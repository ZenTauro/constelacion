\documentclass{article}

\usepackage{amsmath}
\usepackage{amssymb}
\usepackage{booktabs}
\usepackage{hyperref}
\usepackage[utf8]{inputenc}
\usepackage[a4paper, margin=2cm]{geometry}
\usepackage{graphicx}
\usepackage{wrapfig}
\usepackage{float}
\usepackage[spanish]{babel}
\usepackage{pgf, tikz, pgfplots}
\usepackage{mathrsfs}

\pgfplotsset{compat=1.15}
\usetikzlibrary{arrows}

\newcommand{\partder}[1]{\frac{\partial}{\partial #1}}

\title{Cálculo de la región de decisión}
\author{Pedro Gómez Martín}

\begin{document}
\maketitle

\section{Introducción}
En la teoría de detección, para la decisión óptima en función de la
potencia media $(P_e)$ debemos de dividir el un plano cartesiano en
regiones de decisión para determinar la señal transmitida teniendo
en cuenta la distorsión y atenuación que produce la transmisión.

Para realizar las divisiones debemos tener en cuenta la probabilidad
de cada una de las señales, en este caso consideraremos señales
equiprobables.

\section{Intuición}
Asumiendo que no hay dos señales representadas en el mismo punto del
plano cartesiano y que conforme aumentamos la distancia al punto que
se obtiene colocando el vector que representa una señal en el origen
la probabilidad de que esa fuera la señal transmitida decrece, podemos
asignar a cada punto del plano un valor proporcional a dicha
probabilidad.

\begin{align}
  \label{eq:intuition}
  f_n(x,y) = \frac{1}{\left( x - x_n \right)^2 + \left( y - y_n \right)^2  }
\end{align}

\begin{figure}[H]
  \centering
  \includegraphics[width=7cm]{./media/potencial_punto.png}
  \caption{Grafica potencial un punto}
  \label{fig:potential1}
\end{figure}

\section{Desarrollo}
Pensando en la función $f_n(x,y)$\ref{eq:intuition} como una función de
potencial, podemos desarrollar el campo vectorial asociado a ella y
mediante el principio de superposición podemos obtener el campo que
dicta la tendencia a la señal transmitida.

\begin{align}
  \label{eq:development}
  F_n\left( x, y \right) &= -\nabla f_n\left( x, y \right)\\
  &= - \partder{x} f_n \left( x, y \right) \vec{i} - \partder{y} f_n \left( x, y \right) \vec{j}
\end{align}

Procedemos a derivar una función más general.

\begin{align}
  \label{eq:derivative}
  - \partder{n} f_i \left( n, m \right) &= - \partder{n} \left[ \frac{1}{\left( n - n_i \right)^2 + \left( m - m_i \right)^2 } \right]\\
  &= - \partder{n} \left( \left( n - n_i \right)^2 + \left( m - m_i \right)^2 \right)^{-1}\\
  &= - 2\left( \left( \left( n - n_i \right) + \left( m - m_i \right) \right)^2 \right)^{-1} \left( n - n_i \right)\\
  &= - \frac{n - n_i}{\left( \left( n - n_i \right) + \left( m - m_i \right) \right)^2}
\end{align}

Caracterizando $\partder{n}f_i\left( n, m \right)$ con $n = x, m = y$ y viceversa
obtenemos el campo vectorial.

\begin{align}
  \label{eq:development2}
  F_i(x,y) = - 2 \left(
  \begin{array}{c}
    \frac{x - x_n}{\left( \left( x - x_n \right)^2 + \left( y - y_n \right)^2\right)^2}\\
    \frac{y - y_n}{\left( \left( x - x_n \right)^2 + \left( y - y_n \right)^2\right)^2}
  \end{array}
  \right)
\end{align}

Aplicando el principio de superposición, podemos obtener la siguiente fórmula,
donde $i$ representa el índice de los símbolos y $S$ el conjunto de todos los
símbolos posibles.
\begin{align}
  \label{eq:superposition}
  \vec{T_S}(x,y) = \sum\limits_{i} \vec{F_i}\left( x, y\right)
\end{align}

Si resolvemos $\vec{T_S}(x,y) = \vec{0}$, obtendremos la ecuaciones que delimitan
las regiones de decisión.

\subsection{Generalización a $\mathbb{R}^n$}
Aun que se ha desarrollado para una función $f$ de dos variables, se puede
generalizar a $f: \mathbb{R}^n \rightarrow \mathbb{R}$, donde $n$ viene dado por la dimensión de la base
del espacio de símbolos. Y $f$ se puede definir de la siguiente forma:
\begin{align}
  \label{eq:generalf}
  f(x_1, x_2, \cdots ,x_n) = \frac{1}{(x_1 - x_{1i})^2 + (x_2 - x_{2i})^2 + \cdots + (x_n - x_{ni})^2}
\end{align}
Obteniendo lo siguiente: \footnote{Este método se puede utilizar para el cálculo
de diagramas de Voronoi en $\mathbb{R}^n$ con las soluciones a $T_S = \vec{0}$}
\begin{align}
  \label{eq:generalF}
  \vec{T_S}(x_1, x_2, \cdots , x_n) &= \sum_i -\nabla f(x_1, x_2, \cdots ,x_n)
\end{align}

\section{Ejemplo}
Si consideramos dos símbolos con una base compuesta por por dos vectores
$\Phi_1$ y $\Phi_2$, con coordenadas $S_1=(x_{10}, y_{10})$ y $S_2=(x_{20}, y_{20})$, los podemos
representar en $\mathbb{R}^2$
\begin{figure}[h!]
  \centering
  % Title: gl2ps_renderer figure
% Creator: GL2PS 1.4.0, (C) 1999-2017 C. Geuzaine
% For: Octave
% CreationDate: Thu Apr 11 11:20:52 2019
\begin{pgfpicture}
\color[rgb]{1.000000,1.000000,1.000000}
\pgfpathrectanglecorners{\pgfpoint{0pt}{0pt}}{\pgfpoint{576pt}{432pt}}
\pgfusepath{fill}
\begin{pgfscope}
\pgfpathrectangle{\pgfpoint{0pt}{0pt}}{\pgfpoint{576pt}{432pt}}
\pgfusepath{fill}
\pgfpathrectangle{\pgfpoint{0pt}{0pt}}{\pgfpoint{576pt}{432pt}}
\pgfusepath{clip}
\pgfpathmoveto{\pgfpoint{122.039993pt}{399.600006pt}}
\pgflineto{\pgfpoint{474.119995pt}{47.519974pt}}
\pgflineto{\pgfpoint{122.039993pt}{47.519974pt}}
\pgfpathclose
\pgfusepath{fill,stroke}
\pgfpathmoveto{\pgfpoint{122.039993pt}{399.600006pt}}
\pgflineto{\pgfpoint{474.119995pt}{399.600006pt}}
\pgflineto{\pgfpoint{474.119995pt}{47.519974pt}}
\pgfpathclose
\pgfusepath{fill,stroke}
\color[rgb]{0.150000,0.150000,0.150000}
\pgfsetlinewidth{0.500000pt}
\pgfpathmoveto{\pgfpoint{122.039993pt}{51.985001pt}}
\pgflineto{\pgfpoint{122.039993pt}{47.519974pt}}
\pgfusepath{stroke}
\pgfpathmoveto{\pgfpoint{122.039993pt}{395.135010pt}}
\pgflineto{\pgfpoint{122.039993pt}{399.600006pt}}
\pgfusepath{stroke}
\pgfpathmoveto{\pgfpoint{180.720001pt}{51.985001pt}}
\pgflineto{\pgfpoint{180.720001pt}{47.519974pt}}
\pgfusepath{stroke}
\pgfpathmoveto{\pgfpoint{180.720001pt}{395.135010pt}}
\pgflineto{\pgfpoint{180.720001pt}{399.600006pt}}
\pgfusepath{stroke}
\pgfpathmoveto{\pgfpoint{239.399994pt}{51.985001pt}}
\pgflineto{\pgfpoint{239.399994pt}{47.519974pt}}
\pgfusepath{stroke}
\pgfpathmoveto{\pgfpoint{239.399994pt}{395.135010pt}}
\pgflineto{\pgfpoint{239.399994pt}{399.600006pt}}
\pgfusepath{stroke}
\pgfpathmoveto{\pgfpoint{298.079987pt}{51.985001pt}}
\pgflineto{\pgfpoint{298.079987pt}{47.519974pt}}
\pgfusepath{stroke}
\pgfpathmoveto{\pgfpoint{298.079987pt}{395.135010pt}}
\pgflineto{\pgfpoint{298.079987pt}{399.600006pt}}
\pgfusepath{stroke}
\pgfpathmoveto{\pgfpoint{356.759979pt}{51.985001pt}}
\pgflineto{\pgfpoint{356.759979pt}{47.519974pt}}
\pgfusepath{stroke}
\pgfpathmoveto{\pgfpoint{356.759979pt}{395.135010pt}}
\pgflineto{\pgfpoint{356.759979pt}{399.600006pt}}
\pgfusepath{stroke}
\pgfpathmoveto{\pgfpoint{415.440002pt}{51.985001pt}}
\pgflineto{\pgfpoint{415.440002pt}{47.519974pt}}
\pgfusepath{stroke}
\pgfpathmoveto{\pgfpoint{415.440002pt}{395.135010pt}}
\pgflineto{\pgfpoint{415.440002pt}{399.600006pt}}
\pgfusepath{stroke}
\pgfpathmoveto{\pgfpoint{474.119995pt}{51.985001pt}}
\pgflineto{\pgfpoint{474.119995pt}{47.519974pt}}
\pgfusepath{stroke}
\pgfpathmoveto{\pgfpoint{474.119995pt}{395.135010pt}}
\pgflineto{\pgfpoint{474.119995pt}{399.600006pt}}
\pgfusepath{stroke}
{
\pgftransformshift{\pgfpoint{122.039993pt}{40.018265pt}}
\pgfnode{rectangle}{north}{\fontsize{10}{0}\selectfont\textcolor[rgb]{0.15,0.15,0.15}{{-6}}}{}{\pgfusepath{discard}}}
{
\pgftransformshift{\pgfpoint{180.720001pt}{40.018265pt}}
\pgfnode{rectangle}{north}{\fontsize{10}{0}\selectfont\textcolor[rgb]{0.15,0.15,0.15}{{-4}}}{}{\pgfusepath{discard}}}
{
\pgftransformshift{\pgfpoint{239.399994pt}{40.018265pt}}
\pgfnode{rectangle}{north}{\fontsize{10}{0}\selectfont\textcolor[rgb]{0.15,0.15,0.15}{{-2}}}{}{\pgfusepath{discard}}}
{
\pgftransformshift{\pgfpoint{298.079987pt}{40.018265pt}}
\pgfnode{rectangle}{north}{\fontsize{10}{0}\selectfont\textcolor[rgb]{0.15,0.15,0.15}{{0}}}{}{\pgfusepath{discard}}}
{
\pgftransformshift{\pgfpoint{356.759979pt}{40.018265pt}}
\pgfnode{rectangle}{north}{\fontsize{10}{0}\selectfont\textcolor[rgb]{0.15,0.15,0.15}{{2}}}{}{\pgfusepath{discard}}}
{
\pgftransformshift{\pgfpoint{415.440002pt}{40.018265pt}}
\pgfnode{rectangle}{north}{\fontsize{10}{0}\selectfont\textcolor[rgb]{0.15,0.15,0.15}{{4}}}{}{\pgfusepath{discard}}}
{
\pgftransformshift{\pgfpoint{474.119995pt}{40.018265pt}}
\pgfnode{rectangle}{north}{\fontsize{10}{0}\selectfont\textcolor[rgb]{0.15,0.15,0.15}{{6}}}{}{\pgfusepath{discard}}}
\pgfpathmoveto{\pgfpoint{126.504990pt}{47.519974pt}}
\pgflineto{\pgfpoint{122.039993pt}{47.519974pt}}
\pgfusepath{stroke}
\pgfpathmoveto{\pgfpoint{469.654968pt}{47.519974pt}}
\pgflineto{\pgfpoint{474.119995pt}{47.519974pt}}
\pgfusepath{stroke}
\pgfpathmoveto{\pgfpoint{126.504990pt}{106.199989pt}}
\pgflineto{\pgfpoint{122.039993pt}{106.199989pt}}
\pgfusepath{stroke}
\pgfpathmoveto{\pgfpoint{469.654968pt}{106.199989pt}}
\pgflineto{\pgfpoint{474.119995pt}{106.199989pt}}
\pgfusepath{stroke}
\pgfpathmoveto{\pgfpoint{126.504990pt}{164.879990pt}}
\pgflineto{\pgfpoint{122.039993pt}{164.879990pt}}
\pgfusepath{stroke}
\pgfpathmoveto{\pgfpoint{469.654968pt}{164.879990pt}}
\pgflineto{\pgfpoint{474.119995pt}{164.879990pt}}
\pgfusepath{stroke}
\pgfpathmoveto{\pgfpoint{126.504990pt}{223.559998pt}}
\pgflineto{\pgfpoint{122.039993pt}{223.559998pt}}
\pgfusepath{stroke}
\pgfpathmoveto{\pgfpoint{469.654968pt}{223.559998pt}}
\pgflineto{\pgfpoint{474.119995pt}{223.559998pt}}
\pgfusepath{stroke}
\pgfpathmoveto{\pgfpoint{126.504990pt}{282.239990pt}}
\pgflineto{\pgfpoint{122.039993pt}{282.239990pt}}
\pgfusepath{stroke}
\pgfpathmoveto{\pgfpoint{469.654968pt}{282.239990pt}}
\pgflineto{\pgfpoint{474.119995pt}{282.239990pt}}
\pgfusepath{stroke}
\pgfpathmoveto{\pgfpoint{126.504990pt}{340.919983pt}}
\pgflineto{\pgfpoint{122.039993pt}{340.919983pt}}
\pgfusepath{stroke}
\pgfpathmoveto{\pgfpoint{469.654968pt}{340.919983pt}}
\pgflineto{\pgfpoint{474.119995pt}{340.919983pt}}
\pgfusepath{stroke}
\pgfpathmoveto{\pgfpoint{126.504990pt}{399.600006pt}}
\pgflineto{\pgfpoint{122.039993pt}{399.600006pt}}
\pgfusepath{stroke}
\pgfpathmoveto{\pgfpoint{469.654968pt}{399.600006pt}}
\pgflineto{\pgfpoint{474.119995pt}{399.600006pt}}
\pgfusepath{stroke}
{
\pgftransformshift{\pgfpoint{117.038849pt}{47.519989pt}}
\pgfnode{rectangle}{east}{\fontsize{10}{0}\selectfont\textcolor[rgb]{0.15,0.15,0.15}{{-6}}}{}{\pgfusepath{discard}}}
{
\pgftransformshift{\pgfpoint{117.038849pt}{106.199989pt}}
\pgfnode{rectangle}{east}{\fontsize{10}{0}\selectfont\textcolor[rgb]{0.15,0.15,0.15}{{-4}}}{}{\pgfusepath{discard}}}
{
\pgftransformshift{\pgfpoint{117.038849pt}{164.879990pt}}
\pgfnode{rectangle}{east}{\fontsize{10}{0}\selectfont\textcolor[rgb]{0.15,0.15,0.15}{{-2}}}{}{\pgfusepath{discard}}}
{
\pgftransformshift{\pgfpoint{117.038849pt}{223.559998pt}}
\pgfnode{rectangle}{east}{\fontsize{10}{0}\selectfont\textcolor[rgb]{0.15,0.15,0.15}{{0}}}{}{\pgfusepath{discard}}}
{
\pgftransformshift{\pgfpoint{117.038849pt}{282.239990pt}}
\pgfnode{rectangle}{east}{\fontsize{10}{0}\selectfont\textcolor[rgb]{0.15,0.15,0.15}{{2}}}{}{\pgfusepath{discard}}}
{
\pgftransformshift{\pgfpoint{117.038849pt}{340.919983pt}}
\pgfnode{rectangle}{east}{\fontsize{10}{0}\selectfont\textcolor[rgb]{0.15,0.15,0.15}{{4}}}{}{\pgfusepath{discard}}}
{
\pgftransformshift{\pgfpoint{117.038849pt}{399.599976pt}}
\pgfnode{rectangle}{east}{\fontsize{10}{0}\selectfont\textcolor[rgb]{0.15,0.15,0.15}{{6}}}{}{\pgfusepath{discard}}}
\pgfsetrectcap
\pgfsetdash{{16pt}{0pt}}{0pt}
\pgfpathmoveto{\pgfpoint{474.119995pt}{47.519974pt}}
\pgflineto{\pgfpoint{122.039993pt}{47.519974pt}}
\pgfusepath{stroke}
\pgfpathmoveto{\pgfpoint{474.119995pt}{399.600006pt}}
\pgflineto{\pgfpoint{122.039993pt}{399.600006pt}}
\pgfusepath{stroke}
\pgfpathmoveto{\pgfpoint{122.039993pt}{399.600006pt}}
\pgflineto{\pgfpoint{122.039993pt}{47.519974pt}}
\pgfusepath{stroke}
\pgfpathmoveto{\pgfpoint{474.119995pt}{399.600006pt}}
\pgflineto{\pgfpoint{474.119995pt}{47.519974pt}}
\pgfusepath{stroke}
\color[rgb]{0.000000,0.447000,0.741000}
\pgfsetbuttcap
\pgfsetroundjoin
\pgfsetdash{}{0pt}
\pgfpathmoveto{\pgfpoint{151.494461pt}{76.974457pt}}
\pgflineto{\pgfpoint{151.380005pt}{76.859985pt}}
\pgfusepath{stroke}
\pgfpathmoveto{\pgfpoint{151.498657pt}{82.962143pt}}
\pgflineto{\pgfpoint{151.380005pt}{82.847733pt}}
\pgfusepath{stroke}
\pgfpathmoveto{\pgfpoint{151.502991pt}{88.949646pt}}
\pgflineto{\pgfpoint{151.380005pt}{88.835495pt}}
\pgfusepath{stroke}
\pgfpathmoveto{\pgfpoint{151.507492pt}{94.936966pt}}
\pgflineto{\pgfpoint{151.380005pt}{94.823257pt}}
\pgfusepath{stroke}
\pgfpathmoveto{\pgfpoint{151.512131pt}{100.924065pt}}
\pgflineto{\pgfpoint{151.380005pt}{100.811012pt}}
\pgfusepath{stroke}
\pgfpathmoveto{\pgfpoint{151.516891pt}{106.910950pt}}
\pgflineto{\pgfpoint{151.380005pt}{106.798759pt}}
\pgfusepath{stroke}
\pgfpathmoveto{\pgfpoint{151.521790pt}{112.897575pt}}
\pgflineto{\pgfpoint{151.380005pt}{112.786522pt}}
\pgfusepath{stroke}
\pgfpathmoveto{\pgfpoint{151.526825pt}{118.883926pt}}
\pgflineto{\pgfpoint{151.380005pt}{118.774277pt}}
\pgfusepath{stroke}
\pgfpathmoveto{\pgfpoint{151.531937pt}{124.869972pt}}
\pgflineto{\pgfpoint{151.380005pt}{124.762024pt}}
\pgfusepath{stroke}
\pgfpathmoveto{\pgfpoint{151.537155pt}{130.855698pt}}
\pgflineto{\pgfpoint{151.380005pt}{130.749786pt}}
\pgfusepath{stroke}
\pgfpathmoveto{\pgfpoint{151.542419pt}{136.841064pt}}
\pgflineto{\pgfpoint{151.380005pt}{136.737534pt}}
\pgfusepath{stroke}
\pgfpathmoveto{\pgfpoint{151.547714pt}{142.826080pt}}
\pgflineto{\pgfpoint{151.380005pt}{142.725281pt}}
\pgfusepath{stroke}
\pgfpathmoveto{\pgfpoint{151.553009pt}{148.810699pt}}
\pgflineto{\pgfpoint{151.380005pt}{148.713058pt}}
\pgfusepath{stroke}
\pgfpathmoveto{\pgfpoint{151.558273pt}{154.794937pt}}
\pgflineto{\pgfpoint{151.380005pt}{154.700806pt}}
\pgfusepath{stroke}
\pgfpathmoveto{\pgfpoint{151.563477pt}{160.778732pt}}
\pgflineto{\pgfpoint{151.380005pt}{160.688568pt}}
\pgfusepath{stroke}
\pgfpathmoveto{\pgfpoint{151.568527pt}{166.762100pt}}
\pgflineto{\pgfpoint{151.380005pt}{166.676315pt}}
\pgfusepath{stroke}
\pgfpathmoveto{\pgfpoint{151.573441pt}{172.745056pt}}
\pgflineto{\pgfpoint{151.380005pt}{172.664078pt}}
\pgfusepath{stroke}
\pgfpathmoveto{\pgfpoint{151.578140pt}{178.727570pt}}
\pgflineto{\pgfpoint{151.380005pt}{178.651825pt}}
\pgfusepath{stroke}
\pgfpathmoveto{\pgfpoint{151.582596pt}{184.709671pt}}
\pgflineto{\pgfpoint{151.380005pt}{184.639587pt}}
\pgfusepath{stroke}
\pgfpathmoveto{\pgfpoint{151.586700pt}{190.691376pt}}
\pgflineto{\pgfpoint{151.380005pt}{190.627335pt}}
\pgfusepath{stroke}
\pgfpathmoveto{\pgfpoint{151.590454pt}{196.672699pt}}
\pgflineto{\pgfpoint{151.380005pt}{196.615097pt}}
\pgfusepath{stroke}
\pgfpathmoveto{\pgfpoint{151.593796pt}{202.653702pt}}
\pgflineto{\pgfpoint{151.380005pt}{202.602844pt}}
\pgfusepath{stroke}
\pgfpathmoveto{\pgfpoint{151.596680pt}{208.634399pt}}
\pgflineto{\pgfpoint{151.380005pt}{208.590607pt}}
\pgfusepath{stroke}
\pgfpathmoveto{\pgfpoint{151.599091pt}{214.614853pt}}
\pgflineto{\pgfpoint{151.380005pt}{214.578354pt}}
\pgfusepath{stroke}
\pgfpathmoveto{\pgfpoint{151.600998pt}{220.595123pt}}
\pgflineto{\pgfpoint{151.380005pt}{220.566116pt}}
\pgfusepath{stroke}
\pgfpathmoveto{\pgfpoint{151.602356pt}{226.575241pt}}
\pgflineto{\pgfpoint{151.380005pt}{226.553864pt}}
\pgfusepath{stroke}
\pgfpathmoveto{\pgfpoint{151.603180pt}{232.555283pt}}
\pgflineto{\pgfpoint{151.380005pt}{232.541626pt}}
\pgfusepath{stroke}
\pgfpathmoveto{\pgfpoint{151.603424pt}{238.535309pt}}
\pgflineto{\pgfpoint{151.380005pt}{238.529388pt}}
\pgfusepath{stroke}
\pgfpathmoveto{\pgfpoint{151.603149pt}{244.515381pt}}
\pgflineto{\pgfpoint{151.380005pt}{244.517136pt}}
\pgfusepath{stroke}
\pgfpathmoveto{\pgfpoint{151.602356pt}{250.495560pt}}
\pgflineto{\pgfpoint{151.380005pt}{250.504883pt}}
\pgfusepath{stroke}
\pgfpathmoveto{\pgfpoint{151.601044pt}{256.475891pt}}
\pgflineto{\pgfpoint{151.380005pt}{256.492645pt}}
\pgfusepath{stroke}
\pgfpathmoveto{\pgfpoint{151.599274pt}{262.456390pt}}
\pgflineto{\pgfpoint{151.380005pt}{262.480408pt}}
\pgfusepath{stroke}
\pgfpathmoveto{\pgfpoint{151.597061pt}{268.437134pt}}
\pgflineto{\pgfpoint{151.380005pt}{268.468170pt}}
\pgfusepath{stroke}
\pgfpathmoveto{\pgfpoint{151.594452pt}{274.418152pt}}
\pgflineto{\pgfpoint{151.380005pt}{274.455902pt}}
\pgfusepath{stroke}
\pgfpathmoveto{\pgfpoint{151.591461pt}{280.399475pt}}
\pgflineto{\pgfpoint{151.380005pt}{280.443665pt}}
\pgfusepath{stroke}
\pgfpathmoveto{\pgfpoint{151.588196pt}{286.381104pt}}
\pgflineto{\pgfpoint{151.380005pt}{286.431427pt}}
\pgfusepath{stroke}
\pgfpathmoveto{\pgfpoint{151.584579pt}{292.363098pt}}
\pgflineto{\pgfpoint{151.380005pt}{292.419189pt}}
\pgfusepath{stroke}
\pgfpathmoveto{\pgfpoint{151.580780pt}{298.345398pt}}
\pgflineto{\pgfpoint{151.380005pt}{298.406921pt}}
\pgfusepath{stroke}
\pgfpathmoveto{\pgfpoint{151.576736pt}{304.328064pt}}
\pgflineto{\pgfpoint{151.380005pt}{304.394714pt}}
\pgfusepath{stroke}
\pgfpathmoveto{\pgfpoint{151.572540pt}{310.311066pt}}
\pgflineto{\pgfpoint{151.380005pt}{310.382446pt}}
\pgfusepath{stroke}
\pgfpathmoveto{\pgfpoint{151.568207pt}{316.294434pt}}
\pgflineto{\pgfpoint{151.380005pt}{316.370178pt}}
\pgfusepath{stroke}
\pgfpathmoveto{\pgfpoint{151.563766pt}{322.278137pt}}
\pgflineto{\pgfpoint{151.380005pt}{322.357971pt}}
\pgfusepath{stroke}
\pgfpathmoveto{\pgfpoint{151.559235pt}{328.262146pt}}
\pgflineto{\pgfpoint{151.380005pt}{328.345703pt}}
\pgfusepath{stroke}
\pgfpathmoveto{\pgfpoint{151.554688pt}{334.246521pt}}
\pgflineto{\pgfpoint{151.380005pt}{334.333466pt}}
\pgfusepath{stroke}
\pgfpathmoveto{\pgfpoint{151.550079pt}{340.231171pt}}
\pgflineto{\pgfpoint{151.380005pt}{340.321228pt}}
\pgfusepath{stroke}
\pgfpathmoveto{\pgfpoint{151.545471pt}{346.216125pt}}
\pgflineto{\pgfpoint{151.380005pt}{346.308960pt}}
\pgfusepath{stroke}
\pgfpathmoveto{\pgfpoint{151.540894pt}{352.201355pt}}
\pgflineto{\pgfpoint{151.380005pt}{352.296722pt}}
\pgfusepath{stroke}
\pgfpathmoveto{\pgfpoint{151.536331pt}{358.186890pt}}
\pgflineto{\pgfpoint{151.380005pt}{358.284485pt}}
\pgfusepath{stroke}
\pgfpathmoveto{\pgfpoint{151.531815pt}{364.172668pt}}
\pgflineto{\pgfpoint{151.380005pt}{364.272247pt}}
\pgfusepath{stroke}
\pgfpathmoveto{\pgfpoint{151.527374pt}{370.158691pt}}
\pgflineto{\pgfpoint{151.380005pt}{370.260010pt}}
\pgfusepath{stroke}
\pgfpathmoveto{\pgfpoint{157.482132pt}{76.978653pt}}
\pgflineto{\pgfpoint{157.367737pt}{76.859985pt}}
\pgfusepath{stroke}
\pgfpathmoveto{\pgfpoint{157.486496pt}{82.966492pt}}
\pgflineto{\pgfpoint{157.367737pt}{82.847733pt}}
\pgfusepath{stroke}
\pgfpathmoveto{\pgfpoint{157.491013pt}{88.954163pt}}
\pgflineto{\pgfpoint{157.367737pt}{88.835495pt}}
\pgfusepath{stroke}
\pgfpathmoveto{\pgfpoint{157.495712pt}{94.941635pt}}
\pgflineto{\pgfpoint{157.367737pt}{94.823257pt}}
\pgfusepath{stroke}
\pgfpathmoveto{\pgfpoint{157.500565pt}{100.928894pt}}
\pgflineto{\pgfpoint{157.367737pt}{100.811012pt}}
\pgfusepath{stroke}
\pgfpathmoveto{\pgfpoint{157.505585pt}{106.915916pt}}
\pgflineto{\pgfpoint{157.367737pt}{106.798759pt}}
\pgfusepath{stroke}
\pgfpathmoveto{\pgfpoint{157.510757pt}{112.902672pt}}
\pgflineto{\pgfpoint{157.367737pt}{112.786522pt}}
\pgfusepath{stroke}
\pgfpathmoveto{\pgfpoint{157.516083pt}{118.889153pt}}
\pgflineto{\pgfpoint{157.367737pt}{118.774277pt}}
\pgfusepath{stroke}
\pgfpathmoveto{\pgfpoint{157.521530pt}{124.875305pt}}
\pgflineto{\pgfpoint{157.367737pt}{124.762024pt}}
\pgfusepath{stroke}
\pgfpathmoveto{\pgfpoint{157.527069pt}{130.861115pt}}
\pgflineto{\pgfpoint{157.367737pt}{130.749786pt}}
\pgfusepath{stroke}
\pgfpathmoveto{\pgfpoint{157.532700pt}{136.846558pt}}
\pgflineto{\pgfpoint{157.367737pt}{136.737534pt}}
\pgfusepath{stroke}
\pgfpathmoveto{\pgfpoint{157.538406pt}{142.831589pt}}
\pgflineto{\pgfpoint{157.367737pt}{142.725281pt}}
\pgfusepath{stroke}
\pgfpathmoveto{\pgfpoint{157.544113pt}{148.816223pt}}
\pgflineto{\pgfpoint{157.367737pt}{148.713058pt}}
\pgfusepath{stroke}
\pgfpathmoveto{\pgfpoint{157.549820pt}{154.800385pt}}
\pgflineto{\pgfpoint{157.367737pt}{154.700806pt}}
\pgfusepath{stroke}
\pgfpathmoveto{\pgfpoint{157.555450pt}{160.784088pt}}
\pgflineto{\pgfpoint{157.367737pt}{160.688568pt}}
\pgfusepath{stroke}
\pgfpathmoveto{\pgfpoint{157.560989pt}{166.767334pt}}
\pgflineto{\pgfpoint{157.367737pt}{166.676315pt}}
\pgfusepath{stroke}
\pgfpathmoveto{\pgfpoint{157.566345pt}{172.750076pt}}
\pgflineto{\pgfpoint{157.367737pt}{172.664078pt}}
\pgfusepath{stroke}
\pgfpathmoveto{\pgfpoint{157.571487pt}{178.732361pt}}
\pgflineto{\pgfpoint{157.367737pt}{178.651825pt}}
\pgfusepath{stroke}
\pgfpathmoveto{\pgfpoint{157.576355pt}{184.714157pt}}
\pgflineto{\pgfpoint{157.367737pt}{184.639587pt}}
\pgfusepath{stroke}
\pgfpathmoveto{\pgfpoint{157.580887pt}{190.695511pt}}
\pgflineto{\pgfpoint{157.367737pt}{190.627335pt}}
\pgfusepath{stroke}
\pgfpathmoveto{\pgfpoint{157.585022pt}{196.676437pt}}
\pgflineto{\pgfpoint{157.367737pt}{196.615097pt}}
\pgfusepath{stroke}
\pgfpathmoveto{\pgfpoint{157.588699pt}{202.656982pt}}
\pgflineto{\pgfpoint{157.367737pt}{202.602844pt}}
\pgfusepath{stroke}
\pgfpathmoveto{\pgfpoint{157.591888pt}{208.637192pt}}
\pgflineto{\pgfpoint{157.367737pt}{208.590607pt}}
\pgfusepath{stroke}
\pgfpathmoveto{\pgfpoint{157.594528pt}{214.617111pt}}
\pgflineto{\pgfpoint{157.367737pt}{214.578354pt}}
\pgfusepath{stroke}
\pgfpathmoveto{\pgfpoint{157.596588pt}{220.596817pt}}
\pgflineto{\pgfpoint{157.367737pt}{220.566116pt}}
\pgfusepath{stroke}
\pgfpathmoveto{\pgfpoint{157.598068pt}{226.576370pt}}
\pgflineto{\pgfpoint{157.367737pt}{226.553864pt}}
\pgfusepath{stroke}
\pgfpathmoveto{\pgfpoint{157.598938pt}{232.555832pt}}
\pgflineto{\pgfpoint{157.367737pt}{232.541626pt}}
\pgfusepath{stroke}
\pgfpathmoveto{\pgfpoint{157.599167pt}{238.535294pt}}
\pgflineto{\pgfpoint{157.367737pt}{238.529388pt}}
\pgfusepath{stroke}
\pgfpathmoveto{\pgfpoint{157.598831pt}{244.514801pt}}
\pgflineto{\pgfpoint{157.367737pt}{244.517136pt}}
\pgfusepath{stroke}
\pgfpathmoveto{\pgfpoint{157.597885pt}{250.494415pt}}
\pgflineto{\pgfpoint{157.367737pt}{250.504883pt}}
\pgfusepath{stroke}
\pgfpathmoveto{\pgfpoint{157.596405pt}{256.474243pt}}
\pgflineto{\pgfpoint{157.367737pt}{256.492645pt}}
\pgfusepath{stroke}
\pgfpathmoveto{\pgfpoint{157.594391pt}{262.454285pt}}
\pgflineto{\pgfpoint{157.367737pt}{262.480408pt}}
\pgfusepath{stroke}
\pgfpathmoveto{\pgfpoint{157.591904pt}{268.434570pt}}
\pgflineto{\pgfpoint{157.367737pt}{268.468170pt}}
\pgfusepath{stroke}
\pgfpathmoveto{\pgfpoint{157.588989pt}{274.415192pt}}
\pgflineto{\pgfpoint{157.367737pt}{274.455902pt}}
\pgfusepath{stroke}
\pgfpathmoveto{\pgfpoint{157.585663pt}{280.396179pt}}
\pgflineto{\pgfpoint{157.367737pt}{280.443665pt}}
\pgfusepath{stroke}
\pgfpathmoveto{\pgfpoint{157.582031pt}{286.377502pt}}
\pgflineto{\pgfpoint{157.367737pt}{286.431427pt}}
\pgfusepath{stroke}
\pgfpathmoveto{\pgfpoint{157.578094pt}{292.359192pt}}
\pgflineto{\pgfpoint{157.367737pt}{292.419189pt}}
\pgfusepath{stroke}
\pgfpathmoveto{\pgfpoint{157.573898pt}{298.341278pt}}
\pgflineto{\pgfpoint{157.367737pt}{298.406921pt}}
\pgfusepath{stroke}
\pgfpathmoveto{\pgfpoint{157.569504pt}{304.323761pt}}
\pgflineto{\pgfpoint{157.367737pt}{304.394714pt}}
\pgfusepath{stroke}
\pgfpathmoveto{\pgfpoint{157.564926pt}{310.306641pt}}
\pgflineto{\pgfpoint{157.367737pt}{310.382446pt}}
\pgfusepath{stroke}
\pgfpathmoveto{\pgfpoint{157.560242pt}{316.289856pt}}
\pgflineto{\pgfpoint{157.367737pt}{316.370178pt}}
\pgfusepath{stroke}
\pgfpathmoveto{\pgfpoint{157.555435pt}{322.273468pt}}
\pgflineto{\pgfpoint{157.367737pt}{322.357971pt}}
\pgfusepath{stroke}
\pgfpathmoveto{\pgfpoint{157.550583pt}{328.257446pt}}
\pgflineto{\pgfpoint{157.367737pt}{328.345703pt}}
\pgfusepath{stroke}
\pgfpathmoveto{\pgfpoint{157.545685pt}{334.241760pt}}
\pgflineto{\pgfpoint{157.367737pt}{334.333466pt}}
\pgfusepath{stroke}
\pgfpathmoveto{\pgfpoint{157.540771pt}{340.226410pt}}
\pgflineto{\pgfpoint{157.367737pt}{340.321228pt}}
\pgfusepath{stroke}
\pgfpathmoveto{\pgfpoint{157.535873pt}{346.211395pt}}
\pgflineto{\pgfpoint{157.367737pt}{346.308960pt}}
\pgfusepath{stroke}
\pgfpathmoveto{\pgfpoint{157.530991pt}{352.196655pt}}
\pgflineto{\pgfpoint{157.367737pt}{352.296722pt}}
\pgfusepath{stroke}
\pgfpathmoveto{\pgfpoint{157.526169pt}{358.182220pt}}
\pgflineto{\pgfpoint{157.367737pt}{358.284485pt}}
\pgfusepath{stroke}
\pgfpathmoveto{\pgfpoint{157.521393pt}{364.168030pt}}
\pgflineto{\pgfpoint{157.367737pt}{364.272247pt}}
\pgfusepath{stroke}
\pgfpathmoveto{\pgfpoint{157.516708pt}{370.154114pt}}
\pgflineto{\pgfpoint{157.367737pt}{370.260010pt}}
\pgfusepath{stroke}
\pgfpathmoveto{\pgfpoint{163.469635pt}{76.983002pt}}
\pgflineto{\pgfpoint{163.355499pt}{76.859985pt}}
\pgfusepath{stroke}
\pgfpathmoveto{\pgfpoint{163.474152pt}{82.971008pt}}
\pgflineto{\pgfpoint{163.355499pt}{82.847733pt}}
\pgfusepath{stroke}
\pgfpathmoveto{\pgfpoint{163.478867pt}{88.958862pt}}
\pgflineto{\pgfpoint{163.355499pt}{88.835495pt}}
\pgfusepath{stroke}
\pgfpathmoveto{\pgfpoint{163.483765pt}{94.946518pt}}
\pgflineto{\pgfpoint{163.355499pt}{94.823257pt}}
\pgfusepath{stroke}
\pgfpathmoveto{\pgfpoint{163.488861pt}{100.933952pt}}
\pgflineto{\pgfpoint{163.355499pt}{100.811012pt}}
\pgfusepath{stroke}
\pgfpathmoveto{\pgfpoint{163.494125pt}{106.921150pt}}
\pgflineto{\pgfpoint{163.355499pt}{106.798759pt}}
\pgfusepath{stroke}
\pgfpathmoveto{\pgfpoint{163.499573pt}{112.908066pt}}
\pgflineto{\pgfpoint{163.355499pt}{112.786522pt}}
\pgfusepath{stroke}
\pgfpathmoveto{\pgfpoint{163.505203pt}{118.894691pt}}
\pgflineto{\pgfpoint{163.355499pt}{118.774277pt}}
\pgfusepath{stroke}
\pgfpathmoveto{\pgfpoint{163.510986pt}{124.880981pt}}
\pgflineto{\pgfpoint{163.355499pt}{124.762024pt}}
\pgfusepath{stroke}
\pgfpathmoveto{\pgfpoint{163.516907pt}{130.866913pt}}
\pgflineto{\pgfpoint{163.355499pt}{130.749786pt}}
\pgfusepath{stroke}
\pgfpathmoveto{\pgfpoint{163.522949pt}{136.852417pt}}
\pgflineto{\pgfpoint{163.355499pt}{136.737534pt}}
\pgfusepath{stroke}
\pgfpathmoveto{\pgfpoint{163.529053pt}{142.837524pt}}
\pgflineto{\pgfpoint{163.355499pt}{142.725281pt}}
\pgfusepath{stroke}
\pgfpathmoveto{\pgfpoint{163.535217pt}{148.822144pt}}
\pgflineto{\pgfpoint{163.355499pt}{148.713058pt}}
\pgfusepath{stroke}
\pgfpathmoveto{\pgfpoint{163.541397pt}{154.806305pt}}
\pgflineto{\pgfpoint{163.355499pt}{154.700806pt}}
\pgfusepath{stroke}
\pgfpathmoveto{\pgfpoint{163.547546pt}{160.789932pt}}
\pgflineto{\pgfpoint{163.355499pt}{160.688568pt}}
\pgfusepath{stroke}
\pgfpathmoveto{\pgfpoint{163.553574pt}{166.773041pt}}
\pgflineto{\pgfpoint{163.355499pt}{166.676315pt}}
\pgfusepath{stroke}
\pgfpathmoveto{\pgfpoint{163.559448pt}{172.755600pt}}
\pgflineto{\pgfpoint{163.355499pt}{172.664078pt}}
\pgfusepath{stroke}
\pgfpathmoveto{\pgfpoint{163.565094pt}{178.737610pt}}
\pgflineto{\pgfpoint{163.355499pt}{178.651825pt}}
\pgfusepath{stroke}
\pgfpathmoveto{\pgfpoint{163.570465pt}{184.719101pt}}
\pgflineto{\pgfpoint{163.355499pt}{184.639587pt}}
\pgfusepath{stroke}
\pgfpathmoveto{\pgfpoint{163.575470pt}{190.700073pt}}
\pgflineto{\pgfpoint{163.355499pt}{190.627335pt}}
\pgfusepath{stroke}
\pgfpathmoveto{\pgfpoint{163.580032pt}{196.680557pt}}
\pgflineto{\pgfpoint{163.355499pt}{196.615097pt}}
\pgfusepath{stroke}
\pgfpathmoveto{\pgfpoint{163.584106pt}{202.660599pt}}
\pgflineto{\pgfpoint{163.355499pt}{202.602844pt}}
\pgfusepath{stroke}
\pgfpathmoveto{\pgfpoint{163.587631pt}{208.640259pt}}
\pgflineto{\pgfpoint{163.355499pt}{208.590607pt}}
\pgfusepath{stroke}
\pgfpathmoveto{\pgfpoint{163.590546pt}{214.619583pt}}
\pgflineto{\pgfpoint{163.355499pt}{214.578354pt}}
\pgfusepath{stroke}
\pgfpathmoveto{\pgfpoint{163.592804pt}{220.598663pt}}
\pgflineto{\pgfpoint{163.355499pt}{220.566116pt}}
\pgfusepath{stroke}
\pgfpathmoveto{\pgfpoint{163.594391pt}{226.577576pt}}
\pgflineto{\pgfpoint{163.355499pt}{226.553864pt}}
\pgfusepath{stroke}
\pgfpathmoveto{\pgfpoint{163.595306pt}{232.556396pt}}
\pgflineto{\pgfpoint{163.355499pt}{232.541626pt}}
\pgfusepath{stroke}
\pgfpathmoveto{\pgfpoint{163.595535pt}{238.535202pt}}
\pgflineto{\pgfpoint{163.355499pt}{238.529388pt}}
\pgfusepath{stroke}
\pgfpathmoveto{\pgfpoint{163.595078pt}{244.514084pt}}
\pgflineto{\pgfpoint{163.355499pt}{244.517136pt}}
\pgfusepath{stroke}
\pgfpathmoveto{\pgfpoint{163.593979pt}{250.493103pt}}
\pgflineto{\pgfpoint{163.355499pt}{250.504883pt}}
\pgfusepath{stroke}
\pgfpathmoveto{\pgfpoint{163.592270pt}{256.472351pt}}
\pgflineto{\pgfpoint{163.355499pt}{256.492645pt}}
\pgfusepath{stroke}
\pgfpathmoveto{\pgfpoint{163.589966pt}{262.451874pt}}
\pgflineto{\pgfpoint{163.355499pt}{262.480408pt}}
\pgfusepath{stroke}
\pgfpathmoveto{\pgfpoint{163.587173pt}{268.431702pt}}
\pgflineto{\pgfpoint{163.355499pt}{268.468170pt}}
\pgfusepath{stroke}
\pgfpathmoveto{\pgfpoint{163.583893pt}{274.411896pt}}
\pgflineto{\pgfpoint{163.355499pt}{274.455902pt}}
\pgfusepath{stroke}
\pgfpathmoveto{\pgfpoint{163.580215pt}{280.392517pt}}
\pgflineto{\pgfpoint{163.355499pt}{280.443665pt}}
\pgfusepath{stroke}
\pgfpathmoveto{\pgfpoint{163.576172pt}{286.373505pt}}
\pgflineto{\pgfpoint{163.355499pt}{286.431427pt}}
\pgfusepath{stroke}
\pgfpathmoveto{\pgfpoint{163.571823pt}{292.354919pt}}
\pgflineto{\pgfpoint{163.355499pt}{292.419189pt}}
\pgfusepath{stroke}
\pgfpathmoveto{\pgfpoint{163.567230pt}{298.336792pt}}
\pgflineto{\pgfpoint{163.355499pt}{298.406921pt}}
\pgfusepath{stroke}
\pgfpathmoveto{\pgfpoint{163.562424pt}{304.319092pt}}
\pgflineto{\pgfpoint{163.355499pt}{304.394714pt}}
\pgfusepath{stroke}
\pgfpathmoveto{\pgfpoint{163.557449pt}{310.301788pt}}
\pgflineto{\pgfpoint{163.355499pt}{310.382446pt}}
\pgfusepath{stroke}
\pgfpathmoveto{\pgfpoint{163.552368pt}{316.284912pt}}
\pgflineto{\pgfpoint{163.355499pt}{316.370178pt}}
\pgfusepath{stroke}
\pgfpathmoveto{\pgfpoint{163.547195pt}{322.268463pt}}
\pgflineto{\pgfpoint{163.355499pt}{322.357971pt}}
\pgfusepath{stroke}
\pgfpathmoveto{\pgfpoint{163.541962pt}{328.252380pt}}
\pgflineto{\pgfpoint{163.355499pt}{328.345703pt}}
\pgfusepath{stroke}
\pgfpathmoveto{\pgfpoint{163.536713pt}{334.236694pt}}
\pgflineto{\pgfpoint{163.355499pt}{334.333466pt}}
\pgfusepath{stroke}
\pgfpathmoveto{\pgfpoint{163.531464pt}{340.221344pt}}
\pgflineto{\pgfpoint{163.355499pt}{340.321228pt}}
\pgfusepath{stroke}
\pgfpathmoveto{\pgfpoint{163.526245pt}{346.206329pt}}
\pgflineto{\pgfpoint{163.355499pt}{346.308960pt}}
\pgfusepath{stroke}
\pgfpathmoveto{\pgfpoint{163.521072pt}{352.191650pt}}
\pgflineto{\pgfpoint{163.355499pt}{352.296722pt}}
\pgfusepath{stroke}
\pgfpathmoveto{\pgfpoint{163.515961pt}{358.177277pt}}
\pgflineto{\pgfpoint{163.355499pt}{358.284485pt}}
\pgfusepath{stroke}
\pgfpathmoveto{\pgfpoint{163.510910pt}{364.163208pt}}
\pgflineto{\pgfpoint{163.355499pt}{364.272247pt}}
\pgfusepath{stroke}
\pgfpathmoveto{\pgfpoint{163.505951pt}{370.149353pt}}
\pgflineto{\pgfpoint{163.355499pt}{370.260010pt}}
\pgfusepath{stroke}
\pgfpathmoveto{\pgfpoint{169.456970pt}{76.987488pt}}
\pgflineto{\pgfpoint{169.343262pt}{76.859985pt}}
\pgfusepath{stroke}
\pgfpathmoveto{\pgfpoint{169.461639pt}{82.975708pt}}
\pgflineto{\pgfpoint{169.343262pt}{82.847733pt}}
\pgfusepath{stroke}
\pgfpathmoveto{\pgfpoint{169.466522pt}{88.963753pt}}
\pgflineto{\pgfpoint{169.343262pt}{88.835495pt}}
\pgfusepath{stroke}
\pgfpathmoveto{\pgfpoint{169.471619pt}{94.951614pt}}
\pgflineto{\pgfpoint{169.343262pt}{94.823257pt}}
\pgfusepath{stroke}
\pgfpathmoveto{\pgfpoint{169.476944pt}{100.939255pt}}
\pgflineto{\pgfpoint{169.343262pt}{100.811012pt}}
\pgfusepath{stroke}
\pgfpathmoveto{\pgfpoint{169.482452pt}{106.926643pt}}
\pgflineto{\pgfpoint{169.343262pt}{106.798759pt}}
\pgfusepath{stroke}
\pgfpathmoveto{\pgfpoint{169.488220pt}{112.913773pt}}
\pgflineto{\pgfpoint{169.343262pt}{112.786522pt}}
\pgfusepath{stroke}
\pgfpathmoveto{\pgfpoint{169.494171pt}{118.900566pt}}
\pgflineto{\pgfpoint{169.343262pt}{118.774277pt}}
\pgfusepath{stroke}
\pgfpathmoveto{\pgfpoint{169.500305pt}{124.887016pt}}
\pgflineto{\pgfpoint{169.343262pt}{124.762024pt}}
\pgfusepath{stroke}
\pgfpathmoveto{\pgfpoint{169.506622pt}{130.873077pt}}
\pgflineto{\pgfpoint{169.343262pt}{130.749786pt}}
\pgfusepath{stroke}
\pgfpathmoveto{\pgfpoint{169.513092pt}{136.858734pt}}
\pgflineto{\pgfpoint{169.343262pt}{136.737534pt}}
\pgfusepath{stroke}
\pgfpathmoveto{\pgfpoint{169.519669pt}{142.843903pt}}
\pgflineto{\pgfpoint{169.343262pt}{142.725281pt}}
\pgfusepath{stroke}
\pgfpathmoveto{\pgfpoint{169.526337pt}{148.828583pt}}
\pgflineto{\pgfpoint{169.343262pt}{148.713058pt}}
\pgfusepath{stroke}
\pgfpathmoveto{\pgfpoint{169.533051pt}{154.812714pt}}
\pgflineto{\pgfpoint{169.343262pt}{154.700806pt}}
\pgfusepath{stroke}
\pgfpathmoveto{\pgfpoint{169.539734pt}{160.796295pt}}
\pgflineto{\pgfpoint{169.343262pt}{160.688568pt}}
\pgfusepath{stroke}
\pgfpathmoveto{\pgfpoint{169.546326pt}{166.779282pt}}
\pgflineto{\pgfpoint{169.343262pt}{166.676315pt}}
\pgfusepath{stroke}
\pgfpathmoveto{\pgfpoint{169.552780pt}{172.761658pt}}
\pgflineto{\pgfpoint{169.343262pt}{172.664078pt}}
\pgfusepath{stroke}
\pgfpathmoveto{\pgfpoint{169.559006pt}{178.743408pt}}
\pgflineto{\pgfpoint{169.343262pt}{178.651825pt}}
\pgfusepath{stroke}
\pgfpathmoveto{\pgfpoint{169.564941pt}{184.724564pt}}
\pgflineto{\pgfpoint{169.343262pt}{184.639587pt}}
\pgfusepath{stroke}
\pgfpathmoveto{\pgfpoint{169.570480pt}{190.705124pt}}
\pgflineto{\pgfpoint{169.343262pt}{190.627335pt}}
\pgfusepath{stroke}
\pgfpathmoveto{\pgfpoint{169.575546pt}{196.685120pt}}
\pgflineto{\pgfpoint{169.343262pt}{196.615097pt}}
\pgfusepath{stroke}
\pgfpathmoveto{\pgfpoint{169.580063pt}{202.664612pt}}
\pgflineto{\pgfpoint{169.343262pt}{202.602844pt}}
\pgfusepath{stroke}
\pgfpathmoveto{\pgfpoint{169.583969pt}{208.643646pt}}
\pgflineto{\pgfpoint{169.343262pt}{208.590607pt}}
\pgfusepath{stroke}
\pgfpathmoveto{\pgfpoint{169.587204pt}{214.622314pt}}
\pgflineto{\pgfpoint{169.343262pt}{214.578354pt}}
\pgfusepath{stroke}
\pgfpathmoveto{\pgfpoint{169.589691pt}{220.600693pt}}
\pgflineto{\pgfpoint{169.343262pt}{220.566116pt}}
\pgfusepath{stroke}
\pgfpathmoveto{\pgfpoint{169.591431pt}{226.578873pt}}
\pgflineto{\pgfpoint{169.343262pt}{226.553864pt}}
\pgfusepath{stroke}
\pgfpathmoveto{\pgfpoint{169.592377pt}{232.556961pt}}
\pgflineto{\pgfpoint{169.343262pt}{232.541626pt}}
\pgfusepath{stroke}
\pgfpathmoveto{\pgfpoint{169.592560pt}{238.535049pt}}
\pgflineto{\pgfpoint{169.343262pt}{238.529388pt}}
\pgfusepath{stroke}
\pgfpathmoveto{\pgfpoint{169.591980pt}{244.513229pt}}
\pgflineto{\pgfpoint{169.343262pt}{244.517136pt}}
\pgfusepath{stroke}
\pgfpathmoveto{\pgfpoint{169.590683pt}{250.491577pt}}
\pgflineto{\pgfpoint{169.343262pt}{250.504883pt}}
\pgfusepath{stroke}
\pgfpathmoveto{\pgfpoint{169.588684pt}{256.470215pt}}
\pgflineto{\pgfpoint{169.343262pt}{256.492645pt}}
\pgfusepath{stroke}
\pgfpathmoveto{\pgfpoint{169.586090pt}{262.449158pt}}
\pgflineto{\pgfpoint{169.343262pt}{262.480408pt}}
\pgfusepath{stroke}
\pgfpathmoveto{\pgfpoint{169.582916pt}{268.428467pt}}
\pgflineto{\pgfpoint{169.343262pt}{268.468170pt}}
\pgfusepath{stroke}
\pgfpathmoveto{\pgfpoint{169.579239pt}{274.408203pt}}
\pgflineto{\pgfpoint{169.343262pt}{274.455902pt}}
\pgfusepath{stroke}
\pgfpathmoveto{\pgfpoint{169.575104pt}{280.388397pt}}
\pgflineto{\pgfpoint{169.343262pt}{280.443665pt}}
\pgfusepath{stroke}
\pgfpathmoveto{\pgfpoint{169.570618pt}{286.369080pt}}
\pgflineto{\pgfpoint{169.343262pt}{286.431427pt}}
\pgfusepath{stroke}
\pgfpathmoveto{\pgfpoint{169.565826pt}{292.350220pt}}
\pgflineto{\pgfpoint{169.343262pt}{292.419189pt}}
\pgfusepath{stroke}
\pgfpathmoveto{\pgfpoint{169.560776pt}{298.331848pt}}
\pgflineto{\pgfpoint{169.343262pt}{298.406921pt}}
\pgfusepath{stroke}
\pgfpathmoveto{\pgfpoint{169.555511pt}{304.313965pt}}
\pgflineto{\pgfpoint{169.343262pt}{304.394714pt}}
\pgfusepath{stroke}
\pgfpathmoveto{\pgfpoint{169.550095pt}{310.296539pt}}
\pgflineto{\pgfpoint{169.343262pt}{310.382446pt}}
\pgfusepath{stroke}
\pgfpathmoveto{\pgfpoint{169.544586pt}{316.279602pt}}
\pgflineto{\pgfpoint{169.343262pt}{316.370178pt}}
\pgfusepath{stroke}
\pgfpathmoveto{\pgfpoint{169.538986pt}{322.263062pt}}
\pgflineto{\pgfpoint{169.343262pt}{322.357971pt}}
\pgfusepath{stroke}
\pgfpathmoveto{\pgfpoint{169.533356pt}{328.246948pt}}
\pgflineto{\pgfpoint{169.343262pt}{328.345703pt}}
\pgfusepath{stroke}
\pgfpathmoveto{\pgfpoint{169.527740pt}{334.231262pt}}
\pgflineto{\pgfpoint{169.343262pt}{334.333466pt}}
\pgfusepath{stroke}
\pgfpathmoveto{\pgfpoint{169.522125pt}{340.215942pt}}
\pgflineto{\pgfpoint{169.343262pt}{340.321228pt}}
\pgfusepath{stroke}
\pgfpathmoveto{\pgfpoint{169.516571pt}{346.200958pt}}
\pgflineto{\pgfpoint{169.343262pt}{346.308960pt}}
\pgfusepath{stroke}
\pgfpathmoveto{\pgfpoint{169.511078pt}{352.186340pt}}
\pgflineto{\pgfpoint{169.343262pt}{352.296722pt}}
\pgfusepath{stroke}
\pgfpathmoveto{\pgfpoint{169.505646pt}{358.172058pt}}
\pgflineto{\pgfpoint{169.343262pt}{358.284485pt}}
\pgfusepath{stroke}
\pgfpathmoveto{\pgfpoint{169.500336pt}{364.158020pt}}
\pgflineto{\pgfpoint{169.343262pt}{364.272247pt}}
\pgfusepath{stroke}
\pgfpathmoveto{\pgfpoint{169.495117pt}{370.144348pt}}
\pgflineto{\pgfpoint{169.343262pt}{370.260010pt}}
\pgfusepath{stroke}
\pgfpathmoveto{\pgfpoint{175.444077pt}{76.992126pt}}
\pgflineto{\pgfpoint{175.331024pt}{76.859985pt}}
\pgfusepath{stroke}
\pgfpathmoveto{\pgfpoint{175.448898pt}{82.980560pt}}
\pgflineto{\pgfpoint{175.331024pt}{82.847733pt}}
\pgfusepath{stroke}
\pgfpathmoveto{\pgfpoint{175.453964pt}{88.968849pt}}
\pgflineto{\pgfpoint{175.331024pt}{88.835495pt}}
\pgfusepath{stroke}
\pgfpathmoveto{\pgfpoint{175.459259pt}{94.956932pt}}
\pgflineto{\pgfpoint{175.331024pt}{94.823257pt}}
\pgfusepath{stroke}
\pgfpathmoveto{\pgfpoint{175.464813pt}{100.944824pt}}
\pgflineto{\pgfpoint{175.331024pt}{100.811012pt}}
\pgfusepath{stroke}
\pgfpathmoveto{\pgfpoint{175.470612pt}{106.932434pt}}
\pgflineto{\pgfpoint{175.331024pt}{106.798759pt}}
\pgfusepath{stroke}
\pgfpathmoveto{\pgfpoint{175.476654pt}{112.919769pt}}
\pgflineto{\pgfpoint{175.331024pt}{112.786522pt}}
\pgfusepath{stroke}
\pgfpathmoveto{\pgfpoint{175.482941pt}{118.906792pt}}
\pgflineto{\pgfpoint{175.331024pt}{118.774277pt}}
\pgfusepath{stroke}
\pgfpathmoveto{\pgfpoint{175.489441pt}{124.893448pt}}
\pgflineto{\pgfpoint{175.331024pt}{124.762024pt}}
\pgfusepath{stroke}
\pgfpathmoveto{\pgfpoint{175.496201pt}{130.879700pt}}
\pgflineto{\pgfpoint{175.331024pt}{130.749786pt}}
\pgfusepath{stroke}
\pgfpathmoveto{\pgfpoint{175.503128pt}{136.865509pt}}
\pgflineto{\pgfpoint{175.331024pt}{136.737534pt}}
\pgfusepath{stroke}
\pgfpathmoveto{\pgfpoint{175.510208pt}{142.850800pt}}
\pgflineto{\pgfpoint{175.331024pt}{142.725281pt}}
\pgfusepath{stroke}
\pgfpathmoveto{\pgfpoint{175.517410pt}{148.835541pt}}
\pgflineto{\pgfpoint{175.331024pt}{148.713058pt}}
\pgfusepath{stroke}
\pgfpathmoveto{\pgfpoint{175.524689pt}{154.819717pt}}
\pgflineto{\pgfpoint{175.331024pt}{154.700806pt}}
\pgfusepath{stroke}
\pgfpathmoveto{\pgfpoint{175.531982pt}{160.803253pt}}
\pgflineto{\pgfpoint{175.331024pt}{160.688568pt}}
\pgfusepath{stroke}
\pgfpathmoveto{\pgfpoint{175.539230pt}{166.786133pt}}
\pgflineto{\pgfpoint{175.331024pt}{166.676315pt}}
\pgfusepath{stroke}
\pgfpathmoveto{\pgfpoint{175.546341pt}{172.768326pt}}
\pgflineto{\pgfpoint{175.331024pt}{172.664078pt}}
\pgfusepath{stroke}
\pgfpathmoveto{\pgfpoint{175.553223pt}{178.749832pt}}
\pgflineto{\pgfpoint{175.331024pt}{178.651825pt}}
\pgfusepath{stroke}
\pgfpathmoveto{\pgfpoint{175.559799pt}{184.730621pt}}
\pgflineto{\pgfpoint{175.331024pt}{184.639587pt}}
\pgfusepath{stroke}
\pgfpathmoveto{\pgfpoint{175.565948pt}{190.710739pt}}
\pgflineto{\pgfpoint{175.331024pt}{190.627335pt}}
\pgfusepath{stroke}
\pgfpathmoveto{\pgfpoint{175.571609pt}{196.690201pt}}
\pgflineto{\pgfpoint{175.331024pt}{196.615097pt}}
\pgfusepath{stroke}
\pgfpathmoveto{\pgfpoint{175.576645pt}{202.669067pt}}
\pgflineto{\pgfpoint{175.331024pt}{202.602844pt}}
\pgfusepath{stroke}
\pgfpathmoveto{\pgfpoint{175.580994pt}{208.647415pt}}
\pgflineto{\pgfpoint{175.331024pt}{208.590607pt}}
\pgfusepath{stroke}
\pgfpathmoveto{\pgfpoint{175.584579pt}{214.625336pt}}
\pgflineto{\pgfpoint{175.331024pt}{214.578354pt}}
\pgfusepath{stroke}
\pgfpathmoveto{\pgfpoint{175.587341pt}{220.602905pt}}
\pgflineto{\pgfpoint{175.331024pt}{220.566116pt}}
\pgfusepath{stroke}
\pgfpathmoveto{\pgfpoint{175.589218pt}{226.580276pt}}
\pgflineto{\pgfpoint{175.331024pt}{226.553864pt}}
\pgfusepath{stroke}
\pgfpathmoveto{\pgfpoint{175.590225pt}{232.557526pt}}
\pgflineto{\pgfpoint{175.331024pt}{232.541626pt}}
\pgfusepath{stroke}
\pgfpathmoveto{\pgfpoint{175.590363pt}{238.534790pt}}
\pgflineto{\pgfpoint{175.331024pt}{238.529388pt}}
\pgfusepath{stroke}
\pgfpathmoveto{\pgfpoint{175.589615pt}{244.512177pt}}
\pgflineto{\pgfpoint{175.331024pt}{244.517136pt}}
\pgfusepath{stroke}
\pgfpathmoveto{\pgfpoint{175.588074pt}{250.489792pt}}
\pgflineto{\pgfpoint{175.331024pt}{250.504883pt}}
\pgfusepath{stroke}
\pgfpathmoveto{\pgfpoint{175.585754pt}{256.467712pt}}
\pgflineto{\pgfpoint{175.331024pt}{256.492645pt}}
\pgfusepath{stroke}
\pgfpathmoveto{\pgfpoint{175.582779pt}{262.446045pt}}
\pgflineto{\pgfpoint{175.331024pt}{262.480408pt}}
\pgfusepath{stroke}
\pgfpathmoveto{\pgfpoint{175.579147pt}{268.424805pt}}
\pgflineto{\pgfpoint{175.331024pt}{268.468170pt}}
\pgfusepath{stroke}
\pgfpathmoveto{\pgfpoint{175.574982pt}{274.404053pt}}
\pgflineto{\pgfpoint{175.331024pt}{274.455902pt}}
\pgfusepath{stroke}
\pgfpathmoveto{\pgfpoint{175.570389pt}{280.383850pt}}
\pgflineto{\pgfpoint{175.331024pt}{280.443665pt}}
\pgfusepath{stroke}
\pgfpathmoveto{\pgfpoint{175.565369pt}{286.364166pt}}
\pgflineto{\pgfpoint{175.331024pt}{286.431427pt}}
\pgfusepath{stroke}
\pgfpathmoveto{\pgfpoint{175.560059pt}{292.345032pt}}
\pgflineto{\pgfpoint{175.331024pt}{292.419189pt}}
\pgfusepath{stroke}
\pgfpathmoveto{\pgfpoint{175.554504pt}{298.326447pt}}
\pgflineto{\pgfpoint{175.331024pt}{298.406921pt}}
\pgfusepath{stroke}
\pgfpathmoveto{\pgfpoint{175.548737pt}{304.308380pt}}
\pgflineto{\pgfpoint{175.331024pt}{304.394714pt}}
\pgfusepath{stroke}
\pgfpathmoveto{\pgfpoint{175.542847pt}{310.290833pt}}
\pgflineto{\pgfpoint{175.331024pt}{310.382446pt}}
\pgfusepath{stroke}
\pgfpathmoveto{\pgfpoint{175.536865pt}{316.273804pt}}
\pgflineto{\pgfpoint{175.331024pt}{316.370178pt}}
\pgfusepath{stroke}
\pgfpathmoveto{\pgfpoint{175.530823pt}{322.257233pt}}
\pgflineto{\pgfpoint{175.331024pt}{322.357971pt}}
\pgfusepath{stroke}
\pgfpathmoveto{\pgfpoint{175.524780pt}{328.241119pt}}
\pgflineto{\pgfpoint{175.331024pt}{328.345703pt}}
\pgfusepath{stroke}
\pgfpathmoveto{\pgfpoint{175.518753pt}{334.225433pt}}
\pgflineto{\pgfpoint{175.331024pt}{334.333466pt}}
\pgfusepath{stroke}
\pgfpathmoveto{\pgfpoint{175.512756pt}{340.210144pt}}
\pgflineto{\pgfpoint{175.331024pt}{340.321228pt}}
\pgfusepath{stroke}
\pgfpathmoveto{\pgfpoint{175.506836pt}{346.195251pt}}
\pgflineto{\pgfpoint{175.331024pt}{346.308960pt}}
\pgfusepath{stroke}
\pgfpathmoveto{\pgfpoint{175.500992pt}{352.180725pt}}
\pgflineto{\pgfpoint{175.331024pt}{352.296722pt}}
\pgfusepath{stroke}
\pgfpathmoveto{\pgfpoint{175.495270pt}{358.166534pt}}
\pgflineto{\pgfpoint{175.331024pt}{358.284485pt}}
\pgfusepath{stroke}
\pgfpathmoveto{\pgfpoint{175.489639pt}{364.152649pt}}
\pgflineto{\pgfpoint{175.331024pt}{364.272247pt}}
\pgfusepath{stroke}
\pgfpathmoveto{\pgfpoint{175.484161pt}{370.139038pt}}
\pgflineto{\pgfpoint{175.331024pt}{370.260010pt}}
\pgfusepath{stroke}
\pgfpathmoveto{\pgfpoint{181.430954pt}{76.996902pt}}
\pgflineto{\pgfpoint{181.318756pt}{76.859985pt}}
\pgfusepath{stroke}
\pgfpathmoveto{\pgfpoint{181.435913pt}{82.985580pt}}
\pgflineto{\pgfpoint{181.318756pt}{82.847733pt}}
\pgfusepath{stroke}
\pgfpathmoveto{\pgfpoint{181.441162pt}{88.974121pt}}
\pgflineto{\pgfpoint{181.318756pt}{88.835495pt}}
\pgfusepath{stroke}
\pgfpathmoveto{\pgfpoint{181.446655pt}{94.962471pt}}
\pgflineto{\pgfpoint{181.318756pt}{94.823257pt}}
\pgfusepath{stroke}
\pgfpathmoveto{\pgfpoint{181.452438pt}{100.950615pt}}
\pgflineto{\pgfpoint{181.318756pt}{100.811012pt}}
\pgfusepath{stroke}
\pgfpathmoveto{\pgfpoint{181.458496pt}{106.938499pt}}
\pgflineto{\pgfpoint{181.318756pt}{106.798759pt}}
\pgfusepath{stroke}
\pgfpathmoveto{\pgfpoint{181.464859pt}{112.926109pt}}
\pgflineto{\pgfpoint{181.318756pt}{112.786522pt}}
\pgfusepath{stroke}
\pgfpathmoveto{\pgfpoint{181.471481pt}{118.913383pt}}
\pgflineto{\pgfpoint{181.318756pt}{118.774277pt}}
\pgfusepath{stroke}
\pgfpathmoveto{\pgfpoint{181.478394pt}{124.900276pt}}
\pgflineto{\pgfpoint{181.318756pt}{124.762024pt}}
\pgfusepath{stroke}
\pgfpathmoveto{\pgfpoint{181.485565pt}{130.886765pt}}
\pgflineto{\pgfpoint{181.318756pt}{130.749786pt}}
\pgfusepath{stroke}
\pgfpathmoveto{\pgfpoint{181.492981pt}{136.872772pt}}
\pgflineto{\pgfpoint{181.318756pt}{136.737534pt}}
\pgfusepath{stroke}
\pgfpathmoveto{\pgfpoint{181.500610pt}{142.858215pt}}
\pgflineto{\pgfpoint{181.318756pt}{142.725281pt}}
\pgfusepath{stroke}
\pgfpathmoveto{\pgfpoint{181.508408pt}{148.843109pt}}
\pgflineto{\pgfpoint{181.318756pt}{148.713058pt}}
\pgfusepath{stroke}
\pgfpathmoveto{\pgfpoint{181.516327pt}{154.827332pt}}
\pgflineto{\pgfpoint{181.318756pt}{154.700806pt}}
\pgfusepath{stroke}
\pgfpathmoveto{\pgfpoint{181.524292pt}{160.810883pt}}
\pgflineto{\pgfpoint{181.318756pt}{160.688568pt}}
\pgfusepath{stroke}
\pgfpathmoveto{\pgfpoint{181.532257pt}{166.793686pt}}
\pgflineto{\pgfpoint{181.318756pt}{166.676315pt}}
\pgfusepath{stroke}
\pgfpathmoveto{\pgfpoint{181.540115pt}{172.775711pt}}
\pgflineto{\pgfpoint{181.318756pt}{172.664078pt}}
\pgfusepath{stroke}
\pgfpathmoveto{\pgfpoint{181.547760pt}{178.756943pt}}
\pgflineto{\pgfpoint{181.318756pt}{178.651825pt}}
\pgfusepath{stroke}
\pgfpathmoveto{\pgfpoint{181.555054pt}{184.737366pt}}
\pgflineto{\pgfpoint{181.318756pt}{184.639587pt}}
\pgfusepath{stroke}
\pgfpathmoveto{\pgfpoint{181.561951pt}{190.716995pt}}
\pgflineto{\pgfpoint{181.318756pt}{190.627335pt}}
\pgfusepath{stroke}
\pgfpathmoveto{\pgfpoint{181.568268pt}{196.695877pt}}
\pgflineto{\pgfpoint{181.318756pt}{196.615097pt}}
\pgfusepath{stroke}
\pgfpathmoveto{\pgfpoint{181.573914pt}{202.674057pt}}
\pgflineto{\pgfpoint{181.318756pt}{202.602844pt}}
\pgfusepath{stroke}
\pgfpathmoveto{\pgfpoint{181.578796pt}{208.651627pt}}
\pgflineto{\pgfpoint{181.318756pt}{208.590607pt}}
\pgfusepath{stroke}
\pgfpathmoveto{\pgfpoint{181.582794pt}{214.628677pt}}
\pgflineto{\pgfpoint{181.318756pt}{214.578354pt}}
\pgfusepath{stroke}
\pgfpathmoveto{\pgfpoint{181.585846pt}{220.605347pt}}
\pgflineto{\pgfpoint{181.318756pt}{220.566116pt}}
\pgfusepath{stroke}
\pgfpathmoveto{\pgfpoint{181.587906pt}{226.581772pt}}
\pgflineto{\pgfpoint{181.318756pt}{226.553864pt}}
\pgfusepath{stroke}
\pgfpathmoveto{\pgfpoint{181.588959pt}{232.558075pt}}
\pgflineto{\pgfpoint{181.318756pt}{232.541626pt}}
\pgfusepath{stroke}
\pgfpathmoveto{\pgfpoint{181.589005pt}{238.534409pt}}
\pgflineto{\pgfpoint{181.318756pt}{238.529388pt}}
\pgfusepath{stroke}
\pgfpathmoveto{\pgfpoint{181.588074pt}{244.510910pt}}
\pgflineto{\pgfpoint{181.318756pt}{244.517136pt}}
\pgfusepath{stroke}
\pgfpathmoveto{\pgfpoint{181.586212pt}{250.487686pt}}
\pgflineto{\pgfpoint{181.318756pt}{250.504883pt}}
\pgfusepath{stroke}
\pgfpathmoveto{\pgfpoint{181.583527pt}{256.464844pt}}
\pgflineto{\pgfpoint{181.318756pt}{256.492645pt}}
\pgfusepath{stroke}
\pgfpathmoveto{\pgfpoint{181.580048pt}{262.442474pt}}
\pgflineto{\pgfpoint{181.318756pt}{262.480408pt}}
\pgfusepath{stroke}
\pgfpathmoveto{\pgfpoint{181.575928pt}{268.420654pt}}
\pgflineto{\pgfpoint{181.318756pt}{268.468170pt}}
\pgfusepath{stroke}
\pgfpathmoveto{\pgfpoint{181.571228pt}{274.399384pt}}
\pgflineto{\pgfpoint{181.318756pt}{274.455902pt}}
\pgfusepath{stroke}
\pgfpathmoveto{\pgfpoint{181.566040pt}{280.378754pt}}
\pgflineto{\pgfpoint{181.318756pt}{280.443665pt}}
\pgfusepath{stroke}
\pgfpathmoveto{\pgfpoint{181.560455pt}{286.358704pt}}
\pgflineto{\pgfpoint{181.318756pt}{286.431427pt}}
\pgfusepath{stroke}
\pgfpathmoveto{\pgfpoint{181.554565pt}{292.339294pt}}
\pgflineto{\pgfpoint{181.318756pt}{292.419189pt}}
\pgfusepath{stroke}
\pgfpathmoveto{\pgfpoint{181.548431pt}{298.320496pt}}
\pgflineto{\pgfpoint{181.318756pt}{298.406921pt}}
\pgfusepath{stroke}
\pgfpathmoveto{\pgfpoint{181.542114pt}{304.302277pt}}
\pgflineto{\pgfpoint{181.318756pt}{304.394714pt}}
\pgfusepath{stroke}
\pgfpathmoveto{\pgfpoint{181.535690pt}{310.284637pt}}
\pgflineto{\pgfpoint{181.318756pt}{310.382446pt}}
\pgfusepath{stroke}
\pgfpathmoveto{\pgfpoint{181.529190pt}{316.267517pt}}
\pgflineto{\pgfpoint{181.318756pt}{316.370178pt}}
\pgfusepath{stroke}
\pgfpathmoveto{\pgfpoint{181.522675pt}{322.250946pt}}
\pgflineto{\pgfpoint{181.318756pt}{322.357971pt}}
\pgfusepath{stroke}
\pgfpathmoveto{\pgfpoint{181.516174pt}{328.234833pt}}
\pgflineto{\pgfpoint{181.318756pt}{328.345703pt}}
\pgfusepath{stroke}
\pgfpathmoveto{\pgfpoint{181.509705pt}{334.219208pt}}
\pgflineto{\pgfpoint{181.318756pt}{334.333466pt}}
\pgfusepath{stroke}
\pgfpathmoveto{\pgfpoint{181.503311pt}{340.204010pt}}
\pgflineto{\pgfpoint{181.318756pt}{340.321228pt}}
\pgfusepath{stroke}
\pgfpathmoveto{\pgfpoint{181.497009pt}{346.189209pt}}
\pgflineto{\pgfpoint{181.318756pt}{346.308960pt}}
\pgfusepath{stroke}
\pgfpathmoveto{\pgfpoint{181.490814pt}{352.174774pt}}
\pgflineto{\pgfpoint{181.318756pt}{352.296722pt}}
\pgfusepath{stroke}
\pgfpathmoveto{\pgfpoint{181.484756pt}{358.160706pt}}
\pgflineto{\pgfpoint{181.318756pt}{358.284485pt}}
\pgfusepath{stroke}
\pgfpathmoveto{\pgfpoint{181.478821pt}{364.146912pt}}
\pgflineto{\pgfpoint{181.318756pt}{364.272247pt}}
\pgfusepath{stroke}
\pgfpathmoveto{\pgfpoint{181.473053pt}{370.133484pt}}
\pgflineto{\pgfpoint{181.318756pt}{370.260010pt}}
\pgfusepath{stroke}
\pgfpathmoveto{\pgfpoint{187.417572pt}{77.001785pt}}
\pgflineto{\pgfpoint{187.306519pt}{76.859985pt}}
\pgfusepath{stroke}
\pgfpathmoveto{\pgfpoint{187.422684pt}{82.990768pt}}
\pgflineto{\pgfpoint{187.306519pt}{82.847733pt}}
\pgfusepath{stroke}
\pgfpathmoveto{\pgfpoint{187.428070pt}{88.979576pt}}
\pgflineto{\pgfpoint{187.306519pt}{88.835495pt}}
\pgfusepath{stroke}
\pgfpathmoveto{\pgfpoint{187.433777pt}{94.968208pt}}
\pgflineto{\pgfpoint{187.306519pt}{94.823257pt}}
\pgfusepath{stroke}
\pgfpathmoveto{\pgfpoint{187.439789pt}{100.956650pt}}
\pgflineto{\pgfpoint{187.306519pt}{100.811012pt}}
\pgfusepath{stroke}
\pgfpathmoveto{\pgfpoint{187.446106pt}{106.944847pt}}
\pgflineto{\pgfpoint{187.306519pt}{106.798759pt}}
\pgfusepath{stroke}
\pgfpathmoveto{\pgfpoint{187.452759pt}{112.932762pt}}
\pgflineto{\pgfpoint{187.306519pt}{112.786522pt}}
\pgfusepath{stroke}
\pgfpathmoveto{\pgfpoint{187.459747pt}{118.920341pt}}
\pgflineto{\pgfpoint{187.306519pt}{118.774277pt}}
\pgfusepath{stroke}
\pgfpathmoveto{\pgfpoint{187.467072pt}{124.907539pt}}
\pgflineto{\pgfpoint{187.306519pt}{124.762024pt}}
\pgfusepath{stroke}
\pgfpathmoveto{\pgfpoint{187.474701pt}{130.894302pt}}
\pgflineto{\pgfpoint{187.306519pt}{130.749786pt}}
\pgfusepath{stroke}
\pgfpathmoveto{\pgfpoint{187.482635pt}{136.880554pt}}
\pgflineto{\pgfpoint{187.306519pt}{136.737534pt}}
\pgfusepath{stroke}
\pgfpathmoveto{\pgfpoint{187.490845pt}{142.866241pt}}
\pgflineto{\pgfpoint{187.306519pt}{142.725281pt}}
\pgfusepath{stroke}
\pgfpathmoveto{\pgfpoint{187.499268pt}{148.851303pt}}
\pgflineto{\pgfpoint{187.306519pt}{148.713058pt}}
\pgfusepath{stroke}
\pgfpathmoveto{\pgfpoint{187.507904pt}{154.835663pt}}
\pgflineto{\pgfpoint{187.306519pt}{154.700806pt}}
\pgfusepath{stroke}
\pgfpathmoveto{\pgfpoint{187.516632pt}{160.819244pt}}
\pgflineto{\pgfpoint{187.306519pt}{160.688568pt}}
\pgfusepath{stroke}
\pgfpathmoveto{\pgfpoint{187.525391pt}{166.802002pt}}
\pgflineto{\pgfpoint{187.306519pt}{166.676315pt}}
\pgfusepath{stroke}
\pgfpathmoveto{\pgfpoint{187.534088pt}{172.783890pt}}
\pgflineto{\pgfpoint{187.306519pt}{172.664078pt}}
\pgfusepath{stroke}
\pgfpathmoveto{\pgfpoint{187.542603pt}{178.764862pt}}
\pgflineto{\pgfpoint{187.306519pt}{178.651825pt}}
\pgfusepath{stroke}
\pgfpathmoveto{\pgfpoint{187.550766pt}{184.744904pt}}
\pgflineto{\pgfpoint{187.306519pt}{184.639587pt}}
\pgfusepath{stroke}
\pgfpathmoveto{\pgfpoint{187.558487pt}{190.724030pt}}
\pgflineto{\pgfpoint{187.306519pt}{190.627335pt}}
\pgfusepath{stroke}
\pgfpathmoveto{\pgfpoint{187.565613pt}{196.702255pt}}
\pgflineto{\pgfpoint{187.306519pt}{196.615097pt}}
\pgfusepath{stroke}
\pgfpathmoveto{\pgfpoint{187.571991pt}{202.679657pt}}
\pgflineto{\pgfpoint{187.306519pt}{202.602844pt}}
\pgfusepath{stroke}
\pgfpathmoveto{\pgfpoint{187.577454pt}{208.656342pt}}
\pgflineto{\pgfpoint{187.306519pt}{208.590607pt}}
\pgfusepath{stroke}
\pgfpathmoveto{\pgfpoint{187.581955pt}{214.632431pt}}
\pgflineto{\pgfpoint{187.306519pt}{214.578354pt}}
\pgfusepath{stroke}
\pgfpathmoveto{\pgfpoint{187.585358pt}{220.608047pt}}
\pgflineto{\pgfpoint{187.306519pt}{220.566116pt}}
\pgfusepath{stroke}
\pgfpathmoveto{\pgfpoint{187.587616pt}{226.583389pt}}
\pgflineto{\pgfpoint{187.306519pt}{226.553864pt}}
\pgfusepath{stroke}
\pgfpathmoveto{\pgfpoint{187.588684pt}{232.558609pt}}
\pgflineto{\pgfpoint{187.306519pt}{232.541626pt}}
\pgfusepath{stroke}
\pgfpathmoveto{\pgfpoint{187.588623pt}{238.533875pt}}
\pgflineto{\pgfpoint{187.306519pt}{238.529388pt}}
\pgfusepath{stroke}
\pgfpathmoveto{\pgfpoint{187.587433pt}{244.509354pt}}
\pgflineto{\pgfpoint{187.306519pt}{244.517136pt}}
\pgfusepath{stroke}
\pgfpathmoveto{\pgfpoint{187.585205pt}{250.485199pt}}
\pgflineto{\pgfpoint{187.306519pt}{250.504883pt}}
\pgfusepath{stroke}
\pgfpathmoveto{\pgfpoint{187.582031pt}{256.461517pt}}
\pgflineto{\pgfpoint{187.306519pt}{256.492645pt}}
\pgfusepath{stroke}
\pgfpathmoveto{\pgfpoint{187.578033pt}{262.438385pt}}
\pgflineto{\pgfpoint{187.306519pt}{262.480408pt}}
\pgfusepath{stroke}
\pgfpathmoveto{\pgfpoint{187.573288pt}{268.415894pt}}
\pgflineto{\pgfpoint{187.306519pt}{268.468170pt}}
\pgfusepath{stroke}
\pgfpathmoveto{\pgfpoint{187.567932pt}{274.394104pt}}
\pgflineto{\pgfpoint{187.306519pt}{274.455902pt}}
\pgfusepath{stroke}
\pgfpathmoveto{\pgfpoint{187.562088pt}{280.373016pt}}
\pgflineto{\pgfpoint{187.306519pt}{280.443665pt}}
\pgfusepath{stroke}
\pgfpathmoveto{\pgfpoint{187.555847pt}{286.352631pt}}
\pgflineto{\pgfpoint{187.306519pt}{286.431427pt}}
\pgfusepath{stroke}
\pgfpathmoveto{\pgfpoint{187.549301pt}{292.332947pt}}
\pgflineto{\pgfpoint{187.306519pt}{292.419189pt}}
\pgfusepath{stroke}
\pgfpathmoveto{\pgfpoint{187.542526pt}{298.313965pt}}
\pgflineto{\pgfpoint{187.306519pt}{298.406921pt}}
\pgfusepath{stroke}
\pgfpathmoveto{\pgfpoint{187.535614pt}{304.295593pt}}
\pgflineto{\pgfpoint{187.306519pt}{304.394714pt}}
\pgfusepath{stroke}
\pgfpathmoveto{\pgfpoint{187.528595pt}{310.277893pt}}
\pgflineto{\pgfpoint{187.306519pt}{310.382446pt}}
\pgfusepath{stroke}
\pgfpathmoveto{\pgfpoint{187.521545pt}{316.260742pt}}
\pgflineto{\pgfpoint{187.306519pt}{316.370178pt}}
\pgfusepath{stroke}
\pgfpathmoveto{\pgfpoint{187.514511pt}{322.244171pt}}
\pgflineto{\pgfpoint{187.306519pt}{322.357971pt}}
\pgfusepath{stroke}
\pgfpathmoveto{\pgfpoint{187.507507pt}{328.228119pt}}
\pgflineto{\pgfpoint{187.306519pt}{328.345703pt}}
\pgfusepath{stroke}
\pgfpathmoveto{\pgfpoint{187.500580pt}{334.212555pt}}
\pgflineto{\pgfpoint{187.306519pt}{334.333466pt}}
\pgfusepath{stroke}
\pgfpathmoveto{\pgfpoint{187.493774pt}{340.197449pt}}
\pgflineto{\pgfpoint{187.306519pt}{340.321228pt}}
\pgfusepath{stroke}
\pgfpathmoveto{\pgfpoint{187.487061pt}{346.182739pt}}
\pgflineto{\pgfpoint{187.306519pt}{346.308960pt}}
\pgfusepath{stroke}
\pgfpathmoveto{\pgfpoint{187.480515pt}{352.168457pt}}
\pgflineto{\pgfpoint{187.306519pt}{352.296722pt}}
\pgfusepath{stroke}
\pgfpathmoveto{\pgfpoint{187.474106pt}{358.154541pt}}
\pgflineto{\pgfpoint{187.306519pt}{358.284485pt}}
\pgfusepath{stroke}
\pgfpathmoveto{\pgfpoint{187.467865pt}{364.140930pt}}
\pgflineto{\pgfpoint{187.306519pt}{364.272247pt}}
\pgfusepath{stroke}
\pgfpathmoveto{\pgfpoint{187.461792pt}{370.127625pt}}
\pgflineto{\pgfpoint{187.306519pt}{370.260010pt}}
\pgfusepath{stroke}
\pgfpathmoveto{\pgfpoint{193.403931pt}{77.006805pt}}
\pgflineto{\pgfpoint{193.294281pt}{76.859985pt}}
\pgfusepath{stroke}
\pgfpathmoveto{\pgfpoint{193.409149pt}{82.996078pt}}
\pgflineto{\pgfpoint{193.294281pt}{82.847733pt}}
\pgfusepath{stroke}
\pgfpathmoveto{\pgfpoint{193.414703pt}{88.985199pt}}
\pgflineto{\pgfpoint{193.294281pt}{88.835495pt}}
\pgfusepath{stroke}
\pgfpathmoveto{\pgfpoint{193.420563pt}{94.974159pt}}
\pgflineto{\pgfpoint{193.294281pt}{94.823257pt}}
\pgfusepath{stroke}
\pgfpathmoveto{\pgfpoint{193.426788pt}{100.962952pt}}
\pgflineto{\pgfpoint{193.294281pt}{100.811012pt}}
\pgfusepath{stroke}
\pgfpathmoveto{\pgfpoint{193.433380pt}{106.951477pt}}
\pgflineto{\pgfpoint{193.294281pt}{106.798759pt}}
\pgfusepath{stroke}
\pgfpathmoveto{\pgfpoint{193.440338pt}{112.939751pt}}
\pgflineto{\pgfpoint{193.294281pt}{112.786522pt}}
\pgfusepath{stroke}
\pgfpathmoveto{\pgfpoint{193.447693pt}{118.927689pt}}
\pgflineto{\pgfpoint{193.294281pt}{118.774277pt}}
\pgfusepath{stroke}
\pgfpathmoveto{\pgfpoint{193.455414pt}{124.915245pt}}
\pgflineto{\pgfpoint{193.294281pt}{124.762024pt}}
\pgfusepath{stroke}
\pgfpathmoveto{\pgfpoint{193.463531pt}{130.902344pt}}
\pgflineto{\pgfpoint{193.294281pt}{130.749786pt}}
\pgfusepath{stroke}
\pgfpathmoveto{\pgfpoint{193.472015pt}{136.888916pt}}
\pgflineto{\pgfpoint{193.294281pt}{136.737534pt}}
\pgfusepath{stroke}
\pgfpathmoveto{\pgfpoint{193.480835pt}{142.874908pt}}
\pgflineto{\pgfpoint{193.294281pt}{142.725281pt}}
\pgfusepath{stroke}
\pgfpathmoveto{\pgfpoint{193.489975pt}{148.860214pt}}
\pgflineto{\pgfpoint{193.294281pt}{148.713058pt}}
\pgfusepath{stroke}
\pgfpathmoveto{\pgfpoint{193.499359pt}{154.844742pt}}
\pgflineto{\pgfpoint{193.294281pt}{154.700806pt}}
\pgfusepath{stroke}
\pgfpathmoveto{\pgfpoint{193.508942pt}{160.828445pt}}
\pgflineto{\pgfpoint{193.294281pt}{160.688568pt}}
\pgfusepath{stroke}
\pgfpathmoveto{\pgfpoint{193.518616pt}{166.811218pt}}
\pgflineto{\pgfpoint{193.294281pt}{166.676315pt}}
\pgfusepath{stroke}
\pgfpathmoveto{\pgfpoint{193.528259pt}{172.792984pt}}
\pgflineto{\pgfpoint{193.294281pt}{172.664078pt}}
\pgfusepath{stroke}
\pgfpathmoveto{\pgfpoint{193.537750pt}{178.773712pt}}
\pgflineto{\pgfpoint{193.294281pt}{178.651825pt}}
\pgfusepath{stroke}
\pgfpathmoveto{\pgfpoint{193.546936pt}{184.753372pt}}
\pgflineto{\pgfpoint{193.294281pt}{184.639587pt}}
\pgfusepath{stroke}
\pgfpathmoveto{\pgfpoint{193.555649pt}{190.731934pt}}
\pgflineto{\pgfpoint{193.294281pt}{190.627335pt}}
\pgfusepath{stroke}
\pgfpathmoveto{\pgfpoint{193.563705pt}{196.709457pt}}
\pgflineto{\pgfpoint{193.294281pt}{196.615097pt}}
\pgfusepath{stroke}
\pgfpathmoveto{\pgfpoint{193.570923pt}{202.686005pt}}
\pgflineto{\pgfpoint{193.294281pt}{202.602844pt}}
\pgfusepath{stroke}
\pgfpathmoveto{\pgfpoint{193.577148pt}{208.661667pt}}
\pgflineto{\pgfpoint{193.294281pt}{208.590607pt}}
\pgfusepath{stroke}
\pgfpathmoveto{\pgfpoint{193.582214pt}{214.636627pt}}
\pgflineto{\pgfpoint{193.294281pt}{214.578354pt}}
\pgfusepath{stroke}
\pgfpathmoveto{\pgfpoint{193.586029pt}{220.611038pt}}
\pgflineto{\pgfpoint{193.294281pt}{220.566116pt}}
\pgfusepath{stroke}
\pgfpathmoveto{\pgfpoint{193.588501pt}{226.585114pt}}
\pgflineto{\pgfpoint{193.294281pt}{226.553864pt}}
\pgfusepath{stroke}
\pgfpathmoveto{\pgfpoint{193.589600pt}{232.559067pt}}
\pgflineto{\pgfpoint{193.294281pt}{232.541626pt}}
\pgfusepath{stroke}
\pgfpathmoveto{\pgfpoint{193.589371pt}{238.533112pt}}
\pgflineto{\pgfpoint{193.294281pt}{238.529388pt}}
\pgfusepath{stroke}
\pgfpathmoveto{\pgfpoint{193.587860pt}{244.507446pt}}
\pgflineto{\pgfpoint{193.294281pt}{244.517136pt}}
\pgfusepath{stroke}
\pgfpathmoveto{\pgfpoint{193.585159pt}{250.482208pt}}
\pgflineto{\pgfpoint{193.294281pt}{250.504883pt}}
\pgfusepath{stroke}
\pgfpathmoveto{\pgfpoint{193.581390pt}{256.457581pt}}
\pgflineto{\pgfpoint{193.294281pt}{256.492645pt}}
\pgfusepath{stroke}
\pgfpathmoveto{\pgfpoint{193.576706pt}{262.433655pt}}
\pgflineto{\pgfpoint{193.294281pt}{262.480408pt}}
\pgfusepath{stroke}
\pgfpathmoveto{\pgfpoint{193.571259pt}{268.410492pt}}
\pgflineto{\pgfpoint{193.294281pt}{268.468170pt}}
\pgfusepath{stroke}
\pgfpathmoveto{\pgfpoint{193.565155pt}{274.388123pt}}
\pgflineto{\pgfpoint{193.294281pt}{274.455902pt}}
\pgfusepath{stroke}
\pgfpathmoveto{\pgfpoint{193.558533pt}{280.366577pt}}
\pgflineto{\pgfpoint{193.294281pt}{280.443665pt}}
\pgfusepath{stroke}
\pgfpathmoveto{\pgfpoint{193.551529pt}{286.345856pt}}
\pgflineto{\pgfpoint{193.294281pt}{286.431427pt}}
\pgfusepath{stroke}
\pgfpathmoveto{\pgfpoint{193.544250pt}{292.325928pt}}
\pgflineto{\pgfpoint{193.294281pt}{292.419189pt}}
\pgfusepath{stroke}
\pgfpathmoveto{\pgfpoint{193.536774pt}{298.306763pt}}
\pgflineto{\pgfpoint{193.294281pt}{298.406921pt}}
\pgfusepath{stroke}
\pgfpathmoveto{\pgfpoint{193.529190pt}{304.288330pt}}
\pgflineto{\pgfpoint{193.294281pt}{304.394714pt}}
\pgfusepath{stroke}
\pgfpathmoveto{\pgfpoint{193.521545pt}{310.270538pt}}
\pgflineto{\pgfpoint{193.294281pt}{310.382446pt}}
\pgfusepath{stroke}
\pgfpathmoveto{\pgfpoint{193.513901pt}{316.253418pt}}
\pgflineto{\pgfpoint{193.294281pt}{316.370178pt}}
\pgfusepath{stroke}
\pgfpathmoveto{\pgfpoint{193.506302pt}{322.236877pt}}
\pgflineto{\pgfpoint{193.294281pt}{322.357971pt}}
\pgfusepath{stroke}
\pgfpathmoveto{\pgfpoint{193.498779pt}{328.220917pt}}
\pgflineto{\pgfpoint{193.294281pt}{328.345703pt}}
\pgfusepath{stroke}
\pgfpathmoveto{\pgfpoint{193.491379pt}{334.205444pt}}
\pgflineto{\pgfpoint{193.294281pt}{334.333466pt}}
\pgfusepath{stroke}
\pgfpathmoveto{\pgfpoint{193.484100pt}{340.190460pt}}
\pgflineto{\pgfpoint{193.294281pt}{340.321228pt}}
\pgfusepath{stroke}
\pgfpathmoveto{\pgfpoint{193.476990pt}{346.175903pt}}
\pgflineto{\pgfpoint{193.294281pt}{346.308960pt}}
\pgfusepath{stroke}
\pgfpathmoveto{\pgfpoint{193.470047pt}{352.161804pt}}
\pgflineto{\pgfpoint{193.294281pt}{352.296722pt}}
\pgfusepath{stroke}
\pgfpathmoveto{\pgfpoint{193.463287pt}{358.148041pt}}
\pgflineto{\pgfpoint{193.294281pt}{358.284485pt}}
\pgfusepath{stroke}
\pgfpathmoveto{\pgfpoint{193.456726pt}{364.134613pt}}
\pgflineto{\pgfpoint{193.294281pt}{364.272247pt}}
\pgfusepath{stroke}
\pgfpathmoveto{\pgfpoint{193.450363pt}{370.121521pt}}
\pgflineto{\pgfpoint{193.294281pt}{370.260010pt}}
\pgfusepath{stroke}
\pgfpathmoveto{\pgfpoint{199.389984pt}{77.011932pt}}
\pgflineto{\pgfpoint{199.282028pt}{76.859985pt}}
\pgfusepath{stroke}
\pgfpathmoveto{\pgfpoint{199.395309pt}{83.001526pt}}
\pgflineto{\pgfpoint{199.282028pt}{82.847733pt}}
\pgfusepath{stroke}
\pgfpathmoveto{\pgfpoint{199.400970pt}{88.990982pt}}
\pgflineto{\pgfpoint{199.282028pt}{88.835495pt}}
\pgfusepath{stroke}
\pgfpathmoveto{\pgfpoint{199.407013pt}{94.980301pt}}
\pgflineto{\pgfpoint{199.282028pt}{94.823257pt}}
\pgfusepath{stroke}
\pgfpathmoveto{\pgfpoint{199.413452pt}{100.969452pt}}
\pgflineto{\pgfpoint{199.282028pt}{100.811012pt}}
\pgfusepath{stroke}
\pgfpathmoveto{\pgfpoint{199.420288pt}{106.958389pt}}
\pgflineto{\pgfpoint{199.282028pt}{106.798759pt}}
\pgfusepath{stroke}
\pgfpathmoveto{\pgfpoint{199.427551pt}{112.947067pt}}
\pgflineto{\pgfpoint{199.282028pt}{112.786522pt}}
\pgfusepath{stroke}
\pgfpathmoveto{\pgfpoint{199.435242pt}{118.935417pt}}
\pgflineto{\pgfpoint{199.282028pt}{118.774277pt}}
\pgfusepath{stroke}
\pgfpathmoveto{\pgfpoint{199.443390pt}{124.923386pt}}
\pgflineto{\pgfpoint{199.282028pt}{124.762024pt}}
\pgfusepath{stroke}
\pgfpathmoveto{\pgfpoint{199.451996pt}{130.910904pt}}
\pgflineto{\pgfpoint{199.282028pt}{130.749786pt}}
\pgfusepath{stroke}
\pgfpathmoveto{\pgfpoint{199.461044pt}{136.897888pt}}
\pgflineto{\pgfpoint{199.282028pt}{136.737534pt}}
\pgfusepath{stroke}
\pgfpathmoveto{\pgfpoint{199.470520pt}{142.884247pt}}
\pgflineto{\pgfpoint{199.282028pt}{142.725281pt}}
\pgfusepath{stroke}
\pgfpathmoveto{\pgfpoint{199.480408pt}{148.869873pt}}
\pgflineto{\pgfpoint{199.282028pt}{148.713058pt}}
\pgfusepath{stroke}
\pgfpathmoveto{\pgfpoint{199.490631pt}{154.854675pt}}
\pgflineto{\pgfpoint{199.282028pt}{154.700806pt}}
\pgfusepath{stroke}
\pgfpathmoveto{\pgfpoint{199.501144pt}{160.838562pt}}
\pgflineto{\pgfpoint{199.282028pt}{160.688568pt}}
\pgfusepath{stroke}
\pgfpathmoveto{\pgfpoint{199.511841pt}{166.821411pt}}
\pgflineto{\pgfpoint{199.282028pt}{166.676315pt}}
\pgfusepath{stroke}
\pgfpathmoveto{\pgfpoint{199.522583pt}{172.803131pt}}
\pgflineto{\pgfpoint{199.282028pt}{172.664078pt}}
\pgfusepath{stroke}
\pgfpathmoveto{\pgfpoint{199.533218pt}{178.783661pt}}
\pgflineto{\pgfpoint{199.282028pt}{178.651825pt}}
\pgfusepath{stroke}
\pgfpathmoveto{\pgfpoint{199.543579pt}{184.762909pt}}
\pgflineto{\pgfpoint{199.282028pt}{184.639587pt}}
\pgfusepath{stroke}
\pgfpathmoveto{\pgfpoint{199.553467pt}{190.740906pt}}
\pgflineto{\pgfpoint{199.282028pt}{190.627335pt}}
\pgfusepath{stroke}
\pgfpathmoveto{\pgfpoint{199.562637pt}{196.717651pt}}
\pgflineto{\pgfpoint{199.282028pt}{196.615097pt}}
\pgfusepath{stroke}
\pgfpathmoveto{\pgfpoint{199.570877pt}{202.693207pt}}
\pgflineto{\pgfpoint{199.282028pt}{202.602844pt}}
\pgfusepath{stroke}
\pgfpathmoveto{\pgfpoint{199.577972pt}{208.667725pt}}
\pgflineto{\pgfpoint{199.282028pt}{208.590607pt}}
\pgfusepath{stroke}
\pgfpathmoveto{\pgfpoint{199.583740pt}{214.641357pt}}
\pgflineto{\pgfpoint{199.282028pt}{214.578354pt}}
\pgfusepath{stroke}
\pgfpathmoveto{\pgfpoint{199.588013pt}{220.614365pt}}
\pgflineto{\pgfpoint{199.282028pt}{220.566116pt}}
\pgfusepath{stroke}
\pgfpathmoveto{\pgfpoint{199.590744pt}{226.586975pt}}
\pgflineto{\pgfpoint{199.282028pt}{226.553864pt}}
\pgfusepath{stroke}
\pgfpathmoveto{\pgfpoint{199.591858pt}{232.559448pt}}
\pgflineto{\pgfpoint{199.282028pt}{232.541626pt}}
\pgfusepath{stroke}
\pgfpathmoveto{\pgfpoint{199.591400pt}{238.532074pt}}
\pgflineto{\pgfpoint{199.282028pt}{238.529388pt}}
\pgfusepath{stroke}
\pgfpathmoveto{\pgfpoint{199.589462pt}{244.505066pt}}
\pgflineto{\pgfpoint{199.282028pt}{244.517136pt}}
\pgfusepath{stroke}
\pgfpathmoveto{\pgfpoint{199.586166pt}{250.478638pt}}
\pgflineto{\pgfpoint{199.282028pt}{250.504883pt}}
\pgfusepath{stroke}
\pgfpathmoveto{\pgfpoint{199.581680pt}{256.452972pt}}
\pgflineto{\pgfpoint{199.282028pt}{256.492645pt}}
\pgfusepath{stroke}
\pgfpathmoveto{\pgfpoint{199.576172pt}{262.428131pt}}
\pgflineto{\pgfpoint{199.282028pt}{262.480408pt}}
\pgfusepath{stroke}
\pgfpathmoveto{\pgfpoint{199.569839pt}{268.404266pt}}
\pgflineto{\pgfpoint{199.282028pt}{268.468170pt}}
\pgfusepath{stroke}
\pgfpathmoveto{\pgfpoint{199.562866pt}{274.381317pt}}
\pgflineto{\pgfpoint{199.282028pt}{274.455902pt}}
\pgfusepath{stroke}
\pgfpathmoveto{\pgfpoint{199.555359pt}{280.359344pt}}
\pgflineto{\pgfpoint{199.282028pt}{280.443665pt}}
\pgfusepath{stroke}
\pgfpathmoveto{\pgfpoint{199.547501pt}{286.338287pt}}
\pgflineto{\pgfpoint{199.282028pt}{286.431427pt}}
\pgfusepath{stroke}
\pgfpathmoveto{\pgfpoint{199.539398pt}{292.318146pt}}
\pgflineto{\pgfpoint{199.282028pt}{292.419189pt}}
\pgfusepath{stroke}
\pgfpathmoveto{\pgfpoint{199.531143pt}{298.298828pt}}
\pgflineto{\pgfpoint{199.282028pt}{298.406921pt}}
\pgfusepath{stroke}
\pgfpathmoveto{\pgfpoint{199.522827pt}{304.280334pt}}
\pgflineto{\pgfpoint{199.282028pt}{304.394714pt}}
\pgfusepath{stroke}
\pgfpathmoveto{\pgfpoint{199.514496pt}{310.262573pt}}
\pgflineto{\pgfpoint{199.282028pt}{310.382446pt}}
\pgfusepath{stroke}
\pgfpathmoveto{\pgfpoint{199.506210pt}{316.245483pt}}
\pgflineto{\pgfpoint{199.282028pt}{316.370178pt}}
\pgfusepath{stroke}
\pgfpathmoveto{\pgfpoint{199.498016pt}{322.229034pt}}
\pgflineto{\pgfpoint{199.282028pt}{322.357971pt}}
\pgfusepath{stroke}
\pgfpathmoveto{\pgfpoint{199.489960pt}{328.213196pt}}
\pgflineto{\pgfpoint{199.282028pt}{328.345703pt}}
\pgfusepath{stroke}
\pgfpathmoveto{\pgfpoint{199.482025pt}{334.197876pt}}
\pgflineto{\pgfpoint{199.282028pt}{334.333466pt}}
\pgfusepath{stroke}
\pgfpathmoveto{\pgfpoint{199.474289pt}{340.183044pt}}
\pgflineto{\pgfpoint{199.282028pt}{340.321228pt}}
\pgfusepath{stroke}
\pgfpathmoveto{\pgfpoint{199.466736pt}{346.168701pt}}
\pgflineto{\pgfpoint{199.282028pt}{346.308960pt}}
\pgfusepath{stroke}
\pgfpathmoveto{\pgfpoint{199.459396pt}{352.154755pt}}
\pgflineto{\pgfpoint{199.282028pt}{352.296722pt}}
\pgfusepath{stroke}
\pgfpathmoveto{\pgfpoint{199.452271pt}{358.141205pt}}
\pgflineto{\pgfpoint{199.282028pt}{358.284485pt}}
\pgfusepath{stroke}
\pgfpathmoveto{\pgfpoint{199.445374pt}{364.127991pt}}
\pgflineto{\pgfpoint{199.282028pt}{364.272247pt}}
\pgfusepath{stroke}
\pgfpathmoveto{\pgfpoint{199.438721pt}{370.115112pt}}
\pgflineto{\pgfpoint{199.282028pt}{370.260010pt}}
\pgfusepath{stroke}
\pgfpathmoveto{\pgfpoint{205.375702pt}{77.017136pt}}
\pgflineto{\pgfpoint{205.269791pt}{76.859985pt}}
\pgfusepath{stroke}
\pgfpathmoveto{\pgfpoint{205.381119pt}{83.007065pt}}
\pgflineto{\pgfpoint{205.269791pt}{82.847733pt}}
\pgfusepath{stroke}
\pgfpathmoveto{\pgfpoint{205.386902pt}{88.996902pt}}
\pgflineto{\pgfpoint{205.269791pt}{88.835495pt}}
\pgfusepath{stroke}
\pgfpathmoveto{\pgfpoint{205.393097pt}{94.986618pt}}
\pgflineto{\pgfpoint{205.269791pt}{94.823257pt}}
\pgfusepath{stroke}
\pgfpathmoveto{\pgfpoint{205.399689pt}{100.976196pt}}
\pgflineto{\pgfpoint{205.269791pt}{100.811012pt}}
\pgfusepath{stroke}
\pgfpathmoveto{\pgfpoint{205.406769pt}{106.965561pt}}
\pgflineto{\pgfpoint{205.269791pt}{106.798759pt}}
\pgfusepath{stroke}
\pgfpathmoveto{\pgfpoint{205.414307pt}{112.954704pt}}
\pgflineto{\pgfpoint{205.269791pt}{112.786522pt}}
\pgfusepath{stroke}
\pgfpathmoveto{\pgfpoint{205.422348pt}{118.943527pt}}
\pgflineto{\pgfpoint{205.269791pt}{118.774277pt}}
\pgfusepath{stroke}
\pgfpathmoveto{\pgfpoint{205.430908pt}{124.931992pt}}
\pgflineto{\pgfpoint{205.269791pt}{124.762024pt}}
\pgfusepath{stroke}
\pgfpathmoveto{\pgfpoint{205.440002pt}{130.919998pt}}
\pgflineto{\pgfpoint{205.269791pt}{130.749786pt}}
\pgfusepath{stroke}
\pgfpathmoveto{\pgfpoint{205.449631pt}{136.907471pt}}
\pgflineto{\pgfpoint{205.269791pt}{136.737534pt}}
\pgfusepath{stroke}
\pgfpathmoveto{\pgfpoint{205.459808pt}{142.894287pt}}
\pgflineto{\pgfpoint{205.269791pt}{142.725281pt}}
\pgfusepath{stroke}
\pgfpathmoveto{\pgfpoint{205.470490pt}{148.880356pt}}
\pgflineto{\pgfpoint{205.269791pt}{148.713058pt}}
\pgfusepath{stroke}
\pgfpathmoveto{\pgfpoint{205.481628pt}{154.865540pt}}
\pgflineto{\pgfpoint{205.269791pt}{154.700806pt}}
\pgfusepath{stroke}
\pgfpathmoveto{\pgfpoint{205.493164pt}{160.849716pt}}
\pgflineto{\pgfpoint{205.269791pt}{160.688568pt}}
\pgfusepath{stroke}
\pgfpathmoveto{\pgfpoint{205.504990pt}{166.832733pt}}
\pgflineto{\pgfpoint{205.269791pt}{166.676315pt}}
\pgfusepath{stroke}
\pgfpathmoveto{\pgfpoint{205.516983pt}{172.814484pt}}
\pgflineto{\pgfpoint{205.269791pt}{172.664078pt}}
\pgfusepath{stroke}
\pgfpathmoveto{\pgfpoint{205.528961pt}{178.794861pt}}
\pgflineto{\pgfpoint{205.269791pt}{178.651825pt}}
\pgfusepath{stroke}
\pgfpathmoveto{\pgfpoint{205.540710pt}{184.773758pt}}
\pgflineto{\pgfpoint{205.269791pt}{184.639587pt}}
\pgfusepath{stroke}
\pgfpathmoveto{\pgfpoint{205.551971pt}{190.751129pt}}
\pgflineto{\pgfpoint{205.269791pt}{190.627335pt}}
\pgfusepath{stroke}
\pgfpathmoveto{\pgfpoint{205.562500pt}{196.727020pt}}
\pgflineto{\pgfpoint{205.269791pt}{196.615097pt}}
\pgfusepath{stroke}
\pgfpathmoveto{\pgfpoint{205.571976pt}{202.701477pt}}
\pgflineto{\pgfpoint{205.269791pt}{202.602844pt}}
\pgfusepath{stroke}
\pgfpathmoveto{\pgfpoint{205.580139pt}{208.674652pt}}
\pgflineto{\pgfpoint{205.269791pt}{208.590607pt}}
\pgfusepath{stroke}
\pgfpathmoveto{\pgfpoint{205.586746pt}{214.646759pt}}
\pgflineto{\pgfpoint{205.269791pt}{214.578354pt}}
\pgfusepath{stroke}
\pgfpathmoveto{\pgfpoint{205.591614pt}{220.618088pt}}
\pgflineto{\pgfpoint{205.269791pt}{220.566116pt}}
\pgfusepath{stroke}
\pgfpathmoveto{\pgfpoint{205.594604pt}{226.588943pt}}
\pgflineto{\pgfpoint{205.269791pt}{226.553864pt}}
\pgfusepath{stroke}
\pgfpathmoveto{\pgfpoint{205.595688pt}{232.559692pt}}
\pgflineto{\pgfpoint{205.269791pt}{232.541626pt}}
\pgfusepath{stroke}
\pgfpathmoveto{\pgfpoint{205.594925pt}{238.530640pt}}
\pgflineto{\pgfpoint{205.269791pt}{238.529388pt}}
\pgfusepath{stroke}
\pgfpathmoveto{\pgfpoint{205.592422pt}{244.502106pt}}
\pgflineto{\pgfpoint{205.269791pt}{244.517136pt}}
\pgfusepath{stroke}
\pgfpathmoveto{\pgfpoint{205.588379pt}{250.474319pt}}
\pgflineto{\pgfpoint{205.269791pt}{250.504883pt}}
\pgfusepath{stroke}
\pgfpathmoveto{\pgfpoint{205.582977pt}{256.447479pt}}
\pgflineto{\pgfpoint{205.269791pt}{256.492645pt}}
\pgfusepath{stroke}
\pgfpathmoveto{\pgfpoint{205.576477pt}{262.421692pt}}
\pgflineto{\pgfpoint{205.269791pt}{262.480408pt}}
\pgfusepath{stroke}
\pgfpathmoveto{\pgfpoint{205.569107pt}{268.397034pt}}
\pgflineto{\pgfpoint{205.269791pt}{268.468170pt}}
\pgfusepath{stroke}
\pgfpathmoveto{\pgfpoint{205.561066pt}{274.373535pt}}
\pgflineto{\pgfpoint{205.269791pt}{274.455902pt}}
\pgfusepath{stroke}
\pgfpathmoveto{\pgfpoint{205.552551pt}{280.351135pt}}
\pgflineto{\pgfpoint{205.269791pt}{280.443665pt}}
\pgfusepath{stroke}
\pgfpathmoveto{\pgfpoint{205.543701pt}{286.329834pt}}
\pgflineto{\pgfpoint{205.269791pt}{286.431427pt}}
\pgfusepath{stroke}
\pgfpathmoveto{\pgfpoint{205.534698pt}{292.309509pt}}
\pgflineto{\pgfpoint{205.269791pt}{292.419189pt}}
\pgfusepath{stroke}
\pgfpathmoveto{\pgfpoint{205.525574pt}{298.290131pt}}
\pgflineto{\pgfpoint{205.269791pt}{298.406921pt}}
\pgfusepath{stroke}
\pgfpathmoveto{\pgfpoint{205.516449pt}{304.271606pt}}
\pgflineto{\pgfpoint{205.269791pt}{304.394714pt}}
\pgfusepath{stroke}
\pgfpathmoveto{\pgfpoint{205.507385pt}{310.253906pt}}
\pgflineto{\pgfpoint{205.269791pt}{310.382446pt}}
\pgfusepath{stroke}
\pgfpathmoveto{\pgfpoint{205.498413pt}{316.236908pt}}
\pgflineto{\pgfpoint{205.269791pt}{316.370178pt}}
\pgfusepath{stroke}
\pgfpathmoveto{\pgfpoint{205.489594pt}{322.220612pt}}
\pgflineto{\pgfpoint{205.269791pt}{322.357971pt}}
\pgfusepath{stroke}
\pgfpathmoveto{\pgfpoint{205.480957pt}{328.204895pt}}
\pgflineto{\pgfpoint{205.269791pt}{328.345703pt}}
\pgfusepath{stroke}
\pgfpathmoveto{\pgfpoint{205.472504pt}{334.189758pt}}
\pgflineto{\pgfpoint{205.269791pt}{334.333466pt}}
\pgfusepath{stroke}
\pgfpathmoveto{\pgfpoint{205.464264pt}{340.175171pt}}
\pgflineto{\pgfpoint{205.269791pt}{340.321228pt}}
\pgfusepath{stroke}
\pgfpathmoveto{\pgfpoint{205.456284pt}{346.161011pt}}
\pgflineto{\pgfpoint{205.269791pt}{346.308960pt}}
\pgfusepath{stroke}
\pgfpathmoveto{\pgfpoint{205.448517pt}{352.147339pt}}
\pgflineto{\pgfpoint{205.269791pt}{352.296722pt}}
\pgfusepath{stroke}
\pgfpathmoveto{\pgfpoint{205.441025pt}{358.134033pt}}
\pgflineto{\pgfpoint{205.269791pt}{358.284485pt}}
\pgfusepath{stroke}
\pgfpathmoveto{\pgfpoint{205.433777pt}{364.121063pt}}
\pgflineto{\pgfpoint{205.269791pt}{364.272247pt}}
\pgfusepath{stroke}
\pgfpathmoveto{\pgfpoint{205.426819pt}{370.108429pt}}
\pgflineto{\pgfpoint{205.269791pt}{370.260010pt}}
\pgfusepath{stroke}
\pgfpathmoveto{\pgfpoint{211.361084pt}{77.022415pt}}
\pgflineto{\pgfpoint{211.257538pt}{76.859985pt}}
\pgfusepath{stroke}
\pgfpathmoveto{\pgfpoint{211.366547pt}{83.012695pt}}
\pgflineto{\pgfpoint{211.257538pt}{82.847733pt}}
\pgfusepath{stroke}
\pgfpathmoveto{\pgfpoint{211.372421pt}{89.002945pt}}
\pgflineto{\pgfpoint{211.257538pt}{88.835495pt}}
\pgfusepath{stroke}
\pgfpathmoveto{\pgfpoint{211.378723pt}{94.993088pt}}
\pgflineto{\pgfpoint{211.257538pt}{94.823257pt}}
\pgfusepath{stroke}
\pgfpathmoveto{\pgfpoint{211.385498pt}{100.983124pt}}
\pgflineto{\pgfpoint{211.257538pt}{100.811012pt}}
\pgfusepath{stroke}
\pgfpathmoveto{\pgfpoint{211.392761pt}{106.972977pt}}
\pgflineto{\pgfpoint{211.257538pt}{106.798759pt}}
\pgfusepath{stroke}
\pgfpathmoveto{\pgfpoint{211.400558pt}{112.962631pt}}
\pgflineto{\pgfpoint{211.257538pt}{112.786522pt}}
\pgfusepath{stroke}
\pgfpathmoveto{\pgfpoint{211.408936pt}{118.952011pt}}
\pgflineto{\pgfpoint{211.257538pt}{118.774277pt}}
\pgfusepath{stroke}
\pgfpathmoveto{\pgfpoint{211.417877pt}{124.941040pt}}
\pgflineto{\pgfpoint{211.257538pt}{124.762024pt}}
\pgfusepath{stroke}
\pgfpathmoveto{\pgfpoint{211.427460pt}{130.929626pt}}
\pgflineto{\pgfpoint{211.257538pt}{130.749786pt}}
\pgfusepath{stroke}
\pgfpathmoveto{\pgfpoint{211.437698pt}{136.917694pt}}
\pgflineto{\pgfpoint{211.257538pt}{136.737534pt}}
\pgfusepath{stroke}
\pgfpathmoveto{\pgfpoint{211.448578pt}{142.905106pt}}
\pgflineto{\pgfpoint{211.257538pt}{142.725281pt}}
\pgfusepath{stroke}
\pgfpathmoveto{\pgfpoint{211.460083pt}{148.891724pt}}
\pgflineto{\pgfpoint{211.257538pt}{148.713058pt}}
\pgfusepath{stroke}
\pgfpathmoveto{\pgfpoint{211.472198pt}{154.877426pt}}
\pgflineto{\pgfpoint{211.257538pt}{154.700806pt}}
\pgfusepath{stroke}
\pgfpathmoveto{\pgfpoint{211.484879pt}{160.862000pt}}
\pgflineto{\pgfpoint{211.257538pt}{160.688568pt}}
\pgfusepath{stroke}
\pgfpathmoveto{\pgfpoint{211.497986pt}{166.845337pt}}
\pgflineto{\pgfpoint{211.257538pt}{166.676315pt}}
\pgfusepath{stroke}
\pgfpathmoveto{\pgfpoint{211.511414pt}{172.827240pt}}
\pgflineto{\pgfpoint{211.257538pt}{172.664078pt}}
\pgfusepath{stroke}
\pgfpathmoveto{\pgfpoint{211.524933pt}{178.807526pt}}
\pgflineto{\pgfpoint{211.257538pt}{178.651825pt}}
\pgfusepath{stroke}
\pgfpathmoveto{\pgfpoint{211.538300pt}{184.786102pt}}
\pgflineto{\pgfpoint{211.257538pt}{184.639587pt}}
\pgfusepath{stroke}
\pgfpathmoveto{\pgfpoint{211.551239pt}{190.762878pt}}
\pgflineto{\pgfpoint{211.257538pt}{190.627335pt}}
\pgfusepath{stroke}
\pgfpathmoveto{\pgfpoint{211.563385pt}{196.737823pt}}
\pgflineto{\pgfpoint{211.257538pt}{196.615097pt}}
\pgfusepath{stroke}
\pgfpathmoveto{\pgfpoint{211.574387pt}{202.711029pt}}
\pgflineto{\pgfpoint{211.257538pt}{202.602844pt}}
\pgfusepath{stroke}
\pgfpathmoveto{\pgfpoint{211.583878pt}{208.682648pt}}
\pgflineto{\pgfpoint{211.257538pt}{208.590607pt}}
\pgfusepath{stroke}
\pgfpathmoveto{\pgfpoint{211.591522pt}{214.652939pt}}
\pgflineto{\pgfpoint{211.257538pt}{214.578354pt}}
\pgfusepath{stroke}
\pgfpathmoveto{\pgfpoint{211.597076pt}{220.622269pt}}
\pgflineto{\pgfpoint{211.257538pt}{220.566116pt}}
\pgfusepath{stroke}
\pgfpathmoveto{\pgfpoint{211.600372pt}{226.591049pt}}
\pgflineto{\pgfpoint{211.257538pt}{226.553864pt}}
\pgfusepath{stroke}
\pgfpathmoveto{\pgfpoint{211.601410pt}{232.559723pt}}
\pgflineto{\pgfpoint{211.257538pt}{232.541626pt}}
\pgfusepath{stroke}
\pgfpathmoveto{\pgfpoint{211.600189pt}{238.528702pt}}
\pgflineto{\pgfpoint{211.257538pt}{238.529388pt}}
\pgfusepath{stroke}
\pgfpathmoveto{\pgfpoint{211.596954pt}{244.498383pt}}
\pgflineto{\pgfpoint{211.257538pt}{244.517136pt}}
\pgfusepath{stroke}
\pgfpathmoveto{\pgfpoint{211.591904pt}{250.469055pt}}
\pgflineto{\pgfpoint{211.257538pt}{250.504883pt}}
\pgfusepath{stroke}
\pgfpathmoveto{\pgfpoint{211.585373pt}{256.440918pt}}
\pgflineto{\pgfpoint{211.257538pt}{256.492645pt}}
\pgfusepath{stroke}
\pgfpathmoveto{\pgfpoint{211.577667pt}{262.414124pt}}
\pgflineto{\pgfpoint{211.257538pt}{262.480408pt}}
\pgfusepath{stroke}
\pgfpathmoveto{\pgfpoint{211.569031pt}{268.388702pt}}
\pgflineto{\pgfpoint{211.257538pt}{268.468170pt}}
\pgfusepath{stroke}
\pgfpathmoveto{\pgfpoint{211.559753pt}{274.364655pt}}
\pgflineto{\pgfpoint{211.257538pt}{274.455902pt}}
\pgfusepath{stroke}
\pgfpathmoveto{\pgfpoint{211.550049pt}{280.341888pt}}
\pgflineto{\pgfpoint{211.257538pt}{280.443665pt}}
\pgfusepath{stroke}
\pgfpathmoveto{\pgfpoint{211.540100pt}{286.320343pt}}
\pgflineto{\pgfpoint{211.257538pt}{286.431427pt}}
\pgfusepath{stroke}
\pgfpathmoveto{\pgfpoint{211.530060pt}{292.299927pt}}
\pgflineto{\pgfpoint{211.257538pt}{292.419189pt}}
\pgfusepath{stroke}
\pgfpathmoveto{\pgfpoint{211.520004pt}{298.280579pt}}
\pgflineto{\pgfpoint{211.257538pt}{298.406921pt}}
\pgfusepath{stroke}
\pgfpathmoveto{\pgfpoint{211.510010pt}{304.262115pt}}
\pgflineto{\pgfpoint{211.257538pt}{304.394714pt}}
\pgfusepath{stroke}
\pgfpathmoveto{\pgfpoint{211.500168pt}{310.244507pt}}
\pgflineto{\pgfpoint{211.257538pt}{310.382446pt}}
\pgfusepath{stroke}
\pgfpathmoveto{\pgfpoint{211.490494pt}{316.227692pt}}
\pgflineto{\pgfpoint{211.257538pt}{316.370178pt}}
\pgfusepath{stroke}
\pgfpathmoveto{\pgfpoint{211.481003pt}{322.211548pt}}
\pgflineto{\pgfpoint{211.257538pt}{322.357971pt}}
\pgfusepath{stroke}
\pgfpathmoveto{\pgfpoint{211.471771pt}{328.196075pt}}
\pgflineto{\pgfpoint{211.257538pt}{328.345703pt}}
\pgfusepath{stroke}
\pgfpathmoveto{\pgfpoint{211.462769pt}{334.181183pt}}
\pgflineto{\pgfpoint{211.257538pt}{334.333466pt}}
\pgfusepath{stroke}
\pgfpathmoveto{\pgfpoint{211.454041pt}{340.166809pt}}
\pgflineto{\pgfpoint{211.257538pt}{340.321228pt}}
\pgfusepath{stroke}
\pgfpathmoveto{\pgfpoint{211.445572pt}{346.152924pt}}
\pgflineto{\pgfpoint{211.257538pt}{346.308960pt}}
\pgfusepath{stroke}
\pgfpathmoveto{\pgfpoint{211.437408pt}{352.139496pt}}
\pgflineto{\pgfpoint{211.257538pt}{352.296722pt}}
\pgfusepath{stroke}
\pgfpathmoveto{\pgfpoint{211.429535pt}{358.126465pt}}
\pgflineto{\pgfpoint{211.257538pt}{358.284485pt}}
\pgfusepath{stroke}
\pgfpathmoveto{\pgfpoint{211.421936pt}{364.113770pt}}
\pgflineto{\pgfpoint{211.257538pt}{364.272247pt}}
\pgfusepath{stroke}
\pgfpathmoveto{\pgfpoint{211.414658pt}{370.101440pt}}
\pgflineto{\pgfpoint{211.257538pt}{370.260010pt}}
\pgfusepath{stroke}
\pgfpathmoveto{\pgfpoint{217.346100pt}{77.027710pt}}
\pgflineto{\pgfpoint{217.245300pt}{76.859985pt}}
\pgfusepath{stroke}
\pgfpathmoveto{\pgfpoint{217.351593pt}{83.018387pt}}
\pgflineto{\pgfpoint{217.245300pt}{82.847733pt}}
\pgfusepath{stroke}
\pgfpathmoveto{\pgfpoint{217.357513pt}{89.009056pt}}
\pgflineto{\pgfpoint{217.245300pt}{88.835495pt}}
\pgfusepath{stroke}
\pgfpathmoveto{\pgfpoint{217.363892pt}{94.999664pt}}
\pgflineto{\pgfpoint{217.245300pt}{94.823257pt}}
\pgfusepath{stroke}
\pgfpathmoveto{\pgfpoint{217.370789pt}{100.990204pt}}
\pgflineto{\pgfpoint{217.245300pt}{100.811012pt}}
\pgfusepath{stroke}
\pgfpathmoveto{\pgfpoint{217.378235pt}{106.980598pt}}
\pgflineto{\pgfpoint{217.245300pt}{106.798759pt}}
\pgfusepath{stroke}
\pgfpathmoveto{\pgfpoint{217.386246pt}{112.970848pt}}
\pgflineto{\pgfpoint{217.245300pt}{112.786522pt}}
\pgfusepath{stroke}
\pgfpathmoveto{\pgfpoint{217.394897pt}{118.960838pt}}
\pgflineto{\pgfpoint{217.245300pt}{118.774277pt}}
\pgfusepath{stroke}
\pgfpathmoveto{\pgfpoint{217.404236pt}{124.950523pt}}
\pgflineto{\pgfpoint{217.245300pt}{124.762024pt}}
\pgfusepath{stroke}
\pgfpathmoveto{\pgfpoint{217.414291pt}{130.939804pt}}
\pgflineto{\pgfpoint{217.245300pt}{130.749786pt}}
\pgfusepath{stroke}
\pgfpathmoveto{\pgfpoint{217.425110pt}{136.928574pt}}
\pgflineto{\pgfpoint{217.245300pt}{136.737534pt}}
\pgfusepath{stroke}
\pgfpathmoveto{\pgfpoint{217.436691pt}{142.916687pt}}
\pgflineto{\pgfpoint{217.245300pt}{142.725281pt}}
\pgfusepath{stroke}
\pgfpathmoveto{\pgfpoint{217.449097pt}{148.904022pt}}
\pgflineto{\pgfpoint{217.245300pt}{148.713058pt}}
\pgfusepath{stroke}
\pgfpathmoveto{\pgfpoint{217.462250pt}{154.890381pt}}
\pgflineto{\pgfpoint{217.245300pt}{154.700806pt}}
\pgfusepath{stroke}
\pgfpathmoveto{\pgfpoint{217.476151pt}{160.875580pt}}
\pgflineto{\pgfpoint{217.245300pt}{160.688568pt}}
\pgfusepath{stroke}
\pgfpathmoveto{\pgfpoint{217.490692pt}{166.859375pt}}
\pgflineto{\pgfpoint{217.245300pt}{166.676315pt}}
\pgfusepath{stroke}
\pgfpathmoveto{\pgfpoint{217.505722pt}{172.841568pt}}
\pgflineto{\pgfpoint{217.245300pt}{172.664078pt}}
\pgfusepath{stroke}
\pgfpathmoveto{\pgfpoint{217.521027pt}{178.821930pt}}
\pgflineto{\pgfpoint{217.245300pt}{178.651825pt}}
\pgfusepath{stroke}
\pgfpathmoveto{\pgfpoint{217.536346pt}{184.800278pt}}
\pgflineto{\pgfpoint{217.245300pt}{184.639587pt}}
\pgfusepath{stroke}
\pgfpathmoveto{\pgfpoint{217.551300pt}{190.776459pt}}
\pgflineto{\pgfpoint{217.245300pt}{190.627335pt}}
\pgfusepath{stroke}
\pgfpathmoveto{\pgfpoint{217.565430pt}{196.750412pt}}
\pgflineto{\pgfpoint{217.245300pt}{196.615097pt}}
\pgfusepath{stroke}
\pgfpathmoveto{\pgfpoint{217.578339pt}{202.722183pt}}
\pgflineto{\pgfpoint{217.245300pt}{202.602844pt}}
\pgfusepath{stroke}
\pgfpathmoveto{\pgfpoint{217.589447pt}{208.691971pt}}
\pgflineto{\pgfpoint{217.245300pt}{208.590607pt}}
\pgfusepath{stroke}
\pgfpathmoveto{\pgfpoint{217.598404pt}{214.660095pt}}
\pgflineto{\pgfpoint{217.245300pt}{214.578354pt}}
\pgfusepath{stroke}
\pgfpathmoveto{\pgfpoint{217.604813pt}{220.627014pt}}
\pgflineto{\pgfpoint{217.245300pt}{220.566116pt}}
\pgfusepath{stroke}
\pgfpathmoveto{\pgfpoint{217.608459pt}{226.593262pt}}
\pgflineto{\pgfpoint{217.245300pt}{226.553864pt}}
\pgfusepath{stroke}
\pgfpathmoveto{\pgfpoint{217.609344pt}{232.559418pt}}
\pgflineto{\pgfpoint{217.245300pt}{232.541626pt}}
\pgfusepath{stroke}
\pgfpathmoveto{\pgfpoint{217.607513pt}{238.526047pt}}
\pgflineto{\pgfpoint{217.245300pt}{238.529388pt}}
\pgfusepath{stroke}
\pgfpathmoveto{\pgfpoint{217.603271pt}{244.493637pt}}
\pgflineto{\pgfpoint{217.245300pt}{244.517136pt}}
\pgfusepath{stroke}
\pgfpathmoveto{\pgfpoint{217.596954pt}{250.462555pt}}
\pgflineto{\pgfpoint{217.245300pt}{250.504883pt}}
\pgfusepath{stroke}
\pgfpathmoveto{\pgfpoint{217.588974pt}{256.433014pt}}
\pgflineto{\pgfpoint{217.245300pt}{256.492645pt}}
\pgfusepath{stroke}
\pgfpathmoveto{\pgfpoint{217.579727pt}{262.405151pt}}
\pgflineto{\pgfpoint{217.245300pt}{262.480408pt}}
\pgfusepath{stroke}
\pgfpathmoveto{\pgfpoint{217.569580pt}{268.378998pt}}
\pgflineto{\pgfpoint{217.245300pt}{268.468170pt}}
\pgfusepath{stroke}
\pgfpathmoveto{\pgfpoint{217.558853pt}{274.354431pt}}
\pgflineto{\pgfpoint{217.245300pt}{274.455902pt}}
\pgfusepath{stroke}
\pgfpathmoveto{\pgfpoint{217.547821pt}{280.331421pt}}
\pgflineto{\pgfpoint{217.245300pt}{280.443665pt}}
\pgfusepath{stroke}
\pgfpathmoveto{\pgfpoint{217.536621pt}{286.309753pt}}
\pgflineto{\pgfpoint{217.245300pt}{286.431427pt}}
\pgfusepath{stroke}
\pgfpathmoveto{\pgfpoint{217.525436pt}{292.289337pt}}
\pgflineto{\pgfpoint{217.245300pt}{292.419189pt}}
\pgfusepath{stroke}
\pgfpathmoveto{\pgfpoint{217.514359pt}{298.270050pt}}
\pgflineto{\pgfpoint{217.245300pt}{298.406921pt}}
\pgfusepath{stroke}
\pgfpathmoveto{\pgfpoint{217.503464pt}{304.251770pt}}
\pgflineto{\pgfpoint{217.245300pt}{304.394714pt}}
\pgfusepath{stroke}
\pgfpathmoveto{\pgfpoint{217.492767pt}{310.234344pt}}
\pgflineto{\pgfpoint{217.245300pt}{310.382446pt}}
\pgfusepath{stroke}
\pgfpathmoveto{\pgfpoint{217.482346pt}{316.217743pt}}
\pgflineto{\pgfpoint{217.245300pt}{316.370178pt}}
\pgfusepath{stroke}
\pgfpathmoveto{\pgfpoint{217.472198pt}{322.201874pt}}
\pgflineto{\pgfpoint{217.245300pt}{322.357971pt}}
\pgfusepath{stroke}
\pgfpathmoveto{\pgfpoint{217.462326pt}{328.186646pt}}
\pgflineto{\pgfpoint{217.245300pt}{328.345703pt}}
\pgfusepath{stroke}
\pgfpathmoveto{\pgfpoint{217.452759pt}{334.172028pt}}
\pgflineto{\pgfpoint{217.245300pt}{334.333466pt}}
\pgfusepath{stroke}
\pgfpathmoveto{\pgfpoint{217.443512pt}{340.157959pt}}
\pgflineto{\pgfpoint{217.245300pt}{340.321228pt}}
\pgfusepath{stroke}
\pgfpathmoveto{\pgfpoint{217.434586pt}{346.144409pt}}
\pgflineto{\pgfpoint{217.245300pt}{346.308960pt}}
\pgfusepath{stroke}
\pgfpathmoveto{\pgfpoint{217.425980pt}{352.131287pt}}
\pgflineto{\pgfpoint{217.245300pt}{352.296722pt}}
\pgfusepath{stroke}
\pgfpathmoveto{\pgfpoint{217.417725pt}{358.118530pt}}
\pgflineto{\pgfpoint{217.245300pt}{358.284485pt}}
\pgfusepath{stroke}
\pgfpathmoveto{\pgfpoint{217.409790pt}{364.106201pt}}
\pgflineto{\pgfpoint{217.245300pt}{364.272247pt}}
\pgfusepath{stroke}
\pgfpathmoveto{\pgfpoint{217.402191pt}{370.094177pt}}
\pgflineto{\pgfpoint{217.245300pt}{370.260010pt}}
\pgfusepath{stroke}
\pgfpathmoveto{\pgfpoint{223.330719pt}{77.033020pt}}
\pgflineto{\pgfpoint{223.233063pt}{76.859985pt}}
\pgfusepath{stroke}
\pgfpathmoveto{\pgfpoint{223.336212pt}{83.024109pt}}
\pgflineto{\pgfpoint{223.233063pt}{82.847733pt}}
\pgfusepath{stroke}
\pgfpathmoveto{\pgfpoint{223.342148pt}{89.015228pt}}
\pgflineto{\pgfpoint{223.233063pt}{88.835495pt}}
\pgfusepath{stroke}
\pgfpathmoveto{\pgfpoint{223.348587pt}{95.006332pt}}
\pgflineto{\pgfpoint{223.233063pt}{94.823257pt}}
\pgfusepath{stroke}
\pgfpathmoveto{\pgfpoint{223.355560pt}{100.997398pt}}
\pgflineto{\pgfpoint{223.233063pt}{100.811012pt}}
\pgfusepath{stroke}
\pgfpathmoveto{\pgfpoint{223.363098pt}{106.988403pt}}
\pgflineto{\pgfpoint{223.233063pt}{106.798759pt}}
\pgfusepath{stroke}
\pgfpathmoveto{\pgfpoint{223.371307pt}{112.979279pt}}
\pgflineto{\pgfpoint{223.233063pt}{112.786522pt}}
\pgfusepath{stroke}
\pgfpathmoveto{\pgfpoint{223.380219pt}{118.969971pt}}
\pgflineto{\pgfpoint{223.233063pt}{118.774277pt}}
\pgfusepath{stroke}
\pgfpathmoveto{\pgfpoint{223.389862pt}{124.960403pt}}
\pgflineto{\pgfpoint{223.233063pt}{124.762024pt}}
\pgfusepath{stroke}
\pgfpathmoveto{\pgfpoint{223.400360pt}{130.950470pt}}
\pgflineto{\pgfpoint{223.233063pt}{130.749786pt}}
\pgfusepath{stroke}
\pgfpathmoveto{\pgfpoint{223.411728pt}{136.940094pt}}
\pgflineto{\pgfpoint{223.233063pt}{136.737534pt}}
\pgfusepath{stroke}
\pgfpathmoveto{\pgfpoint{223.424026pt}{142.929077pt}}
\pgflineto{\pgfpoint{223.233063pt}{142.725281pt}}
\pgfusepath{stroke}
\pgfpathmoveto{\pgfpoint{223.437302pt}{148.917297pt}}
\pgflineto{\pgfpoint{223.233063pt}{148.713058pt}}
\pgfusepath{stroke}
\pgfpathmoveto{\pgfpoint{223.451538pt}{154.904510pt}}
\pgflineto{\pgfpoint{223.233063pt}{154.700806pt}}
\pgfusepath{stroke}
\pgfpathmoveto{\pgfpoint{223.466766pt}{160.890533pt}}
\pgflineto{\pgfpoint{223.233063pt}{160.688568pt}}
\pgfusepath{stroke}
\pgfpathmoveto{\pgfpoint{223.482864pt}{166.875031pt}}
\pgflineto{\pgfpoint{223.233063pt}{166.676315pt}}
\pgfusepath{stroke}
\pgfpathmoveto{\pgfpoint{223.499725pt}{172.857742pt}}
\pgflineto{\pgfpoint{223.233063pt}{172.664078pt}}
\pgfusepath{stroke}
\pgfpathmoveto{\pgfpoint{223.517120pt}{178.838379pt}}
\pgflineto{\pgfpoint{223.233063pt}{178.651825pt}}
\pgfusepath{stroke}
\pgfpathmoveto{\pgfpoint{223.534729pt}{184.816635pt}}
\pgflineto{\pgfpoint{223.233063pt}{184.639587pt}}
\pgfusepath{stroke}
\pgfpathmoveto{\pgfpoint{223.552124pt}{190.792282pt}}
\pgflineto{\pgfpoint{223.233063pt}{190.627335pt}}
\pgfusepath{stroke}
\pgfpathmoveto{\pgfpoint{223.568756pt}{196.765198pt}}
\pgflineto{\pgfpoint{223.233063pt}{196.615097pt}}
\pgfusepath{stroke}
\pgfpathmoveto{\pgfpoint{223.584015pt}{202.735367pt}}
\pgflineto{\pgfpoint{223.233063pt}{202.602844pt}}
\pgfusepath{stroke}
\pgfpathmoveto{\pgfpoint{223.597260pt}{208.703003pt}}
\pgflineto{\pgfpoint{223.233063pt}{208.590607pt}}
\pgfusepath{stroke}
\pgfpathmoveto{\pgfpoint{223.607849pt}{214.668488pt}}
\pgflineto{\pgfpoint{223.233063pt}{214.578354pt}}
\pgfusepath{stroke}
\pgfpathmoveto{\pgfpoint{223.615326pt}{220.632431pt}}
\pgflineto{\pgfpoint{223.233063pt}{220.566116pt}}
\pgfusepath{stroke}
\pgfpathmoveto{\pgfpoint{223.619385pt}{226.595551pt}}
\pgflineto{\pgfpoint{223.233063pt}{226.553864pt}}
\pgfusepath{stroke}
\pgfpathmoveto{\pgfpoint{223.619980pt}{232.558624pt}}
\pgflineto{\pgfpoint{223.233063pt}{232.541626pt}}
\pgfusepath{stroke}
\pgfpathmoveto{\pgfpoint{223.617279pt}{238.522430pt}}
\pgflineto{\pgfpoint{223.233063pt}{238.529388pt}}
\pgfusepath{stroke}
\pgfpathmoveto{\pgfpoint{223.611664pt}{244.487549pt}}
\pgflineto{\pgfpoint{223.233063pt}{244.517136pt}}
\pgfusepath{stroke}
\pgfpathmoveto{\pgfpoint{223.603668pt}{250.454483pt}}
\pgflineto{\pgfpoint{223.233063pt}{250.504883pt}}
\pgfusepath{stroke}
\pgfpathmoveto{\pgfpoint{223.593826pt}{256.423431pt}}
\pgflineto{\pgfpoint{223.233063pt}{256.492645pt}}
\pgfusepath{stroke}
\pgfpathmoveto{\pgfpoint{223.582687pt}{262.394501pt}}
\pgflineto{\pgfpoint{223.233063pt}{262.480408pt}}
\pgfusepath{stroke}
\pgfpathmoveto{\pgfpoint{223.570709pt}{268.367645pt}}
\pgflineto{\pgfpoint{223.233063pt}{268.468170pt}}
\pgfusepath{stroke}
\pgfpathmoveto{\pgfpoint{223.558273pt}{274.342712pt}}
\pgflineto{\pgfpoint{223.233063pt}{274.455902pt}}
\pgfusepath{stroke}
\pgfpathmoveto{\pgfpoint{223.545685pt}{280.319550pt}}
\pgflineto{\pgfpoint{223.233063pt}{280.443665pt}}
\pgfusepath{stroke}
\pgfpathmoveto{\pgfpoint{223.533112pt}{286.297913pt}}
\pgflineto{\pgfpoint{223.233063pt}{286.431427pt}}
\pgfusepath{stroke}
\pgfpathmoveto{\pgfpoint{223.520691pt}{292.277649pt}}
\pgflineto{\pgfpoint{223.233063pt}{292.419189pt}}
\pgfusepath{stroke}
\pgfpathmoveto{\pgfpoint{223.508545pt}{298.258545pt}}
\pgflineto{\pgfpoint{223.233063pt}{298.406921pt}}
\pgfusepath{stroke}
\pgfpathmoveto{\pgfpoint{223.496674pt}{304.240509pt}}
\pgflineto{\pgfpoint{223.233063pt}{304.394714pt}}
\pgfusepath{stroke}
\pgfpathmoveto{\pgfpoint{223.485138pt}{310.223389pt}}
\pgflineto{\pgfpoint{223.233063pt}{310.382446pt}}
\pgfusepath{stroke}
\pgfpathmoveto{\pgfpoint{223.473938pt}{316.207092pt}}
\pgflineto{\pgfpoint{223.233063pt}{316.370178pt}}
\pgfusepath{stroke}
\pgfpathmoveto{\pgfpoint{223.463074pt}{322.191528pt}}
\pgflineto{\pgfpoint{223.233063pt}{322.357971pt}}
\pgfusepath{stroke}
\pgfpathmoveto{\pgfpoint{223.452576pt}{328.176636pt}}
\pgflineto{\pgfpoint{223.233063pt}{328.345703pt}}
\pgfusepath{stroke}
\pgfpathmoveto{\pgfpoint{223.442444pt}{334.162354pt}}
\pgflineto{\pgfpoint{223.233063pt}{334.333466pt}}
\pgfusepath{stroke}
\pgfpathmoveto{\pgfpoint{223.432678pt}{340.148621pt}}
\pgflineto{\pgfpoint{223.233063pt}{340.321228pt}}
\pgfusepath{stroke}
\pgfpathmoveto{\pgfpoint{223.423264pt}{346.135376pt}}
\pgflineto{\pgfpoint{223.233063pt}{346.308960pt}}
\pgfusepath{stroke}
\pgfpathmoveto{\pgfpoint{223.414246pt}{352.122620pt}}
\pgflineto{\pgfpoint{223.233063pt}{352.296722pt}}
\pgfusepath{stroke}
\pgfpathmoveto{\pgfpoint{223.405579pt}{358.110260pt}}
\pgflineto{\pgfpoint{223.233063pt}{358.284485pt}}
\pgfusepath{stroke}
\pgfpathmoveto{\pgfpoint{223.397308pt}{364.098267pt}}
\pgflineto{\pgfpoint{223.233063pt}{364.272247pt}}
\pgfusepath{stroke}
\pgfpathmoveto{\pgfpoint{223.389404pt}{370.086609pt}}
\pgflineto{\pgfpoint{223.233063pt}{370.260010pt}}
\pgfusepath{stroke}
\pgfpathmoveto{\pgfpoint{229.314941pt}{77.038269pt}}
\pgflineto{\pgfpoint{229.220810pt}{76.859985pt}}
\pgfusepath{stroke}
\pgfpathmoveto{\pgfpoint{229.320374pt}{83.029816pt}}
\pgflineto{\pgfpoint{229.220810pt}{82.847733pt}}
\pgfusepath{stroke}
\pgfpathmoveto{\pgfpoint{229.326294pt}{89.021393pt}}
\pgflineto{\pgfpoint{229.220810pt}{88.835495pt}}
\pgfusepath{stroke}
\pgfpathmoveto{\pgfpoint{229.332718pt}{95.013039pt}}
\pgflineto{\pgfpoint{229.220810pt}{94.823257pt}}
\pgfusepath{stroke}
\pgfpathmoveto{\pgfpoint{229.339722pt}{101.004700pt}}
\pgflineto{\pgfpoint{229.220810pt}{100.811012pt}}
\pgfusepath{stroke}
\pgfpathmoveto{\pgfpoint{229.347351pt}{106.996323pt}}
\pgflineto{\pgfpoint{229.220810pt}{106.798759pt}}
\pgfusepath{stroke}
\pgfpathmoveto{\pgfpoint{229.355652pt}{112.987907pt}}
\pgflineto{\pgfpoint{229.220810pt}{112.786522pt}}
\pgfusepath{stroke}
\pgfpathmoveto{\pgfpoint{229.364746pt}{118.979355pt}}
\pgflineto{\pgfpoint{229.220810pt}{118.774277pt}}
\pgfusepath{stroke}
\pgfpathmoveto{\pgfpoint{229.374680pt}{124.970634pt}}
\pgflineto{\pgfpoint{229.220810pt}{124.762024pt}}
\pgfusepath{stroke}
\pgfpathmoveto{\pgfpoint{229.385544pt}{130.961624pt}}
\pgflineto{\pgfpoint{229.220810pt}{130.749786pt}}
\pgfusepath{stroke}
\pgfpathmoveto{\pgfpoint{229.397415pt}{136.952209pt}}
\pgflineto{\pgfpoint{229.220810pt}{136.737534pt}}
\pgfusepath{stroke}
\pgfpathmoveto{\pgfpoint{229.410385pt}{142.942245pt}}
\pgflineto{\pgfpoint{229.220810pt}{142.725281pt}}
\pgfusepath{stroke}
\pgfpathmoveto{\pgfpoint{229.424530pt}{148.931549pt}}
\pgflineto{\pgfpoint{229.220810pt}{148.713058pt}}
\pgfusepath{stroke}
\pgfpathmoveto{\pgfpoint{229.439880pt}{154.919891pt}}
\pgflineto{\pgfpoint{229.220810pt}{154.700806pt}}
\pgfusepath{stroke}
\pgfpathmoveto{\pgfpoint{229.456467pt}{160.906952pt}}
\pgflineto{\pgfpoint{229.220810pt}{160.688568pt}}
\pgfusepath{stroke}
\pgfpathmoveto{\pgfpoint{229.474274pt}{166.892456pt}}
\pgflineto{\pgfpoint{229.220810pt}{166.676315pt}}
\pgfusepath{stroke}
\pgfpathmoveto{\pgfpoint{229.493195pt}{172.876007pt}}
\pgflineto{\pgfpoint{229.220810pt}{172.664078pt}}
\pgfusepath{stroke}
\pgfpathmoveto{\pgfpoint{229.512985pt}{178.857193pt}}
\pgflineto{\pgfpoint{229.220810pt}{178.651825pt}}
\pgfusepath{stroke}
\pgfpathmoveto{\pgfpoint{229.533325pt}{184.835602pt}}
\pgflineto{\pgfpoint{229.220810pt}{184.639587pt}}
\pgfusepath{stroke}
\pgfpathmoveto{\pgfpoint{229.553696pt}{190.810883pt}}
\pgflineto{\pgfpoint{229.220810pt}{190.627335pt}}
\pgfusepath{stroke}
\pgfpathmoveto{\pgfpoint{229.573441pt}{196.782730pt}}
\pgflineto{\pgfpoint{229.220810pt}{196.615097pt}}
\pgfusepath{stroke}
\pgfpathmoveto{\pgfpoint{229.591736pt}{202.751114pt}}
\pgflineto{\pgfpoint{229.220810pt}{202.602844pt}}
\pgfusepath{stroke}
\pgfpathmoveto{\pgfpoint{229.607697pt}{208.716187pt}}
\pgflineto{\pgfpoint{229.220810pt}{208.590607pt}}
\pgfusepath{stroke}
\pgfpathmoveto{\pgfpoint{229.620453pt}{214.678452pt}}
\pgflineto{\pgfpoint{229.220810pt}{214.578354pt}}
\pgfusepath{stroke}
\pgfpathmoveto{\pgfpoint{229.629303pt}{220.638672pt}}
\pgflineto{\pgfpoint{229.220810pt}{220.566116pt}}
\pgfusepath{stroke}
\pgfpathmoveto{\pgfpoint{229.633820pt}{226.597855pt}}
\pgflineto{\pgfpoint{229.220810pt}{226.553864pt}}
\pgfusepath{stroke}
\pgfpathmoveto{\pgfpoint{229.633942pt}{232.557098pt}}
\pgflineto{\pgfpoint{229.220810pt}{232.541626pt}}
\pgfusepath{stroke}
\pgfpathmoveto{\pgfpoint{229.629959pt}{238.517426pt}}
\pgflineto{\pgfpoint{229.220810pt}{238.529388pt}}
\pgfusepath{stroke}
\pgfpathmoveto{\pgfpoint{229.622467pt}{244.479660pt}}
\pgflineto{\pgfpoint{229.220810pt}{244.517136pt}}
\pgfusepath{stroke}
\pgfpathmoveto{\pgfpoint{229.612213pt}{250.444336pt}}
\pgflineto{\pgfpoint{229.220810pt}{250.504883pt}}
\pgfusepath{stroke}
\pgfpathmoveto{\pgfpoint{229.599976pt}{256.411682pt}}
\pgflineto{\pgfpoint{229.220810pt}{256.492645pt}}
\pgfusepath{stroke}
\pgfpathmoveto{\pgfpoint{229.586456pt}{262.381775pt}}
\pgflineto{\pgfpoint{229.220810pt}{262.480408pt}}
\pgfusepath{stroke}
\pgfpathmoveto{\pgfpoint{229.572266pt}{268.354370pt}}
\pgflineto{\pgfpoint{229.220810pt}{268.468170pt}}
\pgfusepath{stroke}
\pgfpathmoveto{\pgfpoint{229.557861pt}{274.329254pt}}
\pgflineto{\pgfpoint{229.220810pt}{274.455902pt}}
\pgfusepath{stroke}
\pgfpathmoveto{\pgfpoint{229.543503pt}{280.306122pt}}
\pgflineto{\pgfpoint{229.220810pt}{280.443665pt}}
\pgfusepath{stroke}
\pgfpathmoveto{\pgfpoint{229.529419pt}{286.284698pt}}
\pgflineto{\pgfpoint{229.220810pt}{286.431427pt}}
\pgfusepath{stroke}
\pgfpathmoveto{\pgfpoint{229.515717pt}{292.264740pt}}
\pgflineto{\pgfpoint{229.220810pt}{292.419189pt}}
\pgfusepath{stroke}
\pgfpathmoveto{\pgfpoint{229.502426pt}{298.246002pt}}
\pgflineto{\pgfpoint{229.220810pt}{298.406921pt}}
\pgfusepath{stroke}
\pgfpathmoveto{\pgfpoint{229.489563pt}{304.228333pt}}
\pgflineto{\pgfpoint{229.220810pt}{304.394714pt}}
\pgfusepath{stroke}
\pgfpathmoveto{\pgfpoint{229.477158pt}{310.211609pt}}
\pgflineto{\pgfpoint{229.220810pt}{310.382446pt}}
\pgfusepath{stroke}
\pgfpathmoveto{\pgfpoint{229.465179pt}{316.195679pt}}
\pgflineto{\pgfpoint{229.220810pt}{316.370178pt}}
\pgfusepath{stroke}
\pgfpathmoveto{\pgfpoint{229.453613pt}{322.180511pt}}
\pgflineto{\pgfpoint{229.220810pt}{322.357971pt}}
\pgfusepath{stroke}
\pgfpathmoveto{\pgfpoint{229.442474pt}{328.166016pt}}
\pgflineto{\pgfpoint{229.220810pt}{328.345703pt}}
\pgfusepath{stroke}
\pgfpathmoveto{\pgfpoint{229.431763pt}{334.152100pt}}
\pgflineto{\pgfpoint{229.220810pt}{334.333466pt}}
\pgfusepath{stroke}
\pgfpathmoveto{\pgfpoint{229.421463pt}{340.138794pt}}
\pgflineto{\pgfpoint{229.220810pt}{340.321228pt}}
\pgfusepath{stroke}
\pgfpathmoveto{\pgfpoint{229.411575pt}{346.125946pt}}
\pgflineto{\pgfpoint{229.220810pt}{346.308960pt}}
\pgfusepath{stroke}
\pgfpathmoveto{\pgfpoint{229.402115pt}{352.113586pt}}
\pgflineto{\pgfpoint{229.220810pt}{352.296722pt}}
\pgfusepath{stroke}
\pgfpathmoveto{\pgfpoint{229.393066pt}{358.101624pt}}
\pgflineto{\pgfpoint{229.220810pt}{358.284485pt}}
\pgfusepath{stroke}
\pgfpathmoveto{\pgfpoint{229.384445pt}{364.090027pt}}
\pgflineto{\pgfpoint{229.220810pt}{364.272247pt}}
\pgfusepath{stroke}
\pgfpathmoveto{\pgfpoint{229.376236pt}{370.078735pt}}
\pgflineto{\pgfpoint{229.220810pt}{370.260010pt}}
\pgfusepath{stroke}
\pgfpathmoveto{\pgfpoint{235.298737pt}{77.043472pt}}
\pgflineto{\pgfpoint{235.208557pt}{76.859985pt}}
\pgfusepath{stroke}
\pgfpathmoveto{\pgfpoint{235.304092pt}{83.035446pt}}
\pgflineto{\pgfpoint{235.208557pt}{82.847733pt}}
\pgfusepath{stroke}
\pgfpathmoveto{\pgfpoint{235.309937pt}{89.027534pt}}
\pgflineto{\pgfpoint{235.208557pt}{88.835495pt}}
\pgfusepath{stroke}
\pgfpathmoveto{\pgfpoint{235.316299pt}{95.019722pt}}
\pgflineto{\pgfpoint{235.208557pt}{94.823257pt}}
\pgfusepath{stroke}
\pgfpathmoveto{\pgfpoint{235.323257pt}{101.011986pt}}
\pgflineto{\pgfpoint{235.208557pt}{100.811012pt}}
\pgfusepath{stroke}
\pgfpathmoveto{\pgfpoint{235.330887pt}{107.004288pt}}
\pgflineto{\pgfpoint{235.208557pt}{106.798759pt}}
\pgfusepath{stroke}
\pgfpathmoveto{\pgfpoint{235.339249pt}{112.996635pt}}
\pgflineto{\pgfpoint{235.208557pt}{112.786522pt}}
\pgfusepath{stroke}
\pgfpathmoveto{\pgfpoint{235.348434pt}{118.988937pt}}
\pgflineto{\pgfpoint{235.208557pt}{118.774277pt}}
\pgfusepath{stroke}
\pgfpathmoveto{\pgfpoint{235.358566pt}{124.981140pt}}
\pgflineto{\pgfpoint{235.208557pt}{124.762024pt}}
\pgfusepath{stroke}
\pgfpathmoveto{\pgfpoint{235.369720pt}{130.973160pt}}
\pgflineto{\pgfpoint{235.208557pt}{130.749786pt}}
\pgfusepath{stroke}
\pgfpathmoveto{\pgfpoint{235.382019pt}{136.964874pt}}
\pgflineto{\pgfpoint{235.208557pt}{136.737534pt}}
\pgfusepath{stroke}
\pgfpathmoveto{\pgfpoint{235.395569pt}{142.956146pt}}
\pgflineto{\pgfpoint{235.208557pt}{142.725281pt}}
\pgfusepath{stroke}
\pgfpathmoveto{\pgfpoint{235.410522pt}{148.946762pt}}
\pgflineto{\pgfpoint{235.208557pt}{148.713058pt}}
\pgfusepath{stroke}
\pgfpathmoveto{\pgfpoint{235.426956pt}{154.936462pt}}
\pgflineto{\pgfpoint{235.208557pt}{154.700806pt}}
\pgfusepath{stroke}
\pgfpathmoveto{\pgfpoint{235.444977pt}{160.924957pt}}
\pgflineto{\pgfpoint{235.208557pt}{160.688568pt}}
\pgfusepath{stroke}
\pgfpathmoveto{\pgfpoint{235.464569pt}{166.911835pt}}
\pgflineto{\pgfpoint{235.208557pt}{166.676315pt}}
\pgfusepath{stroke}
\pgfpathmoveto{\pgfpoint{235.485733pt}{172.896637pt}}
\pgflineto{\pgfpoint{235.208557pt}{172.664078pt}}
\pgfusepath{stroke}
\pgfpathmoveto{\pgfpoint{235.508286pt}{178.878784pt}}
\pgflineto{\pgfpoint{235.208557pt}{178.651825pt}}
\pgfusepath{stroke}
\pgfpathmoveto{\pgfpoint{235.531860pt}{184.857742pt}}
\pgflineto{\pgfpoint{235.208557pt}{184.639587pt}}
\pgfusepath{stroke}
\pgfpathmoveto{\pgfpoint{235.555878pt}{190.832901pt}}
\pgflineto{\pgfpoint{235.208557pt}{190.627335pt}}
\pgfusepath{stroke}
\pgfpathmoveto{\pgfpoint{235.579559pt}{196.803802pt}}
\pgflineto{\pgfpoint{235.208557pt}{196.615097pt}}
\pgfusepath{stroke}
\pgfpathmoveto{\pgfpoint{235.601807pt}{202.770218pt}}
\pgflineto{\pgfpoint{235.208557pt}{202.602844pt}}
\pgfusepath{stroke}
\pgfpathmoveto{\pgfpoint{235.621368pt}{208.732254pt}}
\pgflineto{\pgfpoint{235.208557pt}{208.590607pt}}
\pgfusepath{stroke}
\pgfpathmoveto{\pgfpoint{235.637009pt}{214.690491pt}}
\pgflineto{\pgfpoint{235.208557pt}{214.578354pt}}
\pgfusepath{stroke}
\pgfpathmoveto{\pgfpoint{235.647675pt}{220.645966pt}}
\pgflineto{\pgfpoint{235.208557pt}{220.566116pt}}
\pgfusepath{stroke}
\pgfpathmoveto{\pgfpoint{235.652710pt}{226.600098pt}}
\pgflineto{\pgfpoint{235.208557pt}{226.553864pt}}
\pgfusepath{stroke}
\pgfpathmoveto{\pgfpoint{235.652054pt}{232.554459pt}}
\pgflineto{\pgfpoint{235.208557pt}{232.541626pt}}
\pgfusepath{stroke}
\pgfpathmoveto{\pgfpoint{235.646194pt}{238.510498pt}}
\pgflineto{\pgfpoint{235.208557pt}{238.529388pt}}
\pgfusepath{stroke}
\pgfpathmoveto{\pgfpoint{235.636047pt}{244.469269pt}}
\pgflineto{\pgfpoint{235.208557pt}{244.517136pt}}
\pgfusepath{stroke}
\pgfpathmoveto{\pgfpoint{235.622726pt}{250.431458pt}}
\pgflineto{\pgfpoint{235.208557pt}{250.504883pt}}
\pgfusepath{stroke}
\pgfpathmoveto{\pgfpoint{235.607361pt}{256.397247pt}}
\pgflineto{\pgfpoint{235.208557pt}{256.492645pt}}
\pgfusepath{stroke}
\pgfpathmoveto{\pgfpoint{235.590881pt}{262.366455pt}}
\pgflineto{\pgfpoint{235.208557pt}{262.480408pt}}
\pgfusepath{stroke}
\pgfpathmoveto{\pgfpoint{235.574051pt}{268.338806pt}}
\pgflineto{\pgfpoint{235.208557pt}{268.468170pt}}
\pgfusepath{stroke}
\pgfpathmoveto{\pgfpoint{235.557343pt}{274.313782pt}}
\pgflineto{\pgfpoint{235.208557pt}{274.455902pt}}
\pgfusepath{stroke}
\pgfpathmoveto{\pgfpoint{235.541077pt}{280.291016pt}}
\pgflineto{\pgfpoint{235.208557pt}{280.443665pt}}
\pgfusepath{stroke}
\pgfpathmoveto{\pgfpoint{235.525375pt}{286.270050pt}}
\pgflineto{\pgfpoint{235.208557pt}{286.431427pt}}
\pgfusepath{stroke}
\pgfpathmoveto{\pgfpoint{235.510300pt}{292.250610pt}}
\pgflineto{\pgfpoint{235.208557pt}{292.419189pt}}
\pgfusepath{stroke}
\pgfpathmoveto{\pgfpoint{235.495865pt}{298.232391pt}}
\pgflineto{\pgfpoint{235.208557pt}{298.406921pt}}
\pgfusepath{stroke}
\pgfpathmoveto{\pgfpoint{235.482025pt}{304.215240pt}}
\pgflineto{\pgfpoint{235.208557pt}{304.394714pt}}
\pgfusepath{stroke}
\pgfpathmoveto{\pgfpoint{235.468750pt}{310.198975pt}}
\pgflineto{\pgfpoint{235.208557pt}{310.382446pt}}
\pgfusepath{stroke}
\pgfpathmoveto{\pgfpoint{235.455994pt}{316.183533pt}}
\pgflineto{\pgfpoint{235.208557pt}{316.370178pt}}
\pgfusepath{stroke}
\pgfpathmoveto{\pgfpoint{235.443741pt}{322.168823pt}}
\pgflineto{\pgfpoint{235.208557pt}{322.357971pt}}
\pgfusepath{stroke}
\pgfpathmoveto{\pgfpoint{235.431961pt}{328.154785pt}}
\pgflineto{\pgfpoint{235.208557pt}{328.345703pt}}
\pgfusepath{stroke}
\pgfpathmoveto{\pgfpoint{235.420654pt}{334.141327pt}}
\pgflineto{\pgfpoint{235.208557pt}{334.333466pt}}
\pgfusepath{stroke}
\pgfpathmoveto{\pgfpoint{235.409821pt}{340.128418pt}}
\pgflineto{\pgfpoint{235.208557pt}{340.321228pt}}
\pgfusepath{stroke}
\pgfpathmoveto{\pgfpoint{235.399460pt}{346.116028pt}}
\pgflineto{\pgfpoint{235.208557pt}{346.308960pt}}
\pgfusepath{stroke}
\pgfpathmoveto{\pgfpoint{235.389557pt}{352.104126pt}}
\pgflineto{\pgfpoint{235.208557pt}{352.296722pt}}
\pgfusepath{stroke}
\pgfpathmoveto{\pgfpoint{235.380127pt}{358.092590pt}}
\pgflineto{\pgfpoint{235.208557pt}{358.284485pt}}
\pgfusepath{stroke}
\pgfpathmoveto{\pgfpoint{235.371170pt}{364.081451pt}}
\pgflineto{\pgfpoint{235.208557pt}{364.272247pt}}
\pgfusepath{stroke}
\pgfpathmoveto{\pgfpoint{235.362656pt}{370.070618pt}}
\pgflineto{\pgfpoint{235.208557pt}{370.260010pt}}
\pgfusepath{stroke}
\pgfpathmoveto{\pgfpoint{241.282104pt}{77.048538pt}}
\pgflineto{\pgfpoint{241.196320pt}{76.859985pt}}
\pgfusepath{stroke}
\pgfpathmoveto{\pgfpoint{241.287323pt}{83.040970pt}}
\pgflineto{\pgfpoint{241.196320pt}{82.847733pt}}
\pgfusepath{stroke}
\pgfpathmoveto{\pgfpoint{241.293030pt}{89.033569pt}}
\pgflineto{\pgfpoint{241.196320pt}{88.835495pt}}
\pgfusepath{stroke}
\pgfpathmoveto{\pgfpoint{241.299286pt}{95.026329pt}}
\pgflineto{\pgfpoint{241.196320pt}{94.823257pt}}
\pgfusepath{stroke}
\pgfpathmoveto{\pgfpoint{241.306137pt}{101.019234pt}}
\pgflineto{\pgfpoint{241.196320pt}{100.811012pt}}
\pgfusepath{stroke}
\pgfpathmoveto{\pgfpoint{241.313690pt}{107.012260pt}}
\pgflineto{\pgfpoint{241.196320pt}{106.798759pt}}
\pgfusepath{stroke}
\pgfpathmoveto{\pgfpoint{241.322006pt}{113.005402pt}}
\pgflineto{\pgfpoint{241.196320pt}{112.786522pt}}
\pgfusepath{stroke}
\pgfpathmoveto{\pgfpoint{241.331207pt}{118.998604pt}}
\pgflineto{\pgfpoint{241.196320pt}{118.774277pt}}
\pgfusepath{stroke}
\pgfpathmoveto{\pgfpoint{241.341415pt}{124.991829pt}}
\pgflineto{\pgfpoint{241.196320pt}{124.762024pt}}
\pgfusepath{stroke}
\pgfpathmoveto{\pgfpoint{241.352737pt}{130.985001pt}}
\pgflineto{\pgfpoint{241.196320pt}{130.749786pt}}
\pgfusepath{stroke}
\pgfpathmoveto{\pgfpoint{241.365341pt}{136.977997pt}}
\pgflineto{\pgfpoint{241.196320pt}{136.737534pt}}
\pgfusepath{stroke}
\pgfpathmoveto{\pgfpoint{241.379379pt}{142.970688pt}}
\pgflineto{\pgfpoint{241.196320pt}{142.725281pt}}
\pgfusepath{stroke}
\pgfpathmoveto{\pgfpoint{241.395020pt}{148.962860pt}}
\pgflineto{\pgfpoint{241.196320pt}{148.713058pt}}
\pgfusepath{stroke}
\pgfpathmoveto{\pgfpoint{241.412460pt}{154.954285pt}}
\pgflineto{\pgfpoint{241.196320pt}{154.700806pt}}
\pgfusepath{stroke}
\pgfpathmoveto{\pgfpoint{241.431839pt}{160.944580pt}}
\pgflineto{\pgfpoint{241.196320pt}{160.688568pt}}
\pgfusepath{stroke}
\pgfpathmoveto{\pgfpoint{241.453308pt}{166.933289pt}}
\pgflineto{\pgfpoint{241.196320pt}{166.676315pt}}
\pgfusepath{stroke}
\pgfpathmoveto{\pgfpoint{241.476883pt}{172.919876pt}}
\pgflineto{\pgfpoint{241.196320pt}{172.664078pt}}
\pgfusepath{stroke}
\pgfpathmoveto{\pgfpoint{241.502502pt}{178.903595pt}}
\pgflineto{\pgfpoint{241.196320pt}{178.651825pt}}
\pgfusepath{stroke}
\pgfpathmoveto{\pgfpoint{241.529877pt}{184.883667pt}}
\pgflineto{\pgfpoint{241.196320pt}{184.639587pt}}
\pgfusepath{stroke}
\pgfpathmoveto{\pgfpoint{241.558380pt}{190.859222pt}}
\pgflineto{\pgfpoint{241.196320pt}{190.627335pt}}
\pgfusepath{stroke}
\pgfpathmoveto{\pgfpoint{241.587082pt}{196.829437pt}}
\pgflineto{\pgfpoint{241.196320pt}{196.615097pt}}
\pgfusepath{stroke}
\pgfpathmoveto{\pgfpoint{241.614532pt}{202.793793pt}}
\pgflineto{\pgfpoint{241.196320pt}{202.602844pt}}
\pgfusepath{stroke}
\pgfpathmoveto{\pgfpoint{241.639008pt}{208.752213pt}}
\pgflineto{\pgfpoint{241.196320pt}{208.590607pt}}
\pgfusepath{stroke}
\pgfpathmoveto{\pgfpoint{241.658630pt}{214.705338pt}}
\pgflineto{\pgfpoint{241.196320pt}{214.578354pt}}
\pgfusepath{stroke}
\pgfpathmoveto{\pgfpoint{241.671753pt}{220.654617pt}}
\pgflineto{\pgfpoint{241.196320pt}{220.566116pt}}
\pgfusepath{stroke}
\pgfpathmoveto{\pgfpoint{241.677383pt}{226.602097pt}}
\pgflineto{\pgfpoint{241.196320pt}{226.553864pt}}
\pgfusepath{stroke}
\pgfpathmoveto{\pgfpoint{241.675446pt}{232.550095pt}}
\pgflineto{\pgfpoint{241.196320pt}{232.541626pt}}
\pgfusepath{stroke}
\pgfpathmoveto{\pgfpoint{241.666748pt}{238.500717pt}}
\pgflineto{\pgfpoint{241.196320pt}{238.529388pt}}
\pgfusepath{stroke}
\pgfpathmoveto{\pgfpoint{241.652802pt}{244.455414pt}}
\pgflineto{\pgfpoint{241.196320pt}{244.517136pt}}
\pgfusepath{stroke}
\pgfpathmoveto{\pgfpoint{241.635269pt}{250.414932pt}}
\pgflineto{\pgfpoint{241.196320pt}{250.504883pt}}
\pgfusepath{stroke}
\pgfpathmoveto{\pgfpoint{241.615799pt}{256.379272pt}}
\pgflineto{\pgfpoint{241.196320pt}{256.492645pt}}
\pgfusepath{stroke}
\pgfpathmoveto{\pgfpoint{241.595642pt}{262.348053pt}}
\pgflineto{\pgfpoint{241.196320pt}{262.480408pt}}
\pgfusepath{stroke}
\pgfpathmoveto{\pgfpoint{241.575653pt}{268.320557pt}}
\pgflineto{\pgfpoint{241.196320pt}{268.468170pt}}
\pgfusepath{stroke}
\pgfpathmoveto{\pgfpoint{241.556366pt}{274.296143pt}}
\pgflineto{\pgfpoint{241.196320pt}{274.455902pt}}
\pgfusepath{stroke}
\pgfpathmoveto{\pgfpoint{241.538025pt}{280.274109pt}}
\pgflineto{\pgfpoint{241.196320pt}{280.443665pt}}
\pgfusepath{stroke}
\pgfpathmoveto{\pgfpoint{241.520660pt}{286.253906pt}}
\pgflineto{\pgfpoint{241.196320pt}{286.431427pt}}
\pgfusepath{stroke}
\pgfpathmoveto{\pgfpoint{241.504242pt}{292.235229pt}}
\pgflineto{\pgfpoint{241.196320pt}{292.419189pt}}
\pgfusepath{stroke}
\pgfpathmoveto{\pgfpoint{241.488708pt}{298.217743pt}}
\pgflineto{\pgfpoint{241.196320pt}{298.406921pt}}
\pgfusepath{stroke}
\pgfpathmoveto{\pgfpoint{241.473907pt}{304.201233pt}}
\pgflineto{\pgfpoint{241.196320pt}{304.394714pt}}
\pgfusepath{stroke}
\pgfpathmoveto{\pgfpoint{241.459808pt}{310.185608pt}}
\pgflineto{\pgfpoint{241.196320pt}{310.382446pt}}
\pgfusepath{stroke}
\pgfpathmoveto{\pgfpoint{241.446304pt}{316.170715pt}}
\pgflineto{\pgfpoint{241.196320pt}{316.370178pt}}
\pgfusepath{stroke}
\pgfpathmoveto{\pgfpoint{241.433380pt}{322.156494pt}}
\pgflineto{\pgfpoint{241.196320pt}{322.357971pt}}
\pgfusepath{stroke}
\pgfpathmoveto{\pgfpoint{241.420959pt}{328.142944pt}}
\pgflineto{\pgfpoint{241.196320pt}{328.345703pt}}
\pgfusepath{stroke}
\pgfpathmoveto{\pgfpoint{241.409088pt}{334.129974pt}}
\pgflineto{\pgfpoint{241.196320pt}{334.333466pt}}
\pgfusepath{stroke}
\pgfpathmoveto{\pgfpoint{241.397736pt}{340.117584pt}}
\pgflineto{\pgfpoint{241.196320pt}{340.321228pt}}
\pgfusepath{stroke}
\pgfpathmoveto{\pgfpoint{241.386887pt}{346.105713pt}}
\pgflineto{\pgfpoint{241.196320pt}{346.308960pt}}
\pgfusepath{stroke}
\pgfpathmoveto{\pgfpoint{241.376556pt}{352.094238pt}}
\pgflineto{\pgfpoint{241.196320pt}{352.296722pt}}
\pgfusepath{stroke}
\pgfpathmoveto{\pgfpoint{241.366730pt}{358.083191pt}}
\pgflineto{\pgfpoint{241.196320pt}{358.284485pt}}
\pgfusepath{stroke}
\pgfpathmoveto{\pgfpoint{241.357422pt}{364.072571pt}}
\pgflineto{\pgfpoint{241.196320pt}{364.272247pt}}
\pgfusepath{stroke}
\pgfpathmoveto{\pgfpoint{241.348633pt}{370.062195pt}}
\pgflineto{\pgfpoint{241.196320pt}{370.260010pt}}
\pgfusepath{stroke}
\pgfpathmoveto{\pgfpoint{247.265045pt}{77.053452pt}}
\pgflineto{\pgfpoint{247.184067pt}{76.859985pt}}
\pgfusepath{stroke}
\pgfpathmoveto{\pgfpoint{247.270081pt}{83.046326pt}}
\pgflineto{\pgfpoint{247.184067pt}{82.847733pt}}
\pgfusepath{stroke}
\pgfpathmoveto{\pgfpoint{247.275604pt}{89.039444pt}}
\pgflineto{\pgfpoint{247.184067pt}{88.835495pt}}
\pgfusepath{stroke}
\pgfpathmoveto{\pgfpoint{247.281647pt}{95.032776pt}}
\pgflineto{\pgfpoint{247.184067pt}{94.823257pt}}
\pgfusepath{stroke}
\pgfpathmoveto{\pgfpoint{247.288330pt}{101.026344pt}}
\pgflineto{\pgfpoint{247.184067pt}{100.811012pt}}
\pgfusepath{stroke}
\pgfpathmoveto{\pgfpoint{247.295715pt}{107.020103pt}}
\pgflineto{\pgfpoint{247.184067pt}{106.798759pt}}
\pgfusepath{stroke}
\pgfpathmoveto{\pgfpoint{247.303894pt}{113.014076pt}}
\pgflineto{\pgfpoint{247.184067pt}{112.786522pt}}
\pgfusepath{stroke}
\pgfpathmoveto{\pgfpoint{247.312988pt}{119.008247pt}}
\pgflineto{\pgfpoint{247.184067pt}{118.774277pt}}
\pgfusepath{stroke}
\pgfpathmoveto{\pgfpoint{247.323135pt}{125.002571pt}}
\pgflineto{\pgfpoint{247.184067pt}{124.762024pt}}
\pgfusepath{stroke}
\pgfpathmoveto{\pgfpoint{247.334488pt}{130.996979pt}}
\pgflineto{\pgfpoint{247.184067pt}{130.749786pt}}
\pgfusepath{stroke}
\pgfpathmoveto{\pgfpoint{247.347229pt}{136.991409pt}}
\pgflineto{\pgfpoint{247.184067pt}{136.737534pt}}
\pgfusepath{stroke}
\pgfpathmoveto{\pgfpoint{247.361572pt}{142.985718pt}}
\pgflineto{\pgfpoint{247.184067pt}{142.725281pt}}
\pgfusepath{stroke}
\pgfpathmoveto{\pgfpoint{247.377747pt}{148.979736pt}}
\pgflineto{\pgfpoint{247.184067pt}{148.713058pt}}
\pgfusepath{stroke}
\pgfpathmoveto{\pgfpoint{247.396011pt}{154.973190pt}}
\pgflineto{\pgfpoint{247.184067pt}{154.700806pt}}
\pgfusepath{stroke}
\pgfpathmoveto{\pgfpoint{247.416626pt}{160.965729pt}}
\pgflineto{\pgfpoint{247.184067pt}{160.688568pt}}
\pgfusepath{stroke}
\pgfpathmoveto{\pgfpoint{247.439880pt}{166.956879pt}}
\pgflineto{\pgfpoint{247.184067pt}{166.676315pt}}
\pgfusepath{stroke}
\pgfpathmoveto{\pgfpoint{247.465942pt}{172.945953pt}}
\pgflineto{\pgfpoint{247.184067pt}{172.664078pt}}
\pgfusepath{stroke}
\pgfpathmoveto{\pgfpoint{247.494934pt}{178.932068pt}}
\pgflineto{\pgfpoint{247.184067pt}{178.651825pt}}
\pgfusepath{stroke}
\pgfpathmoveto{\pgfpoint{247.526672pt}{184.914139pt}}
\pgflineto{\pgfpoint{247.184067pt}{184.639587pt}}
\pgfusepath{stroke}
\pgfpathmoveto{\pgfpoint{247.560654pt}{190.890900pt}}
\pgflineto{\pgfpoint{247.184067pt}{190.627335pt}}
\pgfusepath{stroke}
\pgfpathmoveto{\pgfpoint{247.595764pt}{196.861023pt}}
\pgflineto{\pgfpoint{247.184067pt}{196.615097pt}}
\pgfusepath{stroke}
\pgfpathmoveto{\pgfpoint{247.630219pt}{202.823410pt}}
\pgflineto{\pgfpoint{247.184067pt}{202.602844pt}}
\pgfusepath{stroke}
\pgfpathmoveto{\pgfpoint{247.661530pt}{208.777588pt}}
\pgflineto{\pgfpoint{247.184067pt}{208.590607pt}}
\pgfusepath{stroke}
\pgfpathmoveto{\pgfpoint{247.686829pt}{214.724167pt}}
\pgflineto{\pgfpoint{247.184067pt}{214.578354pt}}
\pgfusepath{stroke}
\pgfpathmoveto{\pgfpoint{247.703476pt}{220.665100pt}}
\pgflineto{\pgfpoint{247.184067pt}{220.566116pt}}
\pgfusepath{stroke}
\pgfpathmoveto{\pgfpoint{247.709793pt}{226.603516pt}}
\pgflineto{\pgfpoint{247.184067pt}{226.553864pt}}
\pgfusepath{stroke}
\pgfpathmoveto{\pgfpoint{247.705688pt}{232.542999pt}}
\pgflineto{\pgfpoint{247.184067pt}{232.541626pt}}
\pgfusepath{stroke}
\pgfpathmoveto{\pgfpoint{247.692627pt}{238.486710pt}}
\pgflineto{\pgfpoint{247.184067pt}{238.529388pt}}
\pgfusepath{stroke}
\pgfpathmoveto{\pgfpoint{247.673065pt}{244.436646pt}}
\pgflineto{\pgfpoint{247.184067pt}{244.517136pt}}
\pgfusepath{stroke}
\pgfpathmoveto{\pgfpoint{247.649719pt}{250.393494pt}}
\pgflineto{\pgfpoint{247.184067pt}{250.504883pt}}
\pgfusepath{stroke}
\pgfpathmoveto{\pgfpoint{247.624863pt}{256.356873pt}}
\pgflineto{\pgfpoint{247.184067pt}{256.492645pt}}
\pgfusepath{stroke}
\pgfpathmoveto{\pgfpoint{247.600143pt}{262.325867pt}}
\pgflineto{\pgfpoint{247.184067pt}{262.480408pt}}
\pgfusepath{stroke}
\pgfpathmoveto{\pgfpoint{247.576538pt}{268.299316pt}}
\pgflineto{\pgfpoint{247.184067pt}{268.468170pt}}
\pgfusepath{stroke}
\pgfpathmoveto{\pgfpoint{247.554443pt}{274.276093pt}}
\pgflineto{\pgfpoint{247.184067pt}{274.455902pt}}
\pgfusepath{stroke}
\pgfpathmoveto{\pgfpoint{247.533981pt}{280.255310pt}}
\pgflineto{\pgfpoint{247.184067pt}{280.443665pt}}
\pgfusepath{stroke}
\pgfpathmoveto{\pgfpoint{247.515015pt}{286.236328pt}}
\pgflineto{\pgfpoint{247.184067pt}{286.431427pt}}
\pgfusepath{stroke}
\pgfpathmoveto{\pgfpoint{247.497345pt}{292.218689pt}}
\pgflineto{\pgfpoint{247.184067pt}{292.419189pt}}
\pgfusepath{stroke}
\pgfpathmoveto{\pgfpoint{247.480774pt}{298.202118pt}}
\pgflineto{\pgfpoint{247.184067pt}{298.406921pt}}
\pgfusepath{stroke}
\pgfpathmoveto{\pgfpoint{247.465134pt}{304.186401pt}}
\pgflineto{\pgfpoint{247.184067pt}{304.394714pt}}
\pgfusepath{stroke}
\pgfpathmoveto{\pgfpoint{247.450256pt}{310.171417pt}}
\pgflineto{\pgfpoint{247.184067pt}{310.382446pt}}
\pgfusepath{stroke}
\pgfpathmoveto{\pgfpoint{247.436066pt}{316.157135pt}}
\pgflineto{\pgfpoint{247.184067pt}{316.370178pt}}
\pgfusepath{stroke}
\pgfpathmoveto{\pgfpoint{247.422470pt}{322.143524pt}}
\pgflineto{\pgfpoint{247.184067pt}{322.357971pt}}
\pgfusepath{stroke}
\pgfpathmoveto{\pgfpoint{247.409454pt}{328.130493pt}}
\pgflineto{\pgfpoint{247.184067pt}{328.345703pt}}
\pgfusepath{stroke}
\pgfpathmoveto{\pgfpoint{247.397003pt}{334.118103pt}}
\pgflineto{\pgfpoint{247.184067pt}{334.333466pt}}
\pgfusepath{stroke}
\pgfpathmoveto{\pgfpoint{247.385117pt}{340.106232pt}}
\pgflineto{\pgfpoint{247.184067pt}{340.321228pt}}
\pgfusepath{stroke}
\pgfpathmoveto{\pgfpoint{247.373795pt}{346.094849pt}}
\pgflineto{\pgfpoint{247.184067pt}{346.308960pt}}
\pgfusepath{stroke}
\pgfpathmoveto{\pgfpoint{247.363022pt}{352.083984pt}}
\pgflineto{\pgfpoint{247.184067pt}{352.296722pt}}
\pgfusepath{stroke}
\pgfpathmoveto{\pgfpoint{247.352829pt}{358.073486pt}}
\pgflineto{\pgfpoint{247.184067pt}{358.284485pt}}
\pgfusepath{stroke}
\pgfpathmoveto{\pgfpoint{247.343201pt}{364.063354pt}}
\pgflineto{\pgfpoint{247.184067pt}{364.272247pt}}
\pgfusepath{stroke}
\pgfpathmoveto{\pgfpoint{247.334106pt}{370.053497pt}}
\pgflineto{\pgfpoint{247.184067pt}{370.260010pt}}
\pgfusepath{stroke}
\pgfpathmoveto{\pgfpoint{253.247559pt}{77.058151pt}}
\pgflineto{\pgfpoint{253.171829pt}{76.859985pt}}
\pgfusepath{stroke}
\pgfpathmoveto{\pgfpoint{253.252350pt}{83.051483pt}}
\pgflineto{\pgfpoint{253.171829pt}{82.847733pt}}
\pgfusepath{stroke}
\pgfpathmoveto{\pgfpoint{253.257614pt}{89.045097pt}}
\pgflineto{\pgfpoint{253.171829pt}{88.835495pt}}
\pgfusepath{stroke}
\pgfpathmoveto{\pgfpoint{253.263412pt}{95.039009pt}}
\pgflineto{\pgfpoint{253.171829pt}{94.823257pt}}
\pgfusepath{stroke}
\pgfpathmoveto{\pgfpoint{253.269821pt}{101.033218pt}}
\pgflineto{\pgfpoint{253.171829pt}{100.811012pt}}
\pgfusepath{stroke}
\pgfpathmoveto{\pgfpoint{253.276947pt}{107.027748pt}}
\pgflineto{\pgfpoint{253.171829pt}{106.798759pt}}
\pgfusepath{stroke}
\pgfpathmoveto{\pgfpoint{253.284851pt}{113.022591pt}}
\pgflineto{\pgfpoint{253.171829pt}{112.786522pt}}
\pgfusepath{stroke}
\pgfpathmoveto{\pgfpoint{253.293716pt}{119.017754pt}}
\pgflineto{\pgfpoint{253.171829pt}{118.774277pt}}
\pgfusepath{stroke}
\pgfpathmoveto{\pgfpoint{253.303650pt}{125.013222pt}}
\pgflineto{\pgfpoint{253.171829pt}{124.762024pt}}
\pgfusepath{stroke}
\pgfpathmoveto{\pgfpoint{253.314850pt}{131.008972pt}}
\pgflineto{\pgfpoint{253.171829pt}{130.749786pt}}
\pgfusepath{stroke}
\pgfpathmoveto{\pgfpoint{253.327530pt}{137.004929pt}}
\pgflineto{\pgfpoint{253.171829pt}{136.737534pt}}
\pgfusepath{stroke}
\pgfpathmoveto{\pgfpoint{253.341934pt}{143.001038pt}}
\pgflineto{\pgfpoint{253.171829pt}{142.725281pt}}
\pgfusepath{stroke}
\pgfpathmoveto{\pgfpoint{253.358368pt}{148.997131pt}}
\pgflineto{\pgfpoint{253.171829pt}{148.713058pt}}
\pgfusepath{stroke}
\pgfpathmoveto{\pgfpoint{253.377197pt}{154.992981pt}}
\pgflineto{\pgfpoint{253.171829pt}{154.700806pt}}
\pgfusepath{stroke}
\pgfpathmoveto{\pgfpoint{253.398788pt}{160.988281pt}}
\pgflineto{\pgfpoint{253.171829pt}{160.688568pt}}
\pgfusepath{stroke}
\pgfpathmoveto{\pgfpoint{253.423599pt}{166.982498pt}}
\pgflineto{\pgfpoint{253.171829pt}{166.676315pt}}
\pgfusepath{stroke}
\pgfpathmoveto{\pgfpoint{253.452057pt}{172.974930pt}}
\pgflineto{\pgfpoint{253.171829pt}{172.664078pt}}
\pgfusepath{stroke}
\pgfpathmoveto{\pgfpoint{253.484528pt}{178.964539pt}}
\pgflineto{\pgfpoint{253.171829pt}{178.651825pt}}
\pgfusepath{stroke}
\pgfpathmoveto{\pgfpoint{253.521164pt}{184.949905pt}}
\pgflineto{\pgfpoint{253.171829pt}{184.639587pt}}
\pgfusepath{stroke}
\pgfpathmoveto{\pgfpoint{253.561661pt}{190.929245pt}}
\pgflineto{\pgfpoint{253.171829pt}{190.627335pt}}
\pgfusepath{stroke}
\pgfpathmoveto{\pgfpoint{253.604980pt}{196.900452pt}}
\pgflineto{\pgfpoint{253.171829pt}{196.615097pt}}
\pgfusepath{stroke}
\pgfpathmoveto{\pgfpoint{253.648956pt}{202.861420pt}}
\pgflineto{\pgfpoint{253.171829pt}{202.602844pt}}
\pgfusepath{stroke}
\pgfpathmoveto{\pgfpoint{253.690094pt}{208.810806pt}}
\pgflineto{\pgfpoint{253.171829pt}{208.590607pt}}
\pgfusepath{stroke}
\pgfpathmoveto{\pgfpoint{253.723892pt}{214.748856pt}}
\pgflineto{\pgfpoint{253.171829pt}{214.578354pt}}
\pgfusepath{stroke}
\pgfpathmoveto{\pgfpoint{253.745804pt}{220.678207pt}}
\pgflineto{\pgfpoint{253.171829pt}{220.566116pt}}
\pgfusepath{stroke}
\pgfpathmoveto{\pgfpoint{253.752899pt}{226.603790pt}}
\pgflineto{\pgfpoint{253.171829pt}{226.553864pt}}
\pgfusepath{stroke}
\pgfpathmoveto{\pgfpoint{253.745071pt}{232.531494pt}}
\pgflineto{\pgfpoint{253.171829pt}{232.541626pt}}
\pgfusepath{stroke}
\pgfpathmoveto{\pgfpoint{253.725021pt}{238.466248pt}}
\pgflineto{\pgfpoint{253.171829pt}{238.529388pt}}
\pgfusepath{stroke}
\pgfpathmoveto{\pgfpoint{253.697083pt}{244.410767pt}}
\pgflineto{\pgfpoint{253.171829pt}{244.517136pt}}
\pgfusepath{stroke}
\pgfpathmoveto{\pgfpoint{253.665527pt}{250.365387pt}}
\pgflineto{\pgfpoint{253.171829pt}{250.504883pt}}
\pgfusepath{stroke}
\pgfpathmoveto{\pgfpoint{253.633667pt}{256.328857pt}}
\pgflineto{\pgfpoint{253.171829pt}{256.492645pt}}
\pgfusepath{stroke}
\pgfpathmoveto{\pgfpoint{253.603455pt}{262.299286pt}}
\pgflineto{\pgfpoint{253.171829pt}{262.480408pt}}
\pgfusepath{stroke}
\pgfpathmoveto{\pgfpoint{253.575806pt}{268.274719pt}}
\pgflineto{\pgfpoint{253.171829pt}{268.468170pt}}
\pgfusepath{stroke}
\pgfpathmoveto{\pgfpoint{253.550873pt}{274.253601pt}}
\pgflineto{\pgfpoint{253.171829pt}{274.455902pt}}
\pgfusepath{stroke}
\pgfpathmoveto{\pgfpoint{253.528412pt}{280.234772pt}}
\pgflineto{\pgfpoint{253.171829pt}{280.443665pt}}
\pgfusepath{stroke}
\pgfpathmoveto{\pgfpoint{253.508041pt}{286.217438pt}}
\pgflineto{\pgfpoint{253.171829pt}{286.431427pt}}
\pgfusepath{stroke}
\pgfpathmoveto{\pgfpoint{253.489349pt}{292.201141pt}}
\pgflineto{\pgfpoint{253.171829pt}{292.419189pt}}
\pgfusepath{stroke}
\pgfpathmoveto{\pgfpoint{253.471954pt}{298.185638pt}}
\pgflineto{\pgfpoint{253.171829pt}{298.406921pt}}
\pgfusepath{stroke}
\pgfpathmoveto{\pgfpoint{253.455582pt}{304.170807pt}}
\pgflineto{\pgfpoint{253.171829pt}{304.394714pt}}
\pgfusepath{stroke}
\pgfpathmoveto{\pgfpoint{253.440048pt}{310.156555pt}}
\pgflineto{\pgfpoint{253.171829pt}{310.382446pt}}
\pgfusepath{stroke}
\pgfpathmoveto{\pgfpoint{253.425201pt}{316.142944pt}}
\pgflineto{\pgfpoint{253.171829pt}{316.370178pt}}
\pgfusepath{stroke}
\pgfpathmoveto{\pgfpoint{253.410995pt}{322.129913pt}}
\pgflineto{\pgfpoint{253.171829pt}{322.357971pt}}
\pgfusepath{stroke}
\pgfpathmoveto{\pgfpoint{253.397385pt}{328.117493pt}}
\pgflineto{\pgfpoint{253.171829pt}{328.345703pt}}
\pgfusepath{stroke}
\pgfpathmoveto{\pgfpoint{253.384369pt}{334.105652pt}}
\pgflineto{\pgfpoint{253.171829pt}{334.333466pt}}
\pgfusepath{stroke}
\pgfpathmoveto{\pgfpoint{253.371948pt}{340.094360pt}}
\pgflineto{\pgfpoint{253.171829pt}{340.321228pt}}
\pgfusepath{stroke}
\pgfpathmoveto{\pgfpoint{253.360138pt}{346.083588pt}}
\pgflineto{\pgfpoint{253.171829pt}{346.308960pt}}
\pgfusepath{stroke}
\pgfpathmoveto{\pgfpoint{253.348938pt}{352.073303pt}}
\pgflineto{\pgfpoint{253.171829pt}{352.296722pt}}
\pgfusepath{stroke}
\pgfpathmoveto{\pgfpoint{253.338348pt}{358.063416pt}}
\pgflineto{\pgfpoint{253.171829pt}{358.284485pt}}
\pgfusepath{stroke}
\pgfpathmoveto{\pgfpoint{253.328400pt}{364.053833pt}}
\pgflineto{\pgfpoint{253.171829pt}{364.272247pt}}
\pgfusepath{stroke}
\pgfpathmoveto{\pgfpoint{253.319046pt}{370.044556pt}}
\pgflineto{\pgfpoint{253.171829pt}{370.260010pt}}
\pgfusepath{stroke}
\pgfpathmoveto{\pgfpoint{259.229675pt}{77.062592pt}}
\pgflineto{\pgfpoint{259.159576pt}{76.859985pt}}
\pgfusepath{stroke}
\pgfpathmoveto{\pgfpoint{259.234161pt}{83.056351pt}}
\pgflineto{\pgfpoint{259.159576pt}{82.847733pt}}
\pgfusepath{stroke}
\pgfpathmoveto{\pgfpoint{259.239105pt}{89.050453pt}}
\pgflineto{\pgfpoint{259.159576pt}{88.835495pt}}
\pgfusepath{stroke}
\pgfpathmoveto{\pgfpoint{259.244568pt}{95.044945pt}}
\pgflineto{\pgfpoint{259.159576pt}{94.823257pt}}
\pgfusepath{stroke}
\pgfpathmoveto{\pgfpoint{259.250610pt}{101.039795pt}}
\pgflineto{\pgfpoint{259.159576pt}{100.811012pt}}
\pgfusepath{stroke}
\pgfpathmoveto{\pgfpoint{259.257355pt}{107.035065pt}}
\pgflineto{\pgfpoint{259.159576pt}{106.798759pt}}
\pgfusepath{stroke}
\pgfpathmoveto{\pgfpoint{259.264893pt}{113.030762pt}}
\pgflineto{\pgfpoint{259.159576pt}{112.786522pt}}
\pgfusepath{stroke}
\pgfpathmoveto{\pgfpoint{259.273376pt}{119.026939pt}}
\pgflineto{\pgfpoint{259.159576pt}{118.774277pt}}
\pgfusepath{stroke}
\pgfpathmoveto{\pgfpoint{259.282898pt}{125.023582pt}}
\pgflineto{\pgfpoint{259.159576pt}{124.762024pt}}
\pgfusepath{stroke}
\pgfpathmoveto{\pgfpoint{259.293762pt}{131.020706pt}}
\pgflineto{\pgfpoint{259.159576pt}{130.749786pt}}
\pgfusepath{stroke}
\pgfpathmoveto{\pgfpoint{259.306122pt}{137.018311pt}}
\pgflineto{\pgfpoint{259.159576pt}{136.737534pt}}
\pgfusepath{stroke}
\pgfpathmoveto{\pgfpoint{259.320282pt}{143.016342pt}}
\pgflineto{\pgfpoint{259.159576pt}{142.725281pt}}
\pgfusepath{stroke}
\pgfpathmoveto{\pgfpoint{259.336639pt}{149.014740pt}}
\pgflineto{\pgfpoint{259.159576pt}{148.713058pt}}
\pgfusepath{stroke}
\pgfpathmoveto{\pgfpoint{259.355591pt}{155.013321pt}}
\pgflineto{\pgfpoint{259.159576pt}{154.700806pt}}
\pgfusepath{stroke}
\pgfpathmoveto{\pgfpoint{259.377747pt}{161.011856pt}}
\pgflineto{\pgfpoint{259.159576pt}{160.688568pt}}
\pgfusepath{stroke}
\pgfpathmoveto{\pgfpoint{259.403656pt}{167.009872pt}}
\pgflineto{\pgfpoint{259.159576pt}{166.676315pt}}
\pgfusepath{stroke}
\pgfpathmoveto{\pgfpoint{259.434143pt}{173.006683pt}}
\pgflineto{\pgfpoint{259.159576pt}{172.664078pt}}
\pgfusepath{stroke}
\pgfpathmoveto{\pgfpoint{259.469910pt}{179.001160pt}}
\pgflineto{\pgfpoint{259.159576pt}{178.651825pt}}
\pgfusepath{stroke}
\pgfpathmoveto{\pgfpoint{259.511658pt}{184.991653pt}}
\pgflineto{\pgfpoint{259.159576pt}{184.639587pt}}
\pgfusepath{stroke}
\pgfpathmoveto{\pgfpoint{259.559662pt}{190.975739pt}}
\pgflineto{\pgfpoint{259.159576pt}{190.627335pt}}
\pgfusepath{stroke}
\pgfpathmoveto{\pgfpoint{259.613312pt}{196.950195pt}}
\pgflineto{\pgfpoint{259.159576pt}{196.615097pt}}
\pgfusepath{stroke}
\pgfpathmoveto{\pgfpoint{259.670319pt}{202.911316pt}}
\pgflineto{\pgfpoint{259.159576pt}{202.602844pt}}
\pgfusepath{stroke}
\pgfpathmoveto{\pgfpoint{259.726013pt}{208.855789pt}}
\pgflineto{\pgfpoint{259.159576pt}{208.590607pt}}
\pgfusepath{stroke}
\pgfpathmoveto{\pgfpoint{259.773102pt}{214.782639pt}}
\pgflineto{\pgfpoint{259.159576pt}{214.578354pt}}
\pgfusepath{stroke}
\pgfpathmoveto{\pgfpoint{259.803467pt}{220.695267pt}}
\pgflineto{\pgfpoint{259.159576pt}{220.566116pt}}
\pgfusepath{stroke}
\pgfpathmoveto{\pgfpoint{259.811462pt}{226.601776pt}}
\pgflineto{\pgfpoint{259.159576pt}{226.553864pt}}
\pgfusepath{stroke}
\pgfpathmoveto{\pgfpoint{259.796936pt}{232.512497pt}}
\pgflineto{\pgfpoint{259.159576pt}{232.541626pt}}
\pgfusepath{stroke}
\pgfpathmoveto{\pgfpoint{259.765320pt}{238.435608pt}}
\pgflineto{\pgfpoint{259.159576pt}{238.529388pt}}
\pgfusepath{stroke}
\pgfpathmoveto{\pgfpoint{259.724487pt}{244.374512pt}}
\pgflineto{\pgfpoint{259.159576pt}{244.517136pt}}
\pgfusepath{stroke}
\pgfpathmoveto{\pgfpoint{259.681427pt}{250.328339pt}}
\pgflineto{\pgfpoint{259.159576pt}{250.504883pt}}
\pgfusepath{stroke}
\pgfpathmoveto{\pgfpoint{259.640625pt}{256.293945pt}}
\pgflineto{\pgfpoint{259.159576pt}{256.492645pt}}
\pgfusepath{stroke}
\pgfpathmoveto{\pgfpoint{259.604126pt}{262.267761pt}}
\pgflineto{\pgfpoint{259.159576pt}{262.480408pt}}
\pgfusepath{stroke}
\pgfpathmoveto{\pgfpoint{259.572327pt}{268.246796pt}}
\pgflineto{\pgfpoint{259.159576pt}{268.468170pt}}
\pgfusepath{stroke}
\pgfpathmoveto{\pgfpoint{259.544800pt}{274.228912pt}}
\pgflineto{\pgfpoint{259.159576pt}{274.455902pt}}
\pgfusepath{stroke}
\pgfpathmoveto{\pgfpoint{259.520752pt}{280.212769pt}}
\pgflineto{\pgfpoint{259.159576pt}{280.443665pt}}
\pgfusepath{stroke}
\pgfpathmoveto{\pgfpoint{259.499420pt}{286.197540pt}}
\pgflineto{\pgfpoint{259.159576pt}{286.431427pt}}
\pgfusepath{stroke}
\pgfpathmoveto{\pgfpoint{259.480042pt}{292.182831pt}}
\pgflineto{\pgfpoint{259.159576pt}{292.419189pt}}
\pgfusepath{stroke}
\pgfpathmoveto{\pgfpoint{259.462097pt}{298.168518pt}}
\pgflineto{\pgfpoint{259.159576pt}{298.406921pt}}
\pgfusepath{stroke}
\pgfpathmoveto{\pgfpoint{259.445221pt}{304.154602pt}}
\pgflineto{\pgfpoint{259.159576pt}{304.394714pt}}
\pgfusepath{stroke}
\pgfpathmoveto{\pgfpoint{259.429138pt}{310.141113pt}}
\pgflineto{\pgfpoint{259.159576pt}{310.382446pt}}
\pgfusepath{stroke}
\pgfpathmoveto{\pgfpoint{259.413727pt}{316.128113pt}}
\pgflineto{\pgfpoint{259.159576pt}{316.370178pt}}
\pgfusepath{stroke}
\pgfpathmoveto{\pgfpoint{259.398926pt}{322.115723pt}}
\pgflineto{\pgfpoint{259.159576pt}{322.357971pt}}
\pgfusepath{stroke}
\pgfpathmoveto{\pgfpoint{259.384735pt}{328.103882pt}}
\pgflineto{\pgfpoint{259.159576pt}{328.345703pt}}
\pgfusepath{stroke}
\pgfpathmoveto{\pgfpoint{259.371124pt}{334.092651pt}}
\pgflineto{\pgfpoint{259.159576pt}{334.333466pt}}
\pgfusepath{stroke}
\pgfpathmoveto{\pgfpoint{259.358154pt}{340.081970pt}}
\pgflineto{\pgfpoint{259.159576pt}{340.321228pt}}
\pgfusepath{stroke}
\pgfpathmoveto{\pgfpoint{259.345856pt}{346.071838pt}}
\pgflineto{\pgfpoint{259.159576pt}{346.308960pt}}
\pgfusepath{stroke}
\pgfpathmoveto{\pgfpoint{259.334229pt}{352.062195pt}}
\pgflineto{\pgfpoint{259.159576pt}{352.296722pt}}
\pgfusepath{stroke}
\pgfpathmoveto{\pgfpoint{259.323273pt}{358.052917pt}}
\pgflineto{\pgfpoint{259.159576pt}{358.284485pt}}
\pgfusepath{stroke}
\pgfpathmoveto{\pgfpoint{259.312988pt}{364.044037pt}}
\pgflineto{\pgfpoint{259.159576pt}{364.272247pt}}
\pgfusepath{stroke}
\pgfpathmoveto{\pgfpoint{259.303375pt}{370.035400pt}}
\pgflineto{\pgfpoint{259.159576pt}{370.260010pt}}
\pgfusepath{stroke}
\pgfpathmoveto{\pgfpoint{265.211365pt}{77.066696pt}}
\pgflineto{\pgfpoint{265.147339pt}{76.859985pt}}
\pgfusepath{stroke}
\pgfpathmoveto{\pgfpoint{265.215515pt}{83.060883pt}}
\pgflineto{\pgfpoint{265.147339pt}{82.847733pt}}
\pgfusepath{stroke}
\pgfpathmoveto{\pgfpoint{265.220062pt}{89.055458pt}}
\pgflineto{\pgfpoint{265.147339pt}{88.835495pt}}
\pgfusepath{stroke}
\pgfpathmoveto{\pgfpoint{265.225128pt}{95.050476pt}}
\pgflineto{\pgfpoint{265.147339pt}{94.823257pt}}
\pgfusepath{stroke}
\pgfpathmoveto{\pgfpoint{265.230743pt}{101.045952pt}}
\pgflineto{\pgfpoint{265.147339pt}{100.811012pt}}
\pgfusepath{stroke}
\pgfpathmoveto{\pgfpoint{265.237000pt}{107.041946pt}}
\pgflineto{\pgfpoint{265.147339pt}{106.798759pt}}
\pgfusepath{stroke}
\pgfpathmoveto{\pgfpoint{265.244019pt}{113.038490pt}}
\pgflineto{\pgfpoint{265.147339pt}{112.786522pt}}
\pgfusepath{stroke}
\pgfpathmoveto{\pgfpoint{265.251923pt}{119.035652pt}}
\pgflineto{\pgfpoint{265.147339pt}{118.774277pt}}
\pgfusepath{stroke}
\pgfpathmoveto{\pgfpoint{265.260895pt}{125.033455pt}}
\pgflineto{\pgfpoint{265.147339pt}{124.762024pt}}
\pgfusepath{stroke}
\pgfpathmoveto{\pgfpoint{265.271118pt}{131.031982pt}}
\pgflineto{\pgfpoint{265.147339pt}{130.749786pt}}
\pgfusepath{stroke}
\pgfpathmoveto{\pgfpoint{265.282867pt}{137.031250pt}}
\pgflineto{\pgfpoint{265.147339pt}{136.737534pt}}
\pgfusepath{stroke}
\pgfpathmoveto{\pgfpoint{265.296448pt}{143.031296pt}}
\pgflineto{\pgfpoint{265.147339pt}{142.725281pt}}
\pgfusepath{stroke}
\pgfpathmoveto{\pgfpoint{265.312286pt}{149.032135pt}}
\pgflineto{\pgfpoint{265.147339pt}{148.713058pt}}
\pgfusepath{stroke}
\pgfpathmoveto{\pgfpoint{265.330872pt}{155.033691pt}}
\pgflineto{\pgfpoint{265.147339pt}{154.700806pt}}
\pgfusepath{stroke}
\pgfpathmoveto{\pgfpoint{265.352905pt}{161.035889pt}}
\pgflineto{\pgfpoint{265.147339pt}{160.688568pt}}
\pgfusepath{stroke}
\pgfpathmoveto{\pgfpoint{265.379211pt}{167.038391pt}}
\pgflineto{\pgfpoint{265.147339pt}{166.676315pt}}
\pgfusepath{stroke}
\pgfpathmoveto{\pgfpoint{265.410889pt}{173.040649pt}}
\pgflineto{\pgfpoint{265.147339pt}{172.664078pt}}
\pgfusepath{stroke}
\pgfpathmoveto{\pgfpoint{265.449249pt}{179.041656pt}}
\pgflineto{\pgfpoint{265.147339pt}{178.651825pt}}
\pgfusepath{stroke}
\pgfpathmoveto{\pgfpoint{265.495728pt}{185.039658pt}}
\pgflineto{\pgfpoint{265.147339pt}{184.639587pt}}
\pgfusepath{stroke}
\pgfpathmoveto{\pgfpoint{265.551727pt}{191.031723pt}}
\pgflineto{\pgfpoint{265.147339pt}{190.627335pt}}
\pgfusepath{stroke}
\pgfpathmoveto{\pgfpoint{265.617859pt}{197.013351pt}}
\pgflineto{\pgfpoint{265.147339pt}{196.615097pt}}
\pgfusepath{stroke}
\pgfpathmoveto{\pgfpoint{265.692596pt}{202.978195pt}}
\pgflineto{\pgfpoint{265.147339pt}{202.602844pt}}
\pgfusepath{stroke}
\pgfpathmoveto{\pgfpoint{265.770294pt}{208.919144pt}}
\pgflineto{\pgfpoint{265.147339pt}{208.590607pt}}
\pgfusepath{stroke}
\pgfpathmoveto{\pgfpoint{265.839447pt}{214.831573pt}}
\pgflineto{\pgfpoint{265.147339pt}{214.578354pt}}
\pgfusepath{stroke}
\pgfpathmoveto{\pgfpoint{265.884552pt}{220.718826pt}}
\pgflineto{\pgfpoint{265.147339pt}{220.566116pt}}
\pgfusepath{stroke}
\pgfpathmoveto{\pgfpoint{265.893555pt}{226.595154pt}}
\pgflineto{\pgfpoint{265.147339pt}{226.553864pt}}
\pgfusepath{stroke}
\pgfpathmoveto{\pgfpoint{265.866302pt}{232.480240pt}}
\pgflineto{\pgfpoint{265.147339pt}{232.541626pt}}
\pgfusepath{stroke}
\pgfpathmoveto{\pgfpoint{265.814575pt}{238.388382pt}}
\pgflineto{\pgfpoint{265.147339pt}{238.529388pt}}
\pgfusepath{stroke}
\pgfpathmoveto{\pgfpoint{265.753632pt}{244.322968pt}}
\pgflineto{\pgfpoint{265.147339pt}{244.517136pt}}
\pgfusepath{stroke}
\pgfpathmoveto{\pgfpoint{265.694641pt}{250.279465pt}}
\pgflineto{\pgfpoint{265.147339pt}{250.504883pt}}
\pgfusepath{stroke}
\pgfpathmoveto{\pgfpoint{265.643005pt}{256.250885pt}}
\pgflineto{\pgfpoint{265.147339pt}{256.492645pt}}
\pgfusepath{stroke}
\pgfpathmoveto{\pgfpoint{265.599915pt}{262.231140pt}}
\pgflineto{\pgfpoint{265.147339pt}{262.480408pt}}
\pgfusepath{stroke}
\pgfpathmoveto{\pgfpoint{265.564514pt}{268.215881pt}}
\pgflineto{\pgfpoint{265.147339pt}{268.468170pt}}
\pgfusepath{stroke}
\pgfpathmoveto{\pgfpoint{265.535248pt}{274.202606pt}}
\pgflineto{\pgfpoint{265.147339pt}{274.455902pt}}
\pgfusepath{stroke}
\pgfpathmoveto{\pgfpoint{265.510468pt}{280.189880pt}}
\pgflineto{\pgfpoint{265.147339pt}{280.443665pt}}
\pgfusepath{stroke}
\pgfpathmoveto{\pgfpoint{265.488831pt}{286.177124pt}}
\pgflineto{\pgfpoint{265.147339pt}{286.431427pt}}
\pgfusepath{stroke}
\pgfpathmoveto{\pgfpoint{265.469330pt}{292.164154pt}}
\pgflineto{\pgfpoint{265.147339pt}{292.419189pt}}
\pgfusepath{stroke}
\pgfpathmoveto{\pgfpoint{265.451233pt}{298.151031pt}}
\pgflineto{\pgfpoint{265.147339pt}{298.406921pt}}
\pgfusepath{stroke}
\pgfpathmoveto{\pgfpoint{265.434082pt}{304.137939pt}}
\pgflineto{\pgfpoint{265.147339pt}{304.394714pt}}
\pgfusepath{stroke}
\pgfpathmoveto{\pgfpoint{265.417572pt}{310.125153pt}}
\pgflineto{\pgfpoint{265.147339pt}{310.382446pt}}
\pgfusepath{stroke}
\pgfpathmoveto{\pgfpoint{265.401672pt}{316.112732pt}}
\pgflineto{\pgfpoint{265.147339pt}{316.370178pt}}
\pgfusepath{stroke}
\pgfpathmoveto{\pgfpoint{265.386292pt}{322.100891pt}}
\pgflineto{\pgfpoint{265.147339pt}{322.357971pt}}
\pgfusepath{stroke}
\pgfpathmoveto{\pgfpoint{265.371460pt}{328.089661pt}}
\pgflineto{\pgfpoint{265.147339pt}{328.345703pt}}
\pgfusepath{stroke}
\pgfpathmoveto{\pgfpoint{265.357239pt}{334.079071pt}}
\pgflineto{\pgfpoint{265.147339pt}{334.333466pt}}
\pgfusepath{stroke}
\pgfpathmoveto{\pgfpoint{265.343719pt}{340.069092pt}}
\pgflineto{\pgfpoint{265.147339pt}{340.321228pt}}
\pgfusepath{stroke}
\pgfpathmoveto{\pgfpoint{265.330902pt}{346.059631pt}}
\pgflineto{\pgfpoint{265.147339pt}{346.308960pt}}
\pgfusepath{stroke}
\pgfpathmoveto{\pgfpoint{265.318817pt}{352.050690pt}}
\pgflineto{\pgfpoint{265.147339pt}{352.296722pt}}
\pgfusepath{stroke}
\pgfpathmoveto{\pgfpoint{265.307495pt}{358.042175pt}}
\pgflineto{\pgfpoint{265.147339pt}{358.284485pt}}
\pgfusepath{stroke}
\pgfpathmoveto{\pgfpoint{265.296936pt}{364.033966pt}}
\pgflineto{\pgfpoint{265.147339pt}{364.272247pt}}
\pgfusepath{stroke}
\pgfpathmoveto{\pgfpoint{265.287079pt}{370.026001pt}}
\pgflineto{\pgfpoint{265.147339pt}{370.260010pt}}
\pgfusepath{stroke}
\pgfpathmoveto{\pgfpoint{271.192688pt}{77.070435pt}}
\pgflineto{\pgfpoint{271.135101pt}{76.859985pt}}
\pgfusepath{stroke}
\pgfpathmoveto{\pgfpoint{271.196442pt}{83.065002pt}}
\pgflineto{\pgfpoint{271.135101pt}{82.847733pt}}
\pgfusepath{stroke}
\pgfpathmoveto{\pgfpoint{271.200562pt}{89.060028pt}}
\pgflineto{\pgfpoint{271.135101pt}{88.835495pt}}
\pgfusepath{stroke}
\pgfpathmoveto{\pgfpoint{271.205139pt}{95.055542pt}}
\pgflineto{\pgfpoint{271.135101pt}{94.823257pt}}
\pgfusepath{stroke}
\pgfpathmoveto{\pgfpoint{271.210205pt}{101.051605pt}}
\pgflineto{\pgfpoint{271.135101pt}{100.811012pt}}
\pgfusepath{stroke}
\pgfpathmoveto{\pgfpoint{271.215881pt}{107.048271pt}}
\pgflineto{\pgfpoint{271.135101pt}{106.798759pt}}
\pgfusepath{stroke}
\pgfpathmoveto{\pgfpoint{271.222260pt}{113.045609pt}}
\pgflineto{\pgfpoint{271.135101pt}{112.786522pt}}
\pgfusepath{stroke}
\pgfpathmoveto{\pgfpoint{271.229462pt}{119.043709pt}}
\pgflineto{\pgfpoint{271.135101pt}{118.774277pt}}
\pgfusepath{stroke}
\pgfpathmoveto{\pgfpoint{271.237640pt}{125.042633pt}}
\pgflineto{\pgfpoint{271.135101pt}{124.762024pt}}
\pgfusepath{stroke}
\pgfpathmoveto{\pgfpoint{271.247009pt}{131.042496pt}}
\pgflineto{\pgfpoint{271.135101pt}{130.749786pt}}
\pgfusepath{stroke}
\pgfpathmoveto{\pgfpoint{271.257843pt}{137.043396pt}}
\pgflineto{\pgfpoint{271.135101pt}{136.737534pt}}
\pgfusepath{stroke}
\pgfpathmoveto{\pgfpoint{271.270416pt}{143.045441pt}}
\pgflineto{\pgfpoint{271.135101pt}{142.725281pt}}
\pgfusepath{stroke}
\pgfpathmoveto{\pgfpoint{271.285187pt}{149.048767pt}}
\pgflineto{\pgfpoint{271.135101pt}{148.713058pt}}
\pgfusepath{stroke}
\pgfpathmoveto{\pgfpoint{271.302734pt}{155.053436pt}}
\pgflineto{\pgfpoint{271.135101pt}{154.700806pt}}
\pgfusepath{stroke}
\pgfpathmoveto{\pgfpoint{271.323792pt}{161.059555pt}}
\pgflineto{\pgfpoint{271.135101pt}{160.688568pt}}
\pgfusepath{stroke}
\pgfpathmoveto{\pgfpoint{271.349426pt}{167.067078pt}}
\pgflineto{\pgfpoint{271.135101pt}{166.676315pt}}
\pgfusepath{stroke}
\pgfpathmoveto{\pgfpoint{271.381012pt}{173.075760pt}}
\pgflineto{\pgfpoint{271.135101pt}{172.664078pt}}
\pgfusepath{stroke}
\pgfpathmoveto{\pgfpoint{271.420441pt}{179.084976pt}}
\pgflineto{\pgfpoint{271.135101pt}{178.651825pt}}
\pgfusepath{stroke}
\pgfpathmoveto{\pgfpoint{271.470184pt}{185.093307pt}}
\pgflineto{\pgfpoint{271.135101pt}{184.639587pt}}
\pgfusepath{stroke}
\pgfpathmoveto{\pgfpoint{271.533356pt}{191.097855pt}}
\pgflineto{\pgfpoint{271.135101pt}{190.627335pt}}
\pgfusepath{stroke}
\pgfpathmoveto{\pgfpoint{271.613068pt}{197.093063pt}}
\pgflineto{\pgfpoint{271.135101pt}{196.615097pt}}
\pgfusepath{stroke}
\pgfpathmoveto{\pgfpoint{271.710815pt}{203.069336pt}}
\pgflineto{\pgfpoint{271.135101pt}{202.602844pt}}
\pgfusepath{stroke}
\pgfpathmoveto{\pgfpoint{271.822235pt}{209.012405pt}}
\pgflineto{\pgfpoint{271.135101pt}{208.590607pt}}
\pgfusepath{stroke}
\pgfpathmoveto{\pgfpoint{271.930359pt}{214.907806pt}}
\pgflineto{\pgfpoint{271.135101pt}{214.578354pt}}
\pgfusepath{stroke}
\pgfpathmoveto{\pgfpoint{272.003998pt}{220.754440pt}}
\pgflineto{\pgfpoint{271.135101pt}{220.566116pt}}
\pgfusepath{stroke}
\pgfpathmoveto{\pgfpoint{272.014160pt}{226.578644pt}}
\pgflineto{\pgfpoint{271.135101pt}{226.553864pt}}
\pgfusepath{stroke}
\pgfpathmoveto{\pgfpoint{271.960663pt}{232.422836pt}}
\pgflineto{\pgfpoint{271.135101pt}{232.541626pt}}
\pgfusepath{stroke}
\pgfpathmoveto{\pgfpoint{271.872223pt}{238.313354pt}}
\pgflineto{\pgfpoint{271.135101pt}{238.529388pt}}
\pgfusepath{stroke}
\pgfpathmoveto{\pgfpoint{271.779755pt}{244.249176pt}}
\pgflineto{\pgfpoint{271.135101pt}{244.517136pt}}
\pgfusepath{stroke}
\pgfpathmoveto{\pgfpoint{271.699829pt}{250.215897pt}}
\pgflineto{\pgfpoint{271.135101pt}{250.504883pt}}
\pgfusepath{stroke}
\pgfpathmoveto{\pgfpoint{271.636505pt}{256.199371pt}}
\pgflineto{\pgfpoint{271.135101pt}{256.492645pt}}
\pgfusepath{stroke}
\pgfpathmoveto{\pgfpoint{271.587921pt}{262.190247pt}}
\pgflineto{\pgfpoint{271.135101pt}{262.480408pt}}
\pgfusepath{stroke}
\pgfpathmoveto{\pgfpoint{271.550598pt}{268.183228pt}}
\pgflineto{\pgfpoint{271.135101pt}{268.468170pt}}
\pgfusepath{stroke}
\pgfpathmoveto{\pgfpoint{271.521210pt}{274.175873pt}}
\pgflineto{\pgfpoint{271.135101pt}{274.455902pt}}
\pgfusepath{stroke}
\pgfpathmoveto{\pgfpoint{271.497070pt}{280.167175pt}}
\pgflineto{\pgfpoint{271.135101pt}{280.443665pt}}
\pgfusepath{stroke}
\pgfpathmoveto{\pgfpoint{271.476196pt}{286.156982pt}}
\pgflineto{\pgfpoint{271.135101pt}{286.431427pt}}
\pgfusepath{stroke}
\pgfpathmoveto{\pgfpoint{271.457275pt}{292.145630pt}}
\pgflineto{\pgfpoint{271.135101pt}{292.419189pt}}
\pgfusepath{stroke}
\pgfpathmoveto{\pgfpoint{271.439453pt}{298.133484pt}}
\pgflineto{\pgfpoint{271.135101pt}{298.406921pt}}
\pgfusepath{stroke}
\pgfpathmoveto{\pgfpoint{271.422272pt}{304.121063pt}}
\pgflineto{\pgfpoint{271.135101pt}{304.394714pt}}
\pgfusepath{stroke}
\pgfpathmoveto{\pgfpoint{271.405487pt}{310.108734pt}}
\pgflineto{\pgfpoint{271.135101pt}{310.382446pt}}
\pgfusepath{stroke}
\pgfpathmoveto{\pgfpoint{271.389069pt}{316.096802pt}}
\pgflineto{\pgfpoint{271.135101pt}{316.370178pt}}
\pgfusepath{stroke}
\pgfpathmoveto{\pgfpoint{271.373047pt}{322.085480pt}}
\pgflineto{\pgfpoint{271.135101pt}{322.357971pt}}
\pgfusepath{stroke}
\pgfpathmoveto{\pgfpoint{271.357574pt}{328.074829pt}}
\pgflineto{\pgfpoint{271.135101pt}{328.345703pt}}
\pgfusepath{stroke}
\pgfpathmoveto{\pgfpoint{271.342712pt}{334.064880pt}}
\pgflineto{\pgfpoint{271.135101pt}{334.333466pt}}
\pgfusepath{stroke}
\pgfpathmoveto{\pgfpoint{271.328552pt}{340.055603pt}}
\pgflineto{\pgfpoint{271.135101pt}{340.321228pt}}
\pgfusepath{stroke}
\pgfpathmoveto{\pgfpoint{271.315216pt}{346.046936pt}}
\pgflineto{\pgfpoint{271.135101pt}{346.308960pt}}
\pgfusepath{stroke}
\pgfpathmoveto{\pgfpoint{271.302673pt}{352.038788pt}}
\pgflineto{\pgfpoint{271.135101pt}{352.296722pt}}
\pgfusepath{stroke}
\pgfpathmoveto{\pgfpoint{271.290985pt}{358.031036pt}}
\pgflineto{\pgfpoint{271.135101pt}{358.284485pt}}
\pgfusepath{stroke}
\pgfpathmoveto{\pgfpoint{271.280121pt}{364.023621pt}}
\pgflineto{\pgfpoint{271.135101pt}{364.272247pt}}
\pgfusepath{stroke}
\pgfpathmoveto{\pgfpoint{271.270081pt}{370.016418pt}}
\pgflineto{\pgfpoint{271.135101pt}{370.260010pt}}
\pgfusepath{stroke}
\pgfpathmoveto{\pgfpoint{277.173706pt}{77.073807pt}}
\pgflineto{\pgfpoint{277.122833pt}{76.859985pt}}
\pgfusepath{stroke}
\pgfpathmoveto{\pgfpoint{277.176971pt}{83.068710pt}}
\pgflineto{\pgfpoint{277.122833pt}{82.847733pt}}
\pgfusepath{stroke}
\pgfpathmoveto{\pgfpoint{277.180603pt}{89.064110pt}}
\pgflineto{\pgfpoint{277.122833pt}{88.835495pt}}
\pgfusepath{stroke}
\pgfpathmoveto{\pgfpoint{277.184601pt}{95.060059pt}}
\pgflineto{\pgfpoint{277.122833pt}{94.823257pt}}
\pgfusepath{stroke}
\pgfpathmoveto{\pgfpoint{277.189087pt}{101.056633pt}}
\pgflineto{\pgfpoint{277.122833pt}{100.811012pt}}
\pgfusepath{stroke}
\pgfpathmoveto{\pgfpoint{277.194061pt}{107.053925pt}}
\pgflineto{\pgfpoint{277.122833pt}{106.798759pt}}
\pgfusepath{stroke}
\pgfpathmoveto{\pgfpoint{277.199646pt}{113.051979pt}}
\pgflineto{\pgfpoint{277.122833pt}{112.786522pt}}
\pgfusepath{stroke}
\pgfpathmoveto{\pgfpoint{277.205994pt}{119.050919pt}}
\pgflineto{\pgfpoint{277.122833pt}{118.774277pt}}
\pgfusepath{stroke}
\pgfpathmoveto{\pgfpoint{277.213196pt}{125.050873pt}}
\pgflineto{\pgfpoint{277.122833pt}{124.762024pt}}
\pgfusepath{stroke}
\pgfpathmoveto{\pgfpoint{277.221466pt}{131.051971pt}}
\pgflineto{\pgfpoint{277.122833pt}{130.749786pt}}
\pgfusepath{stroke}
\pgfpathmoveto{\pgfpoint{277.231018pt}{137.054382pt}}
\pgflineto{\pgfpoint{277.122833pt}{136.737534pt}}
\pgfusepath{stroke}
\pgfpathmoveto{\pgfpoint{277.242188pt}{143.058319pt}}
\pgflineto{\pgfpoint{277.122833pt}{142.725281pt}}
\pgfusepath{stroke}
\pgfpathmoveto{\pgfpoint{277.255371pt}{149.064026pt}}
\pgflineto{\pgfpoint{277.122833pt}{148.713058pt}}
\pgfusepath{stroke}
\pgfpathmoveto{\pgfpoint{277.271118pt}{155.071747pt}}
\pgflineto{\pgfpoint{277.122833pt}{154.700806pt}}
\pgfusepath{stroke}
\pgfpathmoveto{\pgfpoint{277.290222pt}{161.081787pt}}
\pgflineto{\pgfpoint{277.122833pt}{160.688568pt}}
\pgfusepath{stroke}
\pgfpathmoveto{\pgfpoint{277.313782pt}{167.094528pt}}
\pgflineto{\pgfpoint{277.122833pt}{166.676315pt}}
\pgfusepath{stroke}
\pgfpathmoveto{\pgfpoint{277.343414pt}{173.110214pt}}
\pgflineto{\pgfpoint{277.122833pt}{172.664078pt}}
\pgfusepath{stroke}
\pgfpathmoveto{\pgfpoint{277.381409pt}{179.128967pt}}
\pgflineto{\pgfpoint{277.122833pt}{178.651825pt}}
\pgfusepath{stroke}
\pgfpathmoveto{\pgfpoint{277.431305pt}{185.150330pt}}
\pgflineto{\pgfpoint{277.122833pt}{184.639587pt}}
\pgfusepath{stroke}
\pgfpathmoveto{\pgfpoint{277.498199pt}{191.172592pt}}
\pgflineto{\pgfpoint{277.122833pt}{190.627335pt}}
\pgfusepath{stroke}
\pgfpathmoveto{\pgfpoint{277.589325pt}{197.190826pt}}
\pgflineto{\pgfpoint{277.122833pt}{196.615097pt}}
\pgfusepath{stroke}
\pgfpathmoveto{\pgfpoint{277.713287pt}{203.193298pt}}
\pgflineto{\pgfpoint{277.122833pt}{202.602844pt}}
\pgfusepath{stroke}
\pgfpathmoveto{\pgfpoint{277.874481pt}{209.155411pt}}
\pgflineto{\pgfpoint{277.122833pt}{208.590607pt}}
\pgfusepath{stroke}
\pgfpathmoveto{\pgfpoint{278.055542pt}{215.038239pt}}
\pgflineto{\pgfpoint{277.122833pt}{214.578354pt}}
\pgfusepath{stroke}
\pgfpathmoveto{\pgfpoint{278.192810pt}{220.816101pt}}
\pgflineto{\pgfpoint{277.122833pt}{220.566116pt}}
\pgfusepath{stroke}
\pgfpathmoveto{\pgfpoint{278.204254pt}{226.538528pt}}
\pgflineto{\pgfpoint{277.122833pt}{226.553864pt}}
\pgfusepath{stroke}
\pgfpathmoveto{\pgfpoint{278.089722pt}{232.313629pt}}
\pgflineto{\pgfpoint{277.122833pt}{232.541626pt}}
\pgfusepath{stroke}
\pgfpathmoveto{\pgfpoint{277.930939pt}{238.190979pt}}
\pgflineto{\pgfpoint{277.122833pt}{238.529388pt}}
\pgfusepath{stroke}
\pgfpathmoveto{\pgfpoint{277.791229pt}{244.144882pt}}
\pgflineto{\pgfpoint{277.122833pt}{244.517136pt}}
\pgfusepath{stroke}
\pgfpathmoveto{\pgfpoint{277.687531pt}{250.136444pt}}
\pgflineto{\pgfpoint{277.122833pt}{250.504883pt}}
\pgfusepath{stroke}
\pgfpathmoveto{\pgfpoint{277.615051pt}{256.141205pt}}
\pgflineto{\pgfpoint{277.122833pt}{256.492645pt}}
\pgfusepath{stroke}
\pgfpathmoveto{\pgfpoint{277.564728pt}{262.147644pt}}
\pgflineto{\pgfpoint{277.122833pt}{262.480408pt}}
\pgfusepath{stroke}
\pgfpathmoveto{\pgfpoint{277.528900pt}{268.151215pt}}
\pgflineto{\pgfpoint{277.122833pt}{268.468170pt}}
\pgfusepath{stroke}
\pgfpathmoveto{\pgfpoint{277.502075pt}{274.150604pt}}
\pgflineto{\pgfpoint{277.122833pt}{274.455902pt}}
\pgfusepath{stroke}
\pgfpathmoveto{\pgfpoint{277.480469pt}{280.145935pt}}
\pgflineto{\pgfpoint{277.122833pt}{280.443665pt}}
\pgfusepath{stroke}
\pgfpathmoveto{\pgfpoint{277.461670pt}{286.138062pt}}
\pgflineto{\pgfpoint{277.122833pt}{286.431427pt}}
\pgfusepath{stroke}
\pgfpathmoveto{\pgfpoint{277.444153pt}{292.127838pt}}
\pgflineto{\pgfpoint{277.122833pt}{292.419189pt}}
\pgfusepath{stroke}
\pgfpathmoveto{\pgfpoint{277.427094pt}{298.116241pt}}
\pgflineto{\pgfpoint{277.122833pt}{298.406921pt}}
\pgfusepath{stroke}
\pgfpathmoveto{\pgfpoint{277.410065pt}{304.104065pt}}
\pgflineto{\pgfpoint{277.122833pt}{304.394714pt}}
\pgfusepath{stroke}
\pgfpathmoveto{\pgfpoint{277.393036pt}{310.091919pt}}
\pgflineto{\pgfpoint{277.122833pt}{310.382446pt}}
\pgfusepath{stroke}
\pgfpathmoveto{\pgfpoint{277.376038pt}{316.080261pt}}
\pgflineto{\pgfpoint{277.122833pt}{316.370178pt}}
\pgfusepath{stroke}
\pgfpathmoveto{\pgfpoint{277.359314pt}{322.069336pt}}
\pgflineto{\pgfpoint{277.122833pt}{322.357971pt}}
\pgfusepath{stroke}
\pgfpathmoveto{\pgfpoint{277.343048pt}{328.059265pt}}
\pgflineto{\pgfpoint{277.122833pt}{328.345703pt}}
\pgfusepath{stroke}
\pgfpathmoveto{\pgfpoint{277.327454pt}{334.049988pt}}
\pgflineto{\pgfpoint{277.122833pt}{334.333466pt}}
\pgfusepath{stroke}
\pgfpathmoveto{\pgfpoint{277.312622pt}{340.041534pt}}
\pgflineto{\pgfpoint{277.122833pt}{340.321228pt}}
\pgfusepath{stroke}
\pgfpathmoveto{\pgfpoint{277.298676pt}{346.033722pt}}
\pgflineto{\pgfpoint{277.122833pt}{346.308960pt}}
\pgfusepath{stroke}
\pgfpathmoveto{\pgfpoint{277.285675pt}{352.026489pt}}
\pgflineto{\pgfpoint{277.122833pt}{352.296722pt}}
\pgfusepath{stroke}
\pgfpathmoveto{\pgfpoint{277.273621pt}{358.019653pt}}
\pgflineto{\pgfpoint{277.122833pt}{358.284485pt}}
\pgfusepath{stroke}
\pgfpathmoveto{\pgfpoint{277.262512pt}{364.013062pt}}
\pgflineto{\pgfpoint{277.122833pt}{364.272247pt}}
\pgfusepath{stroke}
\pgfpathmoveto{\pgfpoint{277.252289pt}{370.006714pt}}
\pgflineto{\pgfpoint{277.122833pt}{370.260010pt}}
\pgfusepath{stroke}
\pgfpathmoveto{\pgfpoint{283.154388pt}{77.076691pt}}
\pgflineto{\pgfpoint{283.110596pt}{76.859985pt}}
\pgfusepath{stroke}
\pgfpathmoveto{\pgfpoint{283.157196pt}{83.071884pt}}
\pgflineto{\pgfpoint{283.110596pt}{82.847733pt}}
\pgfusepath{stroke}
\pgfpathmoveto{\pgfpoint{283.160248pt}{89.067627pt}}
\pgflineto{\pgfpoint{283.110596pt}{88.835495pt}}
\pgfusepath{stroke}
\pgfpathmoveto{\pgfpoint{283.163635pt}{95.063957pt}}
\pgflineto{\pgfpoint{283.110596pt}{94.823257pt}}
\pgfusepath{stroke}
\pgfpathmoveto{\pgfpoint{283.167419pt}{101.060989pt}}
\pgflineto{\pgfpoint{283.110596pt}{100.811012pt}}
\pgfusepath{stroke}
\pgfpathmoveto{\pgfpoint{283.171631pt}{107.058792pt}}
\pgflineto{\pgfpoint{283.110596pt}{106.798759pt}}
\pgfusepath{stroke}
\pgfpathmoveto{\pgfpoint{283.176331pt}{113.057465pt}}
\pgflineto{\pgfpoint{283.110596pt}{112.786522pt}}
\pgfusepath{stroke}
\pgfpathmoveto{\pgfpoint{283.181671pt}{119.057144pt}}
\pgflineto{\pgfpoint{283.110596pt}{118.774277pt}}
\pgfusepath{stroke}
\pgfpathmoveto{\pgfpoint{283.187714pt}{125.057968pt}}
\pgflineto{\pgfpoint{283.110596pt}{124.762024pt}}
\pgfusepath{stroke}
\pgfpathmoveto{\pgfpoint{283.194641pt}{131.060150pt}}
\pgflineto{\pgfpoint{283.110596pt}{130.749786pt}}
\pgfusepath{stroke}
\pgfpathmoveto{\pgfpoint{283.202637pt}{137.063873pt}}
\pgflineto{\pgfpoint{283.110596pt}{136.737534pt}}
\pgfusepath{stroke}
\pgfpathmoveto{\pgfpoint{283.211975pt}{143.069458pt}}
\pgflineto{\pgfpoint{283.110596pt}{142.725281pt}}
\pgfusepath{stroke}
\pgfpathmoveto{\pgfpoint{283.222992pt}{149.077255pt}}
\pgflineto{\pgfpoint{283.110596pt}{148.713058pt}}
\pgfusepath{stroke}
\pgfpathmoveto{\pgfpoint{283.236176pt}{155.087692pt}}
\pgflineto{\pgfpoint{283.110596pt}{154.700806pt}}
\pgfusepath{stroke}
\pgfpathmoveto{\pgfpoint{283.252258pt}{161.101364pt}}
\pgflineto{\pgfpoint{283.110596pt}{160.688568pt}}
\pgfusepath{stroke}
\pgfpathmoveto{\pgfpoint{283.272217pt}{167.119003pt}}
\pgflineto{\pgfpoint{283.110596pt}{166.676315pt}}
\pgfusepath{stroke}
\pgfpathmoveto{\pgfpoint{283.297577pt}{173.141541pt}}
\pgflineto{\pgfpoint{283.110596pt}{172.664078pt}}
\pgfusepath{stroke}
\pgfpathmoveto{\pgfpoint{283.330811pt}{179.170105pt}}
\pgflineto{\pgfpoint{283.110596pt}{178.651825pt}}
\pgfusepath{stroke}
\pgfpathmoveto{\pgfpoint{283.375793pt}{185.206009pt}}
\pgflineto{\pgfpoint{283.110596pt}{184.639587pt}}
\pgfusepath{stroke}
\pgfpathmoveto{\pgfpoint{283.439148pt}{191.250290pt}}
\pgflineto{\pgfpoint{283.110596pt}{190.627335pt}}
\pgfusepath{stroke}
\pgfpathmoveto{\pgfpoint{283.532410pt}{197.302246pt}}
\pgflineto{\pgfpoint{283.110596pt}{196.615097pt}}
\pgfusepath{stroke}
\pgfpathmoveto{\pgfpoint{283.675415pt}{203.354492pt}}
\pgflineto{\pgfpoint{283.110596pt}{202.602844pt}}
\pgfusepath{stroke}
\pgfpathmoveto{\pgfpoint{283.898010pt}{209.378021pt}}
\pgflineto{\pgfpoint{283.110596pt}{208.590607pt}}
\pgfusepath{stroke}
\pgfpathmoveto{\pgfpoint{284.217224pt}{215.288315pt}}
\pgflineto{\pgfpoint{283.110596pt}{214.578354pt}}
\pgfusepath{stroke}
\pgfpathmoveto{\pgfpoint{284.525452pt}{220.948761pt}}
\pgflineto{\pgfpoint{283.110596pt}{220.566116pt}}
\pgfusepath{stroke}
\pgfpathmoveto{\pgfpoint{284.538330pt}{226.430145pt}}
\pgflineto{\pgfpoint{283.110596pt}{226.553864pt}}
\pgfusepath{stroke}
\pgfpathmoveto{\pgfpoint{284.255768pt}{232.087555pt}}
\pgflineto{\pgfpoint{283.110596pt}{232.541626pt}}
\pgfusepath{stroke}
\pgfpathmoveto{\pgfpoint{283.961731pt}{237.991730pt}}
\pgflineto{\pgfpoint{283.110596pt}{238.529388pt}}
\pgfusepath{stroke}
\pgfpathmoveto{\pgfpoint{283.763519pt}{244.006042pt}}
\pgflineto{\pgfpoint{283.110596pt}{244.517136pt}}
\pgfusepath{stroke}
\pgfpathmoveto{\pgfpoint{283.643646pt}{250.045929pt}}
\pgflineto{\pgfpoint{283.110596pt}{250.504883pt}}
\pgfusepath{stroke}
\pgfpathmoveto{\pgfpoint{283.571777pt}{256.082458pt}}
\pgflineto{\pgfpoint{283.110596pt}{256.492645pt}}
\pgfusepath{stroke}
\pgfpathmoveto{\pgfpoint{283.527435pt}{262.108459pt}}
\pgflineto{\pgfpoint{283.110596pt}{262.480408pt}}
\pgfusepath{stroke}
\pgfpathmoveto{\pgfpoint{283.498505pt}{268.123566pt}}
\pgflineto{\pgfpoint{283.110596pt}{268.468170pt}}
\pgfusepath{stroke}
\pgfpathmoveto{\pgfpoint{283.477875pt}{274.129425pt}}
\pgflineto{\pgfpoint{283.110596pt}{274.455902pt}}
\pgfusepath{stroke}
\pgfpathmoveto{\pgfpoint{283.461243pt}{280.128021pt}}
\pgflineto{\pgfpoint{283.110596pt}{280.443665pt}}
\pgfusepath{stroke}
\pgfpathmoveto{\pgfpoint{283.445984pt}{286.121429pt}}
\pgflineto{\pgfpoint{283.110596pt}{286.431427pt}}
\pgfusepath{stroke}
\pgfpathmoveto{\pgfpoint{283.430695pt}{292.111420pt}}
\pgflineto{\pgfpoint{283.110596pt}{292.419189pt}}
\pgfusepath{stroke}
\pgfpathmoveto{\pgfpoint{283.414734pt}{298.099548pt}}
\pgflineto{\pgfpoint{283.110596pt}{298.406921pt}}
\pgfusepath{stroke}
\pgfpathmoveto{\pgfpoint{283.397949pt}{304.086975pt}}
\pgflineto{\pgfpoint{283.110596pt}{304.394714pt}}
\pgfusepath{stroke}
\pgfpathmoveto{\pgfpoint{283.380524pt}{310.074585pt}}
\pgflineto{\pgfpoint{283.110596pt}{310.382446pt}}
\pgfusepath{stroke}
\pgfpathmoveto{\pgfpoint{283.362793pt}{316.062958pt}}
\pgflineto{\pgfpoint{283.110596pt}{316.370178pt}}
\pgfusepath{stroke}
\pgfpathmoveto{\pgfpoint{283.345123pt}{322.052368pt}}
\pgflineto{\pgfpoint{283.110596pt}{322.357971pt}}
\pgfusepath{stroke}
\pgfpathmoveto{\pgfpoint{283.327911pt}{328.042847pt}}
\pgflineto{\pgfpoint{283.110596pt}{328.345703pt}}
\pgfusepath{stroke}
\pgfpathmoveto{\pgfpoint{283.311401pt}{334.034363pt}}
\pgflineto{\pgfpoint{283.110596pt}{334.333466pt}}
\pgfusepath{stroke}
\pgfpathmoveto{\pgfpoint{283.295776pt}{340.026794pt}}
\pgflineto{\pgfpoint{283.110596pt}{340.321228pt}}
\pgfusepath{stroke}
\pgfpathmoveto{\pgfpoint{283.281219pt}{346.019958pt}}
\pgflineto{\pgfpoint{283.110596pt}{346.308960pt}}
\pgfusepath{stroke}
\pgfpathmoveto{\pgfpoint{283.267731pt}{352.013733pt}}
\pgflineto{\pgfpoint{283.110596pt}{352.296722pt}}
\pgfusepath{stroke}
\pgfpathmoveto{\pgfpoint{283.255310pt}{358.007874pt}}
\pgflineto{\pgfpoint{283.110596pt}{358.284485pt}}
\pgfusepath{stroke}
\pgfpathmoveto{\pgfpoint{283.243988pt}{364.002319pt}}
\pgflineto{\pgfpoint{283.110596pt}{364.272247pt}}
\pgfusepath{stroke}
\pgfpathmoveto{\pgfpoint{283.233643pt}{369.996887pt}}
\pgflineto{\pgfpoint{283.110596pt}{370.260010pt}}
\pgfusepath{stroke}
\pgfpathmoveto{\pgfpoint{289.134857pt}{77.079086pt}}
\pgflineto{\pgfpoint{289.098358pt}{76.859985pt}}
\pgfusepath{stroke}
\pgfpathmoveto{\pgfpoint{289.137115pt}{83.074524pt}}
\pgflineto{\pgfpoint{289.098358pt}{82.847733pt}}
\pgfusepath{stroke}
\pgfpathmoveto{\pgfpoint{289.139587pt}{89.070534pt}}
\pgflineto{\pgfpoint{289.098358pt}{88.835495pt}}
\pgfusepath{stroke}
\pgfpathmoveto{\pgfpoint{289.142303pt}{95.067192pt}}
\pgflineto{\pgfpoint{289.098358pt}{94.823257pt}}
\pgfusepath{stroke}
\pgfpathmoveto{\pgfpoint{289.145325pt}{101.064583pt}}
\pgflineto{\pgfpoint{289.098358pt}{100.811012pt}}
\pgfusepath{stroke}
\pgfpathmoveto{\pgfpoint{289.148682pt}{107.062790pt}}
\pgflineto{\pgfpoint{289.098358pt}{106.798759pt}}
\pgfusepath{stroke}
\pgfpathmoveto{\pgfpoint{289.152435pt}{113.061943pt}}
\pgflineto{\pgfpoint{289.098358pt}{112.786522pt}}
\pgfusepath{stroke}
\pgfpathmoveto{\pgfpoint{289.156616pt}{119.062210pt}}
\pgflineto{\pgfpoint{289.098358pt}{118.774277pt}}
\pgfusepath{stroke}
\pgfpathmoveto{\pgfpoint{289.161377pt}{125.063736pt}}
\pgflineto{\pgfpoint{289.098358pt}{124.762024pt}}
\pgfusepath{stroke}
\pgfpathmoveto{\pgfpoint{289.166748pt}{131.066742pt}}
\pgflineto{\pgfpoint{289.098358pt}{130.749786pt}}
\pgfusepath{stroke}
\pgfpathmoveto{\pgfpoint{289.172943pt}{137.071518pt}}
\pgflineto{\pgfpoint{289.098358pt}{136.737534pt}}
\pgfusepath{stroke}
\pgfpathmoveto{\pgfpoint{289.180084pt}{143.078400pt}}
\pgflineto{\pgfpoint{289.098358pt}{142.725281pt}}
\pgfusepath{stroke}
\pgfpathmoveto{\pgfpoint{289.188477pt}{149.087845pt}}
\pgflineto{\pgfpoint{289.098358pt}{148.713058pt}}
\pgfusepath{stroke}
\pgfpathmoveto{\pgfpoint{289.198456pt}{155.100449pt}}
\pgflineto{\pgfpoint{289.098358pt}{154.700806pt}}
\pgfusepath{stroke}
\pgfpathmoveto{\pgfpoint{289.210480pt}{161.117004pt}}
\pgflineto{\pgfpoint{289.098358pt}{160.688568pt}}
\pgfusepath{stroke}
\pgfpathmoveto{\pgfpoint{289.225342pt}{167.138626pt}}
\pgflineto{\pgfpoint{289.098358pt}{166.676315pt}}
\pgfusepath{stroke}
\pgfpathmoveto{\pgfpoint{289.244171pt}{173.166840pt}}
\pgflineto{\pgfpoint{289.098358pt}{172.664078pt}}
\pgfusepath{stroke}
\pgfpathmoveto{\pgfpoint{289.268860pt}{179.203903pt}}
\pgflineto{\pgfpoint{289.098358pt}{178.651825pt}}
\pgfusepath{stroke}
\pgfpathmoveto{\pgfpoint{289.302643pt}{185.253098pt}}
\pgflineto{\pgfpoint{289.098358pt}{184.639587pt}}
\pgfusepath{stroke}
\pgfpathmoveto{\pgfpoint{289.351562pt}{191.319458pt}}
\pgflineto{\pgfpoint{289.098358pt}{190.627335pt}}
\pgfusepath{stroke}
\pgfpathmoveto{\pgfpoint{289.427795pt}{197.410370pt}}
\pgflineto{\pgfpoint{289.098358pt}{196.615097pt}}
\pgfusepath{stroke}
\pgfpathmoveto{\pgfpoint{289.558228pt}{203.535538pt}}
\pgflineto{\pgfpoint{289.098358pt}{202.602844pt}}
\pgfusepath{stroke}
\pgfpathmoveto{\pgfpoint{289.808319pt}{209.697235pt}}
\pgflineto{\pgfpoint{289.098358pt}{208.590607pt}}
\pgfusepath{stroke}
\pgfpathmoveto{\pgfpoint{290.333832pt}{215.813828pt}}
\pgflineto{\pgfpoint{289.098358pt}{214.578354pt}}
\pgfusepath{stroke}
\pgfpathmoveto{\pgfpoint{291.219757pt}{221.364685pt}}
\pgflineto{\pgfpoint{289.098358pt}{220.566116pt}}
\pgfusepath{stroke}
\pgfpathmoveto{\pgfpoint{291.234222pt}{226.041534pt}}
\pgflineto{\pgfpoint{289.098358pt}{226.553864pt}}
\pgfusepath{stroke}
\pgfpathmoveto{\pgfpoint{290.377136pt}{231.589096pt}}
\pgflineto{\pgfpoint{289.098358pt}{232.541626pt}}
\pgfusepath{stroke}
\pgfpathmoveto{\pgfpoint{289.880066pt}{237.698990pt}}
\pgflineto{\pgfpoint{289.098358pt}{238.529388pt}}
\pgfusepath{stroke}
\pgfpathmoveto{\pgfpoint{289.657745pt}{243.850494pt}}
\pgflineto{\pgfpoint{289.098358pt}{244.517136pt}}
\pgfusepath{stroke}
\pgfpathmoveto{\pgfpoint{289.553833pt}{249.961807pt}}
\pgflineto{\pgfpoint{289.098358pt}{250.504883pt}}
\pgfusepath{stroke}
\pgfpathmoveto{\pgfpoint{289.502319pt}{256.035156pt}}
\pgflineto{\pgfpoint{289.098358pt}{256.492645pt}}
\pgfusepath{stroke}
\pgfpathmoveto{\pgfpoint{289.475464pt}{262.080383pt}}
\pgflineto{\pgfpoint{289.098358pt}{262.480408pt}}
\pgfusepath{stroke}
\pgfpathmoveto{\pgfpoint{289.460297pt}{268.105255pt}}
\pgflineto{\pgfpoint{289.098358pt}{268.468170pt}}
\pgfusepath{stroke}
\pgfpathmoveto{\pgfpoint{289.450012pt}{274.115479pt}}
\pgflineto{\pgfpoint{289.098358pt}{274.455902pt}}
\pgfusepath{stroke}
\pgfpathmoveto{\pgfpoint{289.440765pt}{280.115295pt}}
\pgflineto{\pgfpoint{289.098358pt}{280.443665pt}}
\pgfusepath{stroke}
\pgfpathmoveto{\pgfpoint{289.430450pt}{286.108154pt}}
\pgflineto{\pgfpoint{289.098358pt}{286.431427pt}}
\pgfusepath{stroke}
\pgfpathmoveto{\pgfpoint{289.417999pt}{292.096802pt}}
\pgflineto{\pgfpoint{289.098358pt}{292.419189pt}}
\pgfusepath{stroke}
\pgfpathmoveto{\pgfpoint{289.403259pt}{298.083435pt}}
\pgflineto{\pgfpoint{289.098358pt}{298.406921pt}}
\pgfusepath{stroke}
\pgfpathmoveto{\pgfpoint{289.386475pt}{304.069641pt}}
\pgflineto{\pgfpoint{289.098358pt}{304.394714pt}}
\pgfusepath{stroke}
\pgfpathmoveto{\pgfpoint{289.368317pt}{310.056519pt}}
\pgflineto{\pgfpoint{289.098358pt}{310.382446pt}}
\pgfusepath{stroke}
\pgfpathmoveto{\pgfpoint{289.349426pt}{316.044617pt}}
\pgflineto{\pgfpoint{289.098358pt}{316.370178pt}}
\pgfusepath{stroke}
\pgfpathmoveto{\pgfpoint{289.330475pt}{322.034241pt}}
\pgflineto{\pgfpoint{289.098358pt}{322.357971pt}}
\pgfusepath{stroke}
\pgfpathmoveto{\pgfpoint{289.312012pt}{328.025330pt}}
\pgflineto{\pgfpoint{289.098358pt}{328.345703pt}}
\pgfusepath{stroke}
\pgfpathmoveto{\pgfpoint{289.294403pt}{334.017761pt}}
\pgflineto{\pgfpoint{289.098358pt}{334.333466pt}}
\pgfusepath{stroke}
\pgfpathmoveto{\pgfpoint{289.277893pt}{340.011292pt}}
\pgflineto{\pgfpoint{289.098358pt}{340.321228pt}}
\pgfusepath{stroke}
\pgfpathmoveto{\pgfpoint{289.262634pt}{346.005646pt}}
\pgflineto{\pgfpoint{289.098358pt}{346.308960pt}}
\pgfusepath{stroke}
\pgfpathmoveto{\pgfpoint{289.248657pt}{352.000610pt}}
\pgflineto{\pgfpoint{289.098358pt}{352.296722pt}}
\pgfusepath{stroke}
\pgfpathmoveto{\pgfpoint{289.235962pt}{357.995911pt}}
\pgflineto{\pgfpoint{289.098358pt}{358.284485pt}}
\pgfusepath{stroke}
\pgfpathmoveto{\pgfpoint{289.224457pt}{363.991425pt}}
\pgflineto{\pgfpoint{289.098358pt}{364.272247pt}}
\pgfusepath{stroke}
\pgfpathmoveto{\pgfpoint{289.214050pt}{369.987030pt}}
\pgflineto{\pgfpoint{289.098358pt}{370.260010pt}}
\pgfusepath{stroke}
\pgfpathmoveto{\pgfpoint{295.115112pt}{77.080994pt}}
\pgflineto{\pgfpoint{295.086121pt}{76.859985pt}}
\pgfusepath{stroke}
\pgfpathmoveto{\pgfpoint{295.116821pt}{83.076599pt}}
\pgflineto{\pgfpoint{295.086121pt}{82.847733pt}}
\pgfusepath{stroke}
\pgfpathmoveto{\pgfpoint{295.118652pt}{89.072800pt}}
\pgflineto{\pgfpoint{295.086121pt}{88.835495pt}}
\pgfusepath{stroke}
\pgfpathmoveto{\pgfpoint{295.120697pt}{95.069687pt}}
\pgflineto{\pgfpoint{295.086121pt}{94.823257pt}}
\pgfusepath{stroke}
\pgfpathmoveto{\pgfpoint{295.122925pt}{101.067337pt}}
\pgflineto{\pgfpoint{295.086121pt}{100.811012pt}}
\pgfusepath{stroke}
\pgfpathmoveto{\pgfpoint{295.125366pt}{107.065842pt}}
\pgflineto{\pgfpoint{295.086121pt}{106.798759pt}}
\pgfusepath{stroke}
\pgfpathmoveto{\pgfpoint{295.128052pt}{113.065361pt}}
\pgflineto{\pgfpoint{295.086121pt}{112.786522pt}}
\pgfusepath{stroke}
\pgfpathmoveto{\pgfpoint{295.131042pt}{119.066017pt}}
\pgflineto{\pgfpoint{295.086121pt}{118.774277pt}}
\pgfusepath{stroke}
\pgfpathmoveto{\pgfpoint{295.134369pt}{125.068024pt}}
\pgflineto{\pgfpoint{295.086121pt}{124.762024pt}}
\pgfusepath{stroke}
\pgfpathmoveto{\pgfpoint{295.138092pt}{131.071609pt}}
\pgflineto{\pgfpoint{295.086121pt}{130.749786pt}}
\pgfusepath{stroke}
\pgfpathmoveto{\pgfpoint{295.142273pt}{137.077072pt}}
\pgflineto{\pgfpoint{295.086121pt}{136.737534pt}}
\pgfusepath{stroke}
\pgfpathmoveto{\pgfpoint{295.147003pt}{143.084808pt}}
\pgflineto{\pgfpoint{295.086121pt}{142.725281pt}}
\pgfusepath{stroke}
\pgfpathmoveto{\pgfpoint{295.152435pt}{149.095337pt}}
\pgflineto{\pgfpoint{295.086121pt}{148.713058pt}}
\pgfusepath{stroke}
\pgfpathmoveto{\pgfpoint{295.158661pt}{155.109299pt}}
\pgflineto{\pgfpoint{295.086121pt}{154.700806pt}}
\pgfusepath{stroke}
\pgfpathmoveto{\pgfpoint{295.165955pt}{161.127670pt}}
\pgflineto{\pgfpoint{295.086121pt}{160.688568pt}}
\pgfusepath{stroke}
\pgfpathmoveto{\pgfpoint{295.174622pt}{167.151749pt}}
\pgflineto{\pgfpoint{295.086121pt}{166.676315pt}}
\pgfusepath{stroke}
\pgfpathmoveto{\pgfpoint{295.185120pt}{173.183472pt}}
\pgflineto{\pgfpoint{295.086121pt}{172.664078pt}}
\pgfusepath{stroke}
\pgfpathmoveto{\pgfpoint{295.198212pt}{179.225815pt}}
\pgflineto{\pgfpoint{295.086121pt}{178.651825pt}}
\pgfusepath{stroke}
\pgfpathmoveto{\pgfpoint{295.215271pt}{185.283478pt}}
\pgflineto{\pgfpoint{295.086121pt}{184.639587pt}}
\pgfusepath{stroke}
\pgfpathmoveto{\pgfpoint{295.238831pt}{191.364548pt}}
\pgflineto{\pgfpoint{295.086121pt}{190.627335pt}}
\pgfusepath{stroke}
\pgfpathmoveto{\pgfpoint{295.274445pt}{197.484009pt}}
\pgflineto{\pgfpoint{295.086121pt}{196.615097pt}}
\pgfusepath{stroke}
\pgfpathmoveto{\pgfpoint{295.336090pt}{203.672806pt}}
\pgflineto{\pgfpoint{295.086121pt}{202.602844pt}}
\pgfusepath{stroke}
\pgfpathmoveto{\pgfpoint{295.468750pt}{210.005447pt}}
\pgflineto{\pgfpoint{295.086121pt}{208.590607pt}}
\pgfusepath{stroke}
\pgfpathmoveto{\pgfpoint{295.884674pt}{216.699753pt}}
\pgflineto{\pgfpoint{295.086121pt}{214.578354pt}}
\pgfusepath{stroke}
\pgfpathmoveto{\pgfpoint{298.516998pt}{223.996994pt}}
\pgflineto{\pgfpoint{295.086121pt}{220.566116pt}}
\pgfusepath{stroke}
\pgfpathmoveto{\pgfpoint{298.533173pt}{223.439835pt}}
\pgflineto{\pgfpoint{295.086121pt}{226.553864pt}}
\pgfusepath{stroke}
\pgfpathmoveto{\pgfpoint{295.933136pt}{230.733597pt}}
\pgflineto{\pgfpoint{295.086121pt}{232.541626pt}}
\pgfusepath{stroke}
\pgfpathmoveto{\pgfpoint{295.549255pt}{237.420776pt}}
\pgflineto{\pgfpoint{295.086121pt}{238.529388pt}}
\pgfusepath{stroke}
\pgfpathmoveto{\pgfpoint{295.448181pt}{243.742371pt}}
\pgflineto{\pgfpoint{295.086121pt}{244.517136pt}}
\pgfusepath{stroke}
\pgfpathmoveto{\pgfpoint{295.417084pt}{249.915878pt}}
\pgflineto{\pgfpoint{295.086121pt}{250.504883pt}}
\pgfusepath{stroke}
\pgfpathmoveto{\pgfpoint{295.410278pt}{256.015533pt}}
\pgflineto{\pgfpoint{295.086121pt}{256.492645pt}}
\pgfusepath{stroke}
\pgfpathmoveto{\pgfpoint{295.412720pt}{262.072266pt}}
\pgflineto{\pgfpoint{295.086121pt}{262.480408pt}}
\pgfusepath{stroke}
\pgfpathmoveto{\pgfpoint{295.417664pt}{268.101379pt}}
\pgflineto{\pgfpoint{295.086121pt}{268.468170pt}}
\pgfusepath{stroke}
\pgfpathmoveto{\pgfpoint{295.421387pt}{274.111755pt}}
\pgflineto{\pgfpoint{295.086121pt}{274.455902pt}}
\pgfusepath{stroke}
\pgfpathmoveto{\pgfpoint{295.421600pt}{280.109436pt}}
\pgflineto{\pgfpoint{295.086121pt}{280.443665pt}}
\pgfusepath{stroke}
\pgfpathmoveto{\pgfpoint{295.417145pt}{286.098999pt}}
\pgflineto{\pgfpoint{295.086121pt}{286.431427pt}}
\pgfusepath{stroke}
\pgfpathmoveto{\pgfpoint{295.407715pt}{292.084137pt}}
\pgflineto{\pgfpoint{295.086121pt}{292.419189pt}}
\pgfusepath{stroke}
\pgfpathmoveto{\pgfpoint{295.393829pt}{298.067657pt}}
\pgflineto{\pgfpoint{295.086121pt}{298.406921pt}}
\pgfusepath{stroke}
\pgfpathmoveto{\pgfpoint{295.376434pt}{304.051575pt}}
\pgflineto{\pgfpoint{295.086121pt}{304.394714pt}}
\pgfusepath{stroke}
\pgfpathmoveto{\pgfpoint{295.356812pt}{310.037079pt}}
\pgflineto{\pgfpoint{295.086121pt}{310.382446pt}}
\pgfusepath{stroke}
\pgfpathmoveto{\pgfpoint{295.336121pt}{316.024719pt}}
\pgflineto{\pgfpoint{295.086121pt}{316.370178pt}}
\pgfusepath{stroke}
\pgfpathmoveto{\pgfpoint{295.315338pt}{322.014618pt}}
\pgflineto{\pgfpoint{295.086121pt}{322.357971pt}}
\pgfusepath{stroke}
\pgfpathmoveto{\pgfpoint{295.295227pt}{328.006561pt}}
\pgflineto{\pgfpoint{295.086121pt}{328.345703pt}}
\pgfusepath{stroke}
\pgfpathmoveto{\pgfpoint{295.276276pt}{334.000122pt}}
\pgflineto{\pgfpoint{295.086121pt}{334.333466pt}}
\pgfusepath{stroke}
\pgfpathmoveto{\pgfpoint{295.258759pt}{339.994995pt}}
\pgflineto{\pgfpoint{295.086121pt}{340.321228pt}}
\pgfusepath{stroke}
\pgfpathmoveto{\pgfpoint{295.242798pt}{345.990723pt}}
\pgflineto{\pgfpoint{295.086121pt}{346.308960pt}}
\pgfusepath{stroke}
\pgfpathmoveto{\pgfpoint{295.228363pt}{351.987061pt}}
\pgflineto{\pgfpoint{295.086121pt}{352.296722pt}}
\pgfusepath{stroke}
\pgfpathmoveto{\pgfpoint{295.215393pt}{357.983704pt}}
\pgflineto{\pgfpoint{295.086121pt}{358.284485pt}}
\pgfusepath{stroke}
\pgfpathmoveto{\pgfpoint{295.203796pt}{363.980469pt}}
\pgflineto{\pgfpoint{295.086121pt}{364.272247pt}}
\pgfusepath{stroke}
\pgfpathmoveto{\pgfpoint{295.193451pt}{369.977203pt}}
\pgflineto{\pgfpoint{295.086121pt}{370.260010pt}}
\pgfusepath{stroke}
\pgfpathmoveto{\pgfpoint{301.095245pt}{77.082352pt}}
\pgflineto{\pgfpoint{301.073853pt}{76.859985pt}}
\pgfusepath{stroke}
\pgfpathmoveto{\pgfpoint{301.096375pt}{83.078064pt}}
\pgflineto{\pgfpoint{301.073853pt}{82.847733pt}}
\pgfusepath{stroke}
\pgfpathmoveto{\pgfpoint{301.097565pt}{89.074394pt}}
\pgflineto{\pgfpoint{301.073853pt}{88.835495pt}}
\pgfusepath{stroke}
\pgfpathmoveto{\pgfpoint{301.098877pt}{95.071426pt}}
\pgflineto{\pgfpoint{301.073853pt}{94.823257pt}}
\pgfusepath{stroke}
\pgfpathmoveto{\pgfpoint{301.100281pt}{101.069214pt}}
\pgflineto{\pgfpoint{301.073853pt}{100.811012pt}}
\pgfusepath{stroke}
\pgfpathmoveto{\pgfpoint{301.101776pt}{107.067902pt}}
\pgflineto{\pgfpoint{301.073853pt}{106.798759pt}}
\pgfusepath{stroke}
\pgfpathmoveto{\pgfpoint{301.103394pt}{113.067612pt}}
\pgflineto{\pgfpoint{301.073853pt}{112.786522pt}}
\pgfusepath{stroke}
\pgfpathmoveto{\pgfpoint{301.105103pt}{119.068489pt}}
\pgflineto{\pgfpoint{301.073853pt}{118.774277pt}}
\pgfusepath{stroke}
\pgfpathmoveto{\pgfpoint{301.106964pt}{125.070740pt}}
\pgflineto{\pgfpoint{301.073853pt}{124.762024pt}}
\pgfusepath{stroke}
\pgfpathmoveto{\pgfpoint{301.108948pt}{131.074615pt}}
\pgflineto{\pgfpoint{301.073853pt}{130.749786pt}}
\pgfusepath{stroke}
\pgfpathmoveto{\pgfpoint{301.111053pt}{137.080383pt}}
\pgflineto{\pgfpoint{301.073853pt}{136.737534pt}}
\pgfusepath{stroke}
\pgfpathmoveto{\pgfpoint{301.113251pt}{143.088470pt}}
\pgflineto{\pgfpoint{301.073853pt}{142.725281pt}}
\pgfusepath{stroke}
\pgfpathmoveto{\pgfpoint{301.115540pt}{149.099396pt}}
\pgflineto{\pgfpoint{301.073853pt}{148.713058pt}}
\pgfusepath{stroke}
\pgfpathmoveto{\pgfpoint{301.117859pt}{155.113815pt}}
\pgflineto{\pgfpoint{301.073853pt}{154.700806pt}}
\pgfusepath{stroke}
\pgfpathmoveto{\pgfpoint{301.120087pt}{161.132706pt}}
\pgflineto{\pgfpoint{301.073853pt}{160.688568pt}}
\pgfusepath{stroke}
\pgfpathmoveto{\pgfpoint{301.122101pt}{167.157379pt}}
\pgflineto{\pgfpoint{301.073853pt}{166.676315pt}}
\pgfusepath{stroke}
\pgfpathmoveto{\pgfpoint{301.123505pt}{173.189789pt}}
\pgflineto{\pgfpoint{301.073853pt}{172.664078pt}}
\pgfusepath{stroke}
\pgfpathmoveto{\pgfpoint{301.123779pt}{179.232910pt}}
\pgflineto{\pgfpoint{301.073853pt}{178.651825pt}}
\pgfusepath{stroke}
\pgfpathmoveto{\pgfpoint{301.121765pt}{185.291458pt}}
\pgflineto{\pgfpoint{301.073853pt}{184.639587pt}}
\pgfusepath{stroke}
\pgfpathmoveto{\pgfpoint{301.115143pt}{191.373535pt}}
\pgflineto{\pgfpoint{301.073853pt}{190.627335pt}}
\pgfusepath{stroke}
\pgfpathmoveto{\pgfpoint{301.098633pt}{197.494156pt}}
\pgflineto{\pgfpoint{301.073853pt}{196.615097pt}}
\pgfusepath{stroke}
\pgfpathmoveto{\pgfpoint{301.058533pt}{203.684250pt}}
\pgflineto{\pgfpoint{301.073853pt}{202.602844pt}}
\pgfusepath{stroke}
\pgfpathmoveto{\pgfpoint{300.950134pt}{210.018341pt}}
\pgflineto{\pgfpoint{301.073853pt}{208.590607pt}}
\pgfusepath{stroke}
\pgfpathmoveto{\pgfpoint{300.561523pt}{216.714218pt}}
\pgflineto{\pgfpoint{301.073853pt}{214.578354pt}}
\pgfusepath{stroke}
\pgfpathmoveto{\pgfpoint{297.959839pt}{224.013168pt}}
\pgflineto{\pgfpoint{301.073853pt}{220.566116pt}}
\pgfusepath{stroke}
\pgfpathmoveto{\pgfpoint{297.977722pt}{223.457733pt}}
\pgflineto{\pgfpoint{301.073853pt}{226.553864pt}}
\pgfusepath{stroke}
\pgfpathmoveto{\pgfpoint{300.615295pt}{230.753174pt}}
\pgflineto{\pgfpoint{301.073853pt}{232.541626pt}}
\pgfusepath{stroke}
\pgfpathmoveto{\pgfpoint{301.039856pt}{237.441788pt}}
\pgflineto{\pgfpoint{301.073853pt}{238.529388pt}}
\pgfusepath{stroke}
\pgfpathmoveto{\pgfpoint{301.184143pt}{243.764328pt}}
\pgflineto{\pgfpoint{301.073853pt}{244.517136pt}}
\pgfusepath{stroke}
\pgfpathmoveto{\pgfpoint{301.259644pt}{249.937988pt}}
\pgflineto{\pgfpoint{301.073853pt}{250.504883pt}}
\pgfusepath{stroke}
\pgfpathmoveto{\pgfpoint{301.310089pt}{256.036621pt}}
\pgflineto{\pgfpoint{301.073853pt}{256.492645pt}}
\pgfusepath{stroke}
\pgfpathmoveto{\pgfpoint{301.347870pt}{262.090851pt}}
\pgflineto{\pgfpoint{301.073853pt}{262.480408pt}}
\pgfusepath{stroke}
\pgfpathmoveto{\pgfpoint{301.376465pt}{268.115814pt}}
\pgflineto{\pgfpoint{301.073853pt}{268.468170pt}}
\pgfusepath{stroke}
\pgfpathmoveto{\pgfpoint{301.396393pt}{274.120544pt}}
\pgflineto{\pgfpoint{301.073853pt}{274.455902pt}}
\pgfusepath{stroke}
\pgfpathmoveto{\pgfpoint{301.407257pt}{280.111511pt}}
\pgflineto{\pgfpoint{301.073853pt}{280.443665pt}}
\pgfusepath{stroke}
\pgfpathmoveto{\pgfpoint{301.408905pt}{286.094177pt}}
\pgflineto{\pgfpoint{301.073853pt}{286.431427pt}}
\pgfusepath{stroke}
\pgfpathmoveto{\pgfpoint{301.401978pt}{292.072937pt}}
\pgflineto{\pgfpoint{301.073853pt}{292.419189pt}}
\pgfusepath{stroke}
\pgfpathmoveto{\pgfpoint{301.387909pt}{298.051392pt}}
\pgflineto{\pgfpoint{301.073853pt}{298.406921pt}}
\pgfusepath{stroke}
\pgfpathmoveto{\pgfpoint{301.368652pt}{304.031799pt}}
\pgflineto{\pgfpoint{301.073853pt}{304.394714pt}}
\pgfusepath{stroke}
\pgfpathmoveto{\pgfpoint{301.346344pt}{310.015411pt}}
\pgflineto{\pgfpoint{301.073853pt}{310.382446pt}}
\pgfusepath{stroke}
\pgfpathmoveto{\pgfpoint{301.322815pt}{316.002563pt}}
\pgflineto{\pgfpoint{301.073853pt}{316.370178pt}}
\pgfusepath{stroke}
\pgfpathmoveto{\pgfpoint{301.299469pt}{321.992981pt}}
\pgflineto{\pgfpoint{301.073853pt}{322.357971pt}}
\pgfusepath{stroke}
\pgfpathmoveto{\pgfpoint{301.277283pt}{327.986053pt}}
\pgflineto{\pgfpoint{301.073853pt}{328.345703pt}}
\pgfusepath{stroke}
\pgfpathmoveto{\pgfpoint{301.256714pt}{333.981201pt}}
\pgflineto{\pgfpoint{301.073853pt}{334.333466pt}}
\pgfusepath{stroke}
\pgfpathmoveto{\pgfpoint{301.238068pt}{339.977783pt}}
\pgflineto{\pgfpoint{301.073853pt}{340.321228pt}}
\pgfusepath{stroke}
\pgfpathmoveto{\pgfpoint{301.221375pt}{345.975281pt}}
\pgflineto{\pgfpoint{301.073853pt}{346.308960pt}}
\pgfusepath{stroke}
\pgfpathmoveto{\pgfpoint{301.206543pt}{351.973236pt}}
\pgflineto{\pgfpoint{301.073853pt}{352.296722pt}}
\pgfusepath{stroke}
\pgfpathmoveto{\pgfpoint{301.193451pt}{357.971436pt}}
\pgflineto{\pgfpoint{301.073853pt}{358.284485pt}}
\pgfusepath{stroke}
\pgfpathmoveto{\pgfpoint{301.181885pt}{363.969574pt}}
\pgflineto{\pgfpoint{301.073853pt}{364.272247pt}}
\pgfusepath{stroke}
\pgfpathmoveto{\pgfpoint{301.171722pt}{369.967529pt}}
\pgflineto{\pgfpoint{301.073853pt}{370.260010pt}}
\pgfusepath{stroke}
\pgfpathmoveto{\pgfpoint{307.075287pt}{77.083160pt}}
\pgflineto{\pgfpoint{307.061615pt}{76.859985pt}}
\pgfusepath{stroke}
\pgfpathmoveto{\pgfpoint{307.075836pt}{83.078934pt}}
\pgflineto{\pgfpoint{307.061615pt}{82.847733pt}}
\pgfusepath{stroke}
\pgfpathmoveto{\pgfpoint{307.076385pt}{89.075302pt}}
\pgflineto{\pgfpoint{307.061615pt}{88.835495pt}}
\pgfusepath{stroke}
\pgfpathmoveto{\pgfpoint{307.076965pt}{95.072365pt}}
\pgflineto{\pgfpoint{307.061615pt}{94.823257pt}}
\pgfusepath{stroke}
\pgfpathmoveto{\pgfpoint{307.077515pt}{101.070229pt}}
\pgflineto{\pgfpoint{307.061615pt}{100.811012pt}}
\pgfusepath{stroke}
\pgfpathmoveto{\pgfpoint{307.078064pt}{107.068947pt}}
\pgflineto{\pgfpoint{307.061615pt}{106.798759pt}}
\pgfusepath{stroke}
\pgfpathmoveto{\pgfpoint{307.078613pt}{113.068695pt}}
\pgflineto{\pgfpoint{307.061615pt}{112.786522pt}}
\pgfusepath{stroke}
\pgfpathmoveto{\pgfpoint{307.079071pt}{119.069595pt}}
\pgflineto{\pgfpoint{307.061615pt}{118.774277pt}}
\pgfusepath{stroke}
\pgfpathmoveto{\pgfpoint{307.079437pt}{125.071854pt}}
\pgflineto{\pgfpoint{307.061615pt}{124.762024pt}}
\pgfusepath{stroke}
\pgfpathmoveto{\pgfpoint{307.079681pt}{131.075684pt}}
\pgflineto{\pgfpoint{307.061615pt}{130.749786pt}}
\pgfusepath{stroke}
\pgfpathmoveto{\pgfpoint{307.079712pt}{137.081390pt}}
\pgflineto{\pgfpoint{307.061615pt}{136.737534pt}}
\pgfusepath{stroke}
\pgfpathmoveto{\pgfpoint{307.079407pt}{143.089325pt}}
\pgflineto{\pgfpoint{307.061615pt}{142.725281pt}}
\pgfusepath{stroke}
\pgfpathmoveto{\pgfpoint{307.078613pt}{149.099976pt}}
\pgflineto{\pgfpoint{307.061615pt}{148.713058pt}}
\pgfusepath{stroke}
\pgfpathmoveto{\pgfpoint{307.077087pt}{155.113937pt}}
\pgflineto{\pgfpoint{307.061615pt}{154.700806pt}}
\pgfusepath{stroke}
\pgfpathmoveto{\pgfpoint{307.074463pt}{161.132050pt}}
\pgflineto{\pgfpoint{307.061615pt}{160.688568pt}}
\pgfusepath{stroke}
\pgfpathmoveto{\pgfpoint{307.070099pt}{167.155441pt}}
\pgflineto{\pgfpoint{307.061615pt}{166.676315pt}}
\pgfusepath{stroke}
\pgfpathmoveto{\pgfpoint{307.062988pt}{173.185684pt}}
\pgflineto{\pgfpoint{307.061615pt}{172.664078pt}}
\pgfusepath{stroke}
\pgfpathmoveto{\pgfpoint{307.051483pt}{179.225067pt}}
\pgflineto{\pgfpoint{307.061615pt}{178.651825pt}}
\pgfusepath{stroke}
\pgfpathmoveto{\pgfpoint{307.032501pt}{185.276932pt}}
\pgflineto{\pgfpoint{307.061615pt}{184.639587pt}}
\pgfusepath{stroke}
\pgfpathmoveto{\pgfpoint{307.000244pt}{191.346298pt}}
\pgflineto{\pgfpoint{307.061615pt}{190.627335pt}}
\pgfusepath{stroke}
\pgfpathmoveto{\pgfpoint{306.942841pt}{197.440659pt}}
\pgflineto{\pgfpoint{307.061615pt}{196.615097pt}}
\pgfusepath{stroke}
\pgfpathmoveto{\pgfpoint{306.833618pt}{203.569717pt}}
\pgflineto{\pgfpoint{307.061615pt}{202.602844pt}}
\pgfusepath{stroke}
\pgfpathmoveto{\pgfpoint{306.607544pt}{209.735764pt}}
\pgflineto{\pgfpoint{307.061615pt}{208.590607pt}}
\pgfusepath{stroke}
\pgfpathmoveto{\pgfpoint{306.109100pt}{215.857132pt}}
\pgflineto{\pgfpoint{307.061615pt}{214.578354pt}}
\pgfusepath{stroke}
\pgfpathmoveto{\pgfpoint{305.253601pt}{221.413132pt}}
\pgflineto{\pgfpoint{307.061615pt}{220.566116pt}}
\pgfusepath{stroke}
\pgfpathmoveto{\pgfpoint{305.273193pt}{226.095306pt}}
\pgflineto{\pgfpoint{307.061615pt}{226.553864pt}}
\pgfusepath{stroke}
\pgfpathmoveto{\pgfpoint{306.168060pt}{231.648056pt}}
\pgflineto{\pgfpoint{307.061615pt}{232.541626pt}}
\pgfusepath{stroke}
\pgfpathmoveto{\pgfpoint{306.706482pt}{237.762512pt}}
\pgflineto{\pgfpoint{307.061615pt}{238.529388pt}}
\pgfusepath{stroke}
\pgfpathmoveto{\pgfpoint{306.973267pt}{243.917175pt}}
\pgflineto{\pgfpoint{307.061615pt}{244.517136pt}}
\pgfusepath{stroke}
\pgfpathmoveto{\pgfpoint{307.123535pt}{250.029297pt}}
\pgflineto{\pgfpoint{307.061615pt}{250.504883pt}}
\pgfusepath{stroke}
\pgfpathmoveto{\pgfpoint{307.221405pt}{256.099945pt}}
\pgflineto{\pgfpoint{307.061615pt}{256.492645pt}}
\pgfusepath{stroke}
\pgfpathmoveto{\pgfpoint{307.291748pt}{262.137817pt}}
\pgflineto{\pgfpoint{307.061615pt}{262.480408pt}}
\pgfusepath{stroke}
\pgfpathmoveto{\pgfpoint{307.343842pt}{268.150116pt}}
\pgflineto{\pgfpoint{307.061615pt}{268.468170pt}}
\pgfusepath{stroke}
\pgfpathmoveto{\pgfpoint{307.380524pt}{274.142853pt}}
\pgflineto{\pgfpoint{307.061615pt}{274.455902pt}}
\pgfusepath{stroke}
\pgfpathmoveto{\pgfpoint{307.402283pt}{280.121857pt}}
\pgflineto{\pgfpoint{307.061615pt}{280.443665pt}}
\pgfusepath{stroke}
\pgfpathmoveto{\pgfpoint{307.409424pt}{286.093018pt}}
\pgflineto{\pgfpoint{307.061615pt}{286.431427pt}}
\pgfusepath{stroke}
\pgfpathmoveto{\pgfpoint{307.403473pt}{292.061920pt}}
\pgflineto{\pgfpoint{307.061615pt}{292.419189pt}}
\pgfusepath{stroke}
\pgfpathmoveto{\pgfpoint{307.387146pt}{298.032898pt}}
\pgflineto{\pgfpoint{307.061615pt}{298.406921pt}}
\pgfusepath{stroke}
\pgfpathmoveto{\pgfpoint{307.363892pt}{304.008667pt}}
\pgflineto{\pgfpoint{307.061615pt}{304.394714pt}}
\pgfusepath{stroke}
\pgfpathmoveto{\pgfpoint{307.337036pt}{309.990143pt}}
\pgflineto{\pgfpoint{307.061615pt}{310.382446pt}}
\pgfusepath{stroke}
\pgfpathmoveto{\pgfpoint{307.309265pt}{315.977081pt}}
\pgflineto{\pgfpoint{307.061615pt}{316.370178pt}}
\pgfusepath{stroke}
\pgfpathmoveto{\pgfpoint{307.282440pt}{321.968597pt}}
\pgflineto{\pgfpoint{307.061615pt}{322.357971pt}}
\pgfusepath{stroke}
\pgfpathmoveto{\pgfpoint{307.257599pt}{327.963562pt}}
\pgflineto{\pgfpoint{307.061615pt}{328.345703pt}}
\pgfusepath{stroke}
\pgfpathmoveto{\pgfpoint{307.235229pt}{333.960876pt}}
\pgflineto{\pgfpoint{307.061615pt}{334.333466pt}}
\pgfusepath{stroke}
\pgfpathmoveto{\pgfpoint{307.215424pt}{339.959656pt}}
\pgflineto{\pgfpoint{307.061615pt}{340.321228pt}}
\pgfusepath{stroke}
\pgfpathmoveto{\pgfpoint{307.198090pt}{345.959290pt}}
\pgflineto{\pgfpoint{307.061615pt}{346.308960pt}}
\pgfusepath{stroke}
\pgfpathmoveto{\pgfpoint{307.183044pt}{351.959229pt}}
\pgflineto{\pgfpoint{307.061615pt}{352.296722pt}}
\pgfusepath{stroke}
\pgfpathmoveto{\pgfpoint{307.169952pt}{357.959167pt}}
\pgflineto{\pgfpoint{307.061615pt}{358.284485pt}}
\pgfusepath{stroke}
\pgfpathmoveto{\pgfpoint{307.158630pt}{363.958862pt}}
\pgflineto{\pgfpoint{307.061615pt}{364.272247pt}}
\pgfusepath{stroke}
\pgfpathmoveto{\pgfpoint{307.148773pt}{369.958191pt}}
\pgflineto{\pgfpoint{307.061615pt}{370.260010pt}}
\pgfusepath{stroke}
\pgfpathmoveto{\pgfpoint{313.055298pt}{77.083435pt}}
\pgflineto{\pgfpoint{313.049377pt}{76.859985pt}}
\pgfusepath{stroke}
\pgfpathmoveto{\pgfpoint{313.055298pt}{83.079178pt}}
\pgflineto{\pgfpoint{313.049377pt}{82.847733pt}}
\pgfusepath{stroke}
\pgfpathmoveto{\pgfpoint{313.055206pt}{89.075531pt}}
\pgflineto{\pgfpoint{313.049377pt}{88.835495pt}}
\pgfusepath{stroke}
\pgfpathmoveto{\pgfpoint{313.055054pt}{95.072556pt}}
\pgflineto{\pgfpoint{313.049377pt}{94.823257pt}}
\pgfusepath{stroke}
\pgfpathmoveto{\pgfpoint{313.054779pt}{101.070351pt}}
\pgflineto{\pgfpoint{313.049377pt}{100.811012pt}}
\pgfusepath{stroke}
\pgfpathmoveto{\pgfpoint{313.054413pt}{107.069000pt}}
\pgflineto{\pgfpoint{313.049377pt}{106.798759pt}}
\pgfusepath{stroke}
\pgfpathmoveto{\pgfpoint{313.053864pt}{113.068619pt}}
\pgflineto{\pgfpoint{313.049377pt}{112.786522pt}}
\pgfusepath{stroke}
\pgfpathmoveto{\pgfpoint{313.053101pt}{119.069374pt}}
\pgflineto{\pgfpoint{313.049377pt}{118.774277pt}}
\pgfusepath{stroke}
\pgfpathmoveto{\pgfpoint{313.052063pt}{125.071396pt}}
\pgflineto{\pgfpoint{313.049377pt}{124.762024pt}}
\pgfusepath{stroke}
\pgfpathmoveto{\pgfpoint{313.050659pt}{131.074921pt}}
\pgflineto{\pgfpoint{313.049377pt}{130.749786pt}}
\pgfusepath{stroke}
\pgfpathmoveto{\pgfpoint{313.048706pt}{137.080200pt}}
\pgflineto{\pgfpoint{313.049377pt}{136.737534pt}}
\pgfusepath{stroke}
\pgfpathmoveto{\pgfpoint{313.046051pt}{143.087509pt}}
\pgflineto{\pgfpoint{313.049377pt}{142.725281pt}}
\pgfusepath{stroke}
\pgfpathmoveto{\pgfpoint{313.042419pt}{149.097275pt}}
\pgflineto{\pgfpoint{313.049377pt}{148.713058pt}}
\pgfusepath{stroke}
\pgfpathmoveto{\pgfpoint{313.037415pt}{155.109955pt}}
\pgflineto{\pgfpoint{313.049377pt}{154.700806pt}}
\pgfusepath{stroke}
\pgfpathmoveto{\pgfpoint{313.030487pt}{161.126205pt}}
\pgflineto{\pgfpoint{313.049377pt}{160.688568pt}}
\pgfusepath{stroke}
\pgfpathmoveto{\pgfpoint{313.020721pt}{167.146759pt}}
\pgflineto{\pgfpoint{313.049377pt}{166.676315pt}}
\pgfusepath{stroke}
\pgfpathmoveto{\pgfpoint{313.006714pt}{173.172623pt}}
\pgflineto{\pgfpoint{313.049377pt}{172.664078pt}}
\pgfusepath{stroke}
\pgfpathmoveto{\pgfpoint{312.986237pt}{179.205017pt}}
\pgflineto{\pgfpoint{313.049377pt}{178.651825pt}}
\pgfusepath{stroke}
\pgfpathmoveto{\pgfpoint{312.955597pt}{185.245300pt}}
\pgflineto{\pgfpoint{313.049377pt}{184.639587pt}}
\pgfusepath{stroke}
\pgfpathmoveto{\pgfpoint{312.908386pt}{191.294586pt}}
\pgflineto{\pgfpoint{313.049377pt}{190.627335pt}}
\pgfusepath{stroke}
\pgfpathmoveto{\pgfpoint{312.833344pt}{197.352234pt}}
\pgflineto{\pgfpoint{313.049377pt}{196.615097pt}}
\pgfusepath{stroke}
\pgfpathmoveto{\pgfpoint{312.710968pt}{203.410950pt}}
\pgflineto{\pgfpoint{313.049377pt}{202.602844pt}}
\pgfusepath{stroke}
\pgfpathmoveto{\pgfpoint{312.511719pt}{209.441742pt}}
\pgflineto{\pgfpoint{313.049377pt}{208.590607pt}}
\pgfusepath{stroke}
\pgfpathmoveto{\pgfpoint{312.218994pt}{215.360077pt}}
\pgflineto{\pgfpoint{313.049377pt}{214.578354pt}}
\pgfusepath{stroke}
\pgfpathmoveto{\pgfpoint{311.940765pt}{221.029266pt}}
\pgflineto{\pgfpoint{313.049377pt}{220.566116pt}}
\pgfusepath{stroke}
\pgfpathmoveto{\pgfpoint{311.961792pt}{226.519867pt}}
\pgflineto{\pgfpoint{313.049377pt}{226.553864pt}}
\pgfusepath{stroke}
\pgfpathmoveto{\pgfpoint{312.282501pt}{232.186478pt}}
\pgflineto{\pgfpoint{313.049377pt}{232.541626pt}}
\pgfusepath{stroke}
\pgfpathmoveto{\pgfpoint{312.619080pt}{238.099091pt}}
\pgflineto{\pgfpoint{313.049377pt}{238.529388pt}}
\pgfusepath{stroke}
\pgfpathmoveto{\pgfpoint{312.864105pt}{244.119797pt}}
\pgflineto{\pgfpoint{313.049377pt}{244.517136pt}}
\pgfusepath{stroke}
\pgfpathmoveto{\pgfpoint{313.034210pt}{250.162354pt}}
\pgflineto{\pgfpoint{313.049377pt}{250.504883pt}}
\pgfusepath{stroke}
\pgfpathmoveto{\pgfpoint{313.158020pt}{256.195618pt}}
\pgflineto{\pgfpoint{313.049377pt}{256.492645pt}}
\pgfusepath{stroke}
\pgfpathmoveto{\pgfpoint{313.252899pt}{262.210114pt}}
\pgflineto{\pgfpoint{313.049377pt}{262.480408pt}}
\pgfusepath{stroke}
\pgfpathmoveto{\pgfpoint{313.326233pt}{268.203979pt}}
\pgflineto{\pgfpoint{313.049377pt}{268.468170pt}}
\pgfusepath{stroke}
\pgfpathmoveto{\pgfpoint{313.379608pt}{274.178986pt}}
\pgflineto{\pgfpoint{313.049377pt}{274.455902pt}}
\pgfusepath{stroke}
\pgfpathmoveto{\pgfpoint{313.412079pt}{280.139954pt}}
\pgflineto{\pgfpoint{313.049377pt}{280.443665pt}}
\pgfusepath{stroke}
\pgfpathmoveto{\pgfpoint{313.423309pt}{286.093933pt}}
\pgflineto{\pgfpoint{313.049377pt}{286.431427pt}}
\pgfusepath{stroke}
\pgfpathmoveto{\pgfpoint{313.415497pt}{292.048462pt}}
\pgflineto{\pgfpoint{313.049377pt}{292.419189pt}}
\pgfusepath{stroke}
\pgfpathmoveto{\pgfpoint{313.393311pt}{298.009399pt}}
\pgflineto{\pgfpoint{313.049377pt}{298.406921pt}}
\pgfusepath{stroke}
\pgfpathmoveto{\pgfpoint{313.362610pt}{303.979614pt}}
\pgflineto{\pgfpoint{313.049377pt}{304.394714pt}}
\pgfusepath{stroke}
\pgfpathmoveto{\pgfpoint{313.328461pt}{309.959290pt}}
\pgflineto{\pgfpoint{313.049377pt}{310.382446pt}}
\pgfusepath{stroke}
\pgfpathmoveto{\pgfpoint{313.294647pt}{315.947021pt}}
\pgflineto{\pgfpoint{313.049377pt}{316.370178pt}}
\pgfusepath{stroke}
\pgfpathmoveto{\pgfpoint{313.263306pt}{321.940796pt}}
\pgflineto{\pgfpoint{313.049377pt}{322.357971pt}}
\pgfusepath{stroke}
\pgfpathmoveto{\pgfpoint{313.235382pt}{327.938660pt}}
\pgflineto{\pgfpoint{313.049377pt}{328.345703pt}}
\pgfusepath{stroke}
\pgfpathmoveto{\pgfpoint{313.211090pt}{333.939056pt}}
\pgflineto{\pgfpoint{313.049377pt}{334.333466pt}}
\pgfusepath{stroke}
\pgfpathmoveto{\pgfpoint{313.190277pt}{339.940796pt}}
\pgflineto{\pgfpoint{313.049377pt}{340.321228pt}}
\pgfusepath{stroke}
\pgfpathmoveto{\pgfpoint{313.172577pt}{345.943024pt}}
\pgflineto{\pgfpoint{313.049377pt}{346.308960pt}}
\pgfusepath{stroke}
\pgfpathmoveto{\pgfpoint{313.157532pt}{351.945282pt}}
\pgflineto{\pgfpoint{313.049377pt}{352.296722pt}}
\pgfusepath{stroke}
\pgfpathmoveto{\pgfpoint{313.144745pt}{357.947205pt}}
\pgflineto{\pgfpoint{313.049377pt}{358.284485pt}}
\pgfusepath{stroke}
\pgfpathmoveto{\pgfpoint{313.133850pt}{363.948547pt}}
\pgflineto{\pgfpoint{313.049377pt}{364.272247pt}}
\pgfusepath{stroke}
\pgfpathmoveto{\pgfpoint{313.124573pt}{369.949280pt}}
\pgflineto{\pgfpoint{313.049377pt}{370.260010pt}}
\pgfusepath{stroke}
\pgfpathmoveto{\pgfpoint{319.035400pt}{77.083145pt}}
\pgflineto{\pgfpoint{319.037140pt}{76.859985pt}}
\pgfusepath{stroke}
\pgfpathmoveto{\pgfpoint{319.034790pt}{83.078827pt}}
\pgflineto{\pgfpoint{319.037140pt}{82.847733pt}}
\pgfusepath{stroke}
\pgfpathmoveto{\pgfpoint{319.034088pt}{89.075081pt}}
\pgflineto{\pgfpoint{319.037140pt}{88.835495pt}}
\pgfusepath{stroke}
\pgfpathmoveto{\pgfpoint{319.033234pt}{95.071983pt}}
\pgflineto{\pgfpoint{319.037140pt}{94.823257pt}}
\pgfusepath{stroke}
\pgfpathmoveto{\pgfpoint{319.032166pt}{101.069611pt}}
\pgflineto{\pgfpoint{319.037140pt}{100.811012pt}}
\pgfusepath{stroke}
\pgfpathmoveto{\pgfpoint{319.030914pt}{107.068062pt}}
\pgflineto{\pgfpoint{319.037140pt}{106.798759pt}}
\pgfusepath{stroke}
\pgfpathmoveto{\pgfpoint{319.029358pt}{113.067429pt}}
\pgflineto{\pgfpoint{319.037140pt}{112.786522pt}}
\pgfusepath{stroke}
\pgfpathmoveto{\pgfpoint{319.027435pt}{119.067856pt}}
\pgflineto{\pgfpoint{319.037140pt}{118.774277pt}}
\pgfusepath{stroke}
\pgfpathmoveto{\pgfpoint{319.025055pt}{125.069458pt}}
\pgflineto{\pgfpoint{319.037140pt}{124.762024pt}}
\pgfusepath{stroke}
\pgfpathmoveto{\pgfpoint{319.022095pt}{131.072418pt}}
\pgflineto{\pgfpoint{319.037140pt}{130.749786pt}}
\pgfusepath{stroke}
\pgfpathmoveto{\pgfpoint{319.018372pt}{137.076950pt}}
\pgflineto{\pgfpoint{319.037140pt}{136.737534pt}}
\pgfusepath{stroke}
\pgfpathmoveto{\pgfpoint{319.013641pt}{143.083282pt}}
\pgflineto{\pgfpoint{319.037140pt}{142.725281pt}}
\pgfusepath{stroke}
\pgfpathmoveto{\pgfpoint{319.007568pt}{149.091675pt}}
\pgflineto{\pgfpoint{319.037140pt}{148.713058pt}}
\pgfusepath{stroke}
\pgfpathmoveto{\pgfpoint{318.999664pt}{155.102478pt}}
\pgflineto{\pgfpoint{319.037140pt}{154.700806pt}}
\pgfusepath{stroke}
\pgfpathmoveto{\pgfpoint{318.989288pt}{161.116058pt}}
\pgflineto{\pgfpoint{319.037140pt}{160.688568pt}}
\pgfusepath{stroke}
\pgfpathmoveto{\pgfpoint{318.975403pt}{167.132782pt}}
\pgflineto{\pgfpoint{319.037140pt}{166.676315pt}}
\pgfusepath{stroke}
\pgfpathmoveto{\pgfpoint{318.956635pt}{173.153076pt}}
\pgflineto{\pgfpoint{319.037140pt}{172.664078pt}}
\pgfusepath{stroke}
\pgfpathmoveto{\pgfpoint{318.930756pt}{179.177078pt}}
\pgflineto{\pgfpoint{319.037140pt}{178.651825pt}}
\pgfusepath{stroke}
\pgfpathmoveto{\pgfpoint{318.894501pt}{185.204483pt}}
\pgflineto{\pgfpoint{319.037140pt}{184.639587pt}}
\pgfusepath{stroke}
\pgfpathmoveto{\pgfpoint{318.842957pt}{191.233643pt}}
\pgflineto{\pgfpoint{319.037140pt}{190.627335pt}}
\pgfusepath{stroke}
\pgfpathmoveto{\pgfpoint{318.769165pt}{197.259766pt}}
\pgflineto{\pgfpoint{319.037140pt}{196.615097pt}}
\pgfusepath{stroke}
\pgfpathmoveto{\pgfpoint{318.664886pt}{203.271225pt}}
\pgflineto{\pgfpoint{319.037140pt}{202.602844pt}}
\pgfusepath{stroke}
\pgfpathmoveto{\pgfpoint{318.526031pt}{209.243515pt}}
\pgflineto{\pgfpoint{319.037140pt}{208.590607pt}}
\pgfusepath{stroke}
\pgfpathmoveto{\pgfpoint{318.370483pt}{215.137741pt}}
\pgflineto{\pgfpoint{319.037140pt}{214.578354pt}}
\pgfusepath{stroke}
\pgfpathmoveto{\pgfpoint{318.262360pt}{220.928177pt}}
\pgflineto{\pgfpoint{319.037140pt}{220.566116pt}}
\pgfusepath{stroke}
\pgfpathmoveto{\pgfpoint{318.284332pt}{226.664154pt}}
\pgflineto{\pgfpoint{319.037140pt}{226.553864pt}}
\pgfusepath{stroke}
\pgfpathmoveto{\pgfpoint{318.437164pt}{232.453262pt}}
\pgflineto{\pgfpoint{319.037140pt}{232.541626pt}}
\pgfusepath{stroke}
\pgfpathmoveto{\pgfpoint{318.639801pt}{238.344101pt}}
\pgflineto{\pgfpoint{319.037140pt}{238.529388pt}}
\pgfusepath{stroke}
\pgfpathmoveto{\pgfpoint{318.829376pt}{244.309387pt}}
\pgflineto{\pgfpoint{319.037140pt}{244.517136pt}}
\pgfusepath{stroke}
\pgfpathmoveto{\pgfpoint{318.988892pt}{250.307693pt}}
\pgflineto{\pgfpoint{319.037140pt}{250.504883pt}}
\pgfusepath{stroke}
\pgfpathmoveto{\pgfpoint{319.121979pt}{256.311035pt}}
\pgflineto{\pgfpoint{319.037140pt}{256.492645pt}}
\pgfusepath{stroke}
\pgfpathmoveto{\pgfpoint{319.234558pt}{262.303650pt}}
\pgflineto{\pgfpoint{319.037140pt}{262.480408pt}}
\pgfusepath{stroke}
\pgfpathmoveto{\pgfpoint{319.328247pt}{268.277222pt}}
\pgflineto{\pgfpoint{319.037140pt}{268.468170pt}}
\pgfusepath{stroke}
\pgfpathmoveto{\pgfpoint{319.399780pt}{274.229553pt}}
\pgflineto{\pgfpoint{319.037140pt}{274.455902pt}}
\pgfusepath{stroke}
\pgfpathmoveto{\pgfpoint{319.443604pt}{280.165070pt}}
\pgflineto{\pgfpoint{319.037140pt}{280.443665pt}}
\pgfusepath{stroke}
\pgfpathmoveto{\pgfpoint{319.456696pt}{286.093964pt}}
\pgflineto{\pgfpoint{319.037140pt}{286.431427pt}}
\pgfusepath{stroke}
\pgfpathmoveto{\pgfpoint{319.442017pt}{292.028137pt}}
\pgflineto{\pgfpoint{319.037140pt}{292.419189pt}}
\pgfusepath{stroke}
\pgfpathmoveto{\pgfpoint{319.407959pt}{297.976135pt}}
\pgflineto{\pgfpoint{319.037140pt}{298.406921pt}}
\pgfusepath{stroke}
\pgfpathmoveto{\pgfpoint{319.364441pt}{303.940796pt}}
\pgflineto{\pgfpoint{319.037140pt}{304.394714pt}}
\pgfusepath{stroke}
\pgfpathmoveto{\pgfpoint{319.319336pt}{309.920288pt}}
\pgflineto{\pgfpoint{319.037140pt}{310.382446pt}}
\pgfusepath{stroke}
\pgfpathmoveto{\pgfpoint{319.277374pt}{315.910950pt}}
\pgflineto{\pgfpoint{319.037140pt}{316.370178pt}}
\pgfusepath{stroke}
\pgfpathmoveto{\pgfpoint{319.240601pt}{321.908997pt}}
\pgflineto{\pgfpoint{319.037140pt}{322.357971pt}}
\pgfusepath{stroke}
\pgfpathmoveto{\pgfpoint{319.209442pt}{327.911377pt}}
\pgflineto{\pgfpoint{319.037140pt}{328.345703pt}}
\pgfusepath{stroke}
\pgfpathmoveto{\pgfpoint{319.183502pt}{333.915985pt}}
\pgflineto{\pgfpoint{319.037140pt}{334.333466pt}}
\pgfusepath{stroke}
\pgfpathmoveto{\pgfpoint{319.162048pt}{339.921417pt}}
\pgflineto{\pgfpoint{319.037140pt}{340.321228pt}}
\pgfusepath{stroke}
\pgfpathmoveto{\pgfpoint{319.144348pt}{345.926819pt}}
\pgflineto{\pgfpoint{319.037140pt}{346.308960pt}}
\pgfusepath{stroke}
\pgfpathmoveto{\pgfpoint{319.129730pt}{351.931671pt}}
\pgflineto{\pgfpoint{319.037140pt}{352.296722pt}}
\pgfusepath{stroke}
\pgfpathmoveto{\pgfpoint{319.117615pt}{357.935730pt}}
\pgflineto{\pgfpoint{319.037140pt}{358.284485pt}}
\pgfusepath{stroke}
\pgfpathmoveto{\pgfpoint{319.107513pt}{363.938904pt}}
\pgflineto{\pgfpoint{319.037140pt}{364.272247pt}}
\pgfusepath{stroke}
\pgfpathmoveto{\pgfpoint{319.099030pt}{369.941071pt}}
\pgflineto{\pgfpoint{319.037140pt}{370.260010pt}}
\pgfusepath{stroke}
\pgfpathmoveto{\pgfpoint{325.015564pt}{77.082352pt}}
\pgflineto{\pgfpoint{325.024902pt}{76.859985pt}}
\pgfusepath{stroke}
\pgfpathmoveto{\pgfpoint{325.014404pt}{83.077881pt}}
\pgflineto{\pgfpoint{325.024902pt}{82.847733pt}}
\pgfusepath{stroke}
\pgfpathmoveto{\pgfpoint{325.013123pt}{89.073975pt}}
\pgflineto{\pgfpoint{325.024902pt}{88.835495pt}}
\pgfusepath{stroke}
\pgfpathmoveto{\pgfpoint{325.011566pt}{95.070679pt}}
\pgflineto{\pgfpoint{325.024902pt}{94.823257pt}}
\pgfusepath{stroke}
\pgfpathmoveto{\pgfpoint{325.009796pt}{101.068069pt}}
\pgflineto{\pgfpoint{325.024902pt}{100.811012pt}}
\pgfusepath{stroke}
\pgfpathmoveto{\pgfpoint{325.007690pt}{107.066216pt}}
\pgflineto{\pgfpoint{325.024902pt}{106.798759pt}}
\pgfusepath{stroke}
\pgfpathmoveto{\pgfpoint{325.005188pt}{113.065216pt}}
\pgflineto{\pgfpoint{325.024902pt}{112.786522pt}}
\pgfusepath{stroke}
\pgfpathmoveto{\pgfpoint{325.002228pt}{119.065155pt}}
\pgflineto{\pgfpoint{325.024902pt}{118.774277pt}}
\pgfusepath{stroke}
\pgfpathmoveto{\pgfpoint{324.998657pt}{125.066170pt}}
\pgflineto{\pgfpoint{325.024902pt}{124.762024pt}}
\pgfusepath{stroke}
\pgfpathmoveto{\pgfpoint{324.994324pt}{131.068375pt}}
\pgflineto{\pgfpoint{325.024902pt}{130.749786pt}}
\pgfusepath{stroke}
\pgfpathmoveto{\pgfpoint{324.989044pt}{137.071899pt}}
\pgflineto{\pgfpoint{325.024902pt}{136.737534pt}}
\pgfusepath{stroke}
\pgfpathmoveto{\pgfpoint{324.982544pt}{143.076950pt}}
\pgflineto{\pgfpoint{325.024902pt}{142.725281pt}}
\pgfusepath{stroke}
\pgfpathmoveto{\pgfpoint{324.974487pt}{149.083664pt}}
\pgflineto{\pgfpoint{325.024902pt}{148.713058pt}}
\pgfusepath{stroke}
\pgfpathmoveto{\pgfpoint{324.964325pt}{155.092224pt}}
\pgflineto{\pgfpoint{325.024902pt}{154.700806pt}}
\pgfusepath{stroke}
\pgfpathmoveto{\pgfpoint{324.951447pt}{161.102722pt}}
\pgflineto{\pgfpoint{325.024902pt}{160.688568pt}}
\pgfusepath{stroke}
\pgfpathmoveto{\pgfpoint{324.934937pt}{167.115265pt}}
\pgflineto{\pgfpoint{325.024902pt}{166.676315pt}}
\pgfusepath{stroke}
\pgfpathmoveto{\pgfpoint{324.913483pt}{173.129715pt}}
\pgflineto{\pgfpoint{325.024902pt}{172.664078pt}}
\pgfusepath{stroke}
\pgfpathmoveto{\pgfpoint{324.885376pt}{179.145523pt}}
\pgflineto{\pgfpoint{325.024902pt}{178.651825pt}}
\pgfusepath{stroke}
\pgfpathmoveto{\pgfpoint{324.848328pt}{185.161423pt}}
\pgflineto{\pgfpoint{325.024902pt}{184.639587pt}}
\pgfusepath{stroke}
\pgfpathmoveto{\pgfpoint{324.799469pt}{191.174652pt}}
\pgflineto{\pgfpoint{325.024902pt}{190.627335pt}}
\pgfusepath{stroke}
\pgfpathmoveto{\pgfpoint{324.735901pt}{197.179840pt}}
\pgflineto{\pgfpoint{325.024902pt}{196.615097pt}}
\pgfusepath{stroke}
\pgfpathmoveto{\pgfpoint{324.656433pt}{203.167526pt}}
\pgflineto{\pgfpoint{325.024902pt}{202.602844pt}}
\pgfusepath{stroke}
\pgfpathmoveto{\pgfpoint{324.565918pt}{209.123657pt}}
\pgflineto{\pgfpoint{325.024902pt}{208.590607pt}}
\pgfusepath{stroke}
\pgfpathmoveto{\pgfpoint{324.481812pt}{215.033844pt}}
\pgflineto{\pgfpoint{325.024902pt}{214.578354pt}}
\pgfusepath{stroke}
\pgfpathmoveto{\pgfpoint{324.435883pt}{220.897095pt}}
\pgflineto{\pgfpoint{325.024902pt}{220.566116pt}}
\pgfusepath{stroke}
\pgfpathmoveto{\pgfpoint{324.457977pt}{226.739639pt}}
\pgflineto{\pgfpoint{325.024902pt}{226.553864pt}}
\pgfusepath{stroke}
\pgfpathmoveto{\pgfpoint{324.549316pt}{232.603531pt}}
\pgflineto{\pgfpoint{325.024902pt}{232.541626pt}}
\pgfusepath{stroke}
\pgfpathmoveto{\pgfpoint{324.682343pt}{238.514206pt}}
\pgflineto{\pgfpoint{325.024902pt}{238.529388pt}}
\pgfusepath{stroke}
\pgfpathmoveto{\pgfpoint{324.827698pt}{244.468887pt}}
\pgflineto{\pgfpoint{325.024902pt}{244.517136pt}}
\pgfusepath{stroke}
\pgfpathmoveto{\pgfpoint{324.970001pt}{250.450012pt}}
\pgflineto{\pgfpoint{325.024902pt}{250.504883pt}}
\pgfusepath{stroke}
\pgfpathmoveto{\pgfpoint{325.105621pt}{256.438354pt}}
\pgflineto{\pgfpoint{325.024902pt}{256.492645pt}}
\pgfusepath{stroke}
\pgfpathmoveto{\pgfpoint{325.234192pt}{262.417114pt}}
\pgflineto{\pgfpoint{325.024902pt}{262.480408pt}}
\pgfusepath{stroke}
\pgfpathmoveto{\pgfpoint{325.351929pt}{268.372833pt}}
\pgflineto{\pgfpoint{325.024902pt}{268.468170pt}}
\pgfusepath{stroke}
\pgfpathmoveto{\pgfpoint{325.448029pt}{274.298065pt}}
\pgflineto{\pgfpoint{325.024902pt}{274.455902pt}}
\pgfusepath{stroke}
\pgfpathmoveto{\pgfpoint{325.507019pt}{280.197205pt}}
\pgflineto{\pgfpoint{325.024902pt}{280.443665pt}}
\pgfusepath{stroke}
\pgfpathmoveto{\pgfpoint{325.518860pt}{286.088013pt}}
\pgflineto{\pgfpoint{325.024902pt}{286.431427pt}}
\pgfusepath{stroke}
\pgfpathmoveto{\pgfpoint{325.488098pt}{291.992889pt}}
\pgflineto{\pgfpoint{325.024902pt}{292.419189pt}}
\pgfusepath{stroke}
\pgfpathmoveto{\pgfpoint{325.431641pt}{297.925507pt}}
\pgflineto{\pgfpoint{325.024902pt}{298.406921pt}}
\pgfusepath{stroke}
\pgfpathmoveto{\pgfpoint{325.367249pt}{303.886780pt}}
\pgflineto{\pgfpoint{325.024902pt}{304.394714pt}}
\pgfusepath{stroke}
\pgfpathmoveto{\pgfpoint{325.306610pt}{309.870148pt}}
\pgflineto{\pgfpoint{325.024902pt}{310.382446pt}}
\pgfusepath{stroke}
\pgfpathmoveto{\pgfpoint{325.254639pt}{315.867645pt}}
\pgflineto{\pgfpoint{325.024902pt}{316.370178pt}}
\pgfusepath{stroke}
\pgfpathmoveto{\pgfpoint{325.212158pt}{321.872986pt}}
\pgflineto{\pgfpoint{325.024902pt}{322.357971pt}}
\pgfusepath{stroke}
\pgfpathmoveto{\pgfpoint{325.178253pt}{327.882019pt}}
\pgflineto{\pgfpoint{325.024902pt}{328.345703pt}}
\pgfusepath{stroke}
\pgfpathmoveto{\pgfpoint{325.151367pt}{333.892181pt}}
\pgflineto{\pgfpoint{325.024902pt}{334.333466pt}}
\pgfusepath{stroke}
\pgfpathmoveto{\pgfpoint{325.130066pt}{339.902161pt}}
\pgflineto{\pgfpoint{325.024902pt}{340.321228pt}}
\pgfusepath{stroke}
\pgfpathmoveto{\pgfpoint{325.113098pt}{345.911133pt}}
\pgflineto{\pgfpoint{325.024902pt}{346.308960pt}}
\pgfusepath{stroke}
\pgfpathmoveto{\pgfpoint{325.099487pt}{351.918884pt}}
\pgflineto{\pgfpoint{325.024902pt}{352.296722pt}}
\pgfusepath{stroke}
\pgfpathmoveto{\pgfpoint{325.088501pt}{357.925232pt}}
\pgflineto{\pgfpoint{325.024902pt}{358.284485pt}}
\pgfusepath{stroke}
\pgfpathmoveto{\pgfpoint{325.079529pt}{363.930176pt}}
\pgflineto{\pgfpoint{325.024902pt}{364.272247pt}}
\pgfusepath{stroke}
\pgfpathmoveto{\pgfpoint{325.072174pt}{369.933838pt}}
\pgflineto{\pgfpoint{325.024902pt}{370.260010pt}}
\pgfusepath{stroke}
\pgfpathmoveto{\pgfpoint{330.995880pt}{77.081055pt}}
\pgflineto{\pgfpoint{331.012634pt}{76.859985pt}}
\pgfusepath{stroke}
\pgfpathmoveto{\pgfpoint{330.994232pt}{83.076401pt}}
\pgflineto{\pgfpoint{331.012634pt}{82.847733pt}}
\pgfusepath{stroke}
\pgfpathmoveto{\pgfpoint{330.992340pt}{89.072258pt}}
\pgflineto{\pgfpoint{331.012634pt}{88.835495pt}}
\pgfusepath{stroke}
\pgfpathmoveto{\pgfpoint{330.990204pt}{95.068695pt}}
\pgflineto{\pgfpoint{331.012634pt}{94.823257pt}}
\pgfusepath{stroke}
\pgfpathmoveto{\pgfpoint{330.987732pt}{101.065765pt}}
\pgflineto{\pgfpoint{331.012634pt}{100.811012pt}}
\pgfusepath{stroke}
\pgfpathmoveto{\pgfpoint{330.984833pt}{107.063515pt}}
\pgflineto{\pgfpoint{331.012634pt}{106.798759pt}}
\pgfusepath{stroke}
\pgfpathmoveto{\pgfpoint{330.981506pt}{113.062035pt}}
\pgflineto{\pgfpoint{331.012634pt}{112.786522pt}}
\pgfusepath{stroke}
\pgfpathmoveto{\pgfpoint{330.977600pt}{119.061394pt}}
\pgflineto{\pgfpoint{331.012634pt}{118.774277pt}}
\pgfusepath{stroke}
\pgfpathmoveto{\pgfpoint{330.972961pt}{125.061684pt}}
\pgflineto{\pgfpoint{331.012634pt}{124.762024pt}}
\pgfusepath{stroke}
\pgfpathmoveto{\pgfpoint{330.967468pt}{131.062988pt}}
\pgflineto{\pgfpoint{331.012634pt}{130.749786pt}}
\pgfusepath{stroke}
\pgfpathmoveto{\pgfpoint{330.960938pt}{137.065369pt}}
\pgflineto{\pgfpoint{331.012634pt}{136.737534pt}}
\pgfusepath{stroke}
\pgfpathmoveto{\pgfpoint{330.953003pt}{143.068970pt}}
\pgflineto{\pgfpoint{331.012634pt}{142.725281pt}}
\pgfusepath{stroke}
\pgfpathmoveto{\pgfpoint{330.943420pt}{149.073822pt}}
\pgflineto{\pgfpoint{331.012634pt}{148.713058pt}}
\pgfusepath{stroke}
\pgfpathmoveto{\pgfpoint{330.931702pt}{155.079956pt}}
\pgflineto{\pgfpoint{331.012634pt}{154.700806pt}}
\pgfusepath{stroke}
\pgfpathmoveto{\pgfpoint{330.917236pt}{161.087357pt}}
\pgflineto{\pgfpoint{331.012634pt}{160.688568pt}}
\pgfusepath{stroke}
\pgfpathmoveto{\pgfpoint{330.899292pt}{167.095795pt}}
\pgflineto{\pgfpoint{331.012634pt}{166.676315pt}}
\pgfusepath{stroke}
\pgfpathmoveto{\pgfpoint{330.876892pt}{173.104858pt}}
\pgflineto{\pgfpoint{331.012634pt}{172.664078pt}}
\pgfusepath{stroke}
\pgfpathmoveto{\pgfpoint{330.848877pt}{179.113663pt}}
\pgflineto{\pgfpoint{331.012634pt}{178.651825pt}}
\pgfusepath{stroke}
\pgfpathmoveto{\pgfpoint{330.813934pt}{185.120621pt}}
\pgflineto{\pgfpoint{331.012634pt}{184.639587pt}}
\pgfusepath{stroke}
\pgfpathmoveto{\pgfpoint{330.770874pt}{191.123001pt}}
\pgflineto{\pgfpoint{331.012634pt}{190.627335pt}}
\pgfusepath{stroke}
\pgfpathmoveto{\pgfpoint{330.719391pt}{197.116501pt}}
\pgflineto{\pgfpoint{331.012634pt}{196.615097pt}}
\pgfusepath{stroke}
\pgfpathmoveto{\pgfpoint{330.661224pt}{203.095047pt}}
\pgflineto{\pgfpoint{331.012634pt}{202.602844pt}}
\pgfusepath{stroke}
\pgfpathmoveto{\pgfpoint{330.602478pt}{209.051773pt}}
\pgflineto{\pgfpoint{331.012634pt}{208.590607pt}}
\pgfusepath{stroke}
\pgfpathmoveto{\pgfpoint{330.555176pt}{214.982315pt}}
\pgflineto{\pgfpoint{331.012634pt}{214.578354pt}}
\pgfusepath{stroke}
\pgfpathmoveto{\pgfpoint{330.535553pt}{220.890274pt}}
\pgflineto{\pgfpoint{331.012634pt}{220.566116pt}}
\pgfusepath{stroke}
\pgfpathmoveto{\pgfpoint{330.556641pt}{226.790100pt}}
\pgflineto{\pgfpoint{331.012634pt}{226.553864pt}}
\pgfusepath{stroke}
\pgfpathmoveto{\pgfpoint{330.619934pt}{232.701416pt}}
\pgflineto{\pgfpoint{331.012634pt}{232.541626pt}}
\pgfusepath{stroke}
\pgfpathmoveto{\pgfpoint{330.715607pt}{238.638031pt}}
\pgflineto{\pgfpoint{331.012634pt}{238.529388pt}}
\pgfusepath{stroke}
\pgfpathmoveto{\pgfpoint{330.831024pt}{244.601990pt}}
\pgflineto{\pgfpoint{331.012634pt}{244.517136pt}}
\pgfusepath{stroke}
\pgfpathmoveto{\pgfpoint{330.958344pt}{250.585632pt}}
\pgflineto{\pgfpoint{331.012634pt}{250.504883pt}}
\pgfusepath{stroke}
\pgfpathmoveto{\pgfpoint{331.095642pt}{256.575653pt}}
\pgflineto{\pgfpoint{331.012634pt}{256.492645pt}}
\pgfusepath{stroke}
\pgfpathmoveto{\pgfpoint{331.243317pt}{262.554565pt}}
\pgflineto{\pgfpoint{331.012634pt}{262.480408pt}}
\pgfusepath{stroke}
\pgfpathmoveto{\pgfpoint{331.396271pt}{268.500916pt}}
\pgflineto{\pgfpoint{331.012634pt}{268.468170pt}}
\pgfusepath{stroke}
\pgfpathmoveto{\pgfpoint{331.534302pt}{274.395355pt}}
\pgflineto{\pgfpoint{331.012634pt}{274.455902pt}}
\pgfusepath{stroke}
\pgfpathmoveto{\pgfpoint{331.620544pt}{280.238525pt}}
\pgflineto{\pgfpoint{331.012634pt}{280.443665pt}}
\pgfusepath{stroke}
\pgfpathmoveto{\pgfpoint{331.625854pt}{286.066223pt}}
\pgflineto{\pgfpoint{331.012634pt}{286.431427pt}}
\pgfusepath{stroke}
\pgfpathmoveto{\pgfpoint{331.559998pt}{291.927429pt}}
\pgflineto{\pgfpoint{331.012634pt}{292.419189pt}}
\pgfusepath{stroke}
\pgfpathmoveto{\pgfpoint{331.461914pt}{297.844940pt}}
\pgflineto{\pgfpoint{331.012634pt}{298.406921pt}}
\pgfusepath{stroke}
\pgfpathmoveto{\pgfpoint{331.365173pt}{303.810608pt}}
\pgflineto{\pgfpoint{331.012634pt}{304.394714pt}}
\pgfusepath{stroke}
\pgfpathmoveto{\pgfpoint{331.284546pt}{309.806213pt}}
\pgflineto{\pgfpoint{331.012634pt}{310.382446pt}}
\pgfusepath{stroke}
\pgfpathmoveto{\pgfpoint{331.222107pt}{315.816864pt}}
\pgflineto{\pgfpoint{331.012634pt}{316.370178pt}}
\pgfusepath{stroke}
\pgfpathmoveto{\pgfpoint{331.175140pt}{321.833588pt}}
\pgflineto{\pgfpoint{331.012634pt}{322.357971pt}}
\pgfusepath{stroke}
\pgfpathmoveto{\pgfpoint{331.140045pt}{327.851624pt}}
\pgflineto{\pgfpoint{331.012634pt}{328.345703pt}}
\pgfusepath{stroke}
\pgfpathmoveto{\pgfpoint{331.113708pt}{333.868713pt}}
\pgflineto{\pgfpoint{331.012634pt}{334.333466pt}}
\pgfusepath{stroke}
\pgfpathmoveto{\pgfpoint{331.093781pt}{339.883850pt}}
\pgflineto{\pgfpoint{331.012634pt}{340.321228pt}}
\pgfusepath{stroke}
\pgfpathmoveto{\pgfpoint{331.078491pt}{345.896790pt}}
\pgflineto{\pgfpoint{331.012634pt}{346.308960pt}}
\pgfusepath{stroke}
\pgfpathmoveto{\pgfpoint{331.066650pt}{351.907532pt}}
\pgflineto{\pgfpoint{331.012634pt}{352.296722pt}}
\pgfusepath{stroke}
\pgfpathmoveto{\pgfpoint{331.057373pt}{357.916138pt}}
\pgflineto{\pgfpoint{331.012634pt}{358.284485pt}}
\pgfusepath{stroke}
\pgfpathmoveto{\pgfpoint{331.049988pt}{363.922821pt}}
\pgflineto{\pgfpoint{331.012634pt}{364.272247pt}}
\pgfusepath{stroke}
\pgfpathmoveto{\pgfpoint{331.044067pt}{369.927795pt}}
\pgflineto{\pgfpoint{331.012634pt}{370.260010pt}}
\pgfusepath{stroke}
\pgfpathmoveto{\pgfpoint{336.976379pt}{77.079269pt}}
\pgflineto{\pgfpoint{337.000397pt}{76.859985pt}}
\pgfusepath{stroke}
\pgfpathmoveto{\pgfpoint{336.974274pt}{83.074387pt}}
\pgflineto{\pgfpoint{337.000397pt}{82.847733pt}}
\pgfusepath{stroke}
\pgfpathmoveto{\pgfpoint{336.971863pt}{89.069969pt}}
\pgflineto{\pgfpoint{337.000397pt}{88.835495pt}}
\pgfusepath{stroke}
\pgfpathmoveto{\pgfpoint{336.969147pt}{95.066093pt}}
\pgflineto{\pgfpoint{337.000397pt}{94.823257pt}}
\pgfusepath{stroke}
\pgfpathmoveto{\pgfpoint{336.966034pt}{101.062767pt}}
\pgflineto{\pgfpoint{337.000397pt}{100.811012pt}}
\pgfusepath{stroke}
\pgfpathmoveto{\pgfpoint{336.962494pt}{107.060051pt}}
\pgflineto{\pgfpoint{337.000397pt}{106.798759pt}}
\pgfusepath{stroke}
\pgfpathmoveto{\pgfpoint{336.958374pt}{113.058022pt}}
\pgflineto{\pgfpoint{337.000397pt}{112.786522pt}}
\pgfusepath{stroke}
\pgfpathmoveto{\pgfpoint{336.953644pt}{119.056709pt}}
\pgflineto{\pgfpoint{337.000397pt}{118.774277pt}}
\pgfusepath{stroke}
\pgfpathmoveto{\pgfpoint{336.948151pt}{125.056175pt}}
\pgflineto{\pgfpoint{337.000397pt}{124.762024pt}}
\pgfusepath{stroke}
\pgfpathmoveto{\pgfpoint{336.941711pt}{131.056488pt}}
\pgflineto{\pgfpoint{337.000397pt}{130.749786pt}}
\pgfusepath{stroke}
\pgfpathmoveto{\pgfpoint{336.934113pt}{137.057648pt}}
\pgflineto{\pgfpoint{337.000397pt}{136.737534pt}}
\pgfusepath{stroke}
\pgfpathmoveto{\pgfpoint{336.925171pt}{143.059723pt}}
\pgflineto{\pgfpoint{337.000397pt}{142.725281pt}}
\pgfusepath{stroke}
\pgfpathmoveto{\pgfpoint{336.914490pt}{149.062683pt}}
\pgflineto{\pgfpoint{337.000397pt}{148.713058pt}}
\pgfusepath{stroke}
\pgfpathmoveto{\pgfpoint{336.901764pt}{155.066467pt}}
\pgflineto{\pgfpoint{337.000397pt}{154.700806pt}}
\pgfusepath{stroke}
\pgfpathmoveto{\pgfpoint{336.886475pt}{161.070892pt}}
\pgflineto{\pgfpoint{337.000397pt}{160.688568pt}}
\pgfusepath{stroke}
\pgfpathmoveto{\pgfpoint{336.868042pt}{167.075638pt}}
\pgflineto{\pgfpoint{337.000397pt}{166.676315pt}}
\pgfusepath{stroke}
\pgfpathmoveto{\pgfpoint{336.845886pt}{173.080139pt}}
\pgflineto{\pgfpoint{337.000397pt}{172.664078pt}}
\pgfusepath{stroke}
\pgfpathmoveto{\pgfpoint{336.819275pt}{179.083466pt}}
\pgflineto{\pgfpoint{337.000397pt}{178.651825pt}}
\pgfusepath{stroke}
\pgfpathmoveto{\pgfpoint{336.787750pt}{185.084122pt}}
\pgflineto{\pgfpoint{337.000397pt}{184.639587pt}}
\pgfusepath{stroke}
\pgfpathmoveto{\pgfpoint{336.751129pt}{191.079910pt}}
\pgflineto{\pgfpoint{337.000397pt}{190.627335pt}}
\pgfusepath{stroke}
\pgfpathmoveto{\pgfpoint{336.710236pt}{197.067932pt}}
\pgflineto{\pgfpoint{337.000397pt}{196.615097pt}}
\pgfusepath{stroke}
\pgfpathmoveto{\pgfpoint{336.667664pt}{203.044739pt}}
\pgflineto{\pgfpoint{337.000397pt}{202.602844pt}}
\pgfusepath{stroke}
\pgfpathmoveto{\pgfpoint{336.628448pt}{209.007431pt}}
\pgflineto{\pgfpoint{337.000397pt}{208.590607pt}}
\pgfusepath{stroke}
\pgfpathmoveto{\pgfpoint{336.600372pt}{214.955460pt}}
\pgflineto{\pgfpoint{337.000397pt}{214.578354pt}}
\pgfusepath{stroke}
\pgfpathmoveto{\pgfpoint{336.592285pt}{220.892715pt}}
\pgflineto{\pgfpoint{337.000397pt}{220.566116pt}}
\pgfusepath{stroke}
\pgfpathmoveto{\pgfpoint{336.610840pt}{226.827866pt}}
\pgflineto{\pgfpoint{337.000397pt}{226.553864pt}}
\pgfusepath{stroke}
\pgfpathmoveto{\pgfpoint{336.657806pt}{232.771744pt}}
\pgflineto{\pgfpoint{337.000397pt}{232.541626pt}}
\pgfusepath{stroke}
\pgfpathmoveto{\pgfpoint{336.730103pt}{238.732895pt}}
\pgflineto{\pgfpoint{337.000397pt}{238.529388pt}}
\pgfusepath{stroke}
\pgfpathmoveto{\pgfpoint{336.823639pt}{244.714554pt}}
\pgflineto{\pgfpoint{337.000397pt}{244.517136pt}}
\pgfusepath{stroke}
\pgfpathmoveto{\pgfpoint{336.937103pt}{250.714203pt}}
\pgflineto{\pgfpoint{337.000397pt}{250.504883pt}}
\pgfusepath{stroke}
\pgfpathmoveto{\pgfpoint{337.074554pt}{256.723328pt}}
\pgflineto{\pgfpoint{337.000397pt}{256.492645pt}}
\pgfusepath{stroke}
\pgfpathmoveto{\pgfpoint{337.244476pt}{262.724487pt}}
\pgflineto{\pgfpoint{337.000397pt}{262.480408pt}}
\pgfusepath{stroke}
\pgfpathmoveto{\pgfpoint{337.451843pt}{268.684143pt}}
\pgflineto{\pgfpoint{337.000397pt}{268.468170pt}}
\pgfusepath{stroke}
\pgfpathmoveto{\pgfpoint{337.672974pt}{274.551147pt}}
\pgflineto{\pgfpoint{337.000397pt}{274.455902pt}}
\pgfusepath{stroke}
\pgfpathmoveto{\pgfpoint{337.823273pt}{280.298462pt}}
\pgflineto{\pgfpoint{337.000397pt}{280.443665pt}}
\pgfusepath{stroke}
\pgfpathmoveto{\pgfpoint{337.809784pt}{286.004883pt}}
\pgflineto{\pgfpoint{337.000397pt}{286.431427pt}}
\pgfusepath{stroke}
\pgfpathmoveto{\pgfpoint{337.662109pt}{291.799866pt}}
\pgflineto{\pgfpoint{337.000397pt}{292.419189pt}}
\pgfusepath{stroke}
\pgfpathmoveto{\pgfpoint{337.487183pt}{297.714935pt}}
\pgflineto{\pgfpoint{337.000397pt}{298.406921pt}}
\pgfusepath{stroke}
\pgfpathmoveto{\pgfpoint{337.344940pt}{303.705383pt}}
\pgflineto{\pgfpoint{337.000397pt}{304.394714pt}}
\pgfusepath{stroke}
\pgfpathmoveto{\pgfpoint{337.243561pt}{309.728241pt}}
\pgflineto{\pgfpoint{337.000397pt}{310.382446pt}}
\pgfusepath{stroke}
\pgfpathmoveto{\pgfpoint{337.174011pt}{315.760742pt}}
\pgflineto{\pgfpoint{337.000397pt}{316.370178pt}}
\pgfusepath{stroke}
\pgfpathmoveto{\pgfpoint{337.126343pt}{321.793274pt}}
\pgflineto{\pgfpoint{337.000397pt}{322.357971pt}}
\pgfusepath{stroke}
\pgfpathmoveto{\pgfpoint{337.093231pt}{327.822388pt}}
\pgflineto{\pgfpoint{337.000397pt}{328.345703pt}}
\pgfusepath{stroke}
\pgfpathmoveto{\pgfpoint{337.069794pt}{333.847229pt}}
\pgflineto{\pgfpoint{337.000397pt}{334.333466pt}}
\pgfusepath{stroke}
\pgfpathmoveto{\pgfpoint{337.052917pt}{339.867859pt}}
\pgflineto{\pgfpoint{337.000397pt}{340.321228pt}}
\pgfusepath{stroke}
\pgfpathmoveto{\pgfpoint{337.040527pt}{345.884705pt}}
\pgflineto{\pgfpoint{337.000397pt}{346.308960pt}}
\pgfusepath{stroke}
\pgfpathmoveto{\pgfpoint{337.031311pt}{351.898254pt}}
\pgflineto{\pgfpoint{337.000397pt}{352.296722pt}}
\pgfusepath{stroke}
\pgfpathmoveto{\pgfpoint{337.024323pt}{357.908936pt}}
\pgflineto{\pgfpoint{337.000397pt}{358.284485pt}}
\pgfusepath{stroke}
\pgfpathmoveto{\pgfpoint{337.018982pt}{363.917175pt}}
\pgflineto{\pgfpoint{337.000397pt}{364.272247pt}}
\pgfusepath{stroke}
\pgfpathmoveto{\pgfpoint{337.014862pt}{369.923340pt}}
\pgflineto{\pgfpoint{337.000397pt}{370.260010pt}}
\pgfusepath{stroke}
\pgfpathmoveto{\pgfpoint{342.957153pt}{77.077057pt}}
\pgflineto{\pgfpoint{342.988159pt}{76.859985pt}}
\pgfusepath{stroke}
\pgfpathmoveto{\pgfpoint{342.954590pt}{83.071899pt}}
\pgflineto{\pgfpoint{342.988159pt}{82.847733pt}}
\pgfusepath{stroke}
\pgfpathmoveto{\pgfpoint{342.951691pt}{89.067177pt}}
\pgflineto{\pgfpoint{342.988159pt}{88.835495pt}}
\pgfusepath{stroke}
\pgfpathmoveto{\pgfpoint{342.948456pt}{95.062920pt}}
\pgflineto{\pgfpoint{342.988159pt}{94.823257pt}}
\pgfusepath{stroke}
\pgfpathmoveto{\pgfpoint{342.944794pt}{101.059143pt}}
\pgflineto{\pgfpoint{342.988159pt}{100.811012pt}}
\pgfusepath{stroke}
\pgfpathmoveto{\pgfpoint{342.940643pt}{107.055931pt}}
\pgflineto{\pgfpoint{342.988159pt}{106.798759pt}}
\pgfusepath{stroke}
\pgfpathmoveto{\pgfpoint{342.935913pt}{113.053284pt}}
\pgflineto{\pgfpoint{342.988159pt}{112.786522pt}}
\pgfusepath{stroke}
\pgfpathmoveto{\pgfpoint{342.930481pt}{119.051247pt}}
\pgflineto{\pgfpoint{342.988159pt}{118.774277pt}}
\pgfusepath{stroke}
\pgfpathmoveto{\pgfpoint{342.924255pt}{125.049843pt}}
\pgflineto{\pgfpoint{342.988159pt}{124.762024pt}}
\pgfusepath{stroke}
\pgfpathmoveto{\pgfpoint{342.917053pt}{131.049103pt}}
\pgflineto{\pgfpoint{342.988159pt}{130.749786pt}}
\pgfusepath{stroke}
\pgfpathmoveto{\pgfpoint{342.908691pt}{137.049042pt}}
\pgflineto{\pgfpoint{342.988159pt}{136.737534pt}}
\pgfusepath{stroke}
\pgfpathmoveto{\pgfpoint{342.898987pt}{143.049591pt}}
\pgflineto{\pgfpoint{342.988159pt}{142.725281pt}}
\pgfusepath{stroke}
\pgfpathmoveto{\pgfpoint{342.887634pt}{149.050720pt}}
\pgflineto{\pgfpoint{342.988159pt}{148.713058pt}}
\pgfusepath{stroke}
\pgfpathmoveto{\pgfpoint{342.874359pt}{155.052277pt}}
\pgflineto{\pgfpoint{342.988159pt}{154.700806pt}}
\pgfusepath{stroke}
\pgfpathmoveto{\pgfpoint{342.858795pt}{161.054047pt}}
\pgflineto{\pgfpoint{342.988159pt}{160.688568pt}}
\pgfusepath{stroke}
\pgfpathmoveto{\pgfpoint{342.840576pt}{167.055664pt}}
\pgflineto{\pgfpoint{342.988159pt}{166.676315pt}}
\pgfusepath{stroke}
\pgfpathmoveto{\pgfpoint{342.819305pt}{173.056534pt}}
\pgflineto{\pgfpoint{342.988159pt}{172.664078pt}}
\pgfusepath{stroke}
\pgfpathmoveto{\pgfpoint{342.794739pt}{179.055817pt}}
\pgflineto{\pgfpoint{342.988159pt}{178.651825pt}}
\pgfusepath{stroke}
\pgfpathmoveto{\pgfpoint{342.766785pt}{185.052322pt}}
\pgflineto{\pgfpoint{342.988159pt}{184.639587pt}}
\pgfusepath{stroke}
\pgfpathmoveto{\pgfpoint{342.735901pt}{191.044525pt}}
\pgflineto{\pgfpoint{342.988159pt}{190.627335pt}}
\pgfusepath{stroke}
\pgfpathmoveto{\pgfpoint{342.703247pt}{197.030609pt}}
\pgflineto{\pgfpoint{342.988159pt}{196.615097pt}}
\pgfusepath{stroke}
\pgfpathmoveto{\pgfpoint{342.671204pt}{203.008911pt}}
\pgflineto{\pgfpoint{342.988159pt}{202.602844pt}}
\pgfusepath{stroke}
\pgfpathmoveto{\pgfpoint{342.643555pt}{208.978516pt}}
\pgflineto{\pgfpoint{342.988159pt}{208.590607pt}}
\pgfusepath{stroke}
\pgfpathmoveto{\pgfpoint{342.625244pt}{214.940308pt}}
\pgflineto{\pgfpoint{342.988159pt}{214.578354pt}}
\pgfusepath{stroke}
\pgfpathmoveto{\pgfpoint{342.621368pt}{220.897659pt}}
\pgflineto{\pgfpoint{342.988159pt}{220.566116pt}}
\pgfusepath{stroke}
\pgfpathmoveto{\pgfpoint{342.635803pt}{226.856476pt}}
\pgflineto{\pgfpoint{342.988159pt}{226.553864pt}}
\pgfusepath{stroke}
\pgfpathmoveto{\pgfpoint{342.670105pt}{232.823837pt}}
\pgflineto{\pgfpoint{342.988159pt}{232.541626pt}}
\pgfusepath{stroke}
\pgfpathmoveto{\pgfpoint{342.723969pt}{238.806244pt}}
\pgflineto{\pgfpoint{342.988159pt}{238.529388pt}}
\pgfusepath{stroke}
\pgfpathmoveto{\pgfpoint{342.797211pt}{244.808243pt}}
\pgflineto{\pgfpoint{342.988159pt}{244.517136pt}}
\pgfusepath{stroke}
\pgfpathmoveto{\pgfpoint{342.892822pt}{250.831940pt}}
\pgflineto{\pgfpoint{342.988159pt}{250.504883pt}}
\pgfusepath{stroke}
\pgfpathmoveto{\pgfpoint{343.020935pt}{256.876251pt}}
\pgflineto{\pgfpoint{342.988159pt}{256.492645pt}}
\pgfusepath{stroke}
\pgfpathmoveto{\pgfpoint{343.204132pt}{262.931824pt}}
\pgflineto{\pgfpoint{342.988159pt}{262.480408pt}}
\pgfusepath{stroke}
\pgfpathmoveto{\pgfpoint{343.481323pt}{268.961334pt}}
\pgflineto{\pgfpoint{342.988159pt}{268.468170pt}}
\pgfusepath{stroke}
\pgfpathmoveto{\pgfpoint{343.875366pt}{274.848877pt}}
\pgflineto{\pgfpoint{342.988159pt}{274.455902pt}}
\pgfusepath{stroke}
\pgfpathmoveto{\pgfpoint{344.219482pt}{280.413940pt}}
\pgflineto{\pgfpoint{342.988159pt}{280.443665pt}}
\pgfusepath{stroke}
\pgfpathmoveto{\pgfpoint{344.142792pt}{285.828796pt}}
\pgflineto{\pgfpoint{342.988159pt}{286.431427pt}}
\pgfusepath{stroke}
\pgfpathmoveto{\pgfpoint{343.777985pt}{291.539551pt}}
\pgflineto{\pgfpoint{342.988159pt}{292.419189pt}}
\pgfusepath{stroke}
\pgfpathmoveto{\pgfpoint{343.470886pt}{297.512207pt}}
\pgflineto{\pgfpoint{342.988159pt}{298.406921pt}}
\pgfusepath{stroke}
\pgfpathmoveto{\pgfpoint{343.281769pt}{303.571625pt}}
\pgflineto{\pgfpoint{342.988159pt}{304.394714pt}}
\pgfusepath{stroke}
\pgfpathmoveto{\pgfpoint{343.170929pt}{309.642761pt}}
\pgflineto{\pgfpoint{342.988159pt}{310.382446pt}}
\pgfusepath{stroke}
\pgfpathmoveto{\pgfpoint{343.104584pt}{315.705475pt}}
\pgflineto{\pgfpoint{342.988159pt}{316.370178pt}}
\pgfusepath{stroke}
\pgfpathmoveto{\pgfpoint{343.063416pt}{321.756714pt}}
\pgflineto{\pgfpoint{342.988159pt}{322.357971pt}}
\pgfusepath{stroke}
\pgfpathmoveto{\pgfpoint{343.036987pt}{327.797607pt}}
\pgflineto{\pgfpoint{342.988159pt}{328.345703pt}}
\pgfusepath{stroke}
\pgfpathmoveto{\pgfpoint{343.019501pt}{333.830048pt}}
\pgflineto{\pgfpoint{342.988159pt}{334.333466pt}}
\pgfusepath{stroke}
\pgfpathmoveto{\pgfpoint{343.007660pt}{339.855682pt}}
\pgflineto{\pgfpoint{342.988159pt}{340.321228pt}}
\pgfusepath{stroke}
\pgfpathmoveto{\pgfpoint{342.999481pt}{345.875916pt}}
\pgflineto{\pgfpoint{342.988159pt}{346.308960pt}}
\pgfusepath{stroke}
\pgfpathmoveto{\pgfpoint{342.993744pt}{351.891785pt}}
\pgflineto{\pgfpoint{342.988159pt}{352.296722pt}}
\pgfusepath{stroke}
\pgfpathmoveto{\pgfpoint{342.989685pt}{357.904144pt}}
\pgflineto{\pgfpoint{342.988159pt}{358.284485pt}}
\pgfusepath{stroke}
\pgfpathmoveto{\pgfpoint{342.986816pt}{363.913574pt}}
\pgflineto{\pgfpoint{342.988159pt}{364.272247pt}}
\pgfusepath{stroke}
\pgfpathmoveto{\pgfpoint{342.984772pt}{369.920624pt}}
\pgflineto{\pgfpoint{342.988159pt}{370.260010pt}}
\pgfusepath{stroke}
\pgfpathmoveto{\pgfpoint{348.938171pt}{77.074448pt}}
\pgflineto{\pgfpoint{348.975922pt}{76.859985pt}}
\pgfusepath{stroke}
\pgfpathmoveto{\pgfpoint{348.935211pt}{83.068985pt}}
\pgflineto{\pgfpoint{348.975922pt}{82.847733pt}}
\pgfusepath{stroke}
\pgfpathmoveto{\pgfpoint{348.931885pt}{89.063904pt}}
\pgflineto{\pgfpoint{348.975922pt}{88.835495pt}}
\pgfusepath{stroke}
\pgfpathmoveto{\pgfpoint{348.928223pt}{95.059235pt}}
\pgflineto{\pgfpoint{348.975922pt}{94.823257pt}}
\pgfusepath{stroke}
\pgfpathmoveto{\pgfpoint{348.924072pt}{101.054985pt}}
\pgflineto{\pgfpoint{348.975922pt}{100.811012pt}}
\pgfusepath{stroke}
\pgfpathmoveto{\pgfpoint{348.919403pt}{107.051216pt}}
\pgflineto{\pgfpoint{348.975922pt}{106.798759pt}}
\pgfusepath{stroke}
\pgfpathmoveto{\pgfpoint{348.914124pt}{113.047935pt}}
\pgflineto{\pgfpoint{348.975922pt}{112.786522pt}}
\pgfusepath{stroke}
\pgfpathmoveto{\pgfpoint{348.908112pt}{119.045143pt}}
\pgflineto{\pgfpoint{348.975922pt}{118.774277pt}}
\pgfusepath{stroke}
\pgfpathmoveto{\pgfpoint{348.901306pt}{125.042862pt}}
\pgflineto{\pgfpoint{348.975922pt}{124.762024pt}}
\pgfusepath{stroke}
\pgfpathmoveto{\pgfpoint{348.893555pt}{131.041077pt}}
\pgflineto{\pgfpoint{348.975922pt}{130.749786pt}}
\pgfusepath{stroke}
\pgfpathmoveto{\pgfpoint{348.884644pt}{137.039764pt}}
\pgflineto{\pgfpoint{348.975922pt}{136.737534pt}}
\pgfusepath{stroke}
\pgfpathmoveto{\pgfpoint{348.874451pt}{143.038849pt}}
\pgflineto{\pgfpoint{348.975922pt}{142.725281pt}}
\pgfusepath{stroke}
\pgfpathmoveto{\pgfpoint{348.862732pt}{149.038284pt}}
\pgflineto{\pgfpoint{348.975922pt}{148.713058pt}}
\pgfusepath{stroke}
\pgfpathmoveto{\pgfpoint{348.849243pt}{155.037857pt}}
\pgflineto{\pgfpoint{348.975922pt}{154.700806pt}}
\pgfusepath{stroke}
\pgfpathmoveto{\pgfpoint{348.833801pt}{161.037354pt}}
\pgflineto{\pgfpoint{348.975922pt}{160.688568pt}}
\pgfusepath{stroke}
\pgfpathmoveto{\pgfpoint{348.816132pt}{167.036377pt}}
\pgflineto{\pgfpoint{348.975922pt}{166.676315pt}}
\pgfusepath{stroke}
\pgfpathmoveto{\pgfpoint{348.796082pt}{173.034454pt}}
\pgflineto{\pgfpoint{348.975922pt}{172.664078pt}}
\pgfusepath{stroke}
\pgfpathmoveto{\pgfpoint{348.773590pt}{179.030884pt}}
\pgflineto{\pgfpoint{348.975922pt}{178.651825pt}}
\pgfusepath{stroke}
\pgfpathmoveto{\pgfpoint{348.748901pt}{185.024796pt}}
\pgflineto{\pgfpoint{348.975922pt}{184.639587pt}}
\pgfusepath{stroke}
\pgfpathmoveto{\pgfpoint{348.722595pt}{191.015244pt}}
\pgflineto{\pgfpoint{348.975922pt}{190.627335pt}}
\pgfusepath{stroke}
\pgfpathmoveto{\pgfpoint{348.695862pt}{197.001221pt}}
\pgflineto{\pgfpoint{348.975922pt}{196.615097pt}}
\pgfusepath{stroke}
\pgfpathmoveto{\pgfpoint{348.670593pt}{202.982071pt}}
\pgflineto{\pgfpoint{348.975922pt}{202.602844pt}}
\pgfusepath{stroke}
\pgfpathmoveto{\pgfpoint{348.649414pt}{208.957886pt}}
\pgflineto{\pgfpoint{348.975922pt}{208.590607pt}}
\pgfusepath{stroke}
\pgfpathmoveto{\pgfpoint{348.635468pt}{214.930023pt}}
\pgflineto{\pgfpoint{348.975922pt}{214.578354pt}}
\pgfusepath{stroke}
\pgfpathmoveto{\pgfpoint{348.631744pt}{220.901367pt}}
\pgflineto{\pgfpoint{348.975922pt}{220.566116pt}}
\pgfusepath{stroke}
\pgfpathmoveto{\pgfpoint{348.640533pt}{226.876389pt}}
\pgflineto{\pgfpoint{348.975922pt}{226.553864pt}}
\pgfusepath{stroke}
\pgfpathmoveto{\pgfpoint{348.662842pt}{232.860535pt}}
\pgflineto{\pgfpoint{348.975922pt}{232.541626pt}}
\pgfusepath{stroke}
\pgfpathmoveto{\pgfpoint{348.698975pt}{238.859619pt}}
\pgflineto{\pgfpoint{348.975922pt}{238.529388pt}}
\pgfusepath{stroke}
\pgfpathmoveto{\pgfpoint{348.749573pt}{244.879761pt}}
\pgflineto{\pgfpoint{348.975922pt}{244.517136pt}}
\pgfusepath{stroke}
\pgfpathmoveto{\pgfpoint{348.818054pt}{250.928040pt}}
\pgflineto{\pgfpoint{348.975922pt}{250.504883pt}}
\pgfusepath{stroke}
\pgfpathmoveto{\pgfpoint{348.915344pt}{257.014282pt}}
\pgflineto{\pgfpoint{348.975922pt}{256.492645pt}}
\pgfusepath{stroke}
\pgfpathmoveto{\pgfpoint{349.071136pt}{263.152985pt}}
\pgflineto{\pgfpoint{348.975922pt}{262.480408pt}}
\pgfusepath{stroke}
\pgfpathmoveto{\pgfpoint{349.368866pt}{269.355377pt}}
\pgflineto{\pgfpoint{348.975922pt}{268.468170pt}}
\pgfusepath{stroke}
\pgfpathmoveto{\pgfpoint{350.041931pt}{275.521912pt}}
\pgflineto{\pgfpoint{348.975922pt}{274.455902pt}}
\pgfusepath{stroke}
\pgfpathmoveto{\pgfpoint{351.194458pt}{280.803833pt}}
\pgflineto{\pgfpoint{348.975922pt}{280.443665pt}}
\pgfusepath{stroke}
\pgfpathmoveto{\pgfpoint{350.774719pt}{285.192505pt}}
\pgflineto{\pgfpoint{348.975922pt}{286.431427pt}}
\pgfusepath{stroke}
\pgfpathmoveto{\pgfpoint{349.768646pt}{291.020660pt}}
\pgflineto{\pgfpoint{348.975922pt}{292.419189pt}}
\pgfusepath{stroke}
\pgfpathmoveto{\pgfpoint{349.327881pt}{297.244141pt}}
\pgflineto{\pgfpoint{348.975922pt}{298.406921pt}}
\pgfusepath{stroke}
\pgfpathmoveto{\pgfpoint{349.143188pt}{303.433960pt}}
\pgflineto{\pgfpoint{348.975922pt}{304.394714pt}}
\pgfusepath{stroke}
\pgfpathmoveto{\pgfpoint{349.056335pt}{309.567810pt}}
\pgflineto{\pgfpoint{348.975922pt}{310.382446pt}}
\pgfusepath{stroke}
\pgfpathmoveto{\pgfpoint{349.011353pt}{315.662231pt}}
\pgflineto{\pgfpoint{348.975922pt}{316.370178pt}}
\pgfusepath{stroke}
\pgfpathmoveto{\pgfpoint{348.986511pt}{321.730591pt}}
\pgflineto{\pgfpoint{348.975922pt}{322.357971pt}}
\pgfusepath{stroke}
\pgfpathmoveto{\pgfpoint{348.972198pt}{327.781250pt}}
\pgflineto{\pgfpoint{348.975922pt}{328.345703pt}}
\pgfusepath{stroke}
\pgfpathmoveto{\pgfpoint{348.963776pt}{333.819550pt}}
\pgflineto{\pgfpoint{348.975922pt}{334.333466pt}}
\pgfusepath{stroke}
\pgfpathmoveto{\pgfpoint{348.958862pt}{339.848846pt}}
\pgflineto{\pgfpoint{348.975922pt}{340.321228pt}}
\pgfusepath{stroke}
\pgfpathmoveto{\pgfpoint{348.956055pt}{345.871429pt}}
\pgflineto{\pgfpoint{348.975922pt}{346.308960pt}}
\pgfusepath{stroke}
\pgfpathmoveto{\pgfpoint{348.954559pt}{351.888855pt}}
\pgflineto{\pgfpoint{348.975922pt}{352.296722pt}}
\pgfusepath{stroke}
\pgfpathmoveto{\pgfpoint{348.953918pt}{357.902252pt}}
\pgflineto{\pgfpoint{348.975922pt}{358.284485pt}}
\pgfusepath{stroke}
\pgfpathmoveto{\pgfpoint{348.953827pt}{363.912415pt}}
\pgflineto{\pgfpoint{348.975922pt}{364.272247pt}}
\pgfusepath{stroke}
\pgfpathmoveto{\pgfpoint{348.954071pt}{369.919922pt}}
\pgflineto{\pgfpoint{348.975922pt}{370.260010pt}}
\pgfusepath{stroke}
\pgfpathmoveto{\pgfpoint{354.919495pt}{77.071472pt}}
\pgflineto{\pgfpoint{354.963684pt}{76.859985pt}}
\pgfusepath{stroke}
\pgfpathmoveto{\pgfpoint{354.916168pt}{83.065659pt}}
\pgflineto{\pgfpoint{354.963684pt}{82.847733pt}}
\pgfusepath{stroke}
\pgfpathmoveto{\pgfpoint{354.912476pt}{89.060211pt}}
\pgflineto{\pgfpoint{354.963684pt}{88.835495pt}}
\pgfusepath{stroke}
\pgfpathmoveto{\pgfpoint{354.908386pt}{95.055115pt}}
\pgflineto{\pgfpoint{354.963684pt}{94.823257pt}}
\pgfusepath{stroke}
\pgfpathmoveto{\pgfpoint{354.903839pt}{101.050392pt}}
\pgflineto{\pgfpoint{354.963684pt}{100.811012pt}}
\pgfusepath{stroke}
\pgfpathmoveto{\pgfpoint{354.898743pt}{107.046043pt}}
\pgflineto{\pgfpoint{354.963684pt}{106.798759pt}}
\pgfusepath{stroke}
\pgfpathmoveto{\pgfpoint{354.893005pt}{113.042084pt}}
\pgflineto{\pgfpoint{354.963684pt}{112.786522pt}}
\pgfusepath{stroke}
\pgfpathmoveto{\pgfpoint{354.886597pt}{119.038536pt}}
\pgflineto{\pgfpoint{354.963684pt}{118.774277pt}}
\pgfusepath{stroke}
\pgfpathmoveto{\pgfpoint{354.879333pt}{125.035362pt}}
\pgflineto{\pgfpoint{354.963684pt}{124.762024pt}}
\pgfusepath{stroke}
\pgfpathmoveto{\pgfpoint{354.871155pt}{131.032562pt}}
\pgflineto{\pgfpoint{354.963684pt}{130.749786pt}}
\pgfusepath{stroke}
\pgfpathmoveto{\pgfpoint{354.861877pt}{137.030060pt}}
\pgflineto{\pgfpoint{354.963684pt}{136.737534pt}}
\pgfusepath{stroke}
\pgfpathmoveto{\pgfpoint{354.851410pt}{143.027802pt}}
\pgflineto{\pgfpoint{354.963684pt}{142.725281pt}}
\pgfusepath{stroke}
\pgfpathmoveto{\pgfpoint{354.839539pt}{149.025681pt}}
\pgflineto{\pgfpoint{354.963684pt}{148.713058pt}}
\pgfusepath{stroke}
\pgfpathmoveto{\pgfpoint{354.826111pt}{155.023514pt}}
\pgflineto{\pgfpoint{354.963684pt}{154.700806pt}}
\pgfusepath{stroke}
\pgfpathmoveto{\pgfpoint{354.811005pt}{161.021057pt}}
\pgflineto{\pgfpoint{354.963684pt}{160.688568pt}}
\pgfusepath{stroke}
\pgfpathmoveto{\pgfpoint{354.794067pt}{167.018021pt}}
\pgflineto{\pgfpoint{354.963684pt}{166.676315pt}}
\pgfusepath{stroke}
\pgfpathmoveto{\pgfpoint{354.775299pt}{173.013977pt}}
\pgflineto{\pgfpoint{354.963684pt}{172.664078pt}}
\pgfusepath{stroke}
\pgfpathmoveto{\pgfpoint{354.754761pt}{179.008423pt}}
\pgflineto{\pgfpoint{354.963684pt}{178.651825pt}}
\pgfusepath{stroke}
\pgfpathmoveto{\pgfpoint{354.732758pt}{185.000778pt}}
\pgflineto{\pgfpoint{354.963684pt}{184.639587pt}}
\pgfusepath{stroke}
\pgfpathmoveto{\pgfpoint{354.709900pt}{190.990463pt}}
\pgflineto{\pgfpoint{354.963684pt}{190.627335pt}}
\pgfusepath{stroke}
\pgfpathmoveto{\pgfpoint{354.687164pt}{196.977051pt}}
\pgflineto{\pgfpoint{354.963684pt}{196.615097pt}}
\pgfusepath{stroke}
\pgfpathmoveto{\pgfpoint{354.665955pt}{202.960464pt}}
\pgflineto{\pgfpoint{354.963684pt}{202.602844pt}}
\pgfusepath{stroke}
\pgfpathmoveto{\pgfpoint{354.648041pt}{208.941238pt}}
\pgflineto{\pgfpoint{354.963684pt}{208.590607pt}}
\pgfusepath{stroke}
\pgfpathmoveto{\pgfpoint{354.635315pt}{214.920776pt}}
\pgflineto{\pgfpoint{354.963684pt}{214.578354pt}}
\pgfusepath{stroke}
\pgfpathmoveto{\pgfpoint{354.629425pt}{220.901596pt}}
\pgflineto{\pgfpoint{354.963684pt}{220.566116pt}}
\pgfusepath{stroke}
\pgfpathmoveto{\pgfpoint{354.631531pt}{226.887268pt}}
\pgflineto{\pgfpoint{354.963684pt}{226.553864pt}}
\pgfusepath{stroke}
\pgfpathmoveto{\pgfpoint{354.641846pt}{232.882278pt}}
\pgflineto{\pgfpoint{354.963684pt}{232.541626pt}}
\pgfusepath{stroke}
\pgfpathmoveto{\pgfpoint{354.659943pt}{238.892090pt}}
\pgflineto{\pgfpoint{354.963684pt}{238.529388pt}}
\pgfusepath{stroke}
\pgfpathmoveto{\pgfpoint{354.685059pt}{244.923615pt}}
\pgflineto{\pgfpoint{354.963684pt}{244.517136pt}}
\pgfusepath{stroke}
\pgfpathmoveto{\pgfpoint{354.717194pt}{250.987030pt}}
\pgflineto{\pgfpoint{354.963684pt}{250.504883pt}}
\pgfusepath{stroke}
\pgfpathmoveto{\pgfpoint{354.758545pt}{257.100555pt}}
\pgflineto{\pgfpoint{354.963684pt}{256.492645pt}}
\pgfusepath{stroke}
\pgfpathmoveto{\pgfpoint{354.818481pt}{263.303284pt}}
\pgflineto{\pgfpoint{354.963684pt}{262.480408pt}}
\pgfusepath{stroke}
\pgfpathmoveto{\pgfpoint{354.933929pt}{269.699463pt}}
\pgflineto{\pgfpoint{354.963684pt}{268.468170pt}}
\pgfusepath{stroke}
\pgfpathmoveto{\pgfpoint{355.323853pt}{276.674469pt}}
\pgflineto{\pgfpoint{354.963684pt}{274.455902pt}}
\pgfusepath{stroke}
\pgfpathmoveto{\pgfpoint{360.244873pt}{285.724884pt}}
\pgflineto{\pgfpoint{354.963684pt}{280.443665pt}}
\pgfusepath{stroke}
\pgfpathmoveto{\pgfpoint{356.501068pt}{282.311035pt}}
\pgflineto{\pgfpoint{354.963684pt}{286.431427pt}}
\pgfusepath{stroke}
\pgfpathmoveto{\pgfpoint{355.153351pt}{290.383453pt}}
\pgflineto{\pgfpoint{354.963684pt}{292.419189pt}}
\pgfusepath{stroke}
\pgfpathmoveto{\pgfpoint{354.970581pt}{297.043915pt}}
\pgflineto{\pgfpoint{354.963684pt}{298.406921pt}}
\pgfusepath{stroke}
\pgfpathmoveto{\pgfpoint{354.920837pt}{303.354004pt}}
\pgflineto{\pgfpoint{354.963684pt}{304.394714pt}}
\pgfusepath{stroke}
\pgfpathmoveto{\pgfpoint{354.904480pt}{309.531189pt}}
\pgflineto{\pgfpoint{354.963684pt}{310.382446pt}}
\pgfusepath{stroke}
\pgfpathmoveto{\pgfpoint{354.899750pt}{315.644287pt}}
\pgflineto{\pgfpoint{354.963684pt}{316.370178pt}}
\pgfusepath{stroke}
\pgfpathmoveto{\pgfpoint{354.899719pt}{321.721619pt}}
\pgflineto{\pgfpoint{354.963684pt}{322.357971pt}}
\pgfusepath{stroke}
\pgfpathmoveto{\pgfpoint{354.901794pt}{327.776978pt}}
\pgflineto{\pgfpoint{354.963684pt}{328.345703pt}}
\pgfusepath{stroke}
\pgfpathmoveto{\pgfpoint{354.904724pt}{333.817841pt}}
\pgflineto{\pgfpoint{354.963684pt}{334.333466pt}}
\pgfusepath{stroke}
\pgfpathmoveto{\pgfpoint{354.908020pt}{339.848602pt}}
\pgflineto{\pgfpoint{354.963684pt}{340.321228pt}}
\pgfusepath{stroke}
\pgfpathmoveto{\pgfpoint{354.911346pt}{345.872009pt}}
\pgflineto{\pgfpoint{354.963684pt}{346.308960pt}}
\pgfusepath{stroke}
\pgfpathmoveto{\pgfpoint{354.914581pt}{351.889954pt}}
\pgflineto{\pgfpoint{354.963684pt}{352.296722pt}}
\pgfusepath{stroke}
\pgfpathmoveto{\pgfpoint{354.917633pt}{357.903564pt}}
\pgflineto{\pgfpoint{354.963684pt}{358.284485pt}}
\pgfusepath{stroke}
\pgfpathmoveto{\pgfpoint{354.920502pt}{363.913849pt}}
\pgflineto{\pgfpoint{354.963684pt}{364.272247pt}}
\pgfusepath{stroke}
\pgfpathmoveto{\pgfpoint{354.923187pt}{369.921387pt}}
\pgflineto{\pgfpoint{354.963684pt}{370.260010pt}}
\pgfusepath{stroke}
\pgfpathmoveto{\pgfpoint{360.901123pt}{77.068176pt}}
\pgflineto{\pgfpoint{360.951416pt}{76.859985pt}}
\pgfusepath{stroke}
\pgfpathmoveto{\pgfpoint{360.897491pt}{83.062042pt}}
\pgflineto{\pgfpoint{360.951416pt}{82.847733pt}}
\pgfusepath{stroke}
\pgfpathmoveto{\pgfpoint{360.893494pt}{89.056183pt}}
\pgflineto{\pgfpoint{360.951416pt}{88.835495pt}}
\pgfusepath{stroke}
\pgfpathmoveto{\pgfpoint{360.889069pt}{95.050621pt}}
\pgflineto{\pgfpoint{360.951416pt}{94.823257pt}}
\pgfusepath{stroke}
\pgfpathmoveto{\pgfpoint{360.884155pt}{101.045387pt}}
\pgflineto{\pgfpoint{360.951416pt}{100.811012pt}}
\pgfusepath{stroke}
\pgfpathmoveto{\pgfpoint{360.878723pt}{107.040459pt}}
\pgflineto{\pgfpoint{360.951416pt}{106.798759pt}}
\pgfusepath{stroke}
\pgfpathmoveto{\pgfpoint{360.872650pt}{113.035851pt}}
\pgflineto{\pgfpoint{360.951416pt}{112.786522pt}}
\pgfusepath{stroke}
\pgfpathmoveto{\pgfpoint{360.865845pt}{119.031532pt}}
\pgflineto{\pgfpoint{360.951416pt}{118.774277pt}}
\pgfusepath{stroke}
\pgfpathmoveto{\pgfpoint{360.858276pt}{125.027504pt}}
\pgflineto{\pgfpoint{360.951416pt}{124.762024pt}}
\pgfusepath{stroke}
\pgfpathmoveto{\pgfpoint{360.849823pt}{131.023712pt}}
\pgflineto{\pgfpoint{360.951416pt}{130.749786pt}}
\pgfusepath{stroke}
\pgfpathmoveto{\pgfpoint{360.840332pt}{137.020111pt}}
\pgflineto{\pgfpoint{360.951416pt}{136.737534pt}}
\pgfusepath{stroke}
\pgfpathmoveto{\pgfpoint{360.829742pt}{143.016617pt}}
\pgflineto{\pgfpoint{360.951416pt}{142.725281pt}}
\pgfusepath{stroke}
\pgfpathmoveto{\pgfpoint{360.817902pt}{149.013107pt}}
\pgflineto{\pgfpoint{360.951416pt}{148.713058pt}}
\pgfusepath{stroke}
\pgfpathmoveto{\pgfpoint{360.804688pt}{155.009430pt}}
\pgflineto{\pgfpoint{360.951416pt}{154.700806pt}}
\pgfusepath{stroke}
\pgfpathmoveto{\pgfpoint{360.790039pt}{161.005371pt}}
\pgflineto{\pgfpoint{360.951416pt}{160.688568pt}}
\pgfusepath{stroke}
\pgfpathmoveto{\pgfpoint{360.773926pt}{167.000671pt}}
\pgflineto{\pgfpoint{360.951416pt}{166.676315pt}}
\pgfusepath{stroke}
\pgfpathmoveto{\pgfpoint{360.756348pt}{172.995010pt}}
\pgflineto{\pgfpoint{360.951416pt}{172.664078pt}}
\pgfusepath{stroke}
\pgfpathmoveto{\pgfpoint{360.737427pt}{178.988037pt}}
\pgflineto{\pgfpoint{360.951416pt}{178.651825pt}}
\pgfusepath{stroke}
\pgfpathmoveto{\pgfpoint{360.717529pt}{184.979416pt}}
\pgflineto{\pgfpoint{360.951416pt}{184.639587pt}}
\pgfusepath{stroke}
\pgfpathmoveto{\pgfpoint{360.697144pt}{190.968842pt}}
\pgflineto{\pgfpoint{360.951416pt}{190.627335pt}}
\pgfusepath{stroke}
\pgfpathmoveto{\pgfpoint{360.676971pt}{196.956192pt}}
\pgflineto{\pgfpoint{360.951416pt}{196.615097pt}}
\pgfusepath{stroke}
\pgfpathmoveto{\pgfpoint{360.658051pt}{202.941681pt}}
\pgflineto{\pgfpoint{360.951416pt}{202.602844pt}}
\pgfusepath{stroke}
\pgfpathmoveto{\pgfpoint{360.641418pt}{208.925980pt}}
\pgflineto{\pgfpoint{360.951416pt}{208.590607pt}}
\pgfusepath{stroke}
\pgfpathmoveto{\pgfpoint{360.628174pt}{214.910446pt}}
\pgflineto{\pgfpoint{360.951416pt}{214.578354pt}}
\pgfusepath{stroke}
\pgfpathmoveto{\pgfpoint{360.619019pt}{220.897141pt}}
\pgflineto{\pgfpoint{360.951416pt}{220.566116pt}}
\pgfusepath{stroke}
\pgfpathmoveto{\pgfpoint{360.614166pt}{226.888916pt}}
\pgflineto{\pgfpoint{360.951416pt}{226.553864pt}}
\pgfusepath{stroke}
\pgfpathmoveto{\pgfpoint{360.613037pt}{232.889435pt}}
\pgflineto{\pgfpoint{360.951416pt}{232.541626pt}}
\pgfusepath{stroke}
\pgfpathmoveto{\pgfpoint{360.613922pt}{238.903305pt}}
\pgflineto{\pgfpoint{360.951416pt}{238.529388pt}}
\pgfusepath{stroke}
\pgfpathmoveto{\pgfpoint{360.613953pt}{244.936691pt}}
\pgflineto{\pgfpoint{360.951416pt}{244.517136pt}}
\pgfusepath{stroke}
\pgfpathmoveto{\pgfpoint{360.608002pt}{250.998840pt}}
\pgflineto{\pgfpoint{360.951416pt}{250.504883pt}}
\pgfusepath{stroke}
\pgfpathmoveto{\pgfpoint{360.586212pt}{257.105865pt}}
\pgflineto{\pgfpoint{360.951416pt}{256.492645pt}}
\pgfusepath{stroke}
\pgfpathmoveto{\pgfpoint{360.524872pt}{263.289764pt}}
\pgflineto{\pgfpoint{360.951416pt}{262.480408pt}}
\pgfusepath{stroke}
\pgfpathmoveto{\pgfpoint{360.348816pt}{269.622803pt}}
\pgflineto{\pgfpoint{360.951416pt}{268.468170pt}}
\pgfusepath{stroke}
\pgfpathmoveto{\pgfpoint{359.712524pt}{276.254730pt}}
\pgflineto{\pgfpoint{360.951416pt}{274.455902pt}}
\pgfusepath{stroke}
\pgfpathmoveto{\pgfpoint{356.831055pt}{281.981079pt}}
\pgflineto{\pgfpoint{360.951416pt}{280.443665pt}}
\pgfusepath{stroke}
\pgfpathmoveto{\pgfpoint{358.458435pt}{283.938446pt}}
\pgflineto{\pgfpoint{360.951416pt}{286.431427pt}}
\pgfusepath{stroke}
\pgfpathmoveto{\pgfpoint{360.132080pt}{290.618317pt}}
\pgflineto{\pgfpoint{360.951416pt}{292.419189pt}}
\pgfusepath{stroke}
\pgfpathmoveto{\pgfpoint{360.528107pt}{297.117920pt}}
\pgflineto{\pgfpoint{360.951416pt}{298.406921pt}}
\pgfusepath{stroke}
\pgfpathmoveto{\pgfpoint{360.672424pt}{303.389893pt}}
\pgflineto{\pgfpoint{360.951416pt}{304.394714pt}}
\pgfusepath{stroke}
\pgfpathmoveto{\pgfpoint{360.742798pt}{309.553345pt}}
\pgflineto{\pgfpoint{360.951416pt}{310.382446pt}}
\pgfusepath{stroke}
\pgfpathmoveto{\pgfpoint{360.783936pt}{315.659973pt}}
\pgflineto{\pgfpoint{360.951416pt}{316.370178pt}}
\pgfusepath{stroke}
\pgfpathmoveto{\pgfpoint{360.811127pt}{321.733704pt}}
\pgflineto{\pgfpoint{360.951416pt}{322.357971pt}}
\pgfusepath{stroke}
\pgfpathmoveto{\pgfpoint{360.830627pt}{327.786774pt}}
\pgflineto{\pgfpoint{360.951416pt}{328.345703pt}}
\pgfusepath{stroke}
\pgfpathmoveto{\pgfpoint{360.845428pt}{333.826019pt}}
\pgflineto{\pgfpoint{360.951416pt}{334.333466pt}}
\pgfusepath{stroke}
\pgfpathmoveto{\pgfpoint{360.857147pt}{339.855560pt}}
\pgflineto{\pgfpoint{360.951416pt}{340.321228pt}}
\pgfusepath{stroke}
\pgfpathmoveto{\pgfpoint{360.866730pt}{345.878052pt}}
\pgflineto{\pgfpoint{360.951416pt}{346.308960pt}}
\pgfusepath{stroke}
\pgfpathmoveto{\pgfpoint{360.874725pt}{351.895203pt}}
\pgflineto{\pgfpoint{360.951416pt}{352.296722pt}}
\pgfusepath{stroke}
\pgfpathmoveto{\pgfpoint{360.881531pt}{357.908203pt}}
\pgflineto{\pgfpoint{360.951416pt}{358.284485pt}}
\pgfusepath{stroke}
\pgfpathmoveto{\pgfpoint{360.887390pt}{363.917969pt}}
\pgflineto{\pgfpoint{360.951416pt}{364.272247pt}}
\pgfusepath{stroke}
\pgfpathmoveto{\pgfpoint{360.892487pt}{369.925110pt}}
\pgflineto{\pgfpoint{360.951416pt}{370.260010pt}}
\pgfusepath{stroke}
\pgfpathmoveto{\pgfpoint{366.883087pt}{77.064575pt}}
\pgflineto{\pgfpoint{366.939178pt}{76.859985pt}}
\pgfusepath{stroke}
\pgfpathmoveto{\pgfpoint{366.879211pt}{83.058075pt}}
\pgflineto{\pgfpoint{366.939178pt}{82.847733pt}}
\pgfusepath{stroke}
\pgfpathmoveto{\pgfpoint{366.874939pt}{89.051826pt}}
\pgflineto{\pgfpoint{366.939178pt}{88.835495pt}}
\pgfusepath{stroke}
\pgfpathmoveto{\pgfpoint{366.870209pt}{95.045822pt}}
\pgflineto{\pgfpoint{366.939178pt}{94.823257pt}}
\pgfusepath{stroke}
\pgfpathmoveto{\pgfpoint{366.865021pt}{101.040070pt}}
\pgflineto{\pgfpoint{366.939178pt}{100.811012pt}}
\pgfusepath{stroke}
\pgfpathmoveto{\pgfpoint{366.859314pt}{107.034561pt}}
\pgflineto{\pgfpoint{366.939178pt}{106.798759pt}}
\pgfusepath{stroke}
\pgfpathmoveto{\pgfpoint{366.852966pt}{113.029297pt}}
\pgflineto{\pgfpoint{366.939178pt}{112.786522pt}}
\pgfusepath{stroke}
\pgfpathmoveto{\pgfpoint{366.845917pt}{119.024254pt}}
\pgflineto{\pgfpoint{366.939178pt}{118.774277pt}}
\pgfusepath{stroke}
\pgfpathmoveto{\pgfpoint{366.838135pt}{125.019394pt}}
\pgflineto{\pgfpoint{366.939178pt}{124.762024pt}}
\pgfusepath{stroke}
\pgfpathmoveto{\pgfpoint{366.829498pt}{131.014694pt}}
\pgflineto{\pgfpoint{366.939178pt}{130.749786pt}}
\pgfusepath{stroke}
\pgfpathmoveto{\pgfpoint{366.819946pt}{137.010071pt}}
\pgflineto{\pgfpoint{366.939178pt}{136.737534pt}}
\pgfusepath{stroke}
\pgfpathmoveto{\pgfpoint{366.809326pt}{143.005432pt}}
\pgflineto{\pgfpoint{366.939178pt}{142.725281pt}}
\pgfusepath{stroke}
\pgfpathmoveto{\pgfpoint{366.797638pt}{149.000702pt}}
\pgflineto{\pgfpoint{366.939178pt}{148.713058pt}}
\pgfusepath{stroke}
\pgfpathmoveto{\pgfpoint{366.784729pt}{154.995712pt}}
\pgflineto{\pgfpoint{366.939178pt}{154.700806pt}}
\pgfusepath{stroke}
\pgfpathmoveto{\pgfpoint{366.770599pt}{160.990295pt}}
\pgflineto{\pgfpoint{366.939178pt}{160.688568pt}}
\pgfusepath{stroke}
\pgfpathmoveto{\pgfpoint{366.755249pt}{166.984253pt}}
\pgflineto{\pgfpoint{366.939178pt}{166.676315pt}}
\pgfusepath{stroke}
\pgfpathmoveto{\pgfpoint{366.738708pt}{172.977356pt}}
\pgflineto{\pgfpoint{366.939178pt}{172.664078pt}}
\pgfusepath{stroke}
\pgfpathmoveto{\pgfpoint{366.721130pt}{178.969345pt}}
\pgflineto{\pgfpoint{366.939178pt}{178.651825pt}}
\pgfusepath{stroke}
\pgfpathmoveto{\pgfpoint{366.702820pt}{184.960052pt}}
\pgflineto{\pgfpoint{366.939178pt}{184.639587pt}}
\pgfusepath{stroke}
\pgfpathmoveto{\pgfpoint{366.684143pt}{190.949341pt}}
\pgflineto{\pgfpoint{366.939178pt}{190.627335pt}}
\pgfusepath{stroke}
\pgfpathmoveto{\pgfpoint{366.665619pt}{196.937271pt}}
\pgflineto{\pgfpoint{366.939178pt}{196.615097pt}}
\pgfusepath{stroke}
\pgfpathmoveto{\pgfpoint{366.647827pt}{202.924164pt}}
\pgflineto{\pgfpoint{366.939178pt}{202.602844pt}}
\pgfusepath{stroke}
\pgfpathmoveto{\pgfpoint{366.631409pt}{208.910706pt}}
\pgflineto{\pgfpoint{366.939178pt}{208.590607pt}}
\pgfusepath{stroke}
\pgfpathmoveto{\pgfpoint{366.616821pt}{214.898010pt}}
\pgflineto{\pgfpoint{366.939178pt}{214.578354pt}}
\pgfusepath{stroke}
\pgfpathmoveto{\pgfpoint{366.604126pt}{220.887711pt}}
\pgflineto{\pgfpoint{366.939178pt}{220.566116pt}}
\pgfusepath{stroke}
\pgfpathmoveto{\pgfpoint{366.592957pt}{226.881973pt}}
\pgflineto{\pgfpoint{366.939178pt}{226.553864pt}}
\pgfusepath{stroke}
\pgfpathmoveto{\pgfpoint{366.581909pt}{232.883484pt}}
\pgflineto{\pgfpoint{366.939178pt}{232.541626pt}}
\pgfusepath{stroke}
\pgfpathmoveto{\pgfpoint{366.568481pt}{238.895493pt}}
\pgflineto{\pgfpoint{366.939178pt}{238.529388pt}}
\pgfusepath{stroke}
\pgfpathmoveto{\pgfpoint{366.548126pt}{244.921997pt}}
\pgflineto{\pgfpoint{366.939178pt}{244.517136pt}}
\pgfusepath{stroke}
\pgfpathmoveto{\pgfpoint{366.512878pt}{250.968109pt}}
\pgflineto{\pgfpoint{366.939178pt}{250.504883pt}}
\pgfusepath{stroke}
\pgfpathmoveto{\pgfpoint{366.447449pt}{257.039978pt}}
\pgflineto{\pgfpoint{366.939178pt}{256.492645pt}}
\pgfusepath{stroke}
\pgfpathmoveto{\pgfpoint{366.319885pt}{263.142090pt}}
\pgflineto{\pgfpoint{366.939178pt}{262.480408pt}}
\pgfusepath{stroke}
\pgfpathmoveto{\pgfpoint{366.059540pt}{269.257996pt}}
\pgflineto{\pgfpoint{366.939178pt}{268.468170pt}}
\pgfusepath{stroke}
\pgfpathmoveto{\pgfpoint{365.540649pt}{275.248657pt}}
\pgflineto{\pgfpoint{366.939178pt}{274.455902pt}}
\pgfusepath{stroke}
\pgfpathmoveto{\pgfpoint{364.903442pt}{280.633362pt}}
\pgflineto{\pgfpoint{366.939178pt}{280.443665pt}}
\pgfusepath{stroke}
\pgfpathmoveto{\pgfpoint{365.138306pt}{285.612122pt}}
\pgflineto{\pgfpoint{366.939178pt}{286.431427pt}}
\pgfusepath{stroke}
\pgfpathmoveto{\pgfpoint{365.834534pt}{291.314545pt}}
\pgflineto{\pgfpoint{366.939178pt}{292.419189pt}}
\pgfusepath{stroke}
\pgfpathmoveto{\pgfpoint{366.262329pt}{297.397339pt}}
\pgflineto{\pgfpoint{366.939178pt}{298.406921pt}}
\pgfusepath{stroke}
\pgfpathmoveto{\pgfpoint{366.484009pt}{303.524048pt}}
\pgflineto{\pgfpoint{366.939178pt}{304.394714pt}}
\pgfusepath{stroke}
\pgfpathmoveto{\pgfpoint{366.606628pt}{309.628296pt}}
\pgflineto{\pgfpoint{366.939178pt}{310.382446pt}}
\pgfusepath{stroke}
\pgfpathmoveto{\pgfpoint{366.680939pt}{315.706879pt}}
\pgflineto{\pgfpoint{366.939178pt}{316.370178pt}}
\pgfusepath{stroke}
\pgfpathmoveto{\pgfpoint{366.729736pt}{321.765656pt}}
\pgflineto{\pgfpoint{366.939178pt}{322.357971pt}}
\pgfusepath{stroke}
\pgfpathmoveto{\pgfpoint{366.763916pt}{327.809937pt}}
\pgflineto{\pgfpoint{366.939178pt}{328.345703pt}}
\pgfusepath{stroke}
\pgfpathmoveto{\pgfpoint{366.789093pt}{333.843628pt}}
\pgflineto{\pgfpoint{366.939178pt}{334.333466pt}}
\pgfusepath{stroke}
\pgfpathmoveto{\pgfpoint{366.808411pt}{339.869446pt}}
\pgflineto{\pgfpoint{366.939178pt}{340.321228pt}}
\pgfusepath{stroke}
\pgfpathmoveto{\pgfpoint{366.823669pt}{345.889282pt}}
\pgflineto{\pgfpoint{366.939178pt}{346.308960pt}}
\pgfusepath{stroke}
\pgfpathmoveto{\pgfpoint{366.836090pt}{351.904510pt}}
\pgflineto{\pgfpoint{366.939178pt}{352.296722pt}}
\pgfusepath{stroke}
\pgfpathmoveto{\pgfpoint{366.846375pt}{357.916077pt}}
\pgflineto{\pgfpoint{366.939178pt}{358.284485pt}}
\pgfusepath{stroke}
\pgfpathmoveto{\pgfpoint{366.855042pt}{363.924683pt}}
\pgflineto{\pgfpoint{366.939178pt}{364.272247pt}}
\pgfusepath{stroke}
\pgfpathmoveto{\pgfpoint{366.862396pt}{369.930908pt}}
\pgflineto{\pgfpoint{366.939178pt}{370.260010pt}}
\pgfusepath{stroke}
\pgfpathmoveto{\pgfpoint{372.865417pt}{77.060776pt}}
\pgflineto{\pgfpoint{372.926941pt}{76.859985pt}}
\pgfusepath{stroke}
\pgfpathmoveto{\pgfpoint{372.861267pt}{83.053879pt}}
\pgflineto{\pgfpoint{372.926941pt}{82.847733pt}}
\pgfusepath{stroke}
\pgfpathmoveto{\pgfpoint{372.856781pt}{89.047218pt}}
\pgflineto{\pgfpoint{372.926941pt}{88.835495pt}}
\pgfusepath{stroke}
\pgfpathmoveto{\pgfpoint{372.851868pt}{95.040771pt}}
\pgflineto{\pgfpoint{372.926941pt}{94.823257pt}}
\pgfusepath{stroke}
\pgfpathmoveto{\pgfpoint{372.846436pt}{101.034492pt}}
\pgflineto{\pgfpoint{372.926941pt}{100.811012pt}}
\pgfusepath{stroke}
\pgfpathmoveto{\pgfpoint{372.840485pt}{107.028419pt}}
\pgflineto{\pgfpoint{372.926941pt}{106.798759pt}}
\pgfusepath{stroke}
\pgfpathmoveto{\pgfpoint{372.833954pt}{113.022522pt}}
\pgflineto{\pgfpoint{372.926941pt}{112.786522pt}}
\pgfusepath{stroke}
\pgfpathmoveto{\pgfpoint{372.826752pt}{119.016777pt}}
\pgflineto{\pgfpoint{372.926941pt}{118.774277pt}}
\pgfusepath{stroke}
\pgfpathmoveto{\pgfpoint{372.818848pt}{125.011139pt}}
\pgflineto{\pgfpoint{372.926941pt}{124.762024pt}}
\pgfusepath{stroke}
\pgfpathmoveto{\pgfpoint{372.810120pt}{131.005554pt}}
\pgflineto{\pgfpoint{372.926941pt}{130.749786pt}}
\pgfusepath{stroke}
\pgfpathmoveto{\pgfpoint{372.800568pt}{137.000000pt}}
\pgflineto{\pgfpoint{372.926941pt}{136.737534pt}}
\pgfusepath{stroke}
\pgfpathmoveto{\pgfpoint{372.790039pt}{142.994354pt}}
\pgflineto{\pgfpoint{372.926941pt}{142.725281pt}}
\pgfusepath{stroke}
\pgfpathmoveto{\pgfpoint{372.778534pt}{148.988556pt}}
\pgflineto{\pgfpoint{372.926941pt}{148.713058pt}}
\pgfusepath{stroke}
\pgfpathmoveto{\pgfpoint{372.765991pt}{154.982422pt}}
\pgflineto{\pgfpoint{372.926941pt}{154.700806pt}}
\pgfusepath{stroke}
\pgfpathmoveto{\pgfpoint{372.752380pt}{160.975861pt}}
\pgflineto{\pgfpoint{372.926941pt}{160.688568pt}}
\pgfusepath{stroke}
\pgfpathmoveto{\pgfpoint{372.737732pt}{166.968704pt}}
\pgflineto{\pgfpoint{372.926941pt}{166.676315pt}}
\pgfusepath{stroke}
\pgfpathmoveto{\pgfpoint{372.722107pt}{172.960785pt}}
\pgflineto{\pgfpoint{372.926941pt}{172.664078pt}}
\pgfusepath{stroke}
\pgfpathmoveto{\pgfpoint{372.705627pt}{178.951950pt}}
\pgflineto{\pgfpoint{372.926941pt}{178.651825pt}}
\pgfusepath{stroke}
\pgfpathmoveto{\pgfpoint{372.688507pt}{184.942108pt}}
\pgflineto{\pgfpoint{372.926941pt}{184.639587pt}}
\pgfusepath{stroke}
\pgfpathmoveto{\pgfpoint{372.671021pt}{190.931229pt}}
\pgflineto{\pgfpoint{372.926941pt}{190.627335pt}}
\pgfusepath{stroke}
\pgfpathmoveto{\pgfpoint{372.653473pt}{196.919464pt}}
\pgflineto{\pgfpoint{372.926941pt}{196.615097pt}}
\pgfusepath{stroke}
\pgfpathmoveto{\pgfpoint{372.636230pt}{202.907104pt}}
\pgflineto{\pgfpoint{372.926941pt}{202.602844pt}}
\pgfusepath{stroke}
\pgfpathmoveto{\pgfpoint{372.619507pt}{208.894730pt}}
\pgflineto{\pgfpoint{372.926941pt}{208.590607pt}}
\pgfusepath{stroke}
\pgfpathmoveto{\pgfpoint{372.603424pt}{214.883255pt}}
\pgflineto{\pgfpoint{372.926941pt}{214.578354pt}}
\pgfusepath{stroke}
\pgfpathmoveto{\pgfpoint{372.587646pt}{220.873825pt}}
\pgflineto{\pgfpoint{372.926941pt}{220.566116pt}}
\pgfusepath{stroke}
\pgfpathmoveto{\pgfpoint{372.571381pt}{226.867905pt}}
\pgflineto{\pgfpoint{372.926941pt}{226.553864pt}}
\pgfusepath{stroke}
\pgfpathmoveto{\pgfpoint{372.552917pt}{232.867157pt}}
\pgflineto{\pgfpoint{372.926941pt}{232.541626pt}}
\pgfusepath{stroke}
\pgfpathmoveto{\pgfpoint{372.529419pt}{238.873322pt}}
\pgflineto{\pgfpoint{372.926941pt}{238.529388pt}}
\pgfusepath{stroke}
\pgfpathmoveto{\pgfpoint{372.496155pt}{244.887955pt}}
\pgflineto{\pgfpoint{372.926941pt}{244.517136pt}}
\pgfusepath{stroke}
\pgfpathmoveto{\pgfpoint{372.445496pt}{250.911621pt}}
\pgflineto{\pgfpoint{372.926941pt}{250.504883pt}}
\pgfusepath{stroke}
\pgfpathmoveto{\pgfpoint{372.364929pt}{256.941925pt}}
\pgflineto{\pgfpoint{372.926941pt}{256.492645pt}}
\pgfusepath{stroke}
\pgfpathmoveto{\pgfpoint{372.234924pt}{262.967194pt}}
\pgflineto{\pgfpoint{372.926941pt}{262.480408pt}}
\pgfusepath{stroke}
\pgfpathmoveto{\pgfpoint{372.032166pt}{268.950897pt}}
\pgflineto{\pgfpoint{372.926941pt}{268.468170pt}}
\pgfusepath{stroke}
\pgfpathmoveto{\pgfpoint{371.764130pt}{274.807892pt}}
\pgflineto{\pgfpoint{372.926941pt}{274.455902pt}}
\pgfusepath{stroke}
\pgfpathmoveto{\pgfpoint{371.563904pt}{280.450562pt}}
\pgflineto{\pgfpoint{372.926941pt}{280.443665pt}}
\pgfusepath{stroke}
\pgfpathmoveto{\pgfpoint{371.637909pt}{286.008118pt}}
\pgflineto{\pgfpoint{372.926941pt}{286.431427pt}}
\pgfusepath{stroke}
\pgfpathmoveto{\pgfpoint{371.917328pt}{291.742340pt}}
\pgflineto{\pgfpoint{372.926941pt}{292.419189pt}}
\pgfusepath{stroke}
\pgfpathmoveto{\pgfpoint{372.190125pt}{297.670105pt}}
\pgflineto{\pgfpoint{372.926941pt}{298.406921pt}}
\pgfusepath{stroke}
\pgfpathmoveto{\pgfpoint{372.385010pt}{303.687134pt}}
\pgflineto{\pgfpoint{372.926941pt}{304.394714pt}}
\pgfusepath{stroke}
\pgfpathmoveto{\pgfpoint{372.514648pt}{309.729553pt}}
\pgflineto{\pgfpoint{372.926941pt}{310.382446pt}}
\pgfusepath{stroke}
\pgfpathmoveto{\pgfpoint{372.601685pt}{315.773438pt}}
\pgflineto{\pgfpoint{372.926941pt}{316.370178pt}}
\pgfusepath{stroke}
\pgfpathmoveto{\pgfpoint{372.662170pt}{321.811829pt}}
\pgflineto{\pgfpoint{372.926941pt}{322.357971pt}}
\pgfusepath{stroke}
\pgfpathmoveto{\pgfpoint{372.705811pt}{327.843475pt}}
\pgflineto{\pgfpoint{372.926941pt}{328.345703pt}}
\pgfusepath{stroke}
\pgfpathmoveto{\pgfpoint{372.738403pt}{333.868958pt}}
\pgflineto{\pgfpoint{372.926941pt}{334.333466pt}}
\pgfusepath{stroke}
\pgfpathmoveto{\pgfpoint{372.763550pt}{339.889221pt}}
\pgflineto{\pgfpoint{372.926941pt}{340.321228pt}}
\pgfusepath{stroke}
\pgfpathmoveto{\pgfpoint{372.783447pt}{345.905151pt}}
\pgflineto{\pgfpoint{372.926941pt}{346.308960pt}}
\pgfusepath{stroke}
\pgfpathmoveto{\pgfpoint{372.799530pt}{351.917480pt}}
\pgflineto{\pgfpoint{372.926941pt}{352.296722pt}}
\pgfusepath{stroke}
\pgfpathmoveto{\pgfpoint{372.812805pt}{357.926880pt}}
\pgflineto{\pgfpoint{372.926941pt}{358.284485pt}}
\pgfusepath{stroke}
\pgfpathmoveto{\pgfpoint{372.823883pt}{363.933807pt}}
\pgflineto{\pgfpoint{372.926941pt}{364.272247pt}}
\pgfusepath{stroke}
\pgfpathmoveto{\pgfpoint{372.833313pt}{369.938721pt}}
\pgflineto{\pgfpoint{372.926941pt}{370.260010pt}}
\pgfusepath{stroke}
\pgfpathmoveto{\pgfpoint{378.848053pt}{77.056747pt}}
\pgflineto{\pgfpoint{378.914673pt}{76.859985pt}}
\pgfusepath{stroke}
\pgfpathmoveto{\pgfpoint{378.843750pt}{83.049500pt}}
\pgflineto{\pgfpoint{378.914673pt}{82.847733pt}}
\pgfusepath{stroke}
\pgfpathmoveto{\pgfpoint{378.839081pt}{89.042427pt}}
\pgflineto{\pgfpoint{378.914673pt}{88.835495pt}}
\pgfusepath{stroke}
\pgfpathmoveto{\pgfpoint{378.833954pt}{95.035507pt}}
\pgflineto{\pgfpoint{378.914673pt}{94.823257pt}}
\pgfusepath{stroke}
\pgfpathmoveto{\pgfpoint{378.828369pt}{101.028748pt}}
\pgflineto{\pgfpoint{378.914673pt}{100.811012pt}}
\pgfusepath{stroke}
\pgfpathmoveto{\pgfpoint{378.822266pt}{107.022110pt}}
\pgflineto{\pgfpoint{378.914673pt}{106.798759pt}}
\pgfusepath{stroke}
\pgfpathmoveto{\pgfpoint{378.815613pt}{113.015610pt}}
\pgflineto{\pgfpoint{378.914673pt}{112.786522pt}}
\pgfusepath{stroke}
\pgfpathmoveto{\pgfpoint{378.808289pt}{119.009186pt}}
\pgflineto{\pgfpoint{378.914673pt}{118.774277pt}}
\pgfusepath{stroke}
\pgfpathmoveto{\pgfpoint{378.800323pt}{125.002823pt}}
\pgflineto{\pgfpoint{378.914673pt}{124.762024pt}}
\pgfusepath{stroke}
\pgfpathmoveto{\pgfpoint{378.791626pt}{130.996445pt}}
\pgflineto{\pgfpoint{378.914673pt}{130.749786pt}}
\pgfusepath{stroke}
\pgfpathmoveto{\pgfpoint{378.782104pt}{136.990021pt}}
\pgflineto{\pgfpoint{378.914673pt}{136.737534pt}}
\pgfusepath{stroke}
\pgfpathmoveto{\pgfpoint{378.771729pt}{142.983459pt}}
\pgflineto{\pgfpoint{378.914673pt}{142.725281pt}}
\pgfusepath{stroke}
\pgfpathmoveto{\pgfpoint{378.760498pt}{148.976685pt}}
\pgflineto{\pgfpoint{378.914673pt}{148.713058pt}}
\pgfusepath{stroke}
\pgfpathmoveto{\pgfpoint{378.748322pt}{154.969574pt}}
\pgflineto{\pgfpoint{378.914673pt}{154.700806pt}}
\pgfusepath{stroke}
\pgfpathmoveto{\pgfpoint{378.735229pt}{160.962021pt}}
\pgflineto{\pgfpoint{378.914673pt}{160.688568pt}}
\pgfusepath{stroke}
\pgfpathmoveto{\pgfpoint{378.721252pt}{166.953918pt}}
\pgflineto{\pgfpoint{378.914673pt}{166.676315pt}}
\pgfusepath{stroke}
\pgfpathmoveto{\pgfpoint{378.706390pt}{172.945129pt}}
\pgflineto{\pgfpoint{378.914673pt}{172.664078pt}}
\pgfusepath{stroke}
\pgfpathmoveto{\pgfpoint{378.690796pt}{178.935577pt}}
\pgflineto{\pgfpoint{378.914673pt}{178.651825pt}}
\pgfusepath{stroke}
\pgfpathmoveto{\pgfpoint{378.674591pt}{184.925232pt}}
\pgflineto{\pgfpoint{378.914673pt}{184.639587pt}}
\pgfusepath{stroke}
\pgfpathmoveto{\pgfpoint{378.657959pt}{190.914062pt}}
\pgflineto{\pgfpoint{378.914673pt}{190.627335pt}}
\pgfusepath{stroke}
\pgfpathmoveto{\pgfpoint{378.641052pt}{196.902267pt}}
\pgflineto{\pgfpoint{378.914673pt}{196.615097pt}}
\pgfusepath{stroke}
\pgfpathmoveto{\pgfpoint{378.624054pt}{202.890076pt}}
\pgflineto{\pgfpoint{378.914673pt}{202.602844pt}}
\pgfusepath{stroke}
\pgfpathmoveto{\pgfpoint{378.606964pt}{208.877945pt}}
\pgflineto{\pgfpoint{378.914673pt}{208.590607pt}}
\pgfusepath{stroke}
\pgfpathmoveto{\pgfpoint{378.589630pt}{214.866486pt}}
\pgflineto{\pgfpoint{378.914673pt}{214.578354pt}}
\pgfusepath{stroke}
\pgfpathmoveto{\pgfpoint{378.571564pt}{220.856445pt}}
\pgflineto{\pgfpoint{378.914673pt}{220.566116pt}}
\pgfusepath{stroke}
\pgfpathmoveto{\pgfpoint{378.551788pt}{226.848663pt}}
\pgflineto{\pgfpoint{378.914673pt}{226.553864pt}}
\pgfusepath{stroke}
\pgfpathmoveto{\pgfpoint{378.528656pt}{232.843887pt}}
\pgflineto{\pgfpoint{378.914673pt}{232.541626pt}}
\pgfusepath{stroke}
\pgfpathmoveto{\pgfpoint{378.499603pt}{238.842590pt}}
\pgflineto{\pgfpoint{378.914673pt}{238.529388pt}}
\pgfusepath{stroke}
\pgfpathmoveto{\pgfpoint{378.460754pt}{244.844421pt}}
\pgflineto{\pgfpoint{378.914673pt}{244.517136pt}}
\pgfusepath{stroke}
\pgfpathmoveto{\pgfpoint{378.406769pt}{250.847260pt}}
\pgflineto{\pgfpoint{378.914673pt}{250.504883pt}}
\pgfusepath{stroke}
\pgfpathmoveto{\pgfpoint{378.330597pt}{256.845184pt}}
\pgflineto{\pgfpoint{378.914673pt}{256.492645pt}}
\pgfusepath{stroke}
\pgfpathmoveto{\pgfpoint{378.225403pt}{262.824951pt}}
\pgflineto{\pgfpoint{378.914673pt}{262.480408pt}}
\pgfusepath{stroke}
\pgfpathmoveto{\pgfpoint{378.091614pt}{268.761780pt}}
\pgflineto{\pgfpoint{378.914673pt}{268.468170pt}}
\pgfusepath{stroke}
\pgfpathmoveto{\pgfpoint{377.953949pt}{274.623199pt}}
\pgflineto{\pgfpoint{378.914673pt}{274.455902pt}}
\pgfusepath{stroke}
\pgfpathmoveto{\pgfpoint{377.874023pt}{280.400818pt}}
\pgflineto{\pgfpoint{378.914673pt}{280.443665pt}}
\pgfusepath{stroke}
\pgfpathmoveto{\pgfpoint{377.909882pt}{286.152435pt}}
\pgflineto{\pgfpoint{378.914673pt}{286.431427pt}}
\pgfusepath{stroke}
\pgfpathmoveto{\pgfpoint{378.044037pt}{291.964050pt}}
\pgflineto{\pgfpoint{378.914673pt}{292.419189pt}}
\pgfusepath{stroke}
\pgfpathmoveto{\pgfpoint{378.207153pt}{297.865021pt}}
\pgflineto{\pgfpoint{378.914673pt}{298.406921pt}}
\pgfusepath{stroke}
\pgfpathmoveto{\pgfpoint{378.351318pt}{303.831329pt}}
\pgflineto{\pgfpoint{378.914673pt}{304.394714pt}}
\pgfusepath{stroke}
\pgfpathmoveto{\pgfpoint{378.463776pt}{309.831757pt}}
\pgflineto{\pgfpoint{378.914673pt}{310.382446pt}}
\pgfusepath{stroke}
\pgfpathmoveto{\pgfpoint{378.547974pt}{315.846313pt}}
\pgflineto{\pgfpoint{378.914673pt}{316.370178pt}}
\pgfusepath{stroke}
\pgfpathmoveto{\pgfpoint{378.610840pt}{321.864929pt}}
\pgflineto{\pgfpoint{378.914673pt}{322.357971pt}}
\pgfusepath{stroke}
\pgfpathmoveto{\pgfpoint{378.658386pt}{327.883270pt}}
\pgflineto{\pgfpoint{378.914673pt}{328.345703pt}}
\pgfusepath{stroke}
\pgfpathmoveto{\pgfpoint{378.695007pt}{333.899567pt}}
\pgflineto{\pgfpoint{378.914673pt}{334.333466pt}}
\pgfusepath{stroke}
\pgfpathmoveto{\pgfpoint{378.723816pt}{339.913300pt}}
\pgflineto{\pgfpoint{378.914673pt}{340.321228pt}}
\pgfusepath{stroke}
\pgfpathmoveto{\pgfpoint{378.746918pt}{345.924500pt}}
\pgflineto{\pgfpoint{378.914673pt}{346.308960pt}}
\pgfusepath{stroke}
\pgfpathmoveto{\pgfpoint{378.765747pt}{351.933350pt}}
\pgflineto{\pgfpoint{378.914673pt}{352.296722pt}}
\pgfusepath{stroke}
\pgfpathmoveto{\pgfpoint{378.781342pt}{357.940063pt}}
\pgflineto{\pgfpoint{378.914673pt}{358.284485pt}}
\pgfusepath{stroke}
\pgfpathmoveto{\pgfpoint{378.794434pt}{363.944946pt}}
\pgflineto{\pgfpoint{378.914673pt}{364.272247pt}}
\pgfusepath{stroke}
\pgfpathmoveto{\pgfpoint{378.805542pt}{369.948242pt}}
\pgflineto{\pgfpoint{378.914673pt}{370.260010pt}}
\pgfusepath{stroke}
\pgfpathmoveto{\pgfpoint{384.831055pt}{77.052551pt}}
\pgflineto{\pgfpoint{384.902435pt}{76.859985pt}}
\pgfusepath{stroke}
\pgfpathmoveto{\pgfpoint{384.826630pt}{83.044922pt}}
\pgflineto{\pgfpoint{384.902435pt}{82.847733pt}}
\pgfusepath{stroke}
\pgfpathmoveto{\pgfpoint{384.821777pt}{89.037445pt}}
\pgflineto{\pgfpoint{384.902435pt}{88.835495pt}}
\pgfusepath{stroke}
\pgfpathmoveto{\pgfpoint{384.816528pt}{95.030098pt}}
\pgflineto{\pgfpoint{384.902435pt}{94.823257pt}}
\pgfusepath{stroke}
\pgfpathmoveto{\pgfpoint{384.810822pt}{101.022842pt}}
\pgflineto{\pgfpoint{384.902435pt}{100.811012pt}}
\pgfusepath{stroke}
\pgfpathmoveto{\pgfpoint{384.804626pt}{107.015701pt}}
\pgflineto{\pgfpoint{384.902435pt}{106.798759pt}}
\pgfusepath{stroke}
\pgfpathmoveto{\pgfpoint{384.797882pt}{113.008591pt}}
\pgflineto{\pgfpoint{384.902435pt}{112.786522pt}}
\pgfusepath{stroke}
\pgfpathmoveto{\pgfpoint{384.790527pt}{119.001541pt}}
\pgflineto{\pgfpoint{384.902435pt}{118.774277pt}}
\pgfusepath{stroke}
\pgfpathmoveto{\pgfpoint{384.782562pt}{124.994484pt}}
\pgflineto{\pgfpoint{384.902435pt}{124.762024pt}}
\pgfusepath{stroke}
\pgfpathmoveto{\pgfpoint{384.773895pt}{130.987381pt}}
\pgflineto{\pgfpoint{384.902435pt}{130.749786pt}}
\pgfusepath{stroke}
\pgfpathmoveto{\pgfpoint{384.764496pt}{136.980164pt}}
\pgflineto{\pgfpoint{384.902435pt}{136.737534pt}}
\pgfusepath{stroke}
\pgfpathmoveto{\pgfpoint{384.754333pt}{142.972778pt}}
\pgflineto{\pgfpoint{384.902435pt}{142.725281pt}}
\pgfusepath{stroke}
\pgfpathmoveto{\pgfpoint{384.743378pt}{148.965149pt}}
\pgflineto{\pgfpoint{384.902435pt}{148.713058pt}}
\pgfusepath{stroke}
\pgfpathmoveto{\pgfpoint{384.731598pt}{154.957153pt}}
\pgflineto{\pgfpoint{384.902435pt}{154.700806pt}}
\pgfusepath{stroke}
\pgfpathmoveto{\pgfpoint{384.718994pt}{160.948746pt}}
\pgflineto{\pgfpoint{384.902435pt}{160.688568pt}}
\pgfusepath{stroke}
\pgfpathmoveto{\pgfpoint{384.705566pt}{166.939804pt}}
\pgflineto{\pgfpoint{384.902435pt}{166.676315pt}}
\pgfusepath{stroke}
\pgfpathmoveto{\pgfpoint{384.691406pt}{172.930267pt}}
\pgflineto{\pgfpoint{384.902435pt}{172.664078pt}}
\pgfusepath{stroke}
\pgfpathmoveto{\pgfpoint{384.676544pt}{178.920044pt}}
\pgflineto{\pgfpoint{384.902435pt}{178.651825pt}}
\pgfusepath{stroke}
\pgfpathmoveto{\pgfpoint{384.661102pt}{184.909149pt}}
\pgflineto{\pgfpoint{384.902435pt}{184.639587pt}}
\pgfusepath{stroke}
\pgfpathmoveto{\pgfpoint{384.645142pt}{190.897583pt}}
\pgflineto{\pgfpoint{384.902435pt}{190.627335pt}}
\pgfusepath{stroke}
\pgfpathmoveto{\pgfpoint{384.628723pt}{196.885483pt}}
\pgflineto{\pgfpoint{384.902435pt}{196.615097pt}}
\pgfusepath{stroke}
\pgfpathmoveto{\pgfpoint{384.611908pt}{202.873032pt}}
\pgflineto{\pgfpoint{384.902435pt}{202.602844pt}}
\pgfusepath{stroke}
\pgfpathmoveto{\pgfpoint{384.594604pt}{208.860519pt}}
\pgflineto{\pgfpoint{384.902435pt}{208.590607pt}}
\pgfusepath{stroke}
\pgfpathmoveto{\pgfpoint{384.576508pt}{214.848312pt}}
\pgflineto{\pgfpoint{384.902435pt}{214.578354pt}}
\pgfusepath{stroke}
\pgfpathmoveto{\pgfpoint{384.557068pt}{220.836823pt}}
\pgflineto{\pgfpoint{384.902435pt}{220.566116pt}}
\pgfusepath{stroke}
\pgfpathmoveto{\pgfpoint{384.535400pt}{226.826355pt}}
\pgflineto{\pgfpoint{384.902435pt}{226.553864pt}}
\pgfusepath{stroke}
\pgfpathmoveto{\pgfpoint{384.510132pt}{232.817032pt}}
\pgflineto{\pgfpoint{384.902435pt}{232.541626pt}}
\pgfusepath{stroke}
\pgfpathmoveto{\pgfpoint{384.479279pt}{238.808472pt}}
\pgflineto{\pgfpoint{384.902435pt}{238.529388pt}}
\pgfusepath{stroke}
\pgfpathmoveto{\pgfpoint{384.440277pt}{244.799332pt}}
\pgflineto{\pgfpoint{384.902435pt}{244.517136pt}}
\pgfusepath{stroke}
\pgfpathmoveto{\pgfpoint{384.390137pt}{250.786606pt}}
\pgflineto{\pgfpoint{384.902435pt}{250.504883pt}}
\pgfusepath{stroke}
\pgfpathmoveto{\pgfpoint{384.326202pt}{256.764526pt}}
\pgflineto{\pgfpoint{384.902435pt}{256.492645pt}}
\pgfusepath{stroke}
\pgfpathmoveto{\pgfpoint{384.248230pt}{262.723572pt}}
\pgflineto{\pgfpoint{384.902435pt}{262.480408pt}}
\pgfusepath{stroke}
\pgfpathmoveto{\pgfpoint{384.162720pt}{268.650940pt}}
\pgflineto{\pgfpoint{384.902435pt}{268.468170pt}}
\pgfusepath{stroke}
\pgfpathmoveto{\pgfpoint{384.087799pt}{274.536346pt}}
\pgflineto{\pgfpoint{384.902435pt}{274.455902pt}}
\pgfusepath{stroke}
\pgfpathmoveto{\pgfpoint{384.051208pt}{280.384491pt}}
\pgflineto{\pgfpoint{384.902435pt}{280.443665pt}}
\pgfusepath{stroke}
\pgfpathmoveto{\pgfpoint{384.073334pt}{286.222778pt}}
\pgflineto{\pgfpoint{384.902435pt}{286.431427pt}}
\pgfusepath{stroke}
\pgfpathmoveto{\pgfpoint{384.148285pt}{292.086639pt}}
\pgflineto{\pgfpoint{384.902435pt}{292.419189pt}}
\pgfusepath{stroke}
\pgfpathmoveto{\pgfpoint{384.249542pt}{297.994629pt}}
\pgflineto{\pgfpoint{384.902435pt}{298.406921pt}}
\pgfusepath{stroke}
\pgfpathmoveto{\pgfpoint{384.351746pt}{303.943787pt}}
\pgflineto{\pgfpoint{384.902435pt}{304.394714pt}}
\pgfusepath{stroke}
\pgfpathmoveto{\pgfpoint{384.441528pt}{309.921509pt}}
\pgflineto{\pgfpoint{384.902435pt}{310.382446pt}}
\pgfusepath{stroke}
\pgfpathmoveto{\pgfpoint{384.515350pt}{315.915894pt}}
\pgflineto{\pgfpoint{384.902435pt}{316.370178pt}}
\pgfusepath{stroke}
\pgfpathmoveto{\pgfpoint{384.574493pt}{321.918762pt}}
\pgflineto{\pgfpoint{384.902435pt}{322.357971pt}}
\pgfusepath{stroke}
\pgfpathmoveto{\pgfpoint{384.621613pt}{327.925262pt}}
\pgflineto{\pgfpoint{384.902435pt}{328.345703pt}}
\pgfusepath{stroke}
\pgfpathmoveto{\pgfpoint{384.659302pt}{333.932770pt}}
\pgflineto{\pgfpoint{384.902435pt}{334.333466pt}}
\pgfusepath{stroke}
\pgfpathmoveto{\pgfpoint{384.689728pt}{339.939972pt}}
\pgflineto{\pgfpoint{384.902435pt}{340.321228pt}}
\pgfusepath{stroke}
\pgfpathmoveto{\pgfpoint{384.714600pt}{345.946228pt}}
\pgflineto{\pgfpoint{384.902435pt}{346.308960pt}}
\pgfusepath{stroke}
\pgfpathmoveto{\pgfpoint{384.735168pt}{351.951324pt}}
\pgflineto{\pgfpoint{384.902435pt}{352.296722pt}}
\pgfusepath{stroke}
\pgfpathmoveto{\pgfpoint{384.752350pt}{357.955139pt}}
\pgflineto{\pgfpoint{384.902435pt}{358.284485pt}}
\pgfusepath{stroke}
\pgfpathmoveto{\pgfpoint{384.766907pt}{363.957703pt}}
\pgflineto{\pgfpoint{384.902435pt}{364.272247pt}}
\pgfusepath{stroke}
\pgfpathmoveto{\pgfpoint{384.779327pt}{369.959167pt}}
\pgflineto{\pgfpoint{384.902435pt}{370.260010pt}}
\pgfusepath{stroke}
\pgfpathmoveto{\pgfpoint{390.814423pt}{77.048203pt}}
\pgflineto{\pgfpoint{390.890198pt}{76.859985pt}}
\pgfusepath{stroke}
\pgfpathmoveto{\pgfpoint{390.809875pt}{83.040237pt}}
\pgflineto{\pgfpoint{390.890198pt}{82.847733pt}}
\pgfusepath{stroke}
\pgfpathmoveto{\pgfpoint{390.804932pt}{89.032364pt}}
\pgflineto{\pgfpoint{390.890198pt}{88.835495pt}}
\pgfusepath{stroke}
\pgfpathmoveto{\pgfpoint{390.799561pt}{95.024574pt}}
\pgflineto{\pgfpoint{390.890198pt}{94.823257pt}}
\pgfusepath{stroke}
\pgfpathmoveto{\pgfpoint{390.793793pt}{101.016869pt}}
\pgflineto{\pgfpoint{390.890198pt}{100.811012pt}}
\pgfusepath{stroke}
\pgfpathmoveto{\pgfpoint{390.787537pt}{107.009193pt}}
\pgflineto{\pgfpoint{390.890198pt}{106.798759pt}}
\pgfusepath{stroke}
\pgfpathmoveto{\pgfpoint{390.780762pt}{113.001549pt}}
\pgflineto{\pgfpoint{390.890198pt}{112.786522pt}}
\pgfusepath{stroke}
\pgfpathmoveto{\pgfpoint{390.773407pt}{118.993904pt}}
\pgflineto{\pgfpoint{390.890198pt}{118.774277pt}}
\pgfusepath{stroke}
\pgfpathmoveto{\pgfpoint{390.765472pt}{124.986206pt}}
\pgflineto{\pgfpoint{390.890198pt}{124.762024pt}}
\pgfusepath{stroke}
\pgfpathmoveto{\pgfpoint{390.756897pt}{130.978424pt}}
\pgflineto{\pgfpoint{390.890198pt}{130.749786pt}}
\pgfusepath{stroke}
\pgfpathmoveto{\pgfpoint{390.747681pt}{136.970490pt}}
\pgflineto{\pgfpoint{390.890198pt}{136.737534pt}}
\pgfusepath{stroke}
\pgfpathmoveto{\pgfpoint{390.737732pt}{142.962341pt}}
\pgflineto{\pgfpoint{390.890198pt}{142.725281pt}}
\pgfusepath{stroke}
\pgfpathmoveto{\pgfpoint{390.727081pt}{148.953949pt}}
\pgflineto{\pgfpoint{390.890198pt}{148.713058pt}}
\pgfusepath{stroke}
\pgfpathmoveto{\pgfpoint{390.715698pt}{154.945175pt}}
\pgflineto{\pgfpoint{390.890198pt}{154.700806pt}}
\pgfusepath{stroke}
\pgfpathmoveto{\pgfpoint{390.703552pt}{160.935989pt}}
\pgflineto{\pgfpoint{390.890198pt}{160.688568pt}}
\pgfusepath{stroke}
\pgfpathmoveto{\pgfpoint{390.690674pt}{166.926300pt}}
\pgflineto{\pgfpoint{390.890198pt}{166.676315pt}}
\pgfusepath{stroke}
\pgfpathmoveto{\pgfpoint{390.677124pt}{172.916061pt}}
\pgflineto{\pgfpoint{390.890198pt}{172.664078pt}}
\pgfusepath{stroke}
\pgfpathmoveto{\pgfpoint{390.662933pt}{178.905212pt}}
\pgflineto{\pgfpoint{390.890198pt}{178.651825pt}}
\pgfusepath{stroke}
\pgfpathmoveto{\pgfpoint{390.648132pt}{184.893738pt}}
\pgflineto{\pgfpoint{390.890198pt}{184.639587pt}}
\pgfusepath{stroke}
\pgfpathmoveto{\pgfpoint{390.632751pt}{190.881668pt}}
\pgflineto{\pgfpoint{390.890198pt}{190.627335pt}}
\pgfusepath{stroke}
\pgfpathmoveto{\pgfpoint{390.616821pt}{196.869064pt}}
\pgflineto{\pgfpoint{390.890198pt}{196.615097pt}}
\pgfusepath{stroke}
\pgfpathmoveto{\pgfpoint{390.600281pt}{202.856049pt}}
\pgflineto{\pgfpoint{390.890198pt}{202.602844pt}}
\pgfusepath{stroke}
\pgfpathmoveto{\pgfpoint{390.582947pt}{208.842773pt}}
\pgflineto{\pgfpoint{390.890198pt}{208.590607pt}}
\pgfusepath{stroke}
\pgfpathmoveto{\pgfpoint{390.564636pt}{214.829422pt}}
\pgflineto{\pgfpoint{390.890198pt}{214.578354pt}}
\pgfusepath{stroke}
\pgfpathmoveto{\pgfpoint{390.544739pt}{220.816116pt}}
\pgflineto{\pgfpoint{390.890198pt}{220.566116pt}}
\pgfusepath{stroke}
\pgfpathmoveto{\pgfpoint{390.522552pt}{226.802826pt}}
\pgflineto{\pgfpoint{390.890198pt}{226.553864pt}}
\pgfusepath{stroke}
\pgfpathmoveto{\pgfpoint{390.497070pt}{232.789276pt}}
\pgflineto{\pgfpoint{390.890198pt}{232.541626pt}}
\pgfusepath{stroke}
\pgfpathmoveto{\pgfpoint{390.467041pt}{238.774658pt}}
\pgflineto{\pgfpoint{390.890198pt}{238.529388pt}}
\pgfusepath{stroke}
\pgfpathmoveto{\pgfpoint{390.430939pt}{244.757370pt}}
\pgflineto{\pgfpoint{390.890198pt}{244.517136pt}}
\pgfusepath{stroke}
\pgfpathmoveto{\pgfpoint{390.387634pt}{250.734634pt}}
\pgflineto{\pgfpoint{390.890198pt}{250.504883pt}}
\pgfusepath{stroke}
\pgfpathmoveto{\pgfpoint{390.336853pt}{256.702087pt}}
\pgflineto{\pgfpoint{390.890198pt}{256.492645pt}}
\pgfusepath{stroke}
\pgfpathmoveto{\pgfpoint{390.280731pt}{262.654022pt}}
\pgflineto{\pgfpoint{390.890198pt}{262.480408pt}}
\pgfusepath{stroke}
\pgfpathmoveto{\pgfpoint{390.225464pt}{268.584595pt}}
\pgflineto{\pgfpoint{390.890198pt}{268.468170pt}}
\pgfusepath{stroke}
\pgfpathmoveto{\pgfpoint{390.182251pt}{274.491364pt}}
\pgflineto{\pgfpoint{390.890198pt}{274.455902pt}}
\pgfusepath{stroke}
\pgfpathmoveto{\pgfpoint{390.164276pt}{280.379761pt}}
\pgflineto{\pgfpoint{390.890198pt}{280.443665pt}}
\pgfusepath{stroke}
\pgfpathmoveto{\pgfpoint{390.179993pt}{286.263977pt}}
\pgflineto{\pgfpoint{390.890198pt}{286.431427pt}}
\pgfusepath{stroke}
\pgfpathmoveto{\pgfpoint{390.226868pt}{292.160950pt}}
\pgflineto{\pgfpoint{390.890198pt}{292.419189pt}}
\pgfusepath{stroke}
\pgfpathmoveto{\pgfpoint{390.293427pt}{298.081696pt}}
\pgflineto{\pgfpoint{390.890198pt}{298.406921pt}}
\pgfusepath{stroke}
\pgfpathmoveto{\pgfpoint{390.366272pt}{304.027985pt}}
\pgflineto{\pgfpoint{390.890198pt}{304.394714pt}}
\pgfusepath{stroke}
\pgfpathmoveto{\pgfpoint{390.435913pt}{309.995361pt}}
\pgflineto{\pgfpoint{390.890198pt}{310.382446pt}}
\pgfusepath{stroke}
\pgfpathmoveto{\pgfpoint{390.497620pt}{315.977631pt}}
\pgflineto{\pgfpoint{390.890198pt}{316.370178pt}}
\pgfusepath{stroke}
\pgfpathmoveto{\pgfpoint{390.550232pt}{321.969299pt}}
\pgflineto{\pgfpoint{390.890198pt}{322.357971pt}}
\pgfusepath{stroke}
\pgfpathmoveto{\pgfpoint{390.594177pt}{327.966431pt}}
\pgflineto{\pgfpoint{390.890198pt}{328.345703pt}}
\pgfusepath{stroke}
\pgfpathmoveto{\pgfpoint{390.630737pt}{333.966400pt}}
\pgflineto{\pgfpoint{390.890198pt}{334.333466pt}}
\pgfusepath{stroke}
\pgfpathmoveto{\pgfpoint{390.661133pt}{339.967651pt}}
\pgflineto{\pgfpoint{390.890198pt}{340.321228pt}}
\pgfusepath{stroke}
\pgfpathmoveto{\pgfpoint{390.686523pt}{345.969238pt}}
\pgflineto{\pgfpoint{390.890198pt}{346.308960pt}}
\pgfusepath{stroke}
\pgfpathmoveto{\pgfpoint{390.707886pt}{351.970581pt}}
\pgflineto{\pgfpoint{390.890198pt}{352.296722pt}}
\pgfusepath{stroke}
\pgfpathmoveto{\pgfpoint{390.726013pt}{357.971436pt}}
\pgflineto{\pgfpoint{390.890198pt}{358.284485pt}}
\pgfusepath{stroke}
\pgfpathmoveto{\pgfpoint{390.741516pt}{363.971619pt}}
\pgflineto{\pgfpoint{390.890198pt}{364.272247pt}}
\pgfusepath{stroke}
\pgfpathmoveto{\pgfpoint{390.754822pt}{369.971130pt}}
\pgflineto{\pgfpoint{390.890198pt}{370.260010pt}}
\pgfusepath{stroke}
\pgfpathmoveto{\pgfpoint{396.798126pt}{77.043747pt}}
\pgflineto{\pgfpoint{396.877960pt}{76.859985pt}}
\pgfusepath{stroke}
\pgfpathmoveto{\pgfpoint{396.793457pt}{83.035431pt}}
\pgflineto{\pgfpoint{396.877960pt}{82.847733pt}}
\pgfusepath{stroke}
\pgfpathmoveto{\pgfpoint{396.788452pt}{89.027184pt}}
\pgflineto{\pgfpoint{396.877960pt}{88.835495pt}}
\pgfusepath{stroke}
\pgfpathmoveto{\pgfpoint{396.783051pt}{95.018990pt}}
\pgflineto{\pgfpoint{396.877960pt}{94.823257pt}}
\pgfusepath{stroke}
\pgfpathmoveto{\pgfpoint{396.777222pt}{101.010826pt}}
\pgflineto{\pgfpoint{396.877960pt}{100.811012pt}}
\pgfusepath{stroke}
\pgfpathmoveto{\pgfpoint{396.770935pt}{107.002670pt}}
\pgflineto{\pgfpoint{396.877960pt}{106.798759pt}}
\pgfusepath{stroke}
\pgfpathmoveto{\pgfpoint{396.764160pt}{112.994507pt}}
\pgflineto{\pgfpoint{396.877960pt}{112.786522pt}}
\pgfusepath{stroke}
\pgfpathmoveto{\pgfpoint{396.756897pt}{118.986298pt}}
\pgflineto{\pgfpoint{396.877960pt}{118.774277pt}}
\pgfusepath{stroke}
\pgfpathmoveto{\pgfpoint{396.749023pt}{124.978012pt}}
\pgflineto{\pgfpoint{396.877960pt}{124.762024pt}}
\pgfusepath{stroke}
\pgfpathmoveto{\pgfpoint{396.740601pt}{130.969604pt}}
\pgflineto{\pgfpoint{396.877960pt}{130.749786pt}}
\pgfusepath{stroke}
\pgfpathmoveto{\pgfpoint{396.731567pt}{136.960999pt}}
\pgflineto{\pgfpoint{396.877960pt}{136.737534pt}}
\pgfusepath{stroke}
\pgfpathmoveto{\pgfpoint{396.721863pt}{142.952194pt}}
\pgflineto{\pgfpoint{396.877960pt}{142.725281pt}}
\pgfusepath{stroke}
\pgfpathmoveto{\pgfpoint{396.711517pt}{148.943085pt}}
\pgflineto{\pgfpoint{396.877960pt}{148.713058pt}}
\pgfusepath{stroke}
\pgfpathmoveto{\pgfpoint{396.700500pt}{154.933624pt}}
\pgflineto{\pgfpoint{396.877960pt}{154.700806pt}}
\pgfusepath{stroke}
\pgfpathmoveto{\pgfpoint{396.688843pt}{160.923737pt}}
\pgflineto{\pgfpoint{396.877960pt}{160.688568pt}}
\pgfusepath{stroke}
\pgfpathmoveto{\pgfpoint{396.676483pt}{166.913361pt}}
\pgflineto{\pgfpoint{396.877960pt}{166.676315pt}}
\pgfusepath{stroke}
\pgfpathmoveto{\pgfpoint{396.663513pt}{172.902466pt}}
\pgflineto{\pgfpoint{396.877960pt}{172.664078pt}}
\pgfusepath{stroke}
\pgfpathmoveto{\pgfpoint{396.649902pt}{178.890991pt}}
\pgflineto{\pgfpoint{396.877960pt}{178.651825pt}}
\pgfusepath{stroke}
\pgfpathmoveto{\pgfpoint{396.635681pt}{184.878937pt}}
\pgflineto{\pgfpoint{396.877960pt}{184.639587pt}}
\pgfusepath{stroke}
\pgfpathmoveto{\pgfpoint{396.620880pt}{190.866272pt}}
\pgflineto{\pgfpoint{396.877960pt}{190.627335pt}}
\pgfusepath{stroke}
\pgfpathmoveto{\pgfpoint{396.605469pt}{196.853058pt}}
\pgflineto{\pgfpoint{396.877960pt}{196.615097pt}}
\pgfusepath{stroke}
\pgfpathmoveto{\pgfpoint{396.589355pt}{202.839325pt}}
\pgflineto{\pgfpoint{396.877960pt}{202.602844pt}}
\pgfusepath{stroke}
\pgfpathmoveto{\pgfpoint{396.572327pt}{208.825119pt}}
\pgflineto{\pgfpoint{396.877960pt}{208.590607pt}}
\pgfusepath{stroke}
\pgfpathmoveto{\pgfpoint{396.554260pt}{214.810486pt}}
\pgflineto{\pgfpoint{396.877960pt}{214.578354pt}}
\pgfusepath{stroke}
\pgfpathmoveto{\pgfpoint{396.534607pt}{220.795349pt}}
\pgflineto{\pgfpoint{396.877960pt}{220.566116pt}}
\pgfusepath{stroke}
\pgfpathmoveto{\pgfpoint{396.512939pt}{226.779480pt}}
\pgflineto{\pgfpoint{396.877960pt}{226.553864pt}}
\pgfusepath{stroke}
\pgfpathmoveto{\pgfpoint{396.488586pt}{232.762436pt}}
\pgflineto{\pgfpoint{396.877960pt}{232.541626pt}}
\pgfusepath{stroke}
\pgfpathmoveto{\pgfpoint{396.460815pt}{238.743301pt}}
\pgflineto{\pgfpoint{396.877960pt}{238.529388pt}}
\pgfusepath{stroke}
\pgfpathmoveto{\pgfpoint{396.428986pt}{244.720596pt}}
\pgflineto{\pgfpoint{396.877960pt}{244.517136pt}}
\pgfusepath{stroke}
\pgfpathmoveto{\pgfpoint{396.392975pt}{250.692169pt}}
\pgflineto{\pgfpoint{396.877960pt}{250.504883pt}}
\pgfusepath{stroke}
\pgfpathmoveto{\pgfpoint{396.353577pt}{256.655151pt}}
\pgflineto{\pgfpoint{396.877960pt}{256.492645pt}}
\pgfusepath{stroke}
\pgfpathmoveto{\pgfpoint{396.313263pt}{262.606354pt}}
\pgflineto{\pgfpoint{396.877960pt}{262.480408pt}}
\pgfusepath{stroke}
\pgfpathmoveto{\pgfpoint{396.276733pt}{268.543427pt}}
\pgflineto{\pgfpoint{396.877960pt}{268.468170pt}}
\pgfusepath{stroke}
\pgfpathmoveto{\pgfpoint{396.250580pt}{274.466492pt}}
\pgflineto{\pgfpoint{396.877960pt}{274.455902pt}}
\pgfusepath{stroke}
\pgfpathmoveto{\pgfpoint{396.241638pt}{280.379730pt}}
\pgflineto{\pgfpoint{396.877960pt}{280.443665pt}}
\pgfusepath{stroke}
\pgfpathmoveto{\pgfpoint{396.253723pt}{286.291138pt}}
\pgflineto{\pgfpoint{396.877960pt}{286.431427pt}}
\pgfusepath{stroke}
\pgfpathmoveto{\pgfpoint{396.285645pt}{292.209747pt}}
\pgflineto{\pgfpoint{396.877960pt}{292.419189pt}}
\pgfusepath{stroke}
\pgfpathmoveto{\pgfpoint{396.331818pt}{298.142181pt}}
\pgflineto{\pgfpoint{396.877960pt}{298.406921pt}}
\pgfusepath{stroke}
\pgfpathmoveto{\pgfpoint{396.384949pt}{304.090820pt}}
\pgflineto{\pgfpoint{396.877960pt}{304.394714pt}}
\pgfusepath{stroke}
\pgfpathmoveto{\pgfpoint{396.438751pt}{310.054504pt}}
\pgflineto{\pgfpoint{396.877960pt}{310.382446pt}}
\pgfusepath{stroke}
\pgfpathmoveto{\pgfpoint{396.489288pt}{316.030212pt}}
\pgflineto{\pgfpoint{396.877960pt}{316.370178pt}}
\pgfusepath{stroke}
\pgfpathmoveto{\pgfpoint{396.534607pt}{322.014618pt}}
\pgflineto{\pgfpoint{396.877960pt}{322.357971pt}}
\pgfusepath{stroke}
\pgfpathmoveto{\pgfpoint{396.574188pt}{328.004883pt}}
\pgflineto{\pgfpoint{396.877960pt}{328.345703pt}}
\pgfusepath{stroke}
\pgfpathmoveto{\pgfpoint{396.608276pt}{333.998871pt}}
\pgflineto{\pgfpoint{396.877960pt}{334.333466pt}}
\pgfusepath{stroke}
\pgfpathmoveto{\pgfpoint{396.637482pt}{339.995056pt}}
\pgflineto{\pgfpoint{396.877960pt}{340.321228pt}}
\pgfusepath{stroke}
\pgfpathmoveto{\pgfpoint{396.662415pt}{345.992432pt}}
\pgflineto{\pgfpoint{396.877960pt}{346.308960pt}}
\pgfusepath{stroke}
\pgfpathmoveto{\pgfpoint{396.683838pt}{351.990356pt}}
\pgflineto{\pgfpoint{396.877960pt}{352.296722pt}}
\pgfusepath{stroke}
\pgfpathmoveto{\pgfpoint{396.702271pt}{357.988373pt}}
\pgflineto{\pgfpoint{396.877960pt}{358.284485pt}}
\pgfusepath{stroke}
\pgfpathmoveto{\pgfpoint{396.718201pt}{363.986237pt}}
\pgflineto{\pgfpoint{396.877960pt}{364.272247pt}}
\pgfusepath{stroke}
\pgfpathmoveto{\pgfpoint{396.732056pt}{369.983826pt}}
\pgflineto{\pgfpoint{396.877960pt}{370.260010pt}}
\pgfusepath{stroke}
\pgfpathmoveto{\pgfpoint{402.782166pt}{77.039230pt}}
\pgflineto{\pgfpoint{402.865692pt}{76.859985pt}}
\pgfusepath{stroke}
\pgfpathmoveto{\pgfpoint{402.777435pt}{83.030579pt}}
\pgflineto{\pgfpoint{402.865692pt}{82.847733pt}}
\pgfusepath{stroke}
\pgfpathmoveto{\pgfpoint{402.772400pt}{89.021957pt}}
\pgflineto{\pgfpoint{402.865692pt}{88.835495pt}}
\pgfusepath{stroke}
\pgfpathmoveto{\pgfpoint{402.766937pt}{95.013359pt}}
\pgflineto{\pgfpoint{402.865692pt}{94.823257pt}}
\pgfusepath{stroke}
\pgfpathmoveto{\pgfpoint{402.761108pt}{101.004776pt}}
\pgflineto{\pgfpoint{402.865692pt}{100.811012pt}}
\pgfusepath{stroke}
\pgfpathmoveto{\pgfpoint{402.754822pt}{106.996170pt}}
\pgflineto{\pgfpoint{402.865692pt}{106.798759pt}}
\pgfusepath{stroke}
\pgfpathmoveto{\pgfpoint{402.748108pt}{112.987518pt}}
\pgflineto{\pgfpoint{402.865692pt}{112.786522pt}}
\pgfusepath{stroke}
\pgfpathmoveto{\pgfpoint{402.740906pt}{118.978783pt}}
\pgflineto{\pgfpoint{402.865692pt}{118.774277pt}}
\pgfusepath{stroke}
\pgfpathmoveto{\pgfpoint{402.733154pt}{124.969940pt}}
\pgflineto{\pgfpoint{402.865692pt}{124.762024pt}}
\pgfusepath{stroke}
\pgfpathmoveto{\pgfpoint{402.724915pt}{130.960953pt}}
\pgflineto{\pgfpoint{402.865692pt}{130.749786pt}}
\pgfusepath{stroke}
\pgfpathmoveto{\pgfpoint{402.716064pt}{136.951767pt}}
\pgflineto{\pgfpoint{402.865692pt}{136.737534pt}}
\pgfusepath{stroke}
\pgfpathmoveto{\pgfpoint{402.706665pt}{142.942322pt}}
\pgflineto{\pgfpoint{402.865692pt}{142.725281pt}}
\pgfusepath{stroke}
\pgfpathmoveto{\pgfpoint{402.696625pt}{148.932587pt}}
\pgflineto{\pgfpoint{402.865692pt}{148.713058pt}}
\pgfusepath{stroke}
\pgfpathmoveto{\pgfpoint{402.686005pt}{154.922485pt}}
\pgflineto{\pgfpoint{402.865692pt}{154.700806pt}}
\pgfusepath{stroke}
\pgfpathmoveto{\pgfpoint{402.674774pt}{160.911957pt}}
\pgflineto{\pgfpoint{402.865692pt}{160.688568pt}}
\pgfusepath{stroke}
\pgfpathmoveto{\pgfpoint{402.662933pt}{166.900970pt}}
\pgflineto{\pgfpoint{402.865692pt}{166.676315pt}}
\pgfusepath{stroke}
\pgfpathmoveto{\pgfpoint{402.650513pt}{172.889465pt}}
\pgflineto{\pgfpoint{402.865692pt}{172.664078pt}}
\pgfusepath{stroke}
\pgfpathmoveto{\pgfpoint{402.637482pt}{178.877380pt}}
\pgflineto{\pgfpoint{402.865692pt}{178.651825pt}}
\pgfusepath{stroke}
\pgfpathmoveto{\pgfpoint{402.623871pt}{184.864731pt}}
\pgflineto{\pgfpoint{402.865692pt}{184.639587pt}}
\pgfusepath{stroke}
\pgfpathmoveto{\pgfpoint{402.609650pt}{190.851456pt}}
\pgflineto{\pgfpoint{402.865692pt}{190.627335pt}}
\pgfusepath{stroke}
\pgfpathmoveto{\pgfpoint{402.594818pt}{196.837570pt}}
\pgflineto{\pgfpoint{402.865692pt}{196.615097pt}}
\pgfusepath{stroke}
\pgfpathmoveto{\pgfpoint{402.579224pt}{202.823059pt}}
\pgflineto{\pgfpoint{402.865692pt}{202.602844pt}}
\pgfusepath{stroke}
\pgfpathmoveto{\pgfpoint{402.562836pt}{208.807907pt}}
\pgflineto{\pgfpoint{402.865692pt}{208.590607pt}}
\pgfusepath{stroke}
\pgfpathmoveto{\pgfpoint{402.545349pt}{214.792007pt}}
\pgflineto{\pgfpoint{402.865692pt}{214.578354pt}}
\pgfusepath{stroke}
\pgfpathmoveto{\pgfpoint{402.526550pt}{220.775238pt}}
\pgflineto{\pgfpoint{402.865692pt}{220.566116pt}}
\pgfusepath{stroke}
\pgfpathmoveto{\pgfpoint{402.506042pt}{226.757278pt}}
\pgflineto{\pgfpoint{402.865692pt}{226.553864pt}}
\pgfusepath{stroke}
\pgfpathmoveto{\pgfpoint{402.483521pt}{232.737595pt}}
\pgflineto{\pgfpoint{402.865692pt}{232.541626pt}}
\pgfusepath{stroke}
\pgfpathmoveto{\pgfpoint{402.458679pt}{238.715378pt}}
\pgflineto{\pgfpoint{402.865692pt}{238.529388pt}}
\pgfusepath{stroke}
\pgfpathmoveto{\pgfpoint{402.431396pt}{244.689438pt}}
\pgflineto{\pgfpoint{402.865692pt}{244.517136pt}}
\pgfusepath{stroke}
\pgfpathmoveto{\pgfpoint{402.401978pt}{250.658264pt}}
\pgflineto{\pgfpoint{402.865692pt}{250.504883pt}}
\pgfusepath{stroke}
\pgfpathmoveto{\pgfpoint{402.371613pt}{256.620056pt}}
\pgflineto{\pgfpoint{402.865692pt}{256.492645pt}}
\pgfusepath{stroke}
\pgfpathmoveto{\pgfpoint{402.342407pt}{262.573242pt}}
\pgflineto{\pgfpoint{402.865692pt}{262.480408pt}}
\pgfusepath{stroke}
\pgfpathmoveto{\pgfpoint{402.317627pt}{268.516998pt}}
\pgflineto{\pgfpoint{402.865692pt}{268.468170pt}}
\pgfusepath{stroke}
\pgfpathmoveto{\pgfpoint{402.301270pt}{274.452209pt}}
\pgflineto{\pgfpoint{402.865692pt}{274.455902pt}}
\pgfusepath{stroke}
\pgfpathmoveto{\pgfpoint{402.296997pt}{280.381805pt}}
\pgflineto{\pgfpoint{402.865692pt}{280.443665pt}}
\pgfusepath{stroke}
\pgfpathmoveto{\pgfpoint{402.306763pt}{286.310608pt}}
\pgflineto{\pgfpoint{402.865692pt}{286.431427pt}}
\pgfusepath{stroke}
\pgfpathmoveto{\pgfpoint{402.329926pt}{292.243896pt}}
\pgflineto{\pgfpoint{402.865692pt}{292.419189pt}}
\pgfusepath{stroke}
\pgfpathmoveto{\pgfpoint{402.363464pt}{298.185791pt}}
\pgflineto{\pgfpoint{402.865692pt}{298.406921pt}}
\pgfusepath{stroke}
\pgfpathmoveto{\pgfpoint{402.403259pt}{304.138367pt}}
\pgflineto{\pgfpoint{402.865692pt}{304.394714pt}}
\pgfusepath{stroke}
\pgfpathmoveto{\pgfpoint{402.445251pt}{310.101624pt}}
\pgflineto{\pgfpoint{402.865692pt}{310.382446pt}}
\pgfusepath{stroke}
\pgfpathmoveto{\pgfpoint{402.486420pt}{316.074188pt}}
\pgflineto{\pgfpoint{402.865692pt}{316.370178pt}}
\pgfusepath{stroke}
\pgfpathmoveto{\pgfpoint{402.524902pt}{322.054199pt}}
\pgflineto{\pgfpoint{402.865692pt}{322.357971pt}}
\pgfusepath{stroke}
\pgfpathmoveto{\pgfpoint{402.559753pt}{328.039764pt}}
\pgflineto{\pgfpoint{402.865692pt}{328.345703pt}}
\pgfusepath{stroke}
\pgfpathmoveto{\pgfpoint{402.590759pt}{334.029236pt}}
\pgflineto{\pgfpoint{402.865692pt}{334.333466pt}}
\pgfusepath{stroke}
\pgfpathmoveto{\pgfpoint{402.618042pt}{340.021362pt}}
\pgflineto{\pgfpoint{402.865692pt}{340.321228pt}}
\pgfusepath{stroke}
\pgfpathmoveto{\pgfpoint{402.641907pt}{346.015167pt}}
\pgflineto{\pgfpoint{402.865692pt}{346.308960pt}}
\pgfusepath{stroke}
\pgfpathmoveto{\pgfpoint{402.662781pt}{352.010071pt}}
\pgflineto{\pgfpoint{402.865692pt}{352.296722pt}}
\pgfusepath{stroke}
\pgfpathmoveto{\pgfpoint{402.681000pt}{358.005493pt}}
\pgflineto{\pgfpoint{402.865692pt}{358.284485pt}}
\pgfusepath{stroke}
\pgfpathmoveto{\pgfpoint{402.696991pt}{364.001190pt}}
\pgflineto{\pgfpoint{402.865692pt}{364.272247pt}}
\pgfusepath{stroke}
\pgfpathmoveto{\pgfpoint{402.711060pt}{369.996887pt}}
\pgflineto{\pgfpoint{402.865692pt}{370.260010pt}}
\pgfusepath{stroke}
\pgfpathmoveto{\pgfpoint{408.766510pt}{77.034668pt}}
\pgflineto{\pgfpoint{408.853455pt}{76.859985pt}}
\pgfusepath{stroke}
\pgfpathmoveto{\pgfpoint{408.761749pt}{83.025696pt}}
\pgflineto{\pgfpoint{408.853455pt}{82.847733pt}}
\pgfusepath{stroke}
\pgfpathmoveto{\pgfpoint{408.756683pt}{89.016708pt}}
\pgflineto{\pgfpoint{408.853455pt}{88.835495pt}}
\pgfusepath{stroke}
\pgfpathmoveto{\pgfpoint{408.751221pt}{95.007736pt}}
\pgflineto{\pgfpoint{408.853455pt}{94.823257pt}}
\pgfusepath{stroke}
\pgfpathmoveto{\pgfpoint{408.745422pt}{100.998749pt}}
\pgflineto{\pgfpoint{408.853455pt}{100.811012pt}}
\pgfusepath{stroke}
\pgfpathmoveto{\pgfpoint{408.739197pt}{106.989700pt}}
\pgflineto{\pgfpoint{408.853455pt}{106.798759pt}}
\pgfusepath{stroke}
\pgfpathmoveto{\pgfpoint{408.732544pt}{112.980591pt}}
\pgflineto{\pgfpoint{408.853455pt}{112.786522pt}}
\pgfusepath{stroke}
\pgfpathmoveto{\pgfpoint{408.725464pt}{118.971382pt}}
\pgflineto{\pgfpoint{408.853455pt}{118.774277pt}}
\pgfusepath{stroke}
\pgfpathmoveto{\pgfpoint{408.717834pt}{124.962029pt}}
\pgflineto{\pgfpoint{408.853455pt}{124.762024pt}}
\pgfusepath{stroke}
\pgfpathmoveto{\pgfpoint{408.709778pt}{130.952515pt}}
\pgflineto{\pgfpoint{408.853455pt}{130.749786pt}}
\pgfusepath{stroke}
\pgfpathmoveto{\pgfpoint{408.701172pt}{136.942764pt}}
\pgflineto{\pgfpoint{408.853455pt}{136.737534pt}}
\pgfusepath{stroke}
\pgfpathmoveto{\pgfpoint{408.692017pt}{142.932770pt}}
\pgflineto{\pgfpoint{408.853455pt}{142.725281pt}}
\pgfusepath{stroke}
\pgfpathmoveto{\pgfpoint{408.682343pt}{148.922455pt}}
\pgflineto{\pgfpoint{408.853455pt}{148.713058pt}}
\pgfusepath{stroke}
\pgfpathmoveto{\pgfpoint{408.672119pt}{154.911758pt}}
\pgflineto{\pgfpoint{408.853455pt}{154.700806pt}}
\pgfusepath{stroke}
\pgfpathmoveto{\pgfpoint{408.661316pt}{160.900665pt}}
\pgflineto{\pgfpoint{408.853455pt}{160.688568pt}}
\pgfusepath{stroke}
\pgfpathmoveto{\pgfpoint{408.649963pt}{166.889099pt}}
\pgflineto{\pgfpoint{408.853455pt}{166.676315pt}}
\pgfusepath{stroke}
\pgfpathmoveto{\pgfpoint{408.638092pt}{172.877014pt}}
\pgflineto{\pgfpoint{408.853455pt}{172.664078pt}}
\pgfusepath{stroke}
\pgfpathmoveto{\pgfpoint{408.625641pt}{178.864365pt}}
\pgflineto{\pgfpoint{408.853455pt}{178.651825pt}}
\pgfusepath{stroke}
\pgfpathmoveto{\pgfpoint{408.612640pt}{184.851135pt}}
\pgflineto{\pgfpoint{408.853455pt}{184.639587pt}}
\pgfusepath{stroke}
\pgfpathmoveto{\pgfpoint{408.599060pt}{190.837250pt}}
\pgflineto{\pgfpoint{408.853455pt}{190.627335pt}}
\pgfusepath{stroke}
\pgfpathmoveto{\pgfpoint{408.584869pt}{196.822708pt}}
\pgflineto{\pgfpoint{408.853455pt}{196.615097pt}}
\pgfusepath{stroke}
\pgfpathmoveto{\pgfpoint{408.570007pt}{202.807449pt}}
\pgflineto{\pgfpoint{408.853455pt}{202.602844pt}}
\pgfusepath{stroke}
\pgfpathmoveto{\pgfpoint{408.554352pt}{208.791382pt}}
\pgflineto{\pgfpoint{408.853455pt}{208.590607pt}}
\pgfusepath{stroke}
\pgfpathmoveto{\pgfpoint{408.537781pt}{214.774399pt}}
\pgflineto{\pgfpoint{408.853455pt}{214.578354pt}}
\pgfusepath{stroke}
\pgfpathmoveto{\pgfpoint{408.520142pt}{220.756287pt}}
\pgflineto{\pgfpoint{408.853455pt}{220.566116pt}}
\pgfusepath{stroke}
\pgfpathmoveto{\pgfpoint{408.501190pt}{226.736725pt}}
\pgflineto{\pgfpoint{408.853455pt}{226.553864pt}}
\pgfusepath{stroke}
\pgfpathmoveto{\pgfpoint{408.480865pt}{232.715225pt}}
\pgflineto{\pgfpoint{408.853455pt}{232.541626pt}}
\pgfusepath{stroke}
\pgfpathmoveto{\pgfpoint{408.459045pt}{238.691101pt}}
\pgflineto{\pgfpoint{408.853455pt}{238.529388pt}}
\pgfusepath{stroke}
\pgfpathmoveto{\pgfpoint{408.435974pt}{244.663498pt}}
\pgflineto{\pgfpoint{408.853455pt}{244.517136pt}}
\pgfusepath{stroke}
\pgfpathmoveto{\pgfpoint{408.412170pt}{250.631378pt}}
\pgflineto{\pgfpoint{408.853455pt}{250.504883pt}}
\pgfusepath{stroke}
\pgfpathmoveto{\pgfpoint{408.388702pt}{256.593719pt}}
\pgflineto{\pgfpoint{408.853455pt}{256.492645pt}}
\pgfusepath{stroke}
\pgfpathmoveto{\pgfpoint{408.367249pt}{262.549805pt}}
\pgflineto{\pgfpoint{408.853455pt}{262.480408pt}}
\pgfusepath{stroke}
\pgfpathmoveto{\pgfpoint{408.350037pt}{268.499512pt}}
\pgflineto{\pgfpoint{408.853455pt}{268.468170pt}}
\pgfusepath{stroke}
\pgfpathmoveto{\pgfpoint{408.339539pt}{274.443787pt}}
\pgflineto{\pgfpoint{408.853455pt}{274.455902pt}}
\pgfusepath{stroke}
\pgfpathmoveto{\pgfpoint{408.337830pt}{280.384766pt}}
\pgflineto{\pgfpoint{408.853455pt}{280.443665pt}}
\pgfusepath{stroke}
\pgfpathmoveto{\pgfpoint{408.346008pt}{286.325439pt}}
\pgflineto{\pgfpoint{408.853455pt}{286.431427pt}}
\pgfusepath{stroke}
\pgfpathmoveto{\pgfpoint{408.363617pt}{292.269104pt}}
\pgflineto{\pgfpoint{408.853455pt}{292.419189pt}}
\pgfusepath{stroke}
\pgfpathmoveto{\pgfpoint{408.388977pt}{298.218414pt}}
\pgflineto{\pgfpoint{408.853455pt}{298.406921pt}}
\pgfusepath{stroke}
\pgfpathmoveto{\pgfpoint{408.419556pt}{304.175018pt}}
\pgflineto{\pgfpoint{408.853455pt}{304.394714pt}}
\pgfusepath{stroke}
\pgfpathmoveto{\pgfpoint{408.452759pt}{310.139282pt}}
\pgflineto{\pgfpoint{408.853455pt}{310.382446pt}}
\pgfusepath{stroke}
\pgfpathmoveto{\pgfpoint{408.486389pt}{316.110718pt}}
\pgflineto{\pgfpoint{408.853455pt}{316.370178pt}}
\pgfusepath{stroke}
\pgfpathmoveto{\pgfpoint{408.518860pt}{322.088287pt}}
\pgflineto{\pgfpoint{408.853455pt}{322.357971pt}}
\pgfusepath{stroke}
\pgfpathmoveto{\pgfpoint{408.549225pt}{328.070740pt}}
\pgflineto{\pgfpoint{408.853455pt}{328.345703pt}}
\pgfusepath{stroke}
\pgfpathmoveto{\pgfpoint{408.576965pt}{334.056976pt}}
\pgflineto{\pgfpoint{408.853455pt}{334.333466pt}}
\pgfusepath{stroke}
\pgfpathmoveto{\pgfpoint{408.601990pt}{340.045959pt}}
\pgflineto{\pgfpoint{408.853455pt}{340.321228pt}}
\pgfusepath{stroke}
\pgfpathmoveto{\pgfpoint{408.624390pt}{346.036926pt}}
\pgflineto{\pgfpoint{408.853455pt}{346.308960pt}}
\pgfusepath{stroke}
\pgfpathmoveto{\pgfpoint{408.644287pt}{352.029175pt}}
\pgflineto{\pgfpoint{408.853455pt}{352.296722pt}}
\pgfusepath{stroke}
\pgfpathmoveto{\pgfpoint{408.661987pt}{358.022369pt}}
\pgflineto{\pgfpoint{408.853455pt}{358.284485pt}}
\pgfusepath{stroke}
\pgfpathmoveto{\pgfpoint{408.677704pt}{364.016083pt}}
\pgflineto{\pgfpoint{408.853455pt}{364.272247pt}}
\pgfusepath{stroke}
\pgfpathmoveto{\pgfpoint{408.691681pt}{370.010101pt}}
\pgflineto{\pgfpoint{408.853455pt}{370.260010pt}}
\pgfusepath{stroke}
\pgfpathmoveto{\pgfpoint{414.751160pt}{77.030075pt}}
\pgflineto{\pgfpoint{414.841217pt}{76.859985pt}}
\pgfusepath{stroke}
\pgfpathmoveto{\pgfpoint{414.746399pt}{83.020767pt}}
\pgflineto{\pgfpoint{414.841217pt}{82.847733pt}}
\pgfusepath{stroke}
\pgfpathmoveto{\pgfpoint{414.741333pt}{89.011467pt}}
\pgflineto{\pgfpoint{414.841217pt}{88.835495pt}}
\pgfusepath{stroke}
\pgfpathmoveto{\pgfpoint{414.735931pt}{95.002121pt}}
\pgflineto{\pgfpoint{414.841217pt}{94.823257pt}}
\pgfusepath{stroke}
\pgfpathmoveto{\pgfpoint{414.730164pt}{100.992752pt}}
\pgflineto{\pgfpoint{414.841217pt}{100.811012pt}}
\pgfusepath{stroke}
\pgfpathmoveto{\pgfpoint{414.723999pt}{106.983307pt}}
\pgflineto{\pgfpoint{414.841217pt}{106.798759pt}}
\pgfusepath{stroke}
\pgfpathmoveto{\pgfpoint{414.717438pt}{112.973770pt}}
\pgflineto{\pgfpoint{414.841217pt}{112.786522pt}}
\pgfusepath{stroke}
\pgfpathmoveto{\pgfpoint{414.710449pt}{118.964104pt}}
\pgflineto{\pgfpoint{414.841217pt}{118.774277pt}}
\pgfusepath{stroke}
\pgfpathmoveto{\pgfpoint{414.703033pt}{124.954285pt}}
\pgflineto{\pgfpoint{414.841217pt}{124.762024pt}}
\pgfusepath{stroke}
\pgfpathmoveto{\pgfpoint{414.695160pt}{130.944275pt}}
\pgflineto{\pgfpoint{414.841217pt}{130.749786pt}}
\pgfusepath{stroke}
\pgfpathmoveto{\pgfpoint{414.686798pt}{136.934036pt}}
\pgflineto{\pgfpoint{414.841217pt}{136.737534pt}}
\pgfusepath{stroke}
\pgfpathmoveto{\pgfpoint{414.677948pt}{142.923523pt}}
\pgflineto{\pgfpoint{414.841217pt}{142.725281pt}}
\pgfusepath{stroke}
\pgfpathmoveto{\pgfpoint{414.668610pt}{148.912674pt}}
\pgflineto{\pgfpoint{414.841217pt}{148.713058pt}}
\pgfusepath{stroke}
\pgfpathmoveto{\pgfpoint{414.658783pt}{154.901459pt}}
\pgflineto{\pgfpoint{414.841217pt}{154.700806pt}}
\pgfusepath{stroke}
\pgfpathmoveto{\pgfpoint{414.648438pt}{160.889832pt}}
\pgflineto{\pgfpoint{414.841217pt}{160.688568pt}}
\pgfusepath{stroke}
\pgfpathmoveto{\pgfpoint{414.637573pt}{166.877731pt}}
\pgflineto{\pgfpoint{414.841217pt}{166.676315pt}}
\pgfusepath{stroke}
\pgfpathmoveto{\pgfpoint{414.626221pt}{172.865128pt}}
\pgflineto{\pgfpoint{414.841217pt}{172.664078pt}}
\pgfusepath{stroke}
\pgfpathmoveto{\pgfpoint{414.614349pt}{178.851944pt}}
\pgflineto{\pgfpoint{414.841217pt}{178.651825pt}}
\pgfusepath{stroke}
\pgfpathmoveto{\pgfpoint{414.601990pt}{184.838165pt}}
\pgflineto{\pgfpoint{414.841217pt}{184.639587pt}}
\pgfusepath{stroke}
\pgfpathmoveto{\pgfpoint{414.589081pt}{190.823730pt}}
\pgflineto{\pgfpoint{414.841217pt}{190.627335pt}}
\pgfusepath{stroke}
\pgfpathmoveto{\pgfpoint{414.575592pt}{196.808563pt}}
\pgflineto{\pgfpoint{414.841217pt}{196.615097pt}}
\pgfusepath{stroke}
\pgfpathmoveto{\pgfpoint{414.561523pt}{202.792618pt}}
\pgflineto{\pgfpoint{414.841217pt}{202.602844pt}}
\pgfusepath{stroke}
\pgfpathmoveto{\pgfpoint{414.546783pt}{208.775787pt}}
\pgflineto{\pgfpoint{414.841217pt}{208.590607pt}}
\pgfusepath{stroke}
\pgfpathmoveto{\pgfpoint{414.531311pt}{214.757904pt}}
\pgflineto{\pgfpoint{414.841217pt}{214.578354pt}}
\pgfusepath{stroke}
\pgfpathmoveto{\pgfpoint{414.514984pt}{220.738770pt}}
\pgflineto{\pgfpoint{414.841217pt}{220.566116pt}}
\pgfusepath{stroke}
\pgfpathmoveto{\pgfpoint{414.497772pt}{226.718079pt}}
\pgflineto{\pgfpoint{414.841217pt}{226.553864pt}}
\pgfusepath{stroke}
\pgfpathmoveto{\pgfpoint{414.479675pt}{232.695435pt}}
\pgflineto{\pgfpoint{414.841217pt}{232.541626pt}}
\pgfusepath{stroke}
\pgfpathmoveto{\pgfpoint{414.460754pt}{238.670288pt}}
\pgflineto{\pgfpoint{414.841217pt}{238.529388pt}}
\pgfusepath{stroke}
\pgfpathmoveto{\pgfpoint{414.441406pt}{244.642044pt}}
\pgflineto{\pgfpoint{414.841217pt}{244.517136pt}}
\pgfusepath{stroke}
\pgfpathmoveto{\pgfpoint{414.422119pt}{250.610062pt}}
\pgflineto{\pgfpoint{414.841217pt}{250.504883pt}}
\pgfusepath{stroke}
\pgfpathmoveto{\pgfpoint{414.403870pt}{256.573792pt}}
\pgflineto{\pgfpoint{414.841217pt}{256.492645pt}}
\pgfusepath{stroke}
\pgfpathmoveto{\pgfpoint{414.387878pt}{262.532928pt}}
\pgflineto{\pgfpoint{414.841217pt}{262.480408pt}}
\pgfusepath{stroke}
\pgfpathmoveto{\pgfpoint{414.375671pt}{268.487671pt}}
\pgflineto{\pgfpoint{414.841217pt}{268.468170pt}}
\pgfusepath{stroke}
\pgfpathmoveto{\pgfpoint{414.368835pt}{274.438843pt}}
\pgflineto{\pgfpoint{414.841217pt}{274.455902pt}}
\pgfusepath{stroke}
\pgfpathmoveto{\pgfpoint{414.368591pt}{280.388031pt}}
\pgflineto{\pgfpoint{414.841217pt}{280.443665pt}}
\pgfusepath{stroke}
\pgfpathmoveto{\pgfpoint{414.375549pt}{286.337158pt}}
\pgflineto{\pgfpoint{414.841217pt}{286.431427pt}}
\pgfusepath{stroke}
\pgfpathmoveto{\pgfpoint{414.389435pt}{292.288391pt}}
\pgflineto{\pgfpoint{414.841217pt}{292.419189pt}}
\pgfusepath{stroke}
\pgfpathmoveto{\pgfpoint{414.409210pt}{298.243561pt}}
\pgflineto{\pgfpoint{414.841217pt}{298.406921pt}}
\pgfusepath{stroke}
\pgfpathmoveto{\pgfpoint{414.433289pt}{304.203827pt}}
\pgflineto{\pgfpoint{414.841217pt}{304.394714pt}}
\pgfusepath{stroke}
\pgfpathmoveto{\pgfpoint{414.459961pt}{310.169739pt}}
\pgflineto{\pgfpoint{414.841217pt}{310.382446pt}}
\pgfusepath{stroke}
\pgfpathmoveto{\pgfpoint{414.487640pt}{316.141113pt}}
\pgflineto{\pgfpoint{414.841217pt}{316.370178pt}}
\pgfusepath{stroke}
\pgfpathmoveto{\pgfpoint{414.515045pt}{322.117493pt}}
\pgflineto{\pgfpoint{414.841217pt}{322.357971pt}}
\pgfusepath{stroke}
\pgfpathmoveto{\pgfpoint{414.541351pt}{328.098022pt}}
\pgflineto{\pgfpoint{414.841217pt}{328.345703pt}}
\pgfusepath{stroke}
\pgfpathmoveto{\pgfpoint{414.565948pt}{334.082001pt}}
\pgflineto{\pgfpoint{414.841217pt}{334.333466pt}}
\pgfusepath{stroke}
\pgfpathmoveto{\pgfpoint{414.588654pt}{340.068665pt}}
\pgflineto{\pgfpoint{414.841217pt}{340.321228pt}}
\pgfusepath{stroke}
\pgfpathmoveto{\pgfpoint{414.609344pt}{346.057312pt}}
\pgflineto{\pgfpoint{414.841217pt}{346.308960pt}}
\pgfusepath{stroke}
\pgfpathmoveto{\pgfpoint{414.628082pt}{352.047485pt}}
\pgflineto{\pgfpoint{414.841217pt}{352.296722pt}}
\pgfusepath{stroke}
\pgfpathmoveto{\pgfpoint{414.644958pt}{358.038727pt}}
\pgflineto{\pgfpoint{414.841217pt}{358.284485pt}}
\pgfusepath{stroke}
\pgfpathmoveto{\pgfpoint{414.660156pt}{364.030701pt}}
\pgflineto{\pgfpoint{414.841217pt}{364.272247pt}}
\pgfusepath{stroke}
\pgfpathmoveto{\pgfpoint{414.673828pt}{370.023163pt}}
\pgflineto{\pgfpoint{414.841217pt}{370.260010pt}}
\pgfusepath{stroke}
\pgfpathmoveto{\pgfpoint{420.736145pt}{77.025482pt}}
\pgflineto{\pgfpoint{420.828979pt}{76.859985pt}}
\pgfusepath{stroke}
\pgfpathmoveto{\pgfpoint{420.731384pt}{83.015884pt}}
\pgflineto{\pgfpoint{420.828979pt}{82.847733pt}}
\pgfusepath{stroke}
\pgfpathmoveto{\pgfpoint{420.726318pt}{89.006241pt}}
\pgflineto{\pgfpoint{420.828979pt}{88.835495pt}}
\pgfusepath{stroke}
\pgfpathmoveto{\pgfpoint{420.720947pt}{94.996574pt}}
\pgflineto{\pgfpoint{420.828979pt}{94.823257pt}}
\pgfusepath{stroke}
\pgfpathmoveto{\pgfpoint{420.715271pt}{100.986832pt}}
\pgflineto{\pgfpoint{420.828979pt}{100.811012pt}}
\pgfusepath{stroke}
\pgfpathmoveto{\pgfpoint{420.709198pt}{106.976997pt}}
\pgflineto{\pgfpoint{420.828979pt}{106.798759pt}}
\pgfusepath{stroke}
\pgfpathmoveto{\pgfpoint{420.702759pt}{112.967064pt}}
\pgflineto{\pgfpoint{420.828979pt}{112.786522pt}}
\pgfusepath{stroke}
\pgfpathmoveto{\pgfpoint{420.695923pt}{118.956985pt}}
\pgflineto{\pgfpoint{420.828979pt}{118.774277pt}}
\pgfusepath{stroke}
\pgfpathmoveto{\pgfpoint{420.688690pt}{124.946739pt}}
\pgflineto{\pgfpoint{420.828979pt}{124.762024pt}}
\pgfusepath{stroke}
\pgfpathmoveto{\pgfpoint{420.681030pt}{130.936279pt}}
\pgflineto{\pgfpoint{420.828979pt}{130.749786pt}}
\pgfusepath{stroke}
\pgfpathmoveto{\pgfpoint{420.672913pt}{136.925568pt}}
\pgflineto{\pgfpoint{420.828979pt}{136.737534pt}}
\pgfusepath{stroke}
\pgfpathmoveto{\pgfpoint{420.664368pt}{142.914581pt}}
\pgflineto{\pgfpoint{420.828979pt}{142.725281pt}}
\pgfusepath{stroke}
\pgfpathmoveto{\pgfpoint{420.655396pt}{148.903259pt}}
\pgflineto{\pgfpoint{420.828979pt}{148.713058pt}}
\pgfusepath{stroke}
\pgfpathmoveto{\pgfpoint{420.645935pt}{154.891571pt}}
\pgflineto{\pgfpoint{420.828979pt}{154.700806pt}}
\pgfusepath{stroke}
\pgfpathmoveto{\pgfpoint{420.636047pt}{160.879471pt}}
\pgflineto{\pgfpoint{420.828979pt}{160.688568pt}}
\pgfusepath{stroke}
\pgfpathmoveto{\pgfpoint{420.625671pt}{166.866882pt}}
\pgflineto{\pgfpoint{420.828979pt}{166.676315pt}}
\pgfusepath{stroke}
\pgfpathmoveto{\pgfpoint{420.614868pt}{172.853790pt}}
\pgflineto{\pgfpoint{420.828979pt}{172.664078pt}}
\pgfusepath{stroke}
\pgfpathmoveto{\pgfpoint{420.603577pt}{178.840134pt}}
\pgflineto{\pgfpoint{420.828979pt}{178.651825pt}}
\pgfusepath{stroke}
\pgfpathmoveto{\pgfpoint{420.591858pt}{184.825851pt}}
\pgflineto{\pgfpoint{420.828979pt}{184.639587pt}}
\pgfusepath{stroke}
\pgfpathmoveto{\pgfpoint{420.579620pt}{190.810913pt}}
\pgflineto{\pgfpoint{420.828979pt}{190.627335pt}}
\pgfusepath{stroke}
\pgfpathmoveto{\pgfpoint{420.566956pt}{196.795212pt}}
\pgflineto{\pgfpoint{420.828979pt}{196.615097pt}}
\pgfusepath{stroke}
\pgfpathmoveto{\pgfpoint{420.553711pt}{202.778687pt}}
\pgflineto{\pgfpoint{420.828979pt}{202.602844pt}}
\pgfusepath{stroke}
\pgfpathmoveto{\pgfpoint{420.539978pt}{208.761215pt}}
\pgflineto{\pgfpoint{420.828979pt}{208.590607pt}}
\pgfusepath{stroke}
\pgfpathmoveto{\pgfpoint{420.525635pt}{214.742645pt}}
\pgflineto{\pgfpoint{420.828979pt}{214.578354pt}}
\pgfusepath{stroke}
\pgfpathmoveto{\pgfpoint{420.510742pt}{220.722794pt}}
\pgflineto{\pgfpoint{420.828979pt}{220.566116pt}}
\pgfusepath{stroke}
\pgfpathmoveto{\pgfpoint{420.495239pt}{226.701385pt}}
\pgflineto{\pgfpoint{420.828979pt}{226.553864pt}}
\pgfusepath{stroke}
\pgfpathmoveto{\pgfpoint{420.479309pt}{232.678101pt}}
\pgflineto{\pgfpoint{420.828979pt}{232.541626pt}}
\pgfusepath{stroke}
\pgfpathmoveto{\pgfpoint{420.463013pt}{238.652573pt}}
\pgflineto{\pgfpoint{420.828979pt}{238.529388pt}}
\pgfusepath{stroke}
\pgfpathmoveto{\pgfpoint{420.446808pt}{244.624359pt}}
\pgflineto{\pgfpoint{420.828979pt}{244.517136pt}}
\pgfusepath{stroke}
\pgfpathmoveto{\pgfpoint{420.431152pt}{250.593094pt}}
\pgflineto{\pgfpoint{420.828979pt}{250.504883pt}}
\pgfusepath{stroke}
\pgfpathmoveto{\pgfpoint{420.416809pt}{256.558502pt}}
\pgflineto{\pgfpoint{420.828979pt}{256.492645pt}}
\pgfusepath{stroke}
\pgfpathmoveto{\pgfpoint{420.404694pt}{262.520538pt}}
\pgflineto{\pgfpoint{420.828979pt}{262.480408pt}}
\pgfusepath{stroke}
\pgfpathmoveto{\pgfpoint{420.395905pt}{268.479492pt}}
\pgflineto{\pgfpoint{420.828979pt}{268.468170pt}}
\pgfusepath{stroke}
\pgfpathmoveto{\pgfpoint{420.391418pt}{274.436066pt}}
\pgflineto{\pgfpoint{420.828979pt}{274.455902pt}}
\pgfusepath{stroke}
\pgfpathmoveto{\pgfpoint{420.392029pt}{280.391357pt}}
\pgflineto{\pgfpoint{420.828979pt}{280.443665pt}}
\pgfusepath{stroke}
\pgfpathmoveto{\pgfpoint{420.398041pt}{286.346741pt}}
\pgflineto{\pgfpoint{420.828979pt}{286.431427pt}}
\pgfusepath{stroke}
\pgfpathmoveto{\pgfpoint{420.409302pt}{292.303680pt}}
\pgflineto{\pgfpoint{420.828979pt}{292.419189pt}}
\pgfusepath{stroke}
\pgfpathmoveto{\pgfpoint{420.425110pt}{298.263428pt}}
\pgflineto{\pgfpoint{420.828979pt}{298.406921pt}}
\pgfusepath{stroke}
\pgfpathmoveto{\pgfpoint{420.444489pt}{304.226929pt}}
\pgflineto{\pgfpoint{420.828979pt}{304.394714pt}}
\pgfusepath{stroke}
\pgfpathmoveto{\pgfpoint{420.466248pt}{310.194580pt}}
\pgflineto{\pgfpoint{420.828979pt}{310.382446pt}}
\pgfusepath{stroke}
\pgfpathmoveto{\pgfpoint{420.489197pt}{316.166534pt}}
\pgflineto{\pgfpoint{420.828979pt}{316.370178pt}}
\pgfusepath{stroke}
\pgfpathmoveto{\pgfpoint{420.512451pt}{322.142456pt}}
\pgflineto{\pgfpoint{420.828979pt}{322.357971pt}}
\pgfusepath{stroke}
\pgfpathmoveto{\pgfpoint{420.535156pt}{328.121918pt}}
\pgflineto{\pgfpoint{420.828979pt}{328.345703pt}}
\pgfusepath{stroke}
\pgfpathmoveto{\pgfpoint{420.556885pt}{334.104401pt}}
\pgflineto{\pgfpoint{420.828979pt}{334.333466pt}}
\pgfusepath{stroke}
\pgfpathmoveto{\pgfpoint{420.577332pt}{340.089355pt}}
\pgflineto{\pgfpoint{420.828979pt}{340.321228pt}}
\pgfusepath{stroke}
\pgfpathmoveto{\pgfpoint{420.596252pt}{346.076294pt}}
\pgflineto{\pgfpoint{420.828979pt}{346.308960pt}}
\pgfusepath{stroke}
\pgfpathmoveto{\pgfpoint{420.613678pt}{352.064728pt}}
\pgflineto{\pgfpoint{420.828979pt}{352.296722pt}}
\pgfusepath{stroke}
\pgfpathmoveto{\pgfpoint{420.629608pt}{358.054352pt}}
\pgflineto{\pgfpoint{420.828979pt}{358.284485pt}}
\pgfusepath{stroke}
\pgfpathmoveto{\pgfpoint{420.644135pt}{364.044830pt}}
\pgflineto{\pgfpoint{420.828979pt}{364.272247pt}}
\pgfusepath{stroke}
\pgfpathmoveto{\pgfpoint{420.657349pt}{370.035950pt}}
\pgflineto{\pgfpoint{420.828979pt}{370.260010pt}}
\pgfusepath{stroke}
\pgfpathmoveto{\pgfpoint{426.721375pt}{77.020905pt}}
\pgflineto{\pgfpoint{426.816711pt}{76.859985pt}}
\pgfusepath{stroke}
\pgfpathmoveto{\pgfpoint{426.716644pt}{83.011002pt}}
\pgflineto{\pgfpoint{426.816711pt}{82.847733pt}}
\pgfusepath{stroke}
\pgfpathmoveto{\pgfpoint{426.711670pt}{89.001060pt}}
\pgflineto{\pgfpoint{426.816711pt}{88.835495pt}}
\pgfusepath{stroke}
\pgfpathmoveto{\pgfpoint{426.706360pt}{94.991074pt}}
\pgflineto{\pgfpoint{426.816711pt}{94.823257pt}}
\pgfusepath{stroke}
\pgfpathmoveto{\pgfpoint{426.700745pt}{100.980995pt}}
\pgflineto{\pgfpoint{426.816711pt}{100.811012pt}}
\pgfusepath{stroke}
\pgfpathmoveto{\pgfpoint{426.694763pt}{106.970818pt}}
\pgflineto{\pgfpoint{426.816711pt}{106.798759pt}}
\pgfusepath{stroke}
\pgfpathmoveto{\pgfpoint{426.688446pt}{112.960510pt}}
\pgflineto{\pgfpoint{426.816711pt}{112.786522pt}}
\pgfusepath{stroke}
\pgfpathmoveto{\pgfpoint{426.681763pt}{118.950043pt}}
\pgflineto{\pgfpoint{426.816711pt}{118.774277pt}}
\pgfusepath{stroke}
\pgfpathmoveto{\pgfpoint{426.674744pt}{124.939392pt}}
\pgflineto{\pgfpoint{426.816711pt}{124.762024pt}}
\pgfusepath{stroke}
\pgfpathmoveto{\pgfpoint{426.667297pt}{130.928528pt}}
\pgflineto{\pgfpoint{426.816711pt}{130.749786pt}}
\pgfusepath{stroke}
\pgfpathmoveto{\pgfpoint{426.659485pt}{136.917404pt}}
\pgflineto{\pgfpoint{426.816711pt}{136.737534pt}}
\pgfusepath{stroke}
\pgfpathmoveto{\pgfpoint{426.651245pt}{142.905975pt}}
\pgflineto{\pgfpoint{426.816711pt}{142.725281pt}}
\pgfusepath{stroke}
\pgfpathmoveto{\pgfpoint{426.642639pt}{148.894226pt}}
\pgflineto{\pgfpoint{426.816711pt}{148.713058pt}}
\pgfusepath{stroke}
\pgfpathmoveto{\pgfpoint{426.633575pt}{154.882111pt}}
\pgflineto{\pgfpoint{426.816711pt}{154.700806pt}}
\pgfusepath{stroke}
\pgfpathmoveto{\pgfpoint{426.624115pt}{160.869568pt}}
\pgflineto{\pgfpoint{426.816711pt}{160.688568pt}}
\pgfusepath{stroke}
\pgfpathmoveto{\pgfpoint{426.614258pt}{166.856567pt}}
\pgflineto{\pgfpoint{426.816711pt}{166.676315pt}}
\pgfusepath{stroke}
\pgfpathmoveto{\pgfpoint{426.603973pt}{172.843018pt}}
\pgflineto{\pgfpoint{426.816711pt}{172.664078pt}}
\pgfusepath{stroke}
\pgfpathmoveto{\pgfpoint{426.593262pt}{178.828934pt}}
\pgflineto{\pgfpoint{426.816711pt}{178.651825pt}}
\pgfusepath{stroke}
\pgfpathmoveto{\pgfpoint{426.582184pt}{184.814224pt}}
\pgflineto{\pgfpoint{426.816711pt}{184.639587pt}}
\pgfusepath{stroke}
\pgfpathmoveto{\pgfpoint{426.570679pt}{190.798828pt}}
\pgflineto{\pgfpoint{426.816711pt}{190.627335pt}}
\pgfusepath{stroke}
\pgfpathmoveto{\pgfpoint{426.558777pt}{196.782669pt}}
\pgflineto{\pgfpoint{426.816711pt}{196.615097pt}}
\pgfusepath{stroke}
\pgfpathmoveto{\pgfpoint{426.546448pt}{202.765671pt}}
\pgflineto{\pgfpoint{426.816711pt}{202.602844pt}}
\pgfusepath{stroke}
\pgfpathmoveto{\pgfpoint{426.533722pt}{208.747726pt}}
\pgflineto{\pgfpoint{426.816711pt}{208.590607pt}}
\pgfusepath{stroke}
\pgfpathmoveto{\pgfpoint{426.520569pt}{214.728668pt}}
\pgflineto{\pgfpoint{426.816711pt}{214.578354pt}}
\pgfusepath{stroke}
\pgfpathmoveto{\pgfpoint{426.507080pt}{220.708359pt}}
\pgflineto{\pgfpoint{426.816711pt}{220.566116pt}}
\pgfusepath{stroke}
\pgfpathmoveto{\pgfpoint{426.493225pt}{226.686554pt}}
\pgflineto{\pgfpoint{426.816711pt}{226.553864pt}}
\pgfusepath{stroke}
\pgfpathmoveto{\pgfpoint{426.479248pt}{232.663040pt}}
\pgflineto{\pgfpoint{426.816711pt}{232.541626pt}}
\pgfusepath{stroke}
\pgfpathmoveto{\pgfpoint{426.465271pt}{238.637512pt}}
\pgflineto{\pgfpoint{426.816711pt}{238.529388pt}}
\pgfusepath{stroke}
\pgfpathmoveto{\pgfpoint{426.451660pt}{244.609741pt}}
\pgflineto{\pgfpoint{426.816711pt}{244.517136pt}}
\pgfusepath{stroke}
\pgfpathmoveto{\pgfpoint{426.438904pt}{250.579483pt}}
\pgflineto{\pgfpoint{426.816711pt}{250.504883pt}}
\pgfusepath{stroke}
\pgfpathmoveto{\pgfpoint{426.427490pt}{256.546661pt}}
\pgflineto{\pgfpoint{426.816711pt}{256.492645pt}}
\pgfusepath{stroke}
\pgfpathmoveto{\pgfpoint{426.418213pt}{262.511322pt}}
\pgflineto{\pgfpoint{426.816711pt}{262.480408pt}}
\pgfusepath{stroke}
\pgfpathmoveto{\pgfpoint{426.411804pt}{268.473755pt}}
\pgflineto{\pgfpoint{426.816711pt}{268.468170pt}}
\pgfusepath{stroke}
\pgfpathmoveto{\pgfpoint{426.408875pt}{274.434570pt}}
\pgflineto{\pgfpoint{426.816711pt}{274.455902pt}}
\pgfusepath{stroke}
\pgfpathmoveto{\pgfpoint{426.409912pt}{280.394592pt}}
\pgflineto{\pgfpoint{426.816711pt}{280.443665pt}}
\pgfusepath{stroke}
\pgfpathmoveto{\pgfpoint{426.415161pt}{286.354736pt}}
\pgflineto{\pgfpoint{426.816711pt}{286.431427pt}}
\pgfusepath{stroke}
\pgfpathmoveto{\pgfpoint{426.424500pt}{292.316101pt}}
\pgflineto{\pgfpoint{426.816711pt}{292.419189pt}}
\pgfusepath{stroke}
\pgfpathmoveto{\pgfpoint{426.437469pt}{298.279541pt}}
\pgflineto{\pgfpoint{426.816711pt}{298.406921pt}}
\pgfusepath{stroke}
\pgfpathmoveto{\pgfpoint{426.453308pt}{304.245758pt}}
\pgflineto{\pgfpoint{426.816711pt}{304.394714pt}}
\pgfusepath{stroke}
\pgfpathmoveto{\pgfpoint{426.471313pt}{310.215149pt}}
\pgflineto{\pgfpoint{426.816711pt}{310.382446pt}}
\pgfusepath{stroke}
\pgfpathmoveto{\pgfpoint{426.490570pt}{316.187897pt}}
\pgflineto{\pgfpoint{426.816711pt}{316.370178pt}}
\pgfusepath{stroke}
\pgfpathmoveto{\pgfpoint{426.510345pt}{322.163849pt}}
\pgflineto{\pgfpoint{426.816711pt}{322.357971pt}}
\pgfusepath{stroke}
\pgfpathmoveto{\pgfpoint{426.530029pt}{328.142761pt}}
\pgflineto{\pgfpoint{426.816711pt}{328.345703pt}}
\pgfusepath{stroke}
\pgfpathmoveto{\pgfpoint{426.549194pt}{334.124329pt}}
\pgflineto{\pgfpoint{426.816711pt}{334.333466pt}}
\pgfusepath{stroke}
\pgfpathmoveto{\pgfpoint{426.567505pt}{340.108063pt}}
\pgflineto{\pgfpoint{426.816711pt}{340.321228pt}}
\pgfusepath{stroke}
\pgfpathmoveto{\pgfpoint{426.584717pt}{346.093689pt}}
\pgflineto{\pgfpoint{426.816711pt}{346.308960pt}}
\pgfusepath{stroke}
\pgfpathmoveto{\pgfpoint{426.600800pt}{352.080811pt}}
\pgflineto{\pgfpoint{426.816711pt}{352.296722pt}}
\pgfusepath{stroke}
\pgfpathmoveto{\pgfpoint{426.615723pt}{358.069092pt}}
\pgflineto{\pgfpoint{426.816711pt}{358.284485pt}}
\pgfusepath{stroke}
\pgfpathmoveto{\pgfpoint{426.629425pt}{364.058350pt}}
\pgflineto{\pgfpoint{426.816711pt}{364.272247pt}}
\pgfusepath{stroke}
\pgfpathmoveto{\pgfpoint{426.642059pt}{370.048279pt}}
\pgflineto{\pgfpoint{426.816711pt}{370.260010pt}}
\pgfusepath{stroke}
\pgfpathmoveto{\pgfpoint{432.706879pt}{77.016327pt}}
\pgflineto{\pgfpoint{432.804474pt}{76.859985pt}}
\pgfusepath{stroke}
\pgfpathmoveto{\pgfpoint{432.702209pt}{83.006165pt}}
\pgflineto{\pgfpoint{432.804474pt}{82.847733pt}}
\pgfusepath{stroke}
\pgfpathmoveto{\pgfpoint{432.697296pt}{88.995956pt}}
\pgflineto{\pgfpoint{432.804474pt}{88.835495pt}}
\pgfusepath{stroke}
\pgfpathmoveto{\pgfpoint{432.692017pt}{94.985641pt}}
\pgflineto{\pgfpoint{432.804474pt}{94.823257pt}}
\pgfusepath{stroke}
\pgfpathmoveto{\pgfpoint{432.686523pt}{100.975258pt}}
\pgflineto{\pgfpoint{432.804474pt}{100.811012pt}}
\pgfusepath{stroke}
\pgfpathmoveto{\pgfpoint{432.680664pt}{106.964752pt}}
\pgflineto{\pgfpoint{432.804474pt}{106.798759pt}}
\pgfusepath{stroke}
\pgfpathmoveto{\pgfpoint{432.674530pt}{112.954094pt}}
\pgflineto{\pgfpoint{432.804474pt}{112.786522pt}}
\pgfusepath{stroke}
\pgfpathmoveto{\pgfpoint{432.668030pt}{118.943275pt}}
\pgflineto{\pgfpoint{432.804474pt}{118.774277pt}}
\pgfusepath{stroke}
\pgfpathmoveto{\pgfpoint{432.661194pt}{124.932274pt}}
\pgflineto{\pgfpoint{432.804474pt}{124.762024pt}}
\pgfusepath{stroke}
\pgfpathmoveto{\pgfpoint{432.653992pt}{130.921021pt}}
\pgflineto{\pgfpoint{432.804474pt}{130.749786pt}}
\pgfusepath{stroke}
\pgfpathmoveto{\pgfpoint{432.646423pt}{136.909515pt}}
\pgflineto{\pgfpoint{432.804474pt}{136.737534pt}}
\pgfusepath{stroke}
\pgfpathmoveto{\pgfpoint{432.638550pt}{142.897720pt}}
\pgflineto{\pgfpoint{432.804474pt}{142.725281pt}}
\pgfusepath{stroke}
\pgfpathmoveto{\pgfpoint{432.630249pt}{148.885590pt}}
\pgflineto{\pgfpoint{432.804474pt}{148.713058pt}}
\pgfusepath{stroke}
\pgfpathmoveto{\pgfpoint{432.621582pt}{154.873077pt}}
\pgflineto{\pgfpoint{432.804474pt}{154.700806pt}}
\pgfusepath{stroke}
\pgfpathmoveto{\pgfpoint{432.612610pt}{160.860138pt}}
\pgflineto{\pgfpoint{432.804474pt}{160.688568pt}}
\pgfusepath{stroke}
\pgfpathmoveto{\pgfpoint{432.603210pt}{166.846741pt}}
\pgflineto{\pgfpoint{432.804474pt}{166.676315pt}}
\pgfusepath{stroke}
\pgfpathmoveto{\pgfpoint{432.593475pt}{172.832825pt}}
\pgflineto{\pgfpoint{432.804474pt}{172.664078pt}}
\pgfusepath{stroke}
\pgfpathmoveto{\pgfpoint{432.583374pt}{178.818359pt}}
\pgflineto{\pgfpoint{432.804474pt}{178.651825pt}}
\pgfusepath{stroke}
\pgfpathmoveto{\pgfpoint{432.572937pt}{184.803268pt}}
\pgflineto{\pgfpoint{432.804474pt}{184.639587pt}}
\pgfusepath{stroke}
\pgfpathmoveto{\pgfpoint{432.562164pt}{190.787506pt}}
\pgflineto{\pgfpoint{432.804474pt}{190.627335pt}}
\pgfusepath{stroke}
\pgfpathmoveto{\pgfpoint{432.551025pt}{196.770981pt}}
\pgflineto{\pgfpoint{432.804474pt}{196.615097pt}}
\pgfusepath{stroke}
\pgfpathmoveto{\pgfpoint{432.539612pt}{202.753616pt}}
\pgflineto{\pgfpoint{432.804474pt}{202.602844pt}}
\pgfusepath{stroke}
\pgfpathmoveto{\pgfpoint{432.527893pt}{208.735321pt}}
\pgflineto{\pgfpoint{432.804474pt}{208.590607pt}}
\pgfusepath{stroke}
\pgfpathmoveto{\pgfpoint{432.515930pt}{214.715958pt}}
\pgflineto{\pgfpoint{432.804474pt}{214.578354pt}}
\pgfusepath{stroke}
\pgfpathmoveto{\pgfpoint{432.503723pt}{220.695389pt}}
\pgflineto{\pgfpoint{432.804474pt}{220.566116pt}}
\pgfusepath{stroke}
\pgfpathmoveto{\pgfpoint{432.491425pt}{226.673447pt}}
\pgflineto{\pgfpoint{432.804474pt}{226.553864pt}}
\pgfusepath{stroke}
\pgfpathmoveto{\pgfpoint{432.479187pt}{232.649963pt}}
\pgflineto{\pgfpoint{432.804474pt}{232.541626pt}}
\pgfusepath{stroke}
\pgfpathmoveto{\pgfpoint{432.467163pt}{238.624725pt}}
\pgflineto{\pgfpoint{432.804474pt}{238.529388pt}}
\pgfusepath{stroke}
\pgfpathmoveto{\pgfpoint{432.455719pt}{244.597610pt}}
\pgflineto{\pgfpoint{432.804474pt}{244.517136pt}}
\pgfusepath{stroke}
\pgfpathmoveto{\pgfpoint{432.445221pt}{250.568497pt}}
\pgflineto{\pgfpoint{432.804474pt}{250.504883pt}}
\pgfusepath{stroke}
\pgfpathmoveto{\pgfpoint{432.436127pt}{256.537354pt}}
\pgflineto{\pgfpoint{432.804474pt}{256.492645pt}}
\pgfusepath{stroke}
\pgfpathmoveto{\pgfpoint{432.428894pt}{262.504333pt}}
\pgflineto{\pgfpoint{432.804474pt}{262.480408pt}}
\pgfusepath{stroke}
\pgfpathmoveto{\pgfpoint{432.424133pt}{268.469696pt}}
\pgflineto{\pgfpoint{432.804474pt}{268.468170pt}}
\pgfusepath{stroke}
\pgfpathmoveto{\pgfpoint{432.422241pt}{274.433929pt}}
\pgflineto{\pgfpoint{432.804474pt}{274.455902pt}}
\pgfusepath{stroke}
\pgfpathmoveto{\pgfpoint{432.423553pt}{280.397644pt}}
\pgflineto{\pgfpoint{432.804474pt}{280.443665pt}}
\pgfusepath{stroke}
\pgfpathmoveto{\pgfpoint{432.428192pt}{286.361542pt}}
\pgflineto{\pgfpoint{432.804474pt}{286.431427pt}}
\pgfusepath{stroke}
\pgfpathmoveto{\pgfpoint{432.436096pt}{292.326355pt}}
\pgflineto{\pgfpoint{432.804474pt}{292.419189pt}}
\pgfusepath{stroke}
\pgfpathmoveto{\pgfpoint{432.446838pt}{298.292786pt}}
\pgflineto{\pgfpoint{432.804474pt}{298.406921pt}}
\pgfusepath{stroke}
\pgfpathmoveto{\pgfpoint{432.460083pt}{304.261353pt}}
\pgflineto{\pgfpoint{432.804474pt}{304.394714pt}}
\pgfusepath{stroke}
\pgfpathmoveto{\pgfpoint{432.475128pt}{310.232361pt}}
\pgflineto{\pgfpoint{432.804474pt}{310.382446pt}}
\pgfusepath{stroke}
\pgfpathmoveto{\pgfpoint{432.491394pt}{316.205994pt}}
\pgflineto{\pgfpoint{432.804474pt}{316.370178pt}}
\pgfusepath{stroke}
\pgfpathmoveto{\pgfpoint{432.508362pt}{322.182281pt}}
\pgflineto{\pgfpoint{432.804474pt}{322.357971pt}}
\pgfusepath{stroke}
\pgfpathmoveto{\pgfpoint{432.525452pt}{328.161011pt}}
\pgflineto{\pgfpoint{432.804474pt}{328.345703pt}}
\pgfusepath{stroke}
\pgfpathmoveto{\pgfpoint{432.542358pt}{334.141998pt}}
\pgflineto{\pgfpoint{432.804474pt}{334.333466pt}}
\pgfusepath{stroke}
\pgfpathmoveto{\pgfpoint{432.558716pt}{340.124969pt}}
\pgflineto{\pgfpoint{432.804474pt}{340.321228pt}}
\pgfusepath{stroke}
\pgfpathmoveto{\pgfpoint{432.574341pt}{346.109619pt}}
\pgflineto{\pgfpoint{432.804474pt}{346.308960pt}}
\pgfusepath{stroke}
\pgfpathmoveto{\pgfpoint{432.589111pt}{352.095703pt}}
\pgflineto{\pgfpoint{432.804474pt}{352.296722pt}}
\pgfusepath{stroke}
\pgfpathmoveto{\pgfpoint{432.602936pt}{358.082947pt}}
\pgflineto{\pgfpoint{432.804474pt}{358.284485pt}}
\pgfusepath{stroke}
\pgfpathmoveto{\pgfpoint{432.615845pt}{364.071167pt}}
\pgflineto{\pgfpoint{432.804474pt}{364.272247pt}}
\pgfusepath{stroke}
\pgfpathmoveto{\pgfpoint{432.627838pt}{370.060089pt}}
\pgflineto{\pgfpoint{432.804474pt}{370.260010pt}}
\pgfusepath{stroke}
\pgfpathmoveto{\pgfpoint{438.692657pt}{77.011810pt}}
\pgflineto{\pgfpoint{438.792236pt}{76.859985pt}}
\pgfusepath{stroke}
\pgfpathmoveto{\pgfpoint{438.688049pt}{83.001389pt}}
\pgflineto{\pgfpoint{438.792236pt}{82.847733pt}}
\pgfusepath{stroke}
\pgfpathmoveto{\pgfpoint{438.683197pt}{88.990906pt}}
\pgflineto{\pgfpoint{438.792236pt}{88.835495pt}}
\pgfusepath{stroke}
\pgfpathmoveto{\pgfpoint{438.678040pt}{94.980324pt}}
\pgflineto{\pgfpoint{438.792236pt}{94.823257pt}}
\pgfusepath{stroke}
\pgfpathmoveto{\pgfpoint{438.672638pt}{100.969627pt}}
\pgflineto{\pgfpoint{438.792236pt}{100.811012pt}}
\pgfusepath{stroke}
\pgfpathmoveto{\pgfpoint{438.666931pt}{106.958832pt}}
\pgflineto{\pgfpoint{438.792236pt}{106.798759pt}}
\pgfusepath{stroke}
\pgfpathmoveto{\pgfpoint{438.660919pt}{112.947861pt}}
\pgflineto{\pgfpoint{438.792236pt}{112.786522pt}}
\pgfusepath{stroke}
\pgfpathmoveto{\pgfpoint{438.654633pt}{118.936722pt}}
\pgflineto{\pgfpoint{438.792236pt}{118.774277pt}}
\pgfusepath{stroke}
\pgfpathmoveto{\pgfpoint{438.648010pt}{124.925369pt}}
\pgflineto{\pgfpoint{438.792236pt}{124.762024pt}}
\pgfusepath{stroke}
\pgfpathmoveto{\pgfpoint{438.641052pt}{130.913788pt}}
\pgflineto{\pgfpoint{438.792236pt}{130.749786pt}}
\pgfusepath{stroke}
\pgfpathmoveto{\pgfpoint{438.633789pt}{136.901947pt}}
\pgflineto{\pgfpoint{438.792236pt}{136.737534pt}}
\pgfusepath{stroke}
\pgfpathmoveto{\pgfpoint{438.626190pt}{142.889801pt}}
\pgflineto{\pgfpoint{438.792236pt}{142.725281pt}}
\pgfusepath{stroke}
\pgfpathmoveto{\pgfpoint{438.618286pt}{148.877319pt}}
\pgflineto{\pgfpoint{438.792236pt}{148.713058pt}}
\pgfusepath{stroke}
\pgfpathmoveto{\pgfpoint{438.610016pt}{154.864441pt}}
\pgflineto{\pgfpoint{438.792236pt}{154.700806pt}}
\pgfusepath{stroke}
\pgfpathmoveto{\pgfpoint{438.601440pt}{160.851166pt}}
\pgflineto{\pgfpoint{438.792236pt}{160.688568pt}}
\pgfusepath{stroke}
\pgfpathmoveto{\pgfpoint{438.592529pt}{166.837433pt}}
\pgflineto{\pgfpoint{438.792236pt}{166.676315pt}}
\pgfusepath{stroke}
\pgfpathmoveto{\pgfpoint{438.583313pt}{172.823196pt}}
\pgflineto{\pgfpoint{438.792236pt}{172.664078pt}}
\pgfusepath{stroke}
\pgfpathmoveto{\pgfpoint{438.573792pt}{178.808395pt}}
\pgflineto{\pgfpoint{438.792236pt}{178.651825pt}}
\pgfusepath{stroke}
\pgfpathmoveto{\pgfpoint{438.564026pt}{184.792999pt}}
\pgflineto{\pgfpoint{438.792236pt}{184.639587pt}}
\pgfusepath{stroke}
\pgfpathmoveto{\pgfpoint{438.553955pt}{190.776932pt}}
\pgflineto{\pgfpoint{438.792236pt}{190.627335pt}}
\pgfusepath{stroke}
\pgfpathmoveto{\pgfpoint{438.543640pt}{196.760117pt}}
\pgflineto{\pgfpoint{438.792236pt}{196.615097pt}}
\pgfusepath{stroke}
\pgfpathmoveto{\pgfpoint{438.533051pt}{202.742508pt}}
\pgflineto{\pgfpoint{438.792236pt}{202.602844pt}}
\pgfusepath{stroke}
\pgfpathmoveto{\pgfpoint{438.522308pt}{208.723984pt}}
\pgflineto{\pgfpoint{438.792236pt}{208.590607pt}}
\pgfusepath{stroke}
\pgfpathmoveto{\pgfpoint{438.511414pt}{214.704453pt}}
\pgflineto{\pgfpoint{438.792236pt}{214.578354pt}}
\pgfusepath{stroke}
\pgfpathmoveto{\pgfpoint{438.500488pt}{220.683792pt}}
\pgflineto{\pgfpoint{438.792236pt}{220.566116pt}}
\pgfusepath{stroke}
\pgfpathmoveto{\pgfpoint{438.489563pt}{226.661896pt}}
\pgflineto{\pgfpoint{438.792236pt}{226.553864pt}}
\pgfusepath{stroke}
\pgfpathmoveto{\pgfpoint{438.478851pt}{232.638626pt}}
\pgflineto{\pgfpoint{438.792236pt}{232.541626pt}}
\pgfusepath{stroke}
\pgfpathmoveto{\pgfpoint{438.468536pt}{238.613861pt}}
\pgflineto{\pgfpoint{438.792236pt}{238.529388pt}}
\pgfusepath{stroke}
\pgfpathmoveto{\pgfpoint{438.458862pt}{244.587509pt}}
\pgflineto{\pgfpoint{438.792236pt}{244.517136pt}}
\pgfusepath{stroke}
\pgfpathmoveto{\pgfpoint{438.450195pt}{250.559540pt}}
\pgflineto{\pgfpoint{438.792236pt}{250.504883pt}}
\pgfusepath{stroke}
\pgfpathmoveto{\pgfpoint{438.442810pt}{256.529999pt}}
\pgflineto{\pgfpoint{438.792236pt}{256.492645pt}}
\pgfusepath{stroke}
\pgfpathmoveto{\pgfpoint{438.437164pt}{262.498993pt}}
\pgflineto{\pgfpoint{438.792236pt}{262.480408pt}}
\pgfusepath{stroke}
\pgfpathmoveto{\pgfpoint{438.433594pt}{268.466827pt}}
\pgflineto{\pgfpoint{438.792236pt}{268.468170pt}}
\pgfusepath{stroke}
\pgfpathmoveto{\pgfpoint{438.432373pt}{274.433838pt}}
\pgflineto{\pgfpoint{438.792236pt}{274.455902pt}}
\pgfusepath{stroke}
\pgfpathmoveto{\pgfpoint{438.433838pt}{280.400513pt}}
\pgflineto{\pgfpoint{438.792236pt}{280.443665pt}}
\pgfusepath{stroke}
\pgfpathmoveto{\pgfpoint{438.437958pt}{286.367401pt}}
\pgflineto{\pgfpoint{438.792236pt}{286.431427pt}}
\pgfusepath{stroke}
\pgfpathmoveto{\pgfpoint{438.444702pt}{292.335022pt}}
\pgflineto{\pgfpoint{438.792236pt}{292.419189pt}}
\pgfusepath{stroke}
\pgfpathmoveto{\pgfpoint{438.453796pt}{298.303894pt}}
\pgflineto{\pgfpoint{438.792236pt}{298.406921pt}}
\pgfusepath{stroke}
\pgfpathmoveto{\pgfpoint{438.464935pt}{304.274445pt}}
\pgflineto{\pgfpoint{438.792236pt}{304.394714pt}}
\pgfusepath{stroke}
\pgfpathmoveto{\pgfpoint{438.477722pt}{310.246918pt}}
\pgflineto{\pgfpoint{438.792236pt}{310.382446pt}}
\pgfusepath{stroke}
\pgfpathmoveto{\pgfpoint{438.491608pt}{316.221497pt}}
\pgflineto{\pgfpoint{438.792236pt}{316.370178pt}}
\pgfusepath{stroke}
\pgfpathmoveto{\pgfpoint{438.506226pt}{322.198212pt}}
\pgflineto{\pgfpoint{438.792236pt}{322.357971pt}}
\pgfusepath{stroke}
\pgfpathmoveto{\pgfpoint{438.521179pt}{328.177002pt}}
\pgflineto{\pgfpoint{438.792236pt}{328.345703pt}}
\pgfusepath{stroke}
\pgfpathmoveto{\pgfpoint{438.536072pt}{334.157715pt}}
\pgflineto{\pgfpoint{438.792236pt}{334.333466pt}}
\pgfusepath{stroke}
\pgfpathmoveto{\pgfpoint{438.550720pt}{340.140167pt}}
\pgflineto{\pgfpoint{438.792236pt}{340.321228pt}}
\pgfusepath{stroke}
\pgfpathmoveto{\pgfpoint{438.564819pt}{346.124146pt}}
\pgflineto{\pgfpoint{438.792236pt}{346.308960pt}}
\pgfusepath{stroke}
\pgfpathmoveto{\pgfpoint{438.578339pt}{352.109436pt}}
\pgflineto{\pgfpoint{438.792236pt}{352.296722pt}}
\pgfusepath{stroke}
\pgfpathmoveto{\pgfpoint{438.591125pt}{358.095856pt}}
\pgflineto{\pgfpoint{438.792236pt}{358.284485pt}}
\pgfusepath{stroke}
\pgfpathmoveto{\pgfpoint{438.603210pt}{364.083191pt}}
\pgflineto{\pgfpoint{438.792236pt}{364.272247pt}}
\pgfusepath{stroke}
\pgfpathmoveto{\pgfpoint{438.614502pt}{370.071320pt}}
\pgflineto{\pgfpoint{438.792236pt}{370.260010pt}}
\pgfusepath{stroke}
\pgfpathmoveto{\pgfpoint{444.678650pt}{77.007370pt}}
\pgflineto{\pgfpoint{444.779968pt}{76.859985pt}}
\pgfusepath{stroke}
\pgfpathmoveto{\pgfpoint{444.674133pt}{82.996704pt}}
\pgflineto{\pgfpoint{444.779968pt}{82.847733pt}}
\pgfusepath{stroke}
\pgfpathmoveto{\pgfpoint{444.669342pt}{88.985962pt}}
\pgflineto{\pgfpoint{444.779968pt}{88.835495pt}}
\pgfusepath{stroke}
\pgfpathmoveto{\pgfpoint{444.664307pt}{94.975113pt}}
\pgflineto{\pgfpoint{444.779968pt}{94.823257pt}}
\pgfusepath{stroke}
\pgfpathmoveto{\pgfpoint{444.659027pt}{100.964142pt}}
\pgflineto{\pgfpoint{444.779968pt}{100.811012pt}}
\pgfusepath{stroke}
\pgfpathmoveto{\pgfpoint{444.653473pt}{106.953064pt}}
\pgflineto{\pgfpoint{444.779968pt}{106.798759pt}}
\pgfusepath{stroke}
\pgfpathmoveto{\pgfpoint{444.647644pt}{112.941803pt}}
\pgflineto{\pgfpoint{444.779968pt}{112.786522pt}}
\pgfusepath{stroke}
\pgfpathmoveto{\pgfpoint{444.641541pt}{118.930359pt}}
\pgflineto{\pgfpoint{444.779968pt}{118.774277pt}}
\pgfusepath{stroke}
\pgfpathmoveto{\pgfpoint{444.635132pt}{124.918709pt}}
\pgflineto{\pgfpoint{444.779968pt}{124.762024pt}}
\pgfusepath{stroke}
\pgfpathmoveto{\pgfpoint{444.628418pt}{130.906815pt}}
\pgflineto{\pgfpoint{444.779968pt}{130.749786pt}}
\pgfusepath{stroke}
\pgfpathmoveto{\pgfpoint{444.621460pt}{136.894653pt}}
\pgflineto{\pgfpoint{444.779968pt}{136.737534pt}}
\pgfusepath{stroke}
\pgfpathmoveto{\pgfpoint{444.614197pt}{142.882202pt}}
\pgflineto{\pgfpoint{444.779968pt}{142.725281pt}}
\pgfusepath{stroke}
\pgfpathmoveto{\pgfpoint{444.606567pt}{148.869415pt}}
\pgflineto{\pgfpoint{444.779968pt}{148.713058pt}}
\pgfusepath{stroke}
\pgfpathmoveto{\pgfpoint{444.598724pt}{154.856232pt}}
\pgflineto{\pgfpoint{444.779968pt}{154.700806pt}}
\pgfusepath{stroke}
\pgfpathmoveto{\pgfpoint{444.590576pt}{160.842651pt}}
\pgflineto{\pgfpoint{444.779968pt}{160.688568pt}}
\pgfusepath{stroke}
\pgfpathmoveto{\pgfpoint{444.582153pt}{166.828629pt}}
\pgflineto{\pgfpoint{444.779968pt}{166.676315pt}}
\pgfusepath{stroke}
\pgfpathmoveto{\pgfpoint{444.573486pt}{172.814102pt}}
\pgflineto{\pgfpoint{444.779968pt}{172.664078pt}}
\pgfusepath{stroke}
\pgfpathmoveto{\pgfpoint{444.564545pt}{178.799042pt}}
\pgflineto{\pgfpoint{444.779968pt}{178.651825pt}}
\pgfusepath{stroke}
\pgfpathmoveto{\pgfpoint{444.555389pt}{184.783386pt}}
\pgflineto{\pgfpoint{444.779968pt}{184.639587pt}}
\pgfusepath{stroke}
\pgfpathmoveto{\pgfpoint{444.546021pt}{190.767090pt}}
\pgflineto{\pgfpoint{444.779968pt}{190.627335pt}}
\pgfusepath{stroke}
\pgfpathmoveto{\pgfpoint{444.536438pt}{196.750076pt}}
\pgflineto{\pgfpoint{444.779968pt}{196.615097pt}}
\pgfusepath{stroke}
\pgfpathmoveto{\pgfpoint{444.526733pt}{202.732285pt}}
\pgflineto{\pgfpoint{444.779968pt}{202.602844pt}}
\pgfusepath{stroke}
\pgfpathmoveto{\pgfpoint{444.516876pt}{208.713638pt}}
\pgflineto{\pgfpoint{444.779968pt}{208.590607pt}}
\pgfusepath{stroke}
\pgfpathmoveto{\pgfpoint{444.507019pt}{214.694061pt}}
\pgflineto{\pgfpoint{444.779968pt}{214.578354pt}}
\pgfusepath{stroke}
\pgfpathmoveto{\pgfpoint{444.497192pt}{220.673447pt}}
\pgflineto{\pgfpoint{444.779968pt}{220.566116pt}}
\pgfusepath{stroke}
\pgfpathmoveto{\pgfpoint{444.487549pt}{226.651718pt}}
\pgflineto{\pgfpoint{444.779968pt}{226.553864pt}}
\pgfusepath{stroke}
\pgfpathmoveto{\pgfpoint{444.478149pt}{232.628784pt}}
\pgflineto{\pgfpoint{444.779968pt}{232.541626pt}}
\pgfusepath{stroke}
\pgfpathmoveto{\pgfpoint{444.469238pt}{238.604568pt}}
\pgflineto{\pgfpoint{444.779968pt}{238.529388pt}}
\pgfusepath{stroke}
\pgfpathmoveto{\pgfpoint{444.461060pt}{244.579025pt}}
\pgflineto{\pgfpoint{444.779968pt}{244.517136pt}}
\pgfusepath{stroke}
\pgfpathmoveto{\pgfpoint{444.453796pt}{250.552185pt}}
\pgflineto{\pgfpoint{444.779968pt}{250.504883pt}}
\pgfusepath{stroke}
\pgfpathmoveto{\pgfpoint{444.447815pt}{256.524078pt}}
\pgflineto{\pgfpoint{444.779968pt}{256.492645pt}}
\pgfusepath{stroke}
\pgfpathmoveto{\pgfpoint{444.443298pt}{262.494873pt}}
\pgflineto{\pgfpoint{444.779968pt}{262.480408pt}}
\pgfusepath{stroke}
\pgfpathmoveto{\pgfpoint{444.440613pt}{268.464783pt}}
\pgflineto{\pgfpoint{444.779968pt}{268.468170pt}}
\pgfusepath{stroke}
\pgfpathmoveto{\pgfpoint{444.439941pt}{274.434082pt}}
\pgflineto{\pgfpoint{444.779968pt}{274.455902pt}}
\pgfusepath{stroke}
\pgfpathmoveto{\pgfpoint{444.441406pt}{280.403198pt}}
\pgflineto{\pgfpoint{444.779968pt}{280.443665pt}}
\pgfusepath{stroke}
\pgfpathmoveto{\pgfpoint{444.445068pt}{286.372498pt}}
\pgflineto{\pgfpoint{444.779968pt}{286.431427pt}}
\pgfusepath{stroke}
\pgfpathmoveto{\pgfpoint{444.450897pt}{292.342407pt}}
\pgflineto{\pgfpoint{444.779968pt}{292.419189pt}}
\pgfusepath{stroke}
\pgfpathmoveto{\pgfpoint{444.458710pt}{298.313324pt}}
\pgflineto{\pgfpoint{444.779968pt}{298.406921pt}}
\pgfusepath{stroke}
\pgfpathmoveto{\pgfpoint{444.468201pt}{304.285553pt}}
\pgflineto{\pgfpoint{444.779968pt}{304.394714pt}}
\pgfusepath{stroke}
\pgfpathmoveto{\pgfpoint{444.479126pt}{310.259338pt}}
\pgflineto{\pgfpoint{444.779968pt}{310.382446pt}}
\pgfusepath{stroke}
\pgfpathmoveto{\pgfpoint{444.491119pt}{316.234833pt}}
\pgflineto{\pgfpoint{444.779968pt}{316.370178pt}}
\pgfusepath{stroke}
\pgfpathmoveto{\pgfpoint{444.503784pt}{322.212067pt}}
\pgflineto{\pgfpoint{444.779968pt}{322.357971pt}}
\pgfusepath{stroke}
\pgfpathmoveto{\pgfpoint{444.516876pt}{328.191040pt}}
\pgflineto{\pgfpoint{444.779968pt}{328.345703pt}}
\pgfusepath{stroke}
\pgfpathmoveto{\pgfpoint{444.530090pt}{334.171692pt}}
\pgflineto{\pgfpoint{444.779968pt}{334.333466pt}}
\pgfusepath{stroke}
\pgfpathmoveto{\pgfpoint{444.543152pt}{340.153839pt}}
\pgflineto{\pgfpoint{444.779968pt}{340.321228pt}}
\pgfusepath{stroke}
\pgfpathmoveto{\pgfpoint{444.555939pt}{346.137329pt}}
\pgflineto{\pgfpoint{444.779968pt}{346.308960pt}}
\pgfusepath{stroke}
\pgfpathmoveto{\pgfpoint{444.568268pt}{352.122070pt}}
\pgflineto{\pgfpoint{444.779968pt}{352.296722pt}}
\pgfusepath{stroke}
\pgfpathmoveto{\pgfpoint{444.580078pt}{358.107849pt}}
\pgflineto{\pgfpoint{444.779968pt}{358.284485pt}}
\pgfusepath{stroke}
\pgfpathmoveto{\pgfpoint{444.591309pt}{364.094513pt}}
\pgflineto{\pgfpoint{444.779968pt}{364.272247pt}}
\pgfusepath{stroke}
\pgfpathmoveto{\pgfpoint{444.601929pt}{370.081909pt}}
\pgflineto{\pgfpoint{444.779968pt}{370.260010pt}}
\pgfusepath{stroke}
\pgfpathmoveto{\pgfpoint{151.494461pt}{76.974457pt}}
\pgflineto{\pgfpoint{151.469025pt}{76.923584pt}}
\pgfusepath{stroke}
\pgfpathmoveto{\pgfpoint{151.443604pt}{76.949020pt}}
\pgflineto{\pgfpoint{151.494461pt}{76.974457pt}}
\pgfusepath{stroke}
\pgfpathmoveto{\pgfpoint{151.498657pt}{82.962143pt}}
\pgflineto{\pgfpoint{151.471802pt}{82.910828pt}}
\pgfusepath{stroke}
\pgfpathmoveto{\pgfpoint{151.446396pt}{82.937195pt}}
\pgflineto{\pgfpoint{151.498657pt}{82.962143pt}}
\pgfusepath{stroke}
\pgfpathmoveto{\pgfpoint{151.502991pt}{88.949646pt}}
\pgflineto{\pgfpoint{151.474686pt}{88.897942pt}}
\pgfusepath{stroke}
\pgfpathmoveto{\pgfpoint{151.449310pt}{88.925270pt}}
\pgflineto{\pgfpoint{151.502991pt}{88.949646pt}}
\pgfusepath{stroke}
\pgfpathmoveto{\pgfpoint{151.507492pt}{94.936966pt}}
\pgflineto{\pgfpoint{151.477631pt}{94.884888pt}}
\pgfusepath{stroke}
\pgfpathmoveto{\pgfpoint{151.452362pt}{94.913239pt}}
\pgflineto{\pgfpoint{151.507492pt}{94.936966pt}}
\pgfusepath{stroke}
\pgfpathmoveto{\pgfpoint{151.512131pt}{100.924065pt}}
\pgflineto{\pgfpoint{151.480637pt}{100.871704pt}}
\pgfusepath{stroke}
\pgfpathmoveto{\pgfpoint{151.455536pt}{100.901070pt}}
\pgflineto{\pgfpoint{151.512131pt}{100.924065pt}}
\pgfusepath{stroke}
\pgfpathmoveto{\pgfpoint{151.516891pt}{106.910950pt}}
\pgflineto{\pgfpoint{151.483734pt}{106.858345pt}}
\pgfusepath{stroke}
\pgfpathmoveto{\pgfpoint{151.458801pt}{106.888763pt}}
\pgflineto{\pgfpoint{151.516891pt}{106.910950pt}}
\pgfusepath{stroke}
\pgfpathmoveto{\pgfpoint{151.521790pt}{112.897575pt}}
\pgflineto{\pgfpoint{151.486877pt}{112.844803pt}}
\pgfusepath{stroke}
\pgfpathmoveto{\pgfpoint{151.462204pt}{112.876320pt}}
\pgflineto{\pgfpoint{151.521790pt}{112.897575pt}}
\pgfusepath{stroke}
\pgfpathmoveto{\pgfpoint{151.526825pt}{118.883926pt}}
\pgflineto{\pgfpoint{151.490067pt}{118.831055pt}}
\pgfusepath{stroke}
\pgfpathmoveto{\pgfpoint{151.465698pt}{118.863686pt}}
\pgflineto{\pgfpoint{151.526825pt}{118.883926pt}}
\pgfusepath{stroke}
\pgfpathmoveto{\pgfpoint{151.531937pt}{124.869972pt}}
\pgflineto{\pgfpoint{151.493271pt}{124.817108pt}}
\pgfusepath{stroke}
\pgfpathmoveto{\pgfpoint{151.469315pt}{124.850876pt}}
\pgflineto{\pgfpoint{151.531937pt}{124.869972pt}}
\pgfusepath{stroke}
\pgfpathmoveto{\pgfpoint{151.537155pt}{130.855698pt}}
\pgflineto{\pgfpoint{151.496536pt}{130.802933pt}}
\pgfusepath{stroke}
\pgfpathmoveto{\pgfpoint{151.473007pt}{130.837860pt}}
\pgflineto{\pgfpoint{151.537155pt}{130.855698pt}}
\pgfusepath{stroke}
\pgfpathmoveto{\pgfpoint{151.542419pt}{136.841064pt}}
\pgflineto{\pgfpoint{151.499786pt}{136.788513pt}}
\pgfusepath{stroke}
\pgfpathmoveto{\pgfpoint{151.476776pt}{136.824615pt}}
\pgflineto{\pgfpoint{151.542419pt}{136.841064pt}}
\pgfusepath{stroke}
\pgfpathmoveto{\pgfpoint{151.547714pt}{142.826080pt}}
\pgflineto{\pgfpoint{151.503006pt}{142.773849pt}}
\pgfusepath{stroke}
\pgfpathmoveto{\pgfpoint{151.480606pt}{142.811127pt}}
\pgflineto{\pgfpoint{151.547714pt}{142.826080pt}}
\pgfusepath{stroke}
\pgfpathmoveto{\pgfpoint{151.553009pt}{148.810699pt}}
\pgflineto{\pgfpoint{151.506180pt}{148.758942pt}}
\pgfusepath{stroke}
\pgfpathmoveto{\pgfpoint{151.484482pt}{148.797379pt}}
\pgflineto{\pgfpoint{151.553009pt}{148.810699pt}}
\pgfusepath{stroke}
\pgfpathmoveto{\pgfpoint{151.558273pt}{154.794937pt}}
\pgflineto{\pgfpoint{151.509308pt}{154.743744pt}}
\pgfusepath{stroke}
\pgfpathmoveto{\pgfpoint{151.488388pt}{154.783356pt}}
\pgflineto{\pgfpoint{151.558273pt}{154.794937pt}}
\pgfusepath{stroke}
\pgfpathmoveto{\pgfpoint{151.563477pt}{160.778732pt}}
\pgflineto{\pgfpoint{151.512329pt}{160.728287pt}}
\pgfusepath{stroke}
\pgfpathmoveto{\pgfpoint{151.492294pt}{160.769058pt}}
\pgflineto{\pgfpoint{151.563477pt}{160.778732pt}}
\pgfusepath{stroke}
\pgfpathmoveto{\pgfpoint{151.568527pt}{166.762100pt}}
\pgflineto{\pgfpoint{151.515213pt}{166.712555pt}}
\pgfusepath{stroke}
\pgfpathmoveto{\pgfpoint{151.496155pt}{166.754456pt}}
\pgflineto{\pgfpoint{151.568527pt}{166.762100pt}}
\pgfusepath{stroke}
\pgfpathmoveto{\pgfpoint{151.573441pt}{172.745056pt}}
\pgflineto{\pgfpoint{151.517960pt}{172.696564pt}}
\pgfusepath{stroke}
\pgfpathmoveto{\pgfpoint{151.499969pt}{172.739548pt}}
\pgflineto{\pgfpoint{151.573441pt}{172.745056pt}}
\pgfusepath{stroke}
\pgfpathmoveto{\pgfpoint{151.578140pt}{178.727570pt}}
\pgflineto{\pgfpoint{151.520508pt}{178.680298pt}}
\pgfusepath{stroke}
\pgfpathmoveto{\pgfpoint{151.503677pt}{178.724335pt}}
\pgflineto{\pgfpoint{151.578140pt}{178.727570pt}}
\pgfusepath{stroke}
\pgfpathmoveto{\pgfpoint{151.582596pt}{184.709671pt}}
\pgflineto{\pgfpoint{151.522842pt}{184.663788pt}}
\pgfusepath{stroke}
\pgfpathmoveto{\pgfpoint{151.507263pt}{184.708817pt}}
\pgflineto{\pgfpoint{151.582596pt}{184.709671pt}}
\pgfusepath{stroke}
\pgfpathmoveto{\pgfpoint{151.586700pt}{190.691376pt}}
\pgflineto{\pgfpoint{151.524918pt}{190.647064pt}}
\pgfusepath{stroke}
\pgfpathmoveto{\pgfpoint{151.510681pt}{190.692993pt}}
\pgflineto{\pgfpoint{151.586700pt}{190.691376pt}}
\pgfusepath{stroke}
\pgfpathmoveto{\pgfpoint{151.590454pt}{196.672699pt}}
\pgflineto{\pgfpoint{151.526718pt}{196.630112pt}}
\pgfusepath{stroke}
\pgfpathmoveto{\pgfpoint{151.513885pt}{196.676880pt}}
\pgflineto{\pgfpoint{151.590454pt}{196.672699pt}}
\pgfusepath{stroke}
\pgfpathmoveto{\pgfpoint{151.593796pt}{202.653702pt}}
\pgflineto{\pgfpoint{151.528168pt}{202.612991pt}}
\pgfusepath{stroke}
\pgfpathmoveto{\pgfpoint{151.516876pt}{202.660507pt}}
\pgflineto{\pgfpoint{151.593796pt}{202.653702pt}}
\pgfusepath{stroke}
\pgfpathmoveto{\pgfpoint{151.596680pt}{208.634399pt}}
\pgflineto{\pgfpoint{151.529327pt}{208.595718pt}}
\pgfusepath{stroke}
\pgfpathmoveto{\pgfpoint{151.519592pt}{208.643875pt}}
\pgflineto{\pgfpoint{151.596680pt}{208.634399pt}}
\pgfusepath{stroke}
\pgfpathmoveto{\pgfpoint{151.599091pt}{214.614853pt}}
\pgflineto{\pgfpoint{151.530121pt}{214.578339pt}}
\pgfusepath{stroke}
\pgfpathmoveto{\pgfpoint{151.522018pt}{214.627045pt}}
\pgflineto{\pgfpoint{151.599091pt}{214.614853pt}}
\pgfusepath{stroke}
\pgfpathmoveto{\pgfpoint{151.600998pt}{220.595123pt}}
\pgflineto{\pgfpoint{151.530548pt}{220.560898pt}}
\pgfusepath{stroke}
\pgfpathmoveto{\pgfpoint{151.524109pt}{220.610001pt}}
\pgflineto{\pgfpoint{151.600998pt}{220.595123pt}}
\pgfusepath{stroke}
\pgfpathmoveto{\pgfpoint{151.602356pt}{226.575241pt}}
\pgflineto{\pgfpoint{151.530609pt}{226.543411pt}}
\pgfusepath{stroke}
\pgfpathmoveto{\pgfpoint{151.525864pt}{226.592819pt}}
\pgflineto{\pgfpoint{151.602356pt}{226.575241pt}}
\pgfusepath{stroke}
\pgfpathmoveto{\pgfpoint{151.603180pt}{232.555283pt}}
\pgflineto{\pgfpoint{151.530289pt}{232.525940pt}}
\pgfusepath{stroke}
\pgfpathmoveto{\pgfpoint{151.527252pt}{232.575531pt}}
\pgflineto{\pgfpoint{151.603180pt}{232.555283pt}}
\pgfusepath{stroke}
\pgfpathmoveto{\pgfpoint{151.603424pt}{238.535309pt}}
\pgflineto{\pgfpoint{151.529602pt}{238.508514pt}}
\pgfusepath{stroke}
\pgfpathmoveto{\pgfpoint{151.528290pt}{238.558167pt}}
\pgflineto{\pgfpoint{151.603424pt}{238.535309pt}}
\pgfusepath{stroke}
\pgfpathmoveto{\pgfpoint{151.603149pt}{244.515381pt}}
\pgflineto{\pgfpoint{151.528564pt}{244.491180pt}}
\pgfusepath{stroke}
\pgfpathmoveto{\pgfpoint{151.528961pt}{244.540771pt}}
\pgflineto{\pgfpoint{151.603149pt}{244.515381pt}}
\pgfusepath{stroke}
\pgfpathmoveto{\pgfpoint{151.602356pt}{250.495560pt}}
\pgflineto{\pgfpoint{151.527191pt}{250.473969pt}}
\pgfusepath{stroke}
\pgfpathmoveto{\pgfpoint{151.529266pt}{250.523376pt}}
\pgflineto{\pgfpoint{151.602356pt}{250.495560pt}}
\pgfusepath{stroke}
\pgfpathmoveto{\pgfpoint{151.601044pt}{256.475891pt}}
\pgflineto{\pgfpoint{151.525497pt}{256.456909pt}}
\pgfusepath{stroke}
\pgfpathmoveto{\pgfpoint{151.529221pt}{256.506042pt}}
\pgflineto{\pgfpoint{151.601044pt}{256.475891pt}}
\pgfusepath{stroke}
\pgfpathmoveto{\pgfpoint{151.599274pt}{262.456390pt}}
\pgflineto{\pgfpoint{151.523529pt}{262.440033pt}}
\pgfusepath{stroke}
\pgfpathmoveto{\pgfpoint{151.528839pt}{262.488770pt}}
\pgflineto{\pgfpoint{151.599274pt}{262.456390pt}}
\pgfusepath{stroke}
\pgfpathmoveto{\pgfpoint{151.597061pt}{268.437134pt}}
\pgflineto{\pgfpoint{151.521255pt}{268.423370pt}}
\pgfusepath{stroke}
\pgfpathmoveto{\pgfpoint{151.528137pt}{268.471588pt}}
\pgflineto{\pgfpoint{151.597061pt}{268.437134pt}}
\pgfusepath{stroke}
\pgfpathmoveto{\pgfpoint{151.594452pt}{274.418152pt}}
\pgflineto{\pgfpoint{151.518768pt}{274.406921pt}}
\pgfusepath{stroke}
\pgfpathmoveto{\pgfpoint{151.527161pt}{274.454590pt}}
\pgflineto{\pgfpoint{151.594452pt}{274.418152pt}}
\pgfusepath{stroke}
\pgfpathmoveto{\pgfpoint{151.591461pt}{280.399475pt}}
\pgflineto{\pgfpoint{151.516068pt}{280.390717pt}}
\pgfusepath{stroke}
\pgfpathmoveto{\pgfpoint{151.525894pt}{280.437714pt}}
\pgflineto{\pgfpoint{151.591461pt}{280.399475pt}}
\pgfusepath{stroke}
\pgfpathmoveto{\pgfpoint{151.588196pt}{286.381104pt}}
\pgflineto{\pgfpoint{151.513184pt}{286.374756pt}}
\pgfusepath{stroke}
\pgfpathmoveto{\pgfpoint{151.524368pt}{286.421021pt}}
\pgflineto{\pgfpoint{151.588196pt}{286.381104pt}}
\pgfusepath{stroke}
\pgfpathmoveto{\pgfpoint{151.584579pt}{292.363098pt}}
\pgflineto{\pgfpoint{151.510147pt}{292.359070pt}}
\pgfusepath{stroke}
\pgfpathmoveto{\pgfpoint{151.522629pt}{292.404510pt}}
\pgflineto{\pgfpoint{151.584579pt}{292.363098pt}}
\pgfusepath{stroke}
\pgfpathmoveto{\pgfpoint{151.580780pt}{298.345398pt}}
\pgflineto{\pgfpoint{151.507004pt}{298.343597pt}}
\pgfusepath{stroke}
\pgfpathmoveto{\pgfpoint{151.520676pt}{298.388214pt}}
\pgflineto{\pgfpoint{151.580780pt}{298.345398pt}}
\pgfusepath{stroke}
\pgfpathmoveto{\pgfpoint{151.576736pt}{304.328064pt}}
\pgflineto{\pgfpoint{151.503754pt}{304.328400pt}}
\pgfusepath{stroke}
\pgfpathmoveto{\pgfpoint{151.518539pt}{304.372131pt}}
\pgflineto{\pgfpoint{151.576736pt}{304.328064pt}}
\pgfusepath{stroke}
\pgfpathmoveto{\pgfpoint{151.572540pt}{310.311066pt}}
\pgflineto{\pgfpoint{151.500412pt}{310.313477pt}}
\pgfusepath{stroke}
\pgfpathmoveto{\pgfpoint{151.516281pt}{310.356262pt}}
\pgflineto{\pgfpoint{151.572540pt}{310.311066pt}}
\pgfusepath{stroke}
\pgfpathmoveto{\pgfpoint{151.568207pt}{316.294434pt}}
\pgflineto{\pgfpoint{151.497055pt}{316.298767pt}}
\pgfusepath{stroke}
\pgfpathmoveto{\pgfpoint{151.513870pt}{316.340607pt}}
\pgflineto{\pgfpoint{151.568207pt}{316.294434pt}}
\pgfusepath{stroke}
\pgfpathmoveto{\pgfpoint{151.563766pt}{322.278137pt}}
\pgflineto{\pgfpoint{151.493637pt}{322.284302pt}}
\pgfusepath{stroke}
\pgfpathmoveto{\pgfpoint{151.511368pt}{322.325165pt}}
\pgflineto{\pgfpoint{151.563766pt}{322.278137pt}}
\pgfusepath{stroke}
\pgfpathmoveto{\pgfpoint{151.559235pt}{328.262146pt}}
\pgflineto{\pgfpoint{151.490204pt}{328.270081pt}}
\pgfusepath{stroke}
\pgfpathmoveto{\pgfpoint{151.508774pt}{328.309937pt}}
\pgflineto{\pgfpoint{151.559235pt}{328.262146pt}}
\pgfusepath{stroke}
\pgfpathmoveto{\pgfpoint{151.554688pt}{334.246521pt}}
\pgflineto{\pgfpoint{151.486786pt}{334.256073pt}}
\pgfusepath{stroke}
\pgfpathmoveto{\pgfpoint{151.506119pt}{334.294922pt}}
\pgflineto{\pgfpoint{151.554688pt}{334.246521pt}}
\pgfusepath{stroke}
\pgfpathmoveto{\pgfpoint{151.550079pt}{340.231171pt}}
\pgflineto{\pgfpoint{151.483368pt}{340.242310pt}}
\pgfusepath{stroke}
\pgfpathmoveto{\pgfpoint{151.503387pt}{340.280090pt}}
\pgflineto{\pgfpoint{151.550079pt}{340.231171pt}}
\pgfusepath{stroke}
\pgfpathmoveto{\pgfpoint{151.545471pt}{346.216125pt}}
\pgflineto{\pgfpoint{151.479996pt}{346.228699pt}}
\pgfusepath{stroke}
\pgfpathmoveto{\pgfpoint{151.500626pt}{346.265472pt}}
\pgflineto{\pgfpoint{151.545471pt}{346.216125pt}}
\pgfusepath{stroke}
\pgfpathmoveto{\pgfpoint{151.540894pt}{352.201355pt}}
\pgflineto{\pgfpoint{151.476654pt}{352.215271pt}}
\pgfusepath{stroke}
\pgfpathmoveto{\pgfpoint{151.497864pt}{352.251038pt}}
\pgflineto{\pgfpoint{151.540894pt}{352.201355pt}}
\pgfusepath{stroke}
\pgfpathmoveto{\pgfpoint{151.536331pt}{358.186890pt}}
\pgflineto{\pgfpoint{151.473389pt}{358.202057pt}}
\pgfusepath{stroke}
\pgfpathmoveto{\pgfpoint{151.495056pt}{358.236816pt}}
\pgflineto{\pgfpoint{151.536331pt}{358.186890pt}}
\pgfusepath{stroke}
\pgfpathmoveto{\pgfpoint{151.531815pt}{364.172668pt}}
\pgflineto{\pgfpoint{151.470139pt}{364.188995pt}}
\pgfusepath{stroke}
\pgfpathmoveto{\pgfpoint{151.492279pt}{364.222717pt}}
\pgflineto{\pgfpoint{151.531815pt}{364.172668pt}}
\pgfusepath{stroke}
\pgfpathmoveto{\pgfpoint{151.527374pt}{370.158691pt}}
\pgflineto{\pgfpoint{151.466980pt}{370.176086pt}}
\pgfusepath{stroke}
\pgfpathmoveto{\pgfpoint{151.489502pt}{370.208832pt}}
\pgflineto{\pgfpoint{151.527374pt}{370.158691pt}}
\pgfusepath{stroke}
\pgfpathmoveto{\pgfpoint{157.482132pt}{76.978653pt}}
\pgflineto{\pgfpoint{157.457199pt}{76.926392pt}}
\pgfusepath{stroke}
\pgfpathmoveto{\pgfpoint{157.430832pt}{76.951813pt}}
\pgflineto{\pgfpoint{157.482132pt}{76.978653pt}}
\pgfusepath{stroke}
\pgfpathmoveto{\pgfpoint{157.486496pt}{82.966492pt}}
\pgflineto{\pgfpoint{157.460114pt}{82.913727pt}}
\pgfusepath{stroke}
\pgfpathmoveto{\pgfpoint{157.433716pt}{82.940094pt}}
\pgflineto{\pgfpoint{157.486496pt}{82.966492pt}}
\pgfusepath{stroke}
\pgfpathmoveto{\pgfpoint{157.491013pt}{88.954163pt}}
\pgflineto{\pgfpoint{157.463120pt}{88.900909pt}}
\pgfusepath{stroke}
\pgfpathmoveto{\pgfpoint{157.436737pt}{88.928307pt}}
\pgflineto{\pgfpoint{157.491013pt}{88.954163pt}}
\pgfusepath{stroke}
\pgfpathmoveto{\pgfpoint{157.495712pt}{94.941635pt}}
\pgflineto{\pgfpoint{157.466202pt}{94.887947pt}}
\pgfusepath{stroke}
\pgfpathmoveto{\pgfpoint{157.439896pt}{94.916389pt}}
\pgflineto{\pgfpoint{157.495712pt}{94.941635pt}}
\pgfusepath{stroke}
\pgfpathmoveto{\pgfpoint{157.500565pt}{100.928894pt}}
\pgflineto{\pgfpoint{157.469406pt}{100.874832pt}}
\pgfusepath{stroke}
\pgfpathmoveto{\pgfpoint{157.443192pt}{100.904358pt}}
\pgflineto{\pgfpoint{157.500565pt}{100.928894pt}}
\pgfusepath{stroke}
\pgfpathmoveto{\pgfpoint{157.505585pt}{106.915916pt}}
\pgflineto{\pgfpoint{157.472656pt}{106.861549pt}}
\pgfusepath{stroke}
\pgfpathmoveto{\pgfpoint{157.446625pt}{106.892181pt}}
\pgflineto{\pgfpoint{157.505585pt}{106.915916pt}}
\pgfusepath{stroke}
\pgfpathmoveto{\pgfpoint{157.510757pt}{112.902672pt}}
\pgflineto{\pgfpoint{157.475998pt}{112.848076pt}}
\pgfusepath{stroke}
\pgfpathmoveto{\pgfpoint{157.450195pt}{112.879845pt}}
\pgflineto{\pgfpoint{157.510757pt}{112.902672pt}}
\pgfusepath{stroke}
\pgfpathmoveto{\pgfpoint{157.516083pt}{118.889153pt}}
\pgflineto{\pgfpoint{157.479401pt}{118.834381pt}}
\pgfusepath{stroke}
\pgfpathmoveto{\pgfpoint{157.453873pt}{118.867340pt}}
\pgflineto{\pgfpoint{157.516083pt}{118.889153pt}}
\pgfusepath{stroke}
\pgfpathmoveto{\pgfpoint{157.521530pt}{124.875305pt}}
\pgflineto{\pgfpoint{157.482849pt}{124.820457pt}}
\pgfusepath{stroke}
\pgfpathmoveto{\pgfpoint{157.457672pt}{124.854630pt}}
\pgflineto{\pgfpoint{157.521530pt}{124.875305pt}}
\pgfusepath{stroke}
\pgfpathmoveto{\pgfpoint{157.527069pt}{130.861115pt}}
\pgflineto{\pgfpoint{157.486328pt}{130.806305pt}}
\pgfusepath{stroke}
\pgfpathmoveto{\pgfpoint{157.461594pt}{130.841705pt}}
\pgflineto{\pgfpoint{157.527069pt}{130.861115pt}}
\pgfusepath{stroke}
\pgfpathmoveto{\pgfpoint{157.532700pt}{136.846558pt}}
\pgflineto{\pgfpoint{157.489838pt}{136.791885pt}}
\pgfusepath{stroke}
\pgfpathmoveto{\pgfpoint{157.465607pt}{136.828552pt}}
\pgflineto{\pgfpoint{157.532700pt}{136.846558pt}}
\pgfusepath{stroke}
\pgfpathmoveto{\pgfpoint{157.538406pt}{142.831589pt}}
\pgflineto{\pgfpoint{157.493317pt}{142.777191pt}}
\pgfusepath{stroke}
\pgfpathmoveto{\pgfpoint{157.469696pt}{142.815125pt}}
\pgflineto{\pgfpoint{157.538406pt}{142.831589pt}}
\pgfusepath{stroke}
\pgfpathmoveto{\pgfpoint{157.544113pt}{148.816223pt}}
\pgflineto{\pgfpoint{157.496796pt}{148.762238pt}}
\pgfusepath{stroke}
\pgfpathmoveto{\pgfpoint{157.473862pt}{148.801422pt}}
\pgflineto{\pgfpoint{157.544113pt}{148.816223pt}}
\pgfusepath{stroke}
\pgfpathmoveto{\pgfpoint{157.549820pt}{154.800385pt}}
\pgflineto{\pgfpoint{157.500183pt}{154.746948pt}}
\pgfusepath{stroke}
\pgfpathmoveto{\pgfpoint{157.478058pt}{154.787415pt}}
\pgflineto{\pgfpoint{157.549820pt}{154.800385pt}}
\pgfusepath{stroke}
\pgfpathmoveto{\pgfpoint{157.555450pt}{160.784088pt}}
\pgflineto{\pgfpoint{157.503494pt}{160.731400pt}}
\pgfusepath{stroke}
\pgfpathmoveto{\pgfpoint{157.482269pt}{160.773102pt}}
\pgflineto{\pgfpoint{157.555450pt}{160.784088pt}}
\pgfusepath{stroke}
\pgfpathmoveto{\pgfpoint{157.560989pt}{166.767334pt}}
\pgflineto{\pgfpoint{157.506683pt}{166.715515pt}}
\pgfusepath{stroke}
\pgfpathmoveto{\pgfpoint{157.486450pt}{166.758453pt}}
\pgflineto{\pgfpoint{157.560989pt}{166.767334pt}}
\pgfusepath{stroke}
\pgfpathmoveto{\pgfpoint{157.566345pt}{172.750076pt}}
\pgflineto{\pgfpoint{157.509705pt}{172.699341pt}}
\pgfusepath{stroke}
\pgfpathmoveto{\pgfpoint{157.490601pt}{172.743484pt}}
\pgflineto{\pgfpoint{157.566345pt}{172.750076pt}}
\pgfusepath{stroke}
\pgfpathmoveto{\pgfpoint{157.571487pt}{178.732361pt}}
\pgflineto{\pgfpoint{157.512512pt}{178.682877pt}}
\pgfusepath{stroke}
\pgfpathmoveto{\pgfpoint{157.494614pt}{178.728149pt}}
\pgflineto{\pgfpoint{157.571487pt}{178.732361pt}}
\pgfusepath{stroke}
\pgfpathmoveto{\pgfpoint{157.576355pt}{184.714157pt}}
\pgflineto{\pgfpoint{157.515106pt}{184.666122pt}}
\pgfusepath{stroke}
\pgfpathmoveto{\pgfpoint{157.498520pt}{184.712479pt}}
\pgflineto{\pgfpoint{157.576355pt}{184.714157pt}}
\pgfusepath{stroke}
\pgfpathmoveto{\pgfpoint{157.580887pt}{190.695511pt}}
\pgflineto{\pgfpoint{157.517410pt}{190.649109pt}}
\pgfusepath{stroke}
\pgfpathmoveto{\pgfpoint{157.502258pt}{190.696472pt}}
\pgflineto{\pgfpoint{157.580887pt}{190.695511pt}}
\pgfusepath{stroke}
\pgfpathmoveto{\pgfpoint{157.585022pt}{196.676437pt}}
\pgflineto{\pgfpoint{157.519409pt}{196.631851pt}}
\pgfusepath{stroke}
\pgfpathmoveto{\pgfpoint{157.505783pt}{196.680130pt}}
\pgflineto{\pgfpoint{157.585022pt}{196.676437pt}}
\pgfusepath{stroke}
\pgfpathmoveto{\pgfpoint{157.588699pt}{202.656982pt}}
\pgflineto{\pgfpoint{157.521057pt}{202.614380pt}}
\pgfusepath{stroke}
\pgfpathmoveto{\pgfpoint{157.509033pt}{202.663483pt}}
\pgflineto{\pgfpoint{157.588699pt}{202.656982pt}}
\pgfusepath{stroke}
\pgfpathmoveto{\pgfpoint{157.591888pt}{208.637192pt}}
\pgflineto{\pgfpoint{157.522354pt}{208.596756pt}}
\pgfusepath{stroke}
\pgfpathmoveto{\pgfpoint{157.511993pt}{208.646561pt}}
\pgflineto{\pgfpoint{157.591888pt}{208.637192pt}}
\pgfusepath{stroke}
\pgfpathmoveto{\pgfpoint{157.594528pt}{214.617111pt}}
\pgflineto{\pgfpoint{157.523239pt}{214.578995pt}}
\pgfusepath{stroke}
\pgfpathmoveto{\pgfpoint{157.514618pt}{214.629395pt}}
\pgflineto{\pgfpoint{157.594528pt}{214.617111pt}}
\pgfusepath{stroke}
\pgfpathmoveto{\pgfpoint{157.596588pt}{220.596817pt}}
\pgflineto{\pgfpoint{157.523727pt}{220.561157pt}}
\pgfusepath{stroke}
\pgfpathmoveto{\pgfpoint{157.516907pt}{220.612015pt}}
\pgflineto{\pgfpoint{157.596588pt}{220.596817pt}}
\pgfusepath{stroke}
\pgfpathmoveto{\pgfpoint{157.598068pt}{226.576370pt}}
\pgflineto{\pgfpoint{157.523788pt}{226.543274pt}}
\pgfusepath{stroke}
\pgfpathmoveto{\pgfpoint{157.518799pt}{226.594452pt}}
\pgflineto{\pgfpoint{157.598068pt}{226.576370pt}}
\pgfusepath{stroke}
\pgfpathmoveto{\pgfpoint{157.598938pt}{232.555832pt}}
\pgflineto{\pgfpoint{157.523438pt}{232.525406pt}}
\pgfusepath{stroke}
\pgfpathmoveto{\pgfpoint{157.520294pt}{232.576782pt}}
\pgflineto{\pgfpoint{157.598938pt}{232.555832pt}}
\pgfusepath{stroke}
\pgfpathmoveto{\pgfpoint{157.599167pt}{238.535294pt}}
\pgflineto{\pgfpoint{157.522690pt}{238.507599pt}}
\pgfusepath{stroke}
\pgfpathmoveto{\pgfpoint{157.521378pt}{238.559036pt}}
\pgflineto{\pgfpoint{157.599167pt}{238.535294pt}}
\pgfusepath{stroke}
\pgfpathmoveto{\pgfpoint{157.598831pt}{244.514801pt}}
\pgflineto{\pgfpoint{157.521545pt}{244.489899pt}}
\pgfusepath{stroke}
\pgfpathmoveto{\pgfpoint{157.522064pt}{244.541245pt}}
\pgflineto{\pgfpoint{157.598831pt}{244.514801pt}}
\pgfusepath{stroke}
\pgfpathmoveto{\pgfpoint{157.597885pt}{250.494415pt}}
\pgflineto{\pgfpoint{157.520020pt}{250.472351pt}}
\pgfusepath{stroke}
\pgfpathmoveto{\pgfpoint{157.522339pt}{250.523483pt}}
\pgflineto{\pgfpoint{157.597885pt}{250.494415pt}}
\pgfusepath{stroke}
\pgfpathmoveto{\pgfpoint{157.596405pt}{256.474243pt}}
\pgflineto{\pgfpoint{157.518143pt}{256.454956pt}}
\pgfusepath{stroke}
\pgfpathmoveto{\pgfpoint{157.522217pt}{256.505768pt}}
\pgflineto{\pgfpoint{157.596405pt}{256.474243pt}}
\pgfusepath{stroke}
\pgfpathmoveto{\pgfpoint{157.594391pt}{262.454285pt}}
\pgflineto{\pgfpoint{157.515945pt}{262.437805pt}}
\pgfusepath{stroke}
\pgfpathmoveto{\pgfpoint{157.521744pt}{262.488159pt}}
\pgflineto{\pgfpoint{157.594391pt}{262.454285pt}}
\pgfusepath{stroke}
\pgfpathmoveto{\pgfpoint{157.591904pt}{268.434570pt}}
\pgflineto{\pgfpoint{157.513443pt}{268.420868pt}}
\pgfusepath{stroke}
\pgfpathmoveto{\pgfpoint{157.520905pt}{268.470673pt}}
\pgflineto{\pgfpoint{157.591904pt}{268.434570pt}}
\pgfusepath{stroke}
\pgfpathmoveto{\pgfpoint{157.588989pt}{274.415192pt}}
\pgflineto{\pgfpoint{157.510712pt}{274.404205pt}}
\pgfusepath{stroke}
\pgfpathmoveto{\pgfpoint{157.519760pt}{274.453369pt}}
\pgflineto{\pgfpoint{157.588989pt}{274.415192pt}}
\pgfusepath{stroke}
\pgfpathmoveto{\pgfpoint{157.585663pt}{280.396179pt}}
\pgflineto{\pgfpoint{157.507751pt}{280.387787pt}}
\pgfusepath{stroke}
\pgfpathmoveto{\pgfpoint{157.518311pt}{280.436218pt}}
\pgflineto{\pgfpoint{157.585663pt}{280.396179pt}}
\pgfusepath{stroke}
\pgfpathmoveto{\pgfpoint{157.582031pt}{286.377502pt}}
\pgflineto{\pgfpoint{157.504608pt}{286.371674pt}}
\pgfusepath{stroke}
\pgfpathmoveto{\pgfpoint{157.516586pt}{286.419281pt}}
\pgflineto{\pgfpoint{157.582031pt}{286.377502pt}}
\pgfusepath{stroke}
\pgfpathmoveto{\pgfpoint{157.578094pt}{292.359192pt}}
\pgflineto{\pgfpoint{157.501312pt}{292.355835pt}}
\pgfusepath{stroke}
\pgfpathmoveto{\pgfpoint{157.514648pt}{292.402588pt}}
\pgflineto{\pgfpoint{157.578094pt}{292.359192pt}}
\pgfusepath{stroke}
\pgfpathmoveto{\pgfpoint{157.573898pt}{298.341278pt}}
\pgflineto{\pgfpoint{157.497879pt}{298.340271pt}}
\pgfusepath{stroke}
\pgfpathmoveto{\pgfpoint{157.512466pt}{298.386078pt}}
\pgflineto{\pgfpoint{157.573898pt}{298.341278pt}}
\pgfusepath{stroke}
\pgfpathmoveto{\pgfpoint{157.569504pt}{304.323761pt}}
\pgflineto{\pgfpoint{157.494370pt}{304.324982pt}}
\pgfusepath{stroke}
\pgfpathmoveto{\pgfpoint{157.510117pt}{304.369812pt}}
\pgflineto{\pgfpoint{157.569504pt}{304.323761pt}}
\pgfusepath{stroke}
\pgfpathmoveto{\pgfpoint{157.564926pt}{310.306641pt}}
\pgflineto{\pgfpoint{157.490768pt}{310.309998pt}}
\pgfusepath{stroke}
\pgfpathmoveto{\pgfpoint{157.507629pt}{310.353821pt}}
\pgflineto{\pgfpoint{157.564926pt}{310.306641pt}}
\pgfusepath{stroke}
\pgfpathmoveto{\pgfpoint{157.560242pt}{316.289856pt}}
\pgflineto{\pgfpoint{157.487152pt}{316.295258pt}}
\pgfusepath{stroke}
\pgfpathmoveto{\pgfpoint{157.505005pt}{316.338013pt}}
\pgflineto{\pgfpoint{157.560242pt}{316.289856pt}}
\pgfusepath{stroke}
\pgfpathmoveto{\pgfpoint{157.555435pt}{322.273468pt}}
\pgflineto{\pgfpoint{157.483490pt}{322.280762pt}}
\pgfusepath{stroke}
\pgfpathmoveto{\pgfpoint{157.502258pt}{322.322510pt}}
\pgflineto{\pgfpoint{157.555435pt}{322.273468pt}}
\pgfusepath{stroke}
\pgfpathmoveto{\pgfpoint{157.550583pt}{328.257446pt}}
\pgflineto{\pgfpoint{157.479843pt}{328.266541pt}}
\pgfusepath{stroke}
\pgfpathmoveto{\pgfpoint{157.499451pt}{328.307190pt}}
\pgflineto{\pgfpoint{157.550583pt}{328.257446pt}}
\pgfusepath{stroke}
\pgfpathmoveto{\pgfpoint{157.545685pt}{334.241760pt}}
\pgflineto{\pgfpoint{157.476181pt}{334.252563pt}}
\pgfusepath{stroke}
\pgfpathmoveto{\pgfpoint{157.496567pt}{334.292114pt}}
\pgflineto{\pgfpoint{157.545685pt}{334.241760pt}}
\pgfusepath{stroke}
\pgfpathmoveto{\pgfpoint{157.540771pt}{340.226410pt}}
\pgflineto{\pgfpoint{157.472580pt}{340.238770pt}}
\pgfusepath{stroke}
\pgfpathmoveto{\pgfpoint{157.493637pt}{340.277252pt}}
\pgflineto{\pgfpoint{157.540771pt}{340.226410pt}}
\pgfusepath{stroke}
\pgfpathmoveto{\pgfpoint{157.535873pt}{346.211395pt}}
\pgflineto{\pgfpoint{157.468994pt}{346.225220pt}}
\pgfusepath{stroke}
\pgfpathmoveto{\pgfpoint{157.490677pt}{346.262604pt}}
\pgflineto{\pgfpoint{157.535873pt}{346.211395pt}}
\pgfusepath{stroke}
\pgfpathmoveto{\pgfpoint{157.530991pt}{352.196655pt}}
\pgflineto{\pgfpoint{157.465454pt}{352.211853pt}}
\pgfusepath{stroke}
\pgfpathmoveto{\pgfpoint{157.487701pt}{352.248169pt}}
\pgflineto{\pgfpoint{157.530991pt}{352.196655pt}}
\pgfusepath{stroke}
\pgfpathmoveto{\pgfpoint{157.526169pt}{358.182220pt}}
\pgflineto{\pgfpoint{157.461990pt}{358.198700pt}}
\pgfusepath{stroke}
\pgfpathmoveto{\pgfpoint{157.484726pt}{358.233917pt}}
\pgflineto{\pgfpoint{157.526169pt}{358.182220pt}}
\pgfusepath{stroke}
\pgfpathmoveto{\pgfpoint{157.521393pt}{364.168030pt}}
\pgflineto{\pgfpoint{157.458618pt}{364.185730pt}}
\pgfusepath{stroke}
\pgfpathmoveto{\pgfpoint{157.481766pt}{364.219849pt}}
\pgflineto{\pgfpoint{157.521393pt}{364.168030pt}}
\pgfusepath{stroke}
\pgfpathmoveto{\pgfpoint{157.516708pt}{370.154114pt}}
\pgflineto{\pgfpoint{157.455276pt}{370.172882pt}}
\pgfusepath{stroke}
\pgfpathmoveto{\pgfpoint{157.478821pt}{370.205994pt}}
\pgflineto{\pgfpoint{157.516708pt}{370.154114pt}}
\pgfusepath{stroke}
\pgfpathmoveto{\pgfpoint{163.469635pt}{76.983002pt}}
\pgflineto{\pgfpoint{163.445267pt}{76.929306pt}}
\pgfusepath{stroke}
\pgfpathmoveto{\pgfpoint{163.417938pt}{76.954666pt}}
\pgflineto{\pgfpoint{163.469635pt}{76.983002pt}}
\pgfusepath{stroke}
\pgfpathmoveto{\pgfpoint{163.474152pt}{82.971008pt}}
\pgflineto{\pgfpoint{163.448303pt}{82.916748pt}}
\pgfusepath{stroke}
\pgfpathmoveto{\pgfpoint{163.420929pt}{82.943115pt}}
\pgflineto{\pgfpoint{163.474152pt}{82.971008pt}}
\pgfusepath{stroke}
\pgfpathmoveto{\pgfpoint{163.478867pt}{88.958862pt}}
\pgflineto{\pgfpoint{163.451447pt}{88.904030pt}}
\pgfusepath{stroke}
\pgfpathmoveto{\pgfpoint{163.424042pt}{88.931450pt}}
\pgflineto{\pgfpoint{163.478867pt}{88.958862pt}}
\pgfusepath{stroke}
\pgfpathmoveto{\pgfpoint{163.483765pt}{94.946518pt}}
\pgflineto{\pgfpoint{163.454712pt}{94.891182pt}}
\pgfusepath{stroke}
\pgfpathmoveto{\pgfpoint{163.427307pt}{94.919685pt}}
\pgflineto{\pgfpoint{163.483765pt}{94.946518pt}}
\pgfusepath{stroke}
\pgfpathmoveto{\pgfpoint{163.488861pt}{100.933952pt}}
\pgflineto{\pgfpoint{163.458054pt}{100.878159pt}}
\pgfusepath{stroke}
\pgfpathmoveto{\pgfpoint{163.430740pt}{100.907784pt}}
\pgflineto{\pgfpoint{163.488861pt}{100.933952pt}}
\pgfusepath{stroke}
\pgfpathmoveto{\pgfpoint{163.494125pt}{106.921150pt}}
\pgflineto{\pgfpoint{163.461517pt}{106.864952pt}}
\pgfusepath{stroke}
\pgfpathmoveto{\pgfpoint{163.434326pt}{106.895760pt}}
\pgflineto{\pgfpoint{163.494125pt}{106.921150pt}}
\pgfusepath{stroke}
\pgfpathmoveto{\pgfpoint{163.499573pt}{112.908066pt}}
\pgflineto{\pgfpoint{163.465057pt}{112.851547pt}}
\pgfusepath{stroke}
\pgfpathmoveto{\pgfpoint{163.438049pt}{112.883553pt}}
\pgflineto{\pgfpoint{163.499573pt}{112.908066pt}}
\pgfusepath{stroke}
\pgfpathmoveto{\pgfpoint{163.505203pt}{118.894691pt}}
\pgflineto{\pgfpoint{163.468674pt}{118.837921pt}}
\pgfusepath{stroke}
\pgfpathmoveto{\pgfpoint{163.441925pt}{118.871185pt}}
\pgflineto{\pgfpoint{163.505203pt}{118.894691pt}}
\pgfusepath{stroke}
\pgfpathmoveto{\pgfpoint{163.510986pt}{124.880981pt}}
\pgflineto{\pgfpoint{163.472382pt}{124.824059pt}}
\pgfusepath{stroke}
\pgfpathmoveto{\pgfpoint{163.445953pt}{124.858604pt}}
\pgflineto{\pgfpoint{163.510986pt}{124.880981pt}}
\pgfusepath{stroke}
\pgfpathmoveto{\pgfpoint{163.516907pt}{130.866913pt}}
\pgflineto{\pgfpoint{163.476135pt}{130.809937pt}}
\pgfusepath{stroke}
\pgfpathmoveto{\pgfpoint{163.450073pt}{130.845795pt}}
\pgflineto{\pgfpoint{163.516907pt}{130.866913pt}}
\pgfusepath{stroke}
\pgfpathmoveto{\pgfpoint{163.522949pt}{136.852417pt}}
\pgflineto{\pgfpoint{163.479904pt}{136.795532pt}}
\pgfusepath{stroke}
\pgfpathmoveto{\pgfpoint{163.454376pt}{136.832733pt}}
\pgflineto{\pgfpoint{163.522949pt}{136.852417pt}}
\pgfusepath{stroke}
\pgfpathmoveto{\pgfpoint{163.529053pt}{142.837524pt}}
\pgflineto{\pgfpoint{163.483673pt}{142.780823pt}}
\pgfusepath{stroke}
\pgfpathmoveto{\pgfpoint{163.458740pt}{142.819397pt}}
\pgflineto{\pgfpoint{163.529053pt}{142.837524pt}}
\pgfusepath{stroke}
\pgfpathmoveto{\pgfpoint{163.535217pt}{148.822144pt}}
\pgflineto{\pgfpoint{163.487427pt}{148.765808pt}}
\pgfusepath{stroke}
\pgfpathmoveto{\pgfpoint{163.463196pt}{148.805756pt}}
\pgflineto{\pgfpoint{163.535217pt}{148.822144pt}}
\pgfusepath{stroke}
\pgfpathmoveto{\pgfpoint{163.541397pt}{154.806305pt}}
\pgflineto{\pgfpoint{163.491165pt}{154.750473pt}}
\pgfusepath{stroke}
\pgfpathmoveto{\pgfpoint{163.467712pt}{154.791794pt}}
\pgflineto{\pgfpoint{163.541397pt}{154.806305pt}}
\pgfusepath{stroke}
\pgfpathmoveto{\pgfpoint{163.547546pt}{160.789932pt}}
\pgflineto{\pgfpoint{163.494781pt}{160.734802pt}}
\pgfusepath{stroke}
\pgfpathmoveto{\pgfpoint{163.472260pt}{160.777481pt}}
\pgflineto{\pgfpoint{163.547546pt}{160.789932pt}}
\pgfusepath{stroke}
\pgfpathmoveto{\pgfpoint{163.553574pt}{166.773041pt}}
\pgflineto{\pgfpoint{163.498291pt}{166.718781pt}}
\pgfusepath{stroke}
\pgfpathmoveto{\pgfpoint{163.476807pt}{166.762802pt}}
\pgflineto{\pgfpoint{163.553574pt}{166.773041pt}}
\pgfusepath{stroke}
\pgfpathmoveto{\pgfpoint{163.559448pt}{172.755600pt}}
\pgflineto{\pgfpoint{163.501648pt}{172.702423pt}}
\pgfusepath{stroke}
\pgfpathmoveto{\pgfpoint{163.481293pt}{172.747742pt}}
\pgflineto{\pgfpoint{163.559448pt}{172.755600pt}}
\pgfusepath{stroke}
\pgfpathmoveto{\pgfpoint{163.565094pt}{178.737610pt}}
\pgflineto{\pgfpoint{163.504776pt}{178.685730pt}}
\pgfusepath{stroke}
\pgfpathmoveto{\pgfpoint{163.485703pt}{178.732315pt}}
\pgflineto{\pgfpoint{163.565094pt}{178.737610pt}}
\pgfusepath{stroke}
\pgfpathmoveto{\pgfpoint{163.570465pt}{184.719101pt}}
\pgflineto{\pgfpoint{163.507645pt}{184.668716pt}}
\pgfusepath{stroke}
\pgfpathmoveto{\pgfpoint{163.489960pt}{184.716476pt}}
\pgflineto{\pgfpoint{163.570465pt}{184.719101pt}}
\pgfusepath{stroke}
\pgfpathmoveto{\pgfpoint{163.575470pt}{190.700073pt}}
\pgflineto{\pgfpoint{163.510223pt}{190.651398pt}}
\pgfusepath{stroke}
\pgfpathmoveto{\pgfpoint{163.494049pt}{190.700272pt}}
\pgflineto{\pgfpoint{163.575470pt}{190.700073pt}}
\pgfusepath{stroke}
\pgfpathmoveto{\pgfpoint{163.580032pt}{196.680557pt}}
\pgflineto{\pgfpoint{163.512466pt}{196.633789pt}}
\pgfusepath{stroke}
\pgfpathmoveto{\pgfpoint{163.497925pt}{196.683685pt}}
\pgflineto{\pgfpoint{163.580032pt}{196.680557pt}}
\pgfusepath{stroke}
\pgfpathmoveto{\pgfpoint{163.584106pt}{202.660599pt}}
\pgflineto{\pgfpoint{163.514313pt}{202.615952pt}}
\pgfusepath{stroke}
\pgfpathmoveto{\pgfpoint{163.501495pt}{202.666748pt}}
\pgflineto{\pgfpoint{163.584106pt}{202.660599pt}}
\pgfusepath{stroke}
\pgfpathmoveto{\pgfpoint{163.587631pt}{208.640259pt}}
\pgflineto{\pgfpoint{163.515778pt}{208.597916pt}}
\pgfusepath{stroke}
\pgfpathmoveto{\pgfpoint{163.504730pt}{208.649506pt}}
\pgflineto{\pgfpoint{163.587631pt}{208.640259pt}}
\pgfusepath{stroke}
\pgfpathmoveto{\pgfpoint{163.590546pt}{214.619583pt}}
\pgflineto{\pgfpoint{163.516785pt}{214.579727pt}}
\pgfusepath{stroke}
\pgfpathmoveto{\pgfpoint{163.507614pt}{214.631958pt}}
\pgflineto{\pgfpoint{163.590546pt}{214.619583pt}}
\pgfusepath{stroke}
\pgfpathmoveto{\pgfpoint{163.592804pt}{220.598663pt}}
\pgflineto{\pgfpoint{163.517334pt}{220.561447pt}}
\pgfusepath{stroke}
\pgfpathmoveto{\pgfpoint{163.510086pt}{220.614182pt}}
\pgflineto{\pgfpoint{163.592804pt}{220.598663pt}}
\pgfusepath{stroke}
\pgfpathmoveto{\pgfpoint{163.594391pt}{226.577576pt}}
\pgflineto{\pgfpoint{163.517410pt}{226.543121pt}}
\pgfusepath{stroke}
\pgfpathmoveto{\pgfpoint{163.512131pt}{226.596222pt}}
\pgflineto{\pgfpoint{163.594391pt}{226.577576pt}}
\pgfusepath{stroke}
\pgfpathmoveto{\pgfpoint{163.595306pt}{232.556396pt}}
\pgflineto{\pgfpoint{163.517014pt}{232.524826pt}}
\pgfusepath{stroke}
\pgfpathmoveto{\pgfpoint{163.513733pt}{232.578110pt}}
\pgflineto{\pgfpoint{163.595306pt}{232.556396pt}}
\pgfusepath{stroke}
\pgfpathmoveto{\pgfpoint{163.595535pt}{238.535202pt}}
\pgflineto{\pgfpoint{163.516174pt}{238.506592pt}}
\pgfusepath{stroke}
\pgfpathmoveto{\pgfpoint{163.514877pt}{238.559937pt}}
\pgflineto{\pgfpoint{163.595535pt}{238.535202pt}}
\pgfusepath{stroke}
\pgfpathmoveto{\pgfpoint{163.595078pt}{244.514084pt}}
\pgflineto{\pgfpoint{163.514877pt}{244.488480pt}}
\pgfusepath{stroke}
\pgfpathmoveto{\pgfpoint{163.515564pt}{244.541718pt}}
\pgflineto{\pgfpoint{163.595078pt}{244.514084pt}}
\pgfusepath{stroke}
\pgfpathmoveto{\pgfpoint{163.593979pt}{250.493103pt}}
\pgflineto{\pgfpoint{163.513184pt}{250.470535pt}}
\pgfusepath{stroke}
\pgfpathmoveto{\pgfpoint{163.515793pt}{250.523529pt}}
\pgflineto{\pgfpoint{163.593979pt}{250.493103pt}}
\pgfusepath{stroke}
\pgfpathmoveto{\pgfpoint{163.592270pt}{256.472351pt}}
\pgflineto{\pgfpoint{163.511078pt}{256.452820pt}}
\pgfusepath{stroke}
\pgfpathmoveto{\pgfpoint{163.515594pt}{256.505432pt}}
\pgflineto{\pgfpoint{163.592270pt}{256.472351pt}}
\pgfusepath{stroke}
\pgfpathmoveto{\pgfpoint{163.589966pt}{262.451874pt}}
\pgflineto{\pgfpoint{163.508636pt}{262.435333pt}}
\pgfusepath{stroke}
\pgfpathmoveto{\pgfpoint{163.514984pt}{262.487427pt}}
\pgflineto{\pgfpoint{163.589966pt}{262.451874pt}}
\pgfusepath{stroke}
\pgfpathmoveto{\pgfpoint{163.587173pt}{268.431702pt}}
\pgflineto{\pgfpoint{163.505905pt}{268.418121pt}}
\pgfusepath{stroke}
\pgfpathmoveto{\pgfpoint{163.513992pt}{268.469604pt}}
\pgflineto{\pgfpoint{163.587173pt}{268.431702pt}}
\pgfusepath{stroke}
\pgfpathmoveto{\pgfpoint{163.583893pt}{274.411896pt}}
\pgflineto{\pgfpoint{163.502884pt}{274.401184pt}}
\pgfusepath{stroke}
\pgfpathmoveto{\pgfpoint{163.512665pt}{274.451965pt}}
\pgflineto{\pgfpoint{163.583893pt}{274.411896pt}}
\pgfusepath{stroke}
\pgfpathmoveto{\pgfpoint{163.580215pt}{280.392517pt}}
\pgflineto{\pgfpoint{163.499634pt}{280.384583pt}}
\pgfusepath{stroke}
\pgfpathmoveto{\pgfpoint{163.511002pt}{280.434509pt}}
\pgflineto{\pgfpoint{163.580215pt}{280.392517pt}}
\pgfusepath{stroke}
\pgfpathmoveto{\pgfpoint{163.576172pt}{286.373505pt}}
\pgflineto{\pgfpoint{163.496185pt}{286.368286pt}}
\pgfusepath{stroke}
\pgfpathmoveto{\pgfpoint{163.509064pt}{286.417328pt}}
\pgflineto{\pgfpoint{163.576172pt}{286.373505pt}}
\pgfusepath{stroke}
\pgfpathmoveto{\pgfpoint{163.571823pt}{292.354919pt}}
\pgflineto{\pgfpoint{163.492569pt}{292.352295pt}}
\pgfusepath{stroke}
\pgfpathmoveto{\pgfpoint{163.506866pt}{292.400391pt}}
\pgflineto{\pgfpoint{163.571823pt}{292.354919pt}}
\pgfusepath{stroke}
\pgfpathmoveto{\pgfpoint{163.567230pt}{298.336792pt}}
\pgflineto{\pgfpoint{163.488861pt}{298.336639pt}}
\pgfusepath{stroke}
\pgfpathmoveto{\pgfpoint{163.504456pt}{298.383698pt}}
\pgflineto{\pgfpoint{163.567230pt}{298.336792pt}}
\pgfusepath{stroke}
\pgfpathmoveto{\pgfpoint{163.562424pt}{304.319092pt}}
\pgflineto{\pgfpoint{163.485046pt}{304.321289pt}}
\pgfusepath{stroke}
\pgfpathmoveto{\pgfpoint{163.501846pt}{304.367279pt}}
\pgflineto{\pgfpoint{163.562424pt}{304.319092pt}}
\pgfusepath{stroke}
\pgfpathmoveto{\pgfpoint{163.557449pt}{310.301788pt}}
\pgflineto{\pgfpoint{163.481171pt}{310.306244pt}}
\pgfusepath{stroke}
\pgfpathmoveto{\pgfpoint{163.499084pt}{310.351105pt}}
\pgflineto{\pgfpoint{163.557449pt}{310.301788pt}}
\pgfusepath{stroke}
\pgfpathmoveto{\pgfpoint{163.552368pt}{316.284912pt}}
\pgflineto{\pgfpoint{163.477264pt}{316.291473pt}}
\pgfusepath{stroke}
\pgfpathmoveto{\pgfpoint{163.496216pt}{316.335205pt}}
\pgflineto{\pgfpoint{163.552368pt}{316.284912pt}}
\pgfusepath{stroke}
\pgfpathmoveto{\pgfpoint{163.547195pt}{322.268463pt}}
\pgflineto{\pgfpoint{163.473358pt}{322.276978pt}}
\pgfusepath{stroke}
\pgfpathmoveto{\pgfpoint{163.493256pt}{322.319580pt}}
\pgflineto{\pgfpoint{163.547195pt}{322.268463pt}}
\pgfusepath{stroke}
\pgfpathmoveto{\pgfpoint{163.541962pt}{328.252380pt}}
\pgflineto{\pgfpoint{163.469452pt}{328.262787pt}}
\pgfusepath{stroke}
\pgfpathmoveto{\pgfpoint{163.490173pt}{328.304199pt}}
\pgflineto{\pgfpoint{163.541962pt}{328.252380pt}}
\pgfusepath{stroke}
\pgfpathmoveto{\pgfpoint{163.536713pt}{334.236694pt}}
\pgflineto{\pgfpoint{163.465561pt}{334.248810pt}}
\pgfusepath{stroke}
\pgfpathmoveto{\pgfpoint{163.487061pt}{334.289062pt}}
\pgflineto{\pgfpoint{163.536713pt}{334.236694pt}}
\pgfusepath{stroke}
\pgfpathmoveto{\pgfpoint{163.531464pt}{340.221344pt}}
\pgflineto{\pgfpoint{163.461716pt}{340.235077pt}}
\pgfusepath{stroke}
\pgfpathmoveto{\pgfpoint{163.483917pt}{340.274170pt}}
\pgflineto{\pgfpoint{163.531464pt}{340.221344pt}}
\pgfusepath{stroke}
\pgfpathmoveto{\pgfpoint{163.526245pt}{346.206329pt}}
\pgflineto{\pgfpoint{163.457947pt}{346.221558pt}}
\pgfusepath{stroke}
\pgfpathmoveto{\pgfpoint{163.480743pt}{346.259521pt}}
\pgflineto{\pgfpoint{163.526245pt}{346.206329pt}}
\pgfusepath{stroke}
\pgfpathmoveto{\pgfpoint{163.521072pt}{352.191650pt}}
\pgflineto{\pgfpoint{163.454193pt}{352.208282pt}}
\pgfusepath{stroke}
\pgfpathmoveto{\pgfpoint{163.477539pt}{352.245056pt}}
\pgflineto{\pgfpoint{163.521072pt}{352.191650pt}}
\pgfusepath{stroke}
\pgfpathmoveto{\pgfpoint{163.515961pt}{358.177277pt}}
\pgflineto{\pgfpoint{163.450562pt}{358.195190pt}}
\pgfusepath{stroke}
\pgfpathmoveto{\pgfpoint{163.474380pt}{358.230835pt}}
\pgflineto{\pgfpoint{163.515961pt}{358.177277pt}}
\pgfusepath{stroke}
\pgfpathmoveto{\pgfpoint{163.510910pt}{364.163208pt}}
\pgflineto{\pgfpoint{163.446991pt}{364.182251pt}}
\pgfusepath{stroke}
\pgfpathmoveto{\pgfpoint{163.471222pt}{364.216797pt}}
\pgflineto{\pgfpoint{163.510910pt}{364.163208pt}}
\pgfusepath{stroke}
\pgfpathmoveto{\pgfpoint{163.505951pt}{370.149353pt}}
\pgflineto{\pgfpoint{163.443512pt}{370.169525pt}}
\pgfusepath{stroke}
\pgfpathmoveto{\pgfpoint{163.468109pt}{370.202942pt}}
\pgflineto{\pgfpoint{163.505951pt}{370.149353pt}}
\pgfusepath{stroke}
\pgfpathmoveto{\pgfpoint{169.456970pt}{76.987488pt}}
\pgflineto{\pgfpoint{169.433228pt}{76.932358pt}}
\pgfusepath{stroke}
\pgfpathmoveto{\pgfpoint{169.404892pt}{76.957626pt}}
\pgflineto{\pgfpoint{169.456970pt}{76.987488pt}}
\pgfusepath{stroke}
\pgfpathmoveto{\pgfpoint{169.461639pt}{82.975708pt}}
\pgflineto{\pgfpoint{169.436401pt}{82.919907pt}}
\pgfusepath{stroke}
\pgfpathmoveto{\pgfpoint{169.407959pt}{82.946198pt}}
\pgflineto{\pgfpoint{169.461639pt}{82.975708pt}}
\pgfusepath{stroke}
\pgfpathmoveto{\pgfpoint{169.466522pt}{88.963753pt}}
\pgflineto{\pgfpoint{169.439682pt}{88.907310pt}}
\pgfusepath{stroke}
\pgfpathmoveto{\pgfpoint{169.411194pt}{88.934692pt}}
\pgflineto{\pgfpoint{169.466522pt}{88.963753pt}}
\pgfusepath{stroke}
\pgfpathmoveto{\pgfpoint{169.471619pt}{94.951614pt}}
\pgflineto{\pgfpoint{169.443100pt}{94.894569pt}}
\pgfusepath{stroke}
\pgfpathmoveto{\pgfpoint{169.414566pt}{94.923096pt}}
\pgflineto{\pgfpoint{169.471619pt}{94.951614pt}}
\pgfusepath{stroke}
\pgfpathmoveto{\pgfpoint{169.476944pt}{100.939255pt}}
\pgflineto{\pgfpoint{169.446625pt}{100.881645pt}}
\pgfusepath{stroke}
\pgfpathmoveto{\pgfpoint{169.418121pt}{100.911362pt}}
\pgflineto{\pgfpoint{169.476944pt}{100.939255pt}}
\pgfusepath{stroke}
\pgfpathmoveto{\pgfpoint{169.482452pt}{106.926643pt}}
\pgflineto{\pgfpoint{169.450272pt}{106.868553pt}}
\pgfusepath{stroke}
\pgfpathmoveto{\pgfpoint{169.421860pt}{106.899490pt}}
\pgflineto{\pgfpoint{169.482452pt}{106.926643pt}}
\pgfusepath{stroke}
\pgfpathmoveto{\pgfpoint{169.488220pt}{112.913773pt}}
\pgflineto{\pgfpoint{169.454041pt}{112.855247pt}}
\pgfusepath{stroke}
\pgfpathmoveto{\pgfpoint{169.425766pt}{112.887459pt}}
\pgflineto{\pgfpoint{169.488220pt}{112.913773pt}}
\pgfusepath{stroke}
\pgfpathmoveto{\pgfpoint{169.494171pt}{118.900566pt}}
\pgflineto{\pgfpoint{169.457901pt}{118.841705pt}}
\pgfusepath{stroke}
\pgfpathmoveto{\pgfpoint{169.429840pt}{118.875237pt}}
\pgflineto{\pgfpoint{169.494171pt}{118.900566pt}}
\pgfusepath{stroke}
\pgfpathmoveto{\pgfpoint{169.500305pt}{124.887016pt}}
\pgflineto{\pgfpoint{169.461853pt}{124.827911pt}}
\pgfusepath{stroke}
\pgfpathmoveto{\pgfpoint{169.434067pt}{124.862808pt}}
\pgflineto{\pgfpoint{169.500305pt}{124.887016pt}}
\pgfusepath{stroke}
\pgfpathmoveto{\pgfpoint{169.506622pt}{130.873077pt}}
\pgflineto{\pgfpoint{169.465881pt}{130.813843pt}}
\pgfusepath{stroke}
\pgfpathmoveto{\pgfpoint{169.438461pt}{130.850143pt}}
\pgflineto{\pgfpoint{169.506622pt}{130.873077pt}}
\pgfusepath{stroke}
\pgfpathmoveto{\pgfpoint{169.513092pt}{136.858734pt}}
\pgflineto{\pgfpoint{169.469940pt}{136.799469pt}}
\pgfusepath{stroke}
\pgfpathmoveto{\pgfpoint{169.443024pt}{136.837189pt}}
\pgflineto{\pgfpoint{169.513092pt}{136.858734pt}}
\pgfusepath{stroke}
\pgfpathmoveto{\pgfpoint{169.519669pt}{142.843903pt}}
\pgflineto{\pgfpoint{169.474045pt}{142.784760pt}}
\pgfusepath{stroke}
\pgfpathmoveto{\pgfpoint{169.447693pt}{142.823975pt}}
\pgflineto{\pgfpoint{169.519669pt}{142.843903pt}}
\pgfusepath{stroke}
\pgfpathmoveto{\pgfpoint{169.526337pt}{148.828583pt}}
\pgflineto{\pgfpoint{169.478149pt}{148.769730pt}}
\pgfusepath{stroke}
\pgfpathmoveto{\pgfpoint{169.452469pt}{148.810410pt}}
\pgflineto{\pgfpoint{169.526337pt}{148.828583pt}}
\pgfusepath{stroke}
\pgfpathmoveto{\pgfpoint{169.533051pt}{154.812714pt}}
\pgflineto{\pgfpoint{169.482208pt}{154.754333pt}}
\pgfusepath{stroke}
\pgfpathmoveto{\pgfpoint{169.457336pt}{154.796509pt}}
\pgflineto{\pgfpoint{169.533051pt}{154.812714pt}}
\pgfusepath{stroke}
\pgfpathmoveto{\pgfpoint{169.539734pt}{160.796295pt}}
\pgflineto{\pgfpoint{169.486206pt}{160.738556pt}}
\pgfusepath{stroke}
\pgfpathmoveto{\pgfpoint{169.462265pt}{160.782227pt}}
\pgflineto{\pgfpoint{169.539734pt}{160.796295pt}}
\pgfusepath{stroke}
\pgfpathmoveto{\pgfpoint{169.546326pt}{166.779282pt}}
\pgflineto{\pgfpoint{169.490082pt}{166.722397pt}}
\pgfusepath{stroke}
\pgfpathmoveto{\pgfpoint{169.467194pt}{166.767517pt}}
\pgflineto{\pgfpoint{169.546326pt}{166.779282pt}}
\pgfusepath{stroke}
\pgfpathmoveto{\pgfpoint{169.552780pt}{172.761658pt}}
\pgflineto{\pgfpoint{169.493774pt}{172.705841pt}}
\pgfusepath{stroke}
\pgfpathmoveto{\pgfpoint{169.472107pt}{172.752411pt}}
\pgflineto{\pgfpoint{169.552780pt}{172.761658pt}}
\pgfusepath{stroke}
\pgfpathmoveto{\pgfpoint{169.559006pt}{178.743408pt}}
\pgflineto{\pgfpoint{169.497269pt}{178.688919pt}}
\pgfusepath{stroke}
\pgfpathmoveto{\pgfpoint{169.476898pt}{178.736862pt}}
\pgflineto{\pgfpoint{169.559006pt}{178.743408pt}}
\pgfusepath{stroke}
\pgfpathmoveto{\pgfpoint{169.564941pt}{184.724564pt}}
\pgflineto{\pgfpoint{169.500488pt}{184.671616pt}}
\pgfusepath{stroke}
\pgfpathmoveto{\pgfpoint{169.481598pt}{184.720871pt}}
\pgflineto{\pgfpoint{169.564941pt}{184.724564pt}}
\pgfusepath{stroke}
\pgfpathmoveto{\pgfpoint{169.570480pt}{190.705124pt}}
\pgflineto{\pgfpoint{169.503387pt}{190.653946pt}}
\pgfusepath{stroke}
\pgfpathmoveto{\pgfpoint{169.486084pt}{190.704437pt}}
\pgflineto{\pgfpoint{169.570480pt}{190.705124pt}}
\pgfusepath{stroke}
\pgfpathmoveto{\pgfpoint{169.575546pt}{196.685120pt}}
\pgflineto{\pgfpoint{169.505905pt}{196.635971pt}}
\pgfusepath{stroke}
\pgfpathmoveto{\pgfpoint{169.490326pt}{196.687592pt}}
\pgflineto{\pgfpoint{169.575546pt}{196.685120pt}}
\pgfusepath{stroke}
\pgfpathmoveto{\pgfpoint{169.580063pt}{202.664612pt}}
\pgflineto{\pgfpoint{169.507996pt}{202.617706pt}}
\pgfusepath{stroke}
\pgfpathmoveto{\pgfpoint{169.494263pt}{202.670334pt}}
\pgflineto{\pgfpoint{169.580063pt}{202.664612pt}}
\pgfusepath{stroke}
\pgfpathmoveto{\pgfpoint{169.583969pt}{208.643646pt}}
\pgflineto{\pgfpoint{169.509613pt}{208.599228pt}}
\pgfusepath{stroke}
\pgfpathmoveto{\pgfpoint{169.497833pt}{208.652710pt}}
\pgflineto{\pgfpoint{169.583969pt}{208.643646pt}}
\pgfusepath{stroke}
\pgfpathmoveto{\pgfpoint{169.587204pt}{214.622314pt}}
\pgflineto{\pgfpoint{169.510773pt}{214.580566pt}}
\pgfusepath{stroke}
\pgfpathmoveto{\pgfpoint{169.500992pt}{214.634766pt}}
\pgflineto{\pgfpoint{169.587204pt}{214.622314pt}}
\pgfusepath{stroke}
\pgfpathmoveto{\pgfpoint{169.589691pt}{220.600693pt}}
\pgflineto{\pgfpoint{169.511383pt}{220.561783pt}}
\pgfusepath{stroke}
\pgfpathmoveto{\pgfpoint{169.503708pt}{220.616547pt}}
\pgflineto{\pgfpoint{169.589691pt}{220.600693pt}}
\pgfusepath{stroke}
\pgfpathmoveto{\pgfpoint{169.591431pt}{226.578873pt}}
\pgflineto{\pgfpoint{169.511475pt}{226.542969pt}}
\pgfusepath{stroke}
\pgfpathmoveto{\pgfpoint{169.505920pt}{226.598114pt}}
\pgflineto{\pgfpoint{169.591431pt}{226.578873pt}}
\pgfusepath{stroke}
\pgfpathmoveto{\pgfpoint{169.592377pt}{232.556961pt}}
\pgflineto{\pgfpoint{169.511032pt}{232.524170pt}}
\pgfusepath{stroke}
\pgfpathmoveto{\pgfpoint{169.507629pt}{232.579529pt}}
\pgflineto{\pgfpoint{169.592377pt}{232.556961pt}}
\pgfusepath{stroke}
\pgfpathmoveto{\pgfpoint{169.592560pt}{238.535049pt}}
\pgflineto{\pgfpoint{169.510086pt}{238.505463pt}}
\pgfusepath{stroke}
\pgfpathmoveto{\pgfpoint{169.508820pt}{238.560852pt}}
\pgflineto{\pgfpoint{169.592560pt}{238.535049pt}}
\pgfusepath{stroke}
\pgfpathmoveto{\pgfpoint{169.591980pt}{244.513229pt}}
\pgflineto{\pgfpoint{169.508636pt}{244.486893pt}}
\pgfusepath{stroke}
\pgfpathmoveto{\pgfpoint{169.509506pt}{244.542160pt}}
\pgflineto{\pgfpoint{169.591980pt}{244.513229pt}}
\pgfusepath{stroke}
\pgfpathmoveto{\pgfpoint{169.590683pt}{250.491577pt}}
\pgflineto{\pgfpoint{169.506729pt}{250.468536pt}}
\pgfusepath{stroke}
\pgfpathmoveto{\pgfpoint{169.509689pt}{250.523514pt}}
\pgflineto{\pgfpoint{169.590683pt}{250.491577pt}}
\pgfusepath{stroke}
\pgfpathmoveto{\pgfpoint{169.588684pt}{256.470215pt}}
\pgflineto{\pgfpoint{169.504395pt}{256.450409pt}}
\pgfusepath{stroke}
\pgfpathmoveto{\pgfpoint{169.509384pt}{256.504944pt}}
\pgflineto{\pgfpoint{169.588684pt}{256.470215pt}}
\pgfusepath{stroke}
\pgfpathmoveto{\pgfpoint{169.586090pt}{262.449158pt}}
\pgflineto{\pgfpoint{169.501678pt}{262.432587pt}}
\pgfusepath{stroke}
\pgfpathmoveto{\pgfpoint{169.508606pt}{262.486542pt}}
\pgflineto{\pgfpoint{169.586090pt}{262.449158pt}}
\pgfusepath{stroke}
\pgfpathmoveto{\pgfpoint{169.582916pt}{268.428467pt}}
\pgflineto{\pgfpoint{169.498596pt}{268.415070pt}}
\pgfusepath{stroke}
\pgfpathmoveto{\pgfpoint{169.507446pt}{268.468323pt}}
\pgflineto{\pgfpoint{169.582916pt}{268.428467pt}}
\pgfusepath{stroke}
\pgfpathmoveto{\pgfpoint{169.579239pt}{274.408203pt}}
\pgflineto{\pgfpoint{169.495270pt}{274.397888pt}}
\pgfusepath{stroke}
\pgfpathmoveto{\pgfpoint{169.505859pt}{274.450317pt}}
\pgflineto{\pgfpoint{169.579239pt}{274.408203pt}}
\pgfusepath{stroke}
\pgfpathmoveto{\pgfpoint{169.575104pt}{280.388397pt}}
\pgflineto{\pgfpoint{169.491699pt}{280.381073pt}}
\pgfusepath{stroke}
\pgfpathmoveto{\pgfpoint{169.503967pt}{280.432587pt}}
\pgflineto{\pgfpoint{169.575104pt}{280.388397pt}}
\pgfusepath{stroke}
\pgfpathmoveto{\pgfpoint{169.570618pt}{286.369080pt}}
\pgflineto{\pgfpoint{169.487915pt}{286.364594pt}}
\pgfusepath{stroke}
\pgfpathmoveto{\pgfpoint{169.501755pt}{286.415131pt}}
\pgflineto{\pgfpoint{169.570618pt}{286.369080pt}}
\pgfusepath{stroke}
\pgfpathmoveto{\pgfpoint{169.565826pt}{292.350220pt}}
\pgflineto{\pgfpoint{169.483978pt}{292.348480pt}}
\pgfusepath{stroke}
\pgfpathmoveto{\pgfpoint{169.499298pt}{292.397949pt}}
\pgflineto{\pgfpoint{169.565826pt}{292.350220pt}}
\pgfusepath{stroke}
\pgfpathmoveto{\pgfpoint{169.560776pt}{298.331848pt}}
\pgflineto{\pgfpoint{169.479919pt}{298.332703pt}}
\pgfusepath{stroke}
\pgfpathmoveto{\pgfpoint{169.496597pt}{298.381042pt}}
\pgflineto{\pgfpoint{169.560776pt}{298.331848pt}}
\pgfusepath{stroke}
\pgfpathmoveto{\pgfpoint{169.555511pt}{304.313965pt}}
\pgflineto{\pgfpoint{169.475800pt}{304.317291pt}}
\pgfusepath{stroke}
\pgfpathmoveto{\pgfpoint{169.493729pt}{304.364441pt}}
\pgflineto{\pgfpoint{169.555511pt}{304.313965pt}}
\pgfusepath{stroke}
\pgfpathmoveto{\pgfpoint{169.550095pt}{310.296539pt}}
\pgflineto{\pgfpoint{169.471619pt}{310.302185pt}}
\pgfusepath{stroke}
\pgfpathmoveto{\pgfpoint{169.490692pt}{310.348145pt}}
\pgflineto{\pgfpoint{169.550095pt}{310.296539pt}}
\pgfusepath{stroke}
\pgfpathmoveto{\pgfpoint{169.544586pt}{316.279602pt}}
\pgflineto{\pgfpoint{169.467407pt}{316.287415pt}}
\pgfusepath{stroke}
\pgfpathmoveto{\pgfpoint{169.487549pt}{316.332153pt}}
\pgflineto{\pgfpoint{169.544586pt}{316.279602pt}}
\pgfusepath{stroke}
\pgfpathmoveto{\pgfpoint{169.538986pt}{322.263062pt}}
\pgflineto{\pgfpoint{169.463196pt}{322.272949pt}}
\pgfusepath{stroke}
\pgfpathmoveto{\pgfpoint{169.484283pt}{322.316437pt}}
\pgflineto{\pgfpoint{169.538986pt}{322.263062pt}}
\pgfusepath{stroke}
\pgfpathmoveto{\pgfpoint{169.533356pt}{328.246948pt}}
\pgflineto{\pgfpoint{169.459030pt}{328.258728pt}}
\pgfusepath{stroke}
\pgfpathmoveto{\pgfpoint{169.480972pt}{328.300995pt}}
\pgflineto{\pgfpoint{169.533356pt}{328.246948pt}}
\pgfusepath{stroke}
\pgfpathmoveto{\pgfpoint{169.527740pt}{334.231262pt}}
\pgflineto{\pgfpoint{169.454895pt}{334.244812pt}}
\pgfusepath{stroke}
\pgfpathmoveto{\pgfpoint{169.477600pt}{334.285828pt}}
\pgflineto{\pgfpoint{169.527740pt}{334.231262pt}}
\pgfusepath{stroke}
\pgfpathmoveto{\pgfpoint{169.522125pt}{340.215942pt}}
\pgflineto{\pgfpoint{169.450806pt}{340.231140pt}}
\pgfusepath{stroke}
\pgfpathmoveto{\pgfpoint{169.474213pt}{340.270905pt}}
\pgflineto{\pgfpoint{169.522125pt}{340.215942pt}}
\pgfusepath{stroke}
\pgfpathmoveto{\pgfpoint{169.516571pt}{346.200958pt}}
\pgflineto{\pgfpoint{169.446808pt}{346.217712pt}}
\pgfusepath{stroke}
\pgfpathmoveto{\pgfpoint{169.470795pt}{346.256226pt}}
\pgflineto{\pgfpoint{169.516571pt}{346.200958pt}}
\pgfusepath{stroke}
\pgfpathmoveto{\pgfpoint{169.511078pt}{352.186340pt}}
\pgflineto{\pgfpoint{169.442871pt}{352.204498pt}}
\pgfusepath{stroke}
\pgfpathmoveto{\pgfpoint{169.467407pt}{352.241760pt}}
\pgflineto{\pgfpoint{169.511078pt}{352.186340pt}}
\pgfusepath{stroke}
\pgfpathmoveto{\pgfpoint{169.505646pt}{358.172058pt}}
\pgflineto{\pgfpoint{169.439026pt}{358.191467pt}}
\pgfusepath{stroke}
\pgfpathmoveto{\pgfpoint{169.464020pt}{358.227600pt}}
\pgflineto{\pgfpoint{169.505646pt}{358.172058pt}}
\pgfusepath{stroke}
\pgfpathmoveto{\pgfpoint{169.500336pt}{364.158020pt}}
\pgflineto{\pgfpoint{169.435287pt}{364.178650pt}}
\pgfusepath{stroke}
\pgfpathmoveto{\pgfpoint{169.460663pt}{364.213562pt}}
\pgflineto{\pgfpoint{169.500336pt}{364.158020pt}}
\pgfusepath{stroke}
\pgfpathmoveto{\pgfpoint{169.495117pt}{370.144348pt}}
\pgflineto{\pgfpoint{169.431641pt}{370.166016pt}}
\pgfusepath{stroke}
\pgfpathmoveto{\pgfpoint{169.457367pt}{370.199768pt}}
\pgflineto{\pgfpoint{169.495117pt}{370.144348pt}}
\pgfusepath{stroke}
\pgfpathmoveto{\pgfpoint{175.444077pt}{76.992126pt}}
\pgflineto{\pgfpoint{175.421066pt}{76.935532pt}}
\pgfusepath{stroke}
\pgfpathmoveto{\pgfpoint{175.391708pt}{76.960632pt}}
\pgflineto{\pgfpoint{175.444077pt}{76.992126pt}}
\pgfusepath{stroke}
\pgfpathmoveto{\pgfpoint{175.448898pt}{82.980560pt}}
\pgflineto{\pgfpoint{175.424347pt}{82.923203pt}}
\pgfusepath{stroke}
\pgfpathmoveto{\pgfpoint{175.394836pt}{82.949402pt}}
\pgflineto{\pgfpoint{175.448898pt}{82.980560pt}}
\pgfusepath{stroke}
\pgfpathmoveto{\pgfpoint{175.453964pt}{88.968849pt}}
\pgflineto{\pgfpoint{175.427780pt}{88.910736pt}}
\pgfusepath{stroke}
\pgfpathmoveto{\pgfpoint{175.398163pt}{88.938057pt}}
\pgflineto{\pgfpoint{175.453964pt}{88.968849pt}}
\pgfusepath{stroke}
\pgfpathmoveto{\pgfpoint{175.459259pt}{94.956932pt}}
\pgflineto{\pgfpoint{175.431366pt}{94.898125pt}}
\pgfusepath{stroke}
\pgfpathmoveto{\pgfpoint{175.401657pt}{94.926628pt}}
\pgflineto{\pgfpoint{175.459259pt}{94.956932pt}}
\pgfusepath{stroke}
\pgfpathmoveto{\pgfpoint{175.464813pt}{100.944824pt}}
\pgflineto{\pgfpoint{175.435074pt}{100.885338pt}}
\pgfusepath{stroke}
\pgfpathmoveto{\pgfpoint{175.405350pt}{100.915070pt}}
\pgflineto{\pgfpoint{175.464813pt}{100.944824pt}}
\pgfusepath{stroke}
\pgfpathmoveto{\pgfpoint{175.470612pt}{106.932434pt}}
\pgflineto{\pgfpoint{175.438934pt}{106.872368pt}}
\pgfusepath{stroke}
\pgfpathmoveto{\pgfpoint{175.409210pt}{106.903397pt}}
\pgflineto{\pgfpoint{175.470612pt}{106.932434pt}}
\pgfusepath{stroke}
\pgfpathmoveto{\pgfpoint{175.476654pt}{112.919769pt}}
\pgflineto{\pgfpoint{175.442902pt}{112.859169pt}}
\pgfusepath{stroke}
\pgfpathmoveto{\pgfpoint{175.413300pt}{112.891541pt}}
\pgflineto{\pgfpoint{175.476654pt}{112.919769pt}}
\pgfusepath{stroke}
\pgfpathmoveto{\pgfpoint{175.482941pt}{118.906792pt}}
\pgflineto{\pgfpoint{175.447021pt}{118.845741pt}}
\pgfusepath{stroke}
\pgfpathmoveto{\pgfpoint{175.417572pt}{118.879501pt}}
\pgflineto{\pgfpoint{175.482941pt}{118.906792pt}}
\pgfusepath{stroke}
\pgfpathmoveto{\pgfpoint{175.489441pt}{124.893448pt}}
\pgflineto{\pgfpoint{175.451248pt}{124.832039pt}}
\pgfusepath{stroke}
\pgfpathmoveto{\pgfpoint{175.422058pt}{124.867249pt}}
\pgflineto{\pgfpoint{175.489441pt}{124.893448pt}}
\pgfusepath{stroke}
\pgfpathmoveto{\pgfpoint{175.496201pt}{130.879700pt}}
\pgflineto{\pgfpoint{175.455566pt}{130.818039pt}}
\pgfusepath{stroke}
\pgfpathmoveto{\pgfpoint{175.426697pt}{130.854752pt}}
\pgflineto{\pgfpoint{175.496201pt}{130.879700pt}}
\pgfusepath{stroke}
\pgfpathmoveto{\pgfpoint{175.503128pt}{136.865509pt}}
\pgflineto{\pgfpoint{175.459961pt}{136.803711pt}}
\pgfusepath{stroke}
\pgfpathmoveto{\pgfpoint{175.431534pt}{136.841965pt}}
\pgflineto{\pgfpoint{175.503128pt}{136.865509pt}}
\pgfusepath{stroke}
\pgfpathmoveto{\pgfpoint{175.510208pt}{142.850800pt}}
\pgflineto{\pgfpoint{175.464417pt}{142.789062pt}}
\pgfusepath{stroke}
\pgfpathmoveto{\pgfpoint{175.436523pt}{142.828873pt}}
\pgflineto{\pgfpoint{175.510208pt}{142.850800pt}}
\pgfusepath{stroke}
\pgfpathmoveto{\pgfpoint{175.517410pt}{148.835541pt}}
\pgflineto{\pgfpoint{175.468887pt}{148.774017pt}}
\pgfusepath{stroke}
\pgfpathmoveto{\pgfpoint{175.441650pt}{148.815430pt}}
\pgflineto{\pgfpoint{175.517410pt}{148.835541pt}}
\pgfusepath{stroke}
\pgfpathmoveto{\pgfpoint{175.524689pt}{154.819717pt}}
\pgflineto{\pgfpoint{175.473343pt}{154.758560pt}}
\pgfusepath{stroke}
\pgfpathmoveto{\pgfpoint{175.446930pt}{154.801605pt}}
\pgflineto{\pgfpoint{175.524689pt}{154.819717pt}}
\pgfusepath{stroke}
\pgfpathmoveto{\pgfpoint{175.531982pt}{160.803253pt}}
\pgflineto{\pgfpoint{175.477753pt}{160.742706pt}}
\pgfusepath{stroke}
\pgfpathmoveto{\pgfpoint{175.452255pt}{160.787354pt}}
\pgflineto{\pgfpoint{175.531982pt}{160.803253pt}}
\pgfusepath{stroke}
\pgfpathmoveto{\pgfpoint{175.539230pt}{166.786133pt}}
\pgflineto{\pgfpoint{175.482025pt}{166.726395pt}}
\pgfusepath{stroke}
\pgfpathmoveto{\pgfpoint{175.457611pt}{166.772675pt}}
\pgflineto{\pgfpoint{175.539230pt}{166.786133pt}}
\pgfusepath{stroke}
\pgfpathmoveto{\pgfpoint{175.546341pt}{172.768326pt}}
\pgflineto{\pgfpoint{175.486145pt}{172.709656pt}}
\pgfusepath{stroke}
\pgfpathmoveto{\pgfpoint{175.462982pt}{172.757507pt}}
\pgflineto{\pgfpoint{175.546341pt}{172.768326pt}}
\pgfusepath{stroke}
\pgfpathmoveto{\pgfpoint{175.553223pt}{178.749832pt}}
\pgflineto{\pgfpoint{175.490051pt}{178.692474pt}}
\pgfusepath{stroke}
\pgfpathmoveto{\pgfpoint{175.468262pt}{178.741852pt}}
\pgflineto{\pgfpoint{175.553223pt}{178.749832pt}}
\pgfusepath{stroke}
\pgfpathmoveto{\pgfpoint{175.559799pt}{184.730621pt}}
\pgflineto{\pgfpoint{175.493652pt}{184.674866pt}}
\pgfusepath{stroke}
\pgfpathmoveto{\pgfpoint{175.473419pt}{184.725693pt}}
\pgflineto{\pgfpoint{175.559799pt}{184.730621pt}}
\pgfusepath{stroke}
\pgfpathmoveto{\pgfpoint{175.565948pt}{190.710739pt}}
\pgflineto{\pgfpoint{175.496918pt}{190.656830pt}}
\pgfusepath{stroke}
\pgfpathmoveto{\pgfpoint{175.478363pt}{190.709045pt}}
\pgflineto{\pgfpoint{175.565948pt}{190.710739pt}}
\pgfusepath{stroke}
\pgfpathmoveto{\pgfpoint{175.571609pt}{196.690201pt}}
\pgflineto{\pgfpoint{175.499756pt}{196.638428pt}}
\pgfusepath{stroke}
\pgfpathmoveto{\pgfpoint{175.483063pt}{196.691895pt}}
\pgflineto{\pgfpoint{175.571609pt}{196.690201pt}}
\pgfusepath{stroke}
\pgfpathmoveto{\pgfpoint{175.576645pt}{202.669067pt}}
\pgflineto{\pgfpoint{175.502136pt}{202.619705pt}}
\pgfusepath{stroke}
\pgfpathmoveto{\pgfpoint{175.487411pt}{202.674286pt}}
\pgflineto{\pgfpoint{175.576645pt}{202.669067pt}}
\pgfusepath{stroke}
\pgfpathmoveto{\pgfpoint{175.580994pt}{208.647415pt}}
\pgflineto{\pgfpoint{175.503967pt}{208.600708pt}}
\pgfusepath{stroke}
\pgfpathmoveto{\pgfpoint{175.491364pt}{208.656250pt}}
\pgflineto{\pgfpoint{175.580994pt}{208.647415pt}}
\pgfusepath{stroke}
\pgfpathmoveto{\pgfpoint{175.584579pt}{214.625336pt}}
\pgflineto{\pgfpoint{175.505280pt}{214.581497pt}}
\pgfusepath{stroke}
\pgfpathmoveto{\pgfpoint{175.494843pt}{214.637848pt}}
\pgflineto{\pgfpoint{175.584579pt}{214.625336pt}}
\pgfusepath{stroke}
\pgfpathmoveto{\pgfpoint{175.587341pt}{220.602905pt}}
\pgflineto{\pgfpoint{175.505981pt}{220.562164pt}}
\pgfusepath{stroke}
\pgfpathmoveto{\pgfpoint{175.497803pt}{220.619125pt}}
\pgflineto{\pgfpoint{175.587341pt}{220.602905pt}}
\pgfusepath{stroke}
\pgfpathmoveto{\pgfpoint{175.589218pt}{226.580276pt}}
\pgflineto{\pgfpoint{175.506088pt}{226.542786pt}}
\pgfusepath{stroke}
\pgfpathmoveto{\pgfpoint{175.500214pt}{226.600159pt}}
\pgflineto{\pgfpoint{175.589218pt}{226.580276pt}}
\pgfusepath{stroke}
\pgfpathmoveto{\pgfpoint{175.590225pt}{232.557526pt}}
\pgflineto{\pgfpoint{175.505585pt}{232.523422pt}}
\pgfusepath{stroke}
\pgfpathmoveto{\pgfpoint{175.502045pt}{232.581024pt}}
\pgflineto{\pgfpoint{175.590225pt}{232.557526pt}}
\pgfusepath{stroke}
\pgfpathmoveto{\pgfpoint{175.590363pt}{238.534790pt}}
\pgflineto{\pgfpoint{175.504517pt}{238.504181pt}}
\pgfusepath{stroke}
\pgfpathmoveto{\pgfpoint{175.503296pt}{238.561798pt}}
\pgflineto{\pgfpoint{175.590363pt}{238.534790pt}}
\pgfusepath{stroke}
\pgfpathmoveto{\pgfpoint{175.589615pt}{244.512177pt}}
\pgflineto{\pgfpoint{175.502869pt}{244.485092pt}}
\pgfusepath{stroke}
\pgfpathmoveto{\pgfpoint{175.503967pt}{244.542572pt}}
\pgflineto{\pgfpoint{175.589615pt}{244.512177pt}}
\pgfusepath{stroke}
\pgfpathmoveto{\pgfpoint{175.588074pt}{250.489792pt}}
\pgflineto{\pgfpoint{175.500702pt}{250.466263pt}}
\pgfusepath{stroke}
\pgfpathmoveto{\pgfpoint{175.504059pt}{250.523392pt}}
\pgflineto{\pgfpoint{175.588074pt}{250.489792pt}}
\pgfusepath{stroke}
\pgfpathmoveto{\pgfpoint{175.585754pt}{256.467712pt}}
\pgflineto{\pgfpoint{175.498077pt}{256.447723pt}}
\pgfusepath{stroke}
\pgfpathmoveto{\pgfpoint{175.503616pt}{256.504333pt}}
\pgflineto{\pgfpoint{175.585754pt}{256.467712pt}}
\pgfusepath{stroke}
\pgfpathmoveto{\pgfpoint{175.582779pt}{262.446045pt}}
\pgflineto{\pgfpoint{175.495026pt}{262.429504pt}}
\pgfusepath{stroke}
\pgfpathmoveto{\pgfpoint{175.502655pt}{262.485474pt}}
\pgflineto{\pgfpoint{175.582779pt}{262.446045pt}}
\pgfusepath{stroke}
\pgfpathmoveto{\pgfpoint{175.579147pt}{268.424805pt}}
\pgflineto{\pgfpoint{175.491623pt}{268.411682pt}}
\pgfusepath{stroke}
\pgfpathmoveto{\pgfpoint{175.501251pt}{268.466827pt}}
\pgflineto{\pgfpoint{175.579147pt}{268.424805pt}}
\pgfusepath{stroke}
\pgfpathmoveto{\pgfpoint{175.574982pt}{274.404053pt}}
\pgflineto{\pgfpoint{175.487900pt}{274.394257pt}}
\pgfusepath{stroke}
\pgfpathmoveto{\pgfpoint{175.499435pt}{274.448456pt}}
\pgflineto{\pgfpoint{175.574982pt}{274.404053pt}}
\pgfusepath{stroke}
\pgfpathmoveto{\pgfpoint{175.570389pt}{280.383850pt}}
\pgflineto{\pgfpoint{175.483948pt}{280.377197pt}}
\pgfusepath{stroke}
\pgfpathmoveto{\pgfpoint{175.497238pt}{280.430389pt}}
\pgflineto{\pgfpoint{175.570389pt}{280.383850pt}}
\pgfusepath{stroke}
\pgfpathmoveto{\pgfpoint{175.565369pt}{286.364166pt}}
\pgflineto{\pgfpoint{175.479797pt}{286.360535pt}}
\pgfusepath{stroke}
\pgfpathmoveto{\pgfpoint{175.494720pt}{286.412628pt}}
\pgflineto{\pgfpoint{175.565369pt}{286.364166pt}}
\pgfusepath{stroke}
\pgfpathmoveto{\pgfpoint{175.560059pt}{292.345032pt}}
\pgflineto{\pgfpoint{175.475479pt}{292.344299pt}}
\pgfusepath{stroke}
\pgfpathmoveto{\pgfpoint{175.491959pt}{292.395203pt}}
\pgflineto{\pgfpoint{175.560059pt}{292.345032pt}}
\pgfusepath{stroke}
\pgfpathmoveto{\pgfpoint{175.554504pt}{298.326447pt}}
\pgflineto{\pgfpoint{175.471054pt}{298.328430pt}}
\pgfusepath{stroke}
\pgfpathmoveto{\pgfpoint{175.488953pt}{298.378113pt}}
\pgflineto{\pgfpoint{175.554504pt}{298.326447pt}}
\pgfusepath{stroke}
\pgfpathmoveto{\pgfpoint{175.548737pt}{304.308380pt}}
\pgflineto{\pgfpoint{175.466583pt}{304.312958pt}}
\pgfusepath{stroke}
\pgfpathmoveto{\pgfpoint{175.485764pt}{304.361328pt}}
\pgflineto{\pgfpoint{175.548737pt}{304.308380pt}}
\pgfusepath{stroke}
\pgfpathmoveto{\pgfpoint{175.542847pt}{310.290833pt}}
\pgflineto{\pgfpoint{175.462067pt}{310.297852pt}}
\pgfusepath{stroke}
\pgfpathmoveto{\pgfpoint{175.482422pt}{310.344910pt}}
\pgflineto{\pgfpoint{175.542847pt}{310.290833pt}}
\pgfusepath{stroke}
\pgfpathmoveto{\pgfpoint{175.536865pt}{316.273804pt}}
\pgflineto{\pgfpoint{175.457535pt}{316.283051pt}}
\pgfusepath{stroke}
\pgfpathmoveto{\pgfpoint{175.478973pt}{316.328796pt}}
\pgflineto{\pgfpoint{175.536865pt}{316.273804pt}}
\pgfusepath{stroke}
\pgfpathmoveto{\pgfpoint{175.530823pt}{322.257233pt}}
\pgflineto{\pgfpoint{175.453033pt}{322.268616pt}}
\pgfusepath{stroke}
\pgfpathmoveto{\pgfpoint{175.475403pt}{322.313019pt}}
\pgflineto{\pgfpoint{175.530823pt}{322.257233pt}}
\pgfusepath{stroke}
\pgfpathmoveto{\pgfpoint{175.524780pt}{328.241119pt}}
\pgflineto{\pgfpoint{175.448563pt}{328.254456pt}}
\pgfusepath{stroke}
\pgfpathmoveto{\pgfpoint{175.471817pt}{328.297516pt}}
\pgflineto{\pgfpoint{175.524780pt}{328.241119pt}}
\pgfusepath{stroke}
\pgfpathmoveto{\pgfpoint{175.518753pt}{334.225433pt}}
\pgflineto{\pgfpoint{175.444168pt}{334.240601pt}}
\pgfusepath{stroke}
\pgfpathmoveto{\pgfpoint{175.468170pt}{334.282288pt}}
\pgflineto{\pgfpoint{175.518753pt}{334.225433pt}}
\pgfusepath{stroke}
\pgfpathmoveto{\pgfpoint{175.512756pt}{340.210144pt}}
\pgflineto{\pgfpoint{175.439835pt}{340.226990pt}}
\pgfusepath{stroke}
\pgfpathmoveto{\pgfpoint{175.464508pt}{340.267365pt}}
\pgflineto{\pgfpoint{175.512756pt}{340.210144pt}}
\pgfusepath{stroke}
\pgfpathmoveto{\pgfpoint{175.506836pt}{346.195251pt}}
\pgflineto{\pgfpoint{175.435593pt}{346.213623pt}}
\pgfusepath{stroke}
\pgfpathmoveto{\pgfpoint{175.460861pt}{346.252716pt}}
\pgflineto{\pgfpoint{175.506836pt}{346.195251pt}}
\pgfusepath{stroke}
\pgfpathmoveto{\pgfpoint{175.500992pt}{352.180725pt}}
\pgflineto{\pgfpoint{175.431458pt}{352.200500pt}}
\pgfusepath{stroke}
\pgfpathmoveto{\pgfpoint{175.457214pt}{352.238281pt}}
\pgflineto{\pgfpoint{175.500992pt}{352.180725pt}}
\pgfusepath{stroke}
\pgfpathmoveto{\pgfpoint{175.495270pt}{358.166534pt}}
\pgflineto{\pgfpoint{175.427399pt}{358.187592pt}}
\pgfusepath{stroke}
\pgfpathmoveto{\pgfpoint{175.453613pt}{358.224091pt}}
\pgflineto{\pgfpoint{175.495270pt}{358.166534pt}}
\pgfusepath{stroke}
\pgfpathmoveto{\pgfpoint{175.489639pt}{364.152649pt}}
\pgflineto{\pgfpoint{175.423477pt}{364.174866pt}}
\pgfusepath{stroke}
\pgfpathmoveto{\pgfpoint{175.450058pt}{364.210144pt}}
\pgflineto{\pgfpoint{175.489639pt}{364.152649pt}}
\pgfusepath{stroke}
\pgfpathmoveto{\pgfpoint{175.484161pt}{370.139038pt}}
\pgflineto{\pgfpoint{175.419678pt}{370.162354pt}}
\pgfusepath{stroke}
\pgfpathmoveto{\pgfpoint{175.446548pt}{370.196350pt}}
\pgflineto{\pgfpoint{175.484161pt}{370.139038pt}}
\pgfusepath{stroke}
\pgfpathmoveto{\pgfpoint{181.430954pt}{76.996902pt}}
\pgflineto{\pgfpoint{181.408768pt}{76.938797pt}}
\pgfusepath{stroke}
\pgfpathmoveto{\pgfpoint{181.378357pt}{76.963715pt}}
\pgflineto{\pgfpoint{181.430954pt}{76.996902pt}}
\pgfusepath{stroke}
\pgfpathmoveto{\pgfpoint{181.435913pt}{82.985580pt}}
\pgflineto{\pgfpoint{181.412186pt}{82.926620pt}}
\pgfusepath{stroke}
\pgfpathmoveto{\pgfpoint{181.381561pt}{82.952652pt}}
\pgflineto{\pgfpoint{181.435913pt}{82.985580pt}}
\pgfusepath{stroke}
\pgfpathmoveto{\pgfpoint{181.441162pt}{88.974121pt}}
\pgflineto{\pgfpoint{181.415756pt}{88.914314pt}}
\pgfusepath{stroke}
\pgfpathmoveto{\pgfpoint{181.384949pt}{88.941505pt}}
\pgflineto{\pgfpoint{181.441162pt}{88.974121pt}}
\pgfusepath{stroke}
\pgfpathmoveto{\pgfpoint{181.446655pt}{94.962471pt}}
\pgflineto{\pgfpoint{181.419495pt}{94.901855pt}}
\pgfusepath{stroke}
\pgfpathmoveto{\pgfpoint{181.388550pt}{94.930267pt}}
\pgflineto{\pgfpoint{181.446655pt}{94.962471pt}}
\pgfusepath{stroke}
\pgfpathmoveto{\pgfpoint{181.452438pt}{100.950615pt}}
\pgflineto{\pgfpoint{181.423401pt}{100.889214pt}}
\pgfusepath{stroke}
\pgfpathmoveto{\pgfpoint{181.392365pt}{100.918930pt}}
\pgflineto{\pgfpoint{181.452438pt}{100.950615pt}}
\pgfusepath{stroke}
\pgfpathmoveto{\pgfpoint{181.458496pt}{106.938499pt}}
\pgflineto{\pgfpoint{181.427460pt}{106.876396pt}}
\pgfusepath{stroke}
\pgfpathmoveto{\pgfpoint{181.396393pt}{106.907448pt}}
\pgflineto{\pgfpoint{181.458496pt}{106.938499pt}}
\pgfusepath{stroke}
\pgfpathmoveto{\pgfpoint{181.464859pt}{112.926109pt}}
\pgflineto{\pgfpoint{181.431671pt}{112.863342pt}}
\pgfusepath{stroke}
\pgfpathmoveto{\pgfpoint{181.400635pt}{112.895813pt}}
\pgflineto{\pgfpoint{181.464859pt}{112.926109pt}}
\pgfusepath{stroke}
\pgfpathmoveto{\pgfpoint{181.471481pt}{118.913383pt}}
\pgflineto{\pgfpoint{181.436035pt}{118.850037pt}}
\pgfusepath{stroke}
\pgfpathmoveto{\pgfpoint{181.405121pt}{118.883987pt}}
\pgflineto{\pgfpoint{181.471481pt}{118.913383pt}}
\pgfusepath{stroke}
\pgfpathmoveto{\pgfpoint{181.478394pt}{124.900276pt}}
\pgflineto{\pgfpoint{181.440552pt}{124.836464pt}}
\pgfusepath{stroke}
\pgfpathmoveto{\pgfpoint{181.409821pt}{124.871933pt}}
\pgflineto{\pgfpoint{181.478394pt}{124.900276pt}}
\pgfusepath{stroke}
\pgfpathmoveto{\pgfpoint{181.485565pt}{130.886765pt}}
\pgflineto{\pgfpoint{181.445190pt}{130.822571pt}}
\pgfusepath{stroke}
\pgfpathmoveto{\pgfpoint{181.414749pt}{130.859634pt}}
\pgflineto{\pgfpoint{181.485565pt}{130.886765pt}}
\pgfusepath{stroke}
\pgfpathmoveto{\pgfpoint{181.492981pt}{136.872772pt}}
\pgflineto{\pgfpoint{181.449921pt}{136.808334pt}}
\pgfusepath{stroke}
\pgfpathmoveto{\pgfpoint{181.419891pt}{136.847046pt}}
\pgflineto{\pgfpoint{181.492981pt}{136.872772pt}}
\pgfusepath{stroke}
\pgfpathmoveto{\pgfpoint{181.500610pt}{142.858215pt}}
\pgflineto{\pgfpoint{181.454773pt}{142.793701pt}}
\pgfusepath{stroke}
\pgfpathmoveto{\pgfpoint{181.425232pt}{142.834122pt}}
\pgflineto{\pgfpoint{181.500610pt}{142.858215pt}}
\pgfusepath{stroke}
\pgfpathmoveto{\pgfpoint{181.508408pt}{148.843109pt}}
\pgflineto{\pgfpoint{181.459656pt}{148.778687pt}}
\pgfusepath{stroke}
\pgfpathmoveto{\pgfpoint{181.430756pt}{148.820831pt}}
\pgflineto{\pgfpoint{181.508408pt}{148.843109pt}}
\pgfusepath{stroke}
\pgfpathmoveto{\pgfpoint{181.516327pt}{154.827332pt}}
\pgflineto{\pgfpoint{181.464539pt}{154.763214pt}}
\pgfusepath{stroke}
\pgfpathmoveto{\pgfpoint{181.436401pt}{154.807129pt}}
\pgflineto{\pgfpoint{181.516327pt}{154.827332pt}}
\pgfusepath{stroke}
\pgfpathmoveto{\pgfpoint{181.524292pt}{160.810883pt}}
\pgflineto{\pgfpoint{181.469376pt}{160.747284pt}}
\pgfusepath{stroke}
\pgfpathmoveto{\pgfpoint{181.442200pt}{160.792953pt}}
\pgflineto{\pgfpoint{181.524292pt}{160.810883pt}}
\pgfusepath{stroke}
\pgfpathmoveto{\pgfpoint{181.532257pt}{166.793686pt}}
\pgflineto{\pgfpoint{181.474136pt}{166.730835pt}}
\pgfusepath{stroke}
\pgfpathmoveto{\pgfpoint{181.448059pt}{166.778290pt}}
\pgflineto{\pgfpoint{181.532257pt}{166.793686pt}}
\pgfusepath{stroke}
\pgfpathmoveto{\pgfpoint{181.540115pt}{172.775711pt}}
\pgflineto{\pgfpoint{181.478729pt}{172.713898pt}}
\pgfusepath{stroke}
\pgfpathmoveto{\pgfpoint{181.453918pt}{172.763092pt}}
\pgflineto{\pgfpoint{181.540115pt}{172.775711pt}}
\pgfusepath{stroke}
\pgfpathmoveto{\pgfpoint{181.547760pt}{178.756943pt}}
\pgflineto{\pgfpoint{181.483093pt}{178.696457pt}}
\pgfusepath{stroke}
\pgfpathmoveto{\pgfpoint{181.459747pt}{178.747345pt}}
\pgflineto{\pgfpoint{181.547760pt}{178.756943pt}}
\pgfusepath{stroke}
\pgfpathmoveto{\pgfpoint{181.555054pt}{184.737366pt}}
\pgflineto{\pgfpoint{181.487152pt}{184.678513pt}}
\pgfusepath{stroke}
\pgfpathmoveto{\pgfpoint{181.465424pt}{184.731018pt}}
\pgflineto{\pgfpoint{181.555054pt}{184.737366pt}}
\pgfusepath{stroke}
\pgfpathmoveto{\pgfpoint{181.561951pt}{190.716995pt}}
\pgflineto{\pgfpoint{181.490845pt}{190.660095pt}}
\pgfusepath{stroke}
\pgfpathmoveto{\pgfpoint{181.470917pt}{190.714142pt}}
\pgflineto{\pgfpoint{181.561951pt}{190.716995pt}}
\pgfusepath{stroke}
\pgfpathmoveto{\pgfpoint{181.568268pt}{196.695877pt}}
\pgflineto{\pgfpoint{181.494080pt}{196.641220pt}}
\pgfusepath{stroke}
\pgfpathmoveto{\pgfpoint{181.476135pt}{196.696671pt}}
\pgflineto{\pgfpoint{181.568268pt}{196.695877pt}}
\pgfusepath{stroke}
\pgfpathmoveto{\pgfpoint{181.573914pt}{202.674057pt}}
\pgflineto{\pgfpoint{181.496765pt}{202.621979pt}}
\pgfusepath{stroke}
\pgfpathmoveto{\pgfpoint{181.480957pt}{202.678665pt}}
\pgflineto{\pgfpoint{181.573914pt}{202.674057pt}}
\pgfusepath{stroke}
\pgfpathmoveto{\pgfpoint{181.578796pt}{208.651627pt}}
\pgflineto{\pgfpoint{181.498901pt}{208.602386pt}}
\pgfusepath{stroke}
\pgfpathmoveto{\pgfpoint{181.485336pt}{208.660172pt}}
\pgflineto{\pgfpoint{181.578796pt}{208.651627pt}}
\pgfusepath{stroke}
\pgfpathmoveto{\pgfpoint{181.582794pt}{214.628677pt}}
\pgflineto{\pgfpoint{181.500366pt}{214.582581pt}}
\pgfusepath{stroke}
\pgfpathmoveto{\pgfpoint{181.489197pt}{214.641251pt}}
\pgflineto{\pgfpoint{181.582794pt}{214.628677pt}}
\pgfusepath{stroke}
\pgfpathmoveto{\pgfpoint{181.585846pt}{220.605347pt}}
\pgflineto{\pgfpoint{181.501175pt}{220.562592pt}}
\pgfusepath{stroke}
\pgfpathmoveto{\pgfpoint{181.492462pt}{220.621948pt}}
\pgflineto{\pgfpoint{181.585846pt}{220.605347pt}}
\pgfusepath{stroke}
\pgfpathmoveto{\pgfpoint{181.587906pt}{226.581772pt}}
\pgflineto{\pgfpoint{181.501282pt}{226.542572pt}}
\pgfusepath{stroke}
\pgfpathmoveto{\pgfpoint{181.495087pt}{226.602371pt}}
\pgflineto{\pgfpoint{181.587906pt}{226.581772pt}}
\pgfusepath{stroke}
\pgfpathmoveto{\pgfpoint{181.588959pt}{232.558075pt}}
\pgflineto{\pgfpoint{181.500717pt}{232.522568pt}}
\pgfusepath{stroke}
\pgfpathmoveto{\pgfpoint{181.497070pt}{232.582611pt}}
\pgflineto{\pgfpoint{181.588959pt}{232.558075pt}}
\pgfusepath{stroke}
\pgfpathmoveto{\pgfpoint{181.589005pt}{238.534409pt}}
\pgflineto{\pgfpoint{181.499481pt}{238.502716pt}}
\pgfusepath{stroke}
\pgfpathmoveto{\pgfpoint{181.498367pt}{238.562759pt}}
\pgflineto{\pgfpoint{181.589005pt}{238.534409pt}}
\pgfusepath{stroke}
\pgfpathmoveto{\pgfpoint{181.588074pt}{244.510910pt}}
\pgflineto{\pgfpoint{181.497604pt}{244.483063pt}}
\pgfusepath{stroke}
\pgfpathmoveto{\pgfpoint{181.498993pt}{244.542908pt}}
\pgflineto{\pgfpoint{181.588074pt}{244.510910pt}}
\pgfusepath{stroke}
\pgfpathmoveto{\pgfpoint{181.586212pt}{250.487686pt}}
\pgflineto{\pgfpoint{181.495148pt}{250.463699pt}}
\pgfusepath{stroke}
\pgfpathmoveto{\pgfpoint{181.498962pt}{250.523132pt}}
\pgflineto{\pgfpoint{181.586212pt}{250.487686pt}}
\pgfusepath{stroke}
\pgfpathmoveto{\pgfpoint{181.583527pt}{256.464844pt}}
\pgflineto{\pgfpoint{181.492172pt}{256.444702pt}}
\pgfusepath{stroke}
\pgfpathmoveto{\pgfpoint{181.498352pt}{256.503540pt}}
\pgflineto{\pgfpoint{181.583527pt}{256.464844pt}}
\pgfusepath{stroke}
\pgfpathmoveto{\pgfpoint{181.580048pt}{262.442474pt}}
\pgflineto{\pgfpoint{181.488754pt}{262.426086pt}}
\pgfusepath{stroke}
\pgfpathmoveto{\pgfpoint{181.497162pt}{262.484161pt}}
\pgflineto{\pgfpoint{181.580048pt}{262.442474pt}}
\pgfusepath{stroke}
\pgfpathmoveto{\pgfpoint{181.575928pt}{268.420654pt}}
\pgflineto{\pgfpoint{181.484924pt}{268.407898pt}}
\pgfusepath{stroke}
\pgfpathmoveto{\pgfpoint{181.495483pt}{268.465057pt}}
\pgflineto{\pgfpoint{181.575928pt}{268.420654pt}}
\pgfusepath{stroke}
\pgfpathmoveto{\pgfpoint{181.571228pt}{274.399384pt}}
\pgflineto{\pgfpoint{181.480804pt}{274.390198pt}}
\pgfusepath{stroke}
\pgfpathmoveto{\pgfpoint{181.493347pt}{274.446289pt}}
\pgflineto{\pgfpoint{181.571228pt}{274.399384pt}}
\pgfusepath{stroke}
\pgfpathmoveto{\pgfpoint{181.566040pt}{280.378754pt}}
\pgflineto{\pgfpoint{181.476410pt}{280.372925pt}}
\pgfusepath{stroke}
\pgfpathmoveto{\pgfpoint{181.490829pt}{280.427856pt}}
\pgflineto{\pgfpoint{181.566040pt}{280.378754pt}}
\pgfusepath{stroke}
\pgfpathmoveto{\pgfpoint{181.560455pt}{286.358704pt}}
\pgflineto{\pgfpoint{181.471817pt}{286.356079pt}}
\pgfusepath{stroke}
\pgfpathmoveto{\pgfpoint{181.487976pt}{286.409790pt}}
\pgflineto{\pgfpoint{181.560455pt}{286.358704pt}}
\pgfusepath{stroke}
\pgfpathmoveto{\pgfpoint{181.554565pt}{292.339294pt}}
\pgflineto{\pgfpoint{181.467087pt}{292.339722pt}}
\pgfusepath{stroke}
\pgfpathmoveto{\pgfpoint{181.484833pt}{292.392120pt}}
\pgflineto{\pgfpoint{181.554565pt}{292.339294pt}}
\pgfusepath{stroke}
\pgfpathmoveto{\pgfpoint{181.548431pt}{298.320496pt}}
\pgflineto{\pgfpoint{181.462280pt}{298.323792pt}}
\pgfusepath{stroke}
\pgfpathmoveto{\pgfpoint{181.481476pt}{298.374817pt}}
\pgflineto{\pgfpoint{181.548431pt}{298.320496pt}}
\pgfusepath{stroke}
\pgfpathmoveto{\pgfpoint{181.542114pt}{304.302277pt}}
\pgflineto{\pgfpoint{181.457397pt}{304.308258pt}}
\pgfusepath{stroke}
\pgfpathmoveto{\pgfpoint{181.477936pt}{304.357910pt}}
\pgflineto{\pgfpoint{181.542114pt}{304.302277pt}}
\pgfusepath{stroke}
\pgfpathmoveto{\pgfpoint{181.535690pt}{310.284637pt}}
\pgflineto{\pgfpoint{181.452515pt}{310.293152pt}}
\pgfusepath{stroke}
\pgfpathmoveto{\pgfpoint{181.474243pt}{310.341339pt}}
\pgflineto{\pgfpoint{181.535690pt}{310.284637pt}}
\pgfusepath{stroke}
\pgfpathmoveto{\pgfpoint{181.529190pt}{316.267517pt}}
\pgflineto{\pgfpoint{181.447647pt}{316.278381pt}}
\pgfusepath{stroke}
\pgfpathmoveto{\pgfpoint{181.470459pt}{316.325134pt}}
\pgflineto{\pgfpoint{181.529190pt}{316.267517pt}}
\pgfusepath{stroke}
\pgfpathmoveto{\pgfpoint{181.522675pt}{322.250946pt}}
\pgflineto{\pgfpoint{181.442825pt}{322.263977pt}}
\pgfusepath{stroke}
\pgfpathmoveto{\pgfpoint{181.466599pt}{322.309265pt}}
\pgflineto{\pgfpoint{181.522675pt}{322.250946pt}}
\pgfusepath{stroke}
\pgfpathmoveto{\pgfpoint{181.516174pt}{328.234833pt}}
\pgflineto{\pgfpoint{181.438049pt}{328.249878pt}}
\pgfusepath{stroke}
\pgfpathmoveto{\pgfpoint{181.462677pt}{328.293732pt}}
\pgflineto{\pgfpoint{181.516174pt}{328.234833pt}}
\pgfusepath{stroke}
\pgfpathmoveto{\pgfpoint{181.509705pt}{334.219208pt}}
\pgflineto{\pgfpoint{181.433365pt}{334.236084pt}}
\pgfusepath{stroke}
\pgfpathmoveto{\pgfpoint{181.458755pt}{334.278503pt}}
\pgflineto{\pgfpoint{181.509705pt}{334.219208pt}}
\pgfusepath{stroke}
\pgfpathmoveto{\pgfpoint{181.503311pt}{340.204010pt}}
\pgflineto{\pgfpoint{181.428772pt}{340.222565pt}}
\pgfusepath{stroke}
\pgfpathmoveto{\pgfpoint{181.454803pt}{340.263580pt}}
\pgflineto{\pgfpoint{181.503311pt}{340.204010pt}}
\pgfusepath{stroke}
\pgfpathmoveto{\pgfpoint{181.497009pt}{346.189209pt}}
\pgflineto{\pgfpoint{181.424286pt}{346.209320pt}}
\pgfusepath{stroke}
\pgfpathmoveto{\pgfpoint{181.450897pt}{346.248901pt}}
\pgflineto{\pgfpoint{181.497009pt}{346.189209pt}}
\pgfusepath{stroke}
\pgfpathmoveto{\pgfpoint{181.490814pt}{352.174774pt}}
\pgflineto{\pgfpoint{181.419922pt}{352.196289pt}}
\pgfusepath{stroke}
\pgfpathmoveto{\pgfpoint{181.447021pt}{352.234528pt}}
\pgflineto{\pgfpoint{181.490814pt}{352.174774pt}}
\pgfusepath{stroke}
\pgfpathmoveto{\pgfpoint{181.484756pt}{358.160706pt}}
\pgflineto{\pgfpoint{181.415680pt}{358.183533pt}}
\pgfusepath{stroke}
\pgfpathmoveto{\pgfpoint{181.443176pt}{358.220398pt}}
\pgflineto{\pgfpoint{181.484756pt}{358.160706pt}}
\pgfusepath{stroke}
\pgfpathmoveto{\pgfpoint{181.478821pt}{364.146912pt}}
\pgflineto{\pgfpoint{181.411560pt}{364.170898pt}}
\pgfusepath{stroke}
\pgfpathmoveto{\pgfpoint{181.439392pt}{364.206482pt}}
\pgflineto{\pgfpoint{181.478821pt}{364.146912pt}}
\pgfusepath{stroke}
\pgfpathmoveto{\pgfpoint{181.473053pt}{370.133484pt}}
\pgflineto{\pgfpoint{181.407562pt}{370.158508pt}}
\pgfusepath{stroke}
\pgfpathmoveto{\pgfpoint{181.435684pt}{370.192810pt}}
\pgflineto{\pgfpoint{181.473053pt}{370.133484pt}}
\pgfusepath{stroke}
\pgfpathmoveto{\pgfpoint{187.417572pt}{77.001785pt}}
\pgflineto{\pgfpoint{187.396317pt}{76.942200pt}}
\pgfusepath{stroke}
\pgfpathmoveto{\pgfpoint{187.364807pt}{76.966873pt}}
\pgflineto{\pgfpoint{187.417572pt}{77.001785pt}}
\pgfusepath{stroke}
\pgfpathmoveto{\pgfpoint{187.422684pt}{82.990768pt}}
\pgflineto{\pgfpoint{187.399841pt}{82.930191pt}}
\pgfusepath{stroke}
\pgfpathmoveto{\pgfpoint{187.368073pt}{82.955994pt}}
\pgflineto{\pgfpoint{187.422684pt}{82.990768pt}}
\pgfusepath{stroke}
\pgfpathmoveto{\pgfpoint{187.428070pt}{88.979576pt}}
\pgflineto{\pgfpoint{187.403564pt}{88.918037pt}}
\pgfusepath{stroke}
\pgfpathmoveto{\pgfpoint{187.371552pt}{88.945045pt}}
\pgflineto{\pgfpoint{187.428070pt}{88.979576pt}}
\pgfusepath{stroke}
\pgfpathmoveto{\pgfpoint{187.433777pt}{94.968208pt}}
\pgflineto{\pgfpoint{187.407471pt}{94.905754pt}}
\pgfusepath{stroke}
\pgfpathmoveto{\pgfpoint{187.375244pt}{94.934044pt}}
\pgflineto{\pgfpoint{187.433777pt}{94.968208pt}}
\pgfusepath{stroke}
\pgfpathmoveto{\pgfpoint{187.439789pt}{100.956650pt}}
\pgflineto{\pgfpoint{187.411530pt}{100.893311pt}}
\pgfusepath{stroke}
\pgfpathmoveto{\pgfpoint{187.379181pt}{100.922905pt}}
\pgflineto{\pgfpoint{187.439789pt}{100.956650pt}}
\pgfusepath{stroke}
\pgfpathmoveto{\pgfpoint{187.446106pt}{106.944847pt}}
\pgflineto{\pgfpoint{187.415817pt}{106.880646pt}}
\pgfusepath{stroke}
\pgfpathmoveto{\pgfpoint{187.383347pt}{106.911674pt}}
\pgflineto{\pgfpoint{187.446106pt}{106.944847pt}}
\pgfusepath{stroke}
\pgfpathmoveto{\pgfpoint{187.452759pt}{112.932762pt}}
\pgflineto{\pgfpoint{187.420273pt}{112.867775pt}}
\pgfusepath{stroke}
\pgfpathmoveto{\pgfpoint{187.387756pt}{112.900269pt}}
\pgflineto{\pgfpoint{187.452759pt}{112.932762pt}}
\pgfusepath{stroke}
\pgfpathmoveto{\pgfpoint{187.459747pt}{118.920341pt}}
\pgflineto{\pgfpoint{187.424896pt}{118.854630pt}}
\pgfusepath{stroke}
\pgfpathmoveto{\pgfpoint{187.392456pt}{118.888680pt}}
\pgflineto{\pgfpoint{187.459747pt}{118.920341pt}}
\pgfusepath{stroke}
\pgfpathmoveto{\pgfpoint{187.467072pt}{124.907539pt}}
\pgflineto{\pgfpoint{187.429718pt}{124.841194pt}}
\pgfusepath{stroke}
\pgfpathmoveto{\pgfpoint{187.397385pt}{124.876877pt}}
\pgflineto{\pgfpoint{187.467072pt}{124.907539pt}}
\pgfusepath{stroke}
\pgfpathmoveto{\pgfpoint{187.474701pt}{130.894302pt}}
\pgflineto{\pgfpoint{187.434692pt}{130.827454pt}}
\pgfusepath{stroke}
\pgfpathmoveto{\pgfpoint{187.402588pt}{130.864822pt}}
\pgflineto{\pgfpoint{187.474701pt}{130.894302pt}}
\pgfusepath{stroke}
\pgfpathmoveto{\pgfpoint{187.482635pt}{136.880554pt}}
\pgflineto{\pgfpoint{187.439819pt}{136.813324pt}}
\pgfusepath{stroke}
\pgfpathmoveto{\pgfpoint{187.408035pt}{136.852448pt}}
\pgflineto{\pgfpoint{187.482635pt}{136.880554pt}}
\pgfusepath{stroke}
\pgfpathmoveto{\pgfpoint{187.490845pt}{142.866241pt}}
\pgflineto{\pgfpoint{187.445068pt}{142.798782pt}}
\pgfusepath{stroke}
\pgfpathmoveto{\pgfpoint{187.413727pt}{142.839752pt}}
\pgflineto{\pgfpoint{187.490845pt}{142.866241pt}}
\pgfusepath{stroke}
\pgfpathmoveto{\pgfpoint{187.499268pt}{148.851303pt}}
\pgflineto{\pgfpoint{187.450378pt}{148.783798pt}}
\pgfusepath{stroke}
\pgfpathmoveto{\pgfpoint{187.419678pt}{148.826645pt}}
\pgflineto{\pgfpoint{187.499268pt}{148.851303pt}}
\pgfusepath{stroke}
\pgfpathmoveto{\pgfpoint{187.507904pt}{154.835663pt}}
\pgflineto{\pgfpoint{187.455750pt}{154.768341pt}}
\pgfusepath{stroke}
\pgfpathmoveto{\pgfpoint{187.425797pt}{154.813080pt}}
\pgflineto{\pgfpoint{187.507904pt}{154.835663pt}}
\pgfusepath{stroke}
\pgfpathmoveto{\pgfpoint{187.516632pt}{160.819244pt}}
\pgflineto{\pgfpoint{187.461121pt}{160.752350pt}}
\pgfusepath{stroke}
\pgfpathmoveto{\pgfpoint{187.432083pt}{160.799026pt}}
\pgflineto{\pgfpoint{187.516632pt}{160.819244pt}}
\pgfusepath{stroke}
\pgfpathmoveto{\pgfpoint{187.525391pt}{166.802002pt}}
\pgflineto{\pgfpoint{187.466415pt}{166.735794pt}}
\pgfusepath{stroke}
\pgfpathmoveto{\pgfpoint{187.438477pt}{166.784424pt}}
\pgflineto{\pgfpoint{187.525391pt}{166.802002pt}}
\pgfusepath{stroke}
\pgfpathmoveto{\pgfpoint{187.534088pt}{172.783890pt}}
\pgflineto{\pgfpoint{187.471558pt}{172.718658pt}}
\pgfusepath{stroke}
\pgfpathmoveto{\pgfpoint{187.444916pt}{172.769241pt}}
\pgflineto{\pgfpoint{187.534088pt}{172.783890pt}}
\pgfusepath{stroke}
\pgfpathmoveto{\pgfpoint{187.542603pt}{178.764862pt}}
\pgflineto{\pgfpoint{187.476471pt}{178.700958pt}}
\pgfusepath{stroke}
\pgfpathmoveto{\pgfpoint{187.451340pt}{178.753418pt}}
\pgflineto{\pgfpoint{187.542603pt}{178.764862pt}}
\pgfusepath{stroke}
\pgfpathmoveto{\pgfpoint{187.550766pt}{184.744904pt}}
\pgflineto{\pgfpoint{187.481049pt}{184.682648pt}}
\pgfusepath{stroke}
\pgfpathmoveto{\pgfpoint{187.457657pt}{184.736938pt}}
\pgflineto{\pgfpoint{187.550766pt}{184.744904pt}}
\pgfusepath{stroke}
\pgfpathmoveto{\pgfpoint{187.558487pt}{190.724030pt}}
\pgflineto{\pgfpoint{187.485245pt}{190.663788pt}}
\pgfusepath{stroke}
\pgfpathmoveto{\pgfpoint{187.463760pt}{190.719788pt}}
\pgflineto{\pgfpoint{187.558487pt}{190.724030pt}}
\pgfusepath{stroke}
\pgfpathmoveto{\pgfpoint{187.565613pt}{196.702255pt}}
\pgflineto{\pgfpoint{187.488937pt}{196.644409pt}}
\pgfusepath{stroke}
\pgfpathmoveto{\pgfpoint{187.469559pt}{196.701981pt}}
\pgflineto{\pgfpoint{187.565613pt}{196.702255pt}}
\pgfusepath{stroke}
\pgfpathmoveto{\pgfpoint{187.571991pt}{202.679657pt}}
\pgflineto{\pgfpoint{187.492035pt}{202.624557pt}}
\pgfusepath{stroke}
\pgfpathmoveto{\pgfpoint{187.474960pt}{202.683548pt}}
\pgflineto{\pgfpoint{187.571991pt}{202.679657pt}}
\pgfusepath{stroke}
\pgfpathmoveto{\pgfpoint{187.577454pt}{208.656342pt}}
\pgflineto{\pgfpoint{187.494461pt}{208.604324pt}}
\pgfusepath{stroke}
\pgfpathmoveto{\pgfpoint{187.479843pt}{208.664536pt}}
\pgflineto{\pgfpoint{187.577454pt}{208.656342pt}}
\pgfusepath{stroke}
\pgfpathmoveto{\pgfpoint{187.581955pt}{214.632431pt}}
\pgflineto{\pgfpoint{187.496155pt}{214.583801pt}}
\pgfusepath{stroke}
\pgfpathmoveto{\pgfpoint{187.484131pt}{214.645004pt}}
\pgflineto{\pgfpoint{187.581955pt}{214.632431pt}}
\pgfusepath{stroke}
\pgfpathmoveto{\pgfpoint{187.585358pt}{220.608047pt}}
\pgflineto{\pgfpoint{187.497070pt}{220.563095pt}}
\pgfusepath{stroke}
\pgfpathmoveto{\pgfpoint{187.487747pt}{220.625061pt}}
\pgflineto{\pgfpoint{187.585358pt}{220.608047pt}}
\pgfusepath{stroke}
\pgfpathmoveto{\pgfpoint{187.587616pt}{226.583389pt}}
\pgflineto{\pgfpoint{187.497192pt}{226.542313pt}}
\pgfusepath{stroke}
\pgfpathmoveto{\pgfpoint{187.490631pt}{226.604782pt}}
\pgflineto{\pgfpoint{187.587616pt}{226.583389pt}}
\pgfusepath{stroke}
\pgfpathmoveto{\pgfpoint{187.588684pt}{232.558609pt}}
\pgflineto{\pgfpoint{187.496521pt}{232.521591pt}}
\pgfusepath{stroke}
\pgfpathmoveto{\pgfpoint{187.492752pt}{232.584290pt}}
\pgflineto{\pgfpoint{187.588684pt}{232.558609pt}}
\pgfusepath{stroke}
\pgfpathmoveto{\pgfpoint{187.588623pt}{238.533875pt}}
\pgflineto{\pgfpoint{187.495087pt}{238.501038pt}}
\pgfusepath{stroke}
\pgfpathmoveto{\pgfpoint{187.494080pt}{238.563721pt}}
\pgflineto{\pgfpoint{187.588623pt}{238.533875pt}}
\pgfusepath{stroke}
\pgfpathmoveto{\pgfpoint{187.587433pt}{244.509354pt}}
\pgflineto{\pgfpoint{187.492935pt}{244.480728pt}}
\pgfusepath{stroke}
\pgfpathmoveto{\pgfpoint{187.494659pt}{244.543167pt}}
\pgflineto{\pgfpoint{187.587433pt}{244.509354pt}}
\pgfusepath{stroke}
\pgfpathmoveto{\pgfpoint{187.585205pt}{250.485199pt}}
\pgflineto{\pgfpoint{187.490128pt}{250.460785pt}}
\pgfusepath{stroke}
\pgfpathmoveto{\pgfpoint{187.494507pt}{250.522736pt}}
\pgflineto{\pgfpoint{187.585205pt}{250.485199pt}}
\pgfusepath{stroke}
\pgfpathmoveto{\pgfpoint{187.582031pt}{256.461517pt}}
\pgflineto{\pgfpoint{187.486740pt}{256.441284pt}}
\pgfusepath{stroke}
\pgfpathmoveto{\pgfpoint{187.493652pt}{256.502502pt}}
\pgflineto{\pgfpoint{187.582031pt}{256.461517pt}}
\pgfusepath{stroke}
\pgfpathmoveto{\pgfpoint{187.578033pt}{262.438385pt}}
\pgflineto{\pgfpoint{187.482849pt}{262.422241pt}}
\pgfusepath{stroke}
\pgfpathmoveto{\pgfpoint{187.492188pt}{262.482544pt}}
\pgflineto{\pgfpoint{187.578033pt}{262.438385pt}}
\pgfusepath{stroke}
\pgfpathmoveto{\pgfpoint{187.573288pt}{268.415894pt}}
\pgflineto{\pgfpoint{187.478546pt}{268.403687pt}}
\pgfusepath{stroke}
\pgfpathmoveto{\pgfpoint{187.490173pt}{268.462952pt}}
\pgflineto{\pgfpoint{187.573288pt}{268.415894pt}}
\pgfusepath{stroke}
\pgfpathmoveto{\pgfpoint{187.567932pt}{274.394104pt}}
\pgflineto{\pgfpoint{187.473938pt}{274.385681pt}}
\pgfusepath{stroke}
\pgfpathmoveto{\pgfpoint{187.487671pt}{274.443756pt}}
\pgflineto{\pgfpoint{187.567932pt}{274.394104pt}}
\pgfusepath{stroke}
\pgfpathmoveto{\pgfpoint{187.562088pt}{280.373016pt}}
\pgflineto{\pgfpoint{187.469055pt}{280.368164pt}}
\pgfusepath{stroke}
\pgfpathmoveto{\pgfpoint{187.484741pt}{280.424957pt}}
\pgflineto{\pgfpoint{187.562088pt}{280.373016pt}}
\pgfusepath{stroke}
\pgfpathmoveto{\pgfpoint{187.555847pt}{286.352631pt}}
\pgflineto{\pgfpoint{187.463989pt}{286.351196pt}}
\pgfusepath{stroke}
\pgfpathmoveto{\pgfpoint{187.481491pt}{286.406616pt}}
\pgflineto{\pgfpoint{187.555847pt}{286.352631pt}}
\pgfusepath{stroke}
\pgfpathmoveto{\pgfpoint{187.549301pt}{292.332947pt}}
\pgflineto{\pgfpoint{187.458786pt}{292.334717pt}}
\pgfusepath{stroke}
\pgfpathmoveto{\pgfpoint{187.477951pt}{292.388672pt}}
\pgflineto{\pgfpoint{187.549301pt}{292.332947pt}}
\pgfusepath{stroke}
\pgfpathmoveto{\pgfpoint{187.542526pt}{298.313965pt}}
\pgflineto{\pgfpoint{187.453522pt}{298.318726pt}}
\pgfusepath{stroke}
\pgfpathmoveto{\pgfpoint{187.474197pt}{298.371155pt}}
\pgflineto{\pgfpoint{187.542526pt}{298.313965pt}}
\pgfusepath{stroke}
\pgfpathmoveto{\pgfpoint{187.535614pt}{304.295593pt}}
\pgflineto{\pgfpoint{187.448242pt}{304.303192pt}}
\pgfusepath{stroke}
\pgfpathmoveto{\pgfpoint{187.470261pt}{304.354095pt}}
\pgflineto{\pgfpoint{187.535614pt}{304.295593pt}}
\pgfusepath{stroke}
\pgfpathmoveto{\pgfpoint{187.528595pt}{310.277893pt}}
\pgflineto{\pgfpoint{187.442963pt}{310.288055pt}}
\pgfusepath{stroke}
\pgfpathmoveto{\pgfpoint{187.466187pt}{310.337402pt}}
\pgflineto{\pgfpoint{187.528595pt}{310.277893pt}}
\pgfusepath{stroke}
\pgfpathmoveto{\pgfpoint{187.521545pt}{316.260742pt}}
\pgflineto{\pgfpoint{187.437714pt}{316.273346pt}}
\pgfusepath{stroke}
\pgfpathmoveto{\pgfpoint{187.462036pt}{316.321136pt}}
\pgflineto{\pgfpoint{187.521545pt}{316.260742pt}}
\pgfusepath{stroke}
\pgfpathmoveto{\pgfpoint{187.514511pt}{322.244171pt}}
\pgflineto{\pgfpoint{187.432526pt}{322.259003pt}}
\pgfusepath{stroke}
\pgfpathmoveto{\pgfpoint{187.457825pt}{322.305206pt}}
\pgflineto{\pgfpoint{187.514511pt}{322.244171pt}}
\pgfusepath{stroke}
\pgfpathmoveto{\pgfpoint{187.507507pt}{328.228119pt}}
\pgflineto{\pgfpoint{187.427460pt}{328.244995pt}}
\pgfusepath{stroke}
\pgfpathmoveto{\pgfpoint{187.453583pt}{328.289642pt}}
\pgflineto{\pgfpoint{187.507507pt}{328.228119pt}}
\pgfusepath{stroke}
\pgfpathmoveto{\pgfpoint{187.500580pt}{334.212555pt}}
\pgflineto{\pgfpoint{187.422485pt}{334.231293pt}}
\pgfusepath{stroke}
\pgfpathmoveto{\pgfpoint{187.449341pt}{334.274414pt}}
\pgflineto{\pgfpoint{187.500580pt}{334.212555pt}}
\pgfusepath{stroke}
\pgfpathmoveto{\pgfpoint{187.493774pt}{340.197449pt}}
\pgflineto{\pgfpoint{187.417603pt}{340.217896pt}}
\pgfusepath{stroke}
\pgfpathmoveto{\pgfpoint{187.445099pt}{340.259521pt}}
\pgflineto{\pgfpoint{187.493774pt}{340.197449pt}}
\pgfusepath{stroke}
\pgfpathmoveto{\pgfpoint{187.487061pt}{346.182739pt}}
\pgflineto{\pgfpoint{187.412872pt}{346.204773pt}}
\pgfusepath{stroke}
\pgfpathmoveto{\pgfpoint{187.440918pt}{346.244873pt}}
\pgflineto{\pgfpoint{187.487061pt}{346.182739pt}}
\pgfusepath{stroke}
\pgfpathmoveto{\pgfpoint{187.480515pt}{352.168457pt}}
\pgflineto{\pgfpoint{187.408249pt}{352.191895pt}}
\pgfusepath{stroke}
\pgfpathmoveto{\pgfpoint{187.436768pt}{352.230560pt}}
\pgflineto{\pgfpoint{187.480515pt}{352.168457pt}}
\pgfusepath{stroke}
\pgfpathmoveto{\pgfpoint{187.474106pt}{358.154541pt}}
\pgflineto{\pgfpoint{187.403809pt}{358.179230pt}}
\pgfusepath{stroke}
\pgfpathmoveto{\pgfpoint{187.432678pt}{358.216492pt}}
\pgflineto{\pgfpoint{187.474106pt}{358.154541pt}}
\pgfusepath{stroke}
\pgfpathmoveto{\pgfpoint{187.467865pt}{364.140930pt}}
\pgflineto{\pgfpoint{187.399490pt}{364.166748pt}}
\pgfusepath{stroke}
\pgfpathmoveto{\pgfpoint{187.428680pt}{364.202637pt}}
\pgflineto{\pgfpoint{187.467865pt}{364.140930pt}}
\pgfusepath{stroke}
\pgfpathmoveto{\pgfpoint{187.461792pt}{370.127625pt}}
\pgflineto{\pgfpoint{187.395325pt}{370.154510pt}}
\pgfusepath{stroke}
\pgfpathmoveto{\pgfpoint{187.424744pt}{370.189026pt}}
\pgflineto{\pgfpoint{187.461792pt}{370.127625pt}}
\pgfusepath{stroke}
\pgfpathmoveto{\pgfpoint{193.403931pt}{77.006805pt}}
\pgflineto{\pgfpoint{193.383698pt}{76.945694pt}}
\pgfusepath{stroke}
\pgfpathmoveto{\pgfpoint{193.351059pt}{76.970062pt}}
\pgflineto{\pgfpoint{193.403931pt}{77.006805pt}}
\pgfusepath{stroke}
\pgfpathmoveto{\pgfpoint{193.409149pt}{82.996078pt}}
\pgflineto{\pgfpoint{193.387329pt}{82.933868pt}}
\pgfusepath{stroke}
\pgfpathmoveto{\pgfpoint{193.354370pt}{82.959396pt}}
\pgflineto{\pgfpoint{193.409149pt}{82.996078pt}}
\pgfusepath{stroke}
\pgfpathmoveto{\pgfpoint{193.414703pt}{88.985199pt}}
\pgflineto{\pgfpoint{193.391190pt}{88.921921pt}}
\pgfusepath{stroke}
\pgfpathmoveto{\pgfpoint{193.357910pt}{88.948677pt}}
\pgflineto{\pgfpoint{193.414703pt}{88.985199pt}}
\pgfusepath{stroke}
\pgfpathmoveto{\pgfpoint{193.420563pt}{94.974159pt}}
\pgflineto{\pgfpoint{193.395233pt}{94.909836pt}}
\pgfusepath{stroke}
\pgfpathmoveto{\pgfpoint{193.361694pt}{94.937889pt}}
\pgflineto{\pgfpoint{193.420563pt}{94.974159pt}}
\pgfusepath{stroke}
\pgfpathmoveto{\pgfpoint{193.426788pt}{100.962952pt}}
\pgflineto{\pgfpoint{193.399506pt}{100.897568pt}}
\pgfusepath{stroke}
\pgfpathmoveto{\pgfpoint{193.365753pt}{100.927017pt}}
\pgflineto{\pgfpoint{193.426788pt}{100.962952pt}}
\pgfusepath{stroke}
\pgfpathmoveto{\pgfpoint{193.433380pt}{106.951477pt}}
\pgflineto{\pgfpoint{193.403992pt}{106.885124pt}}
\pgfusepath{stroke}
\pgfpathmoveto{\pgfpoint{193.370041pt}{106.916039pt}}
\pgflineto{\pgfpoint{193.433380pt}{106.951477pt}}
\pgfusepath{stroke}
\pgfpathmoveto{\pgfpoint{193.440338pt}{112.939751pt}}
\pgflineto{\pgfpoint{193.408691pt}{112.872444pt}}
\pgfusepath{stroke}
\pgfpathmoveto{\pgfpoint{193.374634pt}{112.904900pt}}
\pgflineto{\pgfpoint{193.440338pt}{112.939751pt}}
\pgfusepath{stroke}
\pgfpathmoveto{\pgfpoint{193.447693pt}{118.927689pt}}
\pgflineto{\pgfpoint{193.413605pt}{118.859505pt}}
\pgfusepath{stroke}
\pgfpathmoveto{\pgfpoint{193.379517pt}{118.893600pt}}
\pgflineto{\pgfpoint{193.447693pt}{118.927689pt}}
\pgfusepath{stroke}
\pgfpathmoveto{\pgfpoint{193.455414pt}{124.915245pt}}
\pgflineto{\pgfpoint{193.418732pt}{124.846268pt}}
\pgfusepath{stroke}
\pgfpathmoveto{\pgfpoint{193.384689pt}{124.882072pt}}
\pgflineto{\pgfpoint{193.455414pt}{124.915245pt}}
\pgfusepath{stroke}
\pgfpathmoveto{\pgfpoint{193.463531pt}{130.902344pt}}
\pgflineto{\pgfpoint{193.424072pt}{130.832687pt}}
\pgfusepath{stroke}
\pgfpathmoveto{\pgfpoint{193.390167pt}{130.870300pt}}
\pgflineto{\pgfpoint{193.463531pt}{130.902344pt}}
\pgfusepath{stroke}
\pgfpathmoveto{\pgfpoint{193.472015pt}{136.888916pt}}
\pgflineto{\pgfpoint{193.429581pt}{136.818710pt}}
\pgfusepath{stroke}
\pgfpathmoveto{\pgfpoint{193.395950pt}{136.858215pt}}
\pgflineto{\pgfpoint{193.472015pt}{136.888916pt}}
\pgfusepath{stroke}
\pgfpathmoveto{\pgfpoint{193.480835pt}{142.874908pt}}
\pgflineto{\pgfpoint{193.435272pt}{142.804291pt}}
\pgfusepath{stroke}
\pgfpathmoveto{\pgfpoint{193.402023pt}{142.845764pt}}
\pgflineto{\pgfpoint{193.480835pt}{142.874908pt}}
\pgfusepath{stroke}
\pgfpathmoveto{\pgfpoint{193.489975pt}{148.860214pt}}
\pgflineto{\pgfpoint{193.441101pt}{148.789413pt}}
\pgfusepath{stroke}
\pgfpathmoveto{\pgfpoint{193.408386pt}{148.832901pt}}
\pgflineto{\pgfpoint{193.489975pt}{148.860214pt}}
\pgfusepath{stroke}
\pgfpathmoveto{\pgfpoint{193.499359pt}{154.844742pt}}
\pgflineto{\pgfpoint{193.446991pt}{154.773987pt}}
\pgfusepath{stroke}
\pgfpathmoveto{\pgfpoint{193.415009pt}{154.819550pt}}
\pgflineto{\pgfpoint{193.499359pt}{154.844742pt}}
\pgfusepath{stroke}
\pgfpathmoveto{\pgfpoint{193.508942pt}{160.828445pt}}
\pgflineto{\pgfpoint{193.452927pt}{160.757965pt}}
\pgfusepath{stroke}
\pgfpathmoveto{\pgfpoint{193.421844pt}{160.805664pt}}
\pgflineto{\pgfpoint{193.508942pt}{160.828445pt}}
\pgfusepath{stroke}
\pgfpathmoveto{\pgfpoint{193.518616pt}{166.811218pt}}
\pgflineto{\pgfpoint{193.458817pt}{166.741318pt}}
\pgfusepath{stroke}
\pgfpathmoveto{\pgfpoint{193.428848pt}{166.791168pt}}
\pgflineto{\pgfpoint{193.518616pt}{166.811218pt}}
\pgfusepath{stroke}
\pgfpathmoveto{\pgfpoint{193.528259pt}{172.792984pt}}
\pgflineto{\pgfpoint{193.464584pt}{172.724014pt}}
\pgfusepath{stroke}
\pgfpathmoveto{\pgfpoint{193.435944pt}{172.776016pt}}
\pgflineto{\pgfpoint{193.528259pt}{172.792984pt}}
\pgfusepath{stroke}
\pgfpathmoveto{\pgfpoint{193.537750pt}{178.773712pt}}
\pgflineto{\pgfpoint{193.470139pt}{178.706024pt}}
\pgfusepath{stroke}
\pgfpathmoveto{\pgfpoint{193.443054pt}{178.760132pt}}
\pgflineto{\pgfpoint{193.537750pt}{178.773712pt}}
\pgfusepath{stroke}
\pgfpathmoveto{\pgfpoint{193.546936pt}{184.753372pt}}
\pgflineto{\pgfpoint{193.475357pt}{184.687363pt}}
\pgfusepath{stroke}
\pgfpathmoveto{\pgfpoint{193.450073pt}{184.743515pt}}
\pgflineto{\pgfpoint{193.546936pt}{184.753372pt}}
\pgfusepath{stroke}
\pgfpathmoveto{\pgfpoint{193.555649pt}{190.731934pt}}
\pgflineto{\pgfpoint{193.480148pt}{190.668030pt}}
\pgfusepath{stroke}
\pgfpathmoveto{\pgfpoint{193.456909pt}{190.726120pt}}
\pgflineto{\pgfpoint{193.555649pt}{190.731934pt}}
\pgfusepath{stroke}
\pgfpathmoveto{\pgfpoint{193.563705pt}{196.709457pt}}
\pgflineto{\pgfpoint{193.484375pt}{196.648071pt}}
\pgfusepath{stroke}
\pgfpathmoveto{\pgfpoint{193.463409pt}{196.707932pt}}
\pgflineto{\pgfpoint{193.563705pt}{196.709457pt}}
\pgfusepath{stroke}
\pgfpathmoveto{\pgfpoint{193.570923pt}{202.686005pt}}
\pgflineto{\pgfpoint{193.487946pt}{202.627548pt}}
\pgfusepath{stroke}
\pgfpathmoveto{\pgfpoint{193.469467pt}{202.689026pt}}
\pgflineto{\pgfpoint{193.570923pt}{202.686005pt}}
\pgfusepath{stroke}
\pgfpathmoveto{\pgfpoint{193.577148pt}{208.661667pt}}
\pgflineto{\pgfpoint{193.490753pt}{208.606552pt}}
\pgfusepath{stroke}
\pgfpathmoveto{\pgfpoint{193.474945pt}{208.669403pt}}
\pgflineto{\pgfpoint{193.577148pt}{208.661667pt}}
\pgfusepath{stroke}
\pgfpathmoveto{\pgfpoint{193.582214pt}{214.636627pt}}
\pgflineto{\pgfpoint{193.492706pt}{214.585205pt}}
\pgfusepath{stroke}
\pgfpathmoveto{\pgfpoint{193.479767pt}{214.649200pt}}
\pgflineto{\pgfpoint{193.582214pt}{214.636627pt}}
\pgfusepath{stroke}
\pgfpathmoveto{\pgfpoint{193.586029pt}{220.611038pt}}
\pgflineto{\pgfpoint{193.493759pt}{220.563644pt}}
\pgfusepath{stroke}
\pgfpathmoveto{\pgfpoint{193.483780pt}{220.628479pt}}
\pgflineto{\pgfpoint{193.586029pt}{220.611038pt}}
\pgfusepath{stroke}
\pgfpathmoveto{\pgfpoint{193.588501pt}{226.585114pt}}
\pgflineto{\pgfpoint{193.493896pt}{226.542007pt}}
\pgfusepath{stroke}
\pgfpathmoveto{\pgfpoint{193.486938pt}{226.607391pt}}
\pgflineto{\pgfpoint{193.588501pt}{226.585114pt}}
\pgfusepath{stroke}
\pgfpathmoveto{\pgfpoint{193.589600pt}{232.559067pt}}
\pgflineto{\pgfpoint{193.493103pt}{232.520447pt}}
\pgfusepath{stroke}
\pgfpathmoveto{\pgfpoint{193.489227pt}{232.586075pt}}
\pgflineto{\pgfpoint{193.589600pt}{232.559067pt}}
\pgfusepath{stroke}
\pgfpathmoveto{\pgfpoint{193.589371pt}{238.533112pt}}
\pgflineto{\pgfpoint{193.491425pt}{238.499084pt}}
\pgfusepath{stroke}
\pgfpathmoveto{\pgfpoint{193.490601pt}{238.564651pt}}
\pgflineto{\pgfpoint{193.589371pt}{238.533112pt}}
\pgfusepath{stroke}
\pgfpathmoveto{\pgfpoint{193.587860pt}{244.507446pt}}
\pgflineto{\pgfpoint{193.488922pt}{244.478058pt}}
\pgfusepath{stroke}
\pgfpathmoveto{\pgfpoint{193.491074pt}{244.543289pt}}
\pgflineto{\pgfpoint{193.587860pt}{244.507446pt}}
\pgfusepath{stroke}
\pgfpathmoveto{\pgfpoint{193.585159pt}{250.482208pt}}
\pgflineto{\pgfpoint{193.485687pt}{250.457458pt}}
\pgfusepath{stroke}
\pgfpathmoveto{\pgfpoint{193.490723pt}{250.522095pt}}
\pgflineto{\pgfpoint{193.585159pt}{250.482208pt}}
\pgfusepath{stroke}
\pgfpathmoveto{\pgfpoint{193.581390pt}{256.457581pt}}
\pgflineto{\pgfpoint{193.481781pt}{256.437378pt}}
\pgfusepath{stroke}
\pgfpathmoveto{\pgfpoint{193.489578pt}{256.501160pt}}
\pgflineto{\pgfpoint{193.581390pt}{256.457581pt}}
\pgfusepath{stroke}
\pgfpathmoveto{\pgfpoint{193.576706pt}{262.433655pt}}
\pgflineto{\pgfpoint{193.477356pt}{262.417847pt}}
\pgfusepath{stroke}
\pgfpathmoveto{\pgfpoint{193.487762pt}{262.480621pt}}
\pgflineto{\pgfpoint{193.576706pt}{262.433655pt}}
\pgfusepath{stroke}
\pgfpathmoveto{\pgfpoint{193.571259pt}{268.410492pt}}
\pgflineto{\pgfpoint{193.472519pt}{268.398926pt}}
\pgfusepath{stroke}
\pgfpathmoveto{\pgfpoint{193.485321pt}{268.460480pt}}
\pgflineto{\pgfpoint{193.571259pt}{268.410492pt}}
\pgfusepath{stroke}
\pgfpathmoveto{\pgfpoint{193.565155pt}{274.388123pt}}
\pgflineto{\pgfpoint{193.467316pt}{274.380615pt}}
\pgfusepath{stroke}
\pgfpathmoveto{\pgfpoint{193.482391pt}{274.440826pt}}
\pgflineto{\pgfpoint{193.565155pt}{274.388123pt}}
\pgfusepath{stroke}
\pgfpathmoveto{\pgfpoint{193.558533pt}{280.366577pt}}
\pgflineto{\pgfpoint{193.461884pt}{280.362915pt}}
\pgfusepath{stroke}
\pgfpathmoveto{\pgfpoint{193.479004pt}{280.421631pt}}
\pgflineto{\pgfpoint{193.558533pt}{280.366577pt}}
\pgfusepath{stroke}
\pgfpathmoveto{\pgfpoint{193.551529pt}{286.345856pt}}
\pgflineto{\pgfpoint{193.456268pt}{286.345795pt}}
\pgfusepath{stroke}
\pgfpathmoveto{\pgfpoint{193.475281pt}{286.402954pt}}
\pgflineto{\pgfpoint{193.551529pt}{286.345856pt}}
\pgfusepath{stroke}
\pgfpathmoveto{\pgfpoint{193.544250pt}{292.325928pt}}
\pgflineto{\pgfpoint{193.450562pt}{292.329224pt}}
\pgfusepath{stroke}
\pgfpathmoveto{\pgfpoint{193.471283pt}{292.384796pt}}
\pgflineto{\pgfpoint{193.544250pt}{292.325928pt}}
\pgfusepath{stroke}
\pgfpathmoveto{\pgfpoint{193.536774pt}{298.306763pt}}
\pgflineto{\pgfpoint{193.444809pt}{298.313202pt}}
\pgfusepath{stroke}
\pgfpathmoveto{\pgfpoint{193.467072pt}{298.367096pt}}
\pgflineto{\pgfpoint{193.536774pt}{298.306763pt}}
\pgfusepath{stroke}
\pgfpathmoveto{\pgfpoint{193.529190pt}{304.288330pt}}
\pgflineto{\pgfpoint{193.439056pt}{304.297668pt}}
\pgfusepath{stroke}
\pgfpathmoveto{\pgfpoint{193.462708pt}{304.349884pt}}
\pgflineto{\pgfpoint{193.529190pt}{304.288330pt}}
\pgfusepath{stroke}
\pgfpathmoveto{\pgfpoint{193.521545pt}{310.270538pt}}
\pgflineto{\pgfpoint{193.433350pt}{310.282593pt}}
\pgfusepath{stroke}
\pgfpathmoveto{\pgfpoint{193.458221pt}{310.333099pt}}
\pgflineto{\pgfpoint{193.521545pt}{310.270538pt}}
\pgfusepath{stroke}
\pgfpathmoveto{\pgfpoint{193.513901pt}{316.253418pt}}
\pgflineto{\pgfpoint{193.427719pt}{316.267944pt}}
\pgfusepath{stroke}
\pgfpathmoveto{\pgfpoint{193.453674pt}{316.316742pt}}
\pgflineto{\pgfpoint{193.513901pt}{316.253418pt}}
\pgfusepath{stroke}
\pgfpathmoveto{\pgfpoint{193.506302pt}{322.236877pt}}
\pgflineto{\pgfpoint{193.422180pt}{322.253693pt}}
\pgfusepath{stroke}
\pgfpathmoveto{\pgfpoint{193.449081pt}{322.300781pt}}
\pgflineto{\pgfpoint{193.506302pt}{322.236877pt}}
\pgfusepath{stroke}
\pgfpathmoveto{\pgfpoint{193.498779pt}{328.220917pt}}
\pgflineto{\pgfpoint{193.416748pt}{328.239777pt}}
\pgfusepath{stroke}
\pgfpathmoveto{\pgfpoint{193.444489pt}{328.285217pt}}
\pgflineto{\pgfpoint{193.498779pt}{328.220917pt}}
\pgfusepath{stroke}
\pgfpathmoveto{\pgfpoint{193.491379pt}{334.205444pt}}
\pgflineto{\pgfpoint{193.411453pt}{334.226227pt}}
\pgfusepath{stroke}
\pgfpathmoveto{\pgfpoint{193.439911pt}{334.270020pt}}
\pgflineto{\pgfpoint{193.491379pt}{334.205444pt}}
\pgfusepath{stroke}
\pgfpathmoveto{\pgfpoint{193.484100pt}{340.190460pt}}
\pgflineto{\pgfpoint{193.406296pt}{340.212952pt}}
\pgfusepath{stroke}
\pgfpathmoveto{\pgfpoint{193.435364pt}{340.255127pt}}
\pgflineto{\pgfpoint{193.484100pt}{340.190460pt}}
\pgfusepath{stroke}
\pgfpathmoveto{\pgfpoint{193.476990pt}{346.175903pt}}
\pgflineto{\pgfpoint{193.401306pt}{346.199982pt}}
\pgfusepath{stroke}
\pgfpathmoveto{\pgfpoint{193.430862pt}{346.240601pt}}
\pgflineto{\pgfpoint{193.476990pt}{346.175903pt}}
\pgfusepath{stroke}
\pgfpathmoveto{\pgfpoint{193.470047pt}{352.161804pt}}
\pgflineto{\pgfpoint{193.396454pt}{352.187225pt}}
\pgfusepath{stroke}
\pgfpathmoveto{\pgfpoint{193.426453pt}{352.226288pt}}
\pgflineto{\pgfpoint{193.470047pt}{352.161804pt}}
\pgfusepath{stroke}
\pgfpathmoveto{\pgfpoint{193.463287pt}{358.148041pt}}
\pgflineto{\pgfpoint{193.391785pt}{358.174744pt}}
\pgfusepath{stroke}
\pgfpathmoveto{\pgfpoint{193.422119pt}{358.212280pt}}
\pgflineto{\pgfpoint{193.463287pt}{358.148041pt}}
\pgfusepath{stroke}
\pgfpathmoveto{\pgfpoint{193.456726pt}{364.134613pt}}
\pgflineto{\pgfpoint{193.387268pt}{364.162476pt}}
\pgfusepath{stroke}
\pgfpathmoveto{\pgfpoint{193.417862pt}{364.198547pt}}
\pgflineto{\pgfpoint{193.456726pt}{364.134613pt}}
\pgfusepath{stroke}
\pgfpathmoveto{\pgfpoint{193.450363pt}{370.121521pt}}
\pgflineto{\pgfpoint{193.382950pt}{370.150330pt}}
\pgfusepath{stroke}
\pgfpathmoveto{\pgfpoint{193.413712pt}{370.185059pt}}
\pgflineto{\pgfpoint{193.450363pt}{370.121521pt}}
\pgfusepath{stroke}
\pgfpathmoveto{\pgfpoint{199.389984pt}{77.011932pt}}
\pgflineto{\pgfpoint{199.370880pt}{76.949295pt}}
\pgfusepath{stroke}
\pgfpathmoveto{\pgfpoint{199.337112pt}{76.973282pt}}
\pgflineto{\pgfpoint{199.389984pt}{77.011932pt}}
\pgfusepath{stroke}
\pgfpathmoveto{\pgfpoint{199.395309pt}{83.001526pt}}
\pgflineto{\pgfpoint{199.374634pt}{82.937668pt}}
\pgfusepath{stroke}
\pgfpathmoveto{\pgfpoint{199.340454pt}{82.962830pt}}
\pgflineto{\pgfpoint{199.395309pt}{83.001526pt}}
\pgfusepath{stroke}
\pgfpathmoveto{\pgfpoint{199.400970pt}{88.990982pt}}
\pgflineto{\pgfpoint{199.378601pt}{88.925941pt}}
\pgfusepath{stroke}
\pgfpathmoveto{\pgfpoint{199.344055pt}{88.952370pt}}
\pgflineto{\pgfpoint{199.400970pt}{88.990982pt}}
\pgfusepath{stroke}
\pgfpathmoveto{\pgfpoint{199.407013pt}{94.980301pt}}
\pgflineto{\pgfpoint{199.382812pt}{94.914070pt}}
\pgfusepath{stroke}
\pgfpathmoveto{\pgfpoint{199.347900pt}{94.941841pt}}
\pgflineto{\pgfpoint{199.407013pt}{94.980301pt}}
\pgfusepath{stroke}
\pgfpathmoveto{\pgfpoint{199.413452pt}{100.969452pt}}
\pgflineto{\pgfpoint{199.387253pt}{100.902054pt}}
\pgfusepath{stroke}
\pgfpathmoveto{\pgfpoint{199.352036pt}{100.931252pt}}
\pgflineto{\pgfpoint{199.413452pt}{100.969452pt}}
\pgfusepath{stroke}
\pgfpathmoveto{\pgfpoint{199.420288pt}{106.958389pt}}
\pgflineto{\pgfpoint{199.391937pt}{106.889824pt}}
\pgfusepath{stroke}
\pgfpathmoveto{\pgfpoint{199.356476pt}{106.920540pt}}
\pgflineto{\pgfpoint{199.420288pt}{106.958389pt}}
\pgfusepath{stroke}
\pgfpathmoveto{\pgfpoint{199.427551pt}{112.947067pt}}
\pgflineto{\pgfpoint{199.396881pt}{112.877388pt}}
\pgfusepath{stroke}
\pgfpathmoveto{\pgfpoint{199.361206pt}{112.909714pt}}
\pgflineto{\pgfpoint{199.427551pt}{112.947067pt}}
\pgfusepath{stroke}
\pgfpathmoveto{\pgfpoint{199.435242pt}{118.935417pt}}
\pgflineto{\pgfpoint{199.402069pt}{118.864685pt}}
\pgfusepath{stroke}
\pgfpathmoveto{\pgfpoint{199.366272pt}{118.898727pt}}
\pgflineto{\pgfpoint{199.435242pt}{118.935417pt}}
\pgfusepath{stroke}
\pgfpathmoveto{\pgfpoint{199.443390pt}{124.923386pt}}
\pgflineto{\pgfpoint{199.407532pt}{124.851669pt}}
\pgfusepath{stroke}
\pgfpathmoveto{\pgfpoint{199.371674pt}{124.887527pt}}
\pgflineto{\pgfpoint{199.443390pt}{124.923386pt}}
\pgfusepath{stroke}
\pgfpathmoveto{\pgfpoint{199.451996pt}{130.910904pt}}
\pgflineto{\pgfpoint{199.413239pt}{130.838318pt}}
\pgfusepath{stroke}
\pgfpathmoveto{\pgfpoint{199.377441pt}{130.876083pt}}
\pgflineto{\pgfpoint{199.451996pt}{130.910904pt}}
\pgfusepath{stroke}
\pgfpathmoveto{\pgfpoint{199.461044pt}{136.897888pt}}
\pgflineto{\pgfpoint{199.419189pt}{136.824554pt}}
\pgfusepath{stroke}
\pgfpathmoveto{\pgfpoint{199.383560pt}{136.864319pt}}
\pgflineto{\pgfpoint{199.461044pt}{136.897888pt}}
\pgfusepath{stroke}
\pgfpathmoveto{\pgfpoint{199.470520pt}{142.884247pt}}
\pgflineto{\pgfpoint{199.425354pt}{142.810303pt}}
\pgfusepath{stroke}
\pgfpathmoveto{\pgfpoint{199.390030pt}{142.852203pt}}
\pgflineto{\pgfpoint{199.470520pt}{142.884247pt}}
\pgfusepath{stroke}
\pgfpathmoveto{\pgfpoint{199.480408pt}{148.869873pt}}
\pgflineto{\pgfpoint{199.431702pt}{148.795563pt}}
\pgfusepath{stroke}
\pgfpathmoveto{\pgfpoint{199.396851pt}{148.839630pt}}
\pgflineto{\pgfpoint{199.480408pt}{148.869873pt}}
\pgfusepath{stroke}
\pgfpathmoveto{\pgfpoint{199.490631pt}{154.854675pt}}
\pgflineto{\pgfpoint{199.438202pt}{154.780212pt}}
\pgfusepath{stroke}
\pgfpathmoveto{\pgfpoint{199.403992pt}{154.826569pt}}
\pgflineto{\pgfpoint{199.490631pt}{154.854675pt}}
\pgfusepath{stroke}
\pgfpathmoveto{\pgfpoint{199.501144pt}{160.838562pt}}
\pgflineto{\pgfpoint{199.444763pt}{160.764221pt}}
\pgfusepath{stroke}
\pgfpathmoveto{\pgfpoint{199.411438pt}{160.812897pt}}
\pgflineto{\pgfpoint{199.501144pt}{160.838562pt}}
\pgfusepath{stroke}
\pgfpathmoveto{\pgfpoint{199.511841pt}{166.821411pt}}
\pgflineto{\pgfpoint{199.451355pt}{166.747513pt}}
\pgfusepath{stroke}
\pgfpathmoveto{\pgfpoint{199.419098pt}{166.798584pt}}
\pgflineto{\pgfpoint{199.511841pt}{166.821411pt}}
\pgfusepath{stroke}
\pgfpathmoveto{\pgfpoint{199.522583pt}{172.803131pt}}
\pgflineto{\pgfpoint{199.457840pt}{172.730057pt}}
\pgfusepath{stroke}
\pgfpathmoveto{\pgfpoint{199.426941pt}{172.783508pt}}
\pgflineto{\pgfpoint{199.522583pt}{172.803131pt}}
\pgfusepath{stroke}
\pgfpathmoveto{\pgfpoint{199.533218pt}{178.783661pt}}
\pgflineto{\pgfpoint{199.464142pt}{178.711792pt}}
\pgfusepath{stroke}
\pgfpathmoveto{\pgfpoint{199.434845pt}{178.767624pt}}
\pgflineto{\pgfpoint{199.533218pt}{178.783661pt}}
\pgfusepath{stroke}
\pgfpathmoveto{\pgfpoint{199.543579pt}{184.762909pt}}
\pgflineto{\pgfpoint{199.470093pt}{184.692749pt}}
\pgfusepath{stroke}
\pgfpathmoveto{\pgfpoint{199.442688pt}{184.750870pt}}
\pgflineto{\pgfpoint{199.543579pt}{184.762909pt}}
\pgfusepath{stroke}
\pgfpathmoveto{\pgfpoint{199.553467pt}{190.740906pt}}
\pgflineto{\pgfpoint{199.475601pt}{190.672897pt}}
\pgfusepath{stroke}
\pgfpathmoveto{\pgfpoint{199.450363pt}{190.733215pt}}
\pgflineto{\pgfpoint{199.553467pt}{190.740906pt}}
\pgfusepath{stroke}
\pgfpathmoveto{\pgfpoint{199.562637pt}{196.717651pt}}
\pgflineto{\pgfpoint{199.480499pt}{196.652283pt}}
\pgfusepath{stroke}
\pgfpathmoveto{\pgfpoint{199.457703pt}{196.714630pt}}
\pgflineto{\pgfpoint{199.562637pt}{196.717651pt}}
\pgfusepath{stroke}
\pgfpathmoveto{\pgfpoint{199.570877pt}{202.693207pt}}
\pgflineto{\pgfpoint{199.484634pt}{202.630997pt}}
\pgfusepath{stroke}
\pgfpathmoveto{\pgfpoint{199.464554pt}{202.695175pt}}
\pgflineto{\pgfpoint{199.570877pt}{202.693207pt}}
\pgfusepath{stroke}
\pgfpathmoveto{\pgfpoint{199.577972pt}{208.667725pt}}
\pgflineto{\pgfpoint{199.487885pt}{208.609131pt}}
\pgfusepath{stroke}
\pgfpathmoveto{\pgfpoint{199.470749pt}{208.674896pt}}
\pgflineto{\pgfpoint{199.577972pt}{208.667725pt}}
\pgfusepath{stroke}
\pgfpathmoveto{\pgfpoint{199.583740pt}{214.641357pt}}
\pgflineto{\pgfpoint{199.490173pt}{214.586838pt}}
\pgfusepath{stroke}
\pgfpathmoveto{\pgfpoint{199.476166pt}{214.653885pt}}
\pgflineto{\pgfpoint{199.583740pt}{214.641357pt}}
\pgfusepath{stroke}
\pgfpathmoveto{\pgfpoint{199.588013pt}{220.614365pt}}
\pgflineto{\pgfpoint{199.491394pt}{220.564285pt}}
\pgfusepath{stroke}
\pgfpathmoveto{\pgfpoint{199.480667pt}{220.632278pt}}
\pgflineto{\pgfpoint{199.588013pt}{220.614365pt}}
\pgfusepath{stroke}
\pgfpathmoveto{\pgfpoint{199.590744pt}{226.586975pt}}
\pgflineto{\pgfpoint{199.491516pt}{226.541641pt}}
\pgfusepath{stroke}
\pgfpathmoveto{\pgfpoint{199.484161pt}{226.610245pt}}
\pgflineto{\pgfpoint{199.590744pt}{226.586975pt}}
\pgfusepath{stroke}
\pgfpathmoveto{\pgfpoint{199.591858pt}{232.559448pt}}
\pgflineto{\pgfpoint{199.490570pt}{232.519089pt}}
\pgfusepath{stroke}
\pgfpathmoveto{\pgfpoint{199.486603pt}{232.587936pt}}
\pgflineto{\pgfpoint{199.591858pt}{232.559448pt}}
\pgfusepath{stroke}
\pgfpathmoveto{\pgfpoint{199.591400pt}{238.532074pt}}
\pgflineto{\pgfpoint{199.488586pt}{238.496796pt}}
\pgfusepath{stroke}
\pgfpathmoveto{\pgfpoint{199.487976pt}{238.565552pt}}
\pgflineto{\pgfpoint{199.591400pt}{238.532074pt}}
\pgfusepath{stroke}
\pgfpathmoveto{\pgfpoint{199.589462pt}{244.505066pt}}
\pgflineto{\pgfpoint{199.485641pt}{244.474930pt}}
\pgfusepath{stroke}
\pgfpathmoveto{\pgfpoint{199.488312pt}{244.543243pt}}
\pgflineto{\pgfpoint{199.589462pt}{244.505066pt}}
\pgfusepath{stroke}
\pgfpathmoveto{\pgfpoint{199.586166pt}{250.478638pt}}
\pgflineto{\pgfpoint{199.481873pt}{250.453598pt}}
\pgfusepath{stroke}
\pgfpathmoveto{\pgfpoint{199.487701pt}{250.521179pt}}
\pgflineto{\pgfpoint{199.586166pt}{250.478638pt}}
\pgfusepath{stroke}
\pgfpathmoveto{\pgfpoint{199.581680pt}{256.452972pt}}
\pgflineto{\pgfpoint{199.477386pt}{256.432892pt}}
\pgfusepath{stroke}
\pgfpathmoveto{\pgfpoint{199.486206pt}{256.499481pt}}
\pgflineto{\pgfpoint{199.581680pt}{256.452972pt}}
\pgfusepath{stroke}
\pgfpathmoveto{\pgfpoint{199.576172pt}{262.428131pt}}
\pgflineto{\pgfpoint{199.472321pt}{262.412872pt}}
\pgfusepath{stroke}
\pgfpathmoveto{\pgfpoint{199.483932pt}{262.478241pt}}
\pgflineto{\pgfpoint{199.576172pt}{262.428131pt}}
\pgfusepath{stroke}
\pgfpathmoveto{\pgfpoint{199.569839pt}{268.404266pt}}
\pgflineto{\pgfpoint{199.466812pt}{268.393585pt}}
\pgfusepath{stroke}
\pgfpathmoveto{\pgfpoint{199.481018pt}{268.457520pt}}
\pgflineto{\pgfpoint{199.569839pt}{268.404266pt}}
\pgfusepath{stroke}
\pgfpathmoveto{\pgfpoint{199.562866pt}{274.381317pt}}
\pgflineto{\pgfpoint{199.460968pt}{274.375000pt}}
\pgfusepath{stroke}
\pgfpathmoveto{\pgfpoint{199.477539pt}{274.437378pt}}
\pgflineto{\pgfpoint{199.562866pt}{274.381317pt}}
\pgfusepath{stroke}
\pgfpathmoveto{\pgfpoint{199.555359pt}{280.359344pt}}
\pgflineto{\pgfpoint{199.454880pt}{280.357086pt}}
\pgfusepath{stroke}
\pgfpathmoveto{\pgfpoint{199.473618pt}{280.417816pt}}
\pgflineto{\pgfpoint{199.555359pt}{280.359344pt}}
\pgfusepath{stroke}
\pgfpathmoveto{\pgfpoint{199.547501pt}{286.338287pt}}
\pgflineto{\pgfpoint{199.448669pt}{286.339844pt}}
\pgfusepath{stroke}
\pgfpathmoveto{\pgfpoint{199.469360pt}{286.398834pt}}
\pgflineto{\pgfpoint{199.547501pt}{286.338287pt}}
\pgfusepath{stroke}
\pgfpathmoveto{\pgfpoint{199.539398pt}{292.318146pt}}
\pgflineto{\pgfpoint{199.442383pt}{292.323242pt}}
\pgfusepath{stroke}
\pgfpathmoveto{\pgfpoint{199.464844pt}{292.380432pt}}
\pgflineto{\pgfpoint{199.539398pt}{292.318146pt}}
\pgfusepath{stroke}
\pgfpathmoveto{\pgfpoint{199.531143pt}{298.298828pt}}
\pgflineto{\pgfpoint{199.436096pt}{298.307190pt}}
\pgfusepath{stroke}
\pgfpathmoveto{\pgfpoint{199.460114pt}{298.362549pt}}
\pgflineto{\pgfpoint{199.531143pt}{298.298828pt}}
\pgfusepath{stroke}
\pgfpathmoveto{\pgfpoint{199.522827pt}{304.280334pt}}
\pgflineto{\pgfpoint{199.429855pt}{304.291687pt}}
\pgfusepath{stroke}
\pgfpathmoveto{\pgfpoint{199.455261pt}{304.345215pt}}
\pgflineto{\pgfpoint{199.522827pt}{304.280334pt}}
\pgfusepath{stroke}
\pgfpathmoveto{\pgfpoint{199.514496pt}{310.262573pt}}
\pgflineto{\pgfpoint{199.423676pt}{310.276703pt}}
\pgfusepath{stroke}
\pgfpathmoveto{\pgfpoint{199.450317pt}{310.328369pt}}
\pgflineto{\pgfpoint{199.514496pt}{310.262573pt}}
\pgfusepath{stroke}
\pgfpathmoveto{\pgfpoint{199.506210pt}{316.245483pt}}
\pgflineto{\pgfpoint{199.417633pt}{316.262146pt}}
\pgfusepath{stroke}
\pgfpathmoveto{\pgfpoint{199.445343pt}{316.311951pt}}
\pgflineto{\pgfpoint{199.506210pt}{316.245483pt}}
\pgfusepath{stroke}
\pgfpathmoveto{\pgfpoint{199.498016pt}{322.229034pt}}
\pgflineto{\pgfpoint{199.411697pt}{322.248016pt}}
\pgfusepath{stroke}
\pgfpathmoveto{\pgfpoint{199.440338pt}{322.296021pt}}
\pgflineto{\pgfpoint{199.498016pt}{322.229034pt}}
\pgfusepath{stroke}
\pgfpathmoveto{\pgfpoint{199.489960pt}{328.213196pt}}
\pgflineto{\pgfpoint{199.405914pt}{328.234253pt}}
\pgfusepath{stroke}
\pgfpathmoveto{\pgfpoint{199.435364pt}{328.280457pt}}
\pgflineto{\pgfpoint{199.489960pt}{328.213196pt}}
\pgfusepath{stroke}
\pgfpathmoveto{\pgfpoint{199.482025pt}{334.197876pt}}
\pgflineto{\pgfpoint{199.400299pt}{334.220825pt}}
\pgfusepath{stroke}
\pgfpathmoveto{\pgfpoint{199.430435pt}{334.265289pt}}
\pgflineto{\pgfpoint{199.482025pt}{334.197876pt}}
\pgfusepath{stroke}
\pgfpathmoveto{\pgfpoint{199.474289pt}{340.183044pt}}
\pgflineto{\pgfpoint{199.394852pt}{340.207733pt}}
\pgfusepath{stroke}
\pgfpathmoveto{\pgfpoint{199.425552pt}{340.250458pt}}
\pgflineto{\pgfpoint{199.474289pt}{340.183044pt}}
\pgfusepath{stroke}
\pgfpathmoveto{\pgfpoint{199.466736pt}{346.168701pt}}
\pgflineto{\pgfpoint{199.389587pt}{346.194946pt}}
\pgfusepath{stroke}
\pgfpathmoveto{\pgfpoint{199.420746pt}{346.235962pt}}
\pgflineto{\pgfpoint{199.466736pt}{346.168701pt}}
\pgfusepath{stroke}
\pgfpathmoveto{\pgfpoint{199.459396pt}{352.154755pt}}
\pgflineto{\pgfpoint{199.384491pt}{352.182373pt}}
\pgfusepath{stroke}
\pgfpathmoveto{\pgfpoint{199.416046pt}{352.221771pt}}
\pgflineto{\pgfpoint{199.459396pt}{352.154755pt}}
\pgfusepath{stroke}
\pgfpathmoveto{\pgfpoint{199.452271pt}{358.141205pt}}
\pgflineto{\pgfpoint{199.379608pt}{358.170044pt}}
\pgfusepath{stroke}
\pgfpathmoveto{\pgfpoint{199.411438pt}{358.207886pt}}
\pgflineto{\pgfpoint{199.452271pt}{358.141205pt}}
\pgfusepath{stroke}
\pgfpathmoveto{\pgfpoint{199.445374pt}{364.127991pt}}
\pgflineto{\pgfpoint{199.374893pt}{364.157928pt}}
\pgfusepath{stroke}
\pgfpathmoveto{\pgfpoint{199.406952pt}{364.194244pt}}
\pgflineto{\pgfpoint{199.445374pt}{364.127991pt}}
\pgfusepath{stroke}
\pgfpathmoveto{\pgfpoint{199.438721pt}{370.115112pt}}
\pgflineto{\pgfpoint{199.370392pt}{370.145996pt}}
\pgfusepath{stroke}
\pgfpathmoveto{\pgfpoint{199.402573pt}{370.180847pt}}
\pgflineto{\pgfpoint{199.438721pt}{370.115112pt}}
\pgfusepath{stroke}
\pgfpathmoveto{\pgfpoint{205.375702pt}{77.017136pt}}
\pgflineto{\pgfpoint{205.357864pt}{76.953003pt}}
\pgfusepath{stroke}
\pgfpathmoveto{\pgfpoint{205.322937pt}{76.976532pt}}
\pgflineto{\pgfpoint{205.375702pt}{77.017136pt}}
\pgfusepath{stroke}
\pgfpathmoveto{\pgfpoint{205.381119pt}{83.007065pt}}
\pgflineto{\pgfpoint{205.361710pt}{82.941589pt}}
\pgfusepath{stroke}
\pgfpathmoveto{\pgfpoint{205.326294pt}{82.966324pt}}
\pgflineto{\pgfpoint{205.381119pt}{83.007065pt}}
\pgfusepath{stroke}
\pgfpathmoveto{\pgfpoint{205.386902pt}{88.996902pt}}
\pgflineto{\pgfpoint{205.365799pt}{88.930084pt}}
\pgfusepath{stroke}
\pgfpathmoveto{\pgfpoint{205.329926pt}{88.956116pt}}
\pgflineto{\pgfpoint{205.386902pt}{88.996902pt}}
\pgfusepath{stroke}
\pgfpathmoveto{\pgfpoint{205.393097pt}{94.986618pt}}
\pgflineto{\pgfpoint{205.370148pt}{94.918465pt}}
\pgfusepath{stroke}
\pgfpathmoveto{\pgfpoint{205.333832pt}{94.945877pt}}
\pgflineto{\pgfpoint{205.393097pt}{94.986618pt}}
\pgfusepath{stroke}
\pgfpathmoveto{\pgfpoint{205.399689pt}{100.976196pt}}
\pgflineto{\pgfpoint{205.374756pt}{100.906685pt}}
\pgfusepath{stroke}
\pgfpathmoveto{\pgfpoint{205.338043pt}{100.935562pt}}
\pgflineto{\pgfpoint{205.399689pt}{100.976196pt}}
\pgfusepath{stroke}
\pgfpathmoveto{\pgfpoint{205.406769pt}{106.965561pt}}
\pgflineto{\pgfpoint{205.379639pt}{106.894745pt}}
\pgfusepath{stroke}
\pgfpathmoveto{\pgfpoint{205.342560pt}{106.925186pt}}
\pgflineto{\pgfpoint{205.406769pt}{106.965561pt}}
\pgfusepath{stroke}
\pgfpathmoveto{\pgfpoint{205.414307pt}{112.954704pt}}
\pgflineto{\pgfpoint{205.384827pt}{112.882576pt}}
\pgfusepath{stroke}
\pgfpathmoveto{\pgfpoint{205.347443pt}{112.914703pt}}
\pgflineto{\pgfpoint{205.414307pt}{112.954704pt}}
\pgfusepath{stroke}
\pgfpathmoveto{\pgfpoint{205.422348pt}{118.943527pt}}
\pgflineto{\pgfpoint{205.390289pt}{118.870155pt}}
\pgfusepath{stroke}
\pgfpathmoveto{\pgfpoint{205.352692pt}{118.904068pt}}
\pgflineto{\pgfpoint{205.422348pt}{118.943527pt}}
\pgfusepath{stroke}
\pgfpathmoveto{\pgfpoint{205.430908pt}{124.931992pt}}
\pgflineto{\pgfpoint{205.396088pt}{124.857437pt}}
\pgfusepath{stroke}
\pgfpathmoveto{\pgfpoint{205.358307pt}{124.893234pt}}
\pgflineto{\pgfpoint{205.430908pt}{124.931992pt}}
\pgfusepath{stroke}
\pgfpathmoveto{\pgfpoint{205.440002pt}{130.919998pt}}
\pgflineto{\pgfpoint{205.402176pt}{130.844345pt}}
\pgfusepath{stroke}
\pgfpathmoveto{\pgfpoint{205.364349pt}{130.882172pt}}
\pgflineto{\pgfpoint{205.440002pt}{130.919998pt}}
\pgfusepath{stroke}
\pgfpathmoveto{\pgfpoint{205.449631pt}{136.907471pt}}
\pgflineto{\pgfpoint{205.408569pt}{136.830841pt}}
\pgfusepath{stroke}
\pgfpathmoveto{\pgfpoint{205.370804pt}{136.870819pt}}
\pgflineto{\pgfpoint{205.449631pt}{136.907471pt}}
\pgfusepath{stroke}
\pgfpathmoveto{\pgfpoint{205.459808pt}{142.894287pt}}
\pgflineto{\pgfpoint{205.415237pt}{142.816849pt}}
\pgfusepath{stroke}
\pgfpathmoveto{\pgfpoint{205.377686pt}{142.859070pt}}
\pgflineto{\pgfpoint{205.459808pt}{142.894287pt}}
\pgfusepath{stroke}
\pgfpathmoveto{\pgfpoint{205.470490pt}{148.880356pt}}
\pgflineto{\pgfpoint{205.422180pt}{148.802292pt}}
\pgfusepath{stroke}
\pgfpathmoveto{\pgfpoint{205.384995pt}{148.846893pt}}
\pgflineto{\pgfpoint{205.470490pt}{148.880356pt}}
\pgfusepath{stroke}
\pgfpathmoveto{\pgfpoint{205.481628pt}{154.865540pt}}
\pgflineto{\pgfpoint{205.429321pt}{154.787094pt}}
\pgfusepath{stroke}
\pgfpathmoveto{\pgfpoint{205.392700pt}{154.834167pt}}
\pgflineto{\pgfpoint{205.481628pt}{154.865540pt}}
\pgfusepath{stroke}
\pgfpathmoveto{\pgfpoint{205.493164pt}{160.849716pt}}
\pgflineto{\pgfpoint{205.436615pt}{160.771179pt}}
\pgfusepath{stroke}
\pgfpathmoveto{\pgfpoint{205.400787pt}{160.820816pt}}
\pgflineto{\pgfpoint{205.493164pt}{160.849716pt}}
\pgfusepath{stroke}
\pgfpathmoveto{\pgfpoint{205.504990pt}{166.832733pt}}
\pgflineto{\pgfpoint{205.443970pt}{166.754456pt}}
\pgfusepath{stroke}
\pgfpathmoveto{\pgfpoint{205.409210pt}{166.806732pt}}
\pgflineto{\pgfpoint{205.504990pt}{166.832733pt}}
\pgfusepath{stroke}
\pgfpathmoveto{\pgfpoint{205.516983pt}{172.814484pt}}
\pgflineto{\pgfpoint{205.451294pt}{172.736877pt}}
\pgfusepath{stroke}
\pgfpathmoveto{\pgfpoint{205.417877pt}{172.791809pt}}
\pgflineto{\pgfpoint{205.516983pt}{172.814484pt}}
\pgfusepath{stroke}
\pgfpathmoveto{\pgfpoint{205.528961pt}{178.794861pt}}
\pgflineto{\pgfpoint{205.458466pt}{178.718384pt}}
\pgfusepath{stroke}
\pgfpathmoveto{\pgfpoint{205.426682pt}{178.775970pt}}
\pgflineto{\pgfpoint{205.528961pt}{178.794861pt}}
\pgfusepath{stroke}
\pgfpathmoveto{\pgfpoint{205.540710pt}{184.773758pt}}
\pgflineto{\pgfpoint{205.465317pt}{184.698929pt}}
\pgfusepath{stroke}
\pgfpathmoveto{\pgfpoint{205.435486pt}{184.759125pt}}
\pgflineto{\pgfpoint{205.540710pt}{184.773758pt}}
\pgfusepath{stroke}
\pgfpathmoveto{\pgfpoint{205.551971pt}{190.751129pt}}
\pgflineto{\pgfpoint{205.471680pt}{190.678513pt}}
\pgfusepath{stroke}
\pgfpathmoveto{\pgfpoint{205.444153pt}{190.741226pt}}
\pgflineto{\pgfpoint{205.551971pt}{190.751129pt}}
\pgfusepath{stroke}
\pgfpathmoveto{\pgfpoint{205.562500pt}{196.727020pt}}
\pgflineto{\pgfpoint{205.477356pt}{196.657181pt}}
\pgfusepath{stroke}
\pgfpathmoveto{\pgfpoint{205.452484pt}{196.722229pt}}
\pgflineto{\pgfpoint{205.562500pt}{196.727020pt}}
\pgfusepath{stroke}
\pgfpathmoveto{\pgfpoint{205.571976pt}{202.701477pt}}
\pgflineto{\pgfpoint{205.482208pt}{202.635025pt}}
\pgfusepath{stroke}
\pgfpathmoveto{\pgfpoint{205.460297pt}{202.702179pt}}
\pgflineto{\pgfpoint{205.571976pt}{202.701477pt}}
\pgfusepath{stroke}
\pgfpathmoveto{\pgfpoint{205.580139pt}{208.674652pt}}
\pgflineto{\pgfpoint{205.486023pt}{208.612152pt}}
\pgfusepath{stroke}
\pgfpathmoveto{\pgfpoint{205.467346pt}{208.681122pt}}
\pgflineto{\pgfpoint{205.580139pt}{208.674652pt}}
\pgfusepath{stroke}
\pgfpathmoveto{\pgfpoint{205.586746pt}{214.646759pt}}
\pgflineto{\pgfpoint{205.488693pt}{214.588745pt}}
\pgfusepath{stroke}
\pgfpathmoveto{\pgfpoint{205.473495pt}{214.659180pt}}
\pgflineto{\pgfpoint{205.586746pt}{214.646759pt}}
\pgfusepath{stroke}
\pgfpathmoveto{\pgfpoint{205.591614pt}{220.618088pt}}
\pgflineto{\pgfpoint{205.490112pt}{220.565002pt}}
\pgfusepath{stroke}
\pgfpathmoveto{\pgfpoint{205.478546pt}{220.636520pt}}
\pgflineto{\pgfpoint{205.591614pt}{220.618088pt}}
\pgfusepath{stroke}
\pgfpathmoveto{\pgfpoint{205.594604pt}{226.588943pt}}
\pgflineto{\pgfpoint{205.490234pt}{226.541168pt}}
\pgfusepath{stroke}
\pgfpathmoveto{\pgfpoint{205.482437pt}{226.613342pt}}
\pgflineto{\pgfpoint{205.594604pt}{226.588943pt}}
\pgfusepath{stroke}
\pgfpathmoveto{\pgfpoint{205.595688pt}{232.559692pt}}
\pgflineto{\pgfpoint{205.489059pt}{232.517456pt}}
\pgfusepath{stroke}
\pgfpathmoveto{\pgfpoint{205.485046pt}{232.589874pt}}
\pgflineto{\pgfpoint{205.595688pt}{232.559692pt}}
\pgfusepath{stroke}
\pgfpathmoveto{\pgfpoint{205.594925pt}{238.530640pt}}
\pgflineto{\pgfpoint{205.486694pt}{238.494095pt}}
\pgfusepath{stroke}
\pgfpathmoveto{\pgfpoint{205.486404pt}{238.566345pt}}
\pgflineto{\pgfpoint{205.594925pt}{238.530640pt}}
\pgfusepath{stroke}
\pgfpathmoveto{\pgfpoint{205.592422pt}{244.502106pt}}
\pgflineto{\pgfpoint{205.483215pt}{244.471268pt}}
\pgfusepath{stroke}
\pgfpathmoveto{\pgfpoint{205.486542pt}{244.542969pt}}
\pgflineto{\pgfpoint{205.592422pt}{244.502106pt}}
\pgfusepath{stroke}
\pgfpathmoveto{\pgfpoint{205.588379pt}{250.474319pt}}
\pgflineto{\pgfpoint{205.478790pt}{250.449112pt}}
\pgfusepath{stroke}
\pgfpathmoveto{\pgfpoint{205.485565pt}{250.519913pt}}
\pgflineto{\pgfpoint{205.588379pt}{250.474319pt}}
\pgfusepath{stroke}
\pgfpathmoveto{\pgfpoint{205.582977pt}{256.447479pt}}
\pgflineto{\pgfpoint{205.473572pt}{256.427734pt}}
\pgfusepath{stroke}
\pgfpathmoveto{\pgfpoint{205.483597pt}{256.497345pt}}
\pgflineto{\pgfpoint{205.582977pt}{256.447479pt}}
\pgfusepath{stroke}
\pgfpathmoveto{\pgfpoint{205.576477pt}{262.421692pt}}
\pgflineto{\pgfpoint{205.467728pt}{262.407196pt}}
\pgfusepath{stroke}
\pgfpathmoveto{\pgfpoint{205.480774pt}{262.475342pt}}
\pgflineto{\pgfpoint{205.576477pt}{262.421692pt}}
\pgfusepath{stroke}
\pgfpathmoveto{\pgfpoint{205.569107pt}{268.397034pt}}
\pgflineto{\pgfpoint{205.461426pt}{268.387482pt}}
\pgfusepath{stroke}
\pgfpathmoveto{\pgfpoint{205.477234pt}{268.454010pt}}
\pgflineto{\pgfpoint{205.569107pt}{268.397034pt}}
\pgfusepath{stroke}
\pgfpathmoveto{\pgfpoint{205.561066pt}{274.373535pt}}
\pgflineto{\pgfpoint{205.454834pt}{274.368652pt}}
\pgfusepath{stroke}
\pgfpathmoveto{\pgfpoint{205.473114pt}{274.433350pt}}
\pgflineto{\pgfpoint{205.561066pt}{274.373535pt}}
\pgfusepath{stroke}
\pgfpathmoveto{\pgfpoint{205.552551pt}{280.351135pt}}
\pgflineto{\pgfpoint{205.448029pt}{280.350586pt}}
\pgfusepath{stroke}
\pgfpathmoveto{\pgfpoint{205.468567pt}{280.413422pt}}
\pgflineto{\pgfpoint{205.552551pt}{280.351135pt}}
\pgfusepath{stroke}
\pgfpathmoveto{\pgfpoint{205.543701pt}{286.329834pt}}
\pgflineto{\pgfpoint{205.441116pt}{286.333252pt}}
\pgfusepath{stroke}
\pgfpathmoveto{\pgfpoint{205.463699pt}{286.394135pt}}
\pgflineto{\pgfpoint{205.543701pt}{286.329834pt}}
\pgfusepath{stroke}
\pgfpathmoveto{\pgfpoint{205.534698pt}{292.309509pt}}
\pgflineto{\pgfpoint{205.434204pt}{292.316650pt}}
\pgfusepath{stroke}
\pgfpathmoveto{\pgfpoint{205.458572pt}{292.375488pt}}
\pgflineto{\pgfpoint{205.534698pt}{292.309509pt}}
\pgfusepath{stroke}
\pgfpathmoveto{\pgfpoint{205.525574pt}{298.290131pt}}
\pgflineto{\pgfpoint{205.427338pt}{298.300659pt}}
\pgfusepath{stroke}
\pgfpathmoveto{\pgfpoint{205.453278pt}{298.357483pt}}
\pgflineto{\pgfpoint{205.525574pt}{298.290131pt}}
\pgfusepath{stroke}
\pgfpathmoveto{\pgfpoint{205.516449pt}{304.271606pt}}
\pgflineto{\pgfpoint{205.420563pt}{304.285217pt}}
\pgfusepath{stroke}
\pgfpathmoveto{\pgfpoint{205.447906pt}{304.340057pt}}
\pgflineto{\pgfpoint{205.516449pt}{304.271606pt}}
\pgfusepath{stroke}
\pgfpathmoveto{\pgfpoint{205.507385pt}{310.253906pt}}
\pgflineto{\pgfpoint{205.413910pt}{310.270355pt}}
\pgfusepath{stroke}
\pgfpathmoveto{\pgfpoint{205.442474pt}{310.323151pt}}
\pgflineto{\pgfpoint{205.507385pt}{310.253906pt}}
\pgfusepath{stroke}
\pgfpathmoveto{\pgfpoint{205.498413pt}{316.236908pt}}
\pgflineto{\pgfpoint{205.407394pt}{316.255951pt}}
\pgfusepath{stroke}
\pgfpathmoveto{\pgfpoint{205.437012pt}{316.306763pt}}
\pgflineto{\pgfpoint{205.498413pt}{316.236908pt}}
\pgfusepath{stroke}
\pgfpathmoveto{\pgfpoint{205.489594pt}{322.220612pt}}
\pgflineto{\pgfpoint{205.401062pt}{322.241943pt}}
\pgfusepath{stroke}
\pgfpathmoveto{\pgfpoint{205.431580pt}{322.290802pt}}
\pgflineto{\pgfpoint{205.489594pt}{322.220612pt}}
\pgfusepath{stroke}
\pgfpathmoveto{\pgfpoint{205.480957pt}{328.204895pt}}
\pgflineto{\pgfpoint{205.394928pt}{328.228394pt}}
\pgfusepath{stroke}
\pgfpathmoveto{\pgfpoint{205.426208pt}{328.275299pt}}
\pgflineto{\pgfpoint{205.480957pt}{328.204895pt}}
\pgfusepath{stroke}
\pgfpathmoveto{\pgfpoint{205.472504pt}{334.189758pt}}
\pgflineto{\pgfpoint{205.388977pt}{334.215149pt}}
\pgfusepath{stroke}
\pgfpathmoveto{\pgfpoint{205.420898pt}{334.260193pt}}
\pgflineto{\pgfpoint{205.472504pt}{334.189758pt}}
\pgfusepath{stroke}
\pgfpathmoveto{\pgfpoint{205.464264pt}{340.175171pt}}
\pgflineto{\pgfpoint{205.383209pt}{340.202240pt}}
\pgfusepath{stroke}
\pgfpathmoveto{\pgfpoint{205.415680pt}{340.245453pt}}
\pgflineto{\pgfpoint{205.464264pt}{340.175171pt}}
\pgfusepath{stroke}
\pgfpathmoveto{\pgfpoint{205.456284pt}{346.161011pt}}
\pgflineto{\pgfpoint{205.377686pt}{346.189636pt}}
\pgfusepath{stroke}
\pgfpathmoveto{\pgfpoint{205.410553pt}{346.231079pt}}
\pgflineto{\pgfpoint{205.456284pt}{346.161011pt}}
\pgfusepath{stroke}
\pgfpathmoveto{\pgfpoint{205.448517pt}{352.147339pt}}
\pgflineto{\pgfpoint{205.372345pt}{352.177246pt}}
\pgfusepath{stroke}
\pgfpathmoveto{\pgfpoint{205.405548pt}{352.216980pt}}
\pgflineto{\pgfpoint{205.448517pt}{352.147339pt}}
\pgfusepath{stroke}
\pgfpathmoveto{\pgfpoint{205.441025pt}{358.134033pt}}
\pgflineto{\pgfpoint{205.367218pt}{358.165161pt}}
\pgfusepath{stroke}
\pgfpathmoveto{\pgfpoint{205.400665pt}{358.203186pt}}
\pgflineto{\pgfpoint{205.441025pt}{358.134033pt}}
\pgfusepath{stroke}
\pgfpathmoveto{\pgfpoint{205.433777pt}{364.121063pt}}
\pgflineto{\pgfpoint{205.362335pt}{364.153229pt}}
\pgfusepath{stroke}
\pgfpathmoveto{\pgfpoint{205.395920pt}{364.189667pt}}
\pgflineto{\pgfpoint{205.433777pt}{364.121063pt}}
\pgfusepath{stroke}
\pgfpathmoveto{\pgfpoint{205.426819pt}{370.108429pt}}
\pgflineto{\pgfpoint{205.357635pt}{370.141510pt}}
\pgfusepath{stroke}
\pgfpathmoveto{\pgfpoint{205.391312pt}{370.176392pt}}
\pgflineto{\pgfpoint{205.426819pt}{370.108429pt}}
\pgfusepath{stroke}
\pgfpathmoveto{\pgfpoint{211.361084pt}{77.022415pt}}
\pgflineto{\pgfpoint{211.344604pt}{76.956772pt}}
\pgfusepath{stroke}
\pgfpathmoveto{\pgfpoint{211.308502pt}{76.979782pt}}
\pgflineto{\pgfpoint{211.361084pt}{77.022415pt}}
\pgfusepath{stroke}
\pgfpathmoveto{\pgfpoint{211.366547pt}{83.012695pt}}
\pgflineto{\pgfpoint{211.348541pt}{82.945602pt}}
\pgfusepath{stroke}
\pgfpathmoveto{\pgfpoint{211.311890pt}{82.969833pt}}
\pgflineto{\pgfpoint{211.366547pt}{83.012695pt}}
\pgfusepath{stroke}
\pgfpathmoveto{\pgfpoint{211.372421pt}{89.002945pt}}
\pgflineto{\pgfpoint{211.352737pt}{88.934357pt}}
\pgfusepath{stroke}
\pgfpathmoveto{\pgfpoint{211.315521pt}{88.959892pt}}
\pgflineto{\pgfpoint{211.372421pt}{89.002945pt}}
\pgfusepath{stroke}
\pgfpathmoveto{\pgfpoint{211.378723pt}{94.993088pt}}
\pgflineto{\pgfpoint{211.357208pt}{94.923004pt}}
\pgfusepath{stroke}
\pgfpathmoveto{\pgfpoint{211.319458pt}{94.949944pt}}
\pgflineto{\pgfpoint{211.378723pt}{94.993088pt}}
\pgfusepath{stroke}
\pgfpathmoveto{\pgfpoint{211.385498pt}{100.983124pt}}
\pgflineto{\pgfpoint{211.361969pt}{100.911530pt}}
\pgfusepath{stroke}
\pgfpathmoveto{\pgfpoint{211.323730pt}{100.939964pt}}
\pgflineto{\pgfpoint{211.385498pt}{100.983124pt}}
\pgfusepath{stroke}
\pgfpathmoveto{\pgfpoint{211.392761pt}{106.972977pt}}
\pgflineto{\pgfpoint{211.367035pt}{106.899887pt}}
\pgfusepath{stroke}
\pgfpathmoveto{\pgfpoint{211.328339pt}{106.929924pt}}
\pgflineto{\pgfpoint{211.392761pt}{106.972977pt}}
\pgfusepath{stroke}
\pgfpathmoveto{\pgfpoint{211.400558pt}{112.962631pt}}
\pgflineto{\pgfpoint{211.372452pt}{112.888039pt}}
\pgfusepath{stroke}
\pgfpathmoveto{\pgfpoint{211.333313pt}{112.919807pt}}
\pgflineto{\pgfpoint{211.400558pt}{112.962631pt}}
\pgfusepath{stroke}
\pgfpathmoveto{\pgfpoint{211.408936pt}{118.952011pt}}
\pgflineto{\pgfpoint{211.378204pt}{118.875946pt}}
\pgfusepath{stroke}
\pgfpathmoveto{\pgfpoint{211.338715pt}{118.909584pt}}
\pgflineto{\pgfpoint{211.408936pt}{118.952011pt}}
\pgfusepath{stroke}
\pgfpathmoveto{\pgfpoint{211.417877pt}{124.941040pt}}
\pgflineto{\pgfpoint{211.384323pt}{124.863556pt}}
\pgfusepath{stroke}
\pgfpathmoveto{\pgfpoint{211.344543pt}{124.899185pt}}
\pgflineto{\pgfpoint{211.417877pt}{124.941040pt}}
\pgfusepath{stroke}
\pgfpathmoveto{\pgfpoint{211.427460pt}{130.929626pt}}
\pgflineto{\pgfpoint{211.390808pt}{130.850800pt}}
\pgfusepath{stroke}
\pgfpathmoveto{\pgfpoint{211.350845pt}{130.888565pt}}
\pgflineto{\pgfpoint{211.427460pt}{130.929626pt}}
\pgfusepath{stroke}
\pgfpathmoveto{\pgfpoint{211.437698pt}{136.917694pt}}
\pgflineto{\pgfpoint{211.397675pt}{136.837631pt}}
\pgfusepath{stroke}
\pgfpathmoveto{\pgfpoint{211.357635pt}{136.877655pt}}
\pgflineto{\pgfpoint{211.437698pt}{136.917694pt}}
\pgfusepath{stroke}
\pgfpathmoveto{\pgfpoint{211.448578pt}{142.905106pt}}
\pgflineto{\pgfpoint{211.404877pt}{142.823944pt}}
\pgfusepath{stroke}
\pgfpathmoveto{\pgfpoint{211.364929pt}{142.866394pt}}
\pgflineto{\pgfpoint{211.448578pt}{142.905106pt}}
\pgfusepath{stroke}
\pgfpathmoveto{\pgfpoint{211.460083pt}{148.891724pt}}
\pgflineto{\pgfpoint{211.412430pt}{148.809662pt}}
\pgfusepath{stroke}
\pgfpathmoveto{\pgfpoint{211.372711pt}{148.854675pt}}
\pgflineto{\pgfpoint{211.460083pt}{148.891724pt}}
\pgfusepath{stroke}
\pgfpathmoveto{\pgfpoint{211.472198pt}{154.877426pt}}
\pgflineto{\pgfpoint{211.420288pt}{154.794693pt}}
\pgfusepath{stroke}
\pgfpathmoveto{\pgfpoint{211.381027pt}{154.842407pt}}
\pgflineto{\pgfpoint{211.472198pt}{154.877426pt}}
\pgfusepath{stroke}
\pgfpathmoveto{\pgfpoint{211.484879pt}{160.862000pt}}
\pgflineto{\pgfpoint{211.428375pt}{160.778931pt}}
\pgfusepath{stroke}
\pgfpathmoveto{\pgfpoint{211.389832pt}{160.829453pt}}
\pgflineto{\pgfpoint{211.484879pt}{160.862000pt}}
\pgfusepath{stroke}
\pgfpathmoveto{\pgfpoint{211.497986pt}{166.845337pt}}
\pgflineto{\pgfpoint{211.436615pt}{166.762283pt}}
\pgfusepath{stroke}
\pgfpathmoveto{\pgfpoint{211.399063pt}{166.815704pt}}
\pgflineto{\pgfpoint{211.497986pt}{166.845337pt}}
\pgfusepath{stroke}
\pgfpathmoveto{\pgfpoint{211.511414pt}{172.827240pt}}
\pgflineto{\pgfpoint{211.444916pt}{172.744629pt}}
\pgfusepath{stroke}
\pgfpathmoveto{\pgfpoint{211.408661pt}{172.801056pt}}
\pgflineto{\pgfpoint{211.511414pt}{172.827240pt}}
\pgfusepath{stroke}
\pgfpathmoveto{\pgfpoint{211.524933pt}{178.807526pt}}
\pgflineto{\pgfpoint{211.453094pt}{178.725922pt}}
\pgfusepath{stroke}
\pgfpathmoveto{\pgfpoint{211.418503pt}{178.785339pt}}
\pgflineto{\pgfpoint{211.524933pt}{178.807526pt}}
\pgfusepath{stroke}
\pgfpathmoveto{\pgfpoint{211.538300pt}{184.786102pt}}
\pgflineto{\pgfpoint{211.460999pt}{184.706070pt}}
\pgfusepath{stroke}
\pgfpathmoveto{\pgfpoint{211.428436pt}{184.768463pt}}
\pgflineto{\pgfpoint{211.538300pt}{184.786102pt}}
\pgfusepath{stroke}
\pgfpathmoveto{\pgfpoint{211.551239pt}{190.762878pt}}
\pgflineto{\pgfpoint{211.468414pt}{190.685059pt}}
\pgfusepath{stroke}
\pgfpathmoveto{\pgfpoint{211.438278pt}{190.750336pt}}
\pgflineto{\pgfpoint{211.551239pt}{190.762878pt}}
\pgfusepath{stroke}
\pgfpathmoveto{\pgfpoint{211.563385pt}{196.737823pt}}
\pgflineto{\pgfpoint{211.475067pt}{196.662933pt}}
\pgfusepath{stroke}
\pgfpathmoveto{\pgfpoint{211.447815pt}{196.730896pt}}
\pgflineto{\pgfpoint{211.563385pt}{196.737823pt}}
\pgfusepath{stroke}
\pgfpathmoveto{\pgfpoint{211.574387pt}{202.711029pt}}
\pgflineto{\pgfpoint{211.480789pt}{202.639771pt}}
\pgfusepath{stroke}
\pgfpathmoveto{\pgfpoint{211.456757pt}{202.710175pt}}
\pgflineto{\pgfpoint{211.574387pt}{202.711029pt}}
\pgfusepath{stroke}
\pgfpathmoveto{\pgfpoint{211.583878pt}{208.682648pt}}
\pgflineto{\pgfpoint{211.485321pt}{208.615707pt}}
\pgfusepath{stroke}
\pgfpathmoveto{\pgfpoint{211.464874pt}{208.688217pt}}
\pgflineto{\pgfpoint{211.583878pt}{208.682648pt}}
\pgfusepath{stroke}
\pgfpathmoveto{\pgfpoint{211.591522pt}{214.652939pt}}
\pgflineto{\pgfpoint{211.488480pt}{214.590973pt}}
\pgfusepath{stroke}
\pgfpathmoveto{\pgfpoint{211.471909pt}{214.665192pt}}
\pgflineto{\pgfpoint{211.591522pt}{214.652939pt}}
\pgfusepath{stroke}
\pgfpathmoveto{\pgfpoint{211.597076pt}{220.622269pt}}
\pgflineto{\pgfpoint{211.490143pt}{220.565826pt}}
\pgfusepath{stroke}
\pgfpathmoveto{\pgfpoint{211.477661pt}{220.641281pt}}
\pgflineto{\pgfpoint{211.597076pt}{220.622269pt}}
\pgfusepath{stroke}
\pgfpathmoveto{\pgfpoint{211.600372pt}{226.591049pt}}
\pgflineto{\pgfpoint{211.490234pt}{226.540558pt}}
\pgfusepath{stroke}
\pgfpathmoveto{\pgfpoint{211.481964pt}{226.616745pt}}
\pgflineto{\pgfpoint{211.600372pt}{226.591049pt}}
\pgfusepath{stroke}
\pgfpathmoveto{\pgfpoint{211.601410pt}{232.559723pt}}
\pgflineto{\pgfpoint{211.488800pt}{232.515472pt}}
\pgfusepath{stroke}
\pgfpathmoveto{\pgfpoint{211.484772pt}{232.591888pt}}
\pgflineto{\pgfpoint{211.601410pt}{232.559723pt}}
\pgfusepath{stroke}
\pgfpathmoveto{\pgfpoint{211.600189pt}{238.528702pt}}
\pgflineto{\pgfpoint{211.485901pt}{238.490860pt}}
\pgfusepath{stroke}
\pgfpathmoveto{\pgfpoint{211.486053pt}{238.567001pt}}
\pgflineto{\pgfpoint{211.600189pt}{238.528702pt}}
\pgfusepath{stroke}
\pgfpathmoveto{\pgfpoint{211.596954pt}{244.498383pt}}
\pgflineto{\pgfpoint{211.481720pt}{244.466919pt}}
\pgfusepath{stroke}
\pgfpathmoveto{\pgfpoint{211.485901pt}{244.542343pt}}
\pgflineto{\pgfpoint{211.596954pt}{244.498383pt}}
\pgfusepath{stroke}
\pgfpathmoveto{\pgfpoint{211.591904pt}{250.469055pt}}
\pgflineto{\pgfpoint{211.476471pt}{250.443848pt}}
\pgfusepath{stroke}
\pgfpathmoveto{\pgfpoint{211.484436pt}{250.518143pt}}
\pgflineto{\pgfpoint{211.591904pt}{250.469055pt}}
\pgfusepath{stroke}
\pgfpathmoveto{\pgfpoint{211.585373pt}{256.440918pt}}
\pgflineto{\pgfpoint{211.470352pt}{256.421753pt}}
\pgfusepath{stroke}
\pgfpathmoveto{\pgfpoint{211.481842pt}{256.494598pt}}
\pgflineto{\pgfpoint{211.585373pt}{256.440918pt}}
\pgfusepath{stroke}
\pgfpathmoveto{\pgfpoint{211.577667pt}{262.414124pt}}
\pgflineto{\pgfpoint{211.463593pt}{262.400635pt}}
\pgfusepath{stroke}
\pgfpathmoveto{\pgfpoint{211.478317pt}{262.471802pt}}
\pgflineto{\pgfpoint{211.577667pt}{262.414124pt}}
\pgfusepath{stroke}
\pgfpathmoveto{\pgfpoint{211.569031pt}{268.388702pt}}
\pgflineto{\pgfpoint{211.456375pt}{268.380585pt}}
\pgfusepath{stroke}
\pgfpathmoveto{\pgfpoint{211.474030pt}{268.449799pt}}
\pgflineto{\pgfpoint{211.569031pt}{268.388702pt}}
\pgfusepath{stroke}
\pgfpathmoveto{\pgfpoint{211.559753pt}{274.364655pt}}
\pgflineto{\pgfpoint{211.448868pt}{274.361481pt}}
\pgfusepath{stroke}
\pgfpathmoveto{\pgfpoint{211.469162pt}{274.428650pt}}
\pgflineto{\pgfpoint{211.559753pt}{274.364655pt}}
\pgfusepath{stroke}
\pgfpathmoveto{\pgfpoint{211.550049pt}{280.341888pt}}
\pgflineto{\pgfpoint{211.441238pt}{280.343323pt}}
\pgfusepath{stroke}
\pgfpathmoveto{\pgfpoint{211.463867pt}{280.408325pt}}
\pgflineto{\pgfpoint{211.550049pt}{280.341888pt}}
\pgfusepath{stroke}
\pgfpathmoveto{\pgfpoint{211.540100pt}{286.320343pt}}
\pgflineto{\pgfpoint{211.433594pt}{286.325989pt}}
\pgfusepath{stroke}
\pgfpathmoveto{\pgfpoint{211.458267pt}{286.388763pt}}
\pgflineto{\pgfpoint{211.540100pt}{286.320343pt}}
\pgfusepath{stroke}
\pgfpathmoveto{\pgfpoint{211.530060pt}{292.299927pt}}
\pgflineto{\pgfpoint{211.425964pt}{292.309387pt}}
\pgfusepath{stroke}
\pgfpathmoveto{\pgfpoint{211.452469pt}{292.369965pt}}
\pgflineto{\pgfpoint{211.530060pt}{292.299927pt}}
\pgfusepath{stroke}
\pgfpathmoveto{\pgfpoint{211.520004pt}{298.280579pt}}
\pgflineto{\pgfpoint{211.418472pt}{298.293518pt}}
\pgfusepath{stroke}
\pgfpathmoveto{\pgfpoint{211.446564pt}{298.351837pt}}
\pgflineto{\pgfpoint{211.520004pt}{298.280579pt}}
\pgfusepath{stroke}
\pgfpathmoveto{\pgfpoint{211.510010pt}{304.262115pt}}
\pgflineto{\pgfpoint{211.411133pt}{304.278259pt}}
\pgfusepath{stroke}
\pgfpathmoveto{\pgfpoint{211.440582pt}{304.334351pt}}
\pgflineto{\pgfpoint{211.510010pt}{304.262115pt}}
\pgfusepath{stroke}
\pgfpathmoveto{\pgfpoint{211.500168pt}{310.244507pt}}
\pgflineto{\pgfpoint{211.403961pt}{310.263519pt}}
\pgfusepath{stroke}
\pgfpathmoveto{\pgfpoint{211.434616pt}{310.317444pt}}
\pgflineto{\pgfpoint{211.500168pt}{310.244507pt}}
\pgfusepath{stroke}
\pgfpathmoveto{\pgfpoint{211.490494pt}{316.227692pt}}
\pgflineto{\pgfpoint{211.397003pt}{316.249298pt}}
\pgfusepath{stroke}
\pgfpathmoveto{\pgfpoint{211.428680pt}{316.301086pt}}
\pgflineto{\pgfpoint{211.490494pt}{316.227692pt}}
\pgfusepath{stroke}
\pgfpathmoveto{\pgfpoint{211.481003pt}{322.211548pt}}
\pgflineto{\pgfpoint{211.390259pt}{322.235535pt}}
\pgfusepath{stroke}
\pgfpathmoveto{\pgfpoint{211.422791pt}{322.285187pt}}
\pgflineto{\pgfpoint{211.481003pt}{322.211548pt}}
\pgfusepath{stroke}
\pgfpathmoveto{\pgfpoint{211.471771pt}{328.196075pt}}
\pgflineto{\pgfpoint{211.383728pt}{328.222168pt}}
\pgfusepath{stroke}
\pgfpathmoveto{\pgfpoint{211.416992pt}{328.269775pt}}
\pgflineto{\pgfpoint{211.471771pt}{328.196075pt}}
\pgfusepath{stroke}
\pgfpathmoveto{\pgfpoint{211.462769pt}{334.181183pt}}
\pgflineto{\pgfpoint{211.377441pt}{334.209137pt}}
\pgfusepath{stroke}
\pgfpathmoveto{\pgfpoint{211.411285pt}{334.254730pt}}
\pgflineto{\pgfpoint{211.462769pt}{334.181183pt}}
\pgfusepath{stroke}
\pgfpathmoveto{\pgfpoint{211.454041pt}{340.166809pt}}
\pgflineto{\pgfpoint{211.371384pt}{340.196442pt}}
\pgfusepath{stroke}
\pgfpathmoveto{\pgfpoint{211.405685pt}{340.240112pt}}
\pgflineto{\pgfpoint{211.454041pt}{340.166809pt}}
\pgfusepath{stroke}
\pgfpathmoveto{\pgfpoint{211.445572pt}{346.152924pt}}
\pgflineto{\pgfpoint{211.365570pt}{346.184052pt}}
\pgfusepath{stroke}
\pgfpathmoveto{\pgfpoint{211.400238pt}{346.225830pt}}
\pgflineto{\pgfpoint{211.445572pt}{346.152924pt}}
\pgfusepath{stroke}
\pgfpathmoveto{\pgfpoint{211.437408pt}{352.139496pt}}
\pgflineto{\pgfpoint{211.359985pt}{352.171936pt}}
\pgfusepath{stroke}
\pgfpathmoveto{\pgfpoint{211.394928pt}{352.211884pt}}
\pgflineto{\pgfpoint{211.437408pt}{352.139496pt}}
\pgfusepath{stroke}
\pgfpathmoveto{\pgfpoint{211.429535pt}{358.126465pt}}
\pgflineto{\pgfpoint{211.354645pt}{358.160034pt}}
\pgfusepath{stroke}
\pgfpathmoveto{\pgfpoint{211.389755pt}{358.198242pt}}
\pgflineto{\pgfpoint{211.429535pt}{358.126465pt}}
\pgfusepath{stroke}
\pgfpathmoveto{\pgfpoint{211.421936pt}{364.113770pt}}
\pgflineto{\pgfpoint{211.349548pt}{364.148346pt}}
\pgfusepath{stroke}
\pgfpathmoveto{\pgfpoint{211.384750pt}{364.184875pt}}
\pgflineto{\pgfpoint{211.421936pt}{364.113770pt}}
\pgfusepath{stroke}
\pgfpathmoveto{\pgfpoint{211.414658pt}{370.101440pt}}
\pgflineto{\pgfpoint{211.344666pt}{370.136841pt}}
\pgfusepath{stroke}
\pgfpathmoveto{\pgfpoint{211.379898pt}{370.171753pt}}
\pgflineto{\pgfpoint{211.414658pt}{370.101440pt}}
\pgfusepath{stroke}
\pgfpathmoveto{\pgfpoint{217.346100pt}{77.027710pt}}
\pgflineto{\pgfpoint{217.331116pt}{76.960602pt}}
\pgfusepath{stroke}
\pgfpathmoveto{\pgfpoint{217.293854pt}{76.983002pt}}
\pgflineto{\pgfpoint{217.346100pt}{77.027710pt}}
\pgfusepath{stroke}
\pgfpathmoveto{\pgfpoint{217.351593pt}{83.018387pt}}
\pgflineto{\pgfpoint{217.335114pt}{82.949707pt}}
\pgfusepath{stroke}
\pgfpathmoveto{\pgfpoint{217.297195pt}{82.973312pt}}
\pgflineto{\pgfpoint{217.351593pt}{83.018387pt}}
\pgfusepath{stroke}
\pgfpathmoveto{\pgfpoint{217.357513pt}{89.009056pt}}
\pgflineto{\pgfpoint{217.339386pt}{88.938736pt}}
\pgfusepath{stroke}
\pgfpathmoveto{\pgfpoint{217.300827pt}{88.963676pt}}
\pgflineto{\pgfpoint{217.357513pt}{89.009056pt}}
\pgfusepath{stroke}
\pgfpathmoveto{\pgfpoint{217.363892pt}{94.999664pt}}
\pgflineto{\pgfpoint{217.343964pt}{94.927681pt}}
\pgfusepath{stroke}
\pgfpathmoveto{\pgfpoint{217.304764pt}{94.954033pt}}
\pgflineto{\pgfpoint{217.363892pt}{94.999664pt}}
\pgfusepath{stroke}
\pgfpathmoveto{\pgfpoint{217.370789pt}{100.990204pt}}
\pgflineto{\pgfpoint{217.348877pt}{100.916519pt}}
\pgfusepath{stroke}
\pgfpathmoveto{\pgfpoint{217.309052pt}{100.944412pt}}
\pgflineto{\pgfpoint{217.370789pt}{100.990204pt}}
\pgfusepath{stroke}
\pgfpathmoveto{\pgfpoint{217.378235pt}{106.980598pt}}
\pgflineto{\pgfpoint{217.354126pt}{106.905220pt}}
\pgfusepath{stroke}
\pgfpathmoveto{\pgfpoint{217.313721pt}{106.934769pt}}
\pgflineto{\pgfpoint{217.378235pt}{106.980598pt}}
\pgfusepath{stroke}
\pgfpathmoveto{\pgfpoint{217.386246pt}{112.970848pt}}
\pgflineto{\pgfpoint{217.359741pt}{112.893738pt}}
\pgfusepath{stroke}
\pgfpathmoveto{\pgfpoint{217.318787pt}{112.925064pt}}
\pgflineto{\pgfpoint{217.386246pt}{112.970848pt}}
\pgfusepath{stroke}
\pgfpathmoveto{\pgfpoint{217.394897pt}{118.960838pt}}
\pgflineto{\pgfpoint{217.365753pt}{118.882019pt}}
\pgfusepath{stroke}
\pgfpathmoveto{\pgfpoint{217.324310pt}{118.915268pt}}
\pgflineto{\pgfpoint{217.394897pt}{118.960838pt}}
\pgfusepath{stroke}
\pgfpathmoveto{\pgfpoint{217.404236pt}{124.950523pt}}
\pgflineto{\pgfpoint{217.372192pt}{124.870033pt}}
\pgfusepath{stroke}
\pgfpathmoveto{\pgfpoint{217.330322pt}{124.905350pt}}
\pgflineto{\pgfpoint{217.404236pt}{124.950523pt}}
\pgfusepath{stroke}
\pgfpathmoveto{\pgfpoint{217.414291pt}{130.939804pt}}
\pgflineto{\pgfpoint{217.379074pt}{130.857697pt}}
\pgfusepath{stroke}
\pgfpathmoveto{\pgfpoint{217.336853pt}{130.895233pt}}
\pgflineto{\pgfpoint{217.414291pt}{130.939804pt}}
\pgfusepath{stroke}
\pgfpathmoveto{\pgfpoint{217.425110pt}{136.928574pt}}
\pgflineto{\pgfpoint{217.386398pt}{136.844910pt}}
\pgfusepath{stroke}
\pgfpathmoveto{\pgfpoint{217.343948pt}{136.884872pt}}
\pgflineto{\pgfpoint{217.425110pt}{136.928574pt}}
\pgfusepath{stroke}
\pgfpathmoveto{\pgfpoint{217.436691pt}{142.916687pt}}
\pgflineto{\pgfpoint{217.394165pt}{142.831635pt}}
\pgfusepath{stroke}
\pgfpathmoveto{\pgfpoint{217.351639pt}{142.874176pt}}
\pgflineto{\pgfpoint{217.436691pt}{142.916687pt}}
\pgfusepath{stroke}
\pgfpathmoveto{\pgfpoint{217.449097pt}{148.904022pt}}
\pgflineto{\pgfpoint{217.402374pt}{148.817719pt}}
\pgfusepath{stroke}
\pgfpathmoveto{\pgfpoint{217.359940pt}{148.863007pt}}
\pgflineto{\pgfpoint{217.449097pt}{148.904022pt}}
\pgfusepath{stroke}
\pgfpathmoveto{\pgfpoint{217.462250pt}{154.890381pt}}
\pgflineto{\pgfpoint{217.410995pt}{154.803085pt}}
\pgfusepath{stroke}
\pgfpathmoveto{\pgfpoint{217.368866pt}{154.851303pt}}
\pgflineto{\pgfpoint{217.462250pt}{154.890381pt}}
\pgfusepath{stroke}
\pgfpathmoveto{\pgfpoint{217.476151pt}{160.875580pt}}
\pgflineto{\pgfpoint{217.419983pt}{160.787582pt}}
\pgfusepath{stroke}
\pgfpathmoveto{\pgfpoint{217.378418pt}{160.838882pt}}
\pgflineto{\pgfpoint{217.476151pt}{160.875580pt}}
\pgfusepath{stroke}
\pgfpathmoveto{\pgfpoint{217.490692pt}{166.859375pt}}
\pgflineto{\pgfpoint{217.429230pt}{166.771088pt}}
\pgfusepath{stroke}
\pgfpathmoveto{\pgfpoint{217.388550pt}{166.825623pt}}
\pgflineto{\pgfpoint{217.490692pt}{166.859375pt}}
\pgfusepath{stroke}
\pgfpathmoveto{\pgfpoint{217.505722pt}{172.841568pt}}
\pgflineto{\pgfpoint{217.438629pt}{172.753464pt}}
\pgfusepath{stroke}
\pgfpathmoveto{\pgfpoint{217.399185pt}{172.811340pt}}
\pgflineto{\pgfpoint{217.505722pt}{172.841568pt}}
\pgfusepath{stroke}
\pgfpathmoveto{\pgfpoint{217.521027pt}{178.821930pt}}
\pgflineto{\pgfpoint{217.448029pt}{178.734589pt}}
\pgfusepath{stroke}
\pgfpathmoveto{\pgfpoint{217.410217pt}{178.795868pt}}
\pgflineto{\pgfpoint{217.521027pt}{178.821930pt}}
\pgfusepath{stroke}
\pgfpathmoveto{\pgfpoint{217.536346pt}{184.800278pt}}
\pgflineto{\pgfpoint{217.457184pt}{184.714371pt}}
\pgfusepath{stroke}
\pgfpathmoveto{\pgfpoint{217.421478pt}{184.779053pt}}
\pgflineto{\pgfpoint{217.536346pt}{184.800278pt}}
\pgfusepath{stroke}
\pgfpathmoveto{\pgfpoint{217.551300pt}{190.776459pt}}
\pgflineto{\pgfpoint{217.465851pt}{190.692749pt}}
\pgfusepath{stroke}
\pgfpathmoveto{\pgfpoint{217.432724pt}{190.760757pt}}
\pgflineto{\pgfpoint{217.551300pt}{190.776459pt}}
\pgfusepath{stroke}
\pgfpathmoveto{\pgfpoint{217.565430pt}{196.750412pt}}
\pgflineto{\pgfpoint{217.473755pt}{196.669739pt}}
\pgfusepath{stroke}
\pgfpathmoveto{\pgfpoint{217.443695pt}{196.740875pt}}
\pgflineto{\pgfpoint{217.565430pt}{196.750412pt}}
\pgfusepath{stroke}
\pgfpathmoveto{\pgfpoint{217.578339pt}{202.722183pt}}
\pgflineto{\pgfpoint{217.480576pt}{202.645401pt}}
\pgfusepath{stroke}
\pgfpathmoveto{\pgfpoint{217.454056pt}{202.719421pt}}
\pgflineto{\pgfpoint{217.578339pt}{202.722183pt}}
\pgfusepath{stroke}
\pgfpathmoveto{\pgfpoint{217.589447pt}{208.691971pt}}
\pgflineto{\pgfpoint{217.486008pt}{208.619949pt}}
\pgfusepath{stroke}
\pgfpathmoveto{\pgfpoint{217.463470pt}{208.696426pt}}
\pgflineto{\pgfpoint{217.589447pt}{208.691971pt}}
\pgfusepath{stroke}
\pgfpathmoveto{\pgfpoint{217.598404pt}{214.660095pt}}
\pgflineto{\pgfpoint{217.489777pt}{214.593613pt}}
\pgfusepath{stroke}
\pgfpathmoveto{\pgfpoint{217.471619pt}{214.672089pt}}
\pgflineto{\pgfpoint{217.598404pt}{214.660095pt}}
\pgfusepath{stroke}
\pgfpathmoveto{\pgfpoint{217.604813pt}{220.627014pt}}
\pgflineto{\pgfpoint{217.491730pt}{220.566772pt}}
\pgfusepath{stroke}
\pgfpathmoveto{\pgfpoint{217.478210pt}{220.646652pt}}
\pgflineto{\pgfpoint{217.604813pt}{220.627014pt}}
\pgfusepath{stroke}
\pgfpathmoveto{\pgfpoint{217.608459pt}{226.593262pt}}
\pgflineto{\pgfpoint{217.491791pt}{226.539780pt}}
\pgfusepath{stroke}
\pgfpathmoveto{\pgfpoint{217.483032pt}{226.620483pt}}
\pgflineto{\pgfpoint{217.608459pt}{226.593262pt}}
\pgfusepath{stroke}
\pgfpathmoveto{\pgfpoint{217.609344pt}{232.559418pt}}
\pgflineto{\pgfpoint{217.489960pt}{232.513031pt}}
\pgfusepath{stroke}
\pgfpathmoveto{\pgfpoint{217.486023pt}{232.593933pt}}
\pgflineto{\pgfpoint{217.609344pt}{232.559418pt}}
\pgfusepath{stroke}
\pgfpathmoveto{\pgfpoint{217.607513pt}{238.526047pt}}
\pgflineto{\pgfpoint{217.486404pt}{238.486908pt}}
\pgfusepath{stroke}
\pgfpathmoveto{\pgfpoint{217.487152pt}{238.567413pt}}
\pgflineto{\pgfpoint{217.607513pt}{238.526047pt}}
\pgfusepath{stroke}
\pgfpathmoveto{\pgfpoint{217.603271pt}{244.493637pt}}
\pgflineto{\pgfpoint{217.481339pt}{244.461700pt}}
\pgfusepath{stroke}
\pgfpathmoveto{\pgfpoint{217.486572pt}{244.541245pt}}
\pgflineto{\pgfpoint{217.603271pt}{244.493637pt}}
\pgfusepath{stroke}
\pgfpathmoveto{\pgfpoint{217.596954pt}{250.462555pt}}
\pgflineto{\pgfpoint{217.475037pt}{250.437592pt}}
\pgfusepath{stroke}
\pgfpathmoveto{\pgfpoint{217.484436pt}{250.515747pt}}
\pgflineto{\pgfpoint{217.596954pt}{250.462555pt}}
\pgfusepath{stroke}
\pgfpathmoveto{\pgfpoint{217.588974pt}{256.433014pt}}
\pgflineto{\pgfpoint{217.467789pt}{256.414703pt}}
\pgfusepath{stroke}
\pgfpathmoveto{\pgfpoint{217.481049pt}{256.491089pt}}
\pgflineto{\pgfpoint{217.588974pt}{256.433014pt}}
\pgfusepath{stroke}
\pgfpathmoveto{\pgfpoint{217.579727pt}{262.405151pt}}
\pgflineto{\pgfpoint{217.459900pt}{262.393066pt}}
\pgfusepath{stroke}
\pgfpathmoveto{\pgfpoint{217.476624pt}{262.467407pt}}
\pgflineto{\pgfpoint{217.579727pt}{262.405151pt}}
\pgfusepath{stroke}
\pgfpathmoveto{\pgfpoint{217.569580pt}{268.378998pt}}
\pgflineto{\pgfpoint{217.451584pt}{268.372681pt}}
\pgfusepath{stroke}
\pgfpathmoveto{\pgfpoint{217.471405pt}{268.444733pt}}
\pgflineto{\pgfpoint{217.569580pt}{268.378998pt}}
\pgfusepath{stroke}
\pgfpathmoveto{\pgfpoint{217.558853pt}{274.354431pt}}
\pgflineto{\pgfpoint{217.443054pt}{274.353424pt}}
\pgfusepath{stroke}
\pgfpathmoveto{\pgfpoint{217.465607pt}{274.423096pt}}
\pgflineto{\pgfpoint{217.558853pt}{274.354431pt}}
\pgfusepath{stroke}
\pgfpathmoveto{\pgfpoint{217.547821pt}{280.331421pt}}
\pgflineto{\pgfpoint{217.434479pt}{280.335205pt}}
\pgfusepath{stroke}
\pgfpathmoveto{\pgfpoint{217.459442pt}{280.402435pt}}
\pgflineto{\pgfpoint{217.547821pt}{280.331421pt}}
\pgfusepath{stroke}
\pgfpathmoveto{\pgfpoint{217.536621pt}{286.309753pt}}
\pgflineto{\pgfpoint{217.425995pt}{286.317932pt}}
\pgfusepath{stroke}
\pgfpathmoveto{\pgfpoint{217.453033pt}{286.382690pt}}
\pgflineto{\pgfpoint{217.536621pt}{286.309753pt}}
\pgfusepath{stroke}
\pgfpathmoveto{\pgfpoint{217.525436pt}{292.289337pt}}
\pgflineto{\pgfpoint{217.417633pt}{292.301483pt}}
\pgfusepath{stroke}
\pgfpathmoveto{\pgfpoint{217.446472pt}{292.363739pt}}
\pgflineto{\pgfpoint{217.525436pt}{292.289337pt}}
\pgfusepath{stroke}
\pgfpathmoveto{\pgfpoint{217.514359pt}{298.270050pt}}
\pgflineto{\pgfpoint{217.409454pt}{298.285797pt}}
\pgfusepath{stroke}
\pgfpathmoveto{\pgfpoint{217.439880pt}{298.345581pt}}
\pgflineto{\pgfpoint{217.514359pt}{298.270050pt}}
\pgfusepath{stroke}
\pgfpathmoveto{\pgfpoint{217.503464pt}{304.251770pt}}
\pgflineto{\pgfpoint{217.401520pt}{304.270691pt}}
\pgfusepath{stroke}
\pgfpathmoveto{\pgfpoint{217.433289pt}{304.328064pt}}
\pgflineto{\pgfpoint{217.503464pt}{304.251770pt}}
\pgfusepath{stroke}
\pgfpathmoveto{\pgfpoint{217.492767pt}{310.234344pt}}
\pgflineto{\pgfpoint{217.393829pt}{310.256226pt}}
\pgfusepath{stroke}
\pgfpathmoveto{\pgfpoint{217.426727pt}{310.311218pt}}
\pgflineto{\pgfpoint{217.492767pt}{310.234344pt}}
\pgfusepath{stroke}
\pgfpathmoveto{\pgfpoint{217.482346pt}{316.217743pt}}
\pgflineto{\pgfpoint{217.386398pt}{316.242218pt}}
\pgfusepath{stroke}
\pgfpathmoveto{\pgfpoint{217.420273pt}{316.294922pt}}
\pgflineto{\pgfpoint{217.482346pt}{316.217743pt}}
\pgfusepath{stroke}
\pgfpathmoveto{\pgfpoint{217.472198pt}{322.201874pt}}
\pgflineto{\pgfpoint{217.379211pt}{322.228699pt}}
\pgfusepath{stroke}
\pgfpathmoveto{\pgfpoint{217.413910pt}{322.279114pt}}
\pgflineto{\pgfpoint{217.472198pt}{322.201874pt}}
\pgfusepath{stroke}
\pgfpathmoveto{\pgfpoint{217.462326pt}{328.186646pt}}
\pgflineto{\pgfpoint{217.372314pt}{328.215576pt}}
\pgfusepath{stroke}
\pgfpathmoveto{\pgfpoint{217.407654pt}{328.263794pt}}
\pgflineto{\pgfpoint{217.462326pt}{328.186646pt}}
\pgfusepath{stroke}
\pgfpathmoveto{\pgfpoint{217.452759pt}{334.172028pt}}
\pgflineto{\pgfpoint{217.365677pt}{334.202789pt}}
\pgfusepath{stroke}
\pgfpathmoveto{\pgfpoint{217.401550pt}{334.248901pt}}
\pgflineto{\pgfpoint{217.452759pt}{334.172028pt}}
\pgfusepath{stroke}
\pgfpathmoveto{\pgfpoint{217.443512pt}{340.157959pt}}
\pgflineto{\pgfpoint{217.359299pt}{340.190369pt}}
\pgfusepath{stroke}
\pgfpathmoveto{\pgfpoint{217.395584pt}{340.234406pt}}
\pgflineto{\pgfpoint{217.443512pt}{340.157959pt}}
\pgfusepath{stroke}
\pgfpathmoveto{\pgfpoint{217.434586pt}{346.144409pt}}
\pgflineto{\pgfpoint{217.353195pt}{346.178223pt}}
\pgfusepath{stroke}
\pgfpathmoveto{\pgfpoint{217.389771pt}{346.220276pt}}
\pgflineto{\pgfpoint{217.434586pt}{346.144409pt}}
\pgfusepath{stroke}
\pgfpathmoveto{\pgfpoint{217.425980pt}{352.131287pt}}
\pgflineto{\pgfpoint{217.347366pt}{352.166321pt}}
\pgfusepath{stroke}
\pgfpathmoveto{\pgfpoint{217.384140pt}{352.206512pt}}
\pgflineto{\pgfpoint{217.425980pt}{352.131287pt}}
\pgfusepath{stroke}
\pgfpathmoveto{\pgfpoint{217.417725pt}{358.118530pt}}
\pgflineto{\pgfpoint{217.341812pt}{358.154694pt}}
\pgfusepath{stroke}
\pgfpathmoveto{\pgfpoint{217.378677pt}{358.193024pt}}
\pgflineto{\pgfpoint{217.417725pt}{358.118530pt}}
\pgfusepath{stroke}
\pgfpathmoveto{\pgfpoint{217.409790pt}{364.106201pt}}
\pgflineto{\pgfpoint{217.336517pt}{364.143250pt}}
\pgfusepath{stroke}
\pgfpathmoveto{\pgfpoint{217.373413pt}{364.179810pt}}
\pgflineto{\pgfpoint{217.409790pt}{364.106201pt}}
\pgfusepath{stroke}
\pgfpathmoveto{\pgfpoint{217.402191pt}{370.094177pt}}
\pgflineto{\pgfpoint{217.331482pt}{370.132019pt}}
\pgfusepath{stroke}
\pgfpathmoveto{\pgfpoint{217.368317pt}{370.166870pt}}
\pgflineto{\pgfpoint{217.402191pt}{370.094177pt}}
\pgfusepath{stroke}
\pgfpathmoveto{\pgfpoint{223.330719pt}{77.033020pt}}
\pgflineto{\pgfpoint{223.317383pt}{76.964478pt}}
\pgfusepath{stroke}
\pgfpathmoveto{\pgfpoint{223.278931pt}{76.986176pt}}
\pgflineto{\pgfpoint{223.330719pt}{77.033020pt}}
\pgfusepath{stroke}
\pgfpathmoveto{\pgfpoint{223.336212pt}{83.024109pt}}
\pgflineto{\pgfpoint{223.321426pt}{82.953857pt}}
\pgfusepath{stroke}
\pgfpathmoveto{\pgfpoint{223.282227pt}{82.976791pt}}
\pgflineto{\pgfpoint{223.336212pt}{83.024109pt}}
\pgfusepath{stroke}
\pgfpathmoveto{\pgfpoint{223.342148pt}{89.015228pt}}
\pgflineto{\pgfpoint{223.325745pt}{88.943192pt}}
\pgfusepath{stroke}
\pgfpathmoveto{\pgfpoint{223.285812pt}{88.967438pt}}
\pgflineto{\pgfpoint{223.342148pt}{89.015228pt}}
\pgfusepath{stroke}
\pgfpathmoveto{\pgfpoint{223.348587pt}{95.006332pt}}
\pgflineto{\pgfpoint{223.330414pt}{94.932472pt}}
\pgfusepath{stroke}
\pgfpathmoveto{\pgfpoint{223.289734pt}{94.958145pt}}
\pgflineto{\pgfpoint{223.348587pt}{95.006332pt}}
\pgfusepath{stroke}
\pgfpathmoveto{\pgfpoint{223.355560pt}{100.997398pt}}
\pgflineto{\pgfpoint{223.335419pt}{100.921661pt}}
\pgfusepath{stroke}
\pgfpathmoveto{\pgfpoint{223.294006pt}{100.948891pt}}
\pgflineto{\pgfpoint{223.355560pt}{100.997398pt}}
\pgfusepath{stroke}
\pgfpathmoveto{\pgfpoint{223.363098pt}{106.988403pt}}
\pgflineto{\pgfpoint{223.340820pt}{106.910744pt}}
\pgfusepath{stroke}
\pgfpathmoveto{\pgfpoint{223.298676pt}{106.939636pt}}
\pgflineto{\pgfpoint{223.363098pt}{106.988403pt}}
\pgfusepath{stroke}
\pgfpathmoveto{\pgfpoint{223.371307pt}{112.979279pt}}
\pgflineto{\pgfpoint{223.346649pt}{112.899658pt}}
\pgfusepath{stroke}
\pgfpathmoveto{\pgfpoint{223.303802pt}{112.930382pt}}
\pgflineto{\pgfpoint{223.371307pt}{112.979279pt}}
\pgfusepath{stroke}
\pgfpathmoveto{\pgfpoint{223.380219pt}{118.969971pt}}
\pgflineto{\pgfpoint{223.352905pt}{118.888390pt}}
\pgfusepath{stroke}
\pgfpathmoveto{\pgfpoint{223.309418pt}{118.921089pt}}
\pgflineto{\pgfpoint{223.380219pt}{118.969971pt}}
\pgfusepath{stroke}
\pgfpathmoveto{\pgfpoint{223.389862pt}{124.960403pt}}
\pgflineto{\pgfpoint{223.359634pt}{124.876854pt}}
\pgfusepath{stroke}
\pgfpathmoveto{\pgfpoint{223.315552pt}{124.911705pt}}
\pgflineto{\pgfpoint{223.389862pt}{124.960403pt}}
\pgfusepath{stroke}
\pgfpathmoveto{\pgfpoint{223.400360pt}{130.950470pt}}
\pgflineto{\pgfpoint{223.366882pt}{130.864990pt}}
\pgfusepath{stroke}
\pgfpathmoveto{\pgfpoint{223.322296pt}{130.902161pt}}
\pgflineto{\pgfpoint{223.400360pt}{130.950470pt}}
\pgfusepath{stroke}
\pgfpathmoveto{\pgfpoint{223.411728pt}{136.940094pt}}
\pgflineto{\pgfpoint{223.374680pt}{136.852722pt}}
\pgfusepath{stroke}
\pgfpathmoveto{\pgfpoint{223.329666pt}{136.892426pt}}
\pgflineto{\pgfpoint{223.411728pt}{136.940094pt}}
\pgfusepath{stroke}
\pgfpathmoveto{\pgfpoint{223.424026pt}{142.929077pt}}
\pgflineto{\pgfpoint{223.383026pt}{142.839935pt}}
\pgfusepath{stroke}
\pgfpathmoveto{\pgfpoint{223.337723pt}{142.882385pt}}
\pgflineto{\pgfpoint{223.424026pt}{142.929077pt}}
\pgfusepath{stroke}
\pgfpathmoveto{\pgfpoint{223.437302pt}{148.917297pt}}
\pgflineto{\pgfpoint{223.391907pt}{148.826523pt}}
\pgfusepath{stroke}
\pgfpathmoveto{\pgfpoint{223.346527pt}{148.871918pt}}
\pgflineto{\pgfpoint{223.437302pt}{148.917297pt}}
\pgfusepath{stroke}
\pgfpathmoveto{\pgfpoint{223.451538pt}{154.904510pt}}
\pgflineto{\pgfpoint{223.401352pt}{154.812332pt}}
\pgfusepath{stroke}
\pgfpathmoveto{\pgfpoint{223.356079pt}{154.860901pt}}
\pgflineto{\pgfpoint{223.451538pt}{154.904510pt}}
\pgfusepath{stroke}
\pgfpathmoveto{\pgfpoint{223.466766pt}{160.890533pt}}
\pgflineto{\pgfpoint{223.411301pt}{160.797241pt}}
\pgfusepath{stroke}
\pgfpathmoveto{\pgfpoint{223.366425pt}{160.849167pt}}
\pgflineto{\pgfpoint{223.466766pt}{160.890533pt}}
\pgfusepath{stroke}
\pgfpathmoveto{\pgfpoint{223.482864pt}{166.875031pt}}
\pgflineto{\pgfpoint{223.421677pt}{166.781036pt}}
\pgfusepath{stroke}
\pgfpathmoveto{\pgfpoint{223.377518pt}{166.836548pt}}
\pgflineto{\pgfpoint{223.482864pt}{166.875031pt}}
\pgfusepath{stroke}
\pgfpathmoveto{\pgfpoint{223.499725pt}{172.857742pt}}
\pgflineto{\pgfpoint{223.432358pt}{172.763550pt}}
\pgfusepath{stroke}
\pgfpathmoveto{\pgfpoint{223.389313pt}{172.822815pt}}
\pgflineto{\pgfpoint{223.499725pt}{172.857742pt}}
\pgfusepath{stroke}
\pgfpathmoveto{\pgfpoint{223.517120pt}{178.838379pt}}
\pgflineto{\pgfpoint{223.443161pt}{178.744629pt}}
\pgfusepath{stroke}
\pgfpathmoveto{\pgfpoint{223.401703pt}{178.807755pt}}
\pgflineto{\pgfpoint{223.517120pt}{178.838379pt}}
\pgfusepath{stroke}
\pgfpathmoveto{\pgfpoint{223.534729pt}{184.816635pt}}
\pgflineto{\pgfpoint{223.453842pt}{184.724091pt}}
\pgfusepath{stroke}
\pgfpathmoveto{\pgfpoint{223.414505pt}{184.791138pt}}
\pgflineto{\pgfpoint{223.534729pt}{184.816635pt}}
\pgfusepath{stroke}
\pgfpathmoveto{\pgfpoint{223.552124pt}{190.792282pt}}
\pgflineto{\pgfpoint{223.464096pt}{190.701843pt}}
\pgfusepath{stroke}
\pgfpathmoveto{\pgfpoint{223.427444pt}{190.772751pt}}
\pgflineto{\pgfpoint{223.552124pt}{190.792282pt}}
\pgfusepath{stroke}
\pgfpathmoveto{\pgfpoint{223.568756pt}{196.765198pt}}
\pgflineto{\pgfpoint{223.473541pt}{196.677856pt}}
\pgfusepath{stroke}
\pgfpathmoveto{\pgfpoint{223.440186pt}{196.752457pt}}
\pgflineto{\pgfpoint{223.568756pt}{196.765198pt}}
\pgfusepath{stroke}
\pgfpathmoveto{\pgfpoint{223.584015pt}{202.735367pt}}
\pgflineto{\pgfpoint{223.481750pt}{202.652191pt}}
\pgfusepath{stroke}
\pgfpathmoveto{\pgfpoint{223.452301pt}{202.730194pt}}
\pgflineto{\pgfpoint{223.584015pt}{202.735367pt}}
\pgfusepath{stroke}
\pgfpathmoveto{\pgfpoint{223.597260pt}{208.703003pt}}
\pgflineto{\pgfpoint{223.488342pt}{208.625061pt}}
\pgfusepath{stroke}
\pgfpathmoveto{\pgfpoint{223.463364pt}{208.705994pt}}
\pgflineto{\pgfpoint{223.597260pt}{208.703003pt}}
\pgfusepath{stroke}
\pgfpathmoveto{\pgfpoint{223.607849pt}{214.668488pt}}
\pgflineto{\pgfpoint{223.492935pt}{214.596802pt}}
\pgfusepath{stroke}
\pgfpathmoveto{\pgfpoint{223.472900pt}{214.680084pt}}
\pgflineto{\pgfpoint{223.607849pt}{214.668488pt}}
\pgfusepath{stroke}
\pgfpathmoveto{\pgfpoint{223.615326pt}{220.632431pt}}
\pgflineto{\pgfpoint{223.495270pt}{220.567841pt}}
\pgfusepath{stroke}
\pgfpathmoveto{\pgfpoint{223.480530pt}{220.652802pt}}
\pgflineto{\pgfpoint{223.615326pt}{220.632431pt}}
\pgfusepath{stroke}
\pgfpathmoveto{\pgfpoint{223.619385pt}{226.595551pt}}
\pgflineto{\pgfpoint{223.495239pt}{226.538727pt}}
\pgfusepath{stroke}
\pgfpathmoveto{\pgfpoint{223.485977pt}{226.624573pt}}
\pgflineto{\pgfpoint{223.619385pt}{226.595551pt}}
\pgfusepath{stroke}
\pgfpathmoveto{\pgfpoint{223.619980pt}{232.558624pt}}
\pgflineto{\pgfpoint{223.492889pt}{232.509964pt}}
\pgfusepath{stroke}
\pgfpathmoveto{\pgfpoint{223.489105pt}{232.595947pt}}
\pgflineto{\pgfpoint{223.619980pt}{232.558624pt}}
\pgfusepath{stroke}
\pgfpathmoveto{\pgfpoint{223.617279pt}{238.522430pt}}
\pgflineto{\pgfpoint{223.488434pt}{238.482056pt}}
\pgfusepath{stroke}
\pgfpathmoveto{\pgfpoint{223.489975pt}{238.567429pt}}
\pgflineto{\pgfpoint{223.617279pt}{238.522430pt}}
\pgfusepath{stroke}
\pgfpathmoveto{\pgfpoint{223.611664pt}{244.487549pt}}
\pgflineto{\pgfpoint{223.482178pt}{244.455353pt}}
\pgfusepath{stroke}
\pgfpathmoveto{\pgfpoint{223.488739pt}{244.539490pt}}
\pgflineto{\pgfpoint{223.611664pt}{244.487549pt}}
\pgfusepath{stroke}
\pgfpathmoveto{\pgfpoint{223.603668pt}{250.454483pt}}
\pgflineto{\pgfpoint{223.474533pt}{250.430099pt}}
\pgfusepath{stroke}
\pgfpathmoveto{\pgfpoint{223.485733pt}{250.512451pt}}
\pgflineto{\pgfpoint{223.603668pt}{250.454483pt}}
\pgfusepath{stroke}
\pgfpathmoveto{\pgfpoint{223.593826pt}{256.423431pt}}
\pgflineto{\pgfpoint{223.465881pt}{256.406403pt}}
\pgfusepath{stroke}
\pgfpathmoveto{\pgfpoint{223.481262pt}{256.486572pt}}
\pgflineto{\pgfpoint{223.593826pt}{256.423431pt}}
\pgfusepath{stroke}
\pgfpathmoveto{\pgfpoint{223.582687pt}{262.394501pt}}
\pgflineto{\pgfpoint{223.456604pt}{262.384277pt}}
\pgfusepath{stroke}
\pgfpathmoveto{\pgfpoint{223.475677pt}{262.461975pt}}
\pgflineto{\pgfpoint{223.582687pt}{262.394501pt}}
\pgfusepath{stroke}
\pgfpathmoveto{\pgfpoint{223.570709pt}{268.367645pt}}
\pgflineto{\pgfpoint{223.446991pt}{268.363647pt}}
\pgfusepath{stroke}
\pgfpathmoveto{\pgfpoint{223.469330pt}{268.438660pt}}
\pgflineto{\pgfpoint{223.570709pt}{268.367645pt}}
\pgfusepath{stroke}
\pgfpathmoveto{\pgfpoint{223.558273pt}{274.342712pt}}
\pgflineto{\pgfpoint{223.437286pt}{274.344299pt}}
\pgfusepath{stroke}
\pgfpathmoveto{\pgfpoint{223.462448pt}{274.416595pt}}
\pgflineto{\pgfpoint{223.558273pt}{274.342712pt}}
\pgfusepath{stroke}
\pgfpathmoveto{\pgfpoint{223.545685pt}{280.319550pt}}
\pgflineto{\pgfpoint{223.427673pt}{280.326172pt}}
\pgfusepath{stroke}
\pgfpathmoveto{\pgfpoint{223.455261pt}{280.395660pt}}
\pgflineto{\pgfpoint{223.545685pt}{280.319550pt}}
\pgfusepath{stroke}
\pgfpathmoveto{\pgfpoint{223.533112pt}{286.297913pt}}
\pgflineto{\pgfpoint{223.418259pt}{286.309082pt}}
\pgfusepath{stroke}
\pgfpathmoveto{\pgfpoint{223.447922pt}{286.375763pt}}
\pgflineto{\pgfpoint{223.533112pt}{286.297913pt}}
\pgfusepath{stroke}
\pgfpathmoveto{\pgfpoint{223.520691pt}{292.277649pt}}
\pgflineto{\pgfpoint{223.409088pt}{292.292847pt}}
\pgfusepath{stroke}
\pgfpathmoveto{\pgfpoint{223.440552pt}{292.356781pt}}
\pgflineto{\pgfpoint{223.520691pt}{292.277649pt}}
\pgfusepath{stroke}
\pgfpathmoveto{\pgfpoint{223.508545pt}{298.258545pt}}
\pgflineto{\pgfpoint{223.400238pt}{298.277405pt}}
\pgfusepath{stroke}
\pgfpathmoveto{\pgfpoint{223.433197pt}{298.338623pt}}
\pgflineto{\pgfpoint{223.508545pt}{298.258545pt}}
\pgfusepath{stroke}
\pgfpathmoveto{\pgfpoint{223.496674pt}{304.240509pt}}
\pgflineto{\pgfpoint{223.391663pt}{304.262604pt}}
\pgfusepath{stroke}
\pgfpathmoveto{\pgfpoint{223.425934pt}{304.321198pt}}
\pgflineto{\pgfpoint{223.496674pt}{304.240509pt}}
\pgfusepath{stroke}
\pgfpathmoveto{\pgfpoint{223.485138pt}{310.223389pt}}
\pgflineto{\pgfpoint{223.383438pt}{310.248413pt}}
\pgfusepath{stroke}
\pgfpathmoveto{\pgfpoint{223.418793pt}{310.304413pt}}
\pgflineto{\pgfpoint{223.485138pt}{310.223389pt}}
\pgfusepath{stroke}
\pgfpathmoveto{\pgfpoint{223.473938pt}{316.207092pt}}
\pgflineto{\pgfpoint{223.375519pt}{316.234680pt}}
\pgfusepath{stroke}
\pgfpathmoveto{\pgfpoint{223.411774pt}{316.288208pt}}
\pgflineto{\pgfpoint{223.473938pt}{316.207092pt}}
\pgfusepath{stroke}
\pgfpathmoveto{\pgfpoint{223.463074pt}{322.191528pt}}
\pgflineto{\pgfpoint{223.367920pt}{322.221436pt}}
\pgfusepath{stroke}
\pgfpathmoveto{\pgfpoint{223.404907pt}{322.272552pt}}
\pgflineto{\pgfpoint{223.463074pt}{322.191528pt}}
\pgfusepath{stroke}
\pgfpathmoveto{\pgfpoint{223.452576pt}{328.176636pt}}
\pgflineto{\pgfpoint{223.360626pt}{328.208618pt}}
\pgfusepath{stroke}
\pgfpathmoveto{\pgfpoint{223.398193pt}{328.257385pt}}
\pgflineto{\pgfpoint{223.452576pt}{328.176636pt}}
\pgfusepath{stroke}
\pgfpathmoveto{\pgfpoint{223.442444pt}{334.162354pt}}
\pgflineto{\pgfpoint{223.353638pt}{334.196106pt}}
\pgfusepath{stroke}
\pgfpathmoveto{\pgfpoint{223.391663pt}{334.242676pt}}
\pgflineto{\pgfpoint{223.442444pt}{334.162354pt}}
\pgfusepath{stroke}
\pgfpathmoveto{\pgfpoint{223.432678pt}{340.148621pt}}
\pgflineto{\pgfpoint{223.346954pt}{340.183990pt}}
\pgfusepath{stroke}
\pgfpathmoveto{\pgfpoint{223.385315pt}{340.228333pt}}
\pgflineto{\pgfpoint{223.432678pt}{340.148621pt}}
\pgfusepath{stroke}
\pgfpathmoveto{\pgfpoint{223.423264pt}{346.135376pt}}
\pgflineto{\pgfpoint{223.340576pt}{346.172119pt}}
\pgfusepath{stroke}
\pgfpathmoveto{\pgfpoint{223.379150pt}{346.214386pt}}
\pgflineto{\pgfpoint{223.423264pt}{346.135376pt}}
\pgfusepath{stroke}
\pgfpathmoveto{\pgfpoint{223.414246pt}{352.122620pt}}
\pgflineto{\pgfpoint{223.334503pt}{352.160522pt}}
\pgfusepath{stroke}
\pgfpathmoveto{\pgfpoint{223.373199pt}{352.200806pt}}
\pgflineto{\pgfpoint{223.414246pt}{352.122620pt}}
\pgfusepath{stroke}
\pgfpathmoveto{\pgfpoint{223.405579pt}{358.110260pt}}
\pgflineto{\pgfpoint{223.328705pt}{358.149170pt}}
\pgfusepath{stroke}
\pgfpathmoveto{\pgfpoint{223.367432pt}{358.187500pt}}
\pgflineto{\pgfpoint{223.405579pt}{358.110260pt}}
\pgfusepath{stroke}
\pgfpathmoveto{\pgfpoint{223.397308pt}{364.098267pt}}
\pgflineto{\pgfpoint{223.323212pt}{364.138000pt}}
\pgfusepath{stroke}
\pgfpathmoveto{\pgfpoint{223.361877pt}{364.174500pt}}
\pgflineto{\pgfpoint{223.397308pt}{364.098267pt}}
\pgfusepath{stroke}
\pgfpathmoveto{\pgfpoint{223.389404pt}{370.086609pt}}
\pgflineto{\pgfpoint{223.318024pt}{370.127045pt}}
\pgfusepath{stroke}
\pgfpathmoveto{\pgfpoint{223.356567pt}{370.161774pt}}
\pgflineto{\pgfpoint{223.389404pt}{370.086609pt}}
\pgfusepath{stroke}
\pgfpathmoveto{\pgfpoint{229.314941pt}{77.038269pt}}
\pgflineto{\pgfpoint{229.303360pt}{76.968384pt}}
\pgfusepath{stroke}
\pgfpathmoveto{\pgfpoint{229.263748pt}{76.989304pt}}
\pgflineto{\pgfpoint{229.314941pt}{77.038269pt}}
\pgfusepath{stroke}
\pgfpathmoveto{\pgfpoint{229.320374pt}{83.029816pt}}
\pgflineto{\pgfpoint{229.307419pt}{82.958054pt}}
\pgfusepath{stroke}
\pgfpathmoveto{\pgfpoint{229.266968pt}{82.980194pt}}
\pgflineto{\pgfpoint{229.320374pt}{83.029816pt}}
\pgfusepath{stroke}
\pgfpathmoveto{\pgfpoint{229.326294pt}{89.021393pt}}
\pgflineto{\pgfpoint{229.311798pt}{88.947708pt}}
\pgfusepath{stroke}
\pgfpathmoveto{\pgfpoint{229.270477pt}{88.971153pt}}
\pgflineto{\pgfpoint{229.326294pt}{89.021393pt}}
\pgfusepath{stroke}
\pgfpathmoveto{\pgfpoint{229.332718pt}{95.013039pt}}
\pgflineto{\pgfpoint{229.316498pt}{94.937340pt}}
\pgfusepath{stroke}
\pgfpathmoveto{\pgfpoint{229.274338pt}{94.962212pt}}
\pgflineto{\pgfpoint{229.332718pt}{95.013039pt}}
\pgfusepath{stroke}
\pgfpathmoveto{\pgfpoint{229.339722pt}{101.004700pt}}
\pgflineto{\pgfpoint{229.321594pt}{100.926926pt}}
\pgfusepath{stroke}
\pgfpathmoveto{\pgfpoint{229.278564pt}{100.953331pt}}
\pgflineto{\pgfpoint{229.339722pt}{101.004700pt}}
\pgfusepath{stroke}
\pgfpathmoveto{\pgfpoint{229.347351pt}{106.996323pt}}
\pgflineto{\pgfpoint{229.327118pt}{106.916412pt}}
\pgfusepath{stroke}
\pgfpathmoveto{\pgfpoint{229.283218pt}{106.944527pt}}
\pgflineto{\pgfpoint{229.347351pt}{106.996323pt}}
\pgfusepath{stroke}
\pgfpathmoveto{\pgfpoint{229.355652pt}{112.987907pt}}
\pgflineto{\pgfpoint{229.333084pt}{112.905792pt}}
\pgfusepath{stroke}
\pgfpathmoveto{\pgfpoint{229.288330pt}{112.935760pt}}
\pgflineto{\pgfpoint{229.355652pt}{112.987907pt}}
\pgfusepath{stroke}
\pgfpathmoveto{\pgfpoint{229.364746pt}{118.979355pt}}
\pgflineto{\pgfpoint{229.339554pt}{118.895004pt}}
\pgfusepath{stroke}
\pgfpathmoveto{\pgfpoint{229.293976pt}{118.926994pt}}
\pgflineto{\pgfpoint{229.364746pt}{118.979355pt}}
\pgfusepath{stroke}
\pgfpathmoveto{\pgfpoint{229.374680pt}{124.970634pt}}
\pgflineto{\pgfpoint{229.346558pt}{124.884003pt}}
\pgfusepath{stroke}
\pgfpathmoveto{\pgfpoint{229.300217pt}{124.918198pt}}
\pgflineto{\pgfpoint{229.374680pt}{124.970634pt}}
\pgfusepath{stroke}
\pgfpathmoveto{\pgfpoint{229.385544pt}{130.961624pt}}
\pgflineto{\pgfpoint{229.354172pt}{130.872711pt}}
\pgfusepath{stroke}
\pgfpathmoveto{\pgfpoint{229.307098pt}{130.909317pt}}
\pgflineto{\pgfpoint{229.385544pt}{130.961624pt}}
\pgfusepath{stroke}
\pgfpathmoveto{\pgfpoint{229.397415pt}{136.952209pt}}
\pgflineto{\pgfpoint{229.362396pt}{136.861038pt}}
\pgfusepath{stroke}
\pgfpathmoveto{\pgfpoint{229.314697pt}{136.900269pt}}
\pgflineto{\pgfpoint{229.397415pt}{136.952209pt}}
\pgfusepath{stroke}
\pgfpathmoveto{\pgfpoint{229.410385pt}{142.942245pt}}
\pgflineto{\pgfpoint{229.371307pt}{142.848877pt}}
\pgfusepath{stroke}
\pgfpathmoveto{\pgfpoint{229.323090pt}{142.890991pt}}
\pgflineto{\pgfpoint{229.410385pt}{142.942245pt}}
\pgfusepath{stroke}
\pgfpathmoveto{\pgfpoint{229.424530pt}{148.931549pt}}
\pgflineto{\pgfpoint{229.380890pt}{148.836090pt}}
\pgfusepath{stroke}
\pgfpathmoveto{\pgfpoint{229.332336pt}{148.881348pt}}
\pgflineto{\pgfpoint{229.424530pt}{148.931549pt}}
\pgfusepath{stroke}
\pgfpathmoveto{\pgfpoint{229.439880pt}{154.919891pt}}
\pgflineto{\pgfpoint{229.391190pt}{154.822510pt}}
\pgfusepath{stroke}
\pgfpathmoveto{\pgfpoint{229.342514pt}{154.871201pt}}
\pgflineto{\pgfpoint{229.439880pt}{154.919891pt}}
\pgfusepath{stroke}
\pgfpathmoveto{\pgfpoint{229.456467pt}{160.906952pt}}
\pgflineto{\pgfpoint{229.402191pt}{160.807968pt}}
\pgfusepath{stroke}
\pgfpathmoveto{\pgfpoint{229.353653pt}{160.860352pt}}
\pgflineto{\pgfpoint{229.456467pt}{160.906952pt}}
\pgfusepath{stroke}
\pgfpathmoveto{\pgfpoint{229.474274pt}{166.892456pt}}
\pgflineto{\pgfpoint{229.413818pt}{166.792252pt}}
\pgfusepath{stroke}
\pgfpathmoveto{\pgfpoint{229.365768pt}{166.848572pt}}
\pgflineto{\pgfpoint{229.474274pt}{166.892456pt}}
\pgfusepath{stroke}
\pgfpathmoveto{\pgfpoint{229.493195pt}{172.876007pt}}
\pgflineto{\pgfpoint{229.425949pt}{172.775101pt}}
\pgfusepath{stroke}
\pgfpathmoveto{\pgfpoint{229.378845pt}{172.835632pt}}
\pgflineto{\pgfpoint{229.493195pt}{172.876007pt}}
\pgfusepath{stroke}
\pgfpathmoveto{\pgfpoint{229.512985pt}{178.857193pt}}
\pgflineto{\pgfpoint{229.438416pt}{178.756271pt}}
\pgfusepath{stroke}
\pgfpathmoveto{\pgfpoint{229.392776pt}{178.821198pt}}
\pgflineto{\pgfpoint{229.512985pt}{178.857193pt}}
\pgfusepath{stroke}
\pgfpathmoveto{\pgfpoint{229.533325pt}{184.835602pt}}
\pgflineto{\pgfpoint{229.450928pt}{184.735535pt}}
\pgfusepath{stroke}
\pgfpathmoveto{\pgfpoint{229.407379pt}{184.804993pt}}
\pgflineto{\pgfpoint{229.533325pt}{184.835602pt}}
\pgfusepath{stroke}
\pgfpathmoveto{\pgfpoint{229.553696pt}{190.810883pt}}
\pgflineto{\pgfpoint{229.463135pt}{190.712708pt}}
\pgfusepath{stroke}
\pgfpathmoveto{\pgfpoint{229.422348pt}{190.786682pt}}
\pgflineto{\pgfpoint{229.553696pt}{190.810883pt}}
\pgfusepath{stroke}
\pgfpathmoveto{\pgfpoint{229.573441pt}{196.782730pt}}
\pgflineto{\pgfpoint{229.474518pt}{196.687668pt}}
\pgfusepath{stroke}
\pgfpathmoveto{\pgfpoint{229.437271pt}{196.766037pt}}
\pgflineto{\pgfpoint{229.573441pt}{196.782730pt}}
\pgfusepath{stroke}
\pgfpathmoveto{\pgfpoint{229.591736pt}{202.751114pt}}
\pgflineto{\pgfpoint{229.484558pt}{202.660477pt}}
\pgfusepath{stroke}
\pgfpathmoveto{\pgfpoint{229.451614pt}{202.742905pt}}
\pgflineto{\pgfpoint{229.591736pt}{202.751114pt}}
\pgfusepath{stroke}
\pgfpathmoveto{\pgfpoint{229.607697pt}{208.716187pt}}
\pgflineto{\pgfpoint{229.492691pt}{208.631348pt}}
\pgfusepath{stroke}
\pgfpathmoveto{\pgfpoint{229.464783pt}{208.717316pt}}
\pgflineto{\pgfpoint{229.607697pt}{208.716187pt}}
\pgfusepath{stroke}
\pgfpathmoveto{\pgfpoint{229.620453pt}{214.678452pt}}
\pgflineto{\pgfpoint{229.498352pt}{214.600677pt}}
\pgfusepath{stroke}
\pgfpathmoveto{\pgfpoint{229.476120pt}{214.689484pt}}
\pgflineto{\pgfpoint{229.620453pt}{214.678452pt}}
\pgfusepath{stroke}
\pgfpathmoveto{\pgfpoint{229.629303pt}{220.638672pt}}
\pgflineto{\pgfpoint{229.501190pt}{220.569092pt}}
\pgfusepath{stroke}
\pgfpathmoveto{\pgfpoint{229.485077pt}{220.659866pt}}
\pgflineto{\pgfpoint{229.629303pt}{220.638672pt}}
\pgfusepath{stroke}
\pgfpathmoveto{\pgfpoint{229.633820pt}{226.597855pt}}
\pgflineto{\pgfpoint{229.501038pt}{226.537308pt}}
\pgfusepath{stroke}
\pgfpathmoveto{\pgfpoint{229.491257pt}{226.629089pt}}
\pgflineto{\pgfpoint{229.633820pt}{226.597855pt}}
\pgfusepath{stroke}
\pgfpathmoveto{\pgfpoint{229.633942pt}{232.557098pt}}
\pgflineto{\pgfpoint{229.497955pt}{232.506042pt}}
\pgfusepath{stroke}
\pgfpathmoveto{\pgfpoint{229.494507pt}{232.597855pt}}
\pgflineto{\pgfpoint{229.633942pt}{232.557098pt}}
\pgfusepath{stroke}
\pgfpathmoveto{\pgfpoint{229.629959pt}{238.517426pt}}
\pgflineto{\pgfpoint{229.492249pt}{238.475952pt}}
\pgfusepath{stroke}
\pgfpathmoveto{\pgfpoint{229.494904pt}{238.566879pt}}
\pgflineto{\pgfpoint{229.629959pt}{238.517426pt}}
\pgfusepath{stroke}
\pgfpathmoveto{\pgfpoint{229.622467pt}{244.479660pt}}
\pgflineto{\pgfpoint{229.484421pt}{244.447525pt}}
\pgfusepath{stroke}
\pgfpathmoveto{\pgfpoint{229.492752pt}{244.536789pt}}
\pgflineto{\pgfpoint{229.622467pt}{244.479660pt}}
\pgfusepath{stroke}
\pgfpathmoveto{\pgfpoint{229.612213pt}{250.444336pt}}
\pgflineto{\pgfpoint{229.475021pt}{250.421036pt}}
\pgfusepath{stroke}
\pgfpathmoveto{\pgfpoint{229.488464pt}{250.508011pt}}
\pgflineto{\pgfpoint{229.612213pt}{250.444336pt}}
\pgfusepath{stroke}
\pgfpathmoveto{\pgfpoint{229.599976pt}{256.411682pt}}
\pgflineto{\pgfpoint{229.464584pt}{256.396545pt}}
\pgfusepath{stroke}
\pgfpathmoveto{\pgfpoint{229.482574pt}{256.480804pt}}
\pgflineto{\pgfpoint{229.599976pt}{256.411682pt}}
\pgfusepath{stroke}
\pgfpathmoveto{\pgfpoint{229.586456pt}{262.381775pt}}
\pgflineto{\pgfpoint{229.453613pt}{262.374023pt}}
\pgfusepath{stroke}
\pgfpathmoveto{\pgfpoint{229.475525pt}{262.455261pt}}
\pgflineto{\pgfpoint{229.586456pt}{262.381775pt}}
\pgfusepath{stroke}
\pgfpathmoveto{\pgfpoint{229.572266pt}{268.354370pt}}
\pgflineto{\pgfpoint{229.442474pt}{268.353241pt}}
\pgfusepath{stroke}
\pgfpathmoveto{\pgfpoint{229.467758pt}{268.431335pt}}
\pgflineto{\pgfpoint{229.572266pt}{268.354370pt}}
\pgfusepath{stroke}
\pgfpathmoveto{\pgfpoint{229.557861pt}{274.329254pt}}
\pgflineto{\pgfpoint{229.431427pt}{274.334015pt}}
\pgfusepath{stroke}
\pgfpathmoveto{\pgfpoint{229.459579pt}{274.408936pt}}
\pgflineto{\pgfpoint{229.557861pt}{274.329254pt}}
\pgfusepath{stroke}
\pgfpathmoveto{\pgfpoint{229.543503pt}{280.306122pt}}
\pgflineto{\pgfpoint{229.420654pt}{280.316101pt}}
\pgfusepath{stroke}
\pgfpathmoveto{\pgfpoint{229.451233pt}{280.387817pt}}
\pgflineto{\pgfpoint{229.543503pt}{280.306122pt}}
\pgfusepath{stroke}
\pgfpathmoveto{\pgfpoint{229.529419pt}{286.284698pt}}
\pgflineto{\pgfpoint{229.410248pt}{286.299316pt}}
\pgfusepath{stroke}
\pgfpathmoveto{\pgfpoint{229.442856pt}{286.367920pt}}
\pgflineto{\pgfpoint{229.529419pt}{286.284698pt}}
\pgfusepath{stroke}
\pgfpathmoveto{\pgfpoint{229.515717pt}{292.264740pt}}
\pgflineto{\pgfpoint{229.400253pt}{292.283447pt}}
\pgfusepath{stroke}
\pgfpathmoveto{\pgfpoint{229.434570pt}{292.348999pt}}
\pgflineto{\pgfpoint{229.515717pt}{292.264740pt}}
\pgfusepath{stroke}
\pgfpathmoveto{\pgfpoint{229.502426pt}{298.246002pt}}
\pgflineto{\pgfpoint{229.390671pt}{298.268372pt}}
\pgfusepath{stroke}
\pgfpathmoveto{\pgfpoint{229.426437pt}{298.330933pt}}
\pgflineto{\pgfpoint{229.502426pt}{298.246002pt}}
\pgfusepath{stroke}
\pgfpathmoveto{\pgfpoint{229.489563pt}{304.228333pt}}
\pgflineto{\pgfpoint{229.381500pt}{304.253906pt}}
\pgfusepath{stroke}
\pgfpathmoveto{\pgfpoint{229.418472pt}{304.313660pt}}
\pgflineto{\pgfpoint{229.489563pt}{304.228333pt}}
\pgfusepath{stroke}
\pgfpathmoveto{\pgfpoint{229.477158pt}{310.211609pt}}
\pgflineto{\pgfpoint{229.372726pt}{310.240051pt}}
\pgfusepath{stroke}
\pgfpathmoveto{\pgfpoint{229.410690pt}{310.297028pt}}
\pgflineto{\pgfpoint{229.477158pt}{310.211609pt}}
\pgfusepath{stroke}
\pgfpathmoveto{\pgfpoint{229.465179pt}{316.195679pt}}
\pgflineto{\pgfpoint{229.364319pt}{316.226715pt}}
\pgfusepath{stroke}
\pgfpathmoveto{\pgfpoint{229.403107pt}{316.281006pt}}
\pgflineto{\pgfpoint{229.465179pt}{316.195679pt}}
\pgfusepath{stroke}
\pgfpathmoveto{\pgfpoint{229.453613pt}{322.180511pt}}
\pgflineto{\pgfpoint{229.356293pt}{322.213806pt}}
\pgfusepath{stroke}
\pgfpathmoveto{\pgfpoint{229.395721pt}{322.265533pt}}
\pgflineto{\pgfpoint{229.453613pt}{322.180511pt}}
\pgfusepath{stroke}
\pgfpathmoveto{\pgfpoint{229.442474pt}{328.166016pt}}
\pgflineto{\pgfpoint{229.348618pt}{328.201294pt}}
\pgfusepath{stroke}
\pgfpathmoveto{\pgfpoint{229.388550pt}{328.250549pt}}
\pgflineto{\pgfpoint{229.442474pt}{328.166016pt}}
\pgfusepath{stroke}
\pgfpathmoveto{\pgfpoint{229.431763pt}{334.152100pt}}
\pgflineto{\pgfpoint{229.341309pt}{334.189117pt}}
\pgfusepath{stroke}
\pgfpathmoveto{\pgfpoint{229.381592pt}{334.236023pt}}
\pgflineto{\pgfpoint{229.431763pt}{334.152100pt}}
\pgfusepath{stroke}
\pgfpathmoveto{\pgfpoint{229.421463pt}{340.138794pt}}
\pgflineto{\pgfpoint{229.334305pt}{340.177307pt}}
\pgfusepath{stroke}
\pgfpathmoveto{\pgfpoint{229.374847pt}{340.221893pt}}
\pgflineto{\pgfpoint{229.421463pt}{340.138794pt}}
\pgfusepath{stroke}
\pgfpathmoveto{\pgfpoint{229.411575pt}{346.125946pt}}
\pgflineto{\pgfpoint{229.327652pt}{346.165771pt}}
\pgfusepath{stroke}
\pgfpathmoveto{\pgfpoint{229.368317pt}{346.208160pt}}
\pgflineto{\pgfpoint{229.411575pt}{346.125946pt}}
\pgfusepath{stroke}
\pgfpathmoveto{\pgfpoint{229.402115pt}{352.113586pt}}
\pgflineto{\pgfpoint{229.321320pt}{352.154480pt}}
\pgfusepath{stroke}
\pgfpathmoveto{\pgfpoint{229.362030pt}{352.194763pt}}
\pgflineto{\pgfpoint{229.402115pt}{352.113586pt}}
\pgfusepath{stroke}
\pgfpathmoveto{\pgfpoint{229.393066pt}{358.101624pt}}
\pgflineto{\pgfpoint{229.315338pt}{358.143433pt}}
\pgfusepath{stroke}
\pgfpathmoveto{\pgfpoint{229.355972pt}{358.181702pt}}
\pgflineto{\pgfpoint{229.393066pt}{358.101624pt}}
\pgfusepath{stroke}
\pgfpathmoveto{\pgfpoint{229.384445pt}{364.090027pt}}
\pgflineto{\pgfpoint{229.309647pt}{364.132568pt}}
\pgfusepath{stroke}
\pgfpathmoveto{\pgfpoint{229.350143pt}{364.168945pt}}
\pgflineto{\pgfpoint{229.384445pt}{364.090027pt}}
\pgfusepath{stroke}
\pgfpathmoveto{\pgfpoint{229.376236pt}{370.078735pt}}
\pgflineto{\pgfpoint{229.304291pt}{370.121887pt}}
\pgfusepath{stroke}
\pgfpathmoveto{\pgfpoint{229.344574pt}{370.156433pt}}
\pgflineto{\pgfpoint{229.376236pt}{370.078735pt}}
\pgfusepath{stroke}
\pgfpathmoveto{\pgfpoint{235.298737pt}{77.043472pt}}
\pgflineto{\pgfpoint{235.289062pt}{76.972290pt}}
\pgfusepath{stroke}
\pgfpathmoveto{\pgfpoint{235.248291pt}{76.992325pt}}
\pgflineto{\pgfpoint{235.298737pt}{77.043472pt}}
\pgfusepath{stroke}
\pgfpathmoveto{\pgfpoint{235.304092pt}{83.035446pt}}
\pgflineto{\pgfpoint{235.293091pt}{82.962265pt}}
\pgfusepath{stroke}
\pgfpathmoveto{\pgfpoint{235.251404pt}{82.983490pt}}
\pgflineto{\pgfpoint{235.304092pt}{83.035446pt}}
\pgfusepath{stroke}
\pgfpathmoveto{\pgfpoint{235.309937pt}{89.027534pt}}
\pgflineto{\pgfpoint{235.297485pt}{88.952255pt}}
\pgfusepath{stroke}
\pgfpathmoveto{\pgfpoint{235.254807pt}{88.974785pt}}
\pgflineto{\pgfpoint{235.309937pt}{89.027534pt}}
\pgfusepath{stroke}
\pgfpathmoveto{\pgfpoint{235.316299pt}{95.019722pt}}
\pgflineto{\pgfpoint{235.302216pt}{94.942253pt}}
\pgfusepath{stroke}
\pgfpathmoveto{\pgfpoint{235.258560pt}{94.966202pt}}
\pgflineto{\pgfpoint{235.316299pt}{95.019722pt}}
\pgfusepath{stroke}
\pgfpathmoveto{\pgfpoint{235.323257pt}{101.011986pt}}
\pgflineto{\pgfpoint{235.307358pt}{100.932251pt}}
\pgfusepath{stroke}
\pgfpathmoveto{\pgfpoint{235.262695pt}{100.957748pt}}
\pgflineto{\pgfpoint{235.323257pt}{101.011986pt}}
\pgfusepath{stroke}
\pgfpathmoveto{\pgfpoint{235.330887pt}{107.004288pt}}
\pgflineto{\pgfpoint{235.312943pt}{106.922188pt}}
\pgfusepath{stroke}
\pgfpathmoveto{\pgfpoint{235.267273pt}{106.949379pt}}
\pgflineto{\pgfpoint{235.330887pt}{107.004288pt}}
\pgfusepath{stroke}
\pgfpathmoveto{\pgfpoint{235.339249pt}{112.996635pt}}
\pgflineto{\pgfpoint{235.319031pt}{112.912071pt}}
\pgfusepath{stroke}
\pgfpathmoveto{\pgfpoint{235.272339pt}{112.941116pt}}
\pgflineto{\pgfpoint{235.339249pt}{112.996635pt}}
\pgfusepath{stroke}
\pgfpathmoveto{\pgfpoint{235.348434pt}{118.988937pt}}
\pgflineto{\pgfpoint{235.325668pt}{118.901840pt}}
\pgfusepath{stroke}
\pgfpathmoveto{\pgfpoint{235.277954pt}{118.932930pt}}
\pgflineto{\pgfpoint{235.348434pt}{118.988937pt}}
\pgfusepath{stroke}
\pgfpathmoveto{\pgfpoint{235.358566pt}{124.981140pt}}
\pgflineto{\pgfpoint{235.332901pt}{124.891441pt}}
\pgfusepath{stroke}
\pgfpathmoveto{\pgfpoint{235.284210pt}{124.924774pt}}
\pgflineto{\pgfpoint{235.358566pt}{124.981140pt}}
\pgfusepath{stroke}
\pgfpathmoveto{\pgfpoint{235.369720pt}{130.973160pt}}
\pgflineto{\pgfpoint{235.340820pt}{130.880798pt}}
\pgfusepath{stroke}
\pgfpathmoveto{\pgfpoint{235.291168pt}{130.916611pt}}
\pgflineto{\pgfpoint{235.369720pt}{130.973160pt}}
\pgfusepath{stroke}
\pgfpathmoveto{\pgfpoint{235.382019pt}{136.964874pt}}
\pgflineto{\pgfpoint{235.349457pt}{136.869827pt}}
\pgfusepath{stroke}
\pgfpathmoveto{\pgfpoint{235.298935pt}{136.908371pt}}
\pgflineto{\pgfpoint{235.382019pt}{136.964874pt}}
\pgfusepath{stroke}
\pgfpathmoveto{\pgfpoint{235.395569pt}{142.956146pt}}
\pgflineto{\pgfpoint{235.358887pt}{142.858429pt}}
\pgfusepath{stroke}
\pgfpathmoveto{\pgfpoint{235.307587pt}{142.899979pt}}
\pgflineto{\pgfpoint{235.395569pt}{142.956146pt}}
\pgfusepath{stroke}
\pgfpathmoveto{\pgfpoint{235.410522pt}{148.946762pt}}
\pgflineto{\pgfpoint{235.369171pt}{148.846420pt}}
\pgfusepath{stroke}
\pgfpathmoveto{\pgfpoint{235.317230pt}{148.891296pt}}
\pgflineto{\pgfpoint{235.410522pt}{148.946762pt}}
\pgfusepath{stroke}
\pgfpathmoveto{\pgfpoint{235.426956pt}{154.936462pt}}
\pgflineto{\pgfpoint{235.380341pt}{154.833649pt}}
\pgfusepath{stroke}
\pgfpathmoveto{\pgfpoint{235.327972pt}{154.882187pt}}
\pgflineto{\pgfpoint{235.426956pt}{154.936462pt}}
\pgfusepath{stroke}
\pgfpathmoveto{\pgfpoint{235.444977pt}{160.924957pt}}
\pgflineto{\pgfpoint{235.392426pt}{160.819901pt}}
\pgfusepath{stroke}
\pgfpathmoveto{\pgfpoint{235.339905pt}{160.872437pt}}
\pgflineto{\pgfpoint{235.444977pt}{160.924957pt}}
\pgfusepath{stroke}
\pgfpathmoveto{\pgfpoint{235.464569pt}{166.911835pt}}
\pgflineto{\pgfpoint{235.405411pt}{166.804886pt}}
\pgfusepath{stroke}
\pgfpathmoveto{\pgfpoint{235.353073pt}{166.861786pt}}
\pgflineto{\pgfpoint{235.464569pt}{166.911835pt}}
\pgfusepath{stroke}
\pgfpathmoveto{\pgfpoint{235.485733pt}{172.896637pt}}
\pgflineto{\pgfpoint{235.419189pt}{172.788315pt}}
\pgfusepath{stroke}
\pgfpathmoveto{\pgfpoint{235.367508pt}{172.849915pt}}
\pgflineto{\pgfpoint{235.485733pt}{172.896637pt}}
\pgfusepath{stroke}
\pgfpathmoveto{\pgfpoint{235.508286pt}{178.878784pt}}
\pgflineto{\pgfpoint{235.433594pt}{178.769836pt}}
\pgfusepath{stroke}
\pgfpathmoveto{\pgfpoint{235.383148pt}{178.836441pt}}
\pgflineto{\pgfpoint{235.508286pt}{178.878784pt}}
\pgfusepath{stroke}
\pgfpathmoveto{\pgfpoint{235.531860pt}{184.857742pt}}
\pgflineto{\pgfpoint{235.448334pt}{184.749100pt}}
\pgfusepath{stroke}
\pgfpathmoveto{\pgfpoint{235.399841pt}{184.820938pt}}
\pgflineto{\pgfpoint{235.531860pt}{184.857742pt}}
\pgfusepath{stroke}
\pgfpathmoveto{\pgfpoint{235.555878pt}{190.832901pt}}
\pgflineto{\pgfpoint{235.462952pt}{190.725784pt}}
\pgfusepath{stroke}
\pgfpathmoveto{\pgfpoint{235.417267pt}{190.802979pt}}
\pgflineto{\pgfpoint{235.555878pt}{190.832901pt}}
\pgfusepath{stroke}
\pgfpathmoveto{\pgfpoint{235.579559pt}{196.803802pt}}
\pgflineto{\pgfpoint{235.476868pt}{196.699677pt}}
\pgfusepath{stroke}
\pgfpathmoveto{\pgfpoint{235.434921pt}{196.782120pt}}
\pgflineto{\pgfpoint{235.579559pt}{196.803802pt}}
\pgfusepath{stroke}
\pgfpathmoveto{\pgfpoint{235.601807pt}{202.770218pt}}
\pgflineto{\pgfpoint{235.489319pt}{202.670746pt}}
\pgfusepath{stroke}
\pgfpathmoveto{\pgfpoint{235.452118pt}{202.758118pt}}
\pgflineto{\pgfpoint{235.601807pt}{202.770218pt}}
\pgfusepath{stroke}
\pgfpathmoveto{\pgfpoint{235.621368pt}{208.732254pt}}
\pgflineto{\pgfpoint{235.499512pt}{208.639175pt}}
\pgfusepath{stroke}
\pgfpathmoveto{\pgfpoint{235.468033pt}{208.730911pt}}
\pgflineto{\pgfpoint{235.621368pt}{208.732254pt}}
\pgfusepath{stroke}
\pgfpathmoveto{\pgfpoint{235.637009pt}{214.690491pt}}
\pgflineto{\pgfpoint{235.506653pt}{214.605499pt}}
\pgfusepath{stroke}
\pgfpathmoveto{\pgfpoint{235.481735pt}{214.700714pt}}
\pgflineto{\pgfpoint{235.637009pt}{214.690491pt}}
\pgfusepath{stroke}
\pgfpathmoveto{\pgfpoint{235.647675pt}{220.645966pt}}
\pgflineto{\pgfpoint{235.510162pt}{220.570557pt}}
\pgfusepath{stroke}
\pgfpathmoveto{\pgfpoint{235.492432pt}{220.668137pt}}
\pgflineto{\pgfpoint{235.647675pt}{220.645966pt}}
\pgfusepath{stroke}
\pgfpathmoveto{\pgfpoint{235.652710pt}{226.600098pt}}
\pgflineto{\pgfpoint{235.509796pt}{226.535339pt}}
\pgfusepath{stroke}
\pgfpathmoveto{\pgfpoint{235.499527pt}{226.634033pt}}
\pgflineto{\pgfpoint{235.652710pt}{226.600098pt}}
\pgfusepath{stroke}
\pgfpathmoveto{\pgfpoint{235.652054pt}{232.554459pt}}
\pgflineto{\pgfpoint{235.505646pt}{232.500900pt}}
\pgfusepath{stroke}
\pgfpathmoveto{\pgfpoint{235.502792pt}{232.599457pt}}
\pgflineto{\pgfpoint{235.652054pt}{232.554459pt}}
\pgfusepath{stroke}
\pgfpathmoveto{\pgfpoint{235.646194pt}{238.510498pt}}
\pgflineto{\pgfpoint{235.498230pt}{238.468170pt}}
\pgfusepath{stroke}
\pgfpathmoveto{\pgfpoint{235.502411pt}{238.565414pt}}
\pgflineto{\pgfpoint{235.646194pt}{238.510498pt}}
\pgfusepath{stroke}
\pgfpathmoveto{\pgfpoint{235.636047pt}{244.469269pt}}
\pgflineto{\pgfpoint{235.488235pt}{244.437729pt}}
\pgfusepath{stroke}
\pgfpathmoveto{\pgfpoint{235.498871pt}{244.532730pt}}
\pgflineto{\pgfpoint{235.636047pt}{244.469269pt}}
\pgfusepath{stroke}
\pgfpathmoveto{\pgfpoint{235.622726pt}{250.431458pt}}
\pgflineto{\pgfpoint{235.476517pt}{250.409912pt}}
\pgfusepath{stroke}
\pgfpathmoveto{\pgfpoint{235.492828pt}{250.501953pt}}
\pgflineto{\pgfpoint{235.622726pt}{250.431458pt}}
\pgfusepath{stroke}
\pgfpathmoveto{\pgfpoint{235.607361pt}{256.397247pt}}
\pgflineto{\pgfpoint{235.463821pt}{256.384735pt}}
\pgfusepath{stroke}
\pgfpathmoveto{\pgfpoint{235.485031pt}{256.473358pt}}
\pgflineto{\pgfpoint{235.607361pt}{256.397247pt}}
\pgfusepath{stroke}
\pgfpathmoveto{\pgfpoint{235.590881pt}{262.366455pt}}
\pgflineto{\pgfpoint{235.450790pt}{262.361969pt}}
\pgfusepath{stroke}
\pgfpathmoveto{\pgfpoint{235.476105pt}{262.446930pt}}
\pgflineto{\pgfpoint{235.590881pt}{262.366455pt}}
\pgfusepath{stroke}
\pgfpathmoveto{\pgfpoint{235.574051pt}{268.338806pt}}
\pgflineto{\pgfpoint{235.437836pt}{268.341309pt}}
\pgfusepath{stroke}
\pgfpathmoveto{\pgfpoint{235.466599pt}{268.422516pt}}
\pgflineto{\pgfpoint{235.574051pt}{268.338806pt}}
\pgfusepath{stroke}
\pgfpathmoveto{\pgfpoint{235.557343pt}{274.313782pt}}
\pgflineto{\pgfpoint{235.425293pt}{274.322418pt}}
\pgfusepath{stroke}
\pgfpathmoveto{\pgfpoint{235.456879pt}{274.399933pt}}
\pgflineto{\pgfpoint{235.557343pt}{274.313782pt}}
\pgfusepath{stroke}
\pgfpathmoveto{\pgfpoint{235.541077pt}{280.291016pt}}
\pgflineto{\pgfpoint{235.413269pt}{280.304962pt}}
\pgfusepath{stroke}
\pgfpathmoveto{\pgfpoint{235.447189pt}{280.378845pt}}
\pgflineto{\pgfpoint{235.541077pt}{280.291016pt}}
\pgfusepath{stroke}
\pgfpathmoveto{\pgfpoint{235.525375pt}{286.270050pt}}
\pgflineto{\pgfpoint{235.401825pt}{286.288635pt}}
\pgfusepath{stroke}
\pgfpathmoveto{\pgfpoint{235.437683pt}{286.359039pt}}
\pgflineto{\pgfpoint{235.525375pt}{286.270050pt}}
\pgfusepath{stroke}
\pgfpathmoveto{\pgfpoint{235.510300pt}{292.250610pt}}
\pgflineto{\pgfpoint{235.390991pt}{292.273254pt}}
\pgfusepath{stroke}
\pgfpathmoveto{\pgfpoint{235.428452pt}{292.340332pt}}
\pgflineto{\pgfpoint{235.510300pt}{292.250610pt}}
\pgfusepath{stroke}
\pgfpathmoveto{\pgfpoint{235.495865pt}{298.232391pt}}
\pgflineto{\pgfpoint{235.380707pt}{298.258667pt}}
\pgfusepath{stroke}
\pgfpathmoveto{\pgfpoint{235.419495pt}{298.322510pt}}
\pgflineto{\pgfpoint{235.495865pt}{298.232391pt}}
\pgfusepath{stroke}
\pgfpathmoveto{\pgfpoint{235.482025pt}{304.215240pt}}
\pgflineto{\pgfpoint{235.370926pt}{304.244690pt}}
\pgfusepath{stroke}
\pgfpathmoveto{\pgfpoint{235.410812pt}{304.305450pt}}
\pgflineto{\pgfpoint{235.482025pt}{304.215240pt}}
\pgfusepath{stroke}
\pgfpathmoveto{\pgfpoint{235.468750pt}{310.198975pt}}
\pgflineto{\pgfpoint{235.361633pt}{310.231232pt}}
\pgfusepath{stroke}
\pgfpathmoveto{\pgfpoint{235.402405pt}{310.289062pt}}
\pgflineto{\pgfpoint{235.468750pt}{310.198975pt}}
\pgfusepath{stroke}
\pgfpathmoveto{\pgfpoint{235.455994pt}{316.183533pt}}
\pgflineto{\pgfpoint{235.352783pt}{316.218262pt}}
\pgfusepath{stroke}
\pgfpathmoveto{\pgfpoint{235.394257pt}{316.273254pt}}
\pgflineto{\pgfpoint{235.455994pt}{316.183533pt}}
\pgfusepath{stroke}
\pgfpathmoveto{\pgfpoint{235.443741pt}{322.168823pt}}
\pgflineto{\pgfpoint{235.344330pt}{322.205750pt}}
\pgfusepath{stroke}
\pgfpathmoveto{\pgfpoint{235.386353pt}{322.257996pt}}
\pgflineto{\pgfpoint{235.443741pt}{322.168823pt}}
\pgfusepath{stroke}
\pgfpathmoveto{\pgfpoint{235.431961pt}{328.154785pt}}
\pgflineto{\pgfpoint{235.336273pt}{328.193604pt}}
\pgfusepath{stroke}
\pgfpathmoveto{\pgfpoint{235.378708pt}{328.243225pt}}
\pgflineto{\pgfpoint{235.431961pt}{328.154785pt}}
\pgfusepath{stroke}
\pgfpathmoveto{\pgfpoint{235.420654pt}{334.141327pt}}
\pgflineto{\pgfpoint{235.328613pt}{334.181824pt}}
\pgfusepath{stroke}
\pgfpathmoveto{\pgfpoint{235.371307pt}{334.228943pt}}
\pgflineto{\pgfpoint{235.420654pt}{334.141327pt}}
\pgfusepath{stroke}
\pgfpathmoveto{\pgfpoint{235.409821pt}{340.128418pt}}
\pgflineto{\pgfpoint{235.321320pt}{340.170349pt}}
\pgfusepath{stroke}
\pgfpathmoveto{\pgfpoint{235.364166pt}{340.215057pt}}
\pgflineto{\pgfpoint{235.409821pt}{340.128418pt}}
\pgfusepath{stroke}
\pgfpathmoveto{\pgfpoint{235.399460pt}{346.116028pt}}
\pgflineto{\pgfpoint{235.314392pt}{346.159149pt}}
\pgfusepath{stroke}
\pgfpathmoveto{\pgfpoint{235.357269pt}{346.201569pt}}
\pgflineto{\pgfpoint{235.399460pt}{346.116028pt}}
\pgfusepath{stroke}
\pgfpathmoveto{\pgfpoint{235.389557pt}{352.104126pt}}
\pgflineto{\pgfpoint{235.307831pt}{352.148193pt}}
\pgfusepath{stroke}
\pgfpathmoveto{\pgfpoint{235.350632pt}{352.188416pt}}
\pgflineto{\pgfpoint{235.389557pt}{352.104126pt}}
\pgfusepath{stroke}
\pgfpathmoveto{\pgfpoint{235.380127pt}{358.092590pt}}
\pgflineto{\pgfpoint{235.301620pt}{358.137512pt}}
\pgfusepath{stroke}
\pgfpathmoveto{\pgfpoint{235.344269pt}{358.175629pt}}
\pgflineto{\pgfpoint{235.380127pt}{358.092590pt}}
\pgfusepath{stroke}
\pgfpathmoveto{\pgfpoint{235.371170pt}{364.081451pt}}
\pgflineto{\pgfpoint{235.295761pt}{364.126953pt}}
\pgfusepath{stroke}
\pgfpathmoveto{\pgfpoint{235.338165pt}{364.163116pt}}
\pgflineto{\pgfpoint{235.371170pt}{364.081451pt}}
\pgfusepath{stroke}
\pgfpathmoveto{\pgfpoint{235.362656pt}{370.070618pt}}
\pgflineto{\pgfpoint{235.290253pt}{370.116608pt}}
\pgfusepath{stroke}
\pgfpathmoveto{\pgfpoint{235.332336pt}{370.150879pt}}
\pgflineto{\pgfpoint{235.362656pt}{370.070618pt}}
\pgfusepath{stroke}
\pgfpathmoveto{\pgfpoint{241.282104pt}{77.048538pt}}
\pgflineto{\pgfpoint{241.274460pt}{76.976166pt}}
\pgfusepath{stroke}
\pgfpathmoveto{\pgfpoint{241.232559pt}{76.995209pt}}
\pgflineto{\pgfpoint{241.282104pt}{77.048538pt}}
\pgfusepath{stroke}
\pgfpathmoveto{\pgfpoint{241.287323pt}{83.040970pt}}
\pgflineto{\pgfpoint{241.278458pt}{82.966461pt}}
\pgfusepath{stroke}
\pgfpathmoveto{\pgfpoint{241.235519pt}{82.986679pt}}
\pgflineto{\pgfpoint{241.287323pt}{83.040970pt}}
\pgfusepath{stroke}
\pgfpathmoveto{\pgfpoint{241.293030pt}{89.033569pt}}
\pgflineto{\pgfpoint{241.282806pt}{88.956802pt}}
\pgfusepath{stroke}
\pgfpathmoveto{\pgfpoint{241.238785pt}{88.978287pt}}
\pgflineto{\pgfpoint{241.293030pt}{89.033569pt}}
\pgfusepath{stroke}
\pgfpathmoveto{\pgfpoint{241.299286pt}{95.026329pt}}
\pgflineto{\pgfpoint{241.287521pt}{94.947189pt}}
\pgfusepath{stroke}
\pgfpathmoveto{\pgfpoint{241.242401pt}{94.970078pt}}
\pgflineto{\pgfpoint{241.299286pt}{95.026329pt}}
\pgfusepath{stroke}
\pgfpathmoveto{\pgfpoint{241.306137pt}{101.019234pt}}
\pgflineto{\pgfpoint{241.292664pt}{100.937607pt}}
\pgfusepath{stroke}
\pgfpathmoveto{\pgfpoint{241.246399pt}{100.962021pt}}
\pgflineto{\pgfpoint{241.306137pt}{101.019234pt}}
\pgfusepath{stroke}
\pgfpathmoveto{\pgfpoint{241.313690pt}{107.012260pt}}
\pgflineto{\pgfpoint{241.298279pt}{106.928047pt}}
\pgfusepath{stroke}
\pgfpathmoveto{\pgfpoint{241.250839pt}{106.954132pt}}
\pgflineto{\pgfpoint{241.313690pt}{107.012260pt}}
\pgfusepath{stroke}
\pgfpathmoveto{\pgfpoint{241.322006pt}{113.005402pt}}
\pgflineto{\pgfpoint{241.304428pt}{112.918472pt}}
\pgfusepath{stroke}
\pgfpathmoveto{\pgfpoint{241.255798pt}{112.946411pt}}
\pgflineto{\pgfpoint{241.322006pt}{113.005402pt}}
\pgfusepath{stroke}
\pgfpathmoveto{\pgfpoint{241.331207pt}{118.998604pt}}
\pgflineto{\pgfpoint{241.311172pt}{118.908844pt}}
\pgfusepath{stroke}
\pgfpathmoveto{\pgfpoint{241.261322pt}{118.938820pt}}
\pgflineto{\pgfpoint{241.331207pt}{118.998604pt}}
\pgfusepath{stroke}
\pgfpathmoveto{\pgfpoint{241.341415pt}{124.991829pt}}
\pgflineto{\pgfpoint{241.318573pt}{124.899109pt}}
\pgfusepath{stroke}
\pgfpathmoveto{\pgfpoint{241.267517pt}{124.931351pt}}
\pgflineto{\pgfpoint{241.341415pt}{124.991829pt}}
\pgfusepath{stroke}
\pgfpathmoveto{\pgfpoint{241.352737pt}{130.985001pt}}
\pgflineto{\pgfpoint{241.326721pt}{130.889221pt}}
\pgfusepath{stroke}
\pgfpathmoveto{\pgfpoint{241.274460pt}{130.923981pt}}
\pgflineto{\pgfpoint{241.352737pt}{130.985001pt}}
\pgfusepath{stroke}
\pgfpathmoveto{\pgfpoint{241.365341pt}{136.977997pt}}
\pgflineto{\pgfpoint{241.335709pt}{136.879059pt}}
\pgfusepath{stroke}
\pgfpathmoveto{\pgfpoint{241.282288pt}{136.916626pt}}
\pgflineto{\pgfpoint{241.365341pt}{136.977997pt}}
\pgfusepath{stroke}
\pgfpathmoveto{\pgfpoint{241.379379pt}{142.970688pt}}
\pgflineto{\pgfpoint{241.345627pt}{142.868546pt}}
\pgfusepath{stroke}
\pgfpathmoveto{\pgfpoint{241.291092pt}{142.909225pt}}
\pgflineto{\pgfpoint{241.379379pt}{142.970688pt}}
\pgfusepath{stroke}
\pgfpathmoveto{\pgfpoint{241.395020pt}{148.962860pt}}
\pgflineto{\pgfpoint{241.356552pt}{148.857513pt}}
\pgfusepath{stroke}
\pgfpathmoveto{\pgfpoint{241.301025pt}{148.901672pt}}
\pgflineto{\pgfpoint{241.395020pt}{148.962860pt}}
\pgfusepath{stroke}
\pgfpathmoveto{\pgfpoint{241.412460pt}{154.954285pt}}
\pgflineto{\pgfpoint{241.368576pt}{154.845779pt}}
\pgfusepath{stroke}
\pgfpathmoveto{\pgfpoint{241.312256pt}{154.893814pt}}
\pgflineto{\pgfpoint{241.412460pt}{154.954285pt}}
\pgfusepath{stroke}
\pgfpathmoveto{\pgfpoint{241.431839pt}{160.944580pt}}
\pgflineto{\pgfpoint{241.381775pt}{160.833069pt}}
\pgfusepath{stroke}
\pgfpathmoveto{\pgfpoint{241.324890pt}{160.885406pt}}
\pgflineto{\pgfpoint{241.431839pt}{160.944580pt}}
\pgfusepath{stroke}
\pgfpathmoveto{\pgfpoint{241.453308pt}{166.933289pt}}
\pgflineto{\pgfpoint{241.396194pt}{166.819092pt}}
\pgfusepath{stroke}
\pgfpathmoveto{\pgfpoint{241.339081pt}{166.876190pt}}
\pgflineto{\pgfpoint{241.453308pt}{166.933289pt}}
\pgfusepath{stroke}
\pgfpathmoveto{\pgfpoint{241.476883pt}{172.919876pt}}
\pgflineto{\pgfpoint{241.411774pt}{172.803436pt}}
\pgfusepath{stroke}
\pgfpathmoveto{\pgfpoint{241.354935pt}{172.865784pt}}
\pgflineto{\pgfpoint{241.476883pt}{172.919876pt}}
\pgfusepath{stroke}
\pgfpathmoveto{\pgfpoint{241.502502pt}{178.903595pt}}
\pgflineto{\pgfpoint{241.428421pt}{178.785660pt}}
\pgfusepath{stroke}
\pgfpathmoveto{\pgfpoint{241.372467pt}{178.853699pt}}
\pgflineto{\pgfpoint{241.502502pt}{178.903595pt}}
\pgfusepath{stroke}
\pgfpathmoveto{\pgfpoint{241.529877pt}{184.883667pt}}
\pgflineto{\pgfpoint{241.445801pt}{184.765244pt}}
\pgfusepath{stroke}
\pgfpathmoveto{\pgfpoint{241.391571pt}{184.839371pt}}
\pgflineto{\pgfpoint{241.529877pt}{184.883667pt}}
\pgfusepath{stroke}
\pgfpathmoveto{\pgfpoint{241.558380pt}{190.859222pt}}
\pgflineto{\pgfpoint{241.463470pt}{190.741699pt}}
\pgfusepath{stroke}
\pgfpathmoveto{\pgfpoint{241.411926pt}{190.822159pt}}
\pgflineto{\pgfpoint{241.558380pt}{190.859222pt}}
\pgfusepath{stroke}
\pgfpathmoveto{\pgfpoint{241.587082pt}{196.829437pt}}
\pgflineto{\pgfpoint{241.480637pt}{196.714569pt}}
\pgfusepath{stroke}
\pgfpathmoveto{\pgfpoint{241.433014pt}{196.801407pt}}
\pgflineto{\pgfpoint{241.587082pt}{196.829437pt}}
\pgfusepath{stroke}
\pgfpathmoveto{\pgfpoint{241.614532pt}{202.793793pt}}
\pgflineto{\pgfpoint{241.496338pt}{202.683670pt}}
\pgfusepath{stroke}
\pgfpathmoveto{\pgfpoint{241.453903pt}{202.776611pt}}
\pgflineto{\pgfpoint{241.614532pt}{202.793793pt}}
\pgfusepath{stroke}
\pgfpathmoveto{\pgfpoint{241.639008pt}{208.752213pt}}
\pgflineto{\pgfpoint{241.509399pt}{208.649155pt}}
\pgfusepath{stroke}
\pgfpathmoveto{\pgfpoint{241.473480pt}{208.747528pt}}
\pgflineto{\pgfpoint{241.639008pt}{208.752213pt}}
\pgfusepath{stroke}
\pgfpathmoveto{\pgfpoint{241.658630pt}{214.705338pt}}
\pgflineto{\pgfpoint{241.518631pt}{214.611649pt}}
\pgfusepath{stroke}
\pgfpathmoveto{\pgfpoint{241.490417pt}{214.714386pt}}
\pgflineto{\pgfpoint{241.658630pt}{214.705338pt}}
\pgfusepath{stroke}
\pgfpathmoveto{\pgfpoint{241.671753pt}{220.654617pt}}
\pgflineto{\pgfpoint{241.523102pt}{220.572296pt}}
\pgfusepath{stroke}
\pgfpathmoveto{\pgfpoint{241.503448pt}{220.677933pt}}
\pgflineto{\pgfpoint{241.671753pt}{220.654617pt}}
\pgfusepath{stroke}
\pgfpathmoveto{\pgfpoint{241.677383pt}{226.602097pt}}
\pgflineto{\pgfpoint{241.522385pt}{226.532562pt}}
\pgfusepath{stroke}
\pgfpathmoveto{\pgfpoint{241.511658pt}{226.639465pt}}
\pgflineto{\pgfpoint{241.677383pt}{226.602097pt}}
\pgfusepath{stroke}
\pgfpathmoveto{\pgfpoint{241.675446pt}{232.550095pt}}
\pgflineto{\pgfpoint{241.516678pt}{232.494034pt}}
\pgfusepath{stroke}
\pgfpathmoveto{\pgfpoint{241.514786pt}{232.600510pt}}
\pgflineto{\pgfpoint{241.675446pt}{232.550095pt}}
\pgfusepath{stroke}
\pgfpathmoveto{\pgfpoint{241.666748pt}{238.500717pt}}
\pgflineto{\pgfpoint{241.506760pt}{238.457993pt}}
\pgfusepath{stroke}
\pgfpathmoveto{\pgfpoint{241.513123pt}{238.562546pt}}
\pgflineto{\pgfpoint{241.666748pt}{238.500717pt}}
\pgfusepath{stroke}
\pgfpathmoveto{\pgfpoint{241.652802pt}{244.455414pt}}
\pgflineto{\pgfpoint{241.493774pt}{244.425262pt}}
\pgfusepath{stroke}
\pgfpathmoveto{\pgfpoint{241.507492pt}{244.526718pt}}
\pgflineto{\pgfpoint{241.652802pt}{244.455414pt}}
\pgfusepath{stroke}
\pgfpathmoveto{\pgfpoint{241.635269pt}{250.414932pt}}
\pgflineto{\pgfpoint{241.478958pt}{250.396149pt}}
\pgfusepath{stroke}
\pgfpathmoveto{\pgfpoint{241.498947pt}{250.493683pt}}
\pgflineto{\pgfpoint{241.635269pt}{250.414932pt}}
\pgfusepath{stroke}
\pgfpathmoveto{\pgfpoint{241.615799pt}{256.379272pt}}
\pgflineto{\pgfpoint{241.463379pt}{256.370453pt}}
\pgfusepath{stroke}
\pgfpathmoveto{\pgfpoint{241.488556pt}{256.463684pt}}
\pgflineto{\pgfpoint{241.615799pt}{256.379272pt}}
\pgfusepath{stroke}
\pgfpathmoveto{\pgfpoint{241.595642pt}{262.348053pt}}
\pgflineto{\pgfpoint{241.447830pt}{262.347809pt}}
\pgfusepath{stroke}
\pgfpathmoveto{\pgfpoint{241.477234pt}{262.436523pt}}
\pgflineto{\pgfpoint{241.595642pt}{262.348053pt}}
\pgfusepath{stroke}
\pgfpathmoveto{\pgfpoint{241.575653pt}{268.320557pt}}
\pgflineto{\pgfpoint{241.432816pt}{268.327606pt}}
\pgfusepath{stroke}
\pgfpathmoveto{\pgfpoint{241.465607pt}{268.411926pt}}
\pgflineto{\pgfpoint{241.575653pt}{268.320557pt}}
\pgfusepath{stroke}
\pgfpathmoveto{\pgfpoint{241.556366pt}{274.296143pt}}
\pgflineto{\pgfpoint{241.418610pt}{274.309387pt}}
\pgfusepath{stroke}
\pgfpathmoveto{\pgfpoint{241.454117pt}{274.389404pt}}
\pgflineto{\pgfpoint{241.556366pt}{274.296143pt}}
\pgfusepath{stroke}
\pgfpathmoveto{\pgfpoint{241.538025pt}{280.274109pt}}
\pgflineto{\pgfpoint{241.405273pt}{280.292664pt}}
\pgfusepath{stroke}
\pgfpathmoveto{\pgfpoint{241.442963pt}{280.368591pt}}
\pgflineto{\pgfpoint{241.538025pt}{280.274109pt}}
\pgfusepath{stroke}
\pgfpathmoveto{\pgfpoint{241.520660pt}{286.253906pt}}
\pgflineto{\pgfpoint{241.392822pt}{286.277039pt}}
\pgfusepath{stroke}
\pgfpathmoveto{\pgfpoint{241.432281pt}{286.349121pt}}
\pgflineto{\pgfpoint{241.520660pt}{286.253906pt}}
\pgfusepath{stroke}
\pgfpathmoveto{\pgfpoint{241.504242pt}{292.235229pt}}
\pgflineto{\pgfpoint{241.381165pt}{292.262329pt}}
\pgfusepath{stroke}
\pgfpathmoveto{\pgfpoint{241.422043pt}{292.330750pt}}
\pgflineto{\pgfpoint{241.504242pt}{292.235229pt}}
\pgfusepath{stroke}
\pgfpathmoveto{\pgfpoint{241.488708pt}{298.217743pt}}
\pgflineto{\pgfpoint{241.370224pt}{298.248322pt}}
\pgfusepath{stroke}
\pgfpathmoveto{\pgfpoint{241.412262pt}{298.313293pt}}
\pgflineto{\pgfpoint{241.488708pt}{298.217743pt}}
\pgfusepath{stroke}
\pgfpathmoveto{\pgfpoint{241.473907pt}{304.201233pt}}
\pgflineto{\pgfpoint{241.359894pt}{304.234894pt}}
\pgfusepath{stroke}
\pgfpathmoveto{\pgfpoint{241.402878pt}{304.296570pt}}
\pgflineto{\pgfpoint{241.473907pt}{304.201233pt}}
\pgfusepath{stroke}
\pgfpathmoveto{\pgfpoint{241.459808pt}{310.185608pt}}
\pgflineto{\pgfpoint{241.350098pt}{310.221924pt}}
\pgfusepath{stroke}
\pgfpathmoveto{\pgfpoint{241.393860pt}{310.280487pt}}
\pgflineto{\pgfpoint{241.459808pt}{310.185608pt}}
\pgfusepath{stroke}
\pgfpathmoveto{\pgfpoint{241.446304pt}{316.170715pt}}
\pgflineto{\pgfpoint{241.340805pt}{316.209412pt}}
\pgfusepath{stroke}
\pgfpathmoveto{\pgfpoint{241.385147pt}{316.264954pt}}
\pgflineto{\pgfpoint{241.446304pt}{316.170715pt}}
\pgfusepath{stroke}
\pgfpathmoveto{\pgfpoint{241.433380pt}{322.156494pt}}
\pgflineto{\pgfpoint{241.331970pt}{322.197327pt}}
\pgfusepath{stroke}
\pgfpathmoveto{\pgfpoint{241.376740pt}{322.250000pt}}
\pgflineto{\pgfpoint{241.433380pt}{322.156494pt}}
\pgfusepath{stroke}
\pgfpathmoveto{\pgfpoint{241.420959pt}{328.142944pt}}
\pgflineto{\pgfpoint{241.323547pt}{328.185547pt}}
\pgfusepath{stroke}
\pgfpathmoveto{\pgfpoint{241.368622pt}{328.235474pt}}
\pgflineto{\pgfpoint{241.420959pt}{328.142944pt}}
\pgfusepath{stroke}
\pgfpathmoveto{\pgfpoint{241.409088pt}{334.129974pt}}
\pgflineto{\pgfpoint{241.315552pt}{334.174164pt}}
\pgfusepath{stroke}
\pgfpathmoveto{\pgfpoint{241.360779pt}{334.221466pt}}
\pgflineto{\pgfpoint{241.409088pt}{334.129974pt}}
\pgfusepath{stroke}
\pgfpathmoveto{\pgfpoint{241.397736pt}{340.117584pt}}
\pgflineto{\pgfpoint{241.307968pt}{340.163086pt}}
\pgfusepath{stroke}
\pgfpathmoveto{\pgfpoint{241.353210pt}{340.207825pt}}
\pgflineto{\pgfpoint{241.397736pt}{340.117584pt}}
\pgfusepath{stroke}
\pgfpathmoveto{\pgfpoint{241.386887pt}{346.105713pt}}
\pgflineto{\pgfpoint{241.300781pt}{346.152283pt}}
\pgfusepath{stroke}
\pgfpathmoveto{\pgfpoint{241.345947pt}{346.194641pt}}
\pgflineto{\pgfpoint{241.386887pt}{346.105713pt}}
\pgfusepath{stroke}
\pgfpathmoveto{\pgfpoint{241.376556pt}{352.094238pt}}
\pgflineto{\pgfpoint{241.293976pt}{352.141724pt}}
\pgfusepath{stroke}
\pgfpathmoveto{\pgfpoint{241.338974pt}{352.181763pt}}
\pgflineto{\pgfpoint{241.376556pt}{352.094238pt}}
\pgfusepath{stroke}
\pgfpathmoveto{\pgfpoint{241.366730pt}{358.083191pt}}
\pgflineto{\pgfpoint{241.287567pt}{358.131378pt}}
\pgfusepath{stroke}
\pgfpathmoveto{\pgfpoint{241.332291pt}{358.169250pt}}
\pgflineto{\pgfpoint{241.366730pt}{358.083191pt}}
\pgfusepath{stroke}
\pgfpathmoveto{\pgfpoint{241.357422pt}{364.072571pt}}
\pgflineto{\pgfpoint{241.281540pt}{364.121216pt}}
\pgfusepath{stroke}
\pgfpathmoveto{\pgfpoint{241.325912pt}{364.157013pt}}
\pgflineto{\pgfpoint{241.357422pt}{364.072571pt}}
\pgfusepath{stroke}
\pgfpathmoveto{\pgfpoint{241.348633pt}{370.062195pt}}
\pgflineto{\pgfpoint{241.275879pt}{370.111206pt}}
\pgfusepath{stroke}
\pgfpathmoveto{\pgfpoint{241.319839pt}{370.145020pt}}
\pgflineto{\pgfpoint{241.348633pt}{370.062195pt}}
\pgfusepath{stroke}
\pgfpathmoveto{\pgfpoint{247.265045pt}{77.053452pt}}
\pgflineto{\pgfpoint{247.259552pt}{76.979965pt}}
\pgfusepath{stroke}
\pgfpathmoveto{\pgfpoint{247.216553pt}{76.997955pt}}
\pgflineto{\pgfpoint{247.265045pt}{77.053452pt}}
\pgfusepath{stroke}
\pgfpathmoveto{\pgfpoint{247.270081pt}{83.046326pt}}
\pgflineto{\pgfpoint{247.263474pt}{82.970581pt}}
\pgfusepath{stroke}
\pgfpathmoveto{\pgfpoint{247.219345pt}{82.989700pt}}
\pgflineto{\pgfpoint{247.270081pt}{83.046326pt}}
\pgfusepath{stroke}
\pgfpathmoveto{\pgfpoint{247.275604pt}{89.039444pt}}
\pgflineto{\pgfpoint{247.267746pt}{88.961296pt}}
\pgfusepath{stroke}
\pgfpathmoveto{\pgfpoint{247.222427pt}{88.981636pt}}
\pgflineto{\pgfpoint{247.275604pt}{89.039444pt}}
\pgfusepath{stroke}
\pgfpathmoveto{\pgfpoint{247.281647pt}{95.032776pt}}
\pgflineto{\pgfpoint{247.272400pt}{94.952095pt}}
\pgfusepath{stroke}
\pgfpathmoveto{\pgfpoint{247.225845pt}{94.973785pt}}
\pgflineto{\pgfpoint{247.281647pt}{95.032776pt}}
\pgfusepath{stroke}
\pgfpathmoveto{\pgfpoint{247.288330pt}{101.026344pt}}
\pgflineto{\pgfpoint{247.277496pt}{100.942963pt}}
\pgfusepath{stroke}
\pgfpathmoveto{\pgfpoint{247.229660pt}{100.966141pt}}
\pgflineto{\pgfpoint{247.288330pt}{101.026344pt}}
\pgfusepath{stroke}
\pgfpathmoveto{\pgfpoint{247.295715pt}{107.020103pt}}
\pgflineto{\pgfpoint{247.283096pt}{106.933922pt}}
\pgfusepath{stroke}
\pgfpathmoveto{\pgfpoint{247.233917pt}{106.958725pt}}
\pgflineto{\pgfpoint{247.295715pt}{107.020103pt}}
\pgfusepath{stroke}
\pgfpathmoveto{\pgfpoint{247.303894pt}{113.014076pt}}
\pgflineto{\pgfpoint{247.289246pt}{112.924919pt}}
\pgfusepath{stroke}
\pgfpathmoveto{\pgfpoint{247.238663pt}{112.951546pt}}
\pgflineto{\pgfpoint{247.303894pt}{113.014076pt}}
\pgfusepath{stroke}
\pgfpathmoveto{\pgfpoint{247.312988pt}{119.008247pt}}
\pgflineto{\pgfpoint{247.296021pt}{118.915939pt}}
\pgfusepath{stroke}
\pgfpathmoveto{\pgfpoint{247.244019pt}{118.944588pt}}
\pgflineto{\pgfpoint{247.312988pt}{119.008247pt}}
\pgfusepath{stroke}
\pgfpathmoveto{\pgfpoint{247.323135pt}{125.002571pt}}
\pgflineto{\pgfpoint{247.303513pt}{124.906944pt}}
\pgfusepath{stroke}
\pgfpathmoveto{\pgfpoint{247.250061pt}{124.937851pt}}
\pgflineto{\pgfpoint{247.323135pt}{125.002571pt}}
\pgfusepath{stroke}
\pgfpathmoveto{\pgfpoint{247.334488pt}{130.996979pt}}
\pgflineto{\pgfpoint{247.311813pt}{130.897873pt}}
\pgfusepath{stroke}
\pgfpathmoveto{\pgfpoint{247.256882pt}{130.931305pt}}
\pgflineto{\pgfpoint{247.334488pt}{130.996979pt}}
\pgfusepath{stroke}
\pgfpathmoveto{\pgfpoint{247.347229pt}{136.991409pt}}
\pgflineto{\pgfpoint{247.321045pt}{136.888657pt}}
\pgfusepath{stroke}
\pgfpathmoveto{\pgfpoint{247.264633pt}{136.924911pt}}
\pgflineto{\pgfpoint{247.347229pt}{136.991409pt}}
\pgfusepath{stroke}
\pgfpathmoveto{\pgfpoint{247.361572pt}{142.985718pt}}
\pgflineto{\pgfpoint{247.331329pt}{142.879181pt}}
\pgfusepath{stroke}
\pgfpathmoveto{\pgfpoint{247.273468pt}{142.918640pt}}
\pgflineto{\pgfpoint{247.361572pt}{142.985718pt}}
\pgfusepath{stroke}
\pgfpathmoveto{\pgfpoint{247.377747pt}{148.979736pt}}
\pgflineto{\pgfpoint{247.342819pt}{148.869324pt}}
\pgfusepath{stroke}
\pgfpathmoveto{\pgfpoint{247.283554pt}{148.912354pt}}
\pgflineto{\pgfpoint{247.377747pt}{148.979736pt}}
\pgfusepath{stroke}
\pgfpathmoveto{\pgfpoint{247.396011pt}{154.973190pt}}
\pgflineto{\pgfpoint{247.355621pt}{154.858856pt}}
\pgfusepath{stroke}
\pgfpathmoveto{\pgfpoint{247.295105pt}{154.905945pt}}
\pgflineto{\pgfpoint{247.396011pt}{154.973190pt}}
\pgfusepath{stroke}
\pgfpathmoveto{\pgfpoint{247.416626pt}{160.965729pt}}
\pgflineto{\pgfpoint{247.369904pt}{160.847504pt}}
\pgfusepath{stroke}
\pgfpathmoveto{\pgfpoint{247.308319pt}{160.899185pt}}
\pgflineto{\pgfpoint{247.416626pt}{160.965729pt}}
\pgfusepath{stroke}
\pgfpathmoveto{\pgfpoint{247.439880pt}{166.956879pt}}
\pgflineto{\pgfpoint{247.385773pt}{166.834930pt}}
\pgfusepath{stroke}
\pgfpathmoveto{\pgfpoint{247.323425pt}{166.891785pt}}
\pgflineto{\pgfpoint{247.439880pt}{166.956879pt}}
\pgfusepath{stroke}
\pgfpathmoveto{\pgfpoint{247.465942pt}{172.945953pt}}
\pgflineto{\pgfpoint{247.403305pt}{172.820679pt}}
\pgfusepath{stroke}
\pgfpathmoveto{\pgfpoint{247.340668pt}{172.883301pt}}
\pgflineto{\pgfpoint{247.465942pt}{172.945953pt}}
\pgfusepath{stroke}
\pgfpathmoveto{\pgfpoint{247.494934pt}{178.932068pt}}
\pgflineto{\pgfpoint{247.422455pt}{178.804108pt}}
\pgfusepath{stroke}
\pgfpathmoveto{\pgfpoint{247.360168pt}{178.873199pt}}
\pgflineto{\pgfpoint{247.494934pt}{178.932068pt}}
\pgfusepath{stroke}
\pgfpathmoveto{\pgfpoint{247.526672pt}{184.914139pt}}
\pgflineto{\pgfpoint{247.442978pt}{184.784546pt}}
\pgfusepath{stroke}
\pgfpathmoveto{\pgfpoint{247.381973pt}{184.860687pt}}
\pgflineto{\pgfpoint{247.526672pt}{184.914139pt}}
\pgfusepath{stroke}
\pgfpathmoveto{\pgfpoint{247.560654pt}{190.890900pt}}
\pgflineto{\pgfpoint{247.464401pt}{190.761200pt}}
\pgfusepath{stroke}
\pgfpathmoveto{\pgfpoint{247.405838pt}{190.844894pt}}
\pgflineto{\pgfpoint{247.560654pt}{190.890900pt}}
\pgfusepath{stroke}
\pgfpathmoveto{\pgfpoint{247.595764pt}{196.861023pt}}
\pgflineto{\pgfpoint{247.485855pt}{196.733307pt}}
\pgfusepath{stroke}
\pgfpathmoveto{\pgfpoint{247.431213pt}{196.824783pt}}
\pgflineto{\pgfpoint{247.595764pt}{196.861023pt}}
\pgfusepath{stroke}
\pgfpathmoveto{\pgfpoint{247.630219pt}{202.823410pt}}
\pgflineto{\pgfpoint{247.506012pt}{202.700317pt}}
\pgfusepath{stroke}
\pgfpathmoveto{\pgfpoint{247.457001pt}{202.799469pt}}
\pgflineto{\pgfpoint{247.630219pt}{202.823410pt}}
\pgfusepath{stroke}
\pgfpathmoveto{\pgfpoint{247.661530pt}{208.777588pt}}
\pgflineto{\pgfpoint{247.523148pt}{208.662216pt}}
\pgfusepath{stroke}
\pgfpathmoveto{\pgfpoint{247.481598pt}{208.768311pt}}
\pgflineto{\pgfpoint{247.661530pt}{208.777588pt}}
\pgfusepath{stroke}
\pgfpathmoveto{\pgfpoint{247.686829pt}{214.724167pt}}
\pgflineto{\pgfpoint{247.535461pt}{214.619705pt}}
\pgfusepath{stroke}
\pgfpathmoveto{\pgfpoint{247.503052pt}{214.731430pt}}
\pgflineto{\pgfpoint{247.686829pt}{214.724167pt}}
\pgfusepath{stroke}
\pgfpathmoveto{\pgfpoint{247.703476pt}{220.665100pt}}
\pgflineto{\pgfpoint{247.541336pt}{220.574402pt}}
\pgfusepath{stroke}
\pgfpathmoveto{\pgfpoint{247.519348pt}{220.689819pt}}
\pgflineto{\pgfpoint{247.703476pt}{220.665100pt}}
\pgfusepath{stroke}
\pgfpathmoveto{\pgfpoint{247.709793pt}{226.603516pt}}
\pgflineto{\pgfpoint{247.540070pt}{226.528564pt}}
\pgfusepath{stroke}
\pgfpathmoveto{\pgfpoint{247.529037pt}{226.645386pt}}
\pgflineto{\pgfpoint{247.709793pt}{226.603516pt}}
\pgfusepath{stroke}
\pgfpathmoveto{\pgfpoint{247.705688pt}{232.542999pt}}
\pgflineto{\pgfpoint{247.531967pt}{232.484589pt}}
\pgfusepath{stroke}
\pgfpathmoveto{\pgfpoint{247.531662pt}{232.600510pt}}
\pgflineto{\pgfpoint{247.705688pt}{232.542999pt}}
\pgfusepath{stroke}
\pgfpathmoveto{\pgfpoint{247.692627pt}{238.486710pt}}
\pgflineto{\pgfpoint{247.518372pt}{238.444427pt}}
\pgfusepath{stroke}
\pgfpathmoveto{\pgfpoint{247.527847pt}{238.557434pt}}
\pgflineto{\pgfpoint{247.692627pt}{238.486710pt}}
\pgfusepath{stroke}
\pgfpathmoveto{\pgfpoint{247.673065pt}{244.436646pt}}
\pgflineto{\pgfpoint{247.501129pt}{244.409134pt}}
\pgfusepath{stroke}
\pgfpathmoveto{\pgfpoint{247.519012pt}{244.517807pt}}
\pgflineto{\pgfpoint{247.673065pt}{244.436646pt}}
\pgfusepath{stroke}
\pgfpathmoveto{\pgfpoint{247.649719pt}{250.393494pt}}
\pgflineto{\pgfpoint{247.482117pt}{250.378891pt}}
\pgfusepath{stroke}
\pgfpathmoveto{\pgfpoint{247.506866pt}{250.482361pt}}
\pgflineto{\pgfpoint{247.649719pt}{250.393494pt}}
\pgfusepath{stroke}
\pgfpathmoveto{\pgfpoint{247.624863pt}{256.356873pt}}
\pgflineto{\pgfpoint{247.462845pt}{256.353149pt}}
\pgfusepath{stroke}
\pgfpathmoveto{\pgfpoint{247.493011pt}{256.451111pt}}
\pgflineto{\pgfpoint{247.624863pt}{256.356873pt}}
\pgfusepath{stroke}
\pgfpathmoveto{\pgfpoint{247.600143pt}{262.325867pt}}
\pgflineto{\pgfpoint{247.444275pt}{262.331146pt}}
\pgfusepath{stroke}
\pgfpathmoveto{\pgfpoint{247.478622pt}{262.423615pt}}
\pgflineto{\pgfpoint{247.600143pt}{262.325867pt}}
\pgfusepath{stroke}
\pgfpathmoveto{\pgfpoint{247.576538pt}{268.299316pt}}
\pgflineto{\pgfpoint{247.426941pt}{268.311981pt}}
\pgfusepath{stroke}
\pgfpathmoveto{\pgfpoint{247.464478pt}{268.399200pt}}
\pgflineto{\pgfpoint{247.576538pt}{268.299316pt}}
\pgfusepath{stroke}
\pgfpathmoveto{\pgfpoint{247.554443pt}{274.276093pt}}
\pgflineto{\pgfpoint{247.411011pt}{274.294861pt}}
\pgfusepath{stroke}
\pgfpathmoveto{\pgfpoint{247.450974pt}{274.377167pt}}
\pgflineto{\pgfpoint{247.554443pt}{274.276093pt}}
\pgfusepath{stroke}
\pgfpathmoveto{\pgfpoint{247.533981pt}{280.255310pt}}
\pgflineto{\pgfpoint{247.396423pt}{280.279236pt}}
\pgfusepath{stroke}
\pgfpathmoveto{\pgfpoint{247.438263pt}{280.356964pt}}
\pgflineto{\pgfpoint{247.533981pt}{280.255310pt}}
\pgfusepath{stroke}
\pgfpathmoveto{\pgfpoint{247.515015pt}{286.236328pt}}
\pgflineto{\pgfpoint{247.383026pt}{286.264587pt}}
\pgfusepath{stroke}
\pgfpathmoveto{\pgfpoint{247.426361pt}{286.338135pt}}
\pgflineto{\pgfpoint{247.515015pt}{286.236328pt}}
\pgfusepath{stroke}
\pgfpathmoveto{\pgfpoint{247.497345pt}{292.218689pt}}
\pgflineto{\pgfpoint{247.370636pt}{292.250732pt}}
\pgfusepath{stroke}
\pgfpathmoveto{\pgfpoint{247.415192pt}{292.320343pt}}
\pgflineto{\pgfpoint{247.497345pt}{292.218689pt}}
\pgfusepath{stroke}
\pgfpathmoveto{\pgfpoint{247.480774pt}{298.202118pt}}
\pgflineto{\pgfpoint{247.359131pt}{298.237427pt}}
\pgfusepath{stroke}
\pgfpathmoveto{\pgfpoint{247.404633pt}{298.303345pt}}
\pgflineto{\pgfpoint{247.480774pt}{298.202118pt}}
\pgfusepath{stroke}
\pgfpathmoveto{\pgfpoint{247.465134pt}{304.186401pt}}
\pgflineto{\pgfpoint{247.348297pt}{304.224609pt}}
\pgfusepath{stroke}
\pgfpathmoveto{\pgfpoint{247.394592pt}{304.287048pt}}
\pgflineto{\pgfpoint{247.465134pt}{304.186401pt}}
\pgfusepath{stroke}
\pgfpathmoveto{\pgfpoint{247.450256pt}{310.171417pt}}
\pgflineto{\pgfpoint{247.338074pt}{310.212189pt}}
\pgfusepath{stroke}
\pgfpathmoveto{\pgfpoint{247.384979pt}{310.271332pt}}
\pgflineto{\pgfpoint{247.450256pt}{310.171417pt}}
\pgfusepath{stroke}
\pgfpathmoveto{\pgfpoint{247.436066pt}{316.157135pt}}
\pgflineto{\pgfpoint{247.328400pt}{316.200165pt}}
\pgfusepath{stroke}
\pgfpathmoveto{\pgfpoint{247.375732pt}{316.256165pt}}
\pgflineto{\pgfpoint{247.436066pt}{316.157135pt}}
\pgfusepath{stroke}
\pgfpathmoveto{\pgfpoint{247.422470pt}{322.143524pt}}
\pgflineto{\pgfpoint{247.319183pt}{322.188507pt}}
\pgfusepath{stroke}
\pgfpathmoveto{\pgfpoint{247.366837pt}{322.241486pt}}
\pgflineto{\pgfpoint{247.422470pt}{322.143524pt}}
\pgfusepath{stroke}
\pgfpathmoveto{\pgfpoint{247.409454pt}{328.130493pt}}
\pgflineto{\pgfpoint{247.310425pt}{328.177185pt}}
\pgfusepath{stroke}
\pgfpathmoveto{\pgfpoint{247.358246pt}{328.227295pt}}
\pgflineto{\pgfpoint{247.409454pt}{328.130493pt}}
\pgfusepath{stroke}
\pgfpathmoveto{\pgfpoint{247.397003pt}{334.118103pt}}
\pgflineto{\pgfpoint{247.302094pt}{334.166229pt}}
\pgfusepath{stroke}
\pgfpathmoveto{\pgfpoint{247.349960pt}{334.213531pt}}
\pgflineto{\pgfpoint{247.397003pt}{334.118103pt}}
\pgfusepath{stroke}
\pgfpathmoveto{\pgfpoint{247.385117pt}{340.106232pt}}
\pgflineto{\pgfpoint{247.294220pt}{340.155548pt}}
\pgfusepath{stroke}
\pgfpathmoveto{\pgfpoint{247.341995pt}{340.200226pt}}
\pgflineto{\pgfpoint{247.385117pt}{340.106232pt}}
\pgfusepath{stroke}
\pgfpathmoveto{\pgfpoint{247.373795pt}{346.094849pt}}
\pgflineto{\pgfpoint{247.286758pt}{346.145142pt}}
\pgfusepath{stroke}
\pgfpathmoveto{\pgfpoint{247.334335pt}{346.187317pt}}
\pgflineto{\pgfpoint{247.373795pt}{346.094849pt}}
\pgfusepath{stroke}
\pgfpathmoveto{\pgfpoint{247.363022pt}{352.083984pt}}
\pgflineto{\pgfpoint{247.279724pt}{352.135010pt}}
\pgfusepath{stroke}
\pgfpathmoveto{\pgfpoint{247.327026pt}{352.174774pt}}
\pgflineto{\pgfpoint{247.363022pt}{352.083984pt}}
\pgfusepath{stroke}
\pgfpathmoveto{\pgfpoint{247.352829pt}{358.073486pt}}
\pgflineto{\pgfpoint{247.273132pt}{358.125061pt}}
\pgfusepath{stroke}
\pgfpathmoveto{\pgfpoint{247.320023pt}{358.162567pt}}
\pgflineto{\pgfpoint{247.352829pt}{358.073486pt}}
\pgfusepath{stroke}
\pgfpathmoveto{\pgfpoint{247.343201pt}{364.063354pt}}
\pgflineto{\pgfpoint{247.266937pt}{364.115295pt}}
\pgfusepath{stroke}
\pgfpathmoveto{\pgfpoint{247.313370pt}{364.150665pt}}
\pgflineto{\pgfpoint{247.343201pt}{364.063354pt}}
\pgfusepath{stroke}
\pgfpathmoveto{\pgfpoint{247.334106pt}{370.053497pt}}
\pgflineto{\pgfpoint{247.261154pt}{370.105652pt}}
\pgfusepath{stroke}
\pgfpathmoveto{\pgfpoint{247.307037pt}{370.138977pt}}
\pgflineto{\pgfpoint{247.334106pt}{370.053497pt}}
\pgfusepath{stroke}
\pgfpathmoveto{\pgfpoint{253.247559pt}{77.058151pt}}
\pgflineto{\pgfpoint{253.244339pt}{76.983673pt}}
\pgfusepath{stroke}
\pgfpathmoveto{\pgfpoint{253.200302pt}{77.000504pt}}
\pgflineto{\pgfpoint{253.247559pt}{77.058151pt}}
\pgfusepath{stroke}
\pgfpathmoveto{\pgfpoint{253.252350pt}{83.051483pt}}
\pgflineto{\pgfpoint{253.248154pt}{82.974609pt}}
\pgfusepath{stroke}
\pgfpathmoveto{\pgfpoint{253.202881pt}{82.992508pt}}
\pgflineto{\pgfpoint{253.252350pt}{83.051483pt}}
\pgfusepath{stroke}
\pgfpathmoveto{\pgfpoint{253.257614pt}{89.045097pt}}
\pgflineto{\pgfpoint{253.252319pt}{88.965698pt}}
\pgfusepath{stroke}
\pgfpathmoveto{\pgfpoint{253.205719pt}{88.984764pt}}
\pgflineto{\pgfpoint{253.257614pt}{89.045097pt}}
\pgfusepath{stroke}
\pgfpathmoveto{\pgfpoint{253.263412pt}{95.039009pt}}
\pgflineto{\pgfpoint{253.256866pt}{94.956909pt}}
\pgfusepath{stroke}
\pgfpathmoveto{\pgfpoint{253.208908pt}{94.977264pt}}
\pgflineto{\pgfpoint{253.263412pt}{95.039009pt}}
\pgfusepath{stroke}
\pgfpathmoveto{\pgfpoint{253.269821pt}{101.033218pt}}
\pgflineto{\pgfpoint{253.261841pt}{100.948257pt}}
\pgfusepath{stroke}
\pgfpathmoveto{\pgfpoint{253.212463pt}{100.970039pt}}
\pgflineto{\pgfpoint{253.269821pt}{101.033218pt}}
\pgfusepath{stroke}
\pgfpathmoveto{\pgfpoint{253.276947pt}{107.027748pt}}
\pgflineto{\pgfpoint{253.267334pt}{106.939735pt}}
\pgfusepath{stroke}
\pgfpathmoveto{\pgfpoint{253.216461pt}{106.963089pt}}
\pgflineto{\pgfpoint{253.276947pt}{107.027748pt}}
\pgfusepath{stroke}
\pgfpathmoveto{\pgfpoint{253.284851pt}{113.022591pt}}
\pgflineto{\pgfpoint{253.273407pt}{112.931335pt}}
\pgfusepath{stroke}
\pgfpathmoveto{\pgfpoint{253.220947pt}{112.956451pt}}
\pgflineto{\pgfpoint{253.284851pt}{113.022591pt}}
\pgfusepath{stroke}
\pgfpathmoveto{\pgfpoint{253.293716pt}{119.017754pt}}
\pgflineto{\pgfpoint{253.280136pt}{118.923050pt}}
\pgfusepath{stroke}
\pgfpathmoveto{\pgfpoint{253.226044pt}{118.950134pt}}
\pgflineto{\pgfpoint{253.293716pt}{119.017754pt}}
\pgfusepath{stroke}
\pgfpathmoveto{\pgfpoint{253.303650pt}{125.013222pt}}
\pgflineto{\pgfpoint{253.287628pt}{124.914841pt}}
\pgfusepath{stroke}
\pgfpathmoveto{\pgfpoint{253.231796pt}{124.944138pt}}
\pgflineto{\pgfpoint{253.303650pt}{125.013222pt}}
\pgfusepath{stroke}
\pgfpathmoveto{\pgfpoint{253.314850pt}{131.008972pt}}
\pgflineto{\pgfpoint{253.295975pt}{130.906677pt}}
\pgfusepath{stroke}
\pgfpathmoveto{\pgfpoint{253.238373pt}{130.938461pt}}
\pgflineto{\pgfpoint{253.314850pt}{131.008972pt}}
\pgfusepath{stroke}
\pgfpathmoveto{\pgfpoint{253.327530pt}{137.004929pt}}
\pgflineto{\pgfpoint{253.305344pt}{136.898499pt}}
\pgfusepath{stroke}
\pgfpathmoveto{\pgfpoint{253.245911pt}{136.933105pt}}
\pgflineto{\pgfpoint{253.327530pt}{137.004929pt}}
\pgfusepath{stroke}
\pgfpathmoveto{\pgfpoint{253.341934pt}{143.001038pt}}
\pgflineto{\pgfpoint{253.315872pt}{142.890228pt}}
\pgfusepath{stroke}
\pgfpathmoveto{\pgfpoint{253.254593pt}{142.928009pt}}
\pgflineto{\pgfpoint{253.341934pt}{143.001038pt}}
\pgfusepath{stroke}
\pgfpathmoveto{\pgfpoint{253.358368pt}{148.997131pt}}
\pgflineto{\pgfpoint{253.327759pt}{148.881714pt}}
\pgfusepath{stroke}
\pgfpathmoveto{\pgfpoint{253.264618pt}{148.923157pt}}
\pgflineto{\pgfpoint{253.358368pt}{148.997131pt}}
\pgfusepath{stroke}
\pgfpathmoveto{\pgfpoint{253.377197pt}{154.992981pt}}
\pgflineto{\pgfpoint{253.341202pt}{154.872772pt}}
\pgfusepath{stroke}
\pgfpathmoveto{\pgfpoint{253.276276pt}{154.918411pt}}
\pgflineto{\pgfpoint{253.377197pt}{154.992981pt}}
\pgfusepath{stroke}
\pgfpathmoveto{\pgfpoint{253.398788pt}{160.988281pt}}
\pgflineto{\pgfpoint{253.356445pt}{160.863159pt}}
\pgfusepath{stroke}
\pgfpathmoveto{\pgfpoint{253.289825pt}{160.913605pt}}
\pgflineto{\pgfpoint{253.398788pt}{160.988281pt}}
\pgfusepath{stroke}
\pgfpathmoveto{\pgfpoint{253.423599pt}{166.982498pt}}
\pgflineto{\pgfpoint{253.373703pt}{166.852463pt}}
\pgfusepath{stroke}
\pgfpathmoveto{\pgfpoint{253.305664pt}{166.908417pt}}
\pgflineto{\pgfpoint{253.423599pt}{166.982498pt}}
\pgfusepath{stroke}
\pgfpathmoveto{\pgfpoint{253.452057pt}{172.974930pt}}
\pgflineto{\pgfpoint{253.393188pt}{172.840179pt}}
\pgfusepath{stroke}
\pgfpathmoveto{\pgfpoint{253.324112pt}{172.902451pt}}
\pgflineto{\pgfpoint{253.452057pt}{172.974930pt}}
\pgfusepath{stroke}
\pgfpathmoveto{\pgfpoint{253.484528pt}{178.964539pt}}
\pgflineto{\pgfpoint{253.415039pt}{178.825562pt}}
\pgfusepath{stroke}
\pgfpathmoveto{\pgfpoint{253.345551pt}{178.895050pt}}
\pgflineto{\pgfpoint{253.484528pt}{178.964539pt}}
\pgfusepath{stroke}
\pgfpathmoveto{\pgfpoint{253.521164pt}{184.949905pt}}
\pgflineto{\pgfpoint{253.439194pt}{184.807648pt}}
\pgfusepath{stroke}
\pgfpathmoveto{\pgfpoint{253.370239pt}{184.885284pt}}
\pgflineto{\pgfpoint{253.521164pt}{184.949905pt}}
\pgfusepath{stroke}
\pgfpathmoveto{\pgfpoint{253.561661pt}{190.929245pt}}
\pgflineto{\pgfpoint{253.465256pt}{190.785294pt}}
\pgfusepath{stroke}
\pgfpathmoveto{\pgfpoint{253.398163pt}{190.871918pt}}
\pgflineto{\pgfpoint{253.561661pt}{190.929245pt}}
\pgfusepath{stroke}
\pgfpathmoveto{\pgfpoint{253.604980pt}{196.900452pt}}
\pgflineto{\pgfpoint{253.492310pt}{196.757202pt}}
\pgfusepath{stroke}
\pgfpathmoveto{\pgfpoint{253.428894pt}{196.853455pt}}
\pgflineto{\pgfpoint{253.604980pt}{196.900452pt}}
\pgfusepath{stroke}
\pgfpathmoveto{\pgfpoint{253.648956pt}{202.861420pt}}
\pgflineto{\pgfpoint{253.518646pt}{202.722214pt}}
\pgfusepath{stroke}
\pgfpathmoveto{\pgfpoint{253.461182pt}{202.828247pt}}
\pgflineto{\pgfpoint{253.648956pt}{202.861420pt}}
\pgfusepath{stroke}
\pgfpathmoveto{\pgfpoint{253.690094pt}{208.810806pt}}
\pgflineto{\pgfpoint{253.541809pt}{208.679825pt}}
\pgfusepath{stroke}
\pgfpathmoveto{\pgfpoint{253.492874pt}{208.794998pt}}
\pgflineto{\pgfpoint{253.690094pt}{208.810806pt}}
\pgfusepath{stroke}
\pgfpathmoveto{\pgfpoint{253.723892pt}{214.748856pt}}
\pgflineto{\pgfpoint{253.558823pt}{214.630676pt}}
\pgfusepath{stroke}
\pgfpathmoveto{\pgfpoint{253.520935pt}{214.753357pt}}
\pgflineto{\pgfpoint{253.723892pt}{214.748856pt}}
\pgfusepath{stroke}
\pgfpathmoveto{\pgfpoint{253.745804pt}{220.678207pt}}
\pgflineto{\pgfpoint{253.566925pt}{220.577072pt}}
\pgfusepath{stroke}
\pgfpathmoveto{\pgfpoint{253.542023pt}{220.704620pt}}
\pgflineto{\pgfpoint{253.745804pt}{220.678207pt}}
\pgfusepath{stroke}
\pgfpathmoveto{\pgfpoint{253.752899pt}{226.603790pt}}
\pgflineto{\pgfpoint{253.564758pt}{226.522583pt}}
\pgfusepath{stroke}
\pgfpathmoveto{\pgfpoint{253.553665pt}{226.651718pt}}
\pgflineto{\pgfpoint{253.752899pt}{226.603790pt}}
\pgfusepath{stroke}
\pgfpathmoveto{\pgfpoint{253.745071pt}{232.531494pt}}
\pgflineto{\pgfpoint{253.552856pt}{232.471176pt}}
\pgfusepath{stroke}
\pgfpathmoveto{\pgfpoint{253.555115pt}{232.598557pt}}
\pgflineto{\pgfpoint{253.745071pt}{232.531494pt}}
\pgfusepath{stroke}
\pgfpathmoveto{\pgfpoint{253.725021pt}{238.466248pt}}
\pgflineto{\pgfpoint{253.533615pt}{238.425827pt}}
\pgfusepath{stroke}
\pgfpathmoveto{\pgfpoint{253.547638pt}{238.548752pt}}
\pgflineto{\pgfpoint{253.725021pt}{238.466248pt}}
\pgfusepath{stroke}
\pgfpathmoveto{\pgfpoint{253.697083pt}{244.410767pt}}
\pgflineto{\pgfpoint{253.510162pt}{244.387863pt}}
\pgfusepath{stroke}
\pgfpathmoveto{\pgfpoint{253.533813pt}{244.504578pt}}
\pgflineto{\pgfpoint{253.697083pt}{244.410767pt}}
\pgfusepath{stroke}
\pgfpathmoveto{\pgfpoint{253.665527pt}{250.365387pt}}
\pgflineto{\pgfpoint{253.485458pt}{250.357040pt}}
\pgfusepath{stroke}
\pgfpathmoveto{\pgfpoint{253.516449pt}{250.466751pt}}
\pgflineto{\pgfpoint{253.665527pt}{250.365387pt}}
\pgfusepath{stroke}
\pgfpathmoveto{\pgfpoint{253.633667pt}{256.328857pt}}
\pgflineto{\pgfpoint{253.461517pt}{256.332153pt}}
\pgfusepath{stroke}
\pgfpathmoveto{\pgfpoint{253.497910pt}{256.434784pt}}
\pgflineto{\pgfpoint{253.633667pt}{256.328857pt}}
\pgfusepath{stroke}
\pgfpathmoveto{\pgfpoint{253.603455pt}{262.299286pt}}
\pgflineto{\pgfpoint{253.439453pt}{262.311707pt}}
\pgfusepath{stroke}
\pgfpathmoveto{\pgfpoint{253.479706pt}{262.407623pt}}
\pgflineto{\pgfpoint{253.603455pt}{262.299286pt}}
\pgfusepath{stroke}
\pgfpathmoveto{\pgfpoint{253.575806pt}{268.274719pt}}
\pgflineto{\pgfpoint{253.419662pt}{268.294312pt}}
\pgfusepath{stroke}
\pgfpathmoveto{\pgfpoint{253.462646pt}{268.384094pt}}
\pgflineto{\pgfpoint{253.575806pt}{268.274719pt}}
\pgfusepath{stroke}
\pgfpathmoveto{\pgfpoint{253.550873pt}{274.253601pt}}
\pgflineto{\pgfpoint{253.402054pt}{274.278931pt}}
\pgfusepath{stroke}
\pgfpathmoveto{\pgfpoint{253.447006pt}{274.363159pt}}
\pgflineto{\pgfpoint{253.550873pt}{274.253601pt}}
\pgfusepath{stroke}
\pgfpathmoveto{\pgfpoint{253.528412pt}{280.234772pt}}
\pgflineto{\pgfpoint{253.386337pt}{280.264771pt}}
\pgfusepath{stroke}
\pgfpathmoveto{\pgfpoint{253.432770pt}{280.344025pt}}
\pgflineto{\pgfpoint{253.528412pt}{280.234772pt}}
\pgfusepath{stroke}
\pgfpathmoveto{\pgfpoint{253.508041pt}{286.217438pt}}
\pgflineto{\pgfpoint{253.372192pt}{286.251404pt}}
\pgfusepath{stroke}
\pgfpathmoveto{\pgfpoint{253.419739pt}{286.326111pt}}
\pgflineto{\pgfpoint{253.508041pt}{286.217438pt}}
\pgfusepath{stroke}
\pgfpathmoveto{\pgfpoint{253.489349pt}{292.201141pt}}
\pgflineto{\pgfpoint{253.359283pt}{292.238556pt}}
\pgfusepath{stroke}
\pgfpathmoveto{\pgfpoint{253.407730pt}{292.309113pt}}
\pgflineto{\pgfpoint{253.489349pt}{292.201141pt}}
\pgfusepath{stroke}
\pgfpathmoveto{\pgfpoint{253.471954pt}{298.185638pt}}
\pgflineto{\pgfpoint{253.347321pt}{298.226074pt}}
\pgfusepath{stroke}
\pgfpathmoveto{\pgfpoint{253.396500pt}{298.292755pt}}
\pgflineto{\pgfpoint{253.471954pt}{298.185638pt}}
\pgfusepath{stroke}
\pgfpathmoveto{\pgfpoint{253.455582pt}{304.170807pt}}
\pgflineto{\pgfpoint{253.336121pt}{304.213898pt}}
\pgfusepath{stroke}
\pgfpathmoveto{\pgfpoint{253.385880pt}{304.276947pt}}
\pgflineto{\pgfpoint{253.455582pt}{304.170807pt}}
\pgfusepath{stroke}
\pgfpathmoveto{\pgfpoint{253.440048pt}{310.156555pt}}
\pgflineto{\pgfpoint{253.325546pt}{310.202057pt}}
\pgfusepath{stroke}
\pgfpathmoveto{\pgfpoint{253.375732pt}{310.261658pt}}
\pgflineto{\pgfpoint{253.440048pt}{310.156555pt}}
\pgfusepath{stroke}
\pgfpathmoveto{\pgfpoint{253.425201pt}{316.142944pt}}
\pgflineto{\pgfpoint{253.315491pt}{316.190552pt}}
\pgfusepath{stroke}
\pgfpathmoveto{\pgfpoint{253.365997pt}{316.246826pt}}
\pgflineto{\pgfpoint{253.425201pt}{316.142944pt}}
\pgfusepath{stroke}
\pgfpathmoveto{\pgfpoint{253.410995pt}{322.129913pt}}
\pgflineto{\pgfpoint{253.305939pt}{322.179352pt}}
\pgfusepath{stroke}
\pgfpathmoveto{\pgfpoint{253.356613pt}{322.232483pt}}
\pgflineto{\pgfpoint{253.410995pt}{322.129913pt}}
\pgfusepath{stroke}
\pgfpathmoveto{\pgfpoint{253.397385pt}{328.117493pt}}
\pgflineto{\pgfpoint{253.296844pt}{328.168518pt}}
\pgfusepath{stroke}
\pgfpathmoveto{\pgfpoint{253.347565pt}{328.218628pt}}
\pgflineto{\pgfpoint{253.397385pt}{328.117493pt}}
\pgfusepath{stroke}
\pgfpathmoveto{\pgfpoint{253.384369pt}{334.105652pt}}
\pgflineto{\pgfpoint{253.288208pt}{334.157959pt}}
\pgfusepath{stroke}
\pgfpathmoveto{\pgfpoint{253.338837pt}{334.205200pt}}
\pgflineto{\pgfpoint{253.384369pt}{334.105652pt}}
\pgfusepath{stroke}
\pgfpathmoveto{\pgfpoint{253.371948pt}{340.094360pt}}
\pgflineto{\pgfpoint{253.280029pt}{340.147766pt}}
\pgfusepath{stroke}
\pgfpathmoveto{\pgfpoint{253.330444pt}{340.192230pt}}
\pgflineto{\pgfpoint{253.371948pt}{340.094360pt}}
\pgfusepath{stroke}
\pgfpathmoveto{\pgfpoint{253.360138pt}{346.083588pt}}
\pgflineto{\pgfpoint{253.272324pt}{346.137817pt}}
\pgfusepath{stroke}
\pgfpathmoveto{\pgfpoint{253.322403pt}{346.179626pt}}
\pgflineto{\pgfpoint{253.360138pt}{346.083588pt}}
\pgfusepath{stroke}
\pgfpathmoveto{\pgfpoint{253.348938pt}{352.073303pt}}
\pgflineto{\pgfpoint{253.265076pt}{352.128082pt}}
\pgfusepath{stroke}
\pgfpathmoveto{\pgfpoint{253.314728pt}{352.167450pt}}
\pgflineto{\pgfpoint{253.348938pt}{352.073303pt}}
\pgfusepath{stroke}
\pgfpathmoveto{\pgfpoint{253.338348pt}{358.063416pt}}
\pgflineto{\pgfpoint{253.258286pt}{358.118591pt}}
\pgfusepath{stroke}
\pgfpathmoveto{\pgfpoint{253.307419pt}{358.155579pt}}
\pgflineto{\pgfpoint{253.338348pt}{358.063416pt}}
\pgfusepath{stroke}
\pgfpathmoveto{\pgfpoint{253.328400pt}{364.053833pt}}
\pgflineto{\pgfpoint{253.251938pt}{364.109222pt}}
\pgfusepath{stroke}
\pgfpathmoveto{\pgfpoint{253.300476pt}{364.144012pt}}
\pgflineto{\pgfpoint{253.328400pt}{364.053833pt}}
\pgfusepath{stroke}
\pgfpathmoveto{\pgfpoint{253.319046pt}{370.044556pt}}
\pgflineto{\pgfpoint{253.246033pt}{370.100037pt}}
\pgfusepath{stroke}
\pgfpathmoveto{\pgfpoint{253.293915pt}{370.132751pt}}
\pgflineto{\pgfpoint{253.319046pt}{370.044556pt}}
\pgfusepath{stroke}
\pgfpathmoveto{\pgfpoint{259.229675pt}{77.062592pt}}
\pgflineto{\pgfpoint{259.228821pt}{76.987259pt}}
\pgfusepath{stroke}
\pgfpathmoveto{\pgfpoint{259.183777pt}{77.002838pt}}
\pgflineto{\pgfpoint{259.229675pt}{77.062592pt}}
\pgfusepath{stroke}
\pgfpathmoveto{\pgfpoint{259.234161pt}{83.056351pt}}
\pgflineto{\pgfpoint{259.232483pt}{82.978516pt}}
\pgfusepath{stroke}
\pgfpathmoveto{\pgfpoint{259.186127pt}{82.995102pt}}
\pgflineto{\pgfpoint{259.234161pt}{83.056351pt}}
\pgfusepath{stroke}
\pgfpathmoveto{\pgfpoint{259.239105pt}{89.050453pt}}
\pgflineto{\pgfpoint{259.236481pt}{88.969971pt}}
\pgfusepath{stroke}
\pgfpathmoveto{\pgfpoint{259.188721pt}{88.987633pt}}
\pgflineto{\pgfpoint{259.239105pt}{89.050453pt}}
\pgfusepath{stroke}
\pgfpathmoveto{\pgfpoint{259.244568pt}{95.044945pt}}
\pgflineto{\pgfpoint{259.240875pt}{94.961594pt}}
\pgfusepath{stroke}
\pgfpathmoveto{\pgfpoint{259.191620pt}{94.980484pt}}
\pgflineto{\pgfpoint{259.244568pt}{95.044945pt}}
\pgfusepath{stroke}
\pgfpathmoveto{\pgfpoint{259.250610pt}{101.039795pt}}
\pgflineto{\pgfpoint{259.245697pt}{100.953423pt}}
\pgfusepath{stroke}
\pgfpathmoveto{\pgfpoint{259.194855pt}{100.973648pt}}
\pgflineto{\pgfpoint{259.250610pt}{101.039795pt}}
\pgfusepath{stroke}
\pgfpathmoveto{\pgfpoint{259.257355pt}{107.035065pt}}
\pgflineto{\pgfpoint{259.251038pt}{106.945427pt}}
\pgfusepath{stroke}
\pgfpathmoveto{\pgfpoint{259.198517pt}{106.967163pt}}
\pgflineto{\pgfpoint{259.257355pt}{107.035065pt}}
\pgfusepath{stroke}
\pgfpathmoveto{\pgfpoint{259.264893pt}{113.030762pt}}
\pgflineto{\pgfpoint{259.256927pt}{112.937653pt}}
\pgfusepath{stroke}
\pgfpathmoveto{\pgfpoint{259.202667pt}{112.961044pt}}
\pgflineto{\pgfpoint{259.264893pt}{113.030762pt}}
\pgfusepath{stroke}
\pgfpathmoveto{\pgfpoint{259.273376pt}{119.026939pt}}
\pgflineto{\pgfpoint{259.263519pt}{118.930077pt}}
\pgfusepath{stroke}
\pgfpathmoveto{\pgfpoint{259.207367pt}{118.955360pt}}
\pgflineto{\pgfpoint{259.273376pt}{119.026939pt}}
\pgfusepath{stroke}
\pgfpathmoveto{\pgfpoint{259.282898pt}{125.023582pt}}
\pgflineto{\pgfpoint{259.270874pt}{124.922691pt}}
\pgfusepath{stroke}
\pgfpathmoveto{\pgfpoint{259.212738pt}{124.950104pt}}
\pgflineto{\pgfpoint{259.282898pt}{125.023582pt}}
\pgfusepath{stroke}
\pgfpathmoveto{\pgfpoint{259.293762pt}{131.020706pt}}
\pgflineto{\pgfpoint{259.279144pt}{130.915497pt}}
\pgfusepath{stroke}
\pgfpathmoveto{\pgfpoint{259.218933pt}{130.945312pt}}
\pgflineto{\pgfpoint{259.293762pt}{131.020706pt}}
\pgfusepath{stroke}
\pgfpathmoveto{\pgfpoint{259.306122pt}{137.018311pt}}
\pgflineto{\pgfpoint{259.288452pt}{136.908447pt}}
\pgfusepath{stroke}
\pgfpathmoveto{\pgfpoint{259.226074pt}{136.940994pt}}
\pgflineto{\pgfpoint{259.306122pt}{137.018311pt}}
\pgfusepath{stroke}
\pgfpathmoveto{\pgfpoint{259.320282pt}{143.016342pt}}
\pgflineto{\pgfpoint{259.299042pt}{142.901474pt}}
\pgfusepath{stroke}
\pgfpathmoveto{\pgfpoint{259.234375pt}{142.937195pt}}
\pgflineto{\pgfpoint{259.320282pt}{143.016342pt}}
\pgfusepath{stroke}
\pgfpathmoveto{\pgfpoint{259.336639pt}{149.014740pt}}
\pgflineto{\pgfpoint{259.311127pt}{148.894501pt}}
\pgfusepath{stroke}
\pgfpathmoveto{\pgfpoint{259.244080pt}{148.933838pt}}
\pgflineto{\pgfpoint{259.336639pt}{149.014740pt}}
\pgfusepath{stroke}
\pgfpathmoveto{\pgfpoint{259.355591pt}{155.013321pt}}
\pgflineto{\pgfpoint{259.324982pt}{154.887375pt}}
\pgfusepath{stroke}
\pgfpathmoveto{\pgfpoint{259.255554pt}{154.930939pt}}
\pgflineto{\pgfpoint{259.355591pt}{155.013321pt}}
\pgfusepath{stroke}
\pgfpathmoveto{\pgfpoint{259.377747pt}{161.011856pt}}
\pgflineto{\pgfpoint{259.340942pt}{160.879852pt}}
\pgfusepath{stroke}
\pgfpathmoveto{\pgfpoint{259.269104pt}{160.928329pt}}
\pgflineto{\pgfpoint{259.377747pt}{161.011856pt}}
\pgfusepath{stroke}
\pgfpathmoveto{\pgfpoint{259.403656pt}{167.009872pt}}
\pgflineto{\pgfpoint{259.359375pt}{166.871567pt}}
\pgfusepath{stroke}
\pgfpathmoveto{\pgfpoint{259.285248pt}{166.925812pt}}
\pgflineto{\pgfpoint{259.403656pt}{167.009872pt}}
\pgfusepath{stroke}
\pgfpathmoveto{\pgfpoint{259.434143pt}{173.006683pt}}
\pgflineto{\pgfpoint{259.380676pt}{172.861969pt}}
\pgfusepath{stroke}
\pgfpathmoveto{\pgfpoint{259.304565pt}{172.922974pt}}
\pgflineto{\pgfpoint{259.434143pt}{173.006683pt}}
\pgfusepath{stroke}
\pgfpathmoveto{\pgfpoint{259.469910pt}{179.001160pt}}
\pgflineto{\pgfpoint{259.405273pt}{178.850235pt}}
\pgfusepath{stroke}
\pgfpathmoveto{\pgfpoint{259.327637pt}{178.919189pt}}
\pgflineto{\pgfpoint{259.469910pt}{179.001160pt}}
\pgfusepath{stroke}
\pgfpathmoveto{\pgfpoint{259.511658pt}{184.991653pt}}
\pgflineto{\pgfpoint{259.433411pt}{184.835175pt}}
\pgfusepath{stroke}
\pgfpathmoveto{\pgfpoint{259.355164pt}{184.913406pt}}
\pgflineto{\pgfpoint{259.511658pt}{184.991653pt}}
\pgfusepath{stroke}
\pgfpathmoveto{\pgfpoint{259.559662pt}{190.975739pt}}
\pgflineto{\pgfpoint{259.464996pt}{190.815155pt}}
\pgfusepath{stroke}
\pgfpathmoveto{\pgfpoint{259.387573pt}{190.904053pt}}
\pgflineto{\pgfpoint{259.559662pt}{190.975739pt}}
\pgfusepath{stroke}
\pgfpathmoveto{\pgfpoint{259.613312pt}{196.950195pt}}
\pgflineto{\pgfpoint{259.499298pt}{196.788086pt}}
\pgfusepath{stroke}
\pgfpathmoveto{\pgfpoint{259.424835pt}{196.888916pt}}
\pgflineto{\pgfpoint{259.613312pt}{196.950195pt}}
\pgfusepath{stroke}
\pgfpathmoveto{\pgfpoint{259.670319pt}{202.911316pt}}
\pgflineto{\pgfpoint{259.534363pt}{202.751740pt}}
\pgfusepath{stroke}
\pgfpathmoveto{\pgfpoint{259.465820pt}{202.865234pt}}
\pgflineto{\pgfpoint{259.670319pt}{202.911316pt}}
\pgfusepath{stroke}
\pgfpathmoveto{\pgfpoint{259.726013pt}{208.855789pt}}
\pgflineto{\pgfpoint{259.566650pt}{208.704453pt}}
\pgfusepath{stroke}
\pgfpathmoveto{\pgfpoint{259.507751pt}{208.830322pt}}
\pgflineto{\pgfpoint{259.726013pt}{208.855789pt}}
\pgfusepath{stroke}
\pgfpathmoveto{\pgfpoint{259.773102pt}{214.782639pt}}
\pgflineto{\pgfpoint{259.591278pt}{214.646378pt}}
\pgfusepath{stroke}
\pgfpathmoveto{\pgfpoint{259.545898pt}{214.782715pt}}
\pgflineto{\pgfpoint{259.773102pt}{214.782639pt}}
\pgfusepath{stroke}
\pgfpathmoveto{\pgfpoint{259.803467pt}{220.695267pt}}
\pgflineto{\pgfpoint{259.603180pt}{220.580673pt}}
\pgfusepath{stroke}
\pgfpathmoveto{\pgfpoint{259.574493pt}{220.723755pt}}
\pgflineto{\pgfpoint{259.803467pt}{220.695267pt}}
\pgfusepath{stroke}
\pgfpathmoveto{\pgfpoint{259.811462pt}{226.601776pt}}
\pgflineto{\pgfpoint{259.599487pt}{226.513382pt}}
\pgfusepath{stroke}
\pgfpathmoveto{\pgfpoint{259.588837pt}{226.658234pt}}
\pgflineto{\pgfpoint{259.811462pt}{226.601776pt}}
\pgfusepath{stroke}
\pgfpathmoveto{\pgfpoint{259.796936pt}{232.512497pt}}
\pgflineto{\pgfpoint{259.581238pt}{232.451385pt}}
\pgfusepath{stroke}
\pgfpathmoveto{\pgfpoint{259.587708pt}{232.593033pt}}
\pgflineto{\pgfpoint{259.796936pt}{232.512497pt}}
\pgfusepath{stroke}
\pgfpathmoveto{\pgfpoint{259.765320pt}{238.435608pt}}
\pgflineto{\pgfpoint{259.552979pt}{238.399567pt}}
\pgfusepath{stroke}
\pgfpathmoveto{\pgfpoint{259.573822pt}{238.534164pt}}
\pgflineto{\pgfpoint{259.765320pt}{238.435608pt}}
\pgfusepath{stroke}
\pgfpathmoveto{\pgfpoint{259.724487pt}{244.374512pt}}
\pgflineto{\pgfpoint{259.520325pt}{244.359283pt}}
\pgfusepath{stroke}
\pgfpathmoveto{\pgfpoint{259.552032pt}{244.484818pt}}
\pgflineto{\pgfpoint{259.724487pt}{244.374512pt}}
\pgfusepath{stroke}
\pgfpathmoveto{\pgfpoint{259.681427pt}{250.328339pt}}
\pgflineto{\pgfpoint{259.487854pt}{250.329193pt}}
\pgfusepath{stroke}
\pgfpathmoveto{\pgfpoint{259.527100pt}{250.445175pt}}
\pgflineto{\pgfpoint{259.681427pt}{250.328339pt}}
\pgfusepath{stroke}
\pgfpathmoveto{\pgfpoint{259.640625pt}{256.293945pt}}
\pgflineto{\pgfpoint{259.458191pt}{256.306732pt}}
\pgfusepath{stroke}
\pgfpathmoveto{\pgfpoint{259.502350pt}{256.413635pt}}
\pgflineto{\pgfpoint{259.640625pt}{256.293945pt}}
\pgfusepath{stroke}
\pgfpathmoveto{\pgfpoint{259.604126pt}{262.267761pt}}
\pgflineto{\pgfpoint{259.432312pt}{262.289246pt}}
\pgfusepath{stroke}
\pgfpathmoveto{\pgfpoint{259.479553pt}{262.388031pt}}
\pgflineto{\pgfpoint{259.604126pt}{262.267761pt}}
\pgfusepath{stroke}
\pgfpathmoveto{\pgfpoint{259.572327pt}{268.246796pt}}
\pgflineto{\pgfpoint{259.410156pt}{268.274719pt}}
\pgfusepath{stroke}
\pgfpathmoveto{\pgfpoint{259.459351pt}{268.366455pt}}
\pgflineto{\pgfpoint{259.572327pt}{268.246796pt}}
\pgfusepath{stroke}
\pgfpathmoveto{\pgfpoint{259.544800pt}{274.228912pt}}
\pgflineto{\pgfpoint{259.391174pt}{274.261780pt}}
\pgfusepath{stroke}
\pgfpathmoveto{\pgfpoint{259.441620pt}{274.347382pt}}
\pgflineto{\pgfpoint{259.544800pt}{274.228912pt}}
\pgfusepath{stroke}
\pgfpathmoveto{\pgfpoint{259.520752pt}{280.212769pt}}
\pgflineto{\pgfpoint{259.374725pt}{280.249603pt}}
\pgfusepath{stroke}
\pgfpathmoveto{\pgfpoint{259.426025pt}{280.329865pt}}
\pgflineto{\pgfpoint{259.520752pt}{280.212769pt}}
\pgfusepath{stroke}
\pgfpathmoveto{\pgfpoint{259.499420pt}{286.197540pt}}
\pgflineto{\pgfpoint{259.360138pt}{286.237732pt}}
\pgfusepath{stroke}
\pgfpathmoveto{\pgfpoint{259.412140pt}{286.313232pt}}
\pgflineto{\pgfpoint{259.499420pt}{286.197540pt}}
\pgfusepath{stroke}
\pgfpathmoveto{\pgfpoint{259.480042pt}{292.182831pt}}
\pgflineto{\pgfpoint{259.346954pt}{292.226013pt}}
\pgfusepath{stroke}
\pgfpathmoveto{\pgfpoint{259.399475pt}{292.297211pt}}
\pgflineto{\pgfpoint{259.480042pt}{292.182831pt}}
\pgfusepath{stroke}
\pgfpathmoveto{\pgfpoint{259.462097pt}{298.168518pt}}
\pgflineto{\pgfpoint{259.334778pt}{298.214386pt}}
\pgfusepath{stroke}
\pgfpathmoveto{\pgfpoint{259.387756pt}{298.281616pt}}
\pgflineto{\pgfpoint{259.462097pt}{298.168518pt}}
\pgfusepath{stroke}
\pgfpathmoveto{\pgfpoint{259.445221pt}{304.154602pt}}
\pgflineto{\pgfpoint{259.323334pt}{304.202881pt}}
\pgfusepath{stroke}
\pgfpathmoveto{\pgfpoint{259.376678pt}{304.266357pt}}
\pgflineto{\pgfpoint{259.445221pt}{304.154602pt}}
\pgfusepath{stroke}
\pgfpathmoveto{\pgfpoint{259.429138pt}{310.141113pt}}
\pgflineto{\pgfpoint{259.312469pt}{310.191589pt}}
\pgfusepath{stroke}
\pgfpathmoveto{\pgfpoint{259.366119pt}{310.251495pt}}
\pgflineto{\pgfpoint{259.429138pt}{310.141113pt}}
\pgfusepath{stroke}
\pgfpathmoveto{\pgfpoint{259.413727pt}{316.128113pt}}
\pgflineto{\pgfpoint{259.302124pt}{316.180573pt}}
\pgfusepath{stroke}
\pgfpathmoveto{\pgfpoint{259.355927pt}{316.237061pt}}
\pgflineto{\pgfpoint{259.413727pt}{316.128113pt}}
\pgfusepath{stroke}
\pgfpathmoveto{\pgfpoint{259.398926pt}{322.115723pt}}
\pgflineto{\pgfpoint{259.292236pt}{322.169861pt}}
\pgfusepath{stroke}
\pgfpathmoveto{\pgfpoint{259.346069pt}{322.223053pt}}
\pgflineto{\pgfpoint{259.398926pt}{322.115723pt}}
\pgfusepath{stroke}
\pgfpathmoveto{\pgfpoint{259.384735pt}{328.103882pt}}
\pgflineto{\pgfpoint{259.282806pt}{328.159485pt}}
\pgfusepath{stroke}
\pgfpathmoveto{\pgfpoint{259.336548pt}{328.209503pt}}
\pgflineto{\pgfpoint{259.384735pt}{328.103882pt}}
\pgfusepath{stroke}
\pgfpathmoveto{\pgfpoint{259.371124pt}{334.092651pt}}
\pgflineto{\pgfpoint{259.273865pt}{334.149414pt}}
\pgfusepath{stroke}
\pgfpathmoveto{\pgfpoint{259.327362pt}{334.196411pt}}
\pgflineto{\pgfpoint{259.371124pt}{334.092651pt}}
\pgfusepath{stroke}
\pgfpathmoveto{\pgfpoint{259.358154pt}{340.081970pt}}
\pgflineto{\pgfpoint{259.265381pt}{340.139648pt}}
\pgfusepath{stroke}
\pgfpathmoveto{\pgfpoint{259.318542pt}{340.183777pt}}
\pgflineto{\pgfpoint{259.358154pt}{340.081970pt}}
\pgfusepath{stroke}
\pgfpathmoveto{\pgfpoint{259.345856pt}{346.071838pt}}
\pgflineto{\pgfpoint{259.257416pt}{346.130188pt}}
\pgfusepath{stroke}
\pgfpathmoveto{\pgfpoint{259.310120pt}{346.171570pt}}
\pgflineto{\pgfpoint{259.345856pt}{346.071838pt}}
\pgfusepath{stroke}
\pgfpathmoveto{\pgfpoint{259.334229pt}{352.062195pt}}
\pgflineto{\pgfpoint{259.249939pt}{352.120972pt}}
\pgfusepath{stroke}
\pgfpathmoveto{\pgfpoint{259.302063pt}{352.159790pt}}
\pgflineto{\pgfpoint{259.334229pt}{352.062195pt}}
\pgfusepath{stroke}
\pgfpathmoveto{\pgfpoint{259.323273pt}{358.052917pt}}
\pgflineto{\pgfpoint{259.242981pt}{358.111938pt}}
\pgfusepath{stroke}
\pgfpathmoveto{\pgfpoint{259.294434pt}{358.148315pt}}
\pgflineto{\pgfpoint{259.323273pt}{358.052917pt}}
\pgfusepath{stroke}
\pgfpathmoveto{\pgfpoint{259.312988pt}{364.044037pt}}
\pgflineto{\pgfpoint{259.236511pt}{364.103058pt}}
\pgfusepath{stroke}
\pgfpathmoveto{\pgfpoint{259.287201pt}{364.137146pt}}
\pgflineto{\pgfpoint{259.312988pt}{364.044037pt}}
\pgfusepath{stroke}
\pgfpathmoveto{\pgfpoint{259.303375pt}{370.035400pt}}
\pgflineto{\pgfpoint{259.230499pt}{370.094299pt}}
\pgfusepath{stroke}
\pgfpathmoveto{\pgfpoint{259.280396pt}{370.126221pt}}
\pgflineto{\pgfpoint{259.303375pt}{370.035400pt}}
\pgfusepath{stroke}
\pgfpathmoveto{\pgfpoint{265.211365pt}{77.066696pt}}
\pgflineto{\pgfpoint{265.212982pt}{76.990677pt}}
\pgfusepath{stroke}
\pgfpathmoveto{\pgfpoint{265.167053pt}{77.004913pt}}
\pgflineto{\pgfpoint{265.211365pt}{77.066696pt}}
\pgfusepath{stroke}
\pgfpathmoveto{\pgfpoint{265.215515pt}{83.060883pt}}
\pgflineto{\pgfpoint{265.216461pt}{82.982269pt}}
\pgfusepath{stroke}
\pgfpathmoveto{\pgfpoint{265.169098pt}{82.997406pt}}
\pgflineto{\pgfpoint{265.215515pt}{83.060883pt}}
\pgfusepath{stroke}
\pgfpathmoveto{\pgfpoint{265.220062pt}{89.055458pt}}
\pgflineto{\pgfpoint{265.220276pt}{88.974052pt}}
\pgfusepath{stroke}
\pgfpathmoveto{\pgfpoint{265.171387pt}{88.990227pt}}
\pgflineto{\pgfpoint{265.220062pt}{89.055458pt}}
\pgfusepath{stroke}
\pgfpathmoveto{\pgfpoint{265.225128pt}{95.050476pt}}
\pgflineto{\pgfpoint{265.224426pt}{94.966087pt}}
\pgfusepath{stroke}
\pgfpathmoveto{\pgfpoint{265.173950pt}{94.983376pt}}
\pgflineto{\pgfpoint{265.225128pt}{95.050476pt}}
\pgfusepath{stroke}
\pgfpathmoveto{\pgfpoint{265.230743pt}{101.045952pt}}
\pgflineto{\pgfpoint{265.229034pt}{100.958374pt}}
\pgfusepath{stroke}
\pgfpathmoveto{\pgfpoint{265.176819pt}{100.976906pt}}
\pgflineto{\pgfpoint{265.230743pt}{101.045952pt}}
\pgfusepath{stroke}
\pgfpathmoveto{\pgfpoint{265.237000pt}{107.041946pt}}
\pgflineto{\pgfpoint{265.234131pt}{106.950928pt}}
\pgfusepath{stroke}
\pgfpathmoveto{\pgfpoint{265.180084pt}{106.970840pt}}
\pgflineto{\pgfpoint{265.237000pt}{107.041946pt}}
\pgfusepath{stroke}
\pgfpathmoveto{\pgfpoint{265.244019pt}{113.038490pt}}
\pgflineto{\pgfpoint{265.239807pt}{112.943756pt}}
\pgfusepath{stroke}
\pgfpathmoveto{\pgfpoint{265.183807pt}{112.965248pt}}
\pgflineto{\pgfpoint{265.244019pt}{113.038490pt}}
\pgfusepath{stroke}
\pgfpathmoveto{\pgfpoint{265.251923pt}{119.035652pt}}
\pgflineto{\pgfpoint{265.246124pt}{118.936897pt}}
\pgfusepath{stroke}
\pgfpathmoveto{\pgfpoint{265.188019pt}{118.960144pt}}
\pgflineto{\pgfpoint{265.251923pt}{119.035652pt}}
\pgfusepath{stroke}
\pgfpathmoveto{\pgfpoint{265.260895pt}{125.033455pt}}
\pgflineto{\pgfpoint{265.253204pt}{124.930367pt}}
\pgfusepath{stroke}
\pgfpathmoveto{\pgfpoint{265.192902pt}{124.955605pt}}
\pgflineto{\pgfpoint{265.260895pt}{125.033455pt}}
\pgfusepath{stroke}
\pgfpathmoveto{\pgfpoint{265.271118pt}{131.031982pt}}
\pgflineto{\pgfpoint{265.261230pt}{130.924164pt}}
\pgfusepath{stroke}
\pgfpathmoveto{\pgfpoint{265.198517pt}{130.951660pt}}
\pgflineto{\pgfpoint{265.271118pt}{131.031982pt}}
\pgfusepath{stroke}
\pgfpathmoveto{\pgfpoint{265.282867pt}{137.031250pt}}
\pgflineto{\pgfpoint{265.270325pt}{136.918274pt}}
\pgfusepath{stroke}
\pgfpathmoveto{\pgfpoint{265.205078pt}{136.948395pt}}
\pgflineto{\pgfpoint{265.282867pt}{137.031250pt}}
\pgfusepath{stroke}
\pgfpathmoveto{\pgfpoint{265.296448pt}{143.031296pt}}
\pgflineto{\pgfpoint{265.280762pt}{142.912720pt}}
\pgfusepath{stroke}
\pgfpathmoveto{\pgfpoint{265.212738pt}{142.945862pt}}
\pgflineto{\pgfpoint{265.296448pt}{143.031296pt}}
\pgfusepath{stroke}
\pgfpathmoveto{\pgfpoint{265.312286pt}{149.032135pt}}
\pgflineto{\pgfpoint{265.292755pt}{148.907440pt}}
\pgfusepath{stroke}
\pgfpathmoveto{\pgfpoint{265.221863pt}{148.944092pt}}
\pgflineto{\pgfpoint{265.312286pt}{149.032135pt}}
\pgfusepath{stroke}
\pgfpathmoveto{\pgfpoint{265.330872pt}{155.033691pt}}
\pgflineto{\pgfpoint{265.306671pt}{154.902344pt}}
\pgfusepath{stroke}
\pgfpathmoveto{\pgfpoint{265.232697pt}{154.943130pt}}
\pgflineto{\pgfpoint{265.330872pt}{155.033691pt}}
\pgfusepath{stroke}
\pgfpathmoveto{\pgfpoint{265.352905pt}{161.035889pt}}
\pgflineto{\pgfpoint{265.322968pt}{160.897278pt}}
\pgfusepath{stroke}
\pgfpathmoveto{\pgfpoint{265.245789pt}{160.942947pt}}
\pgflineto{\pgfpoint{265.352905pt}{161.035889pt}}
\pgfusepath{stroke}
\pgfpathmoveto{\pgfpoint{265.379211pt}{167.038391pt}}
\pgflineto{\pgfpoint{265.342163pt}{166.891937pt}}
\pgfusepath{stroke}
\pgfpathmoveto{\pgfpoint{265.261688pt}{166.943466pt}}
\pgflineto{\pgfpoint{265.379211pt}{167.038391pt}}
\pgfusepath{stroke}
\pgfpathmoveto{\pgfpoint{265.410889pt}{173.040649pt}}
\pgflineto{\pgfpoint{265.364899pt}{172.885834pt}}
\pgfusepath{stroke}
\pgfpathmoveto{\pgfpoint{265.281189pt}{172.944412pt}}
\pgflineto{\pgfpoint{265.410889pt}{173.040649pt}}
\pgfusepath{stroke}
\pgfpathmoveto{\pgfpoint{265.449249pt}{179.041656pt}}
\pgflineto{\pgfpoint{265.391907pt}{178.878174pt}}
\pgfusepath{stroke}
\pgfpathmoveto{\pgfpoint{265.305298pt}{178.945251pt}}
\pgflineto{\pgfpoint{265.449249pt}{179.041656pt}}
\pgfusepath{stroke}
\pgfpathmoveto{\pgfpoint{265.495728pt}{185.039658pt}}
\pgflineto{\pgfpoint{265.424042pt}{184.867584pt}}
\pgfusepath{stroke}
\pgfpathmoveto{\pgfpoint{265.335144pt}{184.945007pt}}
\pgflineto{\pgfpoint{265.495728pt}{185.039658pt}}
\pgfusepath{stroke}
\pgfpathmoveto{\pgfpoint{265.551727pt}{191.031723pt}}
\pgflineto{\pgfpoint{265.461853pt}{190.852005pt}}
\pgfusepath{stroke}
\pgfpathmoveto{\pgfpoint{265.372009pt}{190.941864pt}}
\pgflineto{\pgfpoint{265.551727pt}{191.031723pt}}
\pgfusepath{stroke}
\pgfpathmoveto{\pgfpoint{265.617859pt}{197.013351pt}}
\pgflineto{\pgfpoint{265.505249pt}{196.828323pt}}
\pgfusepath{stroke}
\pgfpathmoveto{\pgfpoint{265.416748pt}{196.932877pt}}
\pgflineto{\pgfpoint{265.617859pt}{197.013351pt}}
\pgfusepath{stroke}
\pgfpathmoveto{\pgfpoint{265.692596pt}{202.978195pt}}
\pgflineto{\pgfpoint{265.552551pt}{202.792496pt}}
\pgfusepath{stroke}
\pgfpathmoveto{\pgfpoint{265.469147pt}{202.913666pt}}
\pgflineto{\pgfpoint{265.692596pt}{202.978195pt}}
\pgfusepath{stroke}
\pgfpathmoveto{\pgfpoint{265.770294pt}{208.919144pt}}
\pgflineto{\pgfpoint{265.599152pt}{208.740417pt}}
\pgfusepath{stroke}
\pgfpathmoveto{\pgfpoint{265.526123pt}{208.878845pt}}
\pgflineto{\pgfpoint{265.770294pt}{208.919144pt}}
\pgfusepath{stroke}
\pgfpathmoveto{\pgfpoint{265.839447pt}{214.831573pt}}
\pgflineto{\pgfpoint{265.636871pt}{214.670273pt}}
\pgfusepath{stroke}
\pgfpathmoveto{\pgfpoint{265.580597pt}{214.824066pt}}
\pgflineto{\pgfpoint{265.839447pt}{214.831573pt}}
\pgfusepath{stroke}
\pgfpathmoveto{\pgfpoint{265.884552pt}{220.718826pt}}
\pgflineto{\pgfpoint{265.655762pt}{220.586014pt}}
\pgfusepath{stroke}
\pgfpathmoveto{\pgfpoint{265.621826pt}{220.749832pt}}
\pgflineto{\pgfpoint{265.884552pt}{220.718826pt}}
\pgfusepath{stroke}
\pgfpathmoveto{\pgfpoint{265.893555pt}{226.595154pt}}
\pgflineto{\pgfpoint{265.649384pt}{226.498474pt}}
\pgfusepath{stroke}
\pgfpathmoveto{\pgfpoint{265.640228pt}{226.664307pt}}
\pgflineto{\pgfpoint{265.893555pt}{226.595154pt}}
\pgfusepath{stroke}
\pgfpathmoveto{\pgfpoint{265.866302pt}{232.480240pt}}
\pgflineto{\pgfpoint{265.619812pt}{232.420822pt}}
\pgfusepath{stroke}
\pgfpathmoveto{\pgfpoint{265.633453pt}{232.580597pt}}
\pgflineto{\pgfpoint{265.866302pt}{232.480240pt}}
\pgfusepath{stroke}
\pgfpathmoveto{\pgfpoint{265.814575pt}{238.388382pt}}
\pgflineto{\pgfpoint{265.576508pt}{238.361237pt}}
\pgfusepath{stroke}
\pgfpathmoveto{\pgfpoint{265.607849pt}{238.509521pt}}
\pgflineto{\pgfpoint{265.814575pt}{238.388382pt}}
\pgfusepath{stroke}
\pgfpathmoveto{\pgfpoint{265.753632pt}{244.322968pt}}
\pgflineto{\pgfpoint{265.529968pt}{244.320328pt}}
\pgfusepath{stroke}
\pgfpathmoveto{\pgfpoint{265.573120pt}{244.455048pt}}
\pgflineto{\pgfpoint{265.753632pt}{244.322968pt}}
\pgfusepath{stroke}
\pgfpathmoveto{\pgfpoint{265.694641pt}{250.279465pt}}
\pgflineto{\pgfpoint{265.487152pt}{250.293793pt}}
\pgfusepath{stroke}
\pgfpathmoveto{\pgfpoint{265.537262pt}{250.415421pt}}
\pgflineto{\pgfpoint{265.694641pt}{250.279465pt}}
\pgfusepath{stroke}
\pgfpathmoveto{\pgfpoint{265.643005pt}{256.250885pt}}
\pgflineto{\pgfpoint{265.450928pt}{256.276398pt}}
\pgfusepath{stroke}
\pgfpathmoveto{\pgfpoint{265.504639pt}{256.386536pt}}
\pgflineto{\pgfpoint{265.643005pt}{256.250885pt}}
\pgfusepath{stroke}
\pgfpathmoveto{\pgfpoint{265.599915pt}{262.231140pt}}
\pgflineto{\pgfpoint{265.421356pt}{262.263916pt}}
\pgfusepath{stroke}
\pgfpathmoveto{\pgfpoint{265.476746pt}{262.364502pt}}
\pgflineto{\pgfpoint{265.599915pt}{262.231140pt}}
\pgfusepath{stroke}
\pgfpathmoveto{\pgfpoint{265.564514pt}{268.215881pt}}
\pgflineto{\pgfpoint{265.397430pt}{268.253632pt}}
\pgfusepath{stroke}
\pgfpathmoveto{\pgfpoint{265.453491pt}{268.346344pt}}
\pgflineto{\pgfpoint{265.564514pt}{268.215881pt}}
\pgfusepath{stroke}
\pgfpathmoveto{\pgfpoint{265.535248pt}{274.202606pt}}
\pgflineto{\pgfpoint{265.377808pt}{274.243958pt}}
\pgfusepath{stroke}
\pgfpathmoveto{\pgfpoint{265.434082pt}{274.330139pt}}
\pgflineto{\pgfpoint{265.535248pt}{274.202606pt}}
\pgfusepath{stroke}
\pgfpathmoveto{\pgfpoint{265.510468pt}{280.189880pt}}
\pgflineto{\pgfpoint{265.361237pt}{280.234131pt}}
\pgfusepath{stroke}
\pgfpathmoveto{\pgfpoint{265.417603pt}{280.314819pt}}
\pgflineto{\pgfpoint{265.510468pt}{280.189880pt}}
\pgfusepath{stroke}
\pgfpathmoveto{\pgfpoint{265.488831pt}{286.177124pt}}
\pgflineto{\pgfpoint{265.346741pt}{286.223938pt}}
\pgfusepath{stroke}
\pgfpathmoveto{\pgfpoint{265.403259pt}{286.299835pt}}
\pgflineto{\pgfpoint{265.488831pt}{286.177124pt}}
\pgfusepath{stroke}
\pgfpathmoveto{\pgfpoint{265.469330pt}{292.164154pt}}
\pgflineto{\pgfpoint{265.333679pt}{292.213379pt}}
\pgfusepath{stroke}
\pgfpathmoveto{\pgfpoint{265.390350pt}{292.284943pt}}
\pgflineto{\pgfpoint{265.469330pt}{292.164154pt}}
\pgfusepath{stroke}
\pgfpathmoveto{\pgfpoint{265.451233pt}{298.151031pt}}
\pgflineto{\pgfpoint{265.321503pt}{298.202576pt}}
\pgfusepath{stroke}
\pgfpathmoveto{\pgfpoint{265.378357pt}{298.270081pt}}
\pgflineto{\pgfpoint{265.451233pt}{298.151031pt}}
\pgfusepath{stroke}
\pgfpathmoveto{\pgfpoint{265.434082pt}{304.137939pt}}
\pgflineto{\pgfpoint{265.309967pt}{304.191681pt}}
\pgfusepath{stroke}
\pgfpathmoveto{\pgfpoint{265.367004pt}{304.255402pt}}
\pgflineto{\pgfpoint{265.434082pt}{304.137939pt}}
\pgfusepath{stroke}
\pgfpathmoveto{\pgfpoint{265.417572pt}{310.125153pt}}
\pgflineto{\pgfpoint{265.298920pt}{310.180878pt}}
\pgfusepath{stroke}
\pgfpathmoveto{\pgfpoint{265.356079pt}{310.240936pt}}
\pgflineto{\pgfpoint{265.417572pt}{310.125153pt}}
\pgfusepath{stroke}
\pgfpathmoveto{\pgfpoint{265.401672pt}{316.112732pt}}
\pgflineto{\pgfpoint{265.288269pt}{316.170288pt}}
\pgfusepath{stroke}
\pgfpathmoveto{\pgfpoint{265.345490pt}{316.226807pt}}
\pgflineto{\pgfpoint{265.401672pt}{316.112732pt}}
\pgfusepath{stroke}
\pgfpathmoveto{\pgfpoint{265.386292pt}{322.100891pt}}
\pgflineto{\pgfpoint{265.278076pt}{322.160034pt}}
\pgfusepath{stroke}
\pgfpathmoveto{\pgfpoint{265.335205pt}{322.213135pt}}
\pgflineto{\pgfpoint{265.386292pt}{322.100891pt}}
\pgfusepath{stroke}
\pgfpathmoveto{\pgfpoint{265.371460pt}{328.089661pt}}
\pgflineto{\pgfpoint{265.268311pt}{328.150085pt}}
\pgfusepath{stroke}
\pgfpathmoveto{\pgfpoint{265.325195pt}{328.199921pt}}
\pgflineto{\pgfpoint{265.371460pt}{328.089661pt}}
\pgfusepath{stroke}
\pgfpathmoveto{\pgfpoint{265.357239pt}{334.079071pt}}
\pgflineto{\pgfpoint{265.259033pt}{334.140533pt}}
\pgfusepath{stroke}
\pgfpathmoveto{\pgfpoint{265.315552pt}{334.187195pt}}
\pgflineto{\pgfpoint{265.357239pt}{334.079071pt}}
\pgfusepath{stroke}
\pgfpathmoveto{\pgfpoint{265.343719pt}{340.069092pt}}
\pgflineto{\pgfpoint{265.250244pt}{340.131287pt}}
\pgfusepath{stroke}
\pgfpathmoveto{\pgfpoint{265.306274pt}{340.174927pt}}
\pgflineto{\pgfpoint{265.343719pt}{340.069092pt}}
\pgfusepath{stroke}
\pgfpathmoveto{\pgfpoint{265.330902pt}{346.059631pt}}
\pgflineto{\pgfpoint{265.242004pt}{346.122375pt}}
\pgfusepath{stroke}
\pgfpathmoveto{\pgfpoint{265.297424pt}{346.163147pt}}
\pgflineto{\pgfpoint{265.330902pt}{346.059631pt}}
\pgfusepath{stroke}
\pgfpathmoveto{\pgfpoint{265.318817pt}{352.050690pt}}
\pgflineto{\pgfpoint{265.234314pt}{352.113647pt}}
\pgfusepath{stroke}
\pgfpathmoveto{\pgfpoint{265.289001pt}{352.151764pt}}
\pgflineto{\pgfpoint{265.318817pt}{352.050690pt}}
\pgfusepath{stroke}
\pgfpathmoveto{\pgfpoint{265.307495pt}{358.042175pt}}
\pgflineto{\pgfpoint{265.227173pt}{358.105133pt}}
\pgfusepath{stroke}
\pgfpathmoveto{\pgfpoint{265.281036pt}{358.140747pt}}
\pgflineto{\pgfpoint{265.307495pt}{358.042175pt}}
\pgfusepath{stroke}
\pgfpathmoveto{\pgfpoint{265.296936pt}{364.033966pt}}
\pgflineto{\pgfpoint{265.220581pt}{364.096771pt}}
\pgfusepath{stroke}
\pgfpathmoveto{\pgfpoint{265.273529pt}{364.130005pt}}
\pgflineto{\pgfpoint{265.296936pt}{364.033966pt}}
\pgfusepath{stroke}
\pgfpathmoveto{\pgfpoint{265.287079pt}{370.026001pt}}
\pgflineto{\pgfpoint{265.214508pt}{370.088501pt}}
\pgfusepath{stroke}
\pgfpathmoveto{\pgfpoint{265.266510pt}{370.119537pt}}
\pgflineto{\pgfpoint{265.287079pt}{370.026001pt}}
\pgfusepath{stroke}
\pgfpathmoveto{\pgfpoint{271.192688pt}{77.070435pt}}
\pgflineto{\pgfpoint{271.196899pt}{76.993881pt}}
\pgfusepath{stroke}
\pgfpathmoveto{\pgfpoint{271.150116pt}{77.006714pt}}
\pgflineto{\pgfpoint{271.192688pt}{77.070435pt}}
\pgfusepath{stroke}
\pgfpathmoveto{\pgfpoint{271.196442pt}{83.065002pt}}
\pgflineto{\pgfpoint{271.200134pt}{82.985764pt}}
\pgfusepath{stroke}
\pgfpathmoveto{\pgfpoint{271.151855pt}{82.999420pt}}
\pgflineto{\pgfpoint{271.196442pt}{83.065002pt}}
\pgfusepath{stroke}
\pgfpathmoveto{\pgfpoint{271.200562pt}{89.060028pt}}
\pgflineto{\pgfpoint{271.203674pt}{88.977913pt}}
\pgfusepath{stroke}
\pgfpathmoveto{\pgfpoint{271.153778pt}{88.992462pt}}
\pgflineto{\pgfpoint{271.200562pt}{89.060028pt}}
\pgfusepath{stroke}
\pgfpathmoveto{\pgfpoint{271.205139pt}{95.055542pt}}
\pgflineto{\pgfpoint{271.207581pt}{94.970322pt}}
\pgfusepath{stroke}
\pgfpathmoveto{\pgfpoint{271.155975pt}{94.985901pt}}
\pgflineto{\pgfpoint{271.205139pt}{95.055542pt}}
\pgfusepath{stroke}
\pgfpathmoveto{\pgfpoint{271.210205pt}{101.051605pt}}
\pgflineto{\pgfpoint{271.211884pt}{100.963066pt}}
\pgfusepath{stroke}
\pgfpathmoveto{\pgfpoint{271.158447pt}{100.979752pt}}
\pgflineto{\pgfpoint{271.210205pt}{101.051605pt}}
\pgfusepath{stroke}
\pgfpathmoveto{\pgfpoint{271.215881pt}{107.048271pt}}
\pgflineto{\pgfpoint{271.216675pt}{106.956131pt}}
\pgfusepath{stroke}
\pgfpathmoveto{\pgfpoint{271.161224pt}{106.974075pt}}
\pgflineto{\pgfpoint{271.215881pt}{107.048271pt}}
\pgfusepath{stroke}
\pgfpathmoveto{\pgfpoint{271.222260pt}{113.045609pt}}
\pgflineto{\pgfpoint{271.221985pt}{112.949562pt}}
\pgfusepath{stroke}
\pgfpathmoveto{\pgfpoint{271.164398pt}{112.968925pt}}
\pgflineto{\pgfpoint{271.222260pt}{113.045609pt}}
\pgfusepath{stroke}
\pgfpathmoveto{\pgfpoint{271.229462pt}{119.043709pt}}
\pgflineto{\pgfpoint{271.227936pt}{118.943405pt}}
\pgfusepath{stroke}
\pgfpathmoveto{\pgfpoint{271.168060pt}{118.964378pt}}
\pgflineto{\pgfpoint{271.229462pt}{119.043709pt}}
\pgfusepath{stroke}
\pgfpathmoveto{\pgfpoint{271.237640pt}{125.042633pt}}
\pgflineto{\pgfpoint{271.234650pt}{124.937706pt}}
\pgfusepath{stroke}
\pgfpathmoveto{\pgfpoint{271.172272pt}{124.960487pt}}
\pgflineto{\pgfpoint{271.237640pt}{125.042633pt}}
\pgfusepath{stroke}
\pgfpathmoveto{\pgfpoint{271.247009pt}{131.042496pt}}
\pgflineto{\pgfpoint{271.242218pt}{130.932495pt}}
\pgfusepath{stroke}
\pgfpathmoveto{\pgfpoint{271.177185pt}{130.957352pt}}
\pgflineto{\pgfpoint{271.247009pt}{131.042496pt}}
\pgfusepath{stroke}
\pgfpathmoveto{\pgfpoint{271.257843pt}{137.043396pt}}
\pgflineto{\pgfpoint{271.250916pt}{136.927795pt}}
\pgfusepath{stroke}
\pgfpathmoveto{\pgfpoint{271.182922pt}{136.955078pt}}
\pgflineto{\pgfpoint{271.257843pt}{137.043396pt}}
\pgfusepath{stroke}
\pgfpathmoveto{\pgfpoint{271.270416pt}{143.045441pt}}
\pgflineto{\pgfpoint{271.260864pt}{142.923691pt}}
\pgfusepath{stroke}
\pgfpathmoveto{\pgfpoint{271.189728pt}{142.953766pt}}
\pgflineto{\pgfpoint{271.270416pt}{143.045441pt}}
\pgfusepath{stroke}
\pgfpathmoveto{\pgfpoint{271.285187pt}{149.048767pt}}
\pgflineto{\pgfpoint{271.272461pt}{148.920181pt}}
\pgfusepath{stroke}
\pgfpathmoveto{\pgfpoint{271.197845pt}{148.953537pt}}
\pgflineto{\pgfpoint{271.285187pt}{149.048767pt}}
\pgfusepath{stroke}
\pgfpathmoveto{\pgfpoint{271.302734pt}{155.053436pt}}
\pgflineto{\pgfpoint{271.286041pt}{154.917267pt}}
\pgfusepath{stroke}
\pgfpathmoveto{\pgfpoint{271.207672pt}{154.954529pt}}
\pgflineto{\pgfpoint{271.302734pt}{155.053436pt}}
\pgfusepath{stroke}
\pgfpathmoveto{\pgfpoint{271.323792pt}{161.059555pt}}
\pgflineto{\pgfpoint{271.302124pt}{160.914917pt}}
\pgfusepath{stroke}
\pgfpathmoveto{\pgfpoint{271.219666pt}{160.956863pt}}
\pgflineto{\pgfpoint{271.323792pt}{161.059555pt}}
\pgfusepath{stroke}
\pgfpathmoveto{\pgfpoint{271.349426pt}{167.067078pt}}
\pgflineto{\pgfpoint{271.321411pt}{166.913010pt}}
\pgfusepath{stroke}
\pgfpathmoveto{\pgfpoint{271.234558pt}{166.960648pt}}
\pgflineto{\pgfpoint{271.349426pt}{167.067078pt}}
\pgfusepath{stroke}
\pgfpathmoveto{\pgfpoint{271.381012pt}{173.075760pt}}
\pgflineto{\pgfpoint{271.344788pt}{172.911209pt}}
\pgfusepath{stroke}
\pgfpathmoveto{\pgfpoint{271.253296pt}{172.965851pt}}
\pgflineto{\pgfpoint{271.381012pt}{173.075760pt}}
\pgfusepath{stroke}
\pgfpathmoveto{\pgfpoint{271.420441pt}{179.084976pt}}
\pgflineto{\pgfpoint{271.373444pt}{178.908890pt}}
\pgfusepath{stroke}
\pgfpathmoveto{\pgfpoint{271.277191pt}{178.972305pt}}
\pgflineto{\pgfpoint{271.420441pt}{179.084976pt}}
\pgfusepath{stroke}
\pgfpathmoveto{\pgfpoint{271.470184pt}{185.093307pt}}
\pgflineto{\pgfpoint{271.408905pt}{184.904831pt}}
\pgfusepath{stroke}
\pgfpathmoveto{\pgfpoint{271.308075pt}{184.979309pt}}
\pgflineto{\pgfpoint{271.470184pt}{185.093307pt}}
\pgfusepath{stroke}
\pgfpathmoveto{\pgfpoint{271.533356pt}{191.097855pt}}
\pgflineto{\pgfpoint{271.452881pt}{190.896759pt}}
\pgfusepath{stroke}
\pgfpathmoveto{\pgfpoint{271.348328pt}{190.985260pt}}
\pgflineto{\pgfpoint{271.533356pt}{191.097855pt}}
\pgfusepath{stroke}
\pgfpathmoveto{\pgfpoint{271.613068pt}{197.093063pt}}
\pgflineto{\pgfpoint{271.506836pt}{196.880630pt}}
\pgfusepath{stroke}
\pgfpathmoveto{\pgfpoint{271.400635pt}{196.986847pt}}
\pgflineto{\pgfpoint{271.613068pt}{197.093063pt}}
\pgfusepath{stroke}
\pgfpathmoveto{\pgfpoint{271.710815pt}{203.069336pt}}
\pgflineto{\pgfpoint{271.570740pt}{202.849869pt}}
\pgfusepath{stroke}
\pgfpathmoveto{\pgfpoint{271.467102pt}{202.977814pt}}
\pgflineto{\pgfpoint{271.710815pt}{203.069336pt}}
\pgfusepath{stroke}
\pgfpathmoveto{\pgfpoint{271.822235pt}{209.012405pt}}
\pgflineto{\pgfpoint{271.640045pt}{208.795456pt}}
\pgfusepath{stroke}
\pgfpathmoveto{\pgfpoint{271.546326pt}{208.948151pt}}
\pgflineto{\pgfpoint{271.822235pt}{209.012405pt}}
\pgfusepath{stroke}
\pgfpathmoveto{\pgfpoint{271.930359pt}{214.907806pt}}
\pgflineto{\pgfpoint{271.701874pt}{214.709625pt}}
\pgfusepath{stroke}
\pgfpathmoveto{\pgfpoint{271.628662pt}{214.886353pt}}
\pgflineto{\pgfpoint{271.930359pt}{214.907806pt}}
\pgfusepath{stroke}
\pgfpathmoveto{\pgfpoint{272.003998pt}{220.754440pt}}
\pgflineto{\pgfpoint{271.735291pt}{220.595123pt}}
\pgfusepath{stroke}
\pgfpathmoveto{\pgfpoint{271.693451pt}{220.788208pt}}
\pgflineto{\pgfpoint{272.003998pt}{220.754440pt}}
\pgfusepath{stroke}
\pgfpathmoveto{\pgfpoint{272.014160pt}{226.578644pt}}
\pgflineto{\pgfpoint{271.723877pt}{226.472717pt}}
\pgfusepath{stroke}
\pgfpathmoveto{\pgfpoint{271.718384pt}{226.668060pt}}
\pgflineto{\pgfpoint{272.014160pt}{226.578644pt}}
\pgfusepath{stroke}
\pgfpathmoveto{\pgfpoint{271.960663pt}{232.422836pt}}
\pgflineto{\pgfpoint{271.672272pt}{232.370697pt}}
\pgfusepath{stroke}
\pgfpathmoveto{\pgfpoint{271.698669pt}{232.554153pt}}
\pgflineto{\pgfpoint{271.960663pt}{232.422836pt}}
\pgfusepath{stroke}
\pgfpathmoveto{\pgfpoint{271.872223pt}{238.313354pt}}
\pgflineto{\pgfpoint{271.602509pt}{238.303467pt}}
\pgfusepath{stroke}
\pgfpathmoveto{\pgfpoint{271.650513pt}{238.467270pt}}
\pgflineto{\pgfpoint{271.872223pt}{238.313354pt}}
\pgfusepath{stroke}
\pgfpathmoveto{\pgfpoint{271.779755pt}{244.249176pt}}
\pgflineto{\pgfpoint{271.535095pt}{244.266861pt}}
\pgfusepath{stroke}
\pgfpathmoveto{\pgfpoint{271.594635pt}{244.410126pt}}
\pgflineto{\pgfpoint{271.779755pt}{244.249176pt}}
\pgfusepath{stroke}
\pgfpathmoveto{\pgfpoint{271.699829pt}{250.215897pt}}
\pgflineto{\pgfpoint{271.479492pt}{250.249481pt}}
\pgfusepath{stroke}
\pgfpathmoveto{\pgfpoint{271.543701pt}{250.374969pt}}
\pgflineto{\pgfpoint{271.699829pt}{250.215897pt}}
\pgfusepath{stroke}
\pgfpathmoveto{\pgfpoint{271.636505pt}{256.199371pt}}
\pgflineto{\pgfpoint{271.436768pt}{256.241425pt}}
\pgfusepath{stroke}
\pgfpathmoveto{\pgfpoint{271.501953pt}{256.352844pt}}
\pgflineto{\pgfpoint{271.636505pt}{256.199371pt}}
\pgfusepath{stroke}
\pgfpathmoveto{\pgfpoint{271.587921pt}{262.190247pt}}
\pgflineto{\pgfpoint{271.404755pt}{262.236633pt}}
\pgfusepath{stroke}
\pgfpathmoveto{\pgfpoint{271.469208pt}{262.337280pt}}
\pgflineto{\pgfpoint{271.587921pt}{262.190247pt}}
\pgfusepath{stroke}
\pgfpathmoveto{\pgfpoint{271.550598pt}{268.183228pt}}
\pgflineto{\pgfpoint{271.380432pt}{268.232056pt}}
\pgfusepath{stroke}
\pgfpathmoveto{\pgfpoint{271.443756pt}{268.324371pt}}
\pgflineto{\pgfpoint{271.550598pt}{268.183228pt}}
\pgfusepath{stroke}
\pgfpathmoveto{\pgfpoint{271.521210pt}{274.175873pt}}
\pgflineto{\pgfpoint{271.361389pt}{274.226318pt}}
\pgfusepath{stroke}
\pgfpathmoveto{\pgfpoint{271.423615pt}{274.312134pt}}
\pgflineto{\pgfpoint{271.521210pt}{274.175873pt}}
\pgfusepath{stroke}
\pgfpathmoveto{\pgfpoint{271.497070pt}{280.167175pt}}
\pgflineto{\pgfpoint{271.345673pt}{280.219116pt}}
\pgfusepath{stroke}
\pgfpathmoveto{\pgfpoint{271.407135pt}{280.299561pt}}
\pgflineto{\pgfpoint{271.497070pt}{280.167175pt}}
\pgfusepath{stroke}
\pgfpathmoveto{\pgfpoint{271.476196pt}{286.156982pt}}
\pgflineto{\pgfpoint{271.332001pt}{286.210571pt}}
\pgfusepath{stroke}
\pgfpathmoveto{\pgfpoint{271.392975pt}{286.286377pt}}
\pgflineto{\pgfpoint{271.476196pt}{286.156982pt}}
\pgfusepath{stroke}
\pgfpathmoveto{\pgfpoint{271.457275pt}{292.145630pt}}
\pgflineto{\pgfpoint{271.319489pt}{292.200989pt}}
\pgfusepath{stroke}
\pgfpathmoveto{\pgfpoint{271.380280pt}{292.272583pt}}
\pgflineto{\pgfpoint{271.457275pt}{292.145630pt}}
\pgfusepath{stroke}
\pgfpathmoveto{\pgfpoint{271.439453pt}{298.133484pt}}
\pgflineto{\pgfpoint{271.307617pt}{298.190826pt}}
\pgfusepath{stroke}
\pgfpathmoveto{\pgfpoint{271.368378pt}{298.258453pt}}
\pgflineto{\pgfpoint{271.439453pt}{298.133484pt}}
\pgfusepath{stroke}
\pgfpathmoveto{\pgfpoint{271.422272pt}{304.121063pt}}
\pgflineto{\pgfpoint{271.296143pt}{304.180359pt}}
\pgfusepath{stroke}
\pgfpathmoveto{\pgfpoint{271.356934pt}{304.244171pt}}
\pgflineto{\pgfpoint{271.422272pt}{304.121063pt}}
\pgfusepath{stroke}
\pgfpathmoveto{\pgfpoint{271.405487pt}{310.108734pt}}
\pgflineto{\pgfpoint{271.284943pt}{310.169922pt}}
\pgfusepath{stroke}
\pgfpathmoveto{\pgfpoint{271.345764pt}{310.230011pt}}
\pgflineto{\pgfpoint{271.405487pt}{310.108734pt}}
\pgfusepath{stroke}
\pgfpathmoveto{\pgfpoint{271.389069pt}{316.096802pt}}
\pgflineto{\pgfpoint{271.274017pt}{316.159729pt}}
\pgfusepath{stroke}
\pgfpathmoveto{\pgfpoint{271.334778pt}{316.216156pt}}
\pgflineto{\pgfpoint{271.389069pt}{316.096802pt}}
\pgfusepath{stroke}
\pgfpathmoveto{\pgfpoint{271.373047pt}{322.085480pt}}
\pgflineto{\pgfpoint{271.263458pt}{322.149841pt}}
\pgfusepath{stroke}
\pgfpathmoveto{\pgfpoint{271.324005pt}{322.202728pt}}
\pgflineto{\pgfpoint{271.373047pt}{322.085480pt}}
\pgfusepath{stroke}
\pgfpathmoveto{\pgfpoint{271.357574pt}{328.074829pt}}
\pgflineto{\pgfpoint{271.253326pt}{328.140381pt}}
\pgfusepath{stroke}
\pgfpathmoveto{\pgfpoint{271.313507pt}{328.189819pt}}
\pgflineto{\pgfpoint{271.357574pt}{328.074829pt}}
\pgfusepath{stroke}
\pgfpathmoveto{\pgfpoint{271.342712pt}{334.064880pt}}
\pgflineto{\pgfpoint{271.243652pt}{334.131348pt}}
\pgfusepath{stroke}
\pgfpathmoveto{\pgfpoint{271.303345pt}{334.177490pt}}
\pgflineto{\pgfpoint{271.342712pt}{334.064880pt}}
\pgfusepath{stroke}
\pgfpathmoveto{\pgfpoint{271.328552pt}{340.055603pt}}
\pgflineto{\pgfpoint{271.234558pt}{340.122650pt}}
\pgfusepath{stroke}
\pgfpathmoveto{\pgfpoint{271.293579pt}{340.165649pt}}
\pgflineto{\pgfpoint{271.328552pt}{340.055603pt}}
\pgfusepath{stroke}
\pgfpathmoveto{\pgfpoint{271.315216pt}{346.046936pt}}
\pgflineto{\pgfpoint{271.226044pt}{346.114258pt}}
\pgfusepath{stroke}
\pgfpathmoveto{\pgfpoint{271.284271pt}{346.154297pt}}
\pgflineto{\pgfpoint{271.315216pt}{346.046936pt}}
\pgfusepath{stroke}
\pgfpathmoveto{\pgfpoint{271.302673pt}{352.038788pt}}
\pgflineto{\pgfpoint{271.218140pt}{352.106140pt}}
\pgfusepath{stroke}
\pgfpathmoveto{\pgfpoint{271.275482pt}{352.143372pt}}
\pgflineto{\pgfpoint{271.302673pt}{352.038788pt}}
\pgfusepath{stroke}
\pgfpathmoveto{\pgfpoint{271.290985pt}{358.031036pt}}
\pgflineto{\pgfpoint{271.210846pt}{358.098206pt}}
\pgfusepath{stroke}
\pgfpathmoveto{\pgfpoint{271.267181pt}{358.132874pt}}
\pgflineto{\pgfpoint{271.290985pt}{358.031036pt}}
\pgfusepath{stroke}
\pgfpathmoveto{\pgfpoint{271.280121pt}{364.023621pt}}
\pgflineto{\pgfpoint{271.204163pt}{364.090393pt}}
\pgfusepath{stroke}
\pgfpathmoveto{\pgfpoint{271.259399pt}{364.122620pt}}
\pgflineto{\pgfpoint{271.280121pt}{364.023621pt}}
\pgfusepath{stroke}
\pgfpathmoveto{\pgfpoint{271.270081pt}{370.016418pt}}
\pgflineto{\pgfpoint{271.198029pt}{370.082642pt}}
\pgfusepath{stroke}
\pgfpathmoveto{\pgfpoint{271.252136pt}{370.112610pt}}
\pgflineto{\pgfpoint{271.270081pt}{370.016418pt}}
\pgfusepath{stroke}
\pgfpathmoveto{\pgfpoint{277.173706pt}{77.073807pt}}
\pgflineto{\pgfpoint{277.180511pt}{76.996872pt}}
\pgfusepath{stroke}
\pgfpathmoveto{\pgfpoint{277.132996pt}{77.008179pt}}
\pgflineto{\pgfpoint{277.173706pt}{77.073807pt}}
\pgfusepath{stroke}
\pgfpathmoveto{\pgfpoint{277.176971pt}{83.068710pt}}
\pgflineto{\pgfpoint{277.183472pt}{82.989029pt}}
\pgfusepath{stroke}
\pgfpathmoveto{\pgfpoint{277.134399pt}{83.001068pt}}
\pgflineto{\pgfpoint{277.176971pt}{83.068710pt}}
\pgfusepath{stroke}
\pgfpathmoveto{\pgfpoint{277.180603pt}{89.064110pt}}
\pgflineto{\pgfpoint{277.186737pt}{88.981483pt}}
\pgfusepath{stroke}
\pgfpathmoveto{\pgfpoint{277.135956pt}{88.994316pt}}
\pgflineto{\pgfpoint{277.180603pt}{89.064110pt}}
\pgfusepath{stroke}
\pgfpathmoveto{\pgfpoint{277.184601pt}{95.060059pt}}
\pgflineto{\pgfpoint{277.190338pt}{94.974258pt}}
\pgfusepath{stroke}
\pgfpathmoveto{\pgfpoint{277.137695pt}{94.987984pt}}
\pgflineto{\pgfpoint{277.184601pt}{95.060059pt}}
\pgfusepath{stroke}
\pgfpathmoveto{\pgfpoint{277.189087pt}{101.056633pt}}
\pgflineto{\pgfpoint{277.194275pt}{100.967415pt}}
\pgfusepath{stroke}
\pgfpathmoveto{\pgfpoint{277.139709pt}{100.982132pt}}
\pgflineto{\pgfpoint{277.189087pt}{101.056633pt}}
\pgfusepath{stroke}
\pgfpathmoveto{\pgfpoint{277.194061pt}{107.053925pt}}
\pgflineto{\pgfpoint{277.198669pt}{106.960953pt}}
\pgfusepath{stroke}
\pgfpathmoveto{\pgfpoint{277.141968pt}{106.976776pt}}
\pgflineto{\pgfpoint{277.194061pt}{107.053925pt}}
\pgfusepath{stroke}
\pgfpathmoveto{\pgfpoint{277.199646pt}{113.051979pt}}
\pgflineto{\pgfpoint{277.203552pt}{112.954956pt}}
\pgfusepath{stroke}
\pgfpathmoveto{\pgfpoint{277.144562pt}{112.972031pt}}
\pgflineto{\pgfpoint{277.199646pt}{113.051979pt}}
\pgfusepath{stroke}
\pgfpathmoveto{\pgfpoint{277.205994pt}{119.050919pt}}
\pgflineto{\pgfpoint{277.209015pt}{118.949471pt}}
\pgfusepath{stroke}
\pgfpathmoveto{\pgfpoint{277.147552pt}{118.967949pt}}
\pgflineto{\pgfpoint{277.205994pt}{119.050919pt}}
\pgfusepath{stroke}
\pgfpathmoveto{\pgfpoint{277.213196pt}{125.050873pt}}
\pgflineto{\pgfpoint{277.215179pt}{124.944557pt}}
\pgfusepath{stroke}
\pgfpathmoveto{\pgfpoint{277.151001pt}{124.964630pt}}
\pgflineto{\pgfpoint{277.213196pt}{125.050873pt}}
\pgfusepath{stroke}
\pgfpathmoveto{\pgfpoint{277.221466pt}{131.051971pt}}
\pgflineto{\pgfpoint{277.222168pt}{130.940292pt}}
\pgfusepath{stroke}
\pgfpathmoveto{\pgfpoint{277.155029pt}{130.962204pt}}
\pgflineto{\pgfpoint{277.221466pt}{131.051971pt}}
\pgfusepath{stroke}
\pgfpathmoveto{\pgfpoint{277.231018pt}{137.054382pt}}
\pgflineto{\pgfpoint{277.230164pt}{136.936752pt}}
\pgfusepath{stroke}
\pgfpathmoveto{\pgfpoint{277.159760pt}{136.960785pt}}
\pgflineto{\pgfpoint{277.231018pt}{137.054382pt}}
\pgfusepath{stroke}
\pgfpathmoveto{\pgfpoint{277.242188pt}{143.058319pt}}
\pgflineto{\pgfpoint{277.239410pt}{142.934052pt}}
\pgfusepath{stroke}
\pgfpathmoveto{\pgfpoint{277.165405pt}{142.960571pt}}
\pgflineto{\pgfpoint{277.242188pt}{143.058319pt}}
\pgfusepath{stroke}
\pgfpathmoveto{\pgfpoint{277.255371pt}{149.064026pt}}
\pgflineto{\pgfpoint{277.250183pt}{148.932312pt}}
\pgfusepath{stroke}
\pgfpathmoveto{\pgfpoint{277.172180pt}{148.961761pt}}
\pgflineto{\pgfpoint{277.255371pt}{149.064026pt}}
\pgfusepath{stroke}
\pgfpathmoveto{\pgfpoint{277.271118pt}{155.071747pt}}
\pgflineto{\pgfpoint{277.262909pt}{154.931610pt}}
\pgfusepath{stroke}
\pgfpathmoveto{\pgfpoint{277.180481pt}{154.964569pt}}
\pgflineto{\pgfpoint{277.271118pt}{155.071747pt}}
\pgfusepath{stroke}
\pgfpathmoveto{\pgfpoint{277.290222pt}{161.081787pt}}
\pgflineto{\pgfpoint{277.278137pt}{160.932129pt}}
\pgfusepath{stroke}
\pgfpathmoveto{\pgfpoint{277.190735pt}{160.969315pt}}
\pgflineto{\pgfpoint{277.290222pt}{161.081787pt}}
\pgfusepath{stroke}
\pgfpathmoveto{\pgfpoint{277.313782pt}{167.094528pt}}
\pgflineto{\pgfpoint{277.296600pt}{166.933899pt}}
\pgfusepath{stroke}
\pgfpathmoveto{\pgfpoint{277.203674pt}{166.976349pt}}
\pgflineto{\pgfpoint{277.313782pt}{167.094528pt}}
\pgfusepath{stroke}
\pgfpathmoveto{\pgfpoint{277.343414pt}{173.110214pt}}
\pgflineto{\pgfpoint{277.319458pt}{172.936996pt}}
\pgfusepath{stroke}
\pgfpathmoveto{\pgfpoint{277.220306pt}{172.986008pt}}
\pgflineto{\pgfpoint{277.343414pt}{173.110214pt}}
\pgfusepath{stroke}
\pgfpathmoveto{\pgfpoint{277.381409pt}{179.128967pt}}
\pgflineto{\pgfpoint{277.348236pt}{178.941193pt}}
\pgfusepath{stroke}
\pgfpathmoveto{\pgfpoint{277.242218pt}{178.998642pt}}
\pgflineto{\pgfpoint{277.381409pt}{179.128967pt}}
\pgfusepath{stroke}
\pgfpathmoveto{\pgfpoint{277.431305pt}{185.150330pt}}
\pgflineto{\pgfpoint{277.385223pt}{184.945816pt}}
\pgfusepath{stroke}
\pgfpathmoveto{\pgfpoint{277.271729pt}{185.014359pt}}
\pgflineto{\pgfpoint{277.431305pt}{185.150330pt}}
\pgfusepath{stroke}
\pgfpathmoveto{\pgfpoint{277.498199pt}{191.172592pt}}
\pgflineto{\pgfpoint{277.433655pt}{190.949127pt}}
\pgfusepath{stroke}
\pgfpathmoveto{\pgfpoint{277.312500pt}{191.032547pt}}
\pgflineto{\pgfpoint{277.498199pt}{191.172592pt}}
\pgfusepath{stroke}
\pgfpathmoveto{\pgfpoint{277.589325pt}{197.190826pt}}
\pgflineto{\pgfpoint{277.497803pt}{196.947083pt}}
\pgfusepath{stroke}
\pgfpathmoveto{\pgfpoint{277.369873pt}{197.050751pt}}
\pgflineto{\pgfpoint{277.589325pt}{197.190826pt}}
\pgfusepath{stroke}
\pgfpathmoveto{\pgfpoint{277.713287pt}{203.193298pt}}
\pgflineto{\pgfpoint{277.582092pt}{202.930878pt}}
\pgfusepath{stroke}
\pgfpathmoveto{\pgfpoint{277.450867pt}{203.062088pt}}
\pgflineto{\pgfpoint{277.713287pt}{203.193298pt}}
\pgfusepath{stroke}
\pgfpathmoveto{\pgfpoint{277.874481pt}{209.155411pt}}
\pgflineto{\pgfpoint{277.686707pt}{208.883621pt}}
\pgfusepath{stroke}
\pgfpathmoveto{\pgfpoint{277.561188pt}{209.050659pt}}
\pgflineto{\pgfpoint{277.874481pt}{209.155411pt}}
\pgfusepath{stroke}
\pgfpathmoveto{\pgfpoint{278.055542pt}{215.038239pt}}
\pgflineto{\pgfpoint{277.795746pt}{214.781311pt}}
\pgfusepath{stroke}
\pgfpathmoveto{\pgfpoint{277.693542pt}{214.988586pt}}
\pgflineto{\pgfpoint{278.055542pt}{215.038239pt}}
\pgfusepath{stroke}
\pgfpathmoveto{\pgfpoint{278.192810pt}{220.816101pt}}
\pgflineto{\pgfpoint{277.863922pt}{220.613876pt}}
\pgfusepath{stroke}
\pgfpathmoveto{\pgfpoint{277.808380pt}{220.851654pt}}
\pgflineto{\pgfpoint{278.192810pt}{220.816101pt}}
\pgfusepath{stroke}
\pgfpathmoveto{\pgfpoint{278.204254pt}{226.538528pt}}
\pgflineto{\pgfpoint{277.842072pt}{226.423492pt}}
\pgfusepath{stroke}
\pgfpathmoveto{\pgfpoint{277.845490pt}{226.663788pt}}
\pgflineto{\pgfpoint{278.204254pt}{226.538528pt}}
\pgfusepath{stroke}
\pgfpathmoveto{\pgfpoint{278.089722pt}{232.313629pt}}
\pgflineto{\pgfpoint{277.742096pt}{232.282196pt}}
\pgfusepath{stroke}
\pgfpathmoveto{\pgfpoint{277.792755pt}{232.497070pt}}
\pgflineto{\pgfpoint{278.089722pt}{232.313629pt}}
\pgfusepath{stroke}
\pgfpathmoveto{\pgfpoint{277.930939pt}{238.190979pt}}
\pgflineto{\pgfpoint{277.623993pt}{238.213989pt}}
\pgfusepath{stroke}
\pgfpathmoveto{\pgfpoint{277.699188pt}{238.393570pt}}
\pgflineto{\pgfpoint{277.930939pt}{238.190979pt}}
\pgfusepath{stroke}
\pgfpathmoveto{\pgfpoint{277.791229pt}{244.144882pt}}
\pgflineto{\pgfpoint{277.527069pt}{244.194702pt}}
\pgfusepath{stroke}
\pgfpathmoveto{\pgfpoint{277.609802pt}{244.343231pt}}
\pgflineto{\pgfpoint{277.791229pt}{244.144882pt}}
\pgfusepath{stroke}
\pgfpathmoveto{\pgfpoint{277.687531pt}{250.136444pt}}
\pgflineto{\pgfpoint{277.458374pt}{250.196518pt}}
\pgfusepath{stroke}
\pgfpathmoveto{\pgfpoint{277.540253pt}{250.321991pt}}
\pgflineto{\pgfpoint{277.687531pt}{250.136444pt}}
\pgfusepath{stroke}
\pgfpathmoveto{\pgfpoint{277.615051pt}{256.141205pt}}
\pgflineto{\pgfpoint{277.411926pt}{256.203674pt}}
\pgfusepath{stroke}
\pgfpathmoveto{\pgfpoint{277.490021pt}{256.313049pt}}
\pgflineto{\pgfpoint{277.615051pt}{256.141205pt}}
\pgfusepath{stroke}
\pgfpathmoveto{\pgfpoint{277.564728pt}{262.147644pt}}
\pgflineto{\pgfpoint{277.380463pt}{262.209473pt}}
\pgfusepath{stroke}
\pgfpathmoveto{\pgfpoint{277.454407pt}{262.307678pt}}
\pgflineto{\pgfpoint{277.564728pt}{262.147644pt}}
\pgfusepath{stroke}
\pgfpathmoveto{\pgfpoint{277.528900pt}{268.151215pt}}
\pgflineto{\pgfpoint{277.358337pt}{268.211731pt}}
\pgfusepath{stroke}
\pgfpathmoveto{\pgfpoint{277.428772pt}{268.301971pt}}
\pgflineto{\pgfpoint{277.528900pt}{268.151215pt}}
\pgfusepath{stroke}
\pgfpathmoveto{\pgfpoint{277.502075pt}{274.150604pt}}
\pgflineto{\pgfpoint{277.341736pt}{274.210236pt}}
\pgfusepath{stroke}
\pgfpathmoveto{\pgfpoint{277.409576pt}{274.294495pt}}
\pgflineto{\pgfpoint{277.502075pt}{274.150604pt}}
\pgfusepath{stroke}
\pgfpathmoveto{\pgfpoint{277.480469pt}{280.145935pt}}
\pgflineto{\pgfpoint{277.328186pt}{280.205444pt}}
\pgfusepath{stroke}
\pgfpathmoveto{\pgfpoint{277.394348pt}{280.284912pt}}
\pgflineto{\pgfpoint{277.480469pt}{280.145935pt}}
\pgfusepath{stroke}
\pgfpathmoveto{\pgfpoint{277.461670pt}{286.138062pt}}
\pgflineto{\pgfpoint{277.316132pt}{286.198181pt}}
\pgfusepath{stroke}
\pgfpathmoveto{\pgfpoint{277.381317pt}{286.273499pt}}
\pgflineto{\pgfpoint{277.461670pt}{286.138062pt}}
\pgfusepath{stroke}
\pgfpathmoveto{\pgfpoint{277.444153pt}{292.127838pt}}
\pgflineto{\pgfpoint{277.304688pt}{292.189240pt}}
\pgfusepath{stroke}
\pgfpathmoveto{\pgfpoint{277.369446pt}{292.260651pt}}
\pgflineto{\pgfpoint{277.444153pt}{292.127838pt}}
\pgfusepath{stroke}
\pgfpathmoveto{\pgfpoint{277.427094pt}{298.116241pt}}
\pgflineto{\pgfpoint{277.293365pt}{298.179321pt}}
\pgfusepath{stroke}
\pgfpathmoveto{\pgfpoint{277.357971pt}{298.246948pt}}
\pgflineto{\pgfpoint{277.427094pt}{298.116241pt}}
\pgfusepath{stroke}
\pgfpathmoveto{\pgfpoint{277.410065pt}{304.104065pt}}
\pgflineto{\pgfpoint{277.282043pt}{304.169006pt}}
\pgfusepath{stroke}
\pgfpathmoveto{\pgfpoint{277.346619pt}{304.232849pt}}
\pgflineto{\pgfpoint{277.410065pt}{304.104065pt}}
\pgfusepath{stroke}
\pgfpathmoveto{\pgfpoint{277.393036pt}{310.091919pt}}
\pgflineto{\pgfpoint{277.270691pt}{310.158752pt}}
\pgfusepath{stroke}
\pgfpathmoveto{\pgfpoint{277.335236pt}{310.218781pt}}
\pgflineto{\pgfpoint{277.393036pt}{310.091919pt}}
\pgfusepath{stroke}
\pgfpathmoveto{\pgfpoint{277.376038pt}{316.080261pt}}
\pgflineto{\pgfpoint{277.259430pt}{316.148773pt}}
\pgfusepath{stroke}
\pgfpathmoveto{\pgfpoint{277.323853pt}{316.205048pt}}
\pgflineto{\pgfpoint{277.376038pt}{316.080261pt}}
\pgfusepath{stroke}
\pgfpathmoveto{\pgfpoint{277.359314pt}{322.069336pt}}
\pgflineto{\pgfpoint{277.248444pt}{322.139282pt}}
\pgfusepath{stroke}
\pgfpathmoveto{\pgfpoint{277.312561pt}{322.191833pt}}
\pgflineto{\pgfpoint{277.359314pt}{322.069336pt}}
\pgfusepath{stroke}
\pgfpathmoveto{\pgfpoint{277.343048pt}{328.059265pt}}
\pgflineto{\pgfpoint{277.237823pt}{328.130280pt}}
\pgfusepath{stroke}
\pgfpathmoveto{\pgfpoint{277.301483pt}{328.179199pt}}
\pgflineto{\pgfpoint{277.343048pt}{328.059265pt}}
\pgfusepath{stroke}
\pgfpathmoveto{\pgfpoint{277.327454pt}{334.049988pt}}
\pgflineto{\pgfpoint{277.227753pt}{334.121765pt}}
\pgfusepath{stroke}
\pgfpathmoveto{\pgfpoint{277.290741pt}{334.167236pt}}
\pgflineto{\pgfpoint{277.327454pt}{334.049988pt}}
\pgfusepath{stroke}
\pgfpathmoveto{\pgfpoint{277.312622pt}{340.041534pt}}
\pgflineto{\pgfpoint{277.218292pt}{340.113678pt}}
\pgfusepath{stroke}
\pgfpathmoveto{\pgfpoint{277.280426pt}{340.155853pt}}
\pgflineto{\pgfpoint{277.312622pt}{340.041534pt}}
\pgfusepath{stroke}
\pgfpathmoveto{\pgfpoint{277.298676pt}{346.033722pt}}
\pgflineto{\pgfpoint{277.209473pt}{346.105957pt}}
\pgfusepath{stroke}
\pgfpathmoveto{\pgfpoint{277.270660pt}{346.145020pt}}
\pgflineto{\pgfpoint{277.298676pt}{346.033722pt}}
\pgfusepath{stroke}
\pgfpathmoveto{\pgfpoint{277.285675pt}{352.026489pt}}
\pgflineto{\pgfpoint{277.201355pt}{352.098480pt}}
\pgfusepath{stroke}
\pgfpathmoveto{\pgfpoint{277.261414pt}{352.134644pt}}
\pgflineto{\pgfpoint{277.285675pt}{352.026489pt}}
\pgfusepath{stroke}
\pgfpathmoveto{\pgfpoint{277.273621pt}{358.019653pt}}
\pgflineto{\pgfpoint{277.193939pt}{358.091187pt}}
\pgfusepath{stroke}
\pgfpathmoveto{\pgfpoint{277.252808pt}{358.124664pt}}
\pgflineto{\pgfpoint{277.273621pt}{358.019653pt}}
\pgfusepath{stroke}
\pgfpathmoveto{\pgfpoint{277.262512pt}{364.013062pt}}
\pgflineto{\pgfpoint{277.187164pt}{364.083923pt}}
\pgfusepath{stroke}
\pgfpathmoveto{\pgfpoint{277.244751pt}{364.114990pt}}
\pgflineto{\pgfpoint{277.262512pt}{364.013062pt}}
\pgfusepath{stroke}
\pgfpathmoveto{\pgfpoint{277.252289pt}{370.006714pt}}
\pgflineto{\pgfpoint{277.181000pt}{370.076782pt}}
\pgfusepath{stroke}
\pgfpathmoveto{\pgfpoint{277.237274pt}{370.105530pt}}
\pgflineto{\pgfpoint{277.252289pt}{370.006714pt}}
\pgfusepath{stroke}
\pgfpathmoveto{\pgfpoint{283.154388pt}{77.076691pt}}
\pgflineto{\pgfpoint{283.163879pt}{76.999588pt}}
\pgfusepath{stroke}
\pgfpathmoveto{\pgfpoint{283.115723pt}{77.009323pt}}
\pgflineto{\pgfpoint{283.154388pt}{77.076691pt}}
\pgfusepath{stroke}
\pgfpathmoveto{\pgfpoint{283.157196pt}{83.071884pt}}
\pgflineto{\pgfpoint{283.166565pt}{82.992004pt}}
\pgfusepath{stroke}
\pgfpathmoveto{\pgfpoint{283.116760pt}{83.002350pt}}
\pgflineto{\pgfpoint{283.157196pt}{83.071884pt}}
\pgfusepath{stroke}
\pgfpathmoveto{\pgfpoint{283.160248pt}{89.067627pt}}
\pgflineto{\pgfpoint{283.169495pt}{88.984726pt}}
\pgfusepath{stroke}
\pgfpathmoveto{\pgfpoint{283.117920pt}{88.995773pt}}
\pgflineto{\pgfpoint{283.160248pt}{89.067627pt}}
\pgfusepath{stroke}
\pgfpathmoveto{\pgfpoint{283.163635pt}{95.063957pt}}
\pgflineto{\pgfpoint{283.172699pt}{94.977844pt}}
\pgfusepath{stroke}
\pgfpathmoveto{\pgfpoint{283.119232pt}{94.989624pt}}
\pgflineto{\pgfpoint{283.163635pt}{95.063957pt}}
\pgfusepath{stroke}
\pgfpathmoveto{\pgfpoint{283.167419pt}{101.060989pt}}
\pgflineto{\pgfpoint{283.176270pt}{100.971352pt}}
\pgfusepath{stroke}
\pgfpathmoveto{\pgfpoint{283.120697pt}{100.983971pt}}
\pgflineto{\pgfpoint{283.167419pt}{101.060989pt}}
\pgfusepath{stroke}
\pgfpathmoveto{\pgfpoint{283.171631pt}{107.058792pt}}
\pgflineto{\pgfpoint{283.180176pt}{106.965332pt}}
\pgfusepath{stroke}
\pgfpathmoveto{\pgfpoint{283.122406pt}{106.978905pt}}
\pgflineto{\pgfpoint{283.171631pt}{107.058792pt}}
\pgfusepath{stroke}
\pgfpathmoveto{\pgfpoint{283.176331pt}{113.057465pt}}
\pgflineto{\pgfpoint{283.184540pt}{112.959839pt}}
\pgfusepath{stroke}
\pgfpathmoveto{\pgfpoint{283.124329pt}{112.974449pt}}
\pgflineto{\pgfpoint{283.176331pt}{113.057465pt}}
\pgfusepath{stroke}
\pgfpathmoveto{\pgfpoint{283.181671pt}{119.057144pt}}
\pgflineto{\pgfpoint{283.189423pt}{118.954956pt}}
\pgfusepath{stroke}
\pgfpathmoveto{\pgfpoint{283.126556pt}{118.970749pt}}
\pgflineto{\pgfpoint{283.181671pt}{119.057144pt}}
\pgfusepath{stroke}
\pgfpathmoveto{\pgfpoint{283.187714pt}{125.057968pt}}
\pgflineto{\pgfpoint{283.194885pt}{124.950752pt}}
\pgfusepath{stroke}
\pgfpathmoveto{\pgfpoint{283.129120pt}{124.967888pt}}
\pgflineto{\pgfpoint{283.187714pt}{125.057968pt}}
\pgfusepath{stroke}
\pgfpathmoveto{\pgfpoint{283.194641pt}{131.060150pt}}
\pgflineto{\pgfpoint{283.201111pt}{130.947342pt}}
\pgfusepath{stroke}
\pgfpathmoveto{\pgfpoint{283.132141pt}{130.966034pt}}
\pgflineto{\pgfpoint{283.194641pt}{131.060150pt}}
\pgfusepath{stroke}
\pgfpathmoveto{\pgfpoint{283.202637pt}{137.063873pt}}
\pgflineto{\pgfpoint{283.208221pt}{136.944870pt}}
\pgfusepath{stroke}
\pgfpathmoveto{\pgfpoint{283.135712pt}{136.965332pt}}
\pgflineto{\pgfpoint{283.202637pt}{137.063873pt}}
\pgfusepath{stroke}
\pgfpathmoveto{\pgfpoint{283.211975pt}{143.069458pt}}
\pgflineto{\pgfpoint{283.216431pt}{142.943481pt}}
\pgfusepath{stroke}
\pgfpathmoveto{\pgfpoint{283.139954pt}{142.966003pt}}
\pgflineto{\pgfpoint{283.211975pt}{143.069458pt}}
\pgfusepath{stroke}
\pgfpathmoveto{\pgfpoint{283.222992pt}{149.077255pt}}
\pgflineto{\pgfpoint{283.225983pt}{148.943359pt}}
\pgfusepath{stroke}
\pgfpathmoveto{\pgfpoint{283.145050pt}{148.968323pt}}
\pgflineto{\pgfpoint{283.222992pt}{149.077255pt}}
\pgfusepath{stroke}
\pgfpathmoveto{\pgfpoint{283.236176pt}{155.087692pt}}
\pgflineto{\pgfpoint{283.237305pt}{154.944778pt}}
\pgfusepath{stroke}
\pgfpathmoveto{\pgfpoint{283.151337pt}{154.972687pt}}
\pgflineto{\pgfpoint{283.236176pt}{155.087692pt}}
\pgfusepath{stroke}
\pgfpathmoveto{\pgfpoint{283.252258pt}{161.101364pt}}
\pgflineto{\pgfpoint{283.250916pt}{160.948029pt}}
\pgfusepath{stroke}
\pgfpathmoveto{\pgfpoint{283.159180pt}{160.979507pt}}
\pgflineto{\pgfpoint{283.252258pt}{161.101364pt}}
\pgfusepath{stroke}
\pgfpathmoveto{\pgfpoint{283.272217pt}{167.119003pt}}
\pgflineto{\pgfpoint{283.267517pt}{166.953491pt}}
\pgfusepath{stroke}
\pgfpathmoveto{\pgfpoint{283.169159pt}{166.989395pt}}
\pgflineto{\pgfpoint{283.272217pt}{167.119003pt}}
\pgfusepath{stroke}
\pgfpathmoveto{\pgfpoint{283.297577pt}{173.141541pt}}
\pgflineto{\pgfpoint{283.288300pt}{172.961609pt}}
\pgfusepath{stroke}
\pgfpathmoveto{\pgfpoint{283.182220pt}{173.003159pt}}
\pgflineto{\pgfpoint{283.297577pt}{173.141541pt}}
\pgfusepath{stroke}
\pgfpathmoveto{\pgfpoint{283.330811pt}{179.170105pt}}
\pgflineto{\pgfpoint{283.315002pt}{178.972870pt}}
\pgfusepath{stroke}
\pgfpathmoveto{\pgfpoint{283.199829pt}{179.021820pt}}
\pgflineto{\pgfpoint{283.330811pt}{179.170105pt}}
\pgfusepath{stroke}
\pgfpathmoveto{\pgfpoint{283.375793pt}{185.206009pt}}
\pgflineto{\pgfpoint{283.350342pt}{184.987732pt}}
\pgfusepath{stroke}
\pgfpathmoveto{\pgfpoint{283.224457pt}{185.046661pt}}
\pgflineto{\pgfpoint{283.375793pt}{185.206009pt}}
\pgfusepath{stroke}
\pgfpathmoveto{\pgfpoint{283.439148pt}{191.250290pt}}
\pgflineto{\pgfpoint{283.398865pt}{191.006134pt}}
\pgfusepath{stroke}
\pgfpathmoveto{\pgfpoint{283.260406pt}{191.079147pt}}
\pgflineto{\pgfpoint{283.439148pt}{191.250290pt}}
\pgfusepath{stroke}
\pgfpathmoveto{\pgfpoint{283.532410pt}{197.302246pt}}
\pgflineto{\pgfpoint{283.468170pt}{197.026321pt}}
\pgfusepath{stroke}
\pgfpathmoveto{\pgfpoint{283.315460pt}{197.120056pt}}
\pgflineto{\pgfpoint{283.532410pt}{197.302246pt}}
\pgfusepath{stroke}
\pgfpathmoveto{\pgfpoint{283.675415pt}{203.354492pt}}
\pgflineto{\pgfpoint{283.570648pt}{203.041183pt}}
\pgfusepath{stroke}
\pgfpathmoveto{\pgfpoint{283.403625pt}{203.166702pt}}
\pgflineto{\pgfpoint{283.675415pt}{203.354492pt}}
\pgfusepath{stroke}
\pgfpathmoveto{\pgfpoint{283.898010pt}{209.378021pt}}
\pgflineto{\pgfpoint{283.723053pt}{209.028061pt}}
\pgfusepath{stroke}
\pgfpathmoveto{\pgfpoint{283.548065pt}{209.203049pt}}
\pgflineto{\pgfpoint{283.898010pt}{209.378021pt}}
\pgfusepath{stroke}
\pgfpathmoveto{\pgfpoint{284.217224pt}{215.288315pt}}
\pgflineto{\pgfpoint{283.927246pt}{214.928696pt}}
\pgfusepath{stroke}
\pgfpathmoveto{\pgfpoint{283.769470pt}{215.174622pt}}
\pgflineto{\pgfpoint{284.217224pt}{215.288315pt}}
\pgfusepath{stroke}
\pgfpathmoveto{\pgfpoint{284.525452pt}{220.948761pt}}
\pgflineto{\pgfpoint{284.096344pt}{220.664001pt}}
\pgfusepath{stroke}
\pgfpathmoveto{\pgfpoint{284.011322pt}{220.978409pt}}
\pgflineto{\pgfpoint{284.525452pt}{220.948761pt}}
\pgfusepath{stroke}
\pgfpathmoveto{\pgfpoint{284.538330pt}{226.430145pt}}
\pgflineto{\pgfpoint{284.048676pt}{226.312759pt}}
\pgfusepath{stroke}
\pgfpathmoveto{\pgfpoint{284.076172pt}{226.630035pt}}
\pgflineto{\pgfpoint{284.538330pt}{226.430145pt}}
\pgfusepath{stroke}
\pgfpathmoveto{\pgfpoint{284.255768pt}{232.087555pt}}
\pgflineto{\pgfpoint{283.823578pt}{232.111664pt}}
\pgfusepath{stroke}
\pgfpathmoveto{\pgfpoint{283.924500pt}{232.366150pt}}
\pgflineto{\pgfpoint{284.255768pt}{232.087555pt}}
\pgfusepath{stroke}
\pgfpathmoveto{\pgfpoint{283.961731pt}{237.991730pt}}
\pgflineto{\pgfpoint{283.618286pt}{238.076370pt}}
\pgfusepath{stroke}
\pgfpathmoveto{\pgfpoint{283.737762pt}{238.265518pt}}
\pgflineto{\pgfpoint{283.961731pt}{237.991730pt}}
\pgfusepath{stroke}
\pgfpathmoveto{\pgfpoint{283.763519pt}{244.006042pt}}
\pgflineto{\pgfpoint{283.489075pt}{244.103851pt}}
\pgfusepath{stroke}
\pgfpathmoveto{\pgfpoint{283.602661pt}{244.248947pt}}
\pgflineto{\pgfpoint{283.763519pt}{244.006042pt}}
\pgfusepath{stroke}
\pgfpathmoveto{\pgfpoint{283.643646pt}{250.045929pt}}
\pgflineto{\pgfpoint{283.414978pt}{250.139694pt}}
\pgfusepath{stroke}
\pgfpathmoveto{\pgfpoint{283.516968pt}{250.258148pt}}
\pgflineto{\pgfpoint{283.643646pt}{250.045929pt}}
\pgfusepath{stroke}
\pgfpathmoveto{\pgfpoint{283.571777pt}{256.082458pt}}
\pgflineto{\pgfpoint{283.372467pt}{256.167969pt}}
\pgfusepath{stroke}
\pgfpathmoveto{\pgfpoint{283.463623pt}{256.270447pt}}
\pgflineto{\pgfpoint{283.571777pt}{256.082458pt}}
\pgfusepath{stroke}
\pgfpathmoveto{\pgfpoint{283.527435pt}{262.108459pt}}
\pgflineto{\pgfpoint{283.347168pt}{262.186127pt}}
\pgfusepath{stroke}
\pgfpathmoveto{\pgfpoint{283.429810pt}{262.278748pt}}
\pgflineto{\pgfpoint{283.527435pt}{262.108459pt}}
\pgfusepath{stroke}
\pgfpathmoveto{\pgfpoint{283.498505pt}{268.123566pt}}
\pgflineto{\pgfpoint{283.330933pt}{268.195343pt}}
\pgfusepath{stroke}
\pgfpathmoveto{\pgfpoint{283.407501pt}{268.281525pt}}
\pgflineto{\pgfpoint{283.498505pt}{268.123566pt}}
\pgfusepath{stroke}
\pgfpathmoveto{\pgfpoint{283.477875pt}{274.129425pt}}
\pgflineto{\pgfpoint{283.319183pt}{274.197449pt}}
\pgfusepath{stroke}
\pgfpathmoveto{\pgfpoint{283.391724pt}{274.279053pt}}
\pgflineto{\pgfpoint{283.477875pt}{274.129425pt}}
\pgfusepath{stroke}
\pgfpathmoveto{\pgfpoint{283.461243pt}{280.128021pt}}
\pgflineto{\pgfpoint{283.309296pt}{280.194275pt}}
\pgfusepath{stroke}
\pgfpathmoveto{\pgfpoint{283.379425pt}{280.272217pt}}
\pgflineto{\pgfpoint{283.461243pt}{280.128021pt}}
\pgfusepath{stroke}
\pgfpathmoveto{\pgfpoint{283.445984pt}{286.121429pt}}
\pgflineto{\pgfpoint{283.299744pt}{286.187500pt}}
\pgfusepath{stroke}
\pgfpathmoveto{\pgfpoint{283.368622pt}{286.262024pt}}
\pgflineto{\pgfpoint{283.445984pt}{286.121429pt}}
\pgfusepath{stroke}
\pgfpathmoveto{\pgfpoint{283.430695pt}{292.111420pt}}
\pgflineto{\pgfpoint{283.289795pt}{292.178436pt}}
\pgfusepath{stroke}
\pgfpathmoveto{\pgfpoint{283.358185pt}{292.249573pt}}
\pgflineto{\pgfpoint{283.430695pt}{292.111420pt}}
\pgfusepath{stroke}
\pgfpathmoveto{\pgfpoint{283.414734pt}{298.099548pt}}
\pgflineto{\pgfpoint{283.279205pt}{298.168213pt}}
\pgfusepath{stroke}
\pgfpathmoveto{\pgfpoint{283.347504pt}{298.235779pt}}
\pgflineto{\pgfpoint{283.414734pt}{298.099548pt}}
\pgfusepath{stroke}
\pgfpathmoveto{\pgfpoint{283.397949pt}{304.086975pt}}
\pgflineto{\pgfpoint{283.267975pt}{304.157623pt}}
\pgfusepath{stroke}
\pgfpathmoveto{\pgfpoint{283.336365pt}{304.221466pt}}
\pgflineto{\pgfpoint{283.397949pt}{304.086975pt}}
\pgfusepath{stroke}
\pgfpathmoveto{\pgfpoint{283.380524pt}{310.074585pt}}
\pgflineto{\pgfpoint{283.256348pt}{310.147217pt}}
\pgfusepath{stroke}
\pgfpathmoveto{\pgfpoint{283.324738pt}{310.207214pt}}
\pgflineto{\pgfpoint{283.380524pt}{310.074585pt}}
\pgfusepath{stroke}
\pgfpathmoveto{\pgfpoint{283.362793pt}{316.062958pt}}
\pgflineto{\pgfpoint{283.244598pt}{316.137360pt}}
\pgfusepath{stroke}
\pgfpathmoveto{\pgfpoint{283.312866pt}{316.193390pt}}
\pgflineto{\pgfpoint{283.362793pt}{316.062958pt}}
\pgfusepath{stroke}
\pgfpathmoveto{\pgfpoint{283.345123pt}{322.052368pt}}
\pgflineto{\pgfpoint{283.233002pt}{322.128174pt}}
\pgfusepath{stroke}
\pgfpathmoveto{\pgfpoint{283.300903pt}{322.180267pt}}
\pgflineto{\pgfpoint{283.345123pt}{322.052368pt}}
\pgfusepath{stroke}
\pgfpathmoveto{\pgfpoint{283.327911pt}{328.042847pt}}
\pgflineto{\pgfpoint{283.221802pt}{328.119629pt}}
\pgfusepath{stroke}
\pgfpathmoveto{\pgfpoint{283.289124pt}{328.167938pt}}
\pgflineto{\pgfpoint{283.327911pt}{328.042847pt}}
\pgfusepath{stroke}
\pgfpathmoveto{\pgfpoint{283.311401pt}{334.034363pt}}
\pgflineto{\pgfpoint{283.211212pt}{334.111755pt}}
\pgfusepath{stroke}
\pgfpathmoveto{\pgfpoint{283.277679pt}{334.156372pt}}
\pgflineto{\pgfpoint{283.311401pt}{334.034363pt}}
\pgfusepath{stroke}
\pgfpathmoveto{\pgfpoint{283.295776pt}{340.026794pt}}
\pgflineto{\pgfpoint{283.201355pt}{340.104370pt}}
\pgfusepath{stroke}
\pgfpathmoveto{\pgfpoint{283.266785pt}{340.145508pt}}
\pgflineto{\pgfpoint{283.295776pt}{340.026794pt}}
\pgfusepath{stroke}
\pgfpathmoveto{\pgfpoint{283.281219pt}{346.019958pt}}
\pgflineto{\pgfpoint{283.192230pt}{346.097351pt}}
\pgfusepath{stroke}
\pgfpathmoveto{\pgfpoint{283.256439pt}{346.135254pt}}
\pgflineto{\pgfpoint{283.281219pt}{346.019958pt}}
\pgfusepath{stroke}
\pgfpathmoveto{\pgfpoint{283.267731pt}{352.013733pt}}
\pgflineto{\pgfpoint{283.183899pt}{352.090607pt}}
\pgfusepath{stroke}
\pgfpathmoveto{\pgfpoint{283.246796pt}{352.125519pt}}
\pgflineto{\pgfpoint{283.267731pt}{352.013733pt}}
\pgfusepath{stroke}
\pgfpathmoveto{\pgfpoint{283.255310pt}{358.007874pt}}
\pgflineto{\pgfpoint{283.176361pt}{358.083984pt}}
\pgfusepath{stroke}
\pgfpathmoveto{\pgfpoint{283.237823pt}{358.116180pt}}
\pgflineto{\pgfpoint{283.255310pt}{358.007874pt}}
\pgfusepath{stroke}
\pgfpathmoveto{\pgfpoint{283.243988pt}{364.002319pt}}
\pgflineto{\pgfpoint{283.169525pt}{364.077454pt}}
\pgfusepath{stroke}
\pgfpathmoveto{\pgfpoint{283.229523pt}{364.107117pt}}
\pgflineto{\pgfpoint{283.243988pt}{364.002319pt}}
\pgfusepath{stroke}
\pgfpathmoveto{\pgfpoint{283.233643pt}{369.996887pt}}
\pgflineto{\pgfpoint{283.163391pt}{370.070923pt}}
\pgfusepath{stroke}
\pgfpathmoveto{\pgfpoint{283.221863pt}{370.098267pt}}
\pgflineto{\pgfpoint{283.233643pt}{369.996887pt}}
\pgfusepath{stroke}
\pgfpathmoveto{\pgfpoint{289.134857pt}{77.079086pt}}
\pgflineto{\pgfpoint{289.147034pt}{77.002014pt}}
\pgfusepath{stroke}
\pgfpathmoveto{\pgfpoint{289.098358pt}{77.010117pt}}
\pgflineto{\pgfpoint{289.134857pt}{77.079086pt}}
\pgfusepath{stroke}
\pgfpathmoveto{\pgfpoint{289.137115pt}{83.074524pt}}
\pgflineto{\pgfpoint{289.149384pt}{82.994629pt}}
\pgfusepath{stroke}
\pgfpathmoveto{\pgfpoint{289.098999pt}{83.003250pt}}
\pgflineto{\pgfpoint{289.137115pt}{83.074524pt}}
\pgfusepath{stroke}
\pgfpathmoveto{\pgfpoint{289.139587pt}{89.070534pt}}
\pgflineto{\pgfpoint{289.151947pt}{88.987610pt}}
\pgfusepath{stroke}
\pgfpathmoveto{\pgfpoint{289.099731pt}{88.996780pt}}
\pgflineto{\pgfpoint{289.139587pt}{89.070534pt}}
\pgfusepath{stroke}
\pgfpathmoveto{\pgfpoint{289.142303pt}{95.067192pt}}
\pgflineto{\pgfpoint{289.154755pt}{94.980980pt}}
\pgfusepath{stroke}
\pgfpathmoveto{\pgfpoint{289.100555pt}{94.990768pt}}
\pgflineto{\pgfpoint{289.142303pt}{95.067192pt}}
\pgfusepath{stroke}
\pgfpathmoveto{\pgfpoint{289.145325pt}{101.064583pt}}
\pgflineto{\pgfpoint{289.157837pt}{100.974846pt}}
\pgfusepath{stroke}
\pgfpathmoveto{\pgfpoint{289.101501pt}{100.985275pt}}
\pgflineto{\pgfpoint{289.145325pt}{101.064583pt}}
\pgfusepath{stroke}
\pgfpathmoveto{\pgfpoint{289.148682pt}{107.062790pt}}
\pgflineto{\pgfpoint{289.161255pt}{106.969185pt}}
\pgfusepath{stroke}
\pgfpathmoveto{\pgfpoint{289.102570pt}{106.980370pt}}
\pgflineto{\pgfpoint{289.148682pt}{107.062790pt}}
\pgfusepath{stroke}
\pgfpathmoveto{\pgfpoint{289.152435pt}{113.061943pt}}
\pgflineto{\pgfpoint{289.165009pt}{112.964142pt}}
\pgfusepath{stroke}
\pgfpathmoveto{\pgfpoint{289.103790pt}{112.976151pt}}
\pgflineto{\pgfpoint{289.152435pt}{113.061943pt}}
\pgfusepath{stroke}
\pgfpathmoveto{\pgfpoint{289.156616pt}{119.062210pt}}
\pgflineto{\pgfpoint{289.169189pt}{118.959763pt}}
\pgfusepath{stroke}
\pgfpathmoveto{\pgfpoint{289.105194pt}{118.972702pt}}
\pgflineto{\pgfpoint{289.156616pt}{119.062210pt}}
\pgfusepath{stroke}
\pgfpathmoveto{\pgfpoint{289.161377pt}{125.063736pt}}
\pgflineto{\pgfpoint{289.173889pt}{124.956169pt}}
\pgfusepath{stroke}
\pgfpathmoveto{\pgfpoint{289.106842pt}{124.970169pt}}
\pgflineto{\pgfpoint{289.161377pt}{125.063736pt}}
\pgfusepath{stroke}
\pgfpathmoveto{\pgfpoint{289.166748pt}{131.066742pt}}
\pgflineto{\pgfpoint{289.179169pt}{130.953491pt}}
\pgfusepath{stroke}
\pgfpathmoveto{\pgfpoint{289.108734pt}{130.968689pt}}
\pgflineto{\pgfpoint{289.166748pt}{131.066742pt}}
\pgfusepath{stroke}
\pgfpathmoveto{\pgfpoint{289.172943pt}{137.071518pt}}
\pgflineto{\pgfpoint{289.185181pt}{136.951904pt}}
\pgfusepath{stroke}
\pgfpathmoveto{\pgfpoint{289.110962pt}{136.968475pt}}
\pgflineto{\pgfpoint{289.172943pt}{137.071518pt}}
\pgfusepath{stroke}
\pgfpathmoveto{\pgfpoint{289.180084pt}{143.078400pt}}
\pgflineto{\pgfpoint{289.192078pt}{142.951630pt}}
\pgfusepath{stroke}
\pgfpathmoveto{\pgfpoint{289.113617pt}{142.969788pt}}
\pgflineto{\pgfpoint{289.180084pt}{143.078400pt}}
\pgfusepath{stroke}
\pgfpathmoveto{\pgfpoint{289.188477pt}{149.087845pt}}
\pgflineto{\pgfpoint{289.200073pt}{148.952911pt}}
\pgfusepath{stroke}
\pgfpathmoveto{\pgfpoint{289.116791pt}{148.972931pt}}
\pgflineto{\pgfpoint{289.188477pt}{149.087845pt}}
\pgfusepath{stroke}
\pgfpathmoveto{\pgfpoint{289.198456pt}{155.100449pt}}
\pgflineto{\pgfpoint{289.209503pt}{154.956116pt}}
\pgfusepath{stroke}
\pgfpathmoveto{\pgfpoint{289.120667pt}{154.978363pt}}
\pgflineto{\pgfpoint{289.198456pt}{155.100449pt}}
\pgfusepath{stroke}
\pgfpathmoveto{\pgfpoint{289.210480pt}{161.117004pt}}
\pgflineto{\pgfpoint{289.220703pt}{160.961731pt}}
\pgfusepath{stroke}
\pgfpathmoveto{\pgfpoint{289.125488pt}{160.986649pt}}
\pgflineto{\pgfpoint{289.210480pt}{161.117004pt}}
\pgfusepath{stroke}
\pgfpathmoveto{\pgfpoint{289.225342pt}{167.138626pt}}
\pgflineto{\pgfpoint{289.234375pt}{166.970413pt}}
\pgfusepath{stroke}
\pgfpathmoveto{\pgfpoint{289.131653pt}{166.998627pt}}
\pgflineto{\pgfpoint{289.225342pt}{167.138626pt}}
\pgfusepath{stroke}
\pgfpathmoveto{\pgfpoint{289.244171pt}{173.166840pt}}
\pgflineto{\pgfpoint{289.251434pt}{172.983047pt}}
\pgfusepath{stroke}
\pgfpathmoveto{\pgfpoint{289.139709pt}{173.015457pt}}
\pgflineto{\pgfpoint{289.244171pt}{173.166840pt}}
\pgfusepath{stroke}
\pgfpathmoveto{\pgfpoint{289.268860pt}{179.203903pt}}
\pgflineto{\pgfpoint{289.273346pt}{179.000931pt}}
\pgfusepath{stroke}
\pgfpathmoveto{\pgfpoint{289.150665pt}{179.038818pt}}
\pgflineto{\pgfpoint{289.268860pt}{179.203903pt}}
\pgfusepath{stroke}
\pgfpathmoveto{\pgfpoint{289.302643pt}{185.253098pt}}
\pgflineto{\pgfpoint{289.302704pt}{185.025894pt}}
\pgfusepath{stroke}
\pgfpathmoveto{\pgfpoint{289.166382pt}{185.071289pt}}
\pgflineto{\pgfpoint{289.302643pt}{185.253098pt}}
\pgfusepath{stroke}
\pgfpathmoveto{\pgfpoint{289.351562pt}{191.319458pt}}
\pgflineto{\pgfpoint{289.344086pt}{191.060608pt}}
\pgfusepath{stroke}
\pgfpathmoveto{\pgfpoint{289.190277pt}{191.116882pt}}
\pgflineto{\pgfpoint{289.351562pt}{191.319458pt}}
\pgfusepath{stroke}
\pgfpathmoveto{\pgfpoint{289.427795pt}{197.410370pt}}
\pgflineto{\pgfpoint{289.406342pt}{197.108673pt}}
\pgfusepath{stroke}
\pgfpathmoveto{\pgfpoint{289.229614pt}{197.181885pt}}
\pgflineto{\pgfpoint{289.427795pt}{197.410370pt}}
\pgfusepath{stroke}
\pgfpathmoveto{\pgfpoint{289.558228pt}{203.535538pt}}
\pgflineto{\pgfpoint{289.508575pt}{203.173538pt}}
\pgfusepath{stroke}
\pgfpathmoveto{\pgfpoint{289.301300pt}{203.275742pt}}
\pgflineto{\pgfpoint{289.558228pt}{203.535538pt}}
\pgfusepath{stroke}
\pgfpathmoveto{\pgfpoint{289.808319pt}{209.697235pt}}
\pgflineto{\pgfpoint{289.694611pt}{209.249481pt}}
\pgfusepath{stroke}
\pgfpathmoveto{\pgfpoint{289.448700pt}{209.407242pt}}
\pgflineto{\pgfpoint{289.808319pt}{209.697235pt}}
\pgfusepath{stroke}
\pgfpathmoveto{\pgfpoint{290.333832pt}{215.813828pt}}
\pgflineto{\pgfpoint{290.059265pt}{215.264725pt}}
\pgfusepath{stroke}
\pgfpathmoveto{\pgfpoint{289.784729pt}{215.539276pt}}
\pgflineto{\pgfpoint{290.333832pt}{215.813828pt}}
\pgfusepath{stroke}
\pgfpathmoveto{\pgfpoint{291.219757pt}{221.364685pt}}
\pgflineto{\pgfpoint{290.601349pt}{220.862793pt}}
\pgfusepath{stroke}
\pgfpathmoveto{\pgfpoint{290.423889pt}{221.334198pt}}
\pgflineto{\pgfpoint{291.219757pt}{221.364685pt}}
\pgfusepath{stroke}
\pgfpathmoveto{\pgfpoint{291.234222pt}{226.041534pt}}
\pgflineto{\pgfpoint{290.465332pt}{225.975006pt}}
\pgfusepath{stroke}
\pgfpathmoveto{\pgfpoint{290.579193pt}{226.449631pt}}
\pgflineto{\pgfpoint{291.234222pt}{226.041534pt}}
\pgfusepath{stroke}
\pgfpathmoveto{\pgfpoint{290.377136pt}{231.589096pt}}
\pgflineto{\pgfpoint{289.845032pt}{231.764526pt}}
\pgfusepath{stroke}
\pgfpathmoveto{\pgfpoint{290.056702pt}{232.048691pt}}
\pgflineto{\pgfpoint{290.377136pt}{231.589096pt}}
\pgfusepath{stroke}
\pgfpathmoveto{\pgfpoint{289.880066pt}{237.698990pt}}
\pgflineto{\pgfpoint{289.527252pt}{237.888931pt}}
\pgfusepath{stroke}
\pgfpathmoveto{\pgfpoint{289.711761pt}{238.062653pt}}
\pgflineto{\pgfpoint{289.880066pt}{237.698990pt}}
\pgfusepath{stroke}
\pgfpathmoveto{\pgfpoint{289.657745pt}{243.850494pt}}
\pgflineto{\pgfpoint{289.397217pt}{244.010544pt}}
\pgfusepath{stroke}
\pgfpathmoveto{\pgfpoint{289.545349pt}{244.134857pt}}
\pgflineto{\pgfpoint{289.657745pt}{243.850494pt}}
\pgfusepath{stroke}
\pgfpathmoveto{\pgfpoint{289.553833pt}{249.961807pt}}
\pgflineto{\pgfpoint{289.341675pt}{250.092224pt}}
\pgfusepath{stroke}
\pgfpathmoveto{\pgfpoint{289.462372pt}{250.193451pt}}
\pgflineto{\pgfpoint{289.553833pt}{249.961807pt}}
\pgfusepath{stroke}
\pgfpathmoveto{\pgfpoint{289.502319pt}{256.035156pt}}
\pgflineto{\pgfpoint{289.316833pt}{256.142761pt}}
\pgfusepath{stroke}
\pgfpathmoveto{\pgfpoint{289.418488pt}{256.232544pt}}
\pgflineto{\pgfpoint{289.502319pt}{256.035156pt}}
\pgfusepath{stroke}
\pgfpathmoveto{\pgfpoint{289.475464pt}{262.080383pt}}
\pgflineto{\pgfpoint{289.305298pt}{262.171814pt}}
\pgfusepath{stroke}
\pgfpathmoveto{\pgfpoint{289.394196pt}{262.255615pt}}
\pgflineto{\pgfpoint{289.475464pt}{262.080383pt}}
\pgfusepath{stroke}
\pgfpathmoveto{\pgfpoint{289.460297pt}{268.105255pt}}
\pgflineto{\pgfpoint{289.299316pt}{268.186005pt}}
\pgfusepath{stroke}
\pgfpathmoveto{\pgfpoint{289.379974pt}{268.266449pt}}
\pgflineto{\pgfpoint{289.460297pt}{268.105255pt}}
\pgfusepath{stroke}
\pgfpathmoveto{\pgfpoint{289.450012pt}{274.115479pt}}
\pgflineto{\pgfpoint{289.294983pt}{274.189880pt}}
\pgfusepath{stroke}
\pgfpathmoveto{\pgfpoint{289.370636pt}{274.268005pt}}
\pgflineto{\pgfpoint{289.450012pt}{274.115479pt}}
\pgfusepath{stroke}
\pgfpathmoveto{\pgfpoint{289.440765pt}{280.115295pt}}
\pgflineto{\pgfpoint{289.290161pt}{280.186707pt}}
\pgfusepath{stroke}
\pgfpathmoveto{\pgfpoint{289.363129pt}{280.262817pt}}
\pgflineto{\pgfpoint{289.440765pt}{280.115295pt}}
\pgfusepath{stroke}
\pgfpathmoveto{\pgfpoint{289.430450pt}{286.108154pt}}
\pgflineto{\pgfpoint{289.283813pt}{286.179016pt}}
\pgfusepath{stroke}
\pgfpathmoveto{\pgfpoint{289.355652pt}{286.252808pt}}
\pgflineto{\pgfpoint{289.430450pt}{286.108154pt}}
\pgfusepath{stroke}
\pgfpathmoveto{\pgfpoint{289.417999pt}{292.096802pt}}
\pgflineto{\pgfpoint{289.275635pt}{292.168762pt}}
\pgfusepath{stroke}
\pgfpathmoveto{\pgfpoint{289.347260pt}{292.239777pt}}
\pgflineto{\pgfpoint{289.417999pt}{292.096802pt}}
\pgfusepath{stroke}
\pgfpathmoveto{\pgfpoint{289.403259pt}{298.083435pt}}
\pgflineto{\pgfpoint{289.265686pt}{298.157410pt}}
\pgfusepath{stroke}
\pgfpathmoveto{\pgfpoint{289.337555pt}{298.225159pt}}
\pgflineto{\pgfpoint{289.403259pt}{298.083435pt}}
\pgfusepath{stroke}
\pgfpathmoveto{\pgfpoint{289.386475pt}{304.069641pt}}
\pgflineto{\pgfpoint{289.254333pt}{304.145996pt}}
\pgfusepath{stroke}
\pgfpathmoveto{\pgfpoint{289.326569pt}{304.210022pt}}
\pgflineto{\pgfpoint{289.386475pt}{304.069641pt}}
\pgfusepath{stroke}
\pgfpathmoveto{\pgfpoint{289.368317pt}{310.056519pt}}
\pgflineto{\pgfpoint{289.242096pt}{310.135162pt}}
\pgfusepath{stroke}
\pgfpathmoveto{\pgfpoint{289.314545pt}{310.195129pt}}
\pgflineto{\pgfpoint{289.368317pt}{310.056519pt}}
\pgfusepath{stroke}
\pgfpathmoveto{\pgfpoint{289.349426pt}{316.044617pt}}
\pgflineto{\pgfpoint{289.229553pt}{316.125244pt}}
\pgfusepath{stroke}
\pgfpathmoveto{\pgfpoint{289.301910pt}{316.181030pt}}
\pgflineto{\pgfpoint{289.349426pt}{316.044617pt}}
\pgfusepath{stroke}
\pgfpathmoveto{\pgfpoint{289.330475pt}{322.034241pt}}
\pgflineto{\pgfpoint{289.217133pt}{322.116364pt}}
\pgfusepath{stroke}
\pgfpathmoveto{\pgfpoint{289.289062pt}{322.167938pt}}
\pgflineto{\pgfpoint{289.330475pt}{322.034241pt}}
\pgfusepath{stroke}
\pgfpathmoveto{\pgfpoint{289.312012pt}{328.025330pt}}
\pgflineto{\pgfpoint{289.205200pt}{328.108398pt}}
\pgfusepath{stroke}
\pgfpathmoveto{\pgfpoint{289.276398pt}{328.155884pt}}
\pgflineto{\pgfpoint{289.312012pt}{328.025330pt}}
\pgfusepath{stroke}
\pgfpathmoveto{\pgfpoint{289.294403pt}{334.017761pt}}
\pgflineto{\pgfpoint{289.193970pt}{334.101227pt}}
\pgfusepath{stroke}
\pgfpathmoveto{\pgfpoint{289.264130pt}{334.144775pt}}
\pgflineto{\pgfpoint{289.294403pt}{334.017761pt}}
\pgfusepath{stroke}
\pgfpathmoveto{\pgfpoint{289.277893pt}{340.011292pt}}
\pgflineto{\pgfpoint{289.183624pt}{340.094666pt}}
\pgfusepath{stroke}
\pgfpathmoveto{\pgfpoint{289.252502pt}{340.134552pt}}
\pgflineto{\pgfpoint{289.277893pt}{340.011292pt}}
\pgfusepath{stroke}
\pgfpathmoveto{\pgfpoint{289.262634pt}{346.005646pt}}
\pgflineto{\pgfpoint{289.174194pt}{346.088501pt}}
\pgfusepath{stroke}
\pgfpathmoveto{\pgfpoint{289.241577pt}{346.125000pt}}
\pgflineto{\pgfpoint{289.262634pt}{346.005646pt}}
\pgfusepath{stroke}
\pgfpathmoveto{\pgfpoint{289.248657pt}{352.000610pt}}
\pgflineto{\pgfpoint{289.165649pt}{352.082581pt}}
\pgfusepath{stroke}
\pgfpathmoveto{\pgfpoint{289.231476pt}{352.115997pt}}
\pgflineto{\pgfpoint{289.248657pt}{352.000610pt}}
\pgfusepath{stroke}
\pgfpathmoveto{\pgfpoint{289.235962pt}{357.995911pt}}
\pgflineto{\pgfpoint{289.158020pt}{358.076813pt}}
\pgfusepath{stroke}
\pgfpathmoveto{\pgfpoint{289.222168pt}{358.107391pt}}
\pgflineto{\pgfpoint{289.235962pt}{357.995911pt}}
\pgfusepath{stroke}
\pgfpathmoveto{\pgfpoint{289.224457pt}{363.991425pt}}
\pgflineto{\pgfpoint{289.151215pt}{364.071014pt}}
\pgfusepath{stroke}
\pgfpathmoveto{\pgfpoint{289.213623pt}{364.099060pt}}
\pgflineto{\pgfpoint{289.224457pt}{363.991425pt}}
\pgfusepath{stroke}
\pgfpathmoveto{\pgfpoint{289.214050pt}{369.987030pt}}
\pgflineto{\pgfpoint{289.145172pt}{370.065186pt}}
\pgfusepath{stroke}
\pgfpathmoveto{\pgfpoint{289.205811pt}{370.090881pt}}
\pgflineto{\pgfpoint{289.214050pt}{369.987030pt}}
\pgfusepath{stroke}
\pgfpathmoveto{\pgfpoint{295.115112pt}{77.080994pt}}
\pgflineto{\pgfpoint{295.130005pt}{77.004105pt}}
\pgfusepath{stroke}
\pgfpathmoveto{\pgfpoint{295.080902pt}{77.010529pt}}
\pgflineto{\pgfpoint{295.115112pt}{77.080994pt}}
\pgfusepath{stroke}
\pgfpathmoveto{\pgfpoint{295.116821pt}{83.076599pt}}
\pgflineto{\pgfpoint{295.132019pt}{82.996902pt}}
\pgfusepath{stroke}
\pgfpathmoveto{\pgfpoint{295.081146pt}{83.003708pt}}
\pgflineto{\pgfpoint{295.116821pt}{83.076599pt}}
\pgfusepath{stroke}
\pgfpathmoveto{\pgfpoint{295.118652pt}{89.072800pt}}
\pgflineto{\pgfpoint{295.134186pt}{88.990082pt}}
\pgfusepath{stroke}
\pgfpathmoveto{\pgfpoint{295.081451pt}{88.997314pt}}
\pgflineto{\pgfpoint{295.118652pt}{89.072800pt}}
\pgfusepath{stroke}
\pgfpathmoveto{\pgfpoint{295.120697pt}{95.069687pt}}
\pgflineto{\pgfpoint{295.136536pt}{94.983696pt}}
\pgfusepath{stroke}
\pgfpathmoveto{\pgfpoint{295.081787pt}{94.991386pt}}
\pgflineto{\pgfpoint{295.120697pt}{95.069687pt}}
\pgfusepath{stroke}
\pgfpathmoveto{\pgfpoint{295.122925pt}{101.067337pt}}
\pgflineto{\pgfpoint{295.139130pt}{100.977806pt}}
\pgfusepath{stroke}
\pgfpathmoveto{\pgfpoint{295.082153pt}{100.985977pt}}
\pgflineto{\pgfpoint{295.122925pt}{101.067337pt}}
\pgfusepath{stroke}
\pgfpathmoveto{\pgfpoint{295.125366pt}{107.065842pt}}
\pgflineto{\pgfpoint{295.141937pt}{106.972450pt}}
\pgfusepath{stroke}
\pgfpathmoveto{\pgfpoint{295.082611pt}{106.981178pt}}
\pgflineto{\pgfpoint{295.125366pt}{107.065842pt}}
\pgfusepath{stroke}
\pgfpathmoveto{\pgfpoint{295.128052pt}{113.065361pt}}
\pgflineto{\pgfpoint{295.145050pt}{112.967743pt}}
\pgfusepath{stroke}
\pgfpathmoveto{\pgfpoint{295.083099pt}{112.977066pt}}
\pgflineto{\pgfpoint{295.128052pt}{113.065361pt}}
\pgfusepath{stroke}
\pgfpathmoveto{\pgfpoint{295.131042pt}{119.066017pt}}
\pgflineto{\pgfpoint{295.148468pt}{118.963776pt}}
\pgfusepath{stroke}
\pgfpathmoveto{\pgfpoint{295.083649pt}{118.973755pt}}
\pgflineto{\pgfpoint{295.131042pt}{119.066017pt}}
\pgfusepath{stroke}
\pgfpathmoveto{\pgfpoint{295.134369pt}{125.068024pt}}
\pgflineto{\pgfpoint{295.152283pt}{124.960663pt}}
\pgfusepath{stroke}
\pgfpathmoveto{\pgfpoint{295.084290pt}{124.971382pt}}
\pgflineto{\pgfpoint{295.134369pt}{125.068024pt}}
\pgfusepath{stroke}
\pgfpathmoveto{\pgfpoint{295.138092pt}{131.071609pt}}
\pgflineto{\pgfpoint{295.156525pt}{130.958557pt}}
\pgfusepath{stroke}
\pgfpathmoveto{\pgfpoint{295.084991pt}{130.970108pt}}
\pgflineto{\pgfpoint{295.138092pt}{131.071609pt}}
\pgfusepath{stroke}
\pgfpathmoveto{\pgfpoint{295.142273pt}{137.077072pt}}
\pgflineto{\pgfpoint{295.161285pt}{136.957657pt}}
\pgfusepath{stroke}
\pgfpathmoveto{\pgfpoint{295.085815pt}{136.970139pt}}
\pgflineto{\pgfpoint{295.142273pt}{137.077072pt}}
\pgfusepath{stroke}
\pgfpathmoveto{\pgfpoint{295.147003pt}{143.084808pt}}
\pgflineto{\pgfpoint{295.166656pt}{142.958206pt}}
\pgfusepath{stroke}
\pgfpathmoveto{\pgfpoint{295.086761pt}{142.971741pt}}
\pgflineto{\pgfpoint{295.147003pt}{143.084808pt}}
\pgfusepath{stroke}
\pgfpathmoveto{\pgfpoint{295.152435pt}{149.095337pt}}
\pgflineto{\pgfpoint{295.172791pt}{148.960541pt}}
\pgfusepath{stroke}
\pgfpathmoveto{\pgfpoint{295.087830pt}{148.975266pt}}
\pgflineto{\pgfpoint{295.152435pt}{149.095337pt}}
\pgfusepath{stroke}
\pgfpathmoveto{\pgfpoint{295.158661pt}{155.109299pt}}
\pgflineto{\pgfpoint{295.179871pt}{154.965088pt}}
\pgfusepath{stroke}
\pgfpathmoveto{\pgfpoint{295.089081pt}{154.981186pt}}
\pgflineto{\pgfpoint{295.158661pt}{155.109299pt}}
\pgfusepath{stroke}
\pgfpathmoveto{\pgfpoint{295.165955pt}{161.127670pt}}
\pgflineto{\pgfpoint{295.188141pt}{160.972412pt}}
\pgfusepath{stroke}
\pgfpathmoveto{\pgfpoint{295.090546pt}{160.990173pt}}
\pgflineto{\pgfpoint{295.165955pt}{161.127670pt}}
\pgfusepath{stroke}
\pgfpathmoveto{\pgfpoint{295.174622pt}{167.151749pt}}
\pgflineto{\pgfpoint{295.197937pt}{166.983429pt}}
\pgfusepath{stroke}
\pgfpathmoveto{\pgfpoint{295.092285pt}{167.003098pt}}
\pgflineto{\pgfpoint{295.174622pt}{167.151749pt}}
\pgfusepath{stroke}
\pgfpathmoveto{\pgfpoint{295.185120pt}{173.183472pt}}
\pgflineto{\pgfpoint{295.209808pt}{172.999344pt}}
\pgfusepath{stroke}
\pgfpathmoveto{\pgfpoint{295.094391pt}{173.021347pt}}
\pgflineto{\pgfpoint{295.185120pt}{173.183472pt}}
\pgfusepath{stroke}
\pgfpathmoveto{\pgfpoint{295.198212pt}{179.225815pt}}
\pgflineto{\pgfpoint{295.224609pt}{179.022034pt}}
\pgfusepath{stroke}
\pgfpathmoveto{\pgfpoint{295.097076pt}{179.046936pt}}
\pgflineto{\pgfpoint{295.198212pt}{179.225815pt}}
\pgfusepath{stroke}
\pgfpathmoveto{\pgfpoint{295.215271pt}{185.283478pt}}
\pgflineto{\pgfpoint{295.243744pt}{185.054489pt}}
\pgfusepath{stroke}
\pgfpathmoveto{\pgfpoint{295.100677pt}{185.083191pt}}
\pgflineto{\pgfpoint{295.215271pt}{185.283478pt}}
\pgfusepath{stroke}
\pgfpathmoveto{\pgfpoint{295.238831pt}{191.364548pt}}
\pgflineto{\pgfpoint{295.269836pt}{191.101837pt}}
\pgfusepath{stroke}
\pgfpathmoveto{\pgfpoint{295.106018pt}{191.135773pt}}
\pgflineto{\pgfpoint{295.238831pt}{191.364548pt}}
\pgfusepath{stroke}
\pgfpathmoveto{\pgfpoint{295.274445pt}{197.484009pt}}
\pgflineto{\pgfpoint{295.308197pt}{197.173447pt}}
\pgfusepath{stroke}
\pgfpathmoveto{\pgfpoint{295.115112pt}{197.215286pt}}
\pgflineto{\pgfpoint{295.274445pt}{197.484009pt}}
\pgfusepath{stroke}
\pgfpathmoveto{\pgfpoint{295.336090pt}{203.672806pt}}
\pgflineto{\pgfpoint{295.371643pt}{203.288376pt}}
\pgfusepath{stroke}
\pgfpathmoveto{\pgfpoint{295.133881pt}{203.343933pt}}
\pgflineto{\pgfpoint{295.336090pt}{203.672806pt}}
\pgfusepath{stroke}
\pgfpathmoveto{\pgfpoint{295.468750pt}{210.005447pt}}
\pgflineto{\pgfpoint{295.498413pt}{209.491318pt}}
\pgfusepath{stroke}
\pgfpathmoveto{\pgfpoint{295.183990pt}{209.576355pt}}
\pgflineto{\pgfpoint{295.468750pt}{210.005447pt}}
\pgfusepath{stroke}
\pgfpathmoveto{\pgfpoint{295.884674pt}{216.699753pt}}
\pgflineto{\pgfpoint{295.854218pt}{215.903885pt}}
\pgfusepath{stroke}
\pgfpathmoveto{\pgfpoint{295.382782pt}{216.081345pt}}
\pgflineto{\pgfpoint{295.884674pt}{216.699753pt}}
\pgfusepath{stroke}
\pgfpathmoveto{\pgfpoint{298.516998pt}{223.996994pt}}
\pgflineto{\pgfpoint{297.754578pt}{222.472168pt}}
\pgfusepath{stroke}
\pgfpathmoveto{\pgfpoint{296.992157pt}{223.234589pt}}
\pgflineto{\pgfpoint{298.516998pt}{223.996994pt}}
\pgfusepath{stroke}
\pgfpathmoveto{\pgfpoint{298.533173pt}{223.439835pt}}
\pgflineto{\pgfpoint{297.038147pt}{224.094849pt}}
\pgfusepath{stroke}
\pgfpathmoveto{\pgfpoint{297.730164pt}{224.860855pt}}
\pgflineto{\pgfpoint{298.533173pt}{223.439835pt}}
\pgfusepath{stroke}
\pgfpathmoveto{\pgfpoint{295.933136pt}{230.733597pt}}
\pgflineto{\pgfpoint{295.449890pt}{231.242172pt}}
\pgfusepath{stroke}
\pgfpathmoveto{\pgfpoint{295.851685pt}{231.430389pt}}
\pgflineto{\pgfpoint{295.933136pt}{230.733597pt}}
\pgfusepath{stroke}
\pgfpathmoveto{\pgfpoint{295.549255pt}{237.420776pt}}
\pgflineto{\pgfpoint{295.271698pt}{237.738846pt}}
\pgfusepath{stroke}
\pgfpathmoveto{\pgfpoint{295.518066pt}{237.841766pt}}
\pgflineto{\pgfpoint{295.549255pt}{237.420776pt}}
\pgfusepath{stroke}
\pgfpathmoveto{\pgfpoint{295.448181pt}{243.742371pt}}
\pgflineto{\pgfpoint{295.241394pt}{243.960388pt}}
\pgfusepath{stroke}
\pgfpathmoveto{\pgfpoint{295.413574pt}{244.040848pt}}
\pgflineto{\pgfpoint{295.448181pt}{243.742371pt}}
\pgfusepath{stroke}
\pgfpathmoveto{\pgfpoint{295.417084pt}{249.915878pt}}
\pgflineto{\pgfpoint{295.241333pt}{250.075439pt}}
\pgfusepath{stroke}
\pgfpathmoveto{\pgfpoint{295.372223pt}{250.148987pt}}
\pgflineto{\pgfpoint{295.417084pt}{249.915878pt}}
\pgfusepath{stroke}
\pgfpathmoveto{\pgfpoint{295.410278pt}{256.015533pt}}
\pgflineto{\pgfpoint{295.249207pt}{256.138550pt}}
\pgfusepath{stroke}
\pgfpathmoveto{\pgfpoint{295.355225pt}{256.210602pt}}
\pgflineto{\pgfpoint{295.410278pt}{256.015533pt}}
\pgfusepath{stroke}
\pgfpathmoveto{\pgfpoint{295.412720pt}{262.072266pt}}
\pgflineto{\pgfpoint{295.258484pt}{262.172028pt}}
\pgfusepath{stroke}
\pgfpathmoveto{\pgfpoint{295.349182pt}{262.244598pt}}
\pgflineto{\pgfpoint{295.412720pt}{262.072266pt}}
\pgfusepath{stroke}
\pgfpathmoveto{\pgfpoint{295.417664pt}{268.101379pt}}
\pgflineto{\pgfpoint{295.266388pt}{268.186798pt}}
\pgfusepath{stroke}
\pgfpathmoveto{\pgfpoint{295.347900pt}{268.260468pt}}
\pgflineto{\pgfpoint{295.417664pt}{268.101379pt}}
\pgfusepath{stroke}
\pgfpathmoveto{\pgfpoint{295.421387pt}{274.111755pt}}
\pgflineto{\pgfpoint{295.271393pt}{274.189240pt}}
\pgfusepath{stroke}
\pgfpathmoveto{\pgfpoint{295.347870pt}{274.263733pt}}
\pgflineto{\pgfpoint{295.421387pt}{274.111755pt}}
\pgfusepath{stroke}
\pgfpathmoveto{\pgfpoint{295.421600pt}{280.109436pt}}
\pgflineto{\pgfpoint{295.272644pt}{280.183563pt}}
\pgfusepath{stroke}
\pgfpathmoveto{\pgfpoint{295.346893pt}{280.258118pt}}
\pgflineto{\pgfpoint{295.421600pt}{280.109436pt}}
\pgfusepath{stroke}
\pgfpathmoveto{\pgfpoint{295.417145pt}{286.098999pt}}
\pgflineto{\pgfpoint{295.269867pt}{286.173035pt}}
\pgfusepath{stroke}
\pgfpathmoveto{\pgfpoint{295.343719pt}{286.246582pt}}
\pgflineto{\pgfpoint{295.417145pt}{286.098999pt}}
\pgfusepath{stroke}
\pgfpathmoveto{\pgfpoint{295.407715pt}{292.084137pt}}
\pgflineto{\pgfpoint{295.263275pt}{292.160095pt}}
\pgfusepath{stroke}
\pgfpathmoveto{\pgfpoint{295.337738pt}{292.231567pt}}
\pgflineto{\pgfpoint{295.407715pt}{292.084137pt}}
\pgfusepath{stroke}
\pgfpathmoveto{\pgfpoint{295.393829pt}{298.067657pt}}
\pgflineto{\pgfpoint{295.253540pt}{298.146576pt}}
\pgfusepath{stroke}
\pgfpathmoveto{\pgfpoint{295.328949pt}{298.214935pt}}
\pgflineto{\pgfpoint{295.393829pt}{298.067657pt}}
\pgfusepath{stroke}
\pgfpathmoveto{\pgfpoint{295.376434pt}{304.051575pt}}
\pgflineto{\pgfpoint{295.241547pt}{304.133698pt}}
\pgfusepath{stroke}
\pgfpathmoveto{\pgfpoint{295.317780pt}{304.198212pt}}
\pgflineto{\pgfpoint{295.376434pt}{304.051575pt}}
\pgfusepath{stroke}
\pgfpathmoveto{\pgfpoint{295.356812pt}{310.037079pt}}
\pgflineto{\pgfpoint{295.228210pt}{310.122131pt}}
\pgfusepath{stroke}
\pgfpathmoveto{\pgfpoint{295.304962pt}{310.182281pt}}
\pgflineto{\pgfpoint{295.356812pt}{310.037079pt}}
\pgfusepath{stroke}
\pgfpathmoveto{\pgfpoint{295.336121pt}{316.024719pt}}
\pgflineto{\pgfpoint{295.214386pt}{316.112122pt}}
\pgfusepath{stroke}
\pgfpathmoveto{\pgfpoint{295.291168pt}{316.167664pt}}
\pgflineto{\pgfpoint{295.336121pt}{316.024719pt}}
\pgfusepath{stroke}
\pgfpathmoveto{\pgfpoint{295.315338pt}{322.014618pt}}
\pgflineto{\pgfpoint{295.200775pt}{322.103607pt}}
\pgfusepath{stroke}
\pgfpathmoveto{\pgfpoint{295.277069pt}{322.154541pt}}
\pgflineto{\pgfpoint{295.315338pt}{322.014618pt}}
\pgfusepath{stroke}
\pgfpathmoveto{\pgfpoint{295.295227pt}{328.006561pt}}
\pgflineto{\pgfpoint{295.187836pt}{328.096375pt}}
\pgfusepath{stroke}
\pgfpathmoveto{\pgfpoint{295.263214pt}{328.142822pt}}
\pgflineto{\pgfpoint{295.295227pt}{328.006561pt}}
\pgfusepath{stroke}
\pgfpathmoveto{\pgfpoint{295.276276pt}{334.000122pt}}
\pgflineto{\pgfpoint{295.175873pt}{334.090118pt}}
\pgfusepath{stroke}
\pgfpathmoveto{\pgfpoint{295.249939pt}{334.132385pt}}
\pgflineto{\pgfpoint{295.276276pt}{334.000122pt}}
\pgfusepath{stroke}
\pgfpathmoveto{\pgfpoint{295.258759pt}{339.994995pt}}
\pgflineto{\pgfpoint{295.164978pt}{340.084534pt}}
\pgfusepath{stroke}
\pgfpathmoveto{\pgfpoint{295.237457pt}{340.122925pt}}
\pgflineto{\pgfpoint{295.258759pt}{339.994995pt}}
\pgfusepath{stroke}
\pgfpathmoveto{\pgfpoint{295.242798pt}{345.990723pt}}
\pgflineto{\pgfpoint{295.155212pt}{346.079407pt}}
\pgfusepath{stroke}
\pgfpathmoveto{\pgfpoint{295.225922pt}{346.114227pt}}
\pgflineto{\pgfpoint{295.242798pt}{345.990723pt}}
\pgfusepath{stroke}
\pgfpathmoveto{\pgfpoint{295.228363pt}{351.987061pt}}
\pgflineto{\pgfpoint{295.146545pt}{352.074493pt}}
\pgfusepath{stroke}
\pgfpathmoveto{\pgfpoint{295.215332pt}{352.106079pt}}
\pgflineto{\pgfpoint{295.228363pt}{351.987061pt}}
\pgfusepath{stroke}
\pgfpathmoveto{\pgfpoint{295.215393pt}{357.983704pt}}
\pgflineto{\pgfpoint{295.138885pt}{358.069611pt}}
\pgfusepath{stroke}
\pgfpathmoveto{\pgfpoint{295.205719pt}{358.098328pt}}
\pgflineto{\pgfpoint{295.215393pt}{357.983704pt}}
\pgfusepath{stroke}
\pgfpathmoveto{\pgfpoint{295.203796pt}{363.980469pt}}
\pgflineto{\pgfpoint{295.132141pt}{364.064636pt}}
\pgfusepath{stroke}
\pgfpathmoveto{\pgfpoint{295.196991pt}{364.090820pt}}
\pgflineto{\pgfpoint{295.203796pt}{363.980469pt}}
\pgfusepath{stroke}
\pgfpathmoveto{\pgfpoint{295.193451pt}{369.977203pt}}
\pgflineto{\pgfpoint{295.126251pt}{370.059540pt}}
\pgfusepath{stroke}
\pgfpathmoveto{\pgfpoint{295.189087pt}{370.083374pt}}
\pgflineto{\pgfpoint{295.193451pt}{369.977203pt}}
\pgfusepath{stroke}
\pgfpathmoveto{\pgfpoint{301.095245pt}{77.082352pt}}
\pgflineto{\pgfpoint{301.112823pt}{77.005844pt}}
\pgfusepath{stroke}
\pgfpathmoveto{\pgfpoint{301.063416pt}{77.010605pt}}
\pgflineto{\pgfpoint{301.095245pt}{77.082352pt}}
\pgfusepath{stroke}
\pgfpathmoveto{\pgfpoint{301.096375pt}{83.078064pt}}
\pgflineto{\pgfpoint{301.114441pt}{82.998795pt}}
\pgfusepath{stroke}
\pgfpathmoveto{\pgfpoint{301.063263pt}{83.003784pt}}
\pgflineto{\pgfpoint{301.096375pt}{83.078064pt}}
\pgfusepath{stroke}
\pgfpathmoveto{\pgfpoint{301.097565pt}{89.074394pt}}
\pgflineto{\pgfpoint{301.116211pt}{88.992126pt}}
\pgfusepath{stroke}
\pgfpathmoveto{\pgfpoint{301.063110pt}{88.997406pt}}
\pgflineto{\pgfpoint{301.097565pt}{89.074394pt}}
\pgfusepath{stroke}
\pgfpathmoveto{\pgfpoint{301.098877pt}{95.071426pt}}
\pgflineto{\pgfpoint{301.118103pt}{94.985916pt}}
\pgfusepath{stroke}
\pgfpathmoveto{\pgfpoint{301.062958pt}{94.991478pt}}
\pgflineto{\pgfpoint{301.098877pt}{95.071426pt}}
\pgfusepath{stroke}
\pgfpathmoveto{\pgfpoint{301.100281pt}{101.069214pt}}
\pgflineto{\pgfpoint{301.120148pt}{100.980209pt}}
\pgfusepath{stroke}
\pgfpathmoveto{\pgfpoint{301.062775pt}{100.986084pt}}
\pgflineto{\pgfpoint{301.100281pt}{101.069214pt}}
\pgfusepath{stroke}
\pgfpathmoveto{\pgfpoint{301.101776pt}{107.067902pt}}
\pgflineto{\pgfpoint{301.122375pt}{106.975090pt}}
\pgfusepath{stroke}
\pgfpathmoveto{\pgfpoint{301.062561pt}{106.981285pt}}
\pgflineto{\pgfpoint{301.101776pt}{107.067902pt}}
\pgfusepath{stroke}
\pgfpathmoveto{\pgfpoint{301.103394pt}{113.067612pt}}
\pgflineto{\pgfpoint{301.124786pt}{112.970627pt}}
\pgfusepath{stroke}
\pgfpathmoveto{\pgfpoint{301.062317pt}{112.977196pt}}
\pgflineto{\pgfpoint{301.103394pt}{113.067612pt}}
\pgfusepath{stroke}
\pgfpathmoveto{\pgfpoint{301.105103pt}{119.068489pt}}
\pgflineto{\pgfpoint{301.127380pt}{118.966949pt}}
\pgfusepath{stroke}
\pgfpathmoveto{\pgfpoint{301.062012pt}{118.973892pt}}
\pgflineto{\pgfpoint{301.105103pt}{119.068489pt}}
\pgfusepath{stroke}
\pgfpathmoveto{\pgfpoint{301.106964pt}{125.070740pt}}
\pgflineto{\pgfpoint{301.130249pt}{124.964165pt}}
\pgfusepath{stroke}
\pgfpathmoveto{\pgfpoint{301.061646pt}{124.971512pt}}
\pgflineto{\pgfpoint{301.106964pt}{125.070740pt}}
\pgfusepath{stroke}
\pgfpathmoveto{\pgfpoint{301.108948pt}{131.074615pt}}
\pgflineto{\pgfpoint{301.133331pt}{130.962433pt}}
\pgfusepath{stroke}
\pgfpathmoveto{\pgfpoint{301.061157pt}{130.970230pt}}
\pgflineto{\pgfpoint{301.108948pt}{131.074615pt}}
\pgfusepath{stroke}
\pgfpathmoveto{\pgfpoint{301.111053pt}{137.080383pt}}
\pgflineto{\pgfpoint{301.136749pt}{136.961975pt}}
\pgfusepath{stroke}
\pgfpathmoveto{\pgfpoint{301.060547pt}{136.970230pt}}
\pgflineto{\pgfpoint{301.111053pt}{137.080383pt}}
\pgfusepath{stroke}
\pgfpathmoveto{\pgfpoint{301.113251pt}{143.088470pt}}
\pgflineto{\pgfpoint{301.140472pt}{142.963043pt}}
\pgfusepath{stroke}
\pgfpathmoveto{\pgfpoint{301.059784pt}{142.971786pt}}
\pgflineto{\pgfpoint{301.113251pt}{143.088470pt}}
\pgfusepath{stroke}
\pgfpathmoveto{\pgfpoint{301.115540pt}{149.099396pt}}
\pgflineto{\pgfpoint{301.144562pt}{148.965973pt}}
\pgfusepath{stroke}
\pgfpathmoveto{\pgfpoint{301.058716pt}{148.975250pt}}
\pgflineto{\pgfpoint{301.115540pt}{149.099396pt}}
\pgfusepath{stroke}
\pgfpathmoveto{\pgfpoint{301.117859pt}{155.113815pt}}
\pgflineto{\pgfpoint{301.149078pt}{154.971252pt}}
\pgfusepath{stroke}
\pgfpathmoveto{\pgfpoint{301.057312pt}{154.981033pt}}
\pgflineto{\pgfpoint{301.117859pt}{155.113815pt}}
\pgfusepath{stroke}
\pgfpathmoveto{\pgfpoint{301.120087pt}{161.132706pt}}
\pgflineto{\pgfpoint{301.154022pt}{160.979523pt}}
\pgfusepath{stroke}
\pgfpathmoveto{\pgfpoint{301.055328pt}{160.989792pt}}
\pgflineto{\pgfpoint{301.120087pt}{161.132706pt}}
\pgfusepath{stroke}
\pgfpathmoveto{\pgfpoint{301.122101pt}{167.157379pt}}
\pgflineto{\pgfpoint{301.159454pt}{166.991669pt}}
\pgfusepath{stroke}
\pgfpathmoveto{\pgfpoint{301.052551pt}{167.002380pt}}
\pgflineto{\pgfpoint{301.122101pt}{167.157379pt}}
\pgfusepath{stroke}
\pgfpathmoveto{\pgfpoint{301.123505pt}{173.189789pt}}
\pgflineto{\pgfpoint{301.165375pt}{173.009033pt}}
\pgfusepath{stroke}
\pgfpathmoveto{\pgfpoint{301.048553pt}{173.020065pt}}
\pgflineto{\pgfpoint{301.123505pt}{173.189789pt}}
\pgfusepath{stroke}
\pgfpathmoveto{\pgfpoint{301.123779pt}{179.232910pt}}
\pgflineto{\pgfpoint{301.171722pt}{179.033661pt}}
\pgfusepath{stroke}
\pgfpathmoveto{\pgfpoint{301.042603pt}{179.044754pt}}
\pgflineto{\pgfpoint{301.123779pt}{179.232910pt}}
\pgfusepath{stroke}
\pgfpathmoveto{\pgfpoint{301.121765pt}{185.291458pt}}
\pgflineto{\pgfpoint{301.178223pt}{185.068848pt}}
\pgfusepath{stroke}
\pgfpathmoveto{\pgfpoint{301.033386pt}{185.079483pt}}
\pgflineto{\pgfpoint{301.121765pt}{185.291458pt}}
\pgfusepath{stroke}
\pgfpathmoveto{\pgfpoint{301.115143pt}{191.373535pt}}
\pgflineto{\pgfpoint{301.184296pt}{191.120224pt}}
\pgfusepath{stroke}
\pgfpathmoveto{\pgfpoint{301.018463pt}{191.129395pt}}
\pgflineto{\pgfpoint{301.115143pt}{191.373535pt}}
\pgfusepath{stroke}
\pgfpathmoveto{\pgfpoint{301.098633pt}{197.494156pt}}
\pgflineto{\pgfpoint{301.188049pt}{197.198380pt}}
\pgfusepath{stroke}
\pgfpathmoveto{\pgfpoint{300.992706pt}{197.203888pt}}
\pgflineto{\pgfpoint{301.098633pt}{197.494156pt}}
\pgfusepath{stroke}
\pgfpathmoveto{\pgfpoint{301.058533pt}{203.684250pt}}
\pgflineto{\pgfpoint{301.183807pt}{203.325485pt}}
\pgfusepath{stroke}
\pgfpathmoveto{\pgfpoint{300.943481pt}{203.322083pt}}
\pgflineto{\pgfpoint{301.058533pt}{203.684250pt}}
\pgfusepath{stroke}
\pgfpathmoveto{\pgfpoint{300.950134pt}{210.018341pt}}
\pgflineto{\pgfpoint{301.150024pt}{209.556183pt}}
\pgfusepath{stroke}
\pgfpathmoveto{\pgfpoint{300.832764pt}{209.528687pt}}
\pgflineto{\pgfpoint{300.950134pt}{210.018341pt}}
\pgfusepath{stroke}
\pgfpathmoveto{\pgfpoint{300.561523pt}{216.714218pt}}
\pgflineto{\pgfpoint{300.969635pt}{216.059189pt}}
\pgfusepath{stroke}
\pgfpathmoveto{\pgfpoint{300.494995pt}{215.945343pt}}
\pgflineto{\pgfpoint{300.561523pt}{216.714218pt}}
\pgfusepath{stroke}
\pgfpathmoveto{\pgfpoint{297.959839pt}{224.013168pt}}
\pgflineto{\pgfpoint{299.380859pt}{223.210159pt}}
\pgfusepath{stroke}
\pgfpathmoveto{\pgfpoint{298.614838pt}{222.518143pt}}
\pgflineto{\pgfpoint{297.959839pt}{224.013168pt}}
\pgfusepath{stroke}
\pgfpathmoveto{\pgfpoint{297.977722pt}{223.457733pt}}
\pgflineto{\pgfpoint{298.665771pt}{224.833801pt}}
\pgfusepath{stroke}
\pgfpathmoveto{\pgfpoint{299.353790pt}{224.145767pt}}
\pgflineto{\pgfpoint{297.977722pt}{223.457733pt}}
\pgfusepath{stroke}
\pgfpathmoveto{\pgfpoint{300.615295pt}{230.753174pt}}
\pgflineto{\pgfpoint{300.569427pt}{231.400284pt}}
\pgfusepath{stroke}
\pgfpathmoveto{\pgfpoint{300.966858pt}{231.298370pt}}
\pgflineto{\pgfpoint{300.615295pt}{230.753174pt}}
\pgfusepath{stroke}
\pgfpathmoveto{\pgfpoint{301.039856pt}{237.441788pt}}
\pgflineto{\pgfpoint{300.930359pt}{237.808090pt}}
\pgfusepath{stroke}
\pgfpathmoveto{\pgfpoint{301.172058pt}{237.800537pt}}
\pgflineto{\pgfpoint{301.039856pt}{237.441788pt}}
\pgfusepath{stroke}
\pgfpathmoveto{\pgfpoint{301.184143pt}{243.764328pt}}
\pgflineto{\pgfpoint{301.063751pt}{244.003021pt}}
\pgfusepath{stroke}
\pgfpathmoveto{\pgfpoint{301.231049pt}{244.027527pt}}
\pgflineto{\pgfpoint{301.184143pt}{243.764328pt}}
\pgfusepath{stroke}
\pgfpathmoveto{\pgfpoint{301.259644pt}{249.937988pt}}
\pgflineto{\pgfpoint{301.134735pt}{250.106323pt}}
\pgfusepath{stroke}
\pgfpathmoveto{\pgfpoint{301.260712pt}{250.147598pt}}
\pgflineto{\pgfpoint{301.259644pt}{249.937988pt}}
\pgfusepath{stroke}
\pgfpathmoveto{\pgfpoint{301.310089pt}{256.036621pt}}
\pgflineto{\pgfpoint{301.180695pt}{256.162384pt}}
\pgfusepath{stroke}
\pgfpathmoveto{\pgfpoint{301.282013pt}{256.214874pt}}
\pgflineto{\pgfpoint{301.310089pt}{256.036621pt}}
\pgfusepath{stroke}
\pgfpathmoveto{\pgfpoint{301.347870pt}{262.090851pt}}
\pgflineto{\pgfpoint{301.213257pt}{262.190247pt}}
\pgfusepath{stroke}
\pgfpathmoveto{\pgfpoint{301.299805pt}{262.251160pt}}
\pgflineto{\pgfpoint{301.347870pt}{262.090851pt}}
\pgfusepath{stroke}
\pgfpathmoveto{\pgfpoint{301.376465pt}{268.115814pt}}
\pgflineto{\pgfpoint{301.236450pt}{268.199646pt}}
\pgfusepath{stroke}
\pgfpathmoveto{\pgfpoint{301.314758pt}{268.266876pt}}
\pgflineto{\pgfpoint{301.376465pt}{268.115814pt}}
\pgfusepath{stroke}
\pgfpathmoveto{\pgfpoint{301.396393pt}{274.120544pt}}
\pgflineto{\pgfpoint{301.251617pt}{274.196503pt}}
\pgfusepath{stroke}
\pgfpathmoveto{\pgfpoint{301.326141pt}{274.268158pt}}
\pgflineto{\pgfpoint{301.396393pt}{274.120544pt}}
\pgfusepath{stroke}
\pgfpathmoveto{\pgfpoint{301.407257pt}{280.111511pt}}
\pgflineto{\pgfpoint{301.259216pt}{280.185211pt}}
\pgfusepath{stroke}
\pgfpathmoveto{\pgfpoint{301.333038pt}{280.259277pt}}
\pgflineto{\pgfpoint{301.407257pt}{280.111511pt}}
\pgfusepath{stroke}
\pgfpathmoveto{\pgfpoint{301.408905pt}{286.094177pt}}
\pgflineto{\pgfpoint{301.259766pt}{286.169373pt}}
\pgfusepath{stroke}
\pgfpathmoveto{\pgfpoint{301.334717pt}{286.243805pt}}
\pgflineto{\pgfpoint{301.408905pt}{286.094177pt}}
\pgfusepath{stroke}
\pgfpathmoveto{\pgfpoint{301.401978pt}{292.072937pt}}
\pgflineto{\pgfpoint{301.254150pt}{292.151917pt}}
\pgfusepath{stroke}
\pgfpathmoveto{\pgfpoint{301.331085pt}{292.224823pt}}
\pgflineto{\pgfpoint{301.401978pt}{292.072937pt}}
\pgfusepath{stroke}
\pgfpathmoveto{\pgfpoint{301.387909pt}{298.051392pt}}
\pgflineto{\pgfpoint{301.243713pt}{298.135010pt}}
\pgfusepath{stroke}
\pgfpathmoveto{\pgfpoint{301.322723pt}{298.204803pt}}
\pgflineto{\pgfpoint{301.387909pt}{298.051392pt}}
\pgfusepath{stroke}
\pgfpathmoveto{\pgfpoint{301.368652pt}{304.031799pt}}
\pgflineto{\pgfpoint{301.230072pt}{304.119995pt}}
\pgfusepath{stroke}
\pgfpathmoveto{\pgfpoint{301.310730pt}{304.185516pt}}
\pgflineto{\pgfpoint{301.368652pt}{304.031799pt}}
\pgfusepath{stroke}
\pgfpathmoveto{\pgfpoint{301.346344pt}{310.015411pt}}
\pgflineto{\pgfpoint{301.214752pt}{310.107483pt}}
\pgfusepath{stroke}
\pgfpathmoveto{\pgfpoint{301.296295pt}{310.168030pt}}
\pgflineto{\pgfpoint{301.346344pt}{310.015411pt}}
\pgfusepath{stroke}
\pgfpathmoveto{\pgfpoint{301.322815pt}{316.002563pt}}
\pgflineto{\pgfpoint{301.198975pt}{316.097443pt}}
\pgfusepath{stroke}
\pgfpathmoveto{\pgfpoint{301.280701pt}{316.152771pt}}
\pgflineto{\pgfpoint{301.322815pt}{316.002563pt}}
\pgfusepath{stroke}
\pgfpathmoveto{\pgfpoint{301.299469pt}{321.992981pt}}
\pgflineto{\pgfpoint{301.183716pt}{322.089539pt}}
\pgfusepath{stroke}
\pgfpathmoveto{\pgfpoint{301.264832pt}{322.139709pt}}
\pgflineto{\pgfpoint{301.299469pt}{321.992981pt}}
\pgfusepath{stroke}
\pgfpathmoveto{\pgfpoint{301.277283pt}{327.986053pt}}
\pgflineto{\pgfpoint{301.169495pt}{328.083344pt}}
\pgfusepath{stroke}
\pgfpathmoveto{\pgfpoint{301.249420pt}{328.128540pt}}
\pgflineto{\pgfpoint{301.277283pt}{327.986053pt}}
\pgfusepath{stroke}
\pgfpathmoveto{\pgfpoint{301.256714pt}{333.981201pt}}
\pgflineto{\pgfpoint{301.156616pt}{334.078308pt}}
\pgfusepath{stroke}
\pgfpathmoveto{\pgfpoint{301.234924pt}{334.118927pt}}
\pgflineto{\pgfpoint{301.256714pt}{333.981201pt}}
\pgfusepath{stroke}
\pgfpathmoveto{\pgfpoint{301.238068pt}{339.977783pt}}
\pgflineto{\pgfpoint{301.145172pt}{340.074005pt}}
\pgfusepath{stroke}
\pgfpathmoveto{\pgfpoint{301.221497pt}{340.110504pt}}
\pgflineto{\pgfpoint{301.238068pt}{339.977783pt}}
\pgfusepath{stroke}
\pgfpathmoveto{\pgfpoint{301.221375pt}{345.975281pt}}
\pgflineto{\pgfpoint{301.135132pt}{346.070099pt}}
\pgfusepath{stroke}
\pgfpathmoveto{\pgfpoint{301.209290pt}{346.102905pt}}
\pgflineto{\pgfpoint{301.221375pt}{345.975281pt}}
\pgfusepath{stroke}
\pgfpathmoveto{\pgfpoint{301.206543pt}{351.973236pt}}
\pgflineto{\pgfpoint{301.126373pt}{352.066345pt}}
\pgfusepath{stroke}
\pgfpathmoveto{\pgfpoint{301.198273pt}{352.095825pt}}
\pgflineto{\pgfpoint{301.206543pt}{351.973236pt}}
\pgfusepath{stroke}
\pgfpathmoveto{\pgfpoint{301.193451pt}{357.971436pt}}
\pgflineto{\pgfpoint{301.118805pt}{358.062500pt}}
\pgfusepath{stroke}
\pgfpathmoveto{\pgfpoint{301.188385pt}{358.089050pt}}
\pgflineto{\pgfpoint{301.193451pt}{357.971436pt}}
\pgfusepath{stroke}
\pgfpathmoveto{\pgfpoint{301.181885pt}{363.969574pt}}
\pgflineto{\pgfpoint{301.112244pt}{364.058472pt}}
\pgfusepath{stroke}
\pgfpathmoveto{\pgfpoint{301.179504pt}{364.082458pt}}
\pgflineto{\pgfpoint{301.181885pt}{363.969574pt}}
\pgfusepath{stroke}
\pgfpathmoveto{\pgfpoint{301.171722pt}{369.967529pt}}
\pgflineto{\pgfpoint{301.106598pt}{370.054138pt}}
\pgfusepath{stroke}
\pgfpathmoveto{\pgfpoint{301.171600pt}{370.075897pt}}
\pgflineto{\pgfpoint{301.171722pt}{369.967529pt}}
\pgfusepath{stroke}
\pgfpathmoveto{\pgfpoint{307.075287pt}{77.083160pt}}
\pgflineto{\pgfpoint{307.095520pt}{77.007248pt}}
\pgfusepath{stroke}
\pgfpathmoveto{\pgfpoint{307.045929pt}{77.010284pt}}
\pgflineto{\pgfpoint{307.075287pt}{77.083160pt}}
\pgfusepath{stroke}
\pgfpathmoveto{\pgfpoint{307.075836pt}{83.078934pt}}
\pgflineto{\pgfpoint{307.096771pt}{83.000290pt}}
\pgfusepath{stroke}
\pgfpathmoveto{\pgfpoint{307.045410pt}{83.003448pt}}
\pgflineto{\pgfpoint{307.075836pt}{83.078934pt}}
\pgfusepath{stroke}
\pgfpathmoveto{\pgfpoint{307.076385pt}{89.075302pt}}
\pgflineto{\pgfpoint{307.098114pt}{88.993729pt}}
\pgfusepath{stroke}
\pgfpathmoveto{\pgfpoint{307.044830pt}{88.997009pt}}
\pgflineto{\pgfpoint{307.076385pt}{89.075302pt}}
\pgfusepath{stroke}
\pgfpathmoveto{\pgfpoint{307.076965pt}{95.072365pt}}
\pgflineto{\pgfpoint{307.099518pt}{94.987640pt}}
\pgfusepath{stroke}
\pgfpathmoveto{\pgfpoint{307.044159pt}{94.991035pt}}
\pgflineto{\pgfpoint{307.076965pt}{95.072365pt}}
\pgfusepath{stroke}
\pgfpathmoveto{\pgfpoint{307.077515pt}{101.070229pt}}
\pgflineto{\pgfpoint{307.101013pt}{100.982040pt}}
\pgfusepath{stroke}
\pgfpathmoveto{\pgfpoint{307.043427pt}{100.985580pt}}
\pgflineto{\pgfpoint{307.077515pt}{101.070229pt}}
\pgfusepath{stroke}
\pgfpathmoveto{\pgfpoint{307.078064pt}{107.068947pt}}
\pgflineto{\pgfpoint{307.102600pt}{106.977058pt}}
\pgfusepath{stroke}
\pgfpathmoveto{\pgfpoint{307.042572pt}{106.980721pt}}
\pgflineto{\pgfpoint{307.078064pt}{107.068947pt}}
\pgfusepath{stroke}
\pgfpathmoveto{\pgfpoint{307.078613pt}{113.068695pt}}
\pgflineto{\pgfpoint{307.104309pt}{112.972748pt}}
\pgfusepath{stroke}
\pgfpathmoveto{\pgfpoint{307.041595pt}{112.976524pt}}
\pgflineto{\pgfpoint{307.078613pt}{113.068695pt}}
\pgfusepath{stroke}
\pgfpathmoveto{\pgfpoint{307.079071pt}{119.069595pt}}
\pgflineto{\pgfpoint{307.106079pt}{118.969223pt}}
\pgfusepath{stroke}
\pgfpathmoveto{\pgfpoint{307.040436pt}{118.973099pt}}
\pgflineto{\pgfpoint{307.079071pt}{119.069595pt}}
\pgfusepath{stroke}
\pgfpathmoveto{\pgfpoint{307.079437pt}{125.071854pt}}
\pgflineto{\pgfpoint{307.107941pt}{124.966599pt}}
\pgfusepath{stroke}
\pgfpathmoveto{\pgfpoint{307.039093pt}{124.970558pt}}
\pgflineto{\pgfpoint{307.079437pt}{125.071854pt}}
\pgfusepath{stroke}
\pgfpathmoveto{\pgfpoint{307.079681pt}{131.075684pt}}
\pgflineto{\pgfpoint{307.109863pt}{130.965057pt}}
\pgfusepath{stroke}
\pgfpathmoveto{\pgfpoint{307.037445pt}{130.969055pt}}
\pgflineto{\pgfpoint{307.079681pt}{131.075684pt}}
\pgfusepath{stroke}
\pgfpathmoveto{\pgfpoint{307.079712pt}{137.081390pt}}
\pgflineto{\pgfpoint{307.111877pt}{136.964767pt}}
\pgfusepath{stroke}
\pgfpathmoveto{\pgfpoint{307.035492pt}{136.968781pt}}
\pgflineto{\pgfpoint{307.079712pt}{137.081390pt}}
\pgfusepath{stroke}
\pgfpathmoveto{\pgfpoint{307.079407pt}{143.089325pt}}
\pgflineto{\pgfpoint{307.113922pt}{142.966003pt}}
\pgfusepath{stroke}
\pgfpathmoveto{\pgfpoint{307.033020pt}{142.969971pt}}
\pgflineto{\pgfpoint{307.079407pt}{143.089325pt}}
\pgfusepath{stroke}
\pgfpathmoveto{\pgfpoint{307.078613pt}{149.099976pt}}
\pgflineto{\pgfpoint{307.115936pt}{148.969116pt}}
\pgfusepath{stroke}
\pgfpathmoveto{\pgfpoint{307.029968pt}{148.972900pt}}
\pgflineto{\pgfpoint{307.078613pt}{149.099976pt}}
\pgfusepath{stroke}
\pgfpathmoveto{\pgfpoint{307.077087pt}{155.113937pt}}
\pgflineto{\pgfpoint{307.117859pt}{154.974503pt}}
\pgfusepath{stroke}
\pgfpathmoveto{\pgfpoint{307.026031pt}{154.977951pt}}
\pgflineto{\pgfpoint{307.077087pt}{155.113937pt}}
\pgfusepath{stroke}
\pgfpathmoveto{\pgfpoint{307.074463pt}{161.132050pt}}
\pgflineto{\pgfpoint{307.119446pt}{160.982788pt}}
\pgfusepath{stroke}
\pgfpathmoveto{\pgfpoint{307.020905pt}{160.985657pt}}
\pgflineto{\pgfpoint{307.074463pt}{161.132050pt}}
\pgfusepath{stroke}
\pgfpathmoveto{\pgfpoint{307.070099pt}{167.155441pt}}
\pgflineto{\pgfpoint{307.120514pt}{166.994797pt}}
\pgfusepath{stroke}
\pgfpathmoveto{\pgfpoint{307.014038pt}{166.996674pt}}
\pgflineto{\pgfpoint{307.070099pt}{167.155441pt}}
\pgfusepath{stroke}
\pgfpathmoveto{\pgfpoint{307.062988pt}{173.185684pt}}
\pgflineto{\pgfpoint{307.120514pt}{173.011658pt}}
\pgfusepath{stroke}
\pgfpathmoveto{\pgfpoint{307.004578pt}{173.011963pt}}
\pgflineto{\pgfpoint{307.062988pt}{173.185684pt}}
\pgfusepath{stroke}
\pgfpathmoveto{\pgfpoint{307.051483pt}{179.225067pt}}
\pgflineto{\pgfpoint{307.118561pt}{179.035110pt}}
\pgfusepath{stroke}
\pgfpathmoveto{\pgfpoint{306.991180pt}{179.032867pt}}
\pgflineto{\pgfpoint{307.051483pt}{179.225067pt}}
\pgfusepath{stroke}
\pgfpathmoveto{\pgfpoint{307.032501pt}{185.276932pt}}
\pgflineto{\pgfpoint{307.113037pt}{185.067719pt}}
\pgfusepath{stroke}
\pgfpathmoveto{\pgfpoint{306.971405pt}{185.061249pt}}
\pgflineto{\pgfpoint{307.032501pt}{185.276932pt}}
\pgfusepath{stroke}
\pgfpathmoveto{\pgfpoint{307.000244pt}{191.346298pt}}
\pgflineto{\pgfpoint{307.100586pt}{191.113464pt}}
\pgfusepath{stroke}
\pgfpathmoveto{\pgfpoint{306.940826pt}{191.099823pt}}
\pgflineto{\pgfpoint{307.000244pt}{191.346298pt}}
\pgfusepath{stroke}
\pgfpathmoveto{\pgfpoint{306.942841pt}{197.440659pt}}
\pgflineto{\pgfpoint{307.074158pt}{197.178665pt}}
\pgfusepath{stroke}
\pgfpathmoveto{\pgfpoint{306.890686pt}{197.152267pt}}
\pgflineto{\pgfpoint{306.942841pt}{197.440659pt}}
\pgfusepath{stroke}
\pgfpathmoveto{\pgfpoint{306.833618pt}{203.569717pt}}
\pgflineto{\pgfpoint{307.017059pt}{203.272766pt}}
\pgfusepath{stroke}
\pgfpathmoveto{\pgfpoint{306.802185pt}{203.222092pt}}
\pgflineto{\pgfpoint{306.833618pt}{203.569717pt}}
\pgfusepath{stroke}
\pgfpathmoveto{\pgfpoint{306.607544pt}{209.735764pt}}
\pgflineto{\pgfpoint{306.886139pt}{209.404495pt}}
\pgfusepath{stroke}
\pgfpathmoveto{\pgfpoint{306.631653pt}{209.303589pt}}
\pgflineto{\pgfpoint{306.607544pt}{209.735764pt}}
\pgfusepath{stroke}
\pgfpathmoveto{\pgfpoint{306.109100pt}{215.857132pt}}
\pgflineto{\pgfpoint{306.568695pt}{215.536713pt}}
\pgfusepath{stroke}
\pgfpathmoveto{\pgfpoint{306.284515pt}{215.325043pt}}
\pgflineto{\pgfpoint{306.109100pt}{215.857132pt}}
\pgfusepath{stroke}
\pgfpathmoveto{\pgfpoint{305.253601pt}{221.413132pt}}
\pgflineto{\pgfpoint{305.950378pt}{221.331680pt}}
\pgfusepath{stroke}
\pgfpathmoveto{\pgfpoint{305.762146pt}{220.929901pt}}
\pgflineto{\pgfpoint{305.253601pt}{221.413132pt}}
\pgfusepath{stroke}
\pgfpathmoveto{\pgfpoint{305.273193pt}{226.095306pt}}
\pgflineto{\pgfpoint{305.818359pt}{226.446869pt}}
\pgfusepath{stroke}
\pgfpathmoveto{\pgfpoint{305.920288pt}{226.049438pt}}
\pgflineto{\pgfpoint{305.273193pt}{226.095306pt}}
\pgfusepath{stroke}
\pgfpathmoveto{\pgfpoint{306.168060pt}{231.648056pt}}
\pgflineto{\pgfpoint{306.366608pt}{232.045197pt}}
\pgfusepath{stroke}
\pgfpathmoveto{\pgfpoint{306.565186pt}{231.846619pt}}
\pgflineto{\pgfpoint{306.168060pt}{231.648056pt}}
\pgfusepath{stroke}
\pgfpathmoveto{\pgfpoint{306.706482pt}{237.762512pt}}
\pgflineto{\pgfpoint{306.739655pt}{238.057587pt}}
\pgfusepath{stroke}
\pgfpathmoveto{\pgfpoint{306.910065pt}{237.978668pt}}
\pgflineto{\pgfpoint{306.706482pt}{237.762512pt}}
\pgfusepath{stroke}
\pgfpathmoveto{\pgfpoint{306.973267pt}{243.917175pt}}
\pgflineto{\pgfpoint{306.936035pt}{244.126984pt}}
\pgfusepath{stroke}
\pgfpathmoveto{\pgfpoint{307.069366pt}{244.107346pt}}
\pgflineto{\pgfpoint{306.973267pt}{243.917175pt}}
\pgfusepath{stroke}
\pgfpathmoveto{\pgfpoint{307.123535pt}{250.029297pt}}
\pgflineto{\pgfpoint{307.050049pt}{250.180954pt}}
\pgfusepath{stroke}
\pgfpathmoveto{\pgfpoint{307.155731pt}{250.194702pt}}
\pgflineto{\pgfpoint{307.123535pt}{250.029297pt}}
\pgfusepath{stroke}
\pgfpathmoveto{\pgfpoint{307.221405pt}{256.099945pt}}
\pgflineto{\pgfpoint{307.124512pt}{256.213074pt}}
\pgfusepath{stroke}
\pgfpathmoveto{\pgfpoint{307.211792pt}{256.248596pt}}
\pgflineto{\pgfpoint{307.221405pt}{256.099945pt}}
\pgfusepath{stroke}
\pgfpathmoveto{\pgfpoint{307.291748pt}{262.137817pt}}
\pgflineto{\pgfpoint{307.176971pt}{262.226440pt}}
\pgfusepath{stroke}
\pgfpathmoveto{\pgfpoint{307.253113pt}{262.277588pt}}
\pgflineto{\pgfpoint{307.291748pt}{262.137817pt}}
\pgfusepath{stroke}
\pgfpathmoveto{\pgfpoint{307.343842pt}{268.150116pt}}
\pgflineto{\pgfpoint{307.214417pt}{268.224762pt}}
\pgfusepath{stroke}
\pgfpathmoveto{\pgfpoint{307.285095pt}{268.287476pt}}
\pgflineto{\pgfpoint{307.343842pt}{268.150116pt}}
\pgfusepath{stroke}
\pgfpathmoveto{\pgfpoint{307.380524pt}{274.142853pt}}
\pgflineto{\pgfpoint{307.239441pt}{274.211792pt}}
\pgfusepath{stroke}
\pgfpathmoveto{\pgfpoint{307.309021pt}{274.282654pt}}
\pgflineto{\pgfpoint{307.380524pt}{274.142853pt}}
\pgfusepath{stroke}
\pgfpathmoveto{\pgfpoint{307.402283pt}{280.121857pt}}
\pgflineto{\pgfpoint{307.252960pt}{280.191284pt}}
\pgfusepath{stroke}
\pgfpathmoveto{\pgfpoint{307.324463pt}{280.266968pt}}
\pgflineto{\pgfpoint{307.402283pt}{280.121857pt}}
\pgfusepath{stroke}
\pgfpathmoveto{\pgfpoint{307.409424pt}{286.093018pt}}
\pgflineto{\pgfpoint{307.255890pt}{286.167175pt}}
\pgfusepath{stroke}
\pgfpathmoveto{\pgfpoint{307.331085pt}{286.244476pt}}
\pgflineto{\pgfpoint{307.409424pt}{286.093018pt}}
\pgfusepath{stroke}
\pgfpathmoveto{\pgfpoint{307.403473pt}{292.061920pt}}
\pgflineto{\pgfpoint{307.249817pt}{292.143005pt}}
\pgfusepath{stroke}
\pgfpathmoveto{\pgfpoint{307.329224pt}{292.218994pt}}
\pgflineto{\pgfpoint{307.403473pt}{292.061920pt}}
\pgfusepath{stroke}
\pgfpathmoveto{\pgfpoint{307.387146pt}{298.032898pt}}
\pgflineto{\pgfpoint{307.237091pt}{298.121399pt}}
\pgfusepath{stroke}
\pgfpathmoveto{\pgfpoint{307.320190pt}{298.193756pt}}
\pgflineto{\pgfpoint{307.387146pt}{298.032898pt}}
\pgfusepath{stroke}
\pgfpathmoveto{\pgfpoint{307.363892pt}{304.008667pt}}
\pgflineto{\pgfpoint{307.220245pt}{304.103760pt}}
\pgfusepath{stroke}
\pgfpathmoveto{\pgfpoint{307.306030pt}{304.170929pt}}
\pgflineto{\pgfpoint{307.363892pt}{304.008667pt}}
\pgfusepath{stroke}
\pgfpathmoveto{\pgfpoint{307.337036pt}{309.990143pt}}
\pgflineto{\pgfpoint{307.201630pt}{310.090302pt}}
\pgfusepath{stroke}
\pgfpathmoveto{\pgfpoint{307.288818pt}{310.151520pt}}
\pgflineto{\pgfpoint{307.337036pt}{309.990143pt}}
\pgfusepath{stroke}
\pgfpathmoveto{\pgfpoint{307.309265pt}{315.977081pt}}
\pgflineto{\pgfpoint{307.183044pt}{316.080627pt}}
\pgfusepath{stroke}
\pgfpathmoveto{\pgfpoint{307.270386pt}{316.135620pt}}
\pgflineto{\pgfpoint{307.309265pt}{315.977081pt}}
\pgfusepath{stroke}
\pgfpathmoveto{\pgfpoint{307.282440pt}{321.968597pt}}
\pgflineto{\pgfpoint{307.165558pt}{322.073853pt}}
\pgfusepath{stroke}
\pgfpathmoveto{\pgfpoint{307.252075pt}{322.122925pt}}
\pgflineto{\pgfpoint{307.282440pt}{321.968597pt}}
\pgfusepath{stroke}
\pgfpathmoveto{\pgfpoint{307.257599pt}{327.963562pt}}
\pgflineto{\pgfpoint{307.149811pt}{328.069153pt}}
\pgfusepath{stroke}
\pgfpathmoveto{\pgfpoint{307.234741pt}{328.112732pt}}
\pgflineto{\pgfpoint{307.257599pt}{327.963562pt}}
\pgfusepath{stroke}
\pgfpathmoveto{\pgfpoint{307.235229pt}{333.960876pt}}
\pgflineto{\pgfpoint{307.135956pt}{334.065796pt}}
\pgfusepath{stroke}
\pgfpathmoveto{\pgfpoint{307.218750pt}{334.104340pt}}
\pgflineto{\pgfpoint{307.235229pt}{333.960876pt}}
\pgfusepath{stroke}
\pgfpathmoveto{\pgfpoint{307.215424pt}{339.959656pt}}
\pgflineto{\pgfpoint{307.123993pt}{340.063110pt}}
\pgfusepath{stroke}
\pgfpathmoveto{\pgfpoint{307.204315pt}{340.097290pt}}
\pgflineto{\pgfpoint{307.215424pt}{339.959656pt}}
\pgfusepath{stroke}
\pgfpathmoveto{\pgfpoint{307.198090pt}{345.959290pt}}
\pgflineto{\pgfpoint{307.113770pt}{346.060669pt}}
\pgfusepath{stroke}
\pgfpathmoveto{\pgfpoint{307.191467pt}{346.091003pt}}
\pgflineto{\pgfpoint{307.198090pt}{345.959290pt}}
\pgfusepath{stroke}
\pgfpathmoveto{\pgfpoint{307.183044pt}{351.959229pt}}
\pgflineto{\pgfpoint{307.105072pt}{352.058228pt}}
\pgfusepath{stroke}
\pgfpathmoveto{\pgfpoint{307.180054pt}{352.085236pt}}
\pgflineto{\pgfpoint{307.183044pt}{351.959229pt}}
\pgfusepath{stroke}
\pgfpathmoveto{\pgfpoint{307.169952pt}{357.959167pt}}
\pgflineto{\pgfpoint{307.097717pt}{358.055573pt}}
\pgfusepath{stroke}
\pgfpathmoveto{\pgfpoint{307.169983pt}{358.079651pt}}
\pgflineto{\pgfpoint{307.169952pt}{357.959167pt}}
\pgfusepath{stroke}
\pgfpathmoveto{\pgfpoint{307.158630pt}{363.958862pt}}
\pgflineto{\pgfpoint{307.091461pt}{364.052551pt}}
\pgfusepath{stroke}
\pgfpathmoveto{\pgfpoint{307.161102pt}{364.074097pt}}
\pgflineto{\pgfpoint{307.158630pt}{363.958862pt}}
\pgfusepath{stroke}
\pgfpathmoveto{\pgfpoint{307.148773pt}{369.958191pt}}
\pgflineto{\pgfpoint{307.086182pt}{370.049103pt}}
\pgfusepath{stroke}
\pgfpathmoveto{\pgfpoint{307.153259pt}{370.068481pt}}
\pgflineto{\pgfpoint{307.148773pt}{369.958191pt}}
\pgfusepath{stroke}
\pgfpathmoveto{\pgfpoint{313.055298pt}{77.083435pt}}
\pgflineto{\pgfpoint{313.078156pt}{77.008286pt}}
\pgfusepath{stroke}
\pgfpathmoveto{\pgfpoint{313.028503pt}{77.009613pt}}
\pgflineto{\pgfpoint{313.055298pt}{77.083435pt}}
\pgfusepath{stroke}
\pgfpathmoveto{\pgfpoint{313.055298pt}{83.079178pt}}
\pgflineto{\pgfpoint{313.079041pt}{83.001373pt}}
\pgfusepath{stroke}
\pgfpathmoveto{\pgfpoint{313.027588pt}{83.002686pt}}
\pgflineto{\pgfpoint{313.055298pt}{83.079178pt}}
\pgfusepath{stroke}
\pgfpathmoveto{\pgfpoint{313.055206pt}{89.075531pt}}
\pgflineto{\pgfpoint{313.079926pt}{88.994873pt}}
\pgfusepath{stroke}
\pgfpathmoveto{\pgfpoint{313.026581pt}{88.996155pt}}
\pgflineto{\pgfpoint{313.055206pt}{89.075531pt}}
\pgfusepath{stroke}
\pgfpathmoveto{\pgfpoint{313.055054pt}{95.072556pt}}
\pgflineto{\pgfpoint{313.080872pt}{94.988823pt}}
\pgfusepath{stroke}
\pgfpathmoveto{\pgfpoint{313.025452pt}{94.990082pt}}
\pgflineto{\pgfpoint{313.055054pt}{95.072556pt}}
\pgfusepath{stroke}
\pgfpathmoveto{\pgfpoint{313.054779pt}{101.070351pt}}
\pgflineto{\pgfpoint{313.081787pt}{100.983299pt}}
\pgfusepath{stroke}
\pgfpathmoveto{\pgfpoint{313.024170pt}{100.984512pt}}
\pgflineto{\pgfpoint{313.054779pt}{101.070351pt}}
\pgfusepath{stroke}
\pgfpathmoveto{\pgfpoint{313.054413pt}{107.069000pt}}
\pgflineto{\pgfpoint{313.082764pt}{106.978363pt}}
\pgfusepath{stroke}
\pgfpathmoveto{\pgfpoint{313.022705pt}{106.979485pt}}
\pgflineto{\pgfpoint{313.054413pt}{107.069000pt}}
\pgfusepath{stroke}
\pgfpathmoveto{\pgfpoint{313.053864pt}{113.068619pt}}
\pgflineto{\pgfpoint{313.083710pt}{112.974091pt}}
\pgfusepath{stroke}
\pgfpathmoveto{\pgfpoint{313.021027pt}{112.975082pt}}
\pgflineto{\pgfpoint{313.053864pt}{113.068619pt}}
\pgfusepath{stroke}
\pgfpathmoveto{\pgfpoint{313.053101pt}{119.069374pt}}
\pgflineto{\pgfpoint{313.084656pt}{118.970589pt}}
\pgfusepath{stroke}
\pgfpathmoveto{\pgfpoint{313.019073pt}{118.971413pt}}
\pgflineto{\pgfpoint{313.053101pt}{119.069374pt}}
\pgfusepath{stroke}
\pgfpathmoveto{\pgfpoint{313.052063pt}{125.071396pt}}
\pgflineto{\pgfpoint{313.085541pt}{124.967972pt}}
\pgfusepath{stroke}
\pgfpathmoveto{\pgfpoint{313.016785pt}{124.968575pt}}
\pgflineto{\pgfpoint{313.052063pt}{125.071396pt}}
\pgfusepath{stroke}
\pgfpathmoveto{\pgfpoint{313.050659pt}{131.074921pt}}
\pgflineto{\pgfpoint{313.086365pt}{130.966400pt}}
\pgfusepath{stroke}
\pgfpathmoveto{\pgfpoint{313.014099pt}{130.966675pt}}
\pgflineto{\pgfpoint{313.050659pt}{131.074921pt}}
\pgfusepath{stroke}
\pgfpathmoveto{\pgfpoint{313.048706pt}{137.080200pt}}
\pgflineto{\pgfpoint{313.087006pt}{136.966064pt}}
\pgfusepath{stroke}
\pgfpathmoveto{\pgfpoint{313.010864pt}{136.965897pt}}
\pgflineto{\pgfpoint{313.048706pt}{137.080200pt}}
\pgfusepath{stroke}
\pgfpathmoveto{\pgfpoint{313.046051pt}{143.087509pt}}
\pgflineto{\pgfpoint{313.087402pt}{142.967148pt}}
\pgfusepath{stroke}
\pgfpathmoveto{\pgfpoint{313.006897pt}{142.966400pt}}
\pgflineto{\pgfpoint{313.046051pt}{143.087509pt}}
\pgfusepath{stroke}
\pgfpathmoveto{\pgfpoint{313.042419pt}{149.097275pt}}
\pgflineto{\pgfpoint{313.087433pt}{148.969971pt}}
\pgfusepath{stroke}
\pgfpathmoveto{\pgfpoint{313.002045pt}{148.968414pt}}
\pgflineto{\pgfpoint{313.042419pt}{149.097275pt}}
\pgfusepath{stroke}
\pgfpathmoveto{\pgfpoint{313.037415pt}{155.109955pt}}
\pgflineto{\pgfpoint{313.086884pt}{154.974899pt}}
\pgfusepath{stroke}
\pgfpathmoveto{\pgfpoint{312.995941pt}{154.972260pt}}
\pgflineto{\pgfpoint{313.037415pt}{155.109955pt}}
\pgfusepath{stroke}
\pgfpathmoveto{\pgfpoint{313.030487pt}{161.126205pt}}
\pgflineto{\pgfpoint{313.085419pt}{160.982422pt}}
\pgfusepath{stroke}
\pgfpathmoveto{\pgfpoint{312.988159pt}{160.978210pt}}
\pgflineto{\pgfpoint{313.030487pt}{161.126205pt}}
\pgfusepath{stroke}
\pgfpathmoveto{\pgfpoint{313.020721pt}{167.146759pt}}
\pgflineto{\pgfpoint{313.082550pt}{166.993134pt}}
\pgfusepath{stroke}
\pgfpathmoveto{\pgfpoint{312.977997pt}{166.986755pt}}
\pgflineto{\pgfpoint{313.020721pt}{167.146759pt}}
\pgfusepath{stroke}
\pgfpathmoveto{\pgfpoint{313.006714pt}{173.172623pt}}
\pgflineto{\pgfpoint{313.077423pt}{173.007843pt}}
\pgfusepath{stroke}
\pgfpathmoveto{\pgfpoint{312.964417pt}{172.998367pt}}
\pgflineto{\pgfpoint{313.006714pt}{173.172623pt}}
\pgfusepath{stroke}
\pgfpathmoveto{\pgfpoint{312.986237pt}{179.205017pt}}
\pgflineto{\pgfpoint{313.068756pt}{179.027634pt}}
\pgfusepath{stroke}
\pgfpathmoveto{\pgfpoint{312.945831pt}{179.013611pt}}
\pgflineto{\pgfpoint{312.986237pt}{179.205017pt}}
\pgfusepath{stroke}
\pgfpathmoveto{\pgfpoint{312.955597pt}{185.245300pt}}
\pgflineto{\pgfpoint{313.054169pt}{185.053818pt}}
\pgfusepath{stroke}
\pgfpathmoveto{\pgfpoint{312.919556pt}{185.032974pt}}
\pgflineto{\pgfpoint{312.955597pt}{185.245300pt}}
\pgfusepath{stroke}
\pgfpathmoveto{\pgfpoint{312.908386pt}{191.294586pt}}
\pgflineto{\pgfpoint{313.029510pt}{191.087845pt}}
\pgfusepath{stroke}
\pgfpathmoveto{\pgfpoint{312.881226pt}{191.056503pt}}
\pgflineto{\pgfpoint{312.908386pt}{191.294586pt}}
\pgfusepath{stroke}
\pgfpathmoveto{\pgfpoint{312.833344pt}{197.352234pt}}
\pgflineto{\pgfpoint{312.987274pt}{197.130524pt}}
\pgfusepath{stroke}
\pgfpathmoveto{\pgfpoint{312.823456pt}{197.082520pt}}
\pgflineto{\pgfpoint{312.833344pt}{197.352234pt}}
\pgfusepath{stroke}
\pgfpathmoveto{\pgfpoint{312.710968pt}{203.410950pt}}
\pgflineto{\pgfpoint{312.913574pt}{203.179184pt}}
\pgfusepath{stroke}
\pgfpathmoveto{\pgfpoint{312.733978pt}{203.103973pt}}
\pgflineto{\pgfpoint{312.710968pt}{203.410950pt}}
\pgfusepath{stroke}
\pgfpathmoveto{\pgfpoint{312.511719pt}{209.441742pt}}
\pgflineto{\pgfpoint{312.785522pt}{209.217773pt}}
\pgfusepath{stroke}
\pgfpathmoveto{\pgfpoint{312.596375pt}{209.098282pt}}
\pgflineto{\pgfpoint{312.511719pt}{209.441742pt}}
\pgfusepath{stroke}
\pgfpathmoveto{\pgfpoint{312.218994pt}{215.360077pt}}
\pgflineto{\pgfpoint{312.582642pt}{215.191772pt}}
\pgfusepath{stroke}
\pgfpathmoveto{\pgfpoint{312.408936pt}{215.007248pt}}
\pgflineto{\pgfpoint{312.218994pt}{215.360077pt}}
\pgfusepath{stroke}
\pgfpathmoveto{\pgfpoint{311.940765pt}{221.029266pt}}
\pgflineto{\pgfpoint{312.361755pt}{220.998062pt}}
\pgfusepath{stroke}
\pgfpathmoveto{\pgfpoint{312.258850pt}{220.751709pt}}
\pgflineto{\pgfpoint{311.940765pt}{221.029266pt}}
\pgfusepath{stroke}
\pgfpathmoveto{\pgfpoint{311.961792pt}{226.519867pt}}
\pgflineto{\pgfpoint{312.320526pt}{226.652054pt}}
\pgfusepath{stroke}
\pgfpathmoveto{\pgfpoint{312.328094pt}{226.410370pt}}
\pgflineto{\pgfpoint{311.961792pt}{226.519867pt}}
\pgfusepath{stroke}
\pgfpathmoveto{\pgfpoint{312.282501pt}{232.186478pt}}
\pgflineto{\pgfpoint{312.498657pt}{232.390076pt}}
\pgfusepath{stroke}
\pgfpathmoveto{\pgfpoint{312.577576pt}{232.219666pt}}
\pgflineto{\pgfpoint{312.282501pt}{232.186478pt}}
\pgfusepath{stroke}
\pgfpathmoveto{\pgfpoint{312.619080pt}{238.099091pt}}
\pgflineto{\pgfpoint{312.714691pt}{238.290329pt}}
\pgfusepath{stroke}
\pgfpathmoveto{\pgfpoint{312.810333pt}{238.194702pt}}
\pgflineto{\pgfpoint{312.619080pt}{238.099091pt}}
\pgfusepath{stroke}
\pgfpathmoveto{\pgfpoint{312.864105pt}{244.119797pt}}
\pgflineto{\pgfpoint{312.881714pt}{244.272827pt}}
\pgfusepath{stroke}
\pgfpathmoveto{\pgfpoint{312.970001pt}{244.231659pt}}
\pgflineto{\pgfpoint{312.864105pt}{244.119797pt}}
\pgfusepath{stroke}
\pgfpathmoveto{\pgfpoint{313.034210pt}{250.162354pt}}
\pgflineto{\pgfpoint{313.001221pt}{250.278229pt}}
\pgfusepath{stroke}
\pgfpathmoveto{\pgfpoint{313.077332pt}{250.274857pt}}
\pgflineto{\pgfpoint{313.034210pt}{250.162354pt}}
\pgfusepath{stroke}
\pgfpathmoveto{\pgfpoint{313.158020pt}{256.195618pt}}
\pgflineto{\pgfpoint{313.088806pt}{256.282562pt}}
\pgfusepath{stroke}
\pgfpathmoveto{\pgfpoint{313.154816pt}{256.306702pt}}
\pgflineto{\pgfpoint{313.158020pt}{256.195618pt}}
\pgfusepath{stroke}
\pgfpathmoveto{\pgfpoint{313.252899pt}{262.210114pt}}
\pgflineto{\pgfpoint{313.155029pt}{262.277588pt}}
\pgfusepath{stroke}
\pgfpathmoveto{\pgfpoint{313.215088pt}{262.322815pt}}
\pgflineto{\pgfpoint{313.252899pt}{262.210114pt}}
\pgfusepath{stroke}
\pgfpathmoveto{\pgfpoint{313.326233pt}{268.203979pt}}
\pgflineto{\pgfpoint{313.204590pt}{268.261292pt}}
\pgfusepath{stroke}
\pgfpathmoveto{\pgfpoint{313.263306pt}{268.322815pt}}
\pgflineto{\pgfpoint{313.326233pt}{268.203979pt}}
\pgfusepath{stroke}
\pgfpathmoveto{\pgfpoint{313.379608pt}{274.178986pt}}
\pgflineto{\pgfpoint{313.238770pt}{274.234589pt}}
\pgfusepath{stroke}
\pgfpathmoveto{\pgfpoint{313.300323pt}{274.307983pt}}
\pgflineto{\pgfpoint{313.379608pt}{274.178986pt}}
\pgfusepath{stroke}
\pgfpathmoveto{\pgfpoint{313.412079pt}{280.139954pt}}
\pgflineto{\pgfpoint{313.257446pt}{280.200867pt}}
\pgfusepath{stroke}
\pgfpathmoveto{\pgfpoint{313.324921pt}{280.281494pt}}
\pgflineto{\pgfpoint{313.412079pt}{280.139954pt}}
\pgfusepath{stroke}
\pgfpathmoveto{\pgfpoint{313.423309pt}{286.093933pt}}
\pgflineto{\pgfpoint{313.261169pt}{286.164886pt}}
\pgfusepath{stroke}
\pgfpathmoveto{\pgfpoint{313.336151pt}{286.247986pt}}
\pgflineto{\pgfpoint{313.423309pt}{286.093933pt}}
\pgfusepath{stroke}
\pgfpathmoveto{\pgfpoint{313.415497pt}{292.048462pt}}
\pgflineto{\pgfpoint{313.252258pt}{292.131378pt}}
\pgfusepath{stroke}
\pgfpathmoveto{\pgfpoint{313.334625pt}{292.212738pt}}
\pgflineto{\pgfpoint{313.415497pt}{292.048462pt}}
\pgfusepath{stroke}
\pgfpathmoveto{\pgfpoint{313.393311pt}{298.009399pt}}
\pgflineto{\pgfpoint{313.234497pt}{298.103699pt}}
\pgfusepath{stroke}
\pgfpathmoveto{\pgfpoint{313.322845pt}{298.180145pt}}
\pgflineto{\pgfpoint{313.393311pt}{298.009399pt}}
\pgfusepath{stroke}
\pgfpathmoveto{\pgfpoint{313.362610pt}{303.979614pt}}
\pgflineto{\pgfpoint{313.212067pt}{304.083160pt}}
\pgfusepath{stroke}
\pgfpathmoveto{\pgfpoint{313.304321pt}{304.152771pt}}
\pgflineto{\pgfpoint{313.362610pt}{303.979614pt}}
\pgfusepath{stroke}
\pgfpathmoveto{\pgfpoint{313.328461pt}{309.959290pt}}
\pgflineto{\pgfpoint{313.188416pt}{310.069336pt}}
\pgfusepath{stroke}
\pgfpathmoveto{\pgfpoint{313.282471pt}{310.131348pt}}
\pgflineto{\pgfpoint{313.328461pt}{309.959290pt}}
\pgfusepath{stroke}
\pgfpathmoveto{\pgfpoint{313.294647pt}{315.947021pt}}
\pgflineto{\pgfpoint{313.165863pt}{316.060852pt}}
\pgfusepath{stroke}
\pgfpathmoveto{\pgfpoint{313.259918pt}{316.115326pt}}
\pgflineto{\pgfpoint{313.294647pt}{315.947021pt}}
\pgfusepath{stroke}
\pgfpathmoveto{\pgfpoint{313.263306pt}{321.940796pt}}
\pgflineto{\pgfpoint{313.145630pt}{322.056091pt}}
\pgfusepath{stroke}
\pgfpathmoveto{\pgfpoint{313.238342pt}{322.103638pt}}
\pgflineto{\pgfpoint{313.263306pt}{321.940796pt}}
\pgfusepath{stroke}
\pgfpathmoveto{\pgfpoint{313.235382pt}{327.938660pt}}
\pgflineto{\pgfpoint{313.128143pt}{328.053711pt}}
\pgfusepath{stroke}
\pgfpathmoveto{\pgfpoint{313.218597pt}{328.095032pt}}
\pgflineto{\pgfpoint{313.235382pt}{327.938660pt}}
\pgfusepath{stroke}
\pgfpathmoveto{\pgfpoint{313.211090pt}{333.939056pt}}
\pgflineto{\pgfpoint{313.113373pt}{334.052551pt}}
\pgfusepath{stroke}
\pgfpathmoveto{\pgfpoint{313.201019pt}{334.088501pt}}
\pgflineto{\pgfpoint{313.211090pt}{333.939056pt}}
\pgfusepath{stroke}
\pgfpathmoveto{\pgfpoint{313.190277pt}{339.940796pt}}
\pgflineto{\pgfpoint{313.101044pt}{340.051941pt}}
\pgfusepath{stroke}
\pgfpathmoveto{\pgfpoint{313.185577pt}{340.083252pt}}
\pgflineto{\pgfpoint{313.190277pt}{339.940796pt}}
\pgfusepath{stroke}
\pgfpathmoveto{\pgfpoint{313.172577pt}{345.943024pt}}
\pgflineto{\pgfpoint{313.090851pt}{346.051331pt}}
\pgfusepath{stroke}
\pgfpathmoveto{\pgfpoint{313.172150pt}{346.078705pt}}
\pgflineto{\pgfpoint{313.172577pt}{345.943024pt}}
\pgfusepath{stroke}
\pgfpathmoveto{\pgfpoint{313.157532pt}{351.945282pt}}
\pgflineto{\pgfpoint{313.082428pt}{352.050415pt}}
\pgfusepath{stroke}
\pgfpathmoveto{\pgfpoint{313.160522pt}{352.074463pt}}
\pgflineto{\pgfpoint{313.157532pt}{351.945282pt}}
\pgfusepath{stroke}
\pgfpathmoveto{\pgfpoint{313.144745pt}{357.947205pt}}
\pgflineto{\pgfpoint{313.075470pt}{358.049011pt}}
\pgfusepath{stroke}
\pgfpathmoveto{\pgfpoint{313.150421pt}{358.070221pt}}
\pgflineto{\pgfpoint{313.144745pt}{357.947205pt}}
\pgfusepath{stroke}
\pgfpathmoveto{\pgfpoint{313.133850pt}{363.948547pt}}
\pgflineto{\pgfpoint{313.069733pt}{364.047058pt}}
\pgfusepath{stroke}
\pgfpathmoveto{\pgfpoint{313.141663pt}{364.065826pt}}
\pgflineto{\pgfpoint{313.133850pt}{363.948547pt}}
\pgfusepath{stroke}
\pgfpathmoveto{\pgfpoint{313.124573pt}{369.949280pt}}
\pgflineto{\pgfpoint{313.064972pt}{370.044495pt}}
\pgfusepath{stroke}
\pgfpathmoveto{\pgfpoint{313.134033pt}{370.061218pt}}
\pgflineto{\pgfpoint{313.124573pt}{369.949280pt}}
\pgfusepath{stroke}
\pgfpathmoveto{\pgfpoint{319.035400pt}{77.083145pt}}
\pgflineto{\pgfpoint{319.060760pt}{77.008957pt}}
\pgfusepath{stroke}
\pgfpathmoveto{\pgfpoint{319.011169pt}{77.008560pt}}
\pgflineto{\pgfpoint{319.035400pt}{77.083145pt}}
\pgfusepath{stroke}
\pgfpathmoveto{\pgfpoint{319.034790pt}{83.078827pt}}
\pgflineto{\pgfpoint{319.061249pt}{83.002045pt}}
\pgfusepath{stroke}
\pgfpathmoveto{\pgfpoint{319.009888pt}{83.001541pt}}
\pgflineto{\pgfpoint{319.034790pt}{83.078827pt}}
\pgfusepath{stroke}
\pgfpathmoveto{\pgfpoint{319.034088pt}{89.075081pt}}
\pgflineto{\pgfpoint{319.061707pt}{88.995552pt}}
\pgfusepath{stroke}
\pgfpathmoveto{\pgfpoint{319.008484pt}{88.994873pt}}
\pgflineto{\pgfpoint{319.034088pt}{89.075081pt}}
\pgfusepath{stroke}
\pgfpathmoveto{\pgfpoint{319.033234pt}{95.071983pt}}
\pgflineto{\pgfpoint{319.062164pt}{94.989502pt}}
\pgfusepath{stroke}
\pgfpathmoveto{\pgfpoint{319.006897pt}{94.988640pt}}
\pgflineto{\pgfpoint{319.033234pt}{95.071983pt}}
\pgfusepath{stroke}
\pgfpathmoveto{\pgfpoint{319.032166pt}{101.069611pt}}
\pgflineto{\pgfpoint{319.062561pt}{100.983963pt}}
\pgfusepath{stroke}
\pgfpathmoveto{\pgfpoint{319.005096pt}{100.982849pt}}
\pgflineto{\pgfpoint{319.032166pt}{101.069611pt}}
\pgfusepath{stroke}
\pgfpathmoveto{\pgfpoint{319.030914pt}{107.068062pt}}
\pgflineto{\pgfpoint{319.062897pt}{106.978989pt}}
\pgfusepath{stroke}
\pgfpathmoveto{\pgfpoint{319.003052pt}{106.977600pt}}
\pgflineto{\pgfpoint{319.030914pt}{107.068062pt}}
\pgfusepath{stroke}
\pgfpathmoveto{\pgfpoint{319.029358pt}{113.067429pt}}
\pgflineto{\pgfpoint{319.063171pt}{112.974655pt}}
\pgfusepath{stroke}
\pgfpathmoveto{\pgfpoint{319.000732pt}{112.972931pt}}
\pgflineto{\pgfpoint{319.029358pt}{113.067429pt}}
\pgfusepath{stroke}
\pgfpathmoveto{\pgfpoint{319.027435pt}{119.067856pt}}
\pgflineto{\pgfpoint{319.063293pt}{118.971077pt}}
\pgfusepath{stroke}
\pgfpathmoveto{\pgfpoint{318.998047pt}{118.968918pt}}
\pgflineto{\pgfpoint{319.027435pt}{119.067856pt}}
\pgfusepath{stroke}
\pgfpathmoveto{\pgfpoint{319.025055pt}{125.069458pt}}
\pgflineto{\pgfpoint{319.063232pt}{124.968323pt}}
\pgfusepath{stroke}
\pgfpathmoveto{\pgfpoint{318.994934pt}{124.965645pt}}
\pgflineto{\pgfpoint{319.025055pt}{125.069458pt}}
\pgfusepath{stroke}
\pgfpathmoveto{\pgfpoint{319.022095pt}{131.072418pt}}
\pgflineto{\pgfpoint{319.062958pt}{130.966553pt}}
\pgfusepath{stroke}
\pgfpathmoveto{\pgfpoint{318.991272pt}{130.963196pt}}
\pgflineto{\pgfpoint{319.022095pt}{131.072418pt}}
\pgfusepath{stroke}
\pgfpathmoveto{\pgfpoint{319.018372pt}{137.076950pt}}
\pgflineto{\pgfpoint{319.062347pt}{136.965897pt}}
\pgfusepath{stroke}
\pgfpathmoveto{\pgfpoint{318.986908pt}{136.961731pt}}
\pgflineto{\pgfpoint{319.018372pt}{137.076950pt}}
\pgfusepath{stroke}
\pgfpathmoveto{\pgfpoint{319.013641pt}{143.083282pt}}
\pgflineto{\pgfpoint{319.061249pt}{142.966553pt}}
\pgfusepath{stroke}
\pgfpathmoveto{\pgfpoint{318.981689pt}{142.961334pt}}
\pgflineto{\pgfpoint{319.013641pt}{143.083282pt}}
\pgfusepath{stroke}
\pgfpathmoveto{\pgfpoint{319.007568pt}{149.091675pt}}
\pgflineto{\pgfpoint{319.059479pt}{148.968750pt}}
\pgfusepath{stroke}
\pgfpathmoveto{\pgfpoint{318.975342pt}{148.962189pt}}
\pgflineto{\pgfpoint{319.007568pt}{149.091675pt}}
\pgfusepath{stroke}
\pgfpathmoveto{\pgfpoint{318.999664pt}{155.102478pt}}
\pgflineto{\pgfpoint{319.056793pt}{154.972748pt}}
\pgfusepath{stroke}
\pgfpathmoveto{\pgfpoint{318.967529pt}{154.964417pt}}
\pgflineto{\pgfpoint{318.999664pt}{155.102478pt}}
\pgfusepath{stroke}
\pgfpathmoveto{\pgfpoint{318.989288pt}{161.116058pt}}
\pgflineto{\pgfpoint{319.052734pt}{160.978867pt}}
\pgfusepath{stroke}
\pgfpathmoveto{\pgfpoint{318.957733pt}{160.968231pt}}
\pgflineto{\pgfpoint{318.989288pt}{161.116058pt}}
\pgfusepath{stroke}
\pgfpathmoveto{\pgfpoint{318.975403pt}{167.132782pt}}
\pgflineto{\pgfpoint{319.046722pt}{166.987488pt}}
\pgfusepath{stroke}
\pgfpathmoveto{\pgfpoint{318.945282pt}{166.973770pt}}
\pgflineto{\pgfpoint{318.975403pt}{167.132782pt}}
\pgfusepath{stroke}
\pgfpathmoveto{\pgfpoint{318.956635pt}{173.153076pt}}
\pgflineto{\pgfpoint{319.037811pt}{172.999008pt}}
\pgfusepath{stroke}
\pgfpathmoveto{\pgfpoint{318.929138pt}{172.981125pt}}
\pgflineto{\pgfpoint{318.956635pt}{173.153076pt}}
\pgfusepath{stroke}
\pgfpathmoveto{\pgfpoint{318.930756pt}{179.177078pt}}
\pgflineto{\pgfpoint{319.024567pt}{179.013809pt}}
\pgfusepath{stroke}
\pgfpathmoveto{\pgfpoint{318.907867pt}{178.990173pt}}
\pgflineto{\pgfpoint{318.930756pt}{179.177078pt}}
\pgfusepath{stroke}
\pgfpathmoveto{\pgfpoint{318.894501pt}{185.204483pt}}
\pgflineto{\pgfpoint{319.004822pt}{185.032028pt}}
\pgfusepath{stroke}
\pgfpathmoveto{\pgfpoint{318.879272pt}{185.000336pt}}
\pgflineto{\pgfpoint{318.894501pt}{185.204483pt}}
\pgfusepath{stroke}
\pgfpathmoveto{\pgfpoint{318.842957pt}{191.233643pt}}
\pgflineto{\pgfpoint{318.975067pt}{191.053116pt}}
\pgfusepath{stroke}
\pgfpathmoveto{\pgfpoint{318.840332pt}{191.009964pt}}
\pgflineto{\pgfpoint{318.842957pt}{191.233643pt}}
\pgfusepath{stroke}
\pgfpathmoveto{\pgfpoint{318.769165pt}{197.259766pt}}
\pgflineto{\pgfpoint{318.930115pt}{197.074646pt}}
\pgfusepath{stroke}
\pgfpathmoveto{\pgfpoint{318.786865pt}{197.015106pt}}
\pgflineto{\pgfpoint{318.769165pt}{197.259766pt}}
\pgfusepath{stroke}
\pgfpathmoveto{\pgfpoint{318.664886pt}{203.271225pt}}
\pgflineto{\pgfpoint{318.863220pt}{203.089783pt}}
\pgfusepath{stroke}
\pgfpathmoveto{\pgfpoint{318.714691pt}{203.007065pt}}
\pgflineto{\pgfpoint{318.664886pt}{203.271225pt}}
\pgfusepath{stroke}
\pgfpathmoveto{\pgfpoint{318.526031pt}{209.243515pt}}
\pgflineto{\pgfpoint{318.768951pt}{209.082657pt}}
\pgfusepath{stroke}
\pgfpathmoveto{\pgfpoint{318.623840pt}{208.969086pt}}
\pgflineto{\pgfpoint{318.526031pt}{209.243515pt}}
\pgfusepath{stroke}
\pgfpathmoveto{\pgfpoint{318.370483pt}{215.137741pt}}
\pgflineto{\pgfpoint{318.654846pt}{215.025360pt}}
\pgfusepath{stroke}
\pgfpathmoveto{\pgfpoint{318.530548pt}{214.877213pt}}
\pgflineto{\pgfpoint{318.370483pt}{215.137741pt}}
\pgfusepath{stroke}
\pgfpathmoveto{\pgfpoint{318.262360pt}{220.928177pt}}
\pgflineto{\pgfpoint{318.560852pt}{220.893585pt}}
\pgfusepath{stroke}
\pgfpathmoveto{\pgfpoint{318.480408pt}{220.721405pt}}
\pgflineto{\pgfpoint{318.262360pt}{220.928177pt}}
\pgfusepath{stroke}
\pgfpathmoveto{\pgfpoint{318.284332pt}{226.664154pt}}
\pgflineto{\pgfpoint{318.547516pt}{226.711029pt}}
\pgfusepath{stroke}
\pgfpathmoveto{\pgfpoint{318.523010pt}{226.543747pt}}
\pgflineto{\pgfpoint{318.284332pt}{226.664154pt}}
\pgfusepath{stroke}
\pgfpathmoveto{\pgfpoint{318.437164pt}{232.453262pt}}
\pgflineto{\pgfpoint{318.627350pt}{232.549377pt}}
\pgfusepath{stroke}
\pgfpathmoveto{\pgfpoint{318.646973pt}{232.416046pt}}
\pgflineto{\pgfpoint{318.437164pt}{232.453262pt}}
\pgfusepath{stroke}
\pgfpathmoveto{\pgfpoint{318.639801pt}{238.344101pt}}
\pgflineto{\pgfpoint{318.751648pt}{238.450012pt}}
\pgfusepath{stroke}
\pgfpathmoveto{\pgfpoint{318.792816pt}{238.361725pt}}
\pgflineto{\pgfpoint{318.639801pt}{238.344101pt}}
\pgfusepath{stroke}
\pgfpathmoveto{\pgfpoint{318.829376pt}{244.309387pt}}
\pgflineto{\pgfpoint{318.875549pt}{244.401718pt}}
\pgfusepath{stroke}
\pgfpathmoveto{\pgfpoint{318.921722pt}{244.355560pt}}
\pgflineto{\pgfpoint{318.829376pt}{244.309387pt}}
\pgfusepath{stroke}
\pgfpathmoveto{\pgfpoint{318.988892pt}{250.307693pt}}
\pgflineto{\pgfpoint{318.983063pt}{250.378784pt}}
\pgfusepath{stroke}
\pgfpathmoveto{\pgfpoint{319.026886pt}{250.368073pt}}
\pgflineto{\pgfpoint{318.988892pt}{250.307693pt}}
\pgfusepath{stroke}
\pgfpathmoveto{\pgfpoint{319.121979pt}{256.311035pt}}
\pgflineto{\pgfpoint{319.073517pt}{256.362152pt}}
\pgfusepath{stroke}
\pgfpathmoveto{\pgfpoint{319.113892pt}{256.381012pt}}
\pgflineto{\pgfpoint{319.121979pt}{256.311035pt}}
\pgfusepath{stroke}
\pgfpathmoveto{\pgfpoint{319.234558pt}{262.303650pt}}
\pgflineto{\pgfpoint{319.149109pt}{262.340637pt}}
\pgfusepath{stroke}
\pgfpathmoveto{\pgfpoint{319.188385pt}{262.384491pt}}
\pgflineto{\pgfpoint{319.234558pt}{262.303650pt}}
\pgfusepath{stroke}
\pgfpathmoveto{\pgfpoint{319.328247pt}{268.277222pt}}
\pgflineto{\pgfpoint{319.209991pt}{268.308533pt}}
\pgfusepath{stroke}
\pgfpathmoveto{\pgfpoint{319.252411pt}{268.373199pt}}
\pgflineto{\pgfpoint{319.328247pt}{268.277222pt}}
\pgfusepath{stroke}
\pgfpathmoveto{\pgfpoint{319.399780pt}{274.229553pt}}
\pgflineto{\pgfpoint{319.253723pt}{274.264709pt}}
\pgfusepath{stroke}
\pgfpathmoveto{\pgfpoint{319.304047pt}{274.345306pt}}
\pgflineto{\pgfpoint{319.399780pt}{274.229553pt}}
\pgfusepath{stroke}
\pgfpathmoveto{\pgfpoint{319.443604pt}{280.165070pt}}
\pgflineto{\pgfpoint{319.277161pt}{280.212769pt}}
\pgfusepath{stroke}
\pgfpathmoveto{\pgfpoint{319.339081pt}{280.303101pt}}
\pgflineto{\pgfpoint{319.443604pt}{280.165070pt}}
\pgfusepath{stroke}
\pgfpathmoveto{\pgfpoint{319.456696pt}{286.093964pt}}
\pgflineto{\pgfpoint{319.279327pt}{286.159851pt}}
\pgfusepath{stroke}
\pgfpathmoveto{\pgfpoint{319.354340pt}{286.253052pt}}
\pgflineto{\pgfpoint{319.456696pt}{286.093964pt}}
\pgfusepath{stroke}
\pgfpathmoveto{\pgfpoint{319.442017pt}{292.028137pt}}
\pgflineto{\pgfpoint{319.263611pt}{292.113495pt}}
\pgfusepath{stroke}
\pgfpathmoveto{\pgfpoint{319.350494pt}{292.203461pt}}
\pgflineto{\pgfpoint{319.442017pt}{292.028137pt}}
\pgfusepath{stroke}
\pgfpathmoveto{\pgfpoint{319.407959pt}{297.976135pt}}
\pgflineto{\pgfpoint{319.236481pt}{298.078552pt}}
\pgfusepath{stroke}
\pgfpathmoveto{\pgfpoint{319.332214pt}{298.160950pt}}
\pgflineto{\pgfpoint{319.407959pt}{297.976135pt}}
\pgfusepath{stroke}
\pgfpathmoveto{\pgfpoint{319.364441pt}{303.940796pt}}
\pgflineto{\pgfpoint{319.204895pt}{304.055725pt}}
\pgfusepath{stroke}
\pgfpathmoveto{\pgfpoint{319.305756pt}{304.128448pt}}
\pgflineto{\pgfpoint{319.364441pt}{303.940796pt}}
\pgfusepath{stroke}
\pgfpathmoveto{\pgfpoint{319.319336pt}{309.920288pt}}
\pgflineto{\pgfpoint{319.173920pt}{310.042969pt}}
\pgfusepath{stroke}
\pgfpathmoveto{\pgfpoint{319.276611pt}{310.105713pt}}
\pgflineto{\pgfpoint{319.319336pt}{309.920288pt}}
\pgfusepath{stroke}
\pgfpathmoveto{\pgfpoint{319.277374pt}{315.910950pt}}
\pgflineto{\pgfpoint{319.146271pt}{316.037354pt}}
\pgfusepath{stroke}
\pgfpathmoveto{\pgfpoint{319.248322pt}{316.090729pt}}
\pgflineto{\pgfpoint{319.277374pt}{315.910950pt}}
\pgfusepath{stroke}
\pgfpathmoveto{\pgfpoint{319.240601pt}{321.908997pt}}
\pgflineto{\pgfpoint{319.122894pt}{322.036041pt}}
\pgfusepath{stroke}
\pgfpathmoveto{\pgfpoint{319.222656pt}{322.081238pt}}
\pgflineto{\pgfpoint{319.240601pt}{321.908997pt}}
\pgfusepath{stroke}
\pgfpathmoveto{\pgfpoint{319.209442pt}{327.911377pt}}
\pgflineto{\pgfpoint{319.103760pt}{328.037018pt}}
\pgfusepath{stroke}
\pgfpathmoveto{\pgfpoint{319.200256pt}{328.075317pt}}
\pgflineto{\pgfpoint{319.209442pt}{327.911377pt}}
\pgfusepath{stroke}
\pgfpathmoveto{\pgfpoint{319.183502pt}{333.915985pt}}
\pgflineto{\pgfpoint{319.088318pt}{334.038879pt}}
\pgfusepath{stroke}
\pgfpathmoveto{\pgfpoint{319.181091pt}{334.071411pt}}
\pgflineto{\pgfpoint{319.183502pt}{333.915985pt}}
\pgfusepath{stroke}
\pgfpathmoveto{\pgfpoint{319.162048pt}{339.921417pt}}
\pgflineto{\pgfpoint{319.075989pt}{340.040802pt}}
\pgfusepath{stroke}
\pgfpathmoveto{\pgfpoint{319.164825pt}{340.068542pt}}
\pgflineto{\pgfpoint{319.162048pt}{339.921417pt}}
\pgfusepath{stroke}
\pgfpathmoveto{\pgfpoint{319.144348pt}{345.926819pt}}
\pgflineto{\pgfpoint{319.066162pt}{346.042297pt}}
\pgfusepath{stroke}
\pgfpathmoveto{\pgfpoint{319.151062pt}{346.066101pt}}
\pgflineto{\pgfpoint{319.144348pt}{345.926819pt}}
\pgfusepath{stroke}
\pgfpathmoveto{\pgfpoint{319.129730pt}{351.931671pt}}
\pgflineto{\pgfpoint{319.058319pt}{352.043060pt}}
\pgfusepath{stroke}
\pgfpathmoveto{\pgfpoint{319.139435pt}{352.063660pt}}
\pgflineto{\pgfpoint{319.129730pt}{351.931671pt}}
\pgfusepath{stroke}
\pgfpathmoveto{\pgfpoint{319.117615pt}{357.935730pt}}
\pgflineto{\pgfpoint{319.052032pt}{358.043030pt}}
\pgfusepath{stroke}
\pgfpathmoveto{\pgfpoint{319.129547pt}{358.060913pt}}
\pgflineto{\pgfpoint{319.117615pt}{357.935730pt}}
\pgfusepath{stroke}
\pgfpathmoveto{\pgfpoint{319.107513pt}{363.938904pt}}
\pgflineto{\pgfpoint{319.046997pt}{364.042175pt}}
\pgfusepath{stroke}
\pgfpathmoveto{\pgfpoint{319.121094pt}{364.057800pt}}
\pgflineto{\pgfpoint{319.107513pt}{363.938904pt}}
\pgfusepath{stroke}
\pgfpathmoveto{\pgfpoint{319.099030pt}{369.941071pt}}
\pgflineto{\pgfpoint{319.042969pt}{370.040527pt}}
\pgfusepath{stroke}
\pgfpathmoveto{\pgfpoint{319.113831pt}{370.054260pt}}
\pgflineto{\pgfpoint{319.099030pt}{369.941071pt}}
\pgfusepath{stroke}
\pgfpathmoveto{\pgfpoint{325.015564pt}{77.082352pt}}
\pgflineto{\pgfpoint{325.043365pt}{77.009262pt}}
\pgfusepath{stroke}
\pgfpathmoveto{\pgfpoint{324.993958pt}{77.007202pt}}
\pgflineto{\pgfpoint{325.015564pt}{77.082352pt}}
\pgfusepath{stroke}
\pgfpathmoveto{\pgfpoint{325.014404pt}{83.077881pt}}
\pgflineto{\pgfpoint{325.043488pt}{83.002335pt}}
\pgfusepath{stroke}
\pgfpathmoveto{\pgfpoint{324.992340pt}{83.000015pt}}
\pgflineto{\pgfpoint{325.014404pt}{83.077881pt}}
\pgfusepath{stroke}
\pgfpathmoveto{\pgfpoint{325.013123pt}{89.073975pt}}
\pgflineto{\pgfpoint{325.043518pt}{88.995796pt}}
\pgfusepath{stroke}
\pgfpathmoveto{\pgfpoint{324.990540pt}{88.993187pt}}
\pgflineto{\pgfpoint{325.013123pt}{89.073975pt}}
\pgfusepath{stroke}
\pgfpathmoveto{\pgfpoint{325.011566pt}{95.070679pt}}
\pgflineto{\pgfpoint{325.043518pt}{94.989685pt}}
\pgfusepath{stroke}
\pgfpathmoveto{\pgfpoint{324.988525pt}{94.986725pt}}
\pgflineto{\pgfpoint{325.011566pt}{95.070679pt}}
\pgfusepath{stroke}
\pgfpathmoveto{\pgfpoint{325.009796pt}{101.068069pt}}
\pgflineto{\pgfpoint{325.043396pt}{100.984062pt}}
\pgfusepath{stroke}
\pgfpathmoveto{\pgfpoint{324.986267pt}{100.980705pt}}
\pgflineto{\pgfpoint{325.009796pt}{101.068069pt}}
\pgfusepath{stroke}
\pgfpathmoveto{\pgfpoint{325.007690pt}{107.066216pt}}
\pgflineto{\pgfpoint{325.043152pt}{106.978981pt}}
\pgfusepath{stroke}
\pgfpathmoveto{\pgfpoint{324.983704pt}{106.975159pt}}
\pgflineto{\pgfpoint{325.007690pt}{107.066216pt}}
\pgfusepath{stroke}
\pgfpathmoveto{\pgfpoint{325.005188pt}{113.065216pt}}
\pgflineto{\pgfpoint{325.042725pt}{112.974503pt}}
\pgfusepath{stroke}
\pgfpathmoveto{\pgfpoint{324.980774pt}{112.970123pt}}
\pgflineto{\pgfpoint{325.005188pt}{113.065216pt}}
\pgfusepath{stroke}
\pgfpathmoveto{\pgfpoint{325.002228pt}{119.065155pt}}
\pgflineto{\pgfpoint{325.042084pt}{118.970711pt}}
\pgfusepath{stroke}
\pgfpathmoveto{\pgfpoint{324.977448pt}{118.965683pt}}
\pgflineto{\pgfpoint{325.002228pt}{119.065155pt}}
\pgfusepath{stroke}
\pgfpathmoveto{\pgfpoint{324.998657pt}{125.066170pt}}
\pgflineto{\pgfpoint{325.041199pt}{124.967705pt}}
\pgfusepath{stroke}
\pgfpathmoveto{\pgfpoint{324.973602pt}{124.961868pt}}
\pgflineto{\pgfpoint{324.998657pt}{125.066170pt}}
\pgfusepath{stroke}
\pgfpathmoveto{\pgfpoint{324.994324pt}{131.068375pt}}
\pgflineto{\pgfpoint{325.039917pt}{130.965576pt}}
\pgfusepath{stroke}
\pgfpathmoveto{\pgfpoint{324.969116pt}{130.958771pt}}
\pgflineto{\pgfpoint{324.994324pt}{131.068375pt}}
\pgfusepath{stroke}
\pgfpathmoveto{\pgfpoint{324.989044pt}{137.071899pt}}
\pgflineto{\pgfpoint{325.038147pt}{136.964447pt}}
\pgfusepath{stroke}
\pgfpathmoveto{\pgfpoint{324.963837pt}{136.956467pt}}
\pgflineto{\pgfpoint{324.989044pt}{137.071899pt}}
\pgfusepath{stroke}
\pgfpathmoveto{\pgfpoint{324.982544pt}{143.076950pt}}
\pgflineto{\pgfpoint{325.035736pt}{142.964432pt}}
\pgfusepath{stroke}
\pgfpathmoveto{\pgfpoint{324.957581pt}{142.955032pt}}
\pgflineto{\pgfpoint{324.982544pt}{143.076950pt}}
\pgfusepath{stroke}
\pgfpathmoveto{\pgfpoint{324.974487pt}{149.083664pt}}
\pgflineto{\pgfpoint{325.032471pt}{148.965729pt}}
\pgfusepath{stroke}
\pgfpathmoveto{\pgfpoint{324.950104pt}{148.954529pt}}
\pgflineto{\pgfpoint{324.974487pt}{149.083664pt}}
\pgfusepath{stroke}
\pgfpathmoveto{\pgfpoint{324.964325pt}{155.092224pt}}
\pgflineto{\pgfpoint{325.028015pt}{154.968475pt}}
\pgfusepath{stroke}
\pgfpathmoveto{\pgfpoint{324.941040pt}{154.955017pt}}
\pgflineto{\pgfpoint{324.964325pt}{155.092224pt}}
\pgfusepath{stroke}
\pgfpathmoveto{\pgfpoint{324.951447pt}{161.102722pt}}
\pgflineto{\pgfpoint{325.021942pt}{160.972839pt}}
\pgfusepath{stroke}
\pgfpathmoveto{\pgfpoint{324.929932pt}{160.956512pt}}
\pgflineto{\pgfpoint{324.951447pt}{161.102722pt}}
\pgfusepath{stroke}
\pgfpathmoveto{\pgfpoint{324.934937pt}{167.115265pt}}
\pgflineto{\pgfpoint{325.013672pt}{166.978943pt}}
\pgfusepath{stroke}
\pgfpathmoveto{\pgfpoint{324.916138pt}{166.958954pt}}
\pgflineto{\pgfpoint{324.934937pt}{167.115265pt}}
\pgfusepath{stroke}
\pgfpathmoveto{\pgfpoint{324.913483pt}{173.129715pt}}
\pgflineto{\pgfpoint{325.002350pt}{172.986877pt}}
\pgfusepath{stroke}
\pgfpathmoveto{\pgfpoint{324.898865pt}{172.962128pt}}
\pgflineto{\pgfpoint{324.913483pt}{173.129715pt}}
\pgfusepath{stroke}
\pgfpathmoveto{\pgfpoint{324.885376pt}{179.145523pt}}
\pgflineto{\pgfpoint{324.986755pt}{178.996460pt}}
\pgfusepath{stroke}
\pgfpathmoveto{\pgfpoint{324.877014pt}{178.965454pt}}
\pgflineto{\pgfpoint{324.885376pt}{179.145523pt}}
\pgfusepath{stroke}
\pgfpathmoveto{\pgfpoint{324.848328pt}{185.161423pt}}
\pgflineto{\pgfpoint{324.965149pt}{185.007095pt}}
\pgfusepath{stroke}
\pgfpathmoveto{\pgfpoint{324.849213pt}{184.967865pt}}
\pgflineto{\pgfpoint{324.848328pt}{185.161423pt}}
\pgfusepath{stroke}
\pgfpathmoveto{\pgfpoint{324.799469pt}{191.174652pt}}
\pgflineto{\pgfpoint{324.935425pt}{191.017258pt}}
\pgfusepath{stroke}
\pgfpathmoveto{\pgfpoint{324.813782pt}{190.967163pt}}
\pgflineto{\pgfpoint{324.799469pt}{191.174652pt}}
\pgfusepath{stroke}
\pgfpathmoveto{\pgfpoint{324.735901pt}{197.179840pt}}
\pgflineto{\pgfpoint{324.894958pt}{197.023697pt}}
\pgfusepath{stroke}
\pgfpathmoveto{\pgfpoint{324.769470pt}{196.959473pt}}
\pgflineto{\pgfpoint{324.735901pt}{197.179840pt}}
\pgfusepath{stroke}
\pgfpathmoveto{\pgfpoint{324.656433pt}{203.167526pt}}
\pgflineto{\pgfpoint{324.842010pt}{203.020233pt}}
\pgfusepath{stroke}
\pgfpathmoveto{\pgfpoint{324.716522pt}{202.938354pt}}
\pgflineto{\pgfpoint{324.656433pt}{203.167526pt}}
\pgfusepath{stroke}
\pgfpathmoveto{\pgfpoint{324.565918pt}{209.123657pt}}
\pgflineto{\pgfpoint{324.778137pt}{208.996964pt}}
\pgfusepath{stroke}
\pgfpathmoveto{\pgfpoint{324.659698pt}{208.894974pt}}
\pgflineto{\pgfpoint{324.565918pt}{209.123657pt}}
\pgfusepath{stroke}
\pgfpathmoveto{\pgfpoint{324.481812pt}{215.033844pt}}
\pgflineto{\pgfpoint{324.713440pt}{214.942352pt}}
\pgfusepath{stroke}
\pgfpathmoveto{\pgfpoint{324.612213pt}{214.821671pt}}
\pgflineto{\pgfpoint{324.481812pt}{215.033844pt}}
\pgfusepath{stroke}
\pgfpathmoveto{\pgfpoint{324.435883pt}{220.897095pt}}
\pgflineto{\pgfpoint{324.669006pt}{220.852219pt}}
\pgfusepath{stroke}
\pgfpathmoveto{\pgfpoint{324.595428pt}{220.721313pt}}
\pgflineto{\pgfpoint{324.435883pt}{220.897095pt}}
\pgfusepath{stroke}
\pgfpathmoveto{\pgfpoint{324.457977pt}{226.739639pt}}
\pgflineto{\pgfpoint{324.667603pt}{226.740707pt}}
\pgfusepath{stroke}
\pgfpathmoveto{\pgfpoint{324.626312pt}{226.614731pt}}
\pgflineto{\pgfpoint{324.457977pt}{226.739639pt}}
\pgfusepath{stroke}
\pgfpathmoveto{\pgfpoint{324.549316pt}{232.603531pt}}
\pgflineto{\pgfpoint{324.714691pt}{232.635742pt}}
\pgfusepath{stroke}
\pgfpathmoveto{\pgfpoint{324.700958pt}{232.530045pt}}
\pgflineto{\pgfpoint{324.549316pt}{232.603531pt}}
\pgfusepath{stroke}
\pgfpathmoveto{\pgfpoint{324.682343pt}{238.514206pt}}
\pgflineto{\pgfpoint{324.794861pt}{238.557327pt}}
\pgfusepath{stroke}
\pgfpathmoveto{\pgfpoint{324.798218pt}{238.481201pt}}
\pgflineto{\pgfpoint{324.682343pt}{238.514206pt}}
\pgfusepath{stroke}
\pgfpathmoveto{\pgfpoint{324.827698pt}{244.468887pt}}
\pgflineto{\pgfpoint{324.888062pt}{244.506882pt}}
\pgfusepath{stroke}
\pgfpathmoveto{\pgfpoint{324.898773pt}{244.463058pt}}
\pgflineto{\pgfpoint{324.827698pt}{244.468887pt}}
\pgfusepath{stroke}
\pgfpathmoveto{\pgfpoint{324.970001pt}{250.450012pt}}
\pgflineto{\pgfpoint{324.982208pt}{250.474396pt}}
\pgfusepath{stroke}
\pgfpathmoveto{\pgfpoint{324.994385pt}{250.462219pt}}
\pgflineto{\pgfpoint{324.970001pt}{250.450012pt}}
\pgfusepath{stroke}
\pgfpathmoveto{\pgfpoint{325.105621pt}{256.438354pt}}
\pgflineto{\pgfpoint{325.072693pt}{256.447479pt}}
\pgfusepath{stroke}
\pgfpathmoveto{\pgfpoint{325.084747pt}{256.465424pt}}
\pgflineto{\pgfpoint{325.105621pt}{256.438354pt}}
\pgfusepath{stroke}
\pgfpathmoveto{\pgfpoint{325.234192pt}{262.417114pt}}
\pgflineto{\pgfpoint{325.157410pt}{262.414948pt}}
\pgfusepath{stroke}
\pgfpathmoveto{\pgfpoint{325.171448pt}{262.461456pt}}
\pgflineto{\pgfpoint{325.234192pt}{262.417114pt}}
\pgfusepath{stroke}
\pgfpathmoveto{\pgfpoint{325.351929pt}{268.372833pt}}
\pgflineto{\pgfpoint{325.232330pt}{268.368256pt}}
\pgfusepath{stroke}
\pgfpathmoveto{\pgfpoint{325.253510pt}{268.440948pt}}
\pgflineto{\pgfpoint{325.351929pt}{268.372833pt}}
\pgfusepath{stroke}
\pgfpathmoveto{\pgfpoint{325.448029pt}{274.298065pt}}
\pgflineto{\pgfpoint{325.289459pt}{274.303680pt}}
\pgfusepath{stroke}
\pgfpathmoveto{\pgfpoint{325.324524pt}{274.397705pt}}
\pgflineto{\pgfpoint{325.448029pt}{274.298065pt}}
\pgfusepath{stroke}
\pgfpathmoveto{\pgfpoint{325.507019pt}{280.197205pt}}
\pgflineto{\pgfpoint{325.318939pt}{280.225769pt}}
\pgfusepath{stroke}
\pgfpathmoveto{\pgfpoint{325.373718pt}{280.332916pt}}
\pgflineto{\pgfpoint{325.507019pt}{280.197205pt}}
\pgfusepath{stroke}
\pgfpathmoveto{\pgfpoint{325.518860pt}{286.088013pt}}
\pgflineto{\pgfpoint{325.316040pt}{286.147583pt}}
\pgfusepath{stroke}
\pgfpathmoveto{\pgfpoint{325.392334pt}{286.257385pt}}
\pgflineto{\pgfpoint{325.518860pt}{286.088013pt}}
\pgfusepath{stroke}
\pgfpathmoveto{\pgfpoint{325.488098pt}{291.992889pt}}
\pgflineto{\pgfpoint{325.286346pt}{292.083496pt}}
\pgfusepath{stroke}
\pgfpathmoveto{\pgfpoint{325.381073pt}{292.186462pt}}
\pgflineto{\pgfpoint{325.488098pt}{291.992889pt}}
\pgfusepath{stroke}
\pgfpathmoveto{\pgfpoint{325.431641pt}{297.925507pt}}
\pgflineto{\pgfpoint{325.242554pt}{298.040771pt}}
\pgfusepath{stroke}
\pgfpathmoveto{\pgfpoint{325.349548pt}{298.131165pt}}
\pgflineto{\pgfpoint{325.431641pt}{297.925507pt}}
\pgfusepath{stroke}
\pgfpathmoveto{\pgfpoint{325.367249pt}{303.886780pt}}
\pgflineto{\pgfpoint{325.196686pt}{304.018036pt}}
\pgfusepath{stroke}
\pgfpathmoveto{\pgfpoint{325.309570pt}{304.094116pt}}
\pgflineto{\pgfpoint{325.367249pt}{303.886780pt}}
\pgfusepath{stroke}
\pgfpathmoveto{\pgfpoint{325.306610pt}{309.870148pt}}
\pgflineto{\pgfpoint{325.155762pt}{310.009613pt}}
\pgfusepath{stroke}
\pgfpathmoveto{\pgfpoint{325.269623pt}{310.072205pt}}
\pgflineto{\pgfpoint{325.306610pt}{309.870148pt}}
\pgfusepath{stroke}
\pgfpathmoveto{\pgfpoint{325.254639pt}{315.867645pt}}
\pgflineto{\pgfpoint{325.122223pt}{316.009644pt}}
\pgfusepath{stroke}
\pgfpathmoveto{\pgfpoint{325.233887pt}{316.060699pt}}
\pgflineto{\pgfpoint{325.254639pt}{315.867645pt}}
\pgfusepath{stroke}
\pgfpathmoveto{\pgfpoint{325.212158pt}{321.872986pt}}
\pgflineto{\pgfpoint{325.095856pt}{322.013824pt}}
\pgfusepath{stroke}
\pgfpathmoveto{\pgfpoint{325.203613pt}{322.055450pt}}
\pgflineto{\pgfpoint{325.212158pt}{321.872986pt}}
\pgfusepath{stroke}
\pgfpathmoveto{\pgfpoint{325.178253pt}{327.882019pt}}
\pgflineto{\pgfpoint{325.075623pt}{328.019531pt}}
\pgfusepath{stroke}
\pgfpathmoveto{\pgfpoint{325.178650pt}{328.053619pt}}
\pgflineto{\pgfpoint{325.178253pt}{327.882019pt}}
\pgfusepath{stroke}
\pgfpathmoveto{\pgfpoint{325.151367pt}{333.892181pt}}
\pgflineto{\pgfpoint{325.060181pt}{334.025208pt}}
\pgfusepath{stroke}
\pgfpathmoveto{\pgfpoint{325.158234pt}{334.053345pt}}
\pgflineto{\pgfpoint{325.151367pt}{333.892181pt}}
\pgfusepath{stroke}
\pgfpathmoveto{\pgfpoint{325.130066pt}{339.902161pt}}
\pgflineto{\pgfpoint{325.048431pt}{340.030151pt}}
\pgfusepath{stroke}
\pgfpathmoveto{\pgfpoint{325.141571pt}{340.053528pt}}
\pgflineto{\pgfpoint{325.130066pt}{339.902161pt}}
\pgfusepath{stroke}
\pgfpathmoveto{\pgfpoint{325.113098pt}{345.911133pt}}
\pgflineto{\pgfpoint{325.039490pt}{346.033966pt}}
\pgfusepath{stroke}
\pgfpathmoveto{\pgfpoint{325.127899pt}{346.053558pt}}
\pgflineto{\pgfpoint{325.113098pt}{345.911133pt}}
\pgfusepath{stroke}
\pgfpathmoveto{\pgfpoint{325.099487pt}{351.918884pt}}
\pgflineto{\pgfpoint{325.032654pt}{352.036560pt}}
\pgfusepath{stroke}
\pgfpathmoveto{\pgfpoint{325.116608pt}{352.053131pt}}
\pgflineto{\pgfpoint{325.099487pt}{351.918884pt}}
\pgfusepath{stroke}
\pgfpathmoveto{\pgfpoint{325.088501pt}{357.925232pt}}
\pgflineto{\pgfpoint{325.027374pt}{358.037903pt}}
\pgfusepath{stroke}
\pgfpathmoveto{\pgfpoint{325.107208pt}{358.052032pt}}
\pgflineto{\pgfpoint{325.088501pt}{357.925232pt}}
\pgfusepath{stroke}
\pgfpathmoveto{\pgfpoint{325.079529pt}{363.930176pt}}
\pgflineto{\pgfpoint{325.023315pt}{364.038116pt}}
\pgfusepath{stroke}
\pgfpathmoveto{\pgfpoint{325.099335pt}{364.050293pt}}
\pgflineto{\pgfpoint{325.079529pt}{363.930176pt}}
\pgfusepath{stroke}
\pgfpathmoveto{\pgfpoint{325.072174pt}{369.933838pt}}
\pgflineto{\pgfpoint{325.020172pt}{370.037292pt}}
\pgfusepath{stroke}
\pgfpathmoveto{\pgfpoint{325.092651pt}{370.047791pt}}
\pgflineto{\pgfpoint{325.072174pt}{369.933838pt}}
\pgfusepath{stroke}
\pgfpathmoveto{\pgfpoint{330.995880pt}{77.081055pt}}
\pgflineto{\pgfpoint{331.026031pt}{77.009216pt}}
\pgfusepath{stroke}
\pgfpathmoveto{\pgfpoint{330.976898pt}{77.005493pt}}
\pgflineto{\pgfpoint{330.995880pt}{77.081055pt}}
\pgfusepath{stroke}
\pgfpathmoveto{\pgfpoint{330.994232pt}{83.076401pt}}
\pgflineto{\pgfpoint{331.025757pt}{83.002213pt}}
\pgfusepath{stroke}
\pgfpathmoveto{\pgfpoint{330.974976pt}{82.998138pt}}
\pgflineto{\pgfpoint{330.994232pt}{83.076401pt}}
\pgfusepath{stroke}
\pgfpathmoveto{\pgfpoint{330.992340pt}{89.072258pt}}
\pgflineto{\pgfpoint{331.025421pt}{88.995605pt}}
\pgfusepath{stroke}
\pgfpathmoveto{\pgfpoint{330.972809pt}{88.991089pt}}
\pgflineto{\pgfpoint{330.992340pt}{89.072258pt}}
\pgfusepath{stroke}
\pgfpathmoveto{\pgfpoint{330.990204pt}{95.068695pt}}
\pgflineto{\pgfpoint{331.024963pt}{94.989380pt}}
\pgfusepath{stroke}
\pgfpathmoveto{\pgfpoint{330.970398pt}{94.984383pt}}
\pgflineto{\pgfpoint{330.990204pt}{95.068695pt}}
\pgfusepath{stroke}
\pgfpathmoveto{\pgfpoint{330.987732pt}{101.065765pt}}
\pgflineto{\pgfpoint{331.024323pt}{100.983612pt}}
\pgfusepath{stroke}
\pgfpathmoveto{\pgfpoint{330.967712pt}{100.978073pt}}
\pgflineto{\pgfpoint{330.987732pt}{101.065765pt}}
\pgfusepath{stroke}
\pgfpathmoveto{\pgfpoint{330.984833pt}{107.063515pt}}
\pgflineto{\pgfpoint{331.023529pt}{106.978348pt}}
\pgfusepath{stroke}
\pgfpathmoveto{\pgfpoint{330.964691pt}{106.972168pt}}
\pgflineto{\pgfpoint{330.984833pt}{107.063515pt}}
\pgfusepath{stroke}
\pgfpathmoveto{\pgfpoint{330.981506pt}{113.062035pt}}
\pgflineto{\pgfpoint{331.022491pt}{112.973663pt}}
\pgfusepath{stroke}
\pgfpathmoveto{\pgfpoint{330.961273pt}{112.966736pt}}
\pgflineto{\pgfpoint{330.981506pt}{113.062035pt}}
\pgfusepath{stroke}
\pgfpathmoveto{\pgfpoint{330.977600pt}{119.061394pt}}
\pgflineto{\pgfpoint{331.021179pt}{118.969582pt}}
\pgfusepath{stroke}
\pgfpathmoveto{\pgfpoint{330.957367pt}{118.961792pt}}
\pgflineto{\pgfpoint{330.977600pt}{119.061394pt}}
\pgfusepath{stroke}
\pgfpathmoveto{\pgfpoint{330.972961pt}{125.061684pt}}
\pgflineto{\pgfpoint{331.019470pt}{124.966202pt}}
\pgfusepath{stroke}
\pgfpathmoveto{\pgfpoint{330.952881pt}{124.957390pt}}
\pgflineto{\pgfpoint{330.972961pt}{125.061684pt}}
\pgfusepath{stroke}
\pgfpathmoveto{\pgfpoint{330.967468pt}{131.062988pt}}
\pgflineto{\pgfpoint{331.017334pt}{130.963593pt}}
\pgfusepath{stroke}
\pgfpathmoveto{\pgfpoint{330.947723pt}{130.953552pt}}
\pgflineto{\pgfpoint{330.967468pt}{131.062988pt}}
\pgfusepath{stroke}
\pgfpathmoveto{\pgfpoint{330.960938pt}{137.065369pt}}
\pgflineto{\pgfpoint{331.014587pt}{136.961853pt}}
\pgfusepath{stroke}
\pgfpathmoveto{\pgfpoint{330.941742pt}{136.950348pt}}
\pgflineto{\pgfpoint{330.960938pt}{137.065369pt}}
\pgfusepath{stroke}
\pgfpathmoveto{\pgfpoint{330.953003pt}{143.068970pt}}
\pgflineto{\pgfpoint{331.011078pt}{142.961029pt}}
\pgfusepath{stroke}
\pgfpathmoveto{\pgfpoint{330.934692pt}{142.947784pt}}
\pgflineto{\pgfpoint{330.953003pt}{143.068970pt}}
\pgfusepath{stroke}
\pgfpathmoveto{\pgfpoint{330.943420pt}{149.073822pt}}
\pgflineto{\pgfpoint{331.006592pt}{148.961258pt}}
\pgfusepath{stroke}
\pgfpathmoveto{\pgfpoint{330.926422pt}{148.945877pt}}
\pgflineto{\pgfpoint{330.943420pt}{149.073822pt}}
\pgfusepath{stroke}
\pgfpathmoveto{\pgfpoint{330.931702pt}{155.079956pt}}
\pgflineto{\pgfpoint{331.000793pt}{154.962585pt}}
\pgfusepath{stroke}
\pgfpathmoveto{\pgfpoint{330.916534pt}{154.944580pt}}
\pgflineto{\pgfpoint{330.931702pt}{155.079956pt}}
\pgfusepath{stroke}
\pgfpathmoveto{\pgfpoint{330.917236pt}{161.087357pt}}
\pgflineto{\pgfpoint{330.993347pt}{160.965027pt}}
\pgfusepath{stroke}
\pgfpathmoveto{\pgfpoint{330.904724pt}{160.943817pt}}
\pgflineto{\pgfpoint{330.917236pt}{161.087357pt}}
\pgfusepath{stroke}
\pgfpathmoveto{\pgfpoint{330.899292pt}{167.095795pt}}
\pgflineto{\pgfpoint{330.983673pt}{166.968567pt}}
\pgfusepath{stroke}
\pgfpathmoveto{\pgfpoint{330.890472pt}{166.943375pt}}
\pgflineto{\pgfpoint{330.899292pt}{167.095795pt}}
\pgfusepath{stroke}
\pgfpathmoveto{\pgfpoint{330.876892pt}{173.104858pt}}
\pgflineto{\pgfpoint{330.971130pt}{172.973022pt}}
\pgfusepath{stroke}
\pgfpathmoveto{\pgfpoint{330.873169pt}{172.942841pt}}
\pgflineto{\pgfpoint{330.876892pt}{173.104858pt}}
\pgfusepath{stroke}
\pgfpathmoveto{\pgfpoint{330.848877pt}{179.113663pt}}
\pgflineto{\pgfpoint{330.954773pt}{178.977905pt}}
\pgfusepath{stroke}
\pgfpathmoveto{\pgfpoint{330.852142pt}{178.941528pt}}
\pgflineto{\pgfpoint{330.848877pt}{179.113663pt}}
\pgfusepath{stroke}
\pgfpathmoveto{\pgfpoint{330.813934pt}{185.120621pt}}
\pgflineto{\pgfpoint{330.933624pt}{184.982346pt}}
\pgfusepath{stroke}
\pgfpathmoveto{\pgfpoint{330.826721pt}{184.938187pt}}
\pgflineto{\pgfpoint{330.813934pt}{185.120621pt}}
\pgfusepath{stroke}
\pgfpathmoveto{\pgfpoint{330.770874pt}{191.123001pt}}
\pgflineto{\pgfpoint{330.906555pt}{190.984650pt}}
\pgfusepath{stroke}
\pgfpathmoveto{\pgfpoint{330.796387pt}{190.930923pt}}
\pgflineto{\pgfpoint{330.770874pt}{191.123001pt}}
\pgfusepath{stroke}
\pgfpathmoveto{\pgfpoint{330.719391pt}{197.116501pt}}
\pgflineto{\pgfpoint{330.872833pt}{196.981949pt}}
\pgfusepath{stroke}
\pgfpathmoveto{\pgfpoint{330.761414pt}{196.916779pt}}
\pgflineto{\pgfpoint{330.719391pt}{197.116501pt}}
\pgfusepath{stroke}
\pgfpathmoveto{\pgfpoint{330.661224pt}{203.095047pt}}
\pgflineto{\pgfpoint{330.833038pt}{202.970032pt}}
\pgfusepath{stroke}
\pgfpathmoveto{\pgfpoint{330.723663pt}{202.891937pt}}
\pgflineto{\pgfpoint{330.661224pt}{203.095047pt}}
\pgfusepath{stroke}
\pgfpathmoveto{\pgfpoint{330.602478pt}{209.051773pt}}
\pgflineto{\pgfpoint{330.790436pt}{208.943619pt}}
\pgfusepath{stroke}
\pgfpathmoveto{\pgfpoint{330.687958pt}{208.852463pt}}
\pgflineto{\pgfpoint{330.602478pt}{209.051773pt}}
\pgfusepath{stroke}
\pgfpathmoveto{\pgfpoint{330.555176pt}{214.982315pt}}
\pgflineto{\pgfpoint{330.752533pt}{214.898499pt}}
\pgfusepath{stroke}
\pgfpathmoveto{\pgfpoint{330.662781pt}{214.796829pt}}
\pgflineto{\pgfpoint{330.555176pt}{214.982315pt}}
\pgfusepath{stroke}
\pgfpathmoveto{\pgfpoint{330.535553pt}{220.890274pt}}
\pgflineto{\pgfpoint{330.730591pt}{220.835236pt}}
\pgfusepath{stroke}
\pgfpathmoveto{\pgfpoint{330.658569pt}{220.729218pt}}
\pgflineto{\pgfpoint{330.535553pt}{220.890274pt}}
\pgfusepath{stroke}
\pgfpathmoveto{\pgfpoint{330.556641pt}{226.790100pt}}
\pgflineto{\pgfpoint{330.734863pt}{226.762024pt}}
\pgfusepath{stroke}
\pgfpathmoveto{\pgfpoint{330.682373pt}{226.660690pt}}
\pgflineto{\pgfpoint{330.556641pt}{226.790100pt}}
\pgfusepath{stroke}
\pgfpathmoveto{\pgfpoint{330.619934pt}{232.701416pt}}
\pgflineto{\pgfpoint{330.768585pt}{232.691788pt}}
\pgfusepath{stroke}
\pgfpathmoveto{\pgfpoint{330.733093pt}{232.604523pt}}
\pgflineto{\pgfpoint{330.619934pt}{232.701416pt}}
\pgfusepath{stroke}
\pgfpathmoveto{\pgfpoint{330.715607pt}{238.638031pt}}
\pgflineto{\pgfpoint{330.826691pt}{238.634827pt}}
\pgfusepath{stroke}
\pgfpathmoveto{\pgfpoint{330.802551pt}{238.568817pt}}
\pgflineto{\pgfpoint{330.715607pt}{238.638031pt}}
\pgfusepath{stroke}
\pgfpathmoveto{\pgfpoint{330.831024pt}{244.601990pt}}
\pgflineto{\pgfpoint{330.901001pt}{244.593887pt}}
\pgfusepath{stroke}
\pgfpathmoveto{\pgfpoint{330.882141pt}{244.553528pt}}
\pgflineto{\pgfpoint{330.831024pt}{244.601990pt}}
\pgfusepath{stroke}
\pgfpathmoveto{\pgfpoint{330.958344pt}{250.585632pt}}
\pgflineto{\pgfpoint{330.985413pt}{250.564758pt}}
\pgfusepath{stroke}
\pgfpathmoveto{\pgfpoint{330.967468pt}{250.552673pt}}
\pgflineto{\pgfpoint{330.958344pt}{250.585632pt}}
\pgfusepath{stroke}
\pgfpathmoveto{\pgfpoint{331.095642pt}{256.575653pt}}
\pgflineto{\pgfpoint{331.077209pt}{256.538757pt}}
\pgfusepath{stroke}
\pgfpathmoveto{\pgfpoint{331.058746pt}{256.557220pt}}
\pgflineto{\pgfpoint{331.095642pt}{256.575653pt}}
\pgfusepath{stroke}
\pgfpathmoveto{\pgfpoint{331.243317pt}{262.554565pt}}
\pgflineto{\pgfpoint{331.174652pt}{262.504211pt}}
\pgfusepath{stroke}
\pgfpathmoveto{\pgfpoint{331.158173pt}{262.555481pt}}
\pgflineto{\pgfpoint{331.243317pt}{262.554565pt}}
\pgfusepath{stroke}
\pgfpathmoveto{\pgfpoint{331.396271pt}{268.500916pt}}
\pgflineto{\pgfpoint{331.272034pt}{268.447388pt}}
\pgfusepath{stroke}
\pgfpathmoveto{\pgfpoint{331.264740pt}{268.532623pt}}
\pgflineto{\pgfpoint{331.396271pt}{268.500916pt}}
\pgfusepath{stroke}
\pgfpathmoveto{\pgfpoint{331.534302pt}{274.395355pt}}
\pgflineto{\pgfpoint{331.353668pt}{274.357574pt}}
\pgfusepath{stroke}
\pgfpathmoveto{\pgfpoint{331.367126pt}{274.473511pt}}
\pgflineto{\pgfpoint{331.534302pt}{274.395355pt}}
\pgfusepath{stroke}
\pgfpathmoveto{\pgfpoint{331.620544pt}{280.238525pt}}
\pgflineto{\pgfpoint{331.395142pt}{280.239380pt}}
\pgfusepath{stroke}
\pgfpathmoveto{\pgfpoint{331.440704pt}{280.374451pt}}
\pgflineto{\pgfpoint{331.620544pt}{280.238525pt}}
\pgfusepath{stroke}
\pgfpathmoveto{\pgfpoint{331.625854pt}{286.066223pt}}
\pgflineto{\pgfpoint{331.380859pt}{286.119812pt}}
\pgfusepath{stroke}
\pgfpathmoveto{\pgfpoint{331.462036pt}{286.256104pt}}
\pgflineto{\pgfpoint{331.625854pt}{286.066223pt}}
\pgfusepath{stroke}
\pgfpathmoveto{\pgfpoint{331.559998pt}{291.927429pt}}
\pgflineto{\pgfpoint{331.322906pt}{292.030548pt}}
\pgfusepath{stroke}
\pgfpathmoveto{\pgfpoint{331.432190pt}{292.152161pt}}
\pgflineto{\pgfpoint{331.559998pt}{291.927429pt}}
\pgfusepath{stroke}
\pgfpathmoveto{\pgfpoint{331.461914pt}{297.844940pt}}
\pgflineto{\pgfpoint{331.249725pt}{297.982361pt}}
\pgfusepath{stroke}
\pgfpathmoveto{\pgfpoint{331.374603pt}{298.082184pt}}
\pgflineto{\pgfpoint{331.461914pt}{297.844940pt}}
\pgfusepath{stroke}
\pgfpathmoveto{\pgfpoint{331.365173pt}{303.810608pt}}
\pgflineto{\pgfpoint{331.182770pt}{303.966125pt}}
\pgfusepath{stroke}
\pgfpathmoveto{\pgfpoint{331.312561pt}{304.044464pt}}
\pgflineto{\pgfpoint{331.365173pt}{303.810608pt}}
\pgfusepath{stroke}
\pgfpathmoveto{\pgfpoint{331.284546pt}{309.806213pt}}
\pgflineto{\pgfpoint{331.129883pt}{309.968079pt}}
\pgfusepath{stroke}
\pgfpathmoveto{\pgfpoint{331.257935pt}{310.028503pt}}
\pgflineto{\pgfpoint{331.284546pt}{309.806213pt}}
\pgfusepath{stroke}
\pgfpathmoveto{\pgfpoint{331.222107pt}{315.816864pt}}
\pgflineto{\pgfpoint{331.090790pt}{315.978027pt}}
\pgfusepath{stroke}
\pgfpathmoveto{\pgfpoint{331.213776pt}{316.024597pt}}
\pgflineto{\pgfpoint{331.222107pt}{315.816864pt}}
\pgfusepath{stroke}
\pgfpathmoveto{\pgfpoint{331.175140pt}{321.833588pt}}
\pgflineto{\pgfpoint{331.062714pt}{321.990326pt}}
\pgfusepath{stroke}
\pgfpathmoveto{\pgfpoint{331.179230pt}{322.026428pt}}
\pgflineto{\pgfpoint{331.175140pt}{321.833588pt}}
\pgfusepath{stroke}
\pgfpathmoveto{\pgfpoint{331.140045pt}{327.851624pt}}
\pgflineto{\pgfpoint{331.042694pt}{328.002167pt}}
\pgfusepath{stroke}
\pgfpathmoveto{\pgfpoint{331.152496pt}{328.030457pt}}
\pgflineto{\pgfpoint{331.140045pt}{327.851624pt}}
\pgfusepath{stroke}
\pgfpathmoveto{\pgfpoint{331.113708pt}{333.868713pt}}
\pgflineto{\pgfpoint{331.028381pt}{334.012390pt}}
\pgfusepath{stroke}
\pgfpathmoveto{\pgfpoint{331.131683pt}{334.034851pt}}
\pgflineto{\pgfpoint{331.113708pt}{333.868713pt}}
\pgfusepath{stroke}
\pgfpathmoveto{\pgfpoint{331.093781pt}{339.883850pt}}
\pgflineto{\pgfpoint{331.018127pt}{340.020630pt}}
\pgfusepath{stroke}
\pgfpathmoveto{\pgfpoint{331.115326pt}{340.038666pt}}
\pgflineto{\pgfpoint{331.093781pt}{339.883850pt}}
\pgfusepath{stroke}
\pgfpathmoveto{\pgfpoint{331.078491pt}{345.896790pt}}
\pgflineto{\pgfpoint{331.010742pt}{346.026855pt}}
\pgfusepath{stroke}
\pgfpathmoveto{\pgfpoint{331.102356pt}{346.041504pt}}
\pgflineto{\pgfpoint{331.078491pt}{345.896790pt}}
\pgfusepath{stroke}
\pgfpathmoveto{\pgfpoint{331.066650pt}{351.907532pt}}
\pgflineto{\pgfpoint{331.005402pt}{352.031250pt}}
\pgfusepath{stroke}
\pgfpathmoveto{\pgfpoint{331.091888pt}{352.043243pt}}
\pgflineto{\pgfpoint{331.066650pt}{351.907532pt}}
\pgfusepath{stroke}
\pgfpathmoveto{\pgfpoint{331.057373pt}{357.916138pt}}
\pgflineto{\pgfpoint{331.001526pt}{358.033936pt}}
\pgfusepath{stroke}
\pgfpathmoveto{\pgfpoint{331.083374pt}{358.043884pt}}
\pgflineto{\pgfpoint{331.057373pt}{357.916138pt}}
\pgfusepath{stroke}
\pgfpathmoveto{\pgfpoint{331.049988pt}{363.922821pt}}
\pgflineto{\pgfpoint{330.998718pt}{364.035156pt}}
\pgfusepath{stroke}
\pgfpathmoveto{\pgfpoint{331.076355pt}{364.043457pt}}
\pgflineto{\pgfpoint{331.049988pt}{363.922821pt}}
\pgfusepath{stroke}
\pgfpathmoveto{\pgfpoint{331.044067pt}{369.927795pt}}
\pgflineto{\pgfpoint{330.996674pt}{370.035034pt}}
\pgfusepath{stroke}
\pgfpathmoveto{\pgfpoint{331.070496pt}{370.042023pt}}
\pgflineto{\pgfpoint{331.044067pt}{369.927795pt}}
\pgfusepath{stroke}
\pgfpathmoveto{\pgfpoint{336.976379pt}{77.079269pt}}
\pgflineto{\pgfpoint{337.008759pt}{77.008850pt}}
\pgfusepath{stroke}
\pgfpathmoveto{\pgfpoint{336.960022pt}{77.003510pt}}
\pgflineto{\pgfpoint{336.976379pt}{77.079269pt}}
\pgfusepath{stroke}
\pgfpathmoveto{\pgfpoint{336.974274pt}{83.074387pt}}
\pgflineto{\pgfpoint{337.008148pt}{83.001740pt}}
\pgfusepath{stroke}
\pgfpathmoveto{\pgfpoint{336.957794pt}{82.995941pt}}
\pgflineto{\pgfpoint{336.974274pt}{83.074387pt}}
\pgfusepath{stroke}
\pgfpathmoveto{\pgfpoint{336.971863pt}{89.069969pt}}
\pgflineto{\pgfpoint{337.007446pt}{88.994987pt}}
\pgfusepath{stroke}
\pgfpathmoveto{\pgfpoint{336.955322pt}{88.988640pt}}
\pgflineto{\pgfpoint{336.971863pt}{89.069969pt}}
\pgfusepath{stroke}
\pgfpathmoveto{\pgfpoint{336.969147pt}{95.066093pt}}
\pgflineto{\pgfpoint{337.006531pt}{94.988617pt}}
\pgfusepath{stroke}
\pgfpathmoveto{\pgfpoint{336.952576pt}{94.981674pt}}
\pgflineto{\pgfpoint{336.969147pt}{95.066093pt}}
\pgfusepath{stroke}
\pgfpathmoveto{\pgfpoint{336.966034pt}{101.062767pt}}
\pgflineto{\pgfpoint{337.005463pt}{100.982658pt}}
\pgfusepath{stroke}
\pgfpathmoveto{\pgfpoint{336.949524pt}{100.975021pt}}
\pgflineto{\pgfpoint{336.966034pt}{101.062767pt}}
\pgfusepath{stroke}
\pgfpathmoveto{\pgfpoint{336.962494pt}{107.060051pt}}
\pgflineto{\pgfpoint{337.004150pt}{106.977165pt}}
\pgfusepath{stroke}
\pgfpathmoveto{\pgfpoint{336.946075pt}{106.968742pt}}
\pgflineto{\pgfpoint{336.962494pt}{107.060051pt}}
\pgfusepath{stroke}
\pgfpathmoveto{\pgfpoint{336.958374pt}{113.058022pt}}
\pgflineto{\pgfpoint{337.002563pt}{112.972183pt}}
\pgfusepath{stroke}
\pgfpathmoveto{\pgfpoint{336.942230pt}{112.962852pt}}
\pgflineto{\pgfpoint{336.958374pt}{113.058022pt}}
\pgfusepath{stroke}
\pgfpathmoveto{\pgfpoint{336.953644pt}{119.056709pt}}
\pgflineto{\pgfpoint{337.000610pt}{118.967766pt}}
\pgfusepath{stroke}
\pgfpathmoveto{\pgfpoint{336.937866pt}{118.957367pt}}
\pgflineto{\pgfpoint{336.953644pt}{119.056709pt}}
\pgfusepath{stroke}
\pgfpathmoveto{\pgfpoint{336.948151pt}{125.056175pt}}
\pgflineto{\pgfpoint{336.998230pt}{124.963936pt}}
\pgfusepath{stroke}
\pgfpathmoveto{\pgfpoint{336.932861pt}{124.952324pt}}
\pgflineto{\pgfpoint{336.948151pt}{125.056175pt}}
\pgfusepath{stroke}
\pgfpathmoveto{\pgfpoint{336.941711pt}{131.056488pt}}
\pgflineto{\pgfpoint{336.995331pt}{130.960770pt}}
\pgfusepath{stroke}
\pgfpathmoveto{\pgfpoint{336.927185pt}{130.947723pt}}
\pgflineto{\pgfpoint{336.941711pt}{131.056488pt}}
\pgfusepath{stroke}
\pgfpathmoveto{\pgfpoint{336.934113pt}{137.057648pt}}
\pgflineto{\pgfpoint{336.991791pt}{136.958313pt}}
\pgfusepath{stroke}
\pgfpathmoveto{\pgfpoint{336.920654pt}{136.943573pt}}
\pgflineto{\pgfpoint{336.934113pt}{137.057648pt}}
\pgfusepath{stroke}
\pgfpathmoveto{\pgfpoint{336.925171pt}{143.059723pt}}
\pgflineto{\pgfpoint{336.987396pt}{142.956604pt}}
\pgfusepath{stroke}
\pgfpathmoveto{\pgfpoint{336.913086pt}{142.939880pt}}
\pgflineto{\pgfpoint{336.925171pt}{143.059723pt}}
\pgfusepath{stroke}
\pgfpathmoveto{\pgfpoint{336.914490pt}{149.062683pt}}
\pgflineto{\pgfpoint{336.981995pt}{148.955688pt}}
\pgfusepath{stroke}
\pgfpathmoveto{\pgfpoint{336.904297pt}{148.936600pt}}
\pgflineto{\pgfpoint{336.914490pt}{149.062683pt}}
\pgfusepath{stroke}
\pgfpathmoveto{\pgfpoint{336.901764pt}{155.066467pt}}
\pgflineto{\pgfpoint{336.975281pt}{154.955536pt}}
\pgfusepath{stroke}
\pgfpathmoveto{\pgfpoint{336.894012pt}{154.933609pt}}
\pgflineto{\pgfpoint{336.901764pt}{155.066467pt}}
\pgfusepath{stroke}
\pgfpathmoveto{\pgfpoint{336.886475pt}{161.070892pt}}
\pgflineto{\pgfpoint{336.966919pt}{160.956100pt}}
\pgfusepath{stroke}
\pgfpathmoveto{\pgfpoint{336.881958pt}{160.930786pt}}
\pgflineto{\pgfpoint{336.886475pt}{161.070892pt}}
\pgfusepath{stroke}
\pgfpathmoveto{\pgfpoint{336.868042pt}{167.075638pt}}
\pgflineto{\pgfpoint{336.956543pt}{166.957245pt}}
\pgfusepath{stroke}
\pgfpathmoveto{\pgfpoint{336.867798pt}{166.927826pt}}
\pgflineto{\pgfpoint{336.868042pt}{167.075638pt}}
\pgfusepath{stroke}
\pgfpathmoveto{\pgfpoint{336.845886pt}{173.080139pt}}
\pgflineto{\pgfpoint{336.943604pt}{172.958618pt}}
\pgfusepath{stroke}
\pgfpathmoveto{\pgfpoint{336.851166pt}{172.924286pt}}
\pgflineto{\pgfpoint{336.845886pt}{173.080139pt}}
\pgfusepath{stroke}
\pgfpathmoveto{\pgfpoint{336.819275pt}{179.083466pt}}
\pgflineto{\pgfpoint{336.927612pt}{178.959717pt}}
\pgfusepath{stroke}
\pgfpathmoveto{\pgfpoint{336.831696pt}{178.919464pt}}
\pgflineto{\pgfpoint{336.819275pt}{179.083466pt}}
\pgfusepath{stroke}
\pgfpathmoveto{\pgfpoint{336.787750pt}{185.084122pt}}
\pgflineto{\pgfpoint{336.908020pt}{184.959564pt}}
\pgfusepath{stroke}
\pgfpathmoveto{\pgfpoint{336.809235pt}{184.912308pt}}
\pgflineto{\pgfpoint{336.787750pt}{185.084122pt}}
\pgfusepath{stroke}
\pgfpathmoveto{\pgfpoint{336.751129pt}{191.079910pt}}
\pgflineto{\pgfpoint{336.884491pt}{190.956757pt}}
\pgfusepath{stroke}
\pgfpathmoveto{\pgfpoint{336.783936pt}{190.901367pt}}
\pgflineto{\pgfpoint{336.751129pt}{191.079910pt}}
\pgfusepath{stroke}
\pgfpathmoveto{\pgfpoint{336.710236pt}{197.067932pt}}
\pgflineto{\pgfpoint{336.857269pt}{196.949219pt}}
\pgfusepath{stroke}
\pgfpathmoveto{\pgfpoint{336.756653pt}{196.884735pt}}
\pgflineto{\pgfpoint{336.710236pt}{197.067932pt}}
\pgfusepath{stroke}
\pgfpathmoveto{\pgfpoint{336.667664pt}{203.044739pt}}
\pgflineto{\pgfpoint{336.827667pt}{202.934402pt}}
\pgfusepath{stroke}
\pgfpathmoveto{\pgfpoint{336.729462pt}{202.860474pt}}
\pgflineto{\pgfpoint{336.667664pt}{203.044739pt}}
\pgfusepath{stroke}
\pgfpathmoveto{\pgfpoint{336.628448pt}{209.007431pt}}
\pgflineto{\pgfpoint{336.798767pt}{208.909805pt}}
\pgfusepath{stroke}
\pgfpathmoveto{\pgfpoint{336.706116pt}{208.827164pt}}
\pgflineto{\pgfpoint{336.628448pt}{209.007431pt}}
\pgfusepath{stroke}
\pgfpathmoveto{\pgfpoint{336.600372pt}{214.955460pt}}
\pgflineto{\pgfpoint{336.775635pt}{214.874207pt}}
\pgfusepath{stroke}
\pgfpathmoveto{\pgfpoint{336.691833pt}{214.785309pt}}
\pgflineto{\pgfpoint{336.600372pt}{214.955460pt}}
\pgfusepath{stroke}
\pgfpathmoveto{\pgfpoint{336.592285pt}{220.892715pt}}
\pgflineto{\pgfpoint{336.764587pt}{220.829193pt}}
\pgfusepath{stroke}
\pgfpathmoveto{\pgfpoint{336.692017pt}{220.738495pt}}
\pgflineto{\pgfpoint{336.592285pt}{220.892715pt}}
\pgfusepath{stroke}
\pgfpathmoveto{\pgfpoint{336.610840pt}{226.827866pt}}
\pgflineto{\pgfpoint{336.771149pt}{226.779816pt}}
\pgfusepath{stroke}
\pgfpathmoveto{\pgfpoint{336.710266pt}{226.693253pt}}
\pgflineto{\pgfpoint{336.610840pt}{226.827866pt}}
\pgfusepath{stroke}
\pgfpathmoveto{\pgfpoint{336.657806pt}{232.771744pt}}
\pgflineto{\pgfpoint{336.797577pt}{232.733109pt}}
\pgfusepath{stroke}
\pgfpathmoveto{\pgfpoint{336.746429pt}{232.656967pt}}
\pgflineto{\pgfpoint{336.657806pt}{232.771744pt}}
\pgfusepath{stroke}
\pgfpathmoveto{\pgfpoint{336.730103pt}{238.732895pt}}
\pgflineto{\pgfpoint{336.842834pt}{238.695084pt}}
\pgfusepath{stroke}
\pgfpathmoveto{\pgfpoint{336.797607pt}{238.635025pt}}
\pgflineto{\pgfpoint{336.730103pt}{238.732895pt}}
\pgfusepath{stroke}
\pgfpathmoveto{\pgfpoint{336.823639pt}{244.714554pt}}
\pgflineto{\pgfpoint{336.904480pt}{244.668396pt}}
\pgfusepath{stroke}
\pgfpathmoveto{\pgfpoint{336.860626pt}{244.629105pt}}
\pgflineto{\pgfpoint{336.823639pt}{244.714554pt}}
\pgfusepath{stroke}
\pgfpathmoveto{\pgfpoint{336.937103pt}{250.714203pt}}
\pgflineto{\pgfpoint{336.981476pt}{250.651459pt}}
\pgfusepath{stroke}
\pgfpathmoveto{\pgfpoint{336.934967pt}{250.637390pt}}
\pgflineto{\pgfpoint{336.937103pt}{250.714203pt}}
\pgfusepath{stroke}
\pgfpathmoveto{\pgfpoint{337.074554pt}{256.723328pt}}
\pgflineto{\pgfpoint{337.075470pt}{256.638184pt}}
\pgfusepath{stroke}
\pgfpathmoveto{\pgfpoint{337.024200pt}{256.654663pt}}
\pgflineto{\pgfpoint{337.074554pt}{256.723328pt}}
\pgfusepath{stroke}
\pgfpathmoveto{\pgfpoint{337.244476pt}{262.724487pt}}
\pgflineto{\pgfpoint{337.190216pt}{262.615997pt}}
\pgfusepath{stroke}
\pgfpathmoveto{\pgfpoint{337.135986pt}{262.670227pt}}
\pgflineto{\pgfpoint{337.244476pt}{262.724487pt}}
\pgfusepath{stroke}
\pgfpathmoveto{\pgfpoint{337.451843pt}{268.684143pt}}
\pgflineto{\pgfpoint{337.325348pt}{268.561981pt}}
\pgfusepath{stroke}
\pgfpathmoveto{\pgfpoint{337.277344pt}{268.662292pt}}
\pgflineto{\pgfpoint{337.451843pt}{268.684143pt}}
\pgfusepath{stroke}
\pgfpathmoveto{\pgfpoint{337.672974pt}{274.551147pt}}
\pgflineto{\pgfpoint{337.459351pt}{274.444672pt}}
\pgfusepath{stroke}
\pgfpathmoveto{\pgfpoint{337.438202pt}{274.594116pt}}
\pgflineto{\pgfpoint{337.672974pt}{274.551147pt}}
\pgfusepath{stroke}
\pgfpathmoveto{\pgfpoint{337.823273pt}{280.298462pt}}
\pgflineto{\pgfpoint{337.532867pt}{280.255432pt}}
\pgfusepath{stroke}
\pgfpathmoveto{\pgfpoint{337.565125pt}{280.438293pt}}
\pgflineto{\pgfpoint{337.823273pt}{280.298462pt}}
\pgfusepath{stroke}
\pgfpathmoveto{\pgfpoint{337.809784pt}{286.004883pt}}
\pgflineto{\pgfpoint{337.492584pt}{286.057129pt}}
\pgfusepath{stroke}
\pgfpathmoveto{\pgfpoint{337.587372pt}{286.236969pt}}
\pgflineto{\pgfpoint{337.809784pt}{286.004883pt}}
\pgfusepath{stroke}
\pgfpathmoveto{\pgfpoint{337.662109pt}{291.799866pt}}
\pgflineto{\pgfpoint{337.372711pt}{291.932800pt}}
\pgfusepath{stroke}
\pgfpathmoveto{\pgfpoint{337.510345pt}{292.079834pt}}
\pgflineto{\pgfpoint{337.662109pt}{291.799866pt}}
\pgfusepath{stroke}
\pgfpathmoveto{\pgfpoint{337.487183pt}{297.714935pt}}
\pgflineto{\pgfpoint{337.248047pt}{297.891510pt}}
\pgfusepath{stroke}
\pgfpathmoveto{\pgfpoint{337.401825pt}{297.999695pt}}
\pgflineto{\pgfpoint{337.487183pt}{297.714935pt}}
\pgfusepath{stroke}
\pgfpathmoveto{\pgfpoint{337.344940pt}{303.705383pt}}
\pgflineto{\pgfpoint{337.153503pt}{303.896881pt}}
\pgfusepath{stroke}
\pgfpathmoveto{\pgfpoint{337.306702pt}{303.973450pt}}
\pgflineto{\pgfpoint{337.344940pt}{303.705383pt}}
\pgfusepath{stroke}
\pgfpathmoveto{\pgfpoint{337.243561pt}{309.728241pt}}
\pgflineto{\pgfpoint{337.089813pt}{309.919281pt}}
\pgfusepath{stroke}
\pgfpathmoveto{\pgfpoint{337.235199pt}{309.973328pt}}
\pgflineto{\pgfpoint{337.243561pt}{309.728241pt}}
\pgfusepath{stroke}
\pgfpathmoveto{\pgfpoint{337.174011pt}{315.760742pt}}
\pgflineto{\pgfpoint{337.048431pt}{315.944580pt}}
\pgfusepath{stroke}
\pgfpathmoveto{\pgfpoint{337.183868pt}{315.983185pt}}
\pgflineto{\pgfpoint{337.174011pt}{315.760742pt}}
\pgfusepath{stroke}
\pgfpathmoveto{\pgfpoint{337.126343pt}{321.793274pt}}
\pgflineto{\pgfpoint{337.021606pt}{321.967499pt}}
\pgfusepath{stroke}
\pgfpathmoveto{\pgfpoint{337.147095pt}{321.995483pt}}
\pgflineto{\pgfpoint{337.126343pt}{321.793274pt}}
\pgfusepath{stroke}
\pgfpathmoveto{\pgfpoint{337.093231pt}{327.822388pt}}
\pgflineto{\pgfpoint{337.004150pt}{327.986511pt}}
\pgfusepath{stroke}
\pgfpathmoveto{\pgfpoint{337.120422pt}{328.007141pt}}
\pgflineto{\pgfpoint{337.093231pt}{327.822388pt}}
\pgfusepath{stroke}
\pgfpathmoveto{\pgfpoint{337.069794pt}{333.847229pt}}
\pgflineto{\pgfpoint{336.992645pt}{334.001587pt}}
\pgfusepath{stroke}
\pgfpathmoveto{\pgfpoint{337.100677pt}{334.017029pt}}
\pgflineto{\pgfpoint{337.069794pt}{333.847229pt}}
\pgfusepath{stroke}
\pgfpathmoveto{\pgfpoint{337.052917pt}{339.867859pt}}
\pgflineto{\pgfpoint{336.985046pt}{340.013153pt}}
\pgfusepath{stroke}
\pgfpathmoveto{\pgfpoint{337.085785pt}{340.024841pt}}
\pgflineto{\pgfpoint{337.052917pt}{339.867859pt}}
\pgfusepath{stroke}
\pgfpathmoveto{\pgfpoint{337.040527pt}{345.884705pt}}
\pgflineto{\pgfpoint{336.980011pt}{346.021667pt}}
\pgfusepath{stroke}
\pgfpathmoveto{\pgfpoint{337.074280pt}{346.030579pt}}
\pgflineto{\pgfpoint{337.040527pt}{345.884705pt}}
\pgfusepath{stroke}
\pgfpathmoveto{\pgfpoint{337.031311pt}{351.898254pt}}
\pgflineto{\pgfpoint{336.976715pt}{352.027649pt}}
\pgfusepath{stroke}
\pgfpathmoveto{\pgfpoint{337.065277pt}{352.034485pt}}
\pgflineto{\pgfpoint{337.031311pt}{351.898254pt}}
\pgfusepath{stroke}
\pgfpathmoveto{\pgfpoint{337.024323pt}{357.908936pt}}
\pgflineto{\pgfpoint{336.974609pt}{358.031433pt}}
\pgfusepath{stroke}
\pgfpathmoveto{\pgfpoint{337.058075pt}{358.036774pt}}
\pgflineto{\pgfpoint{337.024323pt}{357.908936pt}}
\pgfusepath{stroke}
\pgfpathmoveto{\pgfpoint{337.018982pt}{363.917175pt}}
\pgflineto{\pgfpoint{336.973328pt}{364.033447pt}}
\pgfusepath{stroke}
\pgfpathmoveto{\pgfpoint{337.052246pt}{364.037598pt}}
\pgflineto{\pgfpoint{337.018982pt}{363.917175pt}}
\pgfusepath{stroke}
\pgfpathmoveto{\pgfpoint{337.014862pt}{369.923340pt}}
\pgflineto{\pgfpoint{336.972626pt}{370.033936pt}}
\pgfusepath{stroke}
\pgfpathmoveto{\pgfpoint{337.047455pt}{370.037170pt}}
\pgflineto{\pgfpoint{337.014862pt}{369.923340pt}}
\pgfusepath{stroke}
\pgfpathmoveto{\pgfpoint{342.957153pt}{77.077057pt}}
\pgflineto{\pgfpoint{342.991608pt}{77.008148pt}}
\pgfusepath{stroke}
\pgfpathmoveto{\pgfpoint{342.943359pt}{77.001266pt}}
\pgflineto{\pgfpoint{342.957153pt}{77.077057pt}}
\pgfusepath{stroke}
\pgfpathmoveto{\pgfpoint{342.954590pt}{83.071899pt}}
\pgflineto{\pgfpoint{342.990692pt}{83.000900pt}}
\pgfusepath{stroke}
\pgfpathmoveto{\pgfpoint{342.940857pt}{82.993454pt}}
\pgflineto{\pgfpoint{342.954590pt}{83.071899pt}}
\pgfusepath{stroke}
\pgfpathmoveto{\pgfpoint{342.951691pt}{89.067177pt}}
\pgflineto{\pgfpoint{342.989594pt}{88.993996pt}}
\pgfusepath{stroke}
\pgfpathmoveto{\pgfpoint{342.938110pt}{88.985901pt}}
\pgflineto{\pgfpoint{342.951691pt}{89.067177pt}}
\pgfusepath{stroke}
\pgfpathmoveto{\pgfpoint{342.948456pt}{95.062920pt}}
\pgflineto{\pgfpoint{342.988312pt}{94.987434pt}}
\pgfusepath{stroke}
\pgfpathmoveto{\pgfpoint{342.935059pt}{94.978615pt}}
\pgflineto{\pgfpoint{342.948456pt}{95.062920pt}}
\pgfusepath{stroke}
\pgfpathmoveto{\pgfpoint{342.944794pt}{101.059143pt}}
\pgflineto{\pgfpoint{342.986816pt}{100.981239pt}}
\pgfusepath{stroke}
\pgfpathmoveto{\pgfpoint{342.931671pt}{100.971626pt}}
\pgflineto{\pgfpoint{342.944794pt}{101.059143pt}}
\pgfusepath{stroke}
\pgfpathmoveto{\pgfpoint{342.940643pt}{107.055931pt}}
\pgflineto{\pgfpoint{342.985046pt}{106.975479pt}}
\pgfusepath{stroke}
\pgfpathmoveto{\pgfpoint{342.927917pt}{106.964935pt}}
\pgflineto{\pgfpoint{342.940643pt}{107.055931pt}}
\pgfusepath{stroke}
\pgfpathmoveto{\pgfpoint{342.935913pt}{113.053284pt}}
\pgflineto{\pgfpoint{342.982971pt}{112.970161pt}}
\pgfusepath{stroke}
\pgfpathmoveto{\pgfpoint{342.923676pt}{112.958549pt}}
\pgflineto{\pgfpoint{342.935913pt}{113.053284pt}}
\pgfusepath{stroke}
\pgfpathmoveto{\pgfpoint{342.930481pt}{119.051247pt}}
\pgflineto{\pgfpoint{342.980469pt}{118.965332pt}}
\pgfusepath{stroke}
\pgfpathmoveto{\pgfpoint{342.918945pt}{118.952515pt}}
\pgflineto{\pgfpoint{342.930481pt}{119.051247pt}}
\pgfusepath{stroke}
\pgfpathmoveto{\pgfpoint{342.924255pt}{125.049843pt}}
\pgflineto{\pgfpoint{342.977539pt}{124.961006pt}}
\pgfusepath{stroke}
\pgfpathmoveto{\pgfpoint{342.913574pt}{124.946808pt}}
\pgflineto{\pgfpoint{342.924255pt}{125.049843pt}}
\pgfusepath{stroke}
\pgfpathmoveto{\pgfpoint{342.917053pt}{131.049103pt}}
\pgflineto{\pgfpoint{342.973999pt}{130.957230pt}}
\pgfusepath{stroke}
\pgfpathmoveto{\pgfpoint{342.907501pt}{130.941437pt}}
\pgflineto{\pgfpoint{342.917053pt}{131.049103pt}}
\pgfusepath{stroke}
\pgfpathmoveto{\pgfpoint{342.908691pt}{137.049042pt}}
\pgflineto{\pgfpoint{342.969788pt}{136.954041pt}}
\pgfusepath{stroke}
\pgfpathmoveto{\pgfpoint{342.900574pt}{136.936371pt}}
\pgflineto{\pgfpoint{342.908691pt}{137.049042pt}}
\pgfusepath{stroke}
\pgfpathmoveto{\pgfpoint{342.898987pt}{143.049591pt}}
\pgflineto{\pgfpoint{342.964752pt}{142.951385pt}}
\pgfusepath{stroke}
\pgfpathmoveto{\pgfpoint{342.892670pt}{142.931580pt}}
\pgflineto{\pgfpoint{342.898987pt}{143.049591pt}}
\pgfusepath{stroke}
\pgfpathmoveto{\pgfpoint{342.887634pt}{149.050720pt}}
\pgflineto{\pgfpoint{342.958679pt}{148.949326pt}}
\pgfusepath{stroke}
\pgfpathmoveto{\pgfpoint{342.883636pt}{148.926987pt}}
\pgflineto{\pgfpoint{342.887634pt}{149.050720pt}}
\pgfusepath{stroke}
\pgfpathmoveto{\pgfpoint{342.874359pt}{155.052277pt}}
\pgflineto{\pgfpoint{342.951355pt}{154.947754pt}}
\pgfusepath{stroke}
\pgfpathmoveto{\pgfpoint{342.873230pt}{154.922485pt}}
\pgflineto{\pgfpoint{342.874359pt}{155.052277pt}}
\pgfusepath{stroke}
\pgfpathmoveto{\pgfpoint{342.858795pt}{161.054047pt}}
\pgflineto{\pgfpoint{342.942535pt}{160.946594pt}}
\pgfusepath{stroke}
\pgfpathmoveto{\pgfpoint{342.861298pt}{160.917847pt}}
\pgflineto{\pgfpoint{342.858795pt}{161.054047pt}}
\pgfusepath{stroke}
\pgfpathmoveto{\pgfpoint{342.840576pt}{167.055664pt}}
\pgflineto{\pgfpoint{342.931915pt}{166.945618pt}}
\pgfusepath{stroke}
\pgfpathmoveto{\pgfpoint{342.847595pt}{166.912811pt}}
\pgflineto{\pgfpoint{342.840576pt}{167.055664pt}}
\pgfusepath{stroke}
\pgfpathmoveto{\pgfpoint{342.819305pt}{173.056534pt}}
\pgflineto{\pgfpoint{342.919189pt}{172.944473pt}}
\pgfusepath{stroke}
\pgfpathmoveto{\pgfpoint{342.831970pt}{172.906952pt}}
\pgflineto{\pgfpoint{342.819305pt}{173.056534pt}}
\pgfusepath{stroke}
\pgfpathmoveto{\pgfpoint{342.794739pt}{179.055817pt}}
\pgflineto{\pgfpoint{342.904083pt}{178.942642pt}}
\pgfusepath{stroke}
\pgfpathmoveto{\pgfpoint{342.814331pt}{178.899658pt}}
\pgflineto{\pgfpoint{342.794739pt}{179.055817pt}}
\pgfusepath{stroke}
\pgfpathmoveto{\pgfpoint{342.766785pt}{185.052322pt}}
\pgflineto{\pgfpoint{342.886444pt}{184.939346pt}}
\pgfusepath{stroke}
\pgfpathmoveto{\pgfpoint{342.794708pt}{184.890152pt}}
\pgflineto{\pgfpoint{342.766785pt}{185.052322pt}}
\pgfusepath{stroke}
\pgfpathmoveto{\pgfpoint{342.735901pt}{191.044525pt}}
\pgflineto{\pgfpoint{342.866333pt}{190.933487pt}}
\pgfusepath{stroke}
\pgfpathmoveto{\pgfpoint{342.773621pt}{190.877426pt}}
\pgflineto{\pgfpoint{342.735901pt}{191.044525pt}}
\pgfusepath{stroke}
\pgfpathmoveto{\pgfpoint{342.703247pt}{197.030609pt}}
\pgflineto{\pgfpoint{342.844391pt}{196.923767pt}}
\pgfusepath{stroke}
\pgfpathmoveto{\pgfpoint{342.752045pt}{196.860443pt}}
\pgflineto{\pgfpoint{342.703247pt}{197.030609pt}}
\pgfusepath{stroke}
\pgfpathmoveto{\pgfpoint{342.671204pt}{203.008911pt}}
\pgflineto{\pgfpoint{342.821960pt}{202.908768pt}}
\pgfusepath{stroke}
\pgfpathmoveto{\pgfpoint{342.731750pt}{202.838333pt}}
\pgflineto{\pgfpoint{342.671204pt}{203.008911pt}}
\pgfusepath{stroke}
\pgfpathmoveto{\pgfpoint{342.643555pt}{208.978516pt}}
\pgflineto{\pgfpoint{342.801514pt}{208.887497pt}}
\pgfusepath{stroke}
\pgfpathmoveto{\pgfpoint{342.715332pt}{208.810928pt}}
\pgflineto{\pgfpoint{342.643555pt}{208.978516pt}}
\pgfusepath{stroke}
\pgfpathmoveto{\pgfpoint{342.625244pt}{214.940308pt}}
\pgflineto{\pgfpoint{342.786438pt}{214.859970pt}}
\pgfusepath{stroke}
\pgfpathmoveto{\pgfpoint{342.705994pt}{214.779327pt}}
\pgflineto{\pgfpoint{342.625244pt}{214.940308pt}}
\pgfusepath{stroke}
\pgfpathmoveto{\pgfpoint{342.621368pt}{220.897659pt}}
\pgflineto{\pgfpoint{342.780487pt}{220.827896pt}}
\pgfusepath{stroke}
\pgfpathmoveto{\pgfpoint{342.706787pt}{220.746399pt}}
\pgflineto{\pgfpoint{342.621368pt}{220.897659pt}}
\pgfusepath{stroke}
\pgfpathmoveto{\pgfpoint{342.635803pt}{226.856476pt}}
\pgflineto{\pgfpoint{342.786865pt}{226.794754pt}}
\pgfusepath{stroke}
\pgfpathmoveto{\pgfpoint{342.719635pt}{226.716446pt}}
\pgflineto{\pgfpoint{342.635803pt}{226.856476pt}}
\pgfusepath{stroke}
\pgfpathmoveto{\pgfpoint{342.670105pt}{232.823837pt}}
\pgflineto{\pgfpoint{342.807495pt}{232.765106pt}}
\pgfusepath{stroke}
\pgfpathmoveto{\pgfpoint{342.744781pt}{232.694427pt}}
\pgflineto{\pgfpoint{342.670105pt}{232.823837pt}}
\pgfusepath{stroke}
\pgfpathmoveto{\pgfpoint{342.723969pt}{238.806244pt}}
\pgflineto{\pgfpoint{342.842804pt}{238.743317pt}}
\pgfusepath{stroke}
\pgfpathmoveto{\pgfpoint{342.781281pt}{238.684601pt}}
\pgflineto{\pgfpoint{342.723969pt}{238.806244pt}}
\pgfusepath{stroke}
\pgfpathmoveto{\pgfpoint{342.797211pt}{244.808243pt}}
\pgflineto{\pgfpoint{342.893188pt}{244.732422pt}}
\pgfusepath{stroke}
\pgfpathmoveto{\pgfpoint{342.828522pt}{244.689987pt}}
\pgflineto{\pgfpoint{342.797211pt}{244.808243pt}}
\pgfusepath{stroke}
\pgfpathmoveto{\pgfpoint{342.892822pt}{250.831940pt}}
\pgflineto{\pgfpoint{342.960938pt}{250.733521pt}}
\pgfusepath{stroke}
\pgfpathmoveto{\pgfpoint{342.888245pt}{250.712341pt}}
\pgflineto{\pgfpoint{342.892822pt}{250.831940pt}}
\pgfusepath{stroke}
\pgfpathmoveto{\pgfpoint{343.020935pt}{256.876251pt}}
\pgflineto{\pgfpoint{343.052612pt}{256.744751pt}}
\pgfusepath{stroke}
\pgfpathmoveto{\pgfpoint{342.967377pt}{256.752014pt}}
\pgflineto{\pgfpoint{343.020935pt}{256.876251pt}}
\pgfusepath{stroke}
\pgfpathmoveto{\pgfpoint{343.204132pt}{262.931824pt}}
\pgflineto{\pgfpoint{343.182312pt}{262.757355pt}}
\pgfusepath{stroke}
\pgfpathmoveto{\pgfpoint{343.081970pt}{262.805359pt}}
\pgflineto{\pgfpoint{343.204132pt}{262.931824pt}}
\pgfusepath{stroke}
\pgfpathmoveto{\pgfpoint{343.481323pt}{268.961334pt}}
\pgflineto{\pgfpoint{343.371735pt}{268.742157pt}}
\pgfusepath{stroke}
\pgfpathmoveto{\pgfpoint{343.262146pt}{268.851746pt}}
\pgflineto{\pgfpoint{343.481323pt}{268.961334pt}}
\pgfusepath{stroke}
\pgfpathmoveto{\pgfpoint{343.875366pt}{274.848877pt}}
\pgflineto{\pgfpoint{343.623291pt}{274.619324pt}}
\pgfusepath{stroke}
\pgfpathmoveto{\pgfpoint{343.535950pt}{274.816467pt}}
\pgflineto{\pgfpoint{343.875366pt}{274.848877pt}}
\pgfusepath{stroke}
\pgfpathmoveto{\pgfpoint{344.219482pt}{280.413940pt}}
\pgflineto{\pgfpoint{343.805725pt}{280.287048pt}}
\pgfusepath{stroke}
\pgfpathmoveto{\pgfpoint{343.812347pt}{280.560669pt}}
\pgflineto{\pgfpoint{344.219482pt}{280.413940pt}}
\pgfusepath{stroke}
\pgfpathmoveto{\pgfpoint{344.142792pt}{285.828796pt}}
\pgflineto{\pgfpoint{343.690948pt}{285.901398pt}}
\pgfusepath{stroke}
\pgfpathmoveto{\pgfpoint{343.824860pt}{286.157959pt}}
\pgflineto{\pgfpoint{344.142792pt}{285.828796pt}}
\pgfusepath{stroke}
\pgfpathmoveto{\pgfpoint{343.777985pt}{291.539551pt}}
\pgflineto{\pgfpoint{343.416962pt}{291.744995pt}}
\pgfusepath{stroke}
\pgfpathmoveto{\pgfpoint{343.612427pt}{291.920502pt}}
\pgflineto{\pgfpoint{343.777985pt}{291.539551pt}}
\pgfusepath{stroke}
\pgfpathmoveto{\pgfpoint{343.470886pt}{297.512207pt}}
\pgflineto{\pgfpoint{343.210571pt}{297.756805pt}}
\pgfusepath{stroke}
\pgfpathmoveto{\pgfpoint{343.409393pt}{297.864075pt}}
\pgflineto{\pgfpoint{343.470886pt}{297.512207pt}}
\pgfusepath{stroke}
\pgfpathmoveto{\pgfpoint{343.281769pt}{303.571625pt}}
\pgflineto{\pgfpoint{343.092438pt}{303.813354pt}}
\pgfusepath{stroke}
\pgfpathmoveto{\pgfpoint{343.275360pt}{303.878601pt}}
\pgflineto{\pgfpoint{343.281769pt}{303.571625pt}}
\pgfusepath{stroke}
\pgfpathmoveto{\pgfpoint{343.170929pt}{309.642761pt}}
\pgflineto{\pgfpoint{343.027802pt}{309.869019pt}}
\pgfusepath{stroke}
\pgfpathmoveto{\pgfpoint{343.192200pt}{309.909607pt}}
\pgflineto{\pgfpoint{343.170929pt}{309.642761pt}}
\pgfusepath{stroke}
\pgfpathmoveto{\pgfpoint{343.104584pt}{315.705475pt}}
\pgflineto{\pgfpoint{342.991913pt}{315.914124pt}}
\pgfusepath{stroke}
\pgfpathmoveto{\pgfpoint{343.139618pt}{315.940002pt}}
\pgflineto{\pgfpoint{343.104584pt}{315.705475pt}}
\pgfusepath{stroke}
\pgfpathmoveto{\pgfpoint{343.063416pt}{321.756714pt}}
\pgflineto{\pgfpoint{342.971527pt}{321.948792pt}}
\pgfusepath{stroke}
\pgfpathmoveto{\pgfpoint{343.105133pt}{321.965515pt}}
\pgflineto{\pgfpoint{343.063416pt}{321.756714pt}}
\pgfusepath{stroke}
\pgfpathmoveto{\pgfpoint{343.036987pt}{327.797607pt}}
\pgflineto{\pgfpoint{342.959808pt}{327.974884pt}}
\pgfusepath{stroke}
\pgfpathmoveto{\pgfpoint{343.081604pt}{327.985748pt}}
\pgflineto{\pgfpoint{343.036987pt}{327.797607pt}}
\pgfusepath{stroke}
\pgfpathmoveto{\pgfpoint{343.019501pt}{333.830048pt}}
\pgflineto{\pgfpoint{342.953125pt}{333.994354pt}}
\pgfusepath{stroke}
\pgfpathmoveto{\pgfpoint{343.065002pt}{334.001343pt}}
\pgflineto{\pgfpoint{343.019501pt}{333.830048pt}}
\pgfusepath{stroke}
\pgfpathmoveto{\pgfpoint{343.007660pt}{339.855682pt}}
\pgflineto{\pgfpoint{342.949432pt}{340.008698pt}}
\pgfusepath{stroke}
\pgfpathmoveto{\pgfpoint{343.052887pt}{340.013031pt}}
\pgflineto{\pgfpoint{343.007660pt}{339.855682pt}}
\pgfusepath{stroke}
\pgfpathmoveto{\pgfpoint{342.999481pt}{345.875916pt}}
\pgflineto{\pgfpoint{342.947571pt}{346.019012pt}}
\pgfusepath{stroke}
\pgfpathmoveto{\pgfpoint{343.043823pt}{346.021545pt}}
\pgflineto{\pgfpoint{342.999481pt}{345.875916pt}}
\pgfusepath{stroke}
\pgfpathmoveto{\pgfpoint{342.993744pt}{351.891785pt}}
\pgflineto{\pgfpoint{342.946899pt}{352.026154pt}}
\pgfusepath{stroke}
\pgfpathmoveto{\pgfpoint{343.036865pt}{352.027405pt}}
\pgflineto{\pgfpoint{342.993744pt}{351.891785pt}}
\pgfusepath{stroke}
\pgfpathmoveto{\pgfpoint{342.989685pt}{357.904144pt}}
\pgflineto{\pgfpoint{342.946930pt}{358.030762pt}}
\pgfusepath{stroke}
\pgfpathmoveto{\pgfpoint{343.031433pt}{358.031097pt}}
\pgflineto{\pgfpoint{342.989685pt}{357.904144pt}}
\pgfusepath{stroke}
\pgfpathmoveto{\pgfpoint{342.986816pt}{363.913574pt}}
\pgflineto{\pgfpoint{342.947418pt}{364.033295pt}}
\pgfusepath{stroke}
\pgfpathmoveto{\pgfpoint{343.027100pt}{364.033020pt}}
\pgflineto{\pgfpoint{342.986816pt}{363.913574pt}}
\pgfusepath{stroke}
\pgfpathmoveto{\pgfpoint{342.984772pt}{369.920624pt}}
\pgflineto{\pgfpoint{342.948181pt}{370.034119pt}}
\pgfusepath{stroke}
\pgfpathmoveto{\pgfpoint{343.023621pt}{370.033356pt}}
\pgflineto{\pgfpoint{342.984772pt}{369.920624pt}}
\pgfusepath{stroke}
\pgfpathmoveto{\pgfpoint{348.938171pt}{77.074448pt}}
\pgflineto{\pgfpoint{348.974579pt}{77.007141pt}}
\pgfusepath{stroke}
\pgfpathmoveto{\pgfpoint{348.926910pt}{76.998764pt}}
\pgflineto{\pgfpoint{348.938171pt}{77.074448pt}}
\pgfusepath{stroke}
\pgfpathmoveto{\pgfpoint{348.935211pt}{83.068985pt}}
\pgflineto{\pgfpoint{348.973358pt}{82.999756pt}}
\pgfusepath{stroke}
\pgfpathmoveto{\pgfpoint{348.924194pt}{82.990707pt}}
\pgflineto{\pgfpoint{348.935211pt}{83.068985pt}}
\pgfusepath{stroke}
\pgfpathmoveto{\pgfpoint{348.931885pt}{89.063904pt}}
\pgflineto{\pgfpoint{348.971954pt}{88.992661pt}}
\pgfusepath{stroke}
\pgfpathmoveto{\pgfpoint{348.921204pt}{88.982887pt}}
\pgflineto{\pgfpoint{348.931885pt}{89.063904pt}}
\pgfusepath{stroke}
\pgfpathmoveto{\pgfpoint{348.928223pt}{95.059235pt}}
\pgflineto{\pgfpoint{348.970337pt}{94.985878pt}}
\pgfusepath{stroke}
\pgfpathmoveto{\pgfpoint{348.917877pt}{94.975266pt}}
\pgflineto{\pgfpoint{348.928223pt}{95.059235pt}}
\pgfusepath{stroke}
\pgfpathmoveto{\pgfpoint{348.924072pt}{101.054985pt}}
\pgflineto{\pgfpoint{348.968445pt}{100.979431pt}}
\pgfusepath{stroke}
\pgfpathmoveto{\pgfpoint{348.914246pt}{100.967903pt}}
\pgflineto{\pgfpoint{348.924072pt}{101.054985pt}}
\pgfusepath{stroke}
\pgfpathmoveto{\pgfpoint{348.919403pt}{107.051216pt}}
\pgflineto{\pgfpoint{348.966278pt}{106.973351pt}}
\pgfusepath{stroke}
\pgfpathmoveto{\pgfpoint{348.910187pt}{106.960785pt}}
\pgflineto{\pgfpoint{348.919403pt}{107.051216pt}}
\pgfusepath{stroke}
\pgfpathmoveto{\pgfpoint{348.914124pt}{113.047935pt}}
\pgflineto{\pgfpoint{348.963745pt}{112.967667pt}}
\pgfusepath{stroke}
\pgfpathmoveto{\pgfpoint{348.905670pt}{112.953926pt}}
\pgflineto{\pgfpoint{348.914124pt}{113.047935pt}}
\pgfusepath{stroke}
\pgfpathmoveto{\pgfpoint{348.908112pt}{119.045143pt}}
\pgflineto{\pgfpoint{348.960815pt}{118.962387pt}}
\pgfusepath{stroke}
\pgfpathmoveto{\pgfpoint{348.900635pt}{118.947319pt}}
\pgflineto{\pgfpoint{348.908112pt}{119.045143pt}}
\pgfusepath{stroke}
\pgfpathmoveto{\pgfpoint{348.901306pt}{125.042862pt}}
\pgflineto{\pgfpoint{348.957367pt}{124.957542pt}}
\pgfusepath{stroke}
\pgfpathmoveto{\pgfpoint{348.894958pt}{124.940956pt}}
\pgflineto{\pgfpoint{348.901306pt}{125.042862pt}}
\pgfusepath{stroke}
\pgfpathmoveto{\pgfpoint{348.893555pt}{131.041077pt}}
\pgflineto{\pgfpoint{348.953369pt}{130.953125pt}}
\pgfusepath{stroke}
\pgfpathmoveto{\pgfpoint{348.888641pt}{130.934814pt}}
\pgflineto{\pgfpoint{348.893555pt}{131.041077pt}}
\pgfusepath{stroke}
\pgfpathmoveto{\pgfpoint{348.884644pt}{137.039764pt}}
\pgflineto{\pgfpoint{348.948639pt}{136.949158pt}}
\pgfusepath{stroke}
\pgfpathmoveto{\pgfpoint{348.881470pt}{136.928864pt}}
\pgflineto{\pgfpoint{348.884644pt}{137.039764pt}}
\pgfusepath{stroke}
\pgfpathmoveto{\pgfpoint{348.874451pt}{143.038849pt}}
\pgflineto{\pgfpoint{348.943115pt}{142.945618pt}}
\pgfusepath{stroke}
\pgfpathmoveto{\pgfpoint{348.873413pt}{142.923065pt}}
\pgflineto{\pgfpoint{348.874451pt}{143.038849pt}}
\pgfusepath{stroke}
\pgfpathmoveto{\pgfpoint{348.862732pt}{149.038284pt}}
\pgflineto{\pgfpoint{348.936584pt}{148.942444pt}}
\pgfusepath{stroke}
\pgfpathmoveto{\pgfpoint{348.864319pt}{148.917297pt}}
\pgflineto{\pgfpoint{348.862732pt}{149.038284pt}}
\pgfusepath{stroke}
\pgfpathmoveto{\pgfpoint{348.849243pt}{155.037857pt}}
\pgflineto{\pgfpoint{348.928925pt}{154.939575pt}}
\pgfusepath{stroke}
\pgfpathmoveto{\pgfpoint{348.854004pt}{154.911438pt}}
\pgflineto{\pgfpoint{348.849243pt}{155.037857pt}}
\pgfusepath{stroke}
\pgfpathmoveto{\pgfpoint{348.833801pt}{161.037354pt}}
\pgflineto{\pgfpoint{348.919922pt}{160.936874pt}}
\pgfusepath{stroke}
\pgfpathmoveto{\pgfpoint{348.842407pt}{160.905289pt}}
\pgflineto{\pgfpoint{348.833801pt}{161.037354pt}}
\pgfusepath{stroke}
\pgfpathmoveto{\pgfpoint{348.816132pt}{167.036377pt}}
\pgflineto{\pgfpoint{348.909393pt}{166.934113pt}}
\pgfusepath{stroke}
\pgfpathmoveto{\pgfpoint{348.829376pt}{166.898605pt}}
\pgflineto{\pgfpoint{348.816132pt}{167.036377pt}}
\pgfusepath{stroke}
\pgfpathmoveto{\pgfpoint{348.796082pt}{173.034454pt}}
\pgflineto{\pgfpoint{348.897156pt}{172.930969pt}}
\pgfusepath{stroke}
\pgfpathmoveto{\pgfpoint{348.814880pt}{172.891006pt}}
\pgflineto{\pgfpoint{348.796082pt}{173.034454pt}}
\pgfusepath{stroke}
\pgfpathmoveto{\pgfpoint{348.773590pt}{179.030884pt}}
\pgflineto{\pgfpoint{348.883148pt}{178.927002pt}}
\pgfusepath{stroke}
\pgfpathmoveto{\pgfpoint{348.798920pt}{178.882050pt}}
\pgflineto{\pgfpoint{348.773590pt}{179.030884pt}}
\pgfusepath{stroke}
\pgfpathmoveto{\pgfpoint{348.748901pt}{185.024796pt}}
\pgflineto{\pgfpoint{348.867371pt}{184.921616pt}}
\pgfusepath{stroke}
\pgfpathmoveto{\pgfpoint{348.781769pt}{184.871170pt}}
\pgflineto{\pgfpoint{348.748901pt}{185.024796pt}}
\pgfusepath{stroke}
\pgfpathmoveto{\pgfpoint{348.722595pt}{191.015244pt}}
\pgflineto{\pgfpoint{348.850159pt}{190.914078pt}}
\pgfusepath{stroke}
\pgfpathmoveto{\pgfpoint{348.763947pt}{190.857788pt}}
\pgflineto{\pgfpoint{348.722595pt}{191.015244pt}}
\pgfusepath{stroke}
\pgfpathmoveto{\pgfpoint{348.695862pt}{197.001221pt}}
\pgflineto{\pgfpoint{348.832123pt}{196.903625pt}}
\pgfusepath{stroke}
\pgfpathmoveto{\pgfpoint{348.746307pt}{196.841385pt}}
\pgflineto{\pgfpoint{348.695862pt}{197.001221pt}}
\pgfusepath{stroke}
\pgfpathmoveto{\pgfpoint{348.670593pt}{202.982071pt}}
\pgflineto{\pgfpoint{348.814514pt}{202.889587pt}}
\pgfusepath{stroke}
\pgfpathmoveto{\pgfpoint{348.730225pt}{202.821732pt}}
\pgflineto{\pgfpoint{348.670593pt}{202.982071pt}}
\pgfusepath{stroke}
\pgfpathmoveto{\pgfpoint{348.649414pt}{208.957886pt}}
\pgflineto{\pgfpoint{348.799072pt}{208.871735pt}}
\pgfusepath{stroke}
\pgfpathmoveto{\pgfpoint{348.717438pt}{208.799179pt}}
\pgflineto{\pgfpoint{348.649414pt}{208.957886pt}}
\pgfusepath{stroke}
\pgfpathmoveto{\pgfpoint{348.635468pt}{214.930023pt}}
\pgflineto{\pgfpoint{348.788025pt}{214.850632pt}}
\pgfusepath{stroke}
\pgfpathmoveto{\pgfpoint{348.709869pt}{214.774979pt}}
\pgflineto{\pgfpoint{348.635468pt}{214.930023pt}}
\pgfusepath{stroke}
\pgfpathmoveto{\pgfpoint{348.631744pt}{220.901367pt}}
\pgflineto{\pgfpoint{348.783722pt}{220.827866pt}}
\pgfusepath{stroke}
\pgfpathmoveto{\pgfpoint{348.709229pt}{220.751389pt}}
\pgflineto{\pgfpoint{348.631744pt}{220.901367pt}}
\pgfusepath{stroke}
\pgfpathmoveto{\pgfpoint{348.640533pt}{226.876389pt}}
\pgflineto{\pgfpoint{348.788147pt}{226.806152pt}}
\pgfusepath{stroke}
\pgfpathmoveto{\pgfpoint{348.716492pt}{226.731628pt}}
\pgflineto{\pgfpoint{348.640533pt}{226.876389pt}}
\pgfusepath{stroke}
\pgfpathmoveto{\pgfpoint{348.662842pt}{232.860535pt}}
\pgflineto{\pgfpoint{348.802643pt}{232.789017pt}}
\pgfusepath{stroke}
\pgfpathmoveto{\pgfpoint{348.731781pt}{232.719452pt}}
\pgflineto{\pgfpoint{348.662842pt}{232.860535pt}}
\pgfusepath{stroke}
\pgfpathmoveto{\pgfpoint{348.698975pt}{238.859619pt}}
\pgflineto{\pgfpoint{348.827972pt}{238.780319pt}}
\pgfusepath{stroke}
\pgfpathmoveto{\pgfpoint{348.754608pt}{238.718781pt}}
\pgflineto{\pgfpoint{348.698975pt}{238.859619pt}}
\pgfusepath{stroke}
\pgfpathmoveto{\pgfpoint{348.749573pt}{244.879761pt}}
\pgflineto{\pgfpoint{348.865295pt}{244.784042pt}}
\pgfusepath{stroke}
\pgfpathmoveto{\pgfpoint{348.784729pt}{244.733734pt}}
\pgflineto{\pgfpoint{348.749573pt}{244.879761pt}}
\pgfusepath{stroke}
\pgfpathmoveto{\pgfpoint{348.818054pt}{250.928040pt}}
\pgflineto{\pgfpoint{348.917694pt}{250.804535pt}}
\pgfusepath{stroke}
\pgfpathmoveto{\pgfpoint{348.823669pt}{250.769455pt}}
\pgflineto{\pgfpoint{348.818054pt}{250.928040pt}}
\pgfusepath{stroke}
\pgfpathmoveto{\pgfpoint{348.915344pt}{257.014282pt}}
\pgflineto{\pgfpoint{348.993500pt}{256.847137pt}}
\pgfusepath{stroke}
\pgfpathmoveto{\pgfpoint{348.877563pt}{256.833679pt}}
\pgflineto{\pgfpoint{348.915344pt}{257.014282pt}}
\pgfusepath{stroke}
\pgfpathmoveto{\pgfpoint{349.071136pt}{263.152985pt}}
\pgflineto{\pgfpoint{349.114136pt}{262.918213pt}}
\pgfusepath{stroke}
\pgfpathmoveto{\pgfpoint{348.964661pt}{262.939362pt}}
\pgflineto{\pgfpoint{349.071136pt}{263.152985pt}}
\pgfusepath{stroke}
\pgfpathmoveto{\pgfpoint{349.368866pt}{269.355377pt}}
\pgflineto{\pgfpoint{349.336456pt}{269.015961pt}}
\pgfusepath{stroke}
\pgfpathmoveto{\pgfpoint{349.139313pt}{269.103302pt}}
\pgflineto{\pgfpoint{349.368866pt}{269.355377pt}}
\pgfusepath{stroke}
\pgfpathmoveto{\pgfpoint{350.041931pt}{275.521912pt}}
\pgflineto{\pgfpoint{349.805023pt}{275.048126pt}}
\pgfusepath{stroke}
\pgfpathmoveto{\pgfpoint{349.568146pt}{275.285034pt}}
\pgflineto{\pgfpoint{350.041931pt}{275.521912pt}}
\pgfusepath{stroke}
\pgfpathmoveto{\pgfpoint{351.194458pt}{280.803833pt}}
\pgflineto{\pgfpoint{350.494965pt}{280.437286pt}}
\pgfusepath{stroke}
\pgfpathmoveto{\pgfpoint{350.414917pt}{280.930298pt}}
\pgflineto{\pgfpoint{351.194458pt}{280.803833pt}}
\pgfusepath{stroke}
\pgfpathmoveto{\pgfpoint{350.774719pt}{285.192505pt}}
\pgflineto{\pgfpoint{350.037476pt}{285.405609pt}}
\pgfusepath{stroke}
\pgfpathmoveto{\pgfpoint{350.312775pt}{285.805359pt}}
\pgflineto{\pgfpoint{350.774719pt}{285.192505pt}}
\pgfusepath{stroke}
\pgfpathmoveto{\pgfpoint{349.768646pt}{291.020660pt}}
\pgflineto{\pgfpoint{349.349030pt}{291.398743pt}}
\pgfusepath{stroke}
\pgfpathmoveto{\pgfpoint{349.659790pt}{291.574921pt}}
\pgflineto{\pgfpoint{349.768646pt}{291.020660pt}}
\pgfusepath{stroke}
\pgfpathmoveto{\pgfpoint{349.327881pt}{297.244141pt}}
\pgflineto{\pgfpoint{349.081360pt}{297.592621pt}}
\pgfusepath{stroke}
\pgfpathmoveto{\pgfpoint{349.339752pt}{297.670837pt}}
\pgflineto{\pgfpoint{349.327881pt}{297.244141pt}}
\pgfusepath{stroke}
\pgfpathmoveto{\pgfpoint{349.143188pt}{303.433960pt}}
\pgflineto{\pgfpoint{348.980682pt}{303.735596pt}}
\pgfusepath{stroke}
\pgfpathmoveto{\pgfpoint{349.194183pt}{303.772797pt}}
\pgflineto{\pgfpoint{349.143188pt}{303.433960pt}}
\pgfusepath{stroke}
\pgfpathmoveto{\pgfpoint{349.056335pt}{309.567810pt}}
\pgflineto{\pgfpoint{348.938995pt}{309.830414pt}}
\pgfusepath{stroke}
\pgfpathmoveto{\pgfpoint{349.120056pt}{309.848267pt}}
\pgflineto{\pgfpoint{349.056335pt}{309.567810pt}}
\pgfusepath{stroke}
\pgfpathmoveto{\pgfpoint{349.011353pt}{315.662231pt}}
\pgflineto{\pgfpoint{348.920868pt}{315.894287pt}}
\pgfusepath{stroke}
\pgfpathmoveto{\pgfpoint{349.078217pt}{315.902161pt}}
\pgflineto{\pgfpoint{349.011353pt}{315.662231pt}}
\pgfusepath{stroke}
\pgfpathmoveto{\pgfpoint{348.986511pt}{321.730591pt}}
\pgflineto{\pgfpoint{348.913269pt}{321.938538pt}}
\pgfusepath{stroke}
\pgfpathmoveto{\pgfpoint{349.052673pt}{321.940887pt}}
\pgflineto{\pgfpoint{348.986511pt}{321.730591pt}}
\pgfusepath{stroke}
\pgfpathmoveto{\pgfpoint{348.972198pt}{327.781250pt}}
\pgflineto{\pgfpoint{348.910706pt}{327.969818pt}}
\pgfusepath{stroke}
\pgfpathmoveto{\pgfpoint{349.036133pt}{327.968994pt}}
\pgflineto{\pgfpoint{348.972198pt}{327.781250pt}}
\pgfusepath{stroke}
\pgfpathmoveto{\pgfpoint{348.963776pt}{333.819550pt}}
\pgflineto{\pgfpoint{348.910706pt}{333.992188pt}}
\pgfusepath{stroke}
\pgfpathmoveto{\pgfpoint{349.024933pt}{333.989502pt}}
\pgflineto{\pgfpoint{348.963776pt}{333.819550pt}}
\pgfusepath{stroke}
\pgfpathmoveto{\pgfpoint{348.958862pt}{339.848846pt}}
\pgflineto{\pgfpoint{348.912048pt}{340.008209pt}}
\pgfusepath{stroke}
\pgfpathmoveto{\pgfpoint{349.017029pt}{340.004395pt}}
\pgflineto{\pgfpoint{348.958862pt}{339.848846pt}}
\pgfusepath{stroke}
\pgfpathmoveto{\pgfpoint{348.956055pt}{345.871429pt}}
\pgflineto{\pgfpoint{348.914062pt}{346.019470pt}}
\pgfusepath{stroke}
\pgfpathmoveto{\pgfpoint{349.011292pt}{346.015076pt}}
\pgflineto{\pgfpoint{348.956055pt}{345.871429pt}}
\pgfusepath{stroke}
\pgfpathmoveto{\pgfpoint{348.954559pt}{351.888855pt}}
\pgflineto{\pgfpoint{348.916351pt}{352.027191pt}}
\pgfusepath{stroke}
\pgfpathmoveto{\pgfpoint{349.007019pt}{352.022461pt}}
\pgflineto{\pgfpoint{348.954559pt}{351.888855pt}}
\pgfusepath{stroke}
\pgfpathmoveto{\pgfpoint{348.953918pt}{357.902252pt}}
\pgflineto{\pgfpoint{348.918793pt}{358.032104pt}}
\pgfusepath{stroke}
\pgfpathmoveto{\pgfpoint{349.003723pt}{358.027222pt}}
\pgflineto{\pgfpoint{348.953918pt}{357.902252pt}}
\pgfusepath{stroke}
\pgfpathmoveto{\pgfpoint{348.953827pt}{363.912415pt}}
\pgflineto{\pgfpoint{348.921204pt}{364.034821pt}}
\pgfusepath{stroke}
\pgfpathmoveto{\pgfpoint{349.001160pt}{364.029907pt}}
\pgflineto{\pgfpoint{348.953827pt}{363.912415pt}}
\pgfusepath{stroke}
\pgfpathmoveto{\pgfpoint{348.954071pt}{369.919922pt}}
\pgflineto{\pgfpoint{348.923584pt}{370.035706pt}}
\pgfusepath{stroke}
\pgfpathmoveto{\pgfpoint{348.999146pt}{370.030884pt}}
\pgflineto{\pgfpoint{348.954071pt}{369.919922pt}}
\pgfusepath{stroke}
\pgfpathmoveto{\pgfpoint{354.919495pt}{77.071472pt}}
\pgflineto{\pgfpoint{354.957703pt}{77.005905pt}}
\pgfusepath{stroke}
\pgfpathmoveto{\pgfpoint{354.910706pt}{76.996078pt}}
\pgflineto{\pgfpoint{354.919495pt}{77.071472pt}}
\pgfusepath{stroke}
\pgfpathmoveto{\pgfpoint{354.916168pt}{83.065659pt}}
\pgflineto{\pgfpoint{354.956207pt}{82.998306pt}}
\pgfusepath{stroke}
\pgfpathmoveto{\pgfpoint{354.907776pt}{82.987762pt}}
\pgflineto{\pgfpoint{354.916168pt}{83.065659pt}}
\pgfusepath{stroke}
\pgfpathmoveto{\pgfpoint{354.912476pt}{89.060211pt}}
\pgflineto{\pgfpoint{354.954529pt}{88.990997pt}}
\pgfusepath{stroke}
\pgfpathmoveto{\pgfpoint{354.904602pt}{88.979630pt}}
\pgflineto{\pgfpoint{354.912476pt}{89.060211pt}}
\pgfusepath{stroke}
\pgfpathmoveto{\pgfpoint{354.908386pt}{95.055115pt}}
\pgflineto{\pgfpoint{354.952576pt}{94.983971pt}}
\pgfusepath{stroke}
\pgfpathmoveto{\pgfpoint{354.901062pt}{94.971687pt}}
\pgflineto{\pgfpoint{354.908386pt}{95.055115pt}}
\pgfusepath{stroke}
\pgfpathmoveto{\pgfpoint{354.903839pt}{101.050392pt}}
\pgflineto{\pgfpoint{354.950378pt}{100.977226pt}}
\pgfusepath{stroke}
\pgfpathmoveto{\pgfpoint{354.897186pt}{100.963936pt}}
\pgflineto{\pgfpoint{354.903839pt}{101.050392pt}}
\pgfusepath{stroke}
\pgfpathmoveto{\pgfpoint{354.898743pt}{107.046043pt}}
\pgflineto{\pgfpoint{354.947845pt}{106.970833pt}}
\pgfusepath{stroke}
\pgfpathmoveto{\pgfpoint{354.892914pt}{106.956398pt}}
\pgflineto{\pgfpoint{354.898743pt}{107.046043pt}}
\pgfusepath{stroke}
\pgfpathmoveto{\pgfpoint{354.893005pt}{113.042084pt}}
\pgflineto{\pgfpoint{354.944946pt}{112.964745pt}}
\pgfusepath{stroke}
\pgfpathmoveto{\pgfpoint{354.888184pt}{112.949051pt}}
\pgflineto{\pgfpoint{354.893005pt}{113.042084pt}}
\pgfusepath{stroke}
\pgfpathmoveto{\pgfpoint{354.886597pt}{119.038536pt}}
\pgflineto{\pgfpoint{354.941650pt}{118.959007pt}}
\pgfusepath{stroke}
\pgfpathmoveto{\pgfpoint{354.882935pt}{118.941879pt}}
\pgflineto{\pgfpoint{354.886597pt}{119.038536pt}}
\pgfusepath{stroke}
\pgfpathmoveto{\pgfpoint{354.879333pt}{125.035362pt}}
\pgflineto{\pgfpoint{354.937805pt}{124.953621pt}}
\pgfusepath{stroke}
\pgfpathmoveto{\pgfpoint{354.877075pt}{124.934883pt}}
\pgflineto{\pgfpoint{354.879333pt}{125.035362pt}}
\pgfusepath{stroke}
\pgfpathmoveto{\pgfpoint{354.871155pt}{131.032562pt}}
\pgflineto{\pgfpoint{354.933411pt}{130.948578pt}}
\pgfusepath{stroke}
\pgfpathmoveto{\pgfpoint{354.870575pt}{130.928024pt}}
\pgflineto{\pgfpoint{354.871155pt}{131.032562pt}}
\pgfusepath{stroke}
\pgfpathmoveto{\pgfpoint{354.861877pt}{137.030060pt}}
\pgflineto{\pgfpoint{354.928314pt}{136.943863pt}}
\pgfusepath{stroke}
\pgfpathmoveto{\pgfpoint{354.863312pt}{136.921234pt}}
\pgflineto{\pgfpoint{354.861877pt}{137.030060pt}}
\pgfusepath{stroke}
\pgfpathmoveto{\pgfpoint{354.851410pt}{143.027802pt}}
\pgflineto{\pgfpoint{354.922424pt}{142.939453pt}}
\pgfusepath{stroke}
\pgfpathmoveto{\pgfpoint{354.855225pt}{142.914490pt}}
\pgflineto{\pgfpoint{354.851410pt}{143.027802pt}}
\pgfusepath{stroke}
\pgfpathmoveto{\pgfpoint{354.839539pt}{149.025681pt}}
\pgflineto{\pgfpoint{354.915649pt}{148.935272pt}}
\pgfusepath{stroke}
\pgfpathmoveto{\pgfpoint{354.846161pt}{148.907684pt}}
\pgflineto{\pgfpoint{354.839539pt}{149.025681pt}}
\pgfusepath{stroke}
\pgfpathmoveto{\pgfpoint{354.826111pt}{155.023514pt}}
\pgflineto{\pgfpoint{354.907837pt}{154.931229pt}}
\pgfusepath{stroke}
\pgfpathmoveto{\pgfpoint{354.836121pt}{154.900665pt}}
\pgflineto{\pgfpoint{354.826111pt}{155.023514pt}}
\pgfusepath{stroke}
\pgfpathmoveto{\pgfpoint{354.811005pt}{161.021057pt}}
\pgflineto{\pgfpoint{354.898834pt}{160.927185pt}}
\pgfusepath{stroke}
\pgfpathmoveto{\pgfpoint{354.824951pt}{160.893280pt}}
\pgflineto{\pgfpoint{354.811005pt}{161.021057pt}}
\pgfusepath{stroke}
\pgfpathmoveto{\pgfpoint{354.794067pt}{167.018021pt}}
\pgflineto{\pgfpoint{354.888580pt}{166.922974pt}}
\pgfusepath{stroke}
\pgfpathmoveto{\pgfpoint{354.812653pt}{166.885284pt}}
\pgflineto{\pgfpoint{354.794067pt}{167.018021pt}}
\pgfusepath{stroke}
\pgfpathmoveto{\pgfpoint{354.775299pt}{173.013977pt}}
\pgflineto{\pgfpoint{354.876953pt}{172.918274pt}}
\pgfusepath{stroke}
\pgfpathmoveto{\pgfpoint{354.799225pt}{172.876419pt}}
\pgflineto{\pgfpoint{354.775299pt}{173.013977pt}}
\pgfusepath{stroke}
\pgfpathmoveto{\pgfpoint{354.754761pt}{179.008423pt}}
\pgflineto{\pgfpoint{354.864014pt}{178.912766pt}}
\pgfusepath{stroke}
\pgfpathmoveto{\pgfpoint{354.784790pt}{178.866333pt}}
\pgflineto{\pgfpoint{354.754761pt}{179.008423pt}}
\pgfusepath{stroke}
\pgfpathmoveto{\pgfpoint{354.732758pt}{185.000778pt}}
\pgflineto{\pgfpoint{354.849854pt}{184.906036pt}}
\pgfusepath{stroke}
\pgfpathmoveto{\pgfpoint{354.769592pt}{184.854721pt}}
\pgflineto{\pgfpoint{354.732758pt}{185.000778pt}}
\pgfusepath{stroke}
\pgfpathmoveto{\pgfpoint{354.709900pt}{190.990463pt}}
\pgflineto{\pgfpoint{354.834839pt}{190.897614pt}}
\pgfusepath{stroke}
\pgfpathmoveto{\pgfpoint{354.754150pt}{190.841232pt}}
\pgflineto{\pgfpoint{354.709900pt}{190.990463pt}}
\pgfusepath{stroke}
\pgfpathmoveto{\pgfpoint{354.687164pt}{196.977051pt}}
\pgflineto{\pgfpoint{354.819550pt}{196.887131pt}}
\pgfusepath{stroke}
\pgfpathmoveto{\pgfpoint{354.739105pt}{196.825684pt}}
\pgflineto{\pgfpoint{354.687164pt}{196.977051pt}}
\pgfusepath{stroke}
\pgfpathmoveto{\pgfpoint{354.665955pt}{202.960464pt}}
\pgflineto{\pgfpoint{354.804932pt}{202.874344pt}}
\pgfusepath{stroke}
\pgfpathmoveto{\pgfpoint{354.725464pt}{202.808182pt}}
\pgflineto{\pgfpoint{354.665955pt}{202.960464pt}}
\pgfusepath{stroke}
\pgfpathmoveto{\pgfpoint{354.648041pt}{208.941238pt}}
\pgflineto{\pgfpoint{354.792206pt}{208.859421pt}}
\pgfusepath{stroke}
\pgfpathmoveto{\pgfpoint{354.714294pt}{208.789291pt}}
\pgflineto{\pgfpoint{354.648041pt}{208.941238pt}}
\pgfusepath{stroke}
\pgfpathmoveto{\pgfpoint{354.635315pt}{214.920776pt}}
\pgflineto{\pgfpoint{354.782806pt}{214.843124pt}}
\pgfusepath{stroke}
\pgfpathmoveto{\pgfpoint{354.706696pt}{214.770157pt}}
\pgflineto{\pgfpoint{354.635315pt}{214.920776pt}}
\pgfusepath{stroke}
\pgfpathmoveto{\pgfpoint{354.629425pt}{220.901596pt}}
\pgflineto{\pgfpoint{354.778137pt}{220.826904pt}}
\pgfusepath{stroke}
\pgfpathmoveto{\pgfpoint{354.703552pt}{220.752640pt}}
\pgflineto{\pgfpoint{354.629425pt}{220.901596pt}}
\pgfusepath{stroke}
\pgfpathmoveto{\pgfpoint{354.631531pt}{226.887268pt}}
\pgflineto{\pgfpoint{354.779297pt}{226.813034pt}}
\pgfusepath{stroke}
\pgfpathmoveto{\pgfpoint{354.705200pt}{226.739227pt}}
\pgflineto{\pgfpoint{354.631531pt}{226.887268pt}}
\pgfusepath{stroke}
\pgfpathmoveto{\pgfpoint{354.641846pt}{232.882278pt}}
\pgflineto{\pgfpoint{354.786987pt}{232.804489pt}}
\pgfusepath{stroke}
\pgfpathmoveto{\pgfpoint{354.711273pt}{232.732971pt}}
\pgflineto{\pgfpoint{354.641846pt}{232.882278pt}}
\pgfusepath{stroke}
\pgfpathmoveto{\pgfpoint{354.659943pt}{238.892090pt}}
\pgflineto{\pgfpoint{354.801483pt}{238.804932pt}}
\pgfusepath{stroke}
\pgfpathmoveto{\pgfpoint{354.720886pt}{238.737427pt}}
\pgflineto{\pgfpoint{354.659943pt}{238.892090pt}}
\pgfusepath{stroke}
\pgfpathmoveto{\pgfpoint{354.685059pt}{244.923615pt}}
\pgflineto{\pgfpoint{354.823120pt}{244.819077pt}}
\pgfusepath{stroke}
\pgfpathmoveto{\pgfpoint{354.732788pt}{244.757156pt}}
\pgflineto{\pgfpoint{354.685059pt}{244.923615pt}}
\pgfusepath{stroke}
\pgfpathmoveto{\pgfpoint{354.717194pt}{250.987030pt}}
\pgflineto{\pgfpoint{354.852905pt}{250.853699pt}}
\pgfusepath{stroke}
\pgfpathmoveto{\pgfpoint{354.745789pt}{250.798935pt}}
\pgflineto{\pgfpoint{354.717194pt}{250.987030pt}}
\pgfusepath{stroke}
\pgfpathmoveto{\pgfpoint{354.758545pt}{257.100555pt}}
\pgflineto{\pgfpoint{354.894470pt}{256.920715pt}}
\pgfusepath{stroke}
\pgfpathmoveto{\pgfpoint{354.759369pt}{256.875122pt}}
\pgflineto{\pgfpoint{354.758545pt}{257.100555pt}}
\pgfusepath{stroke}
\pgfpathmoveto{\pgfpoint{354.818481pt}{263.303284pt}}
\pgflineto{\pgfpoint{354.958313pt}{263.045135pt}}
\pgfusepath{stroke}
\pgfpathmoveto{\pgfpoint{354.775421pt}{263.012878pt}}
\pgflineto{\pgfpoint{354.818481pt}{263.303284pt}}
\pgfusepath{stroke}
\pgfpathmoveto{\pgfpoint{354.933929pt}{269.699463pt}}
\pgflineto{\pgfpoint{355.080658pt}{269.292328pt}}
\pgfusepath{stroke}
\pgfpathmoveto{\pgfpoint{354.807037pt}{269.285736pt}}
\pgflineto{\pgfpoint{354.933929pt}{269.699463pt}}
\pgfusepath{stroke}
\pgfpathmoveto{\pgfpoint{355.323853pt}{276.674469pt}}
\pgflineto{\pgfpoint{355.450287pt}{275.894928pt}}
\pgfusepath{stroke}
\pgfpathmoveto{\pgfpoint{354.957275pt}{275.974976pt}}
\pgflineto{\pgfpoint{355.323853pt}{276.674469pt}}
\pgfusepath{stroke}
\pgfpathmoveto{\pgfpoint{360.244873pt}{285.724884pt}}
\pgflineto{\pgfpoint{359.071259pt}{283.377655pt}}
\pgfusepath{stroke}
\pgfpathmoveto{\pgfpoint{357.897644pt}{284.551270pt}}
\pgflineto{\pgfpoint{360.244873pt}{285.724884pt}}
\pgfusepath{stroke}
\pgfpathmoveto{\pgfpoint{356.501068pt}{282.311035pt}}
\pgflineto{\pgfpoint{355.530762pt}{283.513672pt}}
\pgfusepath{stroke}
\pgfpathmoveto{\pgfpoint{356.446411pt}{283.855316pt}}
\pgflineto{\pgfpoint{356.501068pt}{282.311035pt}}
\pgfusepath{stroke}
\pgfpathmoveto{\pgfpoint{355.153351pt}{290.383453pt}}
\pgflineto{\pgfpoint{354.863953pt}{291.040955pt}}
\pgfusepath{stroke}
\pgfpathmoveto{\pgfpoint{355.316315pt}{291.083099pt}}
\pgflineto{\pgfpoint{355.153351pt}{290.383453pt}}
\pgfusepath{stroke}
\pgfpathmoveto{\pgfpoint{354.970581pt}{297.043915pt}}
\pgflineto{\pgfpoint{354.816833pt}{297.497498pt}}
\pgfusepath{stroke}
\pgfpathmoveto{\pgfpoint{355.119720pt}{297.499023pt}}
\pgflineto{\pgfpoint{354.970581pt}{297.043915pt}}
\pgfusepath{stroke}
\pgfpathmoveto{\pgfpoint{354.920837pt}{303.354004pt}}
\pgflineto{\pgfpoint{354.819458pt}{303.705688pt}}
\pgfusepath{stroke}
\pgfpathmoveto{\pgfpoint{355.050720pt}{303.696167pt}}
\pgflineto{\pgfpoint{354.920837pt}{303.354004pt}}
\pgfusepath{stroke}
\pgfpathmoveto{\pgfpoint{354.904480pt}{309.531189pt}}
\pgflineto{\pgfpoint{354.829620pt}{309.821533pt}}
\pgfusepath{stroke}
\pgfpathmoveto{\pgfpoint{355.018799pt}{309.808380pt}}
\pgflineto{\pgfpoint{354.904480pt}{309.531189pt}}
\pgfusepath{stroke}
\pgfpathmoveto{\pgfpoint{354.899750pt}{315.644287pt}}
\pgflineto{\pgfpoint{354.840393pt}{315.893372pt}}
\pgfusepath{stroke}
\pgfpathmoveto{\pgfpoint{355.001709pt}{315.879150pt}}
\pgflineto{\pgfpoint{354.899750pt}{315.644287pt}}
\pgfusepath{stroke}
\pgfpathmoveto{\pgfpoint{354.899719pt}{321.721619pt}}
\pgflineto{\pgfpoint{354.850342pt}{321.940857pt}}
\pgfusepath{stroke}
\pgfpathmoveto{\pgfpoint{354.991760pt}{321.926636pt}}
\pgflineto{\pgfpoint{354.899719pt}{321.721619pt}}
\pgfusepath{stroke}
\pgfpathmoveto{\pgfpoint{354.901794pt}{327.776978pt}}
\pgflineto{\pgfpoint{354.859222pt}{327.973450pt}}
\pgfusepath{stroke}
\pgfpathmoveto{\pgfpoint{354.985596pt}{327.959686pt}}
\pgflineto{\pgfpoint{354.901794pt}{327.776978pt}}
\pgfusepath{stroke}
\pgfpathmoveto{\pgfpoint{354.904724pt}{333.817841pt}}
\pgflineto{\pgfpoint{354.867096pt}{333.996277pt}}
\pgfusepath{stroke}
\pgfpathmoveto{\pgfpoint{354.981689pt}{333.983154pt}}
\pgflineto{\pgfpoint{354.904724pt}{333.817841pt}}
\pgfusepath{stroke}
\pgfpathmoveto{\pgfpoint{354.908020pt}{339.848602pt}}
\pgflineto{\pgfpoint{354.874054pt}{340.012329pt}}
\pgfusepath{stroke}
\pgfpathmoveto{\pgfpoint{354.979065pt}{339.999969pt}}
\pgflineto{\pgfpoint{354.908020pt}{339.848602pt}}
\pgfusepath{stroke}
\pgfpathmoveto{\pgfpoint{354.911346pt}{345.872009pt}}
\pgflineto{\pgfpoint{354.880249pt}{346.023499pt}}
\pgfusepath{stroke}
\pgfpathmoveto{\pgfpoint{354.977325pt}{346.011871pt}}
\pgflineto{\pgfpoint{354.911346pt}{345.872009pt}}
\pgfusepath{stroke}
\pgfpathmoveto{\pgfpoint{354.914581pt}{351.889954pt}}
\pgflineto{\pgfpoint{354.885742pt}{352.030975pt}}
\pgfusepath{stroke}
\pgfpathmoveto{\pgfpoint{354.976135pt}{352.020081pt}}
\pgflineto{\pgfpoint{354.914581pt}{351.889954pt}}
\pgfusepath{stroke}
\pgfpathmoveto{\pgfpoint{354.917633pt}{357.903564pt}}
\pgflineto{\pgfpoint{354.890656pt}{358.035645pt}}
\pgfusepath{stroke}
\pgfpathmoveto{\pgfpoint{354.975281pt}{358.025421pt}}
\pgflineto{\pgfpoint{354.917633pt}{357.903564pt}}
\pgfusepath{stroke}
\pgfpathmoveto{\pgfpoint{354.920502pt}{363.913849pt}}
\pgflineto{\pgfpoint{354.895081pt}{364.038086pt}}
\pgfusepath{stroke}
\pgfpathmoveto{\pgfpoint{354.974731pt}{364.028503pt}}
\pgflineto{\pgfpoint{354.920502pt}{363.913849pt}}
\pgfusepath{stroke}
\pgfpathmoveto{\pgfpoint{354.923187pt}{369.921387pt}}
\pgflineto{\pgfpoint{354.899048pt}{370.038757pt}}
\pgfusepath{stroke}
\pgfpathmoveto{\pgfpoint{354.974304pt}{370.029785pt}}
\pgflineto{\pgfpoint{354.923187pt}{369.921387pt}}
\pgfusepath{stroke}
\pgfpathmoveto{\pgfpoint{360.901123pt}{77.068176pt}}
\pgflineto{\pgfpoint{360.941010pt}{77.004364pt}}
\pgfusepath{stroke}
\pgfpathmoveto{\pgfpoint{360.894745pt}{76.993195pt}}
\pgflineto{\pgfpoint{360.901123pt}{77.068176pt}}
\pgfusepath{stroke}
\pgfpathmoveto{\pgfpoint{360.897491pt}{83.062042pt}}
\pgflineto{\pgfpoint{360.939270pt}{82.996582pt}}
\pgfusepath{stroke}
\pgfpathmoveto{\pgfpoint{360.891663pt}{82.984604pt}}
\pgflineto{\pgfpoint{360.897491pt}{83.062042pt}}
\pgfusepath{stroke}
\pgfpathmoveto{\pgfpoint{360.893494pt}{89.056183pt}}
\pgflineto{\pgfpoint{360.937317pt}{88.989052pt}}
\pgfusepath{stroke}
\pgfpathmoveto{\pgfpoint{360.888275pt}{88.976181pt}}
\pgflineto{\pgfpoint{360.893494pt}{89.056183pt}}
\pgfusepath{stroke}
\pgfpathmoveto{\pgfpoint{360.889069pt}{95.050621pt}}
\pgflineto{\pgfpoint{360.935120pt}{94.981758pt}}
\pgfusepath{stroke}
\pgfpathmoveto{\pgfpoint{360.884583pt}{94.967903pt}}
\pgflineto{\pgfpoint{360.889069pt}{95.050621pt}}
\pgfusepath{stroke}
\pgfpathmoveto{\pgfpoint{360.884155pt}{101.045387pt}}
\pgflineto{\pgfpoint{360.932617pt}{100.974731pt}}
\pgfusepath{stroke}
\pgfpathmoveto{\pgfpoint{360.880554pt}{100.959778pt}}
\pgflineto{\pgfpoint{360.884155pt}{101.045387pt}}
\pgfusepath{stroke}
\pgfpathmoveto{\pgfpoint{360.878723pt}{107.040459pt}}
\pgflineto{\pgfpoint{360.929810pt}{106.967972pt}}
\pgfusepath{stroke}
\pgfpathmoveto{\pgfpoint{360.876099pt}{106.951813pt}}
\pgflineto{\pgfpoint{360.878723pt}{107.040459pt}}
\pgfusepath{stroke}
\pgfpathmoveto{\pgfpoint{360.872650pt}{113.035851pt}}
\pgflineto{\pgfpoint{360.926575pt}{112.961487pt}}
\pgfusepath{stroke}
\pgfpathmoveto{\pgfpoint{360.871216pt}{112.943977pt}}
\pgflineto{\pgfpoint{360.872650pt}{113.035851pt}}
\pgfusepath{stroke}
\pgfpathmoveto{\pgfpoint{360.865845pt}{119.031532pt}}
\pgflineto{\pgfpoint{360.922974pt}{118.955284pt}}
\pgfusepath{stroke}
\pgfpathmoveto{\pgfpoint{360.865784pt}{118.936272pt}}
\pgflineto{\pgfpoint{360.865845pt}{119.031532pt}}
\pgfusepath{stroke}
\pgfpathmoveto{\pgfpoint{360.858276pt}{125.027504pt}}
\pgflineto{\pgfpoint{360.918823pt}{124.949356pt}}
\pgfusepath{stroke}
\pgfpathmoveto{\pgfpoint{360.859833pt}{124.928665pt}}
\pgflineto{\pgfpoint{360.858276pt}{125.027504pt}}
\pgfusepath{stroke}
\pgfpathmoveto{\pgfpoint{360.849823pt}{131.023712pt}}
\pgflineto{\pgfpoint{360.914124pt}{130.943695pt}}
\pgfusepath{stroke}
\pgfpathmoveto{\pgfpoint{360.853241pt}{130.921112pt}}
\pgflineto{\pgfpoint{360.849823pt}{131.023712pt}}
\pgfusepath{stroke}
\pgfpathmoveto{\pgfpoint{360.840332pt}{137.020111pt}}
\pgflineto{\pgfpoint{360.908752pt}{136.938263pt}}
\pgfusepath{stroke}
\pgfpathmoveto{\pgfpoint{360.845978pt}{136.913574pt}}
\pgflineto{\pgfpoint{360.840332pt}{137.020111pt}}
\pgfusepath{stroke}
\pgfpathmoveto{\pgfpoint{360.829742pt}{143.016617pt}}
\pgflineto{\pgfpoint{360.902679pt}{142.933029pt}}
\pgfusepath{stroke}
\pgfpathmoveto{\pgfpoint{360.837921pt}{142.905991pt}}
\pgflineto{\pgfpoint{360.829742pt}{143.016617pt}}
\pgfusepath{stroke}
\pgfpathmoveto{\pgfpoint{360.817902pt}{149.013107pt}}
\pgflineto{\pgfpoint{360.895752pt}{148.927917pt}}
\pgfusepath{stroke}
\pgfpathmoveto{\pgfpoint{360.829071pt}{148.898254pt}}
\pgflineto{\pgfpoint{360.817902pt}{149.013107pt}}
\pgfusepath{stroke}
\pgfpathmoveto{\pgfpoint{360.804688pt}{155.009430pt}}
\pgflineto{\pgfpoint{360.887909pt}{154.922852pt}}
\pgfusepath{stroke}
\pgfpathmoveto{\pgfpoint{360.819305pt}{154.890259pt}}
\pgflineto{\pgfpoint{360.804688pt}{155.009430pt}}
\pgfusepath{stroke}
\pgfpathmoveto{\pgfpoint{360.790039pt}{161.005371pt}}
\pgflineto{\pgfpoint{360.879028pt}{160.917694pt}}
\pgfusepath{stroke}
\pgfpathmoveto{\pgfpoint{360.808655pt}{160.881836pt}}
\pgflineto{\pgfpoint{360.790039pt}{161.005371pt}}
\pgfusepath{stroke}
\pgfpathmoveto{\pgfpoint{360.773926pt}{167.000671pt}}
\pgflineto{\pgfpoint{360.869141pt}{166.912277pt}}
\pgfusepath{stroke}
\pgfpathmoveto{\pgfpoint{360.797058pt}{166.872833pt}}
\pgflineto{\pgfpoint{360.773926pt}{167.000671pt}}
\pgfusepath{stroke}
\pgfpathmoveto{\pgfpoint{360.756348pt}{172.995010pt}}
\pgflineto{\pgfpoint{360.858154pt}{172.906372pt}}
\pgfusepath{stroke}
\pgfpathmoveto{\pgfpoint{360.784607pt}{172.863022pt}}
\pgflineto{\pgfpoint{360.756348pt}{172.995010pt}}
\pgfusepath{stroke}
\pgfpathmoveto{\pgfpoint{360.737427pt}{178.988037pt}}
\pgflineto{\pgfpoint{360.846130pt}{178.899750pt}}
\pgfusepath{stroke}
\pgfpathmoveto{\pgfpoint{360.771393pt}{178.852203pt}}
\pgflineto{\pgfpoint{360.737427pt}{178.988037pt}}
\pgfusepath{stroke}
\pgfpathmoveto{\pgfpoint{360.717529pt}{184.979416pt}}
\pgflineto{\pgfpoint{360.833252pt}{184.892136pt}}
\pgfusepath{stroke}
\pgfpathmoveto{\pgfpoint{360.757721pt}{184.840149pt}}
\pgflineto{\pgfpoint{360.717529pt}{184.979416pt}}
\pgfusepath{stroke}
\pgfpathmoveto{\pgfpoint{360.697144pt}{190.968842pt}}
\pgflineto{\pgfpoint{360.819824pt}{190.883255pt}}
\pgfusepath{stroke}
\pgfpathmoveto{\pgfpoint{360.743958pt}{190.826752pt}}
\pgflineto{\pgfpoint{360.697144pt}{190.968842pt}}
\pgfusepath{stroke}
\pgfpathmoveto{\pgfpoint{360.676971pt}{196.956192pt}}
\pgflineto{\pgfpoint{360.806366pt}{196.872986pt}}
\pgfusepath{stroke}
\pgfpathmoveto{\pgfpoint{360.730560pt}{196.811996pt}}
\pgflineto{\pgfpoint{360.676971pt}{196.956192pt}}
\pgfusepath{stroke}
\pgfpathmoveto{\pgfpoint{360.658051pt}{202.941681pt}}
\pgflineto{\pgfpoint{360.793488pt}{202.861328pt}}
\pgfusepath{stroke}
\pgfpathmoveto{\pgfpoint{360.718201pt}{202.796143pt}}
\pgflineto{\pgfpoint{360.658051pt}{202.941681pt}}
\pgfusepath{stroke}
\pgfpathmoveto{\pgfpoint{360.641418pt}{208.925980pt}}
\pgflineto{\pgfpoint{360.782013pt}{208.848633pt}}
\pgfusepath{stroke}
\pgfpathmoveto{\pgfpoint{360.707489pt}{208.779739pt}}
\pgflineto{\pgfpoint{360.641418pt}{208.925980pt}}
\pgfusepath{stroke}
\pgfpathmoveto{\pgfpoint{360.628174pt}{214.910446pt}}
\pgflineto{\pgfpoint{360.772797pt}{214.835663pt}}
\pgfusepath{stroke}
\pgfpathmoveto{\pgfpoint{360.699036pt}{214.763824pt}}
\pgflineto{\pgfpoint{360.628174pt}{214.910446pt}}
\pgfusepath{stroke}
\pgfpathmoveto{\pgfpoint{360.619019pt}{220.897141pt}}
\pgflineto{\pgfpoint{360.766602pt}{220.823730pt}}
\pgfusepath{stroke}
\pgfpathmoveto{\pgfpoint{360.693024pt}{220.749863pt}}
\pgflineto{\pgfpoint{360.619019pt}{220.897141pt}}
\pgfusepath{stroke}
\pgfpathmoveto{\pgfpoint{360.614166pt}{226.888916pt}}
\pgflineto{\pgfpoint{360.763794pt}{226.814713pt}}
\pgfusepath{stroke}
\pgfpathmoveto{\pgfpoint{360.689362pt}{226.739761pt}}
\pgflineto{\pgfpoint{360.614166pt}{226.888916pt}}
\pgfusepath{stroke}
\pgfpathmoveto{\pgfpoint{360.613037pt}{232.889435pt}}
\pgflineto{\pgfpoint{360.764465pt}{232.811096pt}}
\pgfusepath{stroke}
\pgfpathmoveto{\pgfpoint{360.687195pt}{232.735901pt}}
\pgflineto{\pgfpoint{360.613037pt}{232.889435pt}}
\pgfusepath{stroke}
\pgfpathmoveto{\pgfpoint{360.613922pt}{238.903305pt}}
\pgflineto{\pgfpoint{360.767944pt}{238.816162pt}}
\pgfusepath{stroke}
\pgfpathmoveto{\pgfpoint{360.684875pt}{238.741165pt}}
\pgflineto{\pgfpoint{360.613922pt}{238.903305pt}}
\pgfusepath{stroke}
\pgfpathmoveto{\pgfpoint{360.613953pt}{244.936691pt}}
\pgflineto{\pgfpoint{360.773071pt}{244.834335pt}}
\pgfusepath{stroke}
\pgfpathmoveto{\pgfpoint{360.679840pt}{244.759338pt}}
\pgflineto{\pgfpoint{360.613953pt}{244.936691pt}}
\pgfusepath{stroke}
\pgfpathmoveto{\pgfpoint{360.608002pt}{250.998840pt}}
\pgflineto{\pgfpoint{360.777374pt}{250.872345pt}}
\pgfusepath{stroke}
\pgfpathmoveto{\pgfpoint{360.667603pt}{250.796036pt}}
\pgflineto{\pgfpoint{360.608002pt}{250.998840pt}}
\pgfusepath{stroke}
\pgfpathmoveto{\pgfpoint{360.586212pt}{257.105865pt}}
\pgflineto{\pgfpoint{360.776062pt}{256.942017pt}}
\pgfusepath{stroke}
\pgfpathmoveto{\pgfpoint{360.639832pt}{256.860870pt}}
\pgflineto{\pgfpoint{360.586212pt}{257.105865pt}}
\pgfusepath{stroke}
\pgfpathmoveto{\pgfpoint{360.524872pt}{263.289764pt}}
\pgflineto{\pgfpoint{360.756958pt}{263.067383pt}}
\pgfusepath{stroke}
\pgfpathmoveto{\pgfpoint{360.577118pt}{262.972595pt}}
\pgflineto{\pgfpoint{360.524872pt}{263.289764pt}}
\pgfusepath{stroke}
\pgfpathmoveto{\pgfpoint{360.348816pt}{269.622803pt}}
\pgflineto{\pgfpoint{360.677979pt}{269.304871pt}}
\pgfusepath{stroke}
\pgfpathmoveto{\pgfpoint{360.421387pt}{269.170959pt}}
\pgflineto{\pgfpoint{360.348816pt}{269.622803pt}}
\pgfusepath{stroke}
\pgfpathmoveto{\pgfpoint{359.712524pt}{276.254730pt}}
\pgflineto{\pgfpoint{360.325348pt}{275.792786pt}}
\pgfusepath{stroke}
\pgfpathmoveto{\pgfpoint{359.925598pt}{275.517487pt}}
\pgflineto{\pgfpoint{359.712524pt}{276.254730pt}}
\pgfusepath{stroke}
\pgfpathmoveto{\pgfpoint{356.831055pt}{281.981079pt}}
\pgflineto{\pgfpoint{358.375305pt}{281.926422pt}}
\pgfusepath{stroke}
\pgfpathmoveto{\pgfpoint{358.033691pt}{281.010803pt}}
\pgflineto{\pgfpoint{356.831055pt}{281.981079pt}}
\pgfusepath{stroke}
\pgfpathmoveto{\pgfpoint{358.458435pt}{283.938446pt}}
\pgflineto{\pgfpoint{359.012421pt}{285.046417pt}}
\pgfusepath{stroke}
\pgfpathmoveto{\pgfpoint{359.566406pt}{284.492432pt}}
\pgflineto{\pgfpoint{358.458435pt}{283.938446pt}}
\pgfusepath{stroke}
\pgfpathmoveto{\pgfpoint{360.132080pt}{290.618317pt}}
\pgflineto{\pgfpoint{360.205109pt}{291.309631pt}}
\pgfusepath{stroke}
\pgfpathmoveto{\pgfpoint{360.605286pt}{291.127563pt}}
\pgflineto{\pgfpoint{360.132080pt}{290.618317pt}}
\pgfusepath{stroke}
\pgfpathmoveto{\pgfpoint{360.528107pt}{297.117920pt}}
\pgflineto{\pgfpoint{360.526001pt}{297.594604pt}}
\pgfusepath{stroke}
\pgfpathmoveto{\pgfpoint{360.812439pt}{297.500549pt}}
\pgflineto{\pgfpoint{360.528107pt}{297.117920pt}}
\pgfusepath{stroke}
\pgfpathmoveto{\pgfpoint{360.672424pt}{303.389893pt}}
\pgflineto{\pgfpoint{360.653809pt}{303.755829pt}}
\pgfusepath{stroke}
\pgfpathmoveto{\pgfpoint{360.877075pt}{303.693817pt}}
\pgflineto{\pgfpoint{360.672424pt}{303.389893pt}}
\pgfusepath{stroke}
\pgfpathmoveto{\pgfpoint{360.742798pt}{309.553345pt}}
\pgflineto{\pgfpoint{360.720215pt}{309.852905pt}}
\pgfusepath{stroke}
\pgfpathmoveto{\pgfpoint{360.904449pt}{309.806519pt}}
\pgflineto{\pgfpoint{360.742798pt}{309.553345pt}}
\pgfusepath{stroke}
\pgfpathmoveto{\pgfpoint{360.783936pt}{315.659973pt}}
\pgflineto{\pgfpoint{360.760864pt}{315.915344pt}}
\pgfusepath{stroke}
\pgfpathmoveto{\pgfpoint{360.918671pt}{315.878113pt}}
\pgflineto{\pgfpoint{360.783936pt}{315.659973pt}}
\pgfusepath{stroke}
\pgfpathmoveto{\pgfpoint{360.811127pt}{321.733704pt}}
\pgflineto{\pgfpoint{360.788544pt}{321.957397pt}}
\pgfusepath{stroke}
\pgfpathmoveto{\pgfpoint{360.927246pt}{321.926208pt}}
\pgflineto{\pgfpoint{360.811127pt}{321.733704pt}}
\pgfusepath{stroke}
\pgfpathmoveto{\pgfpoint{360.830627pt}{327.786774pt}}
\pgflineto{\pgfpoint{360.808777pt}{327.986511pt}}
\pgfusepath{stroke}
\pgfpathmoveto{\pgfpoint{360.932983pt}{327.959656pt}}
\pgflineto{\pgfpoint{360.830627pt}{327.786774pt}}
\pgfusepath{stroke}
\pgfpathmoveto{\pgfpoint{360.845428pt}{333.826019pt}}
\pgflineto{\pgfpoint{360.824371pt}{334.006958pt}}
\pgfusepath{stroke}
\pgfpathmoveto{\pgfpoint{360.937134pt}{333.983398pt}}
\pgflineto{\pgfpoint{360.845428pt}{333.826019pt}}
\pgfusepath{stroke}
\pgfpathmoveto{\pgfpoint{360.857147pt}{339.855560pt}}
\pgflineto{\pgfpoint{360.836823pt}{340.021240pt}}
\pgfusepath{stroke}
\pgfpathmoveto{\pgfpoint{360.940308pt}{340.000305pt}}
\pgflineto{\pgfpoint{360.857147pt}{339.855560pt}}
\pgfusepath{stroke}
\pgfpathmoveto{\pgfpoint{360.866730pt}{345.878052pt}}
\pgflineto{\pgfpoint{360.847076pt}{346.031097pt}}
\pgfusepath{stroke}
\pgfpathmoveto{\pgfpoint{360.942841pt}{346.012268pt}}
\pgflineto{\pgfpoint{360.866730pt}{345.878052pt}}
\pgfusepath{stroke}
\pgfpathmoveto{\pgfpoint{360.874725pt}{351.895203pt}}
\pgflineto{\pgfpoint{360.855682pt}{352.037537pt}}
\pgfusepath{stroke}
\pgfpathmoveto{\pgfpoint{360.944916pt}{352.020508pt}}
\pgflineto{\pgfpoint{360.874725pt}{351.895203pt}}
\pgfusepath{stroke}
\pgfpathmoveto{\pgfpoint{360.881531pt}{357.908203pt}}
\pgflineto{\pgfpoint{360.863037pt}{358.041382pt}}
\pgfusepath{stroke}
\pgfpathmoveto{\pgfpoint{360.946655pt}{358.025879pt}}
\pgflineto{\pgfpoint{360.881531pt}{357.908203pt}}
\pgfusepath{stroke}
\pgfpathmoveto{\pgfpoint{360.887390pt}{363.917969pt}}
\pgflineto{\pgfpoint{360.869385pt}{364.043152pt}}
\pgfusepath{stroke}
\pgfpathmoveto{\pgfpoint{360.948090pt}{364.028931pt}}
\pgflineto{\pgfpoint{360.887390pt}{363.917969pt}}
\pgfusepath{stroke}
\pgfpathmoveto{\pgfpoint{360.892487pt}{369.925110pt}}
\pgflineto{\pgfpoint{360.874908pt}{370.043274pt}}
\pgfusepath{stroke}
\pgfpathmoveto{\pgfpoint{360.949341pt}{370.030182pt}}
\pgflineto{\pgfpoint{360.892487pt}{369.925110pt}}
\pgfusepath{stroke}
\pgfpathmoveto{\pgfpoint{366.883087pt}{77.064575pt}}
\pgflineto{\pgfpoint{366.924500pt}{77.002609pt}}
\pgfusepath{stroke}
\pgfpathmoveto{\pgfpoint{366.879059pt}{76.990158pt}}
\pgflineto{\pgfpoint{366.883087pt}{77.064575pt}}
\pgfusepath{stroke}
\pgfpathmoveto{\pgfpoint{366.879211pt}{83.058075pt}}
\pgflineto{\pgfpoint{366.922546pt}{82.994629pt}}
\pgfusepath{stroke}
\pgfpathmoveto{\pgfpoint{366.875824pt}{82.981308pt}}
\pgflineto{\pgfpoint{366.879211pt}{83.058075pt}}
\pgfusepath{stroke}
\pgfpathmoveto{\pgfpoint{366.874939pt}{89.051826pt}}
\pgflineto{\pgfpoint{366.920380pt}{88.986847pt}}
\pgfusepath{stroke}
\pgfpathmoveto{\pgfpoint{366.872314pt}{88.972572pt}}
\pgflineto{\pgfpoint{366.874939pt}{89.051826pt}}
\pgfusepath{stroke}
\pgfpathmoveto{\pgfpoint{366.870209pt}{95.045822pt}}
\pgflineto{\pgfpoint{366.917938pt}{94.979294pt}}
\pgfusepath{stroke}
\pgfpathmoveto{\pgfpoint{366.868469pt}{94.963974pt}}
\pgflineto{\pgfpoint{366.870209pt}{95.045822pt}}
\pgfusepath{stroke}
\pgfpathmoveto{\pgfpoint{366.865021pt}{101.040070pt}}
\pgflineto{\pgfpoint{366.915192pt}{100.971962pt}}
\pgfusepath{stroke}
\pgfpathmoveto{\pgfpoint{366.864288pt}{100.955482pt}}
\pgflineto{\pgfpoint{366.865021pt}{101.040070pt}}
\pgfusepath{stroke}
\pgfpathmoveto{\pgfpoint{366.859314pt}{107.034561pt}}
\pgflineto{\pgfpoint{366.912109pt}{106.964844pt}}
\pgfusepath{stroke}
\pgfpathmoveto{\pgfpoint{366.859741pt}{106.947090pt}}
\pgflineto{\pgfpoint{366.859314pt}{107.034561pt}}
\pgfusepath{stroke}
\pgfpathmoveto{\pgfpoint{366.852966pt}{113.029297pt}}
\pgflineto{\pgfpoint{366.908661pt}{112.957947pt}}
\pgfusepath{stroke}
\pgfpathmoveto{\pgfpoint{366.854706pt}{112.938789pt}}
\pgflineto{\pgfpoint{366.852966pt}{113.029297pt}}
\pgfusepath{stroke}
\pgfpathmoveto{\pgfpoint{366.845917pt}{119.024254pt}}
\pgflineto{\pgfpoint{366.904785pt}{118.951294pt}}
\pgfusepath{stroke}
\pgfpathmoveto{\pgfpoint{366.849243pt}{118.930565pt}}
\pgflineto{\pgfpoint{366.845917pt}{119.024254pt}}
\pgfusepath{stroke}
\pgfpathmoveto{\pgfpoint{366.838135pt}{125.019394pt}}
\pgflineto{\pgfpoint{366.900391pt}{124.944832pt}}
\pgfusepath{stroke}
\pgfpathmoveto{\pgfpoint{366.843231pt}{124.922379pt}}
\pgflineto{\pgfpoint{366.838135pt}{125.019394pt}}
\pgfusepath{stroke}
\pgfpathmoveto{\pgfpoint{366.829498pt}{131.014694pt}}
\pgflineto{\pgfpoint{366.895508pt}{130.938568pt}}
\pgfusepath{stroke}
\pgfpathmoveto{\pgfpoint{366.836609pt}{130.914215pt}}
\pgflineto{\pgfpoint{366.829498pt}{131.014694pt}}
\pgfusepath{stroke}
\pgfpathmoveto{\pgfpoint{366.819946pt}{137.010071pt}}
\pgflineto{\pgfpoint{366.889954pt}{136.932465pt}}
\pgfusepath{stroke}
\pgfpathmoveto{\pgfpoint{366.829407pt}{136.905975pt}}
\pgflineto{\pgfpoint{366.819946pt}{137.010071pt}}
\pgfusepath{stroke}
\pgfpathmoveto{\pgfpoint{366.809326pt}{143.005432pt}}
\pgflineto{\pgfpoint{366.883728pt}{142.926483pt}}
\pgfusepath{stroke}
\pgfpathmoveto{\pgfpoint{366.821472pt}{142.897644pt}}
\pgflineto{\pgfpoint{366.809326pt}{143.005432pt}}
\pgfusepath{stroke}
\pgfpathmoveto{\pgfpoint{366.797638pt}{149.000702pt}}
\pgflineto{\pgfpoint{366.876770pt}{148.920532pt}}
\pgfusepath{stroke}
\pgfpathmoveto{\pgfpoint{366.812836pt}{148.889099pt}}
\pgflineto{\pgfpoint{366.797638pt}{149.000702pt}}
\pgfusepath{stroke}
\pgfpathmoveto{\pgfpoint{366.784729pt}{154.995712pt}}
\pgflineto{\pgfpoint{366.868988pt}{154.914566pt}}
\pgfusepath{stroke}
\pgfpathmoveto{\pgfpoint{366.803436pt}{154.880249pt}}
\pgflineto{\pgfpoint{366.784729pt}{154.995712pt}}
\pgfusepath{stroke}
\pgfpathmoveto{\pgfpoint{366.770599pt}{160.990295pt}}
\pgflineto{\pgfpoint{366.860321pt}{160.908447pt}}
\pgfusepath{stroke}
\pgfpathmoveto{\pgfpoint{366.793274pt}{160.871002pt}}
\pgflineto{\pgfpoint{366.770599pt}{160.990295pt}}
\pgfusepath{stroke}
\pgfpathmoveto{\pgfpoint{366.755249pt}{166.984253pt}}
\pgflineto{\pgfpoint{366.850769pt}{166.902039pt}}
\pgfusepath{stroke}
\pgfpathmoveto{\pgfpoint{366.782349pt}{166.861176pt}}
\pgflineto{\pgfpoint{366.755249pt}{166.984253pt}}
\pgfusepath{stroke}
\pgfpathmoveto{\pgfpoint{366.738708pt}{172.977356pt}}
\pgflineto{\pgfpoint{366.840332pt}{172.895203pt}}
\pgfusepath{stroke}
\pgfpathmoveto{\pgfpoint{366.770721pt}{172.850647pt}}
\pgflineto{\pgfpoint{366.738708pt}{172.977356pt}}
\pgfusepath{stroke}
\pgfpathmoveto{\pgfpoint{366.721130pt}{178.969345pt}}
\pgflineto{\pgfpoint{366.829102pt}{178.887726pt}}
\pgfusepath{stroke}
\pgfpathmoveto{\pgfpoint{366.758545pt}{178.839279pt}}
\pgflineto{\pgfpoint{366.721130pt}{178.969345pt}}
\pgfusepath{stroke}
\pgfpathmoveto{\pgfpoint{366.702820pt}{184.960052pt}}
\pgflineto{\pgfpoint{366.817200pt}{184.879486pt}}
\pgfusepath{stroke}
\pgfpathmoveto{\pgfpoint{366.746002pt}{184.826965pt}}
\pgflineto{\pgfpoint{366.702820pt}{184.960052pt}}
\pgfusepath{stroke}
\pgfpathmoveto{\pgfpoint{366.684143pt}{190.949341pt}}
\pgflineto{\pgfpoint{366.804932pt}{190.870331pt}}
\pgfusepath{stroke}
\pgfpathmoveto{\pgfpoint{366.733398pt}{190.813660pt}}
\pgflineto{\pgfpoint{366.684143pt}{190.949341pt}}
\pgfusepath{stroke}
\pgfpathmoveto{\pgfpoint{366.665619pt}{196.937271pt}}
\pgflineto{\pgfpoint{366.792603pt}{196.860275pt}}
\pgfusepath{stroke}
\pgfpathmoveto{\pgfpoint{366.721008pt}{196.799484pt}}
\pgflineto{\pgfpoint{366.665619pt}{196.937271pt}}
\pgfusepath{stroke}
\pgfpathmoveto{\pgfpoint{366.647827pt}{202.924164pt}}
\pgflineto{\pgfpoint{366.780640pt}{202.849426pt}}
\pgfusepath{stroke}
\pgfpathmoveto{\pgfpoint{366.709229pt}{202.784683pt}}
\pgflineto{\pgfpoint{366.647827pt}{202.924164pt}}
\pgfusepath{stroke}
\pgfpathmoveto{\pgfpoint{366.631409pt}{208.910706pt}}
\pgflineto{\pgfpoint{366.769562pt}{208.838196pt}}
\pgfusepath{stroke}
\pgfpathmoveto{\pgfpoint{366.698425pt}{208.769806pt}}
\pgflineto{\pgfpoint{366.631409pt}{208.910706pt}}
\pgfusepath{stroke}
\pgfpathmoveto{\pgfpoint{366.616821pt}{214.898010pt}}
\pgflineto{\pgfpoint{366.759766pt}{214.827271pt}}
\pgfusepath{stroke}
\pgfpathmoveto{\pgfpoint{366.688751pt}{214.755646pt}}
\pgflineto{\pgfpoint{366.616821pt}{214.898010pt}}
\pgfusepath{stroke}
\pgfpathmoveto{\pgfpoint{366.604126pt}{220.887711pt}}
\pgflineto{\pgfpoint{366.751526pt}{220.817734pt}}
\pgfusepath{stroke}
\pgfpathmoveto{\pgfpoint{366.680084pt}{220.743286pt}}
\pgflineto{\pgfpoint{366.604126pt}{220.887711pt}}
\pgfusepath{stroke}
\pgfpathmoveto{\pgfpoint{366.592957pt}{226.881973pt}}
\pgflineto{\pgfpoint{366.744812pt}{226.811081pt}}
\pgfusepath{stroke}
\pgfpathmoveto{\pgfpoint{366.671906pt}{226.734146pt}}
\pgflineto{\pgfpoint{366.592957pt}{226.881973pt}}
\pgfusepath{stroke}
\pgfpathmoveto{\pgfpoint{366.581909pt}{232.883484pt}}
\pgflineto{\pgfpoint{366.738983pt}{232.809235pt}}
\pgfusepath{stroke}
\pgfpathmoveto{\pgfpoint{366.663025pt}{232.729828pt}}
\pgflineto{\pgfpoint{366.581909pt}{232.883484pt}}
\pgfusepath{stroke}
\pgfpathmoveto{\pgfpoint{366.568481pt}{238.895493pt}}
\pgflineto{\pgfpoint{366.732727pt}{238.814636pt}}
\pgfusepath{stroke}
\pgfpathmoveto{\pgfpoint{366.651367pt}{238.732269pt}}
\pgflineto{\pgfpoint{366.568481pt}{238.895493pt}}
\pgfusepath{stroke}
\pgfpathmoveto{\pgfpoint{366.548126pt}{244.921997pt}}
\pgflineto{\pgfpoint{366.723450pt}{244.830505pt}}
\pgfusepath{stroke}
\pgfpathmoveto{\pgfpoint{366.633484pt}{244.743607pt}}
\pgflineto{\pgfpoint{366.548126pt}{244.921997pt}}
\pgfusepath{stroke}
\pgfpathmoveto{\pgfpoint{366.512878pt}{250.968109pt}}
\pgflineto{\pgfpoint{366.706451pt}{250.861069pt}}
\pgfusepath{stroke}
\pgfpathmoveto{\pgfpoint{366.603516pt}{250.766342pt}}
\pgflineto{\pgfpoint{366.512878pt}{250.968109pt}}
\pgfusepath{stroke}
\pgfpathmoveto{\pgfpoint{366.447449pt}{257.039978pt}}
\pgflineto{\pgfpoint{366.672180pt}{256.912170pt}}
\pgfusepath{stroke}
\pgfpathmoveto{\pgfpoint{366.550537pt}{256.802917pt}}
\pgflineto{\pgfpoint{366.447449pt}{257.039978pt}}
\pgfusepath{stroke}
\pgfpathmoveto{\pgfpoint{366.319885pt}{263.142090pt}}
\pgflineto{\pgfpoint{366.599823pt}{262.990356pt}}
\pgfusepath{stroke}
\pgfpathmoveto{\pgfpoint{366.452789pt}{262.852722pt}}
\pgflineto{\pgfpoint{366.319885pt}{263.142090pt}}
\pgfusepath{stroke}
\pgfpathmoveto{\pgfpoint{366.059540pt}{269.257996pt}}
\pgflineto{\pgfpoint{366.440521pt}{269.092438pt}}
\pgfusepath{stroke}
\pgfpathmoveto{\pgfpoint{366.264984pt}{268.896973pt}}
\pgflineto{\pgfpoint{366.059540pt}{269.257996pt}}
\pgfusepath{stroke}
\pgfpathmoveto{\pgfpoint{365.540649pt}{275.248657pt}}
\pgflineto{\pgfpoint{366.094910pt}{275.139801pt}}
\pgfusepath{stroke}
\pgfpathmoveto{\pgfpoint{365.918762pt}{274.829010pt}}
\pgflineto{\pgfpoint{365.540649pt}{275.248657pt}}
\pgfusepath{stroke}
\pgfpathmoveto{\pgfpoint{364.903442pt}{280.633362pt}}
\pgflineto{\pgfpoint{365.603088pt}{280.796326pt}}
\pgfusepath{stroke}
\pgfpathmoveto{\pgfpoint{365.560944pt}{280.343933pt}}
\pgflineto{\pgfpoint{364.903442pt}{280.633362pt}}
\pgfusepath{stroke}
\pgfpathmoveto{\pgfpoint{365.138306pt}{285.612122pt}}
\pgflineto{\pgfpoint{365.647552pt}{286.085297pt}}
\pgfusepath{stroke}
\pgfpathmoveto{\pgfpoint{365.829620pt}{285.685120pt}}
\pgflineto{\pgfpoint{365.138306pt}{285.612122pt}}
\pgfusepath{stroke}
\pgfpathmoveto{\pgfpoint{365.834534pt}{291.314545pt}}
\pgflineto{\pgfpoint{366.080017pt}{291.805481pt}}
\pgfusepath{stroke}
\pgfpathmoveto{\pgfpoint{366.325500pt}{291.560028pt}}
\pgflineto{\pgfpoint{365.834534pt}{291.314545pt}}
\pgfusepath{stroke}
\pgfpathmoveto{\pgfpoint{366.262329pt}{297.397339pt}}
\pgflineto{\pgfpoint{366.375763pt}{297.809082pt}}
\pgfusepath{stroke}
\pgfpathmoveto{\pgfpoint{366.600128pt}{297.658661pt}}
\pgflineto{\pgfpoint{366.262329pt}{297.397339pt}}
\pgfusepath{stroke}
\pgfpathmoveto{\pgfpoint{366.484009pt}{303.524048pt}}
\pgflineto{\pgfpoint{366.539001pt}{303.864838pt}}
\pgfusepath{stroke}
\pgfpathmoveto{\pgfpoint{366.732483pt}{303.763702pt}}
\pgflineto{\pgfpoint{366.484009pt}{303.524048pt}}
\pgfusepath{stroke}
\pgfpathmoveto{\pgfpoint{366.606628pt}{309.628296pt}}
\pgflineto{\pgfpoint{366.633667pt}{309.916626pt}}
\pgfusepath{stroke}
\pgfpathmoveto{\pgfpoint{366.801270pt}{309.842712pt}}
\pgflineto{\pgfpoint{366.606628pt}{309.628296pt}}
\pgfusepath{stroke}
\pgfpathmoveto{\pgfpoint{366.680939pt}{315.706879pt}}
\pgflineto{\pgfpoint{366.693298pt}{315.956696pt}}
\pgfusepath{stroke}
\pgfpathmoveto{\pgfpoint{366.840729pt}{315.899292pt}}
\pgflineto{\pgfpoint{366.680939pt}{315.706879pt}}
\pgfusepath{stroke}
\pgfpathmoveto{\pgfpoint{366.729736pt}{321.765656pt}}
\pgflineto{\pgfpoint{366.733734pt}{321.986359pt}}
\pgfusepath{stroke}
\pgfpathmoveto{\pgfpoint{366.865356pt}{321.939819pt}}
\pgflineto{\pgfpoint{366.729736pt}{321.765656pt}}
\pgfusepath{stroke}
\pgfpathmoveto{\pgfpoint{366.763916pt}{327.809937pt}}
\pgflineto{\pgfpoint{366.762787pt}{328.007996pt}}
\pgfusepath{stroke}
\pgfpathmoveto{\pgfpoint{366.881836pt}{327.969055pt}}
\pgflineto{\pgfpoint{366.763916pt}{327.809937pt}}
\pgfusepath{stroke}
\pgfpathmoveto{\pgfpoint{366.789093pt}{333.843628pt}}
\pgflineto{\pgfpoint{366.784698pt}{334.023590pt}}
\pgfusepath{stroke}
\pgfpathmoveto{\pgfpoint{366.893555pt}{333.990234pt}}
\pgflineto{\pgfpoint{366.789093pt}{333.843628pt}}
\pgfusepath{stroke}
\pgfpathmoveto{\pgfpoint{366.808411pt}{339.869446pt}}
\pgflineto{\pgfpoint{366.801788pt}{340.034576pt}}
\pgfusepath{stroke}
\pgfpathmoveto{\pgfpoint{366.902191pt}{340.005493pt}}
\pgflineto{\pgfpoint{366.808411pt}{339.869446pt}}
\pgfusepath{stroke}
\pgfpathmoveto{\pgfpoint{366.823669pt}{345.889282pt}}
\pgflineto{\pgfpoint{366.815552pt}{346.042023pt}}
\pgfusepath{stroke}
\pgfpathmoveto{\pgfpoint{366.908813pt}{346.016357pt}}
\pgflineto{\pgfpoint{366.823669pt}{345.889282pt}}
\pgfusepath{stroke}
\pgfpathmoveto{\pgfpoint{366.836090pt}{351.904510pt}}
\pgflineto{\pgfpoint{366.826874pt}{352.046692pt}}
\pgfusepath{stroke}
\pgfpathmoveto{\pgfpoint{366.914032pt}{352.023804pt}}
\pgflineto{\pgfpoint{366.836090pt}{351.904510pt}}
\pgfusepath{stroke}
\pgfpathmoveto{\pgfpoint{366.846375pt}{357.916077pt}}
\pgflineto{\pgfpoint{366.836365pt}{358.049194pt}}
\pgfusepath{stroke}
\pgfpathmoveto{\pgfpoint{366.918213pt}{358.028564pt}}
\pgflineto{\pgfpoint{366.846375pt}{357.916077pt}}
\pgfusepath{stroke}
\pgfpathmoveto{\pgfpoint{366.855042pt}{363.924683pt}}
\pgflineto{\pgfpoint{366.844452pt}{364.049896pt}}
\pgfusepath{stroke}
\pgfpathmoveto{\pgfpoint{366.921692pt}{364.031189pt}}
\pgflineto{\pgfpoint{366.855042pt}{363.924683pt}}
\pgfusepath{stroke}
\pgfpathmoveto{\pgfpoint{366.862396pt}{369.930908pt}}
\pgflineto{\pgfpoint{366.851440pt}{370.049133pt}}
\pgfusepath{stroke}
\pgfpathmoveto{\pgfpoint{366.924561pt}{370.032074pt}}
\pgflineto{\pgfpoint{366.862396pt}{369.930908pt}}
\pgfusepath{stroke}
\pgfpathmoveto{\pgfpoint{372.865417pt}{77.060776pt}}
\pgflineto{\pgfpoint{372.908203pt}{77.000671pt}}
\pgfusepath{stroke}
\pgfpathmoveto{\pgfpoint{372.863586pt}{76.987015pt}}
\pgflineto{\pgfpoint{372.865417pt}{77.060776pt}}
\pgfusepath{stroke}
\pgfpathmoveto{\pgfpoint{372.861267pt}{83.053879pt}}
\pgflineto{\pgfpoint{372.906067pt}{82.992477pt}}
\pgfusepath{stroke}
\pgfpathmoveto{\pgfpoint{372.860260pt}{82.977875pt}}
\pgflineto{\pgfpoint{372.861267pt}{83.053879pt}}
\pgfusepath{stroke}
\pgfpathmoveto{\pgfpoint{372.856781pt}{89.047218pt}}
\pgflineto{\pgfpoint{372.903687pt}{88.984444pt}}
\pgfusepath{stroke}
\pgfpathmoveto{\pgfpoint{372.856628pt}{88.968849pt}}
\pgflineto{\pgfpoint{372.856781pt}{89.047218pt}}
\pgfusepath{stroke}
\pgfpathmoveto{\pgfpoint{372.851868pt}{95.040771pt}}
\pgflineto{\pgfpoint{372.901062pt}{94.976593pt}}
\pgfusepath{stroke}
\pgfpathmoveto{\pgfpoint{372.852722pt}{94.959923pt}}
\pgflineto{\pgfpoint{372.851868pt}{95.040771pt}}
\pgfusepath{stroke}
\pgfpathmoveto{\pgfpoint{372.846436pt}{101.034492pt}}
\pgflineto{\pgfpoint{372.898102pt}{100.968948pt}}
\pgfusepath{stroke}
\pgfpathmoveto{\pgfpoint{372.848450pt}{100.951050pt}}
\pgflineto{\pgfpoint{372.846436pt}{101.034492pt}}
\pgfusepath{stroke}
\pgfpathmoveto{\pgfpoint{372.840485pt}{107.028419pt}}
\pgflineto{\pgfpoint{372.894806pt}{106.961487pt}}
\pgfusepath{stroke}
\pgfpathmoveto{\pgfpoint{372.843781pt}{106.942276pt}}
\pgflineto{\pgfpoint{372.840485pt}{107.028419pt}}
\pgfusepath{stroke}
\pgfpathmoveto{\pgfpoint{372.833954pt}{113.022522pt}}
\pgflineto{\pgfpoint{372.891174pt}{112.954185pt}}
\pgfusepath{stroke}
\pgfpathmoveto{\pgfpoint{372.838715pt}{112.933525pt}}
\pgflineto{\pgfpoint{372.833954pt}{113.022522pt}}
\pgfusepath{stroke}
\pgfpathmoveto{\pgfpoint{372.826752pt}{119.016777pt}}
\pgflineto{\pgfpoint{372.887085pt}{118.947067pt}}
\pgfusepath{stroke}
\pgfpathmoveto{\pgfpoint{372.833191pt}{118.924812pt}}
\pgflineto{\pgfpoint{372.826752pt}{119.016777pt}}
\pgfusepath{stroke}
\pgfpathmoveto{\pgfpoint{372.818848pt}{125.011139pt}}
\pgflineto{\pgfpoint{372.882568pt}{124.940117pt}}
\pgfusepath{stroke}
\pgfpathmoveto{\pgfpoint{372.827209pt}{124.916092pt}}
\pgflineto{\pgfpoint{372.818848pt}{125.011139pt}}
\pgfusepath{stroke}
\pgfpathmoveto{\pgfpoint{372.810120pt}{131.005554pt}}
\pgflineto{\pgfpoint{372.877472pt}{130.933289pt}}
\pgfusepath{stroke}
\pgfpathmoveto{\pgfpoint{372.820648pt}{130.907333pt}}
\pgflineto{\pgfpoint{372.810120pt}{131.005554pt}}
\pgfusepath{stroke}
\pgfpathmoveto{\pgfpoint{372.800568pt}{137.000000pt}}
\pgflineto{\pgfpoint{372.871826pt}{136.926559pt}}
\pgfusepath{stroke}
\pgfpathmoveto{\pgfpoint{372.813507pt}{136.898468pt}}
\pgflineto{\pgfpoint{372.800568pt}{137.000000pt}}
\pgfusepath{stroke}
\pgfpathmoveto{\pgfpoint{372.790039pt}{142.994354pt}}
\pgflineto{\pgfpoint{372.865570pt}{142.919891pt}}
\pgfusepath{stroke}
\pgfpathmoveto{\pgfpoint{372.805786pt}{142.889465pt}}
\pgflineto{\pgfpoint{372.790039pt}{142.994354pt}}
\pgfusepath{stroke}
\pgfpathmoveto{\pgfpoint{372.778534pt}{148.988556pt}}
\pgflineto{\pgfpoint{372.858612pt}{148.913208pt}}
\pgfusepath{stroke}
\pgfpathmoveto{\pgfpoint{372.797394pt}{148.880219pt}}
\pgflineto{\pgfpoint{372.778534pt}{148.988556pt}}
\pgfusepath{stroke}
\pgfpathmoveto{\pgfpoint{372.765991pt}{154.982422pt}}
\pgflineto{\pgfpoint{372.850922pt}{154.906433pt}}
\pgfusepath{stroke}
\pgfpathmoveto{\pgfpoint{372.788330pt}{154.870667pt}}
\pgflineto{\pgfpoint{372.765991pt}{154.982422pt}}
\pgfusepath{stroke}
\pgfpathmoveto{\pgfpoint{372.752380pt}{160.975861pt}}
\pgflineto{\pgfpoint{372.842468pt}{160.899490pt}}
\pgfusepath{stroke}
\pgfpathmoveto{\pgfpoint{372.778656pt}{160.860703pt}}
\pgflineto{\pgfpoint{372.752380pt}{160.975861pt}}
\pgfusepath{stroke}
\pgfpathmoveto{\pgfpoint{372.737732pt}{166.968704pt}}
\pgflineto{\pgfpoint{372.833282pt}{166.892258pt}}
\pgfusepath{stroke}
\pgfpathmoveto{\pgfpoint{372.768311pt}{166.850220pt}}
\pgflineto{\pgfpoint{372.737732pt}{166.968704pt}}
\pgfusepath{stroke}
\pgfpathmoveto{\pgfpoint{372.722107pt}{172.960785pt}}
\pgflineto{\pgfpoint{372.823364pt}{172.884628pt}}
\pgfusepath{stroke}
\pgfpathmoveto{\pgfpoint{372.757416pt}{172.839111pt}}
\pgflineto{\pgfpoint{372.722107pt}{172.960785pt}}
\pgfusepath{stroke}
\pgfpathmoveto{\pgfpoint{372.705627pt}{178.951950pt}}
\pgflineto{\pgfpoint{372.812744pt}{178.876495pt}}
\pgfusepath{stroke}
\pgfpathmoveto{\pgfpoint{372.746063pt}{178.827332pt}}
\pgflineto{\pgfpoint{372.705627pt}{178.951950pt}}
\pgfusepath{stroke}
\pgfpathmoveto{\pgfpoint{372.688507pt}{184.942108pt}}
\pgflineto{\pgfpoint{372.801605pt}{184.867752pt}}
\pgfusepath{stroke}
\pgfpathmoveto{\pgfpoint{372.734375pt}{184.814774pt}}
\pgflineto{\pgfpoint{372.688507pt}{184.942108pt}}
\pgfusepath{stroke}
\pgfpathmoveto{\pgfpoint{372.671021pt}{190.931229pt}}
\pgflineto{\pgfpoint{372.790100pt}{190.858368pt}}
\pgfusepath{stroke}
\pgfpathmoveto{\pgfpoint{372.722565pt}{190.801498pt}}
\pgflineto{\pgfpoint{372.671021pt}{190.931229pt}}
\pgfusepath{stroke}
\pgfpathmoveto{\pgfpoint{372.653473pt}{196.919464pt}}
\pgflineto{\pgfpoint{372.778442pt}{196.848389pt}}
\pgfusepath{stroke}
\pgfpathmoveto{\pgfpoint{372.710815pt}{196.787613pt}}
\pgflineto{\pgfpoint{372.653473pt}{196.919464pt}}
\pgfusepath{stroke}
\pgfpathmoveto{\pgfpoint{372.636230pt}{202.907104pt}}
\pgflineto{\pgfpoint{372.766937pt}{202.837982pt}}
\pgfusepath{stroke}
\pgfpathmoveto{\pgfpoint{372.699341pt}{202.773376pt}}
\pgflineto{\pgfpoint{372.636230pt}{202.907104pt}}
\pgfusepath{stroke}
\pgfpathmoveto{\pgfpoint{372.619507pt}{208.894730pt}}
\pgflineto{\pgfpoint{372.755798pt}{208.827515pt}}
\pgfusepath{stroke}
\pgfpathmoveto{\pgfpoint{372.688202pt}{208.759201pt}}
\pgflineto{\pgfpoint{372.619507pt}{208.894730pt}}
\pgfusepath{stroke}
\pgfpathmoveto{\pgfpoint{372.603424pt}{214.883255pt}}
\pgflineto{\pgfpoint{372.745117pt}{214.817566pt}}
\pgfusepath{stroke}
\pgfpathmoveto{\pgfpoint{372.677399pt}{214.745667pt}}
\pgflineto{\pgfpoint{372.603424pt}{214.883255pt}}
\pgfusepath{stroke}
\pgfpathmoveto{\pgfpoint{372.587646pt}{220.873825pt}}
\pgflineto{\pgfpoint{372.734924pt}{220.808945pt}}
\pgfusepath{stroke}
\pgfpathmoveto{\pgfpoint{372.666565pt}{220.733551pt}}
\pgflineto{\pgfpoint{372.587646pt}{220.873825pt}}
\pgfusepath{stroke}
\pgfpathmoveto{\pgfpoint{372.571381pt}{226.867905pt}}
\pgflineto{\pgfpoint{372.724792pt}{226.802734pt}}
\pgfusepath{stroke}
\pgfpathmoveto{\pgfpoint{372.654999pt}{226.723724pt}}
\pgflineto{\pgfpoint{372.571381pt}{226.867905pt}}
\pgfusepath{stroke}
\pgfpathmoveto{\pgfpoint{372.552917pt}{232.867157pt}}
\pgflineto{\pgfpoint{372.713745pt}{232.800201pt}}
\pgfusepath{stroke}
\pgfpathmoveto{\pgfpoint{372.641418pt}{232.717087pt}}
\pgflineto{\pgfpoint{372.552917pt}{232.867157pt}}
\pgfusepath{stroke}
\pgfpathmoveto{\pgfpoint{372.529419pt}{238.873322pt}}
\pgflineto{\pgfpoint{372.700134pt}{238.802856pt}}
\pgfusepath{stroke}
\pgfpathmoveto{\pgfpoint{372.623688pt}{238.714508pt}}
\pgflineto{\pgfpoint{372.529419pt}{238.873322pt}}
\pgfusepath{stroke}
\pgfpathmoveto{\pgfpoint{372.496155pt}{244.887955pt}}
\pgflineto{\pgfpoint{372.680939pt}{244.812210pt}}
\pgfusepath{stroke}
\pgfpathmoveto{\pgfpoint{372.598541pt}{244.716476pt}}
\pgflineto{\pgfpoint{372.496155pt}{244.887955pt}}
\pgfusepath{stroke}
\pgfpathmoveto{\pgfpoint{372.445496pt}{250.911621pt}}
\pgflineto{\pgfpoint{372.651154pt}{250.829544pt}}
\pgfusepath{stroke}
\pgfpathmoveto{\pgfpoint{372.560791pt}{250.722565pt}}
\pgflineto{\pgfpoint{372.445496pt}{250.911621pt}}
\pgfusepath{stroke}
\pgfpathmoveto{\pgfpoint{372.364929pt}{256.941925pt}}
\pgflineto{\pgfpoint{372.602173pt}{256.854614pt}}
\pgfusepath{stroke}
\pgfpathmoveto{\pgfpoint{372.502350pt}{256.729706pt}}
\pgflineto{\pgfpoint{372.364929pt}{256.941925pt}}
\pgfusepath{stroke}
\pgfpathmoveto{\pgfpoint{372.234924pt}{262.967194pt}}
\pgflineto{\pgfpoint{372.519684pt}{262.881805pt}}
\pgfusepath{stroke}
\pgfpathmoveto{\pgfpoint{372.411499pt}{262.728027pt}}
\pgflineto{\pgfpoint{372.234924pt}{262.967194pt}}
\pgfusepath{stroke}
\pgfpathmoveto{\pgfpoint{372.032166pt}{268.950897pt}}
\pgflineto{\pgfpoint{372.384064pt}{268.889404pt}}
\pgfusepath{stroke}
\pgfpathmoveto{\pgfpoint{372.276794pt}{268.690582pt}}
\pgflineto{\pgfpoint{372.032166pt}{268.950897pt}}
\pgfusepath{stroke}
\pgfpathmoveto{\pgfpoint{371.764130pt}{274.807892pt}}
\pgflineto{\pgfpoint{372.190826pt}{274.819763pt}}
\pgfusepath{stroke}
\pgfpathmoveto{\pgfpoint{372.112610pt}{274.561340pt}}
\pgflineto{\pgfpoint{371.764130pt}{274.807892pt}}
\pgfusepath{stroke}
\pgfpathmoveto{\pgfpoint{371.563904pt}{280.450562pt}}
\pgflineto{\pgfpoint{372.019012pt}{280.599731pt}}
\pgfusepath{stroke}
\pgfpathmoveto{\pgfpoint{372.017487pt}{280.296814pt}}
\pgflineto{\pgfpoint{371.563904pt}{280.450562pt}}
\pgfusepath{stroke}
\pgfpathmoveto{\pgfpoint{371.637909pt}{286.008118pt}}
\pgflineto{\pgfpoint{372.020538pt}{286.292419pt}}
\pgfusepath{stroke}
\pgfpathmoveto{\pgfpoint{372.114624pt}{286.005981pt}}
\pgflineto{\pgfpoint{371.637909pt}{286.008118pt}}
\pgfusepath{stroke}
\pgfpathmoveto{\pgfpoint{371.917328pt}{291.742340pt}}
\pgflineto{\pgfpoint{372.178650pt}{292.080139pt}}
\pgfusepath{stroke}
\pgfpathmoveto{\pgfpoint{372.329071pt}{291.855774pt}}
\pgflineto{\pgfpoint{371.917328pt}{291.742340pt}}
\pgfusepath{stroke}
\pgfpathmoveto{\pgfpoint{372.190125pt}{297.670105pt}}
\pgflineto{\pgfpoint{372.353851pt}{297.997589pt}}
\pgfusepath{stroke}
\pgfpathmoveto{\pgfpoint{372.517578pt}{297.833862pt}}
\pgflineto{\pgfpoint{372.190125pt}{297.670105pt}}
\pgfusepath{stroke}
\pgfpathmoveto{\pgfpoint{372.385010pt}{303.687134pt}}
\pgflineto{\pgfpoint{372.487030pt}{303.983215pt}}
\pgfusepath{stroke}
\pgfpathmoveto{\pgfpoint{372.644287pt}{303.862793pt}}
\pgflineto{\pgfpoint{372.385010pt}{303.687134pt}}
\pgfusepath{stroke}
\pgfpathmoveto{\pgfpoint{372.514648pt}{309.729553pt}}
\pgflineto{\pgfpoint{372.579529pt}{309.992981pt}}
\pgfusepath{stroke}
\pgfpathmoveto{\pgfpoint{372.724609pt}{309.901367pt}}
\pgflineto{\pgfpoint{372.514648pt}{309.729553pt}}
\pgfusepath{stroke}
\pgfpathmoveto{\pgfpoint{372.601685pt}{315.773438pt}}
\pgflineto{\pgfpoint{372.643799pt}{316.008484pt}}
\pgfusepath{stroke}
\pgfpathmoveto{\pgfpoint{372.776428pt}{315.936218pt}}
\pgflineto{\pgfpoint{372.601685pt}{315.773438pt}}
\pgfusepath{stroke}
\pgfpathmoveto{\pgfpoint{372.662170pt}{321.811829pt}}
\pgflineto{\pgfpoint{372.689758pt}{322.023285pt}}
\pgfusepath{stroke}
\pgfpathmoveto{\pgfpoint{372.811096pt}{321.964447pt}}
\pgflineto{\pgfpoint{372.662170pt}{321.811829pt}}
\pgfusepath{stroke}
\pgfpathmoveto{\pgfpoint{372.705811pt}{327.843475pt}}
\pgflineto{\pgfpoint{372.723694pt}{328.035461pt}}
\pgfusepath{stroke}
\pgfpathmoveto{\pgfpoint{372.835327pt}{327.986328pt}}
\pgflineto{\pgfpoint{372.705811pt}{327.843475pt}}
\pgfusepath{stroke}
\pgfpathmoveto{\pgfpoint{372.738403pt}{333.868958pt}}
\pgflineto{\pgfpoint{372.749634pt}{334.044739pt}}
\pgfusepath{stroke}
\pgfpathmoveto{\pgfpoint{372.852844pt}{334.002869pt}}
\pgflineto{\pgfpoint{372.738403pt}{333.868958pt}}
\pgfusepath{stroke}
\pgfpathmoveto{\pgfpoint{372.763550pt}{339.889221pt}}
\pgflineto{\pgfpoint{372.770020pt}{340.051392pt}}
\pgfusepath{stroke}
\pgfpathmoveto{\pgfpoint{372.865997pt}{340.015076pt}}
\pgflineto{\pgfpoint{372.763550pt}{339.889221pt}}
\pgfusepath{stroke}
\pgfpathmoveto{\pgfpoint{372.783447pt}{345.905151pt}}
\pgflineto{\pgfpoint{372.786377pt}{346.055695pt}}
\pgfusepath{stroke}
\pgfpathmoveto{\pgfpoint{372.876129pt}{346.023804pt}}
\pgflineto{\pgfpoint{372.783447pt}{345.905151pt}}
\pgfusepath{stroke}
\pgfpathmoveto{\pgfpoint{372.799530pt}{351.917480pt}}
\pgflineto{\pgfpoint{372.799866pt}{352.058044pt}}
\pgfusepath{stroke}
\pgfpathmoveto{\pgfpoint{372.884125pt}{352.029724pt}}
\pgflineto{\pgfpoint{372.799530pt}{351.917480pt}}
\pgfusepath{stroke}
\pgfpathmoveto{\pgfpoint{372.812805pt}{357.926880pt}}
\pgflineto{\pgfpoint{372.811096pt}{358.058777pt}}
\pgfusepath{stroke}
\pgfpathmoveto{\pgfpoint{372.890564pt}{358.033386pt}}
\pgflineto{\pgfpoint{372.812805pt}{357.926880pt}}
\pgfusepath{stroke}
\pgfpathmoveto{\pgfpoint{372.823883pt}{363.933807pt}}
\pgflineto{\pgfpoint{372.820618pt}{364.058044pt}}
\pgfusepath{stroke}
\pgfpathmoveto{\pgfpoint{372.895844pt}{364.035156pt}}
\pgflineto{\pgfpoint{372.823883pt}{363.933807pt}}
\pgfusepath{stroke}
\pgfpathmoveto{\pgfpoint{372.833313pt}{369.938721pt}}
\pgflineto{\pgfpoint{372.828827pt}{370.056213pt}}
\pgfusepath{stroke}
\pgfpathmoveto{\pgfpoint{372.900208pt}{370.035400pt}}
\pgflineto{\pgfpoint{372.833313pt}{369.938721pt}}
\pgfusepath{stroke}
\pgfpathmoveto{\pgfpoint{378.848053pt}{77.056747pt}}
\pgflineto{\pgfpoint{378.892120pt}{76.998550pt}}
\pgfusepath{stroke}
\pgfpathmoveto{\pgfpoint{378.848419pt}{76.983734pt}}
\pgflineto{\pgfpoint{378.848053pt}{77.056747pt}}
\pgfusepath{stroke}
\pgfpathmoveto{\pgfpoint{378.843750pt}{83.049500pt}}
\pgflineto{\pgfpoint{378.889832pt}{82.990112pt}}
\pgfusepath{stroke}
\pgfpathmoveto{\pgfpoint{378.844971pt}{82.974365pt}}
\pgflineto{\pgfpoint{378.843750pt}{83.049500pt}}
\pgfusepath{stroke}
\pgfpathmoveto{\pgfpoint{378.839081pt}{89.042427pt}}
\pgflineto{\pgfpoint{378.887268pt}{88.981842pt}}
\pgfusepath{stroke}
\pgfpathmoveto{\pgfpoint{378.841278pt}{88.965042pt}}
\pgflineto{\pgfpoint{378.839081pt}{89.042427pt}}
\pgfusepath{stroke}
\pgfpathmoveto{\pgfpoint{378.833954pt}{95.035507pt}}
\pgflineto{\pgfpoint{378.884460pt}{94.973724pt}}
\pgfusepath{stroke}
\pgfpathmoveto{\pgfpoint{378.837280pt}{94.955788pt}}
\pgflineto{\pgfpoint{378.833954pt}{95.035507pt}}
\pgfusepath{stroke}
\pgfpathmoveto{\pgfpoint{378.828369pt}{101.028748pt}}
\pgflineto{\pgfpoint{378.881348pt}{100.965752pt}}
\pgfusepath{stroke}
\pgfpathmoveto{\pgfpoint{378.832947pt}{100.946587pt}}
\pgflineto{\pgfpoint{378.828369pt}{101.028748pt}}
\pgfusepath{stroke}
\pgfpathmoveto{\pgfpoint{378.822266pt}{107.022110pt}}
\pgflineto{\pgfpoint{378.877899pt}{106.957932pt}}
\pgfusepath{stroke}
\pgfpathmoveto{\pgfpoint{378.828247pt}{106.937393pt}}
\pgflineto{\pgfpoint{378.822266pt}{107.022110pt}}
\pgfusepath{stroke}
\pgfpathmoveto{\pgfpoint{378.815613pt}{113.015610pt}}
\pgflineto{\pgfpoint{378.874084pt}{112.950256pt}}
\pgfusepath{stroke}
\pgfpathmoveto{\pgfpoint{378.823181pt}{112.928230pt}}
\pgflineto{\pgfpoint{378.815613pt}{113.015610pt}}
\pgfusepath{stroke}
\pgfpathmoveto{\pgfpoint{378.808289pt}{119.009186pt}}
\pgflineto{\pgfpoint{378.869873pt}{118.942696pt}}
\pgfusepath{stroke}
\pgfpathmoveto{\pgfpoint{378.817657pt}{118.919060pt}}
\pgflineto{\pgfpoint{378.808289pt}{119.009186pt}}
\pgfusepath{stroke}
\pgfpathmoveto{\pgfpoint{378.800323pt}{125.002823pt}}
\pgflineto{\pgfpoint{378.865204pt}{124.935257pt}}
\pgfusepath{stroke}
\pgfpathmoveto{\pgfpoint{378.811707pt}{124.909851pt}}
\pgflineto{\pgfpoint{378.800323pt}{125.002823pt}}
\pgfusepath{stroke}
\pgfpathmoveto{\pgfpoint{378.791626pt}{130.996445pt}}
\pgflineto{\pgfpoint{378.860046pt}{130.927887pt}}
\pgfusepath{stroke}
\pgfpathmoveto{\pgfpoint{378.805237pt}{130.900543pt}}
\pgflineto{\pgfpoint{378.791626pt}{130.996445pt}}
\pgfusepath{stroke}
\pgfpathmoveto{\pgfpoint{378.782104pt}{136.990021pt}}
\pgflineto{\pgfpoint{378.854340pt}{136.920593pt}}
\pgfusepath{stroke}
\pgfpathmoveto{\pgfpoint{378.798248pt}{136.891129pt}}
\pgflineto{\pgfpoint{378.782104pt}{136.990021pt}}
\pgfusepath{stroke}
\pgfpathmoveto{\pgfpoint{378.771729pt}{142.983459pt}}
\pgflineto{\pgfpoint{378.848083pt}{142.913300pt}}
\pgfusepath{stroke}
\pgfpathmoveto{\pgfpoint{378.790710pt}{142.881531pt}}
\pgflineto{\pgfpoint{378.771729pt}{142.983459pt}}
\pgfusepath{stroke}
\pgfpathmoveto{\pgfpoint{378.760498pt}{148.976685pt}}
\pgflineto{\pgfpoint{378.841187pt}{148.905945pt}}
\pgfusepath{stroke}
\pgfpathmoveto{\pgfpoint{378.782593pt}{148.871674pt}}
\pgflineto{\pgfpoint{378.760498pt}{148.976685pt}}
\pgfusepath{stroke}
\pgfpathmoveto{\pgfpoint{378.748322pt}{154.969574pt}}
\pgflineto{\pgfpoint{378.833649pt}{154.898468pt}}
\pgfusepath{stroke}
\pgfpathmoveto{\pgfpoint{378.773926pt}{154.861496pt}}
\pgflineto{\pgfpoint{378.748322pt}{154.969574pt}}
\pgfusepath{stroke}
\pgfpathmoveto{\pgfpoint{378.735229pt}{160.962021pt}}
\pgflineto{\pgfpoint{378.825439pt}{160.890808pt}}
\pgfusepath{stroke}
\pgfpathmoveto{\pgfpoint{378.764648pt}{160.850922pt}}
\pgflineto{\pgfpoint{378.735229pt}{160.962021pt}}
\pgfusepath{stroke}
\pgfpathmoveto{\pgfpoint{378.721252pt}{166.953918pt}}
\pgflineto{\pgfpoint{378.816559pt}{166.882874pt}}
\pgfusepath{stroke}
\pgfpathmoveto{\pgfpoint{378.754883pt}{166.839890pt}}
\pgflineto{\pgfpoint{378.721252pt}{166.953918pt}}
\pgfusepath{stroke}
\pgfpathmoveto{\pgfpoint{378.706390pt}{172.945129pt}}
\pgflineto{\pgfpoint{378.807068pt}{172.874588pt}}
\pgfusepath{stroke}
\pgfpathmoveto{\pgfpoint{378.744598pt}{172.828308pt}}
\pgflineto{\pgfpoint{378.706390pt}{172.945129pt}}
\pgfusepath{stroke}
\pgfpathmoveto{\pgfpoint{378.690796pt}{178.935577pt}}
\pgflineto{\pgfpoint{378.796936pt}{178.865875pt}}
\pgfusepath{stroke}
\pgfpathmoveto{\pgfpoint{378.733887pt}{178.816116pt}}
\pgflineto{\pgfpoint{378.690796pt}{178.935577pt}}
\pgfusepath{stroke}
\pgfpathmoveto{\pgfpoint{378.674591pt}{184.925232pt}}
\pgflineto{\pgfpoint{378.786377pt}{184.856689pt}}
\pgfusepath{stroke}
\pgfpathmoveto{\pgfpoint{378.722900pt}{184.803329pt}}
\pgflineto{\pgfpoint{378.674591pt}{184.925232pt}}
\pgfusepath{stroke}
\pgfpathmoveto{\pgfpoint{378.657959pt}{190.914062pt}}
\pgflineto{\pgfpoint{378.775391pt}{190.847015pt}}
\pgfusepath{stroke}
\pgfpathmoveto{\pgfpoint{378.711670pt}{190.789963pt}}
\pgflineto{\pgfpoint{378.657959pt}{190.914062pt}}
\pgfusepath{stroke}
\pgfpathmoveto{\pgfpoint{378.641052pt}{196.902267pt}}
\pgflineto{\pgfpoint{378.764160pt}{196.836945pt}}
\pgfusepath{stroke}
\pgfpathmoveto{\pgfpoint{378.700378pt}{196.776138pt}}
\pgflineto{\pgfpoint{378.641052pt}{196.902267pt}}
\pgfusepath{stroke}
\pgfpathmoveto{\pgfpoint{378.624054pt}{202.890076pt}}
\pgflineto{\pgfpoint{378.752838pt}{202.826630pt}}
\pgfusepath{stroke}
\pgfpathmoveto{\pgfpoint{378.689026pt}{202.762039pt}}
\pgflineto{\pgfpoint{378.624054pt}{202.890076pt}}
\pgfusepath{stroke}
\pgfpathmoveto{\pgfpoint{378.606964pt}{208.877945pt}}
\pgflineto{\pgfpoint{378.741455pt}{208.816360pt}}
\pgfusepath{stroke}
\pgfpathmoveto{\pgfpoint{378.677612pt}{208.747971pt}}
\pgflineto{\pgfpoint{378.606964pt}{208.877945pt}}
\pgfusepath{stroke}
\pgfpathmoveto{\pgfpoint{378.589630pt}{214.866486pt}}
\pgflineto{\pgfpoint{378.729980pt}{214.806564pt}}
\pgfusepath{stroke}
\pgfpathmoveto{\pgfpoint{378.665955pt}{214.734329pt}}
\pgflineto{\pgfpoint{378.589630pt}{214.866486pt}}
\pgfusepath{stroke}
\pgfpathmoveto{\pgfpoint{378.571564pt}{220.856445pt}}
\pgflineto{\pgfpoint{378.718201pt}{220.797791pt}}
\pgfusepath{stroke}
\pgfpathmoveto{\pgfpoint{378.653687pt}{220.721542pt}}
\pgflineto{\pgfpoint{378.571564pt}{220.856445pt}}
\pgfusepath{stroke}
\pgfpathmoveto{\pgfpoint{378.551788pt}{226.848663pt}}
\pgflineto{\pgfpoint{378.705505pt}{226.790710pt}}
\pgfusepath{stroke}
\pgfpathmoveto{\pgfpoint{378.640015pt}{226.710083pt}}
\pgflineto{\pgfpoint{378.551788pt}{226.848663pt}}
\pgfusepath{stroke}
\pgfpathmoveto{\pgfpoint{378.528656pt}{232.843887pt}}
\pgflineto{\pgfpoint{378.690918pt}{232.786026pt}}
\pgfusepath{stroke}
\pgfpathmoveto{\pgfpoint{378.623749pt}{232.700241pt}}
\pgflineto{\pgfpoint{378.528656pt}{232.843887pt}}
\pgfusepath{stroke}
\pgfpathmoveto{\pgfpoint{378.499603pt}{238.842590pt}}
\pgflineto{\pgfpoint{378.672760pt}{238.784317pt}}
\pgfusepath{stroke}
\pgfpathmoveto{\pgfpoint{378.603149pt}{238.692078pt}}
\pgflineto{\pgfpoint{378.499603pt}{238.842590pt}}
\pgfusepath{stroke}
\pgfpathmoveto{\pgfpoint{378.460754pt}{244.844421pt}}
\pgflineto{\pgfpoint{378.648438pt}{244.785767pt}}
\pgfusepath{stroke}
\pgfpathmoveto{\pgfpoint{378.575714pt}{244.684906pt}}
\pgflineto{\pgfpoint{378.460754pt}{244.844421pt}}
\pgfusepath{stroke}
\pgfpathmoveto{\pgfpoint{378.406769pt}{250.847260pt}}
\pgflineto{\pgfpoint{378.614136pt}{250.789566pt}}
\pgfusepath{stroke}
\pgfpathmoveto{\pgfpoint{378.538025pt}{250.676697pt}}
\pgflineto{\pgfpoint{378.406769pt}{250.847260pt}}
\pgfusepath{stroke}
\pgfpathmoveto{\pgfpoint{378.330597pt}{256.845184pt}}
\pgflineto{\pgfpoint{378.564453pt}{256.792572pt}}
\pgfusepath{stroke}
\pgfpathmoveto{\pgfpoint{378.486145pt}{256.662781pt}}
\pgflineto{\pgfpoint{378.330597pt}{256.845184pt}}
\pgfusepath{stroke}
\pgfpathmoveto{\pgfpoint{378.225403pt}{262.824951pt}}
\pgflineto{\pgfpoint{378.493439pt}{262.786682pt}}
\pgfusepath{stroke}
\pgfpathmoveto{\pgfpoint{378.416870pt}{262.633514pt}}
\pgflineto{\pgfpoint{378.225403pt}{262.824951pt}}
\pgfusepath{stroke}
\pgfpathmoveto{\pgfpoint{378.091614pt}{268.761780pt}}
\pgflineto{\pgfpoint{378.398590pt}{268.755341pt}}
\pgfusepath{stroke}
\pgfpathmoveto{\pgfpoint{378.333344pt}{268.572449pt}}
\pgflineto{\pgfpoint{378.091614pt}{268.761780pt}}
\pgfusepath{stroke}
\pgfpathmoveto{\pgfpoint{377.953949pt}{274.623199pt}}
\pgflineto{\pgfpoint{378.292786pt}{274.674194pt}}
\pgfusepath{stroke}
\pgfpathmoveto{\pgfpoint{378.255615pt}{274.460693pt}}
\pgflineto{\pgfpoint{377.953949pt}{274.623199pt}}
\pgfusepath{stroke}
\pgfpathmoveto{\pgfpoint{377.874023pt}{280.400818pt}}
\pgflineto{\pgfpoint{378.216156pt}{280.530731pt}}
\pgfusepath{stroke}
\pgfpathmoveto{\pgfpoint{378.225647pt}{280.299500pt}}
\pgflineto{\pgfpoint{377.874023pt}{280.400818pt}}
\pgfusepath{stroke}
\pgfpathmoveto{\pgfpoint{377.909882pt}{286.152435pt}}
\pgflineto{\pgfpoint{378.213806pt}{286.357056pt}}
\pgfusepath{stroke}
\pgfpathmoveto{\pgfpoint{378.275818pt}{286.133789pt}}
\pgflineto{\pgfpoint{377.909882pt}{286.152435pt}}
\pgfusepath{stroke}
\pgfpathmoveto{\pgfpoint{378.044037pt}{291.964050pt}}
\pgflineto{\pgfpoint{378.283691pt}{292.212494pt}}
\pgfusepath{stroke}
\pgfpathmoveto{\pgfpoint{378.384827pt}{292.019012pt}}
\pgflineto{\pgfpoint{378.044037pt}{291.964050pt}}
\pgfusepath{stroke}
\pgfpathmoveto{\pgfpoint{378.207153pt}{297.865021pt}}
\pgflineto{\pgfpoint{378.382751pt}{298.124268pt}}
\pgfusepath{stroke}
\pgfpathmoveto{\pgfpoint{378.503204pt}{297.967041pt}}
\pgflineto{\pgfpoint{378.207153pt}{297.865021pt}}
\pgfusepath{stroke}
\pgfpathmoveto{\pgfpoint{378.351318pt}{303.831329pt}}
\pgflineto{\pgfpoint{378.476532pt}{304.081726pt}}
\pgfusepath{stroke}
\pgfpathmoveto{\pgfpoint{378.601715pt}{303.956543pt}}
\pgflineto{\pgfpoint{378.351318pt}{303.831329pt}}
\pgfusepath{stroke}
\pgfpathmoveto{\pgfpoint{378.463776pt}{309.831757pt}}
\pgflineto{\pgfpoint{378.552887pt}{310.065430pt}}
\pgfusepath{stroke}
\pgfpathmoveto{\pgfpoint{378.675262pt}{309.965210pt}}
\pgflineto{\pgfpoint{378.463776pt}{309.831757pt}}
\pgfusepath{stroke}
\pgfpathmoveto{\pgfpoint{378.547974pt}{315.846313pt}}
\pgflineto{\pgfpoint{378.612000pt}{316.061676pt}}
\pgfusepath{stroke}
\pgfpathmoveto{\pgfpoint{378.728424pt}{315.980164pt}}
\pgflineto{\pgfpoint{378.547974pt}{315.846313pt}}
\pgfusepath{stroke}
\pgfpathmoveto{\pgfpoint{378.610840pt}{321.864929pt}}
\pgflineto{\pgfpoint{378.657349pt}{322.063049pt}}
\pgfusepath{stroke}
\pgfpathmoveto{\pgfpoint{378.766907pt}{321.995514pt}}
\pgflineto{\pgfpoint{378.610840pt}{321.864929pt}}
\pgfusepath{stroke}
\pgfpathmoveto{\pgfpoint{378.658386pt}{327.883270pt}}
\pgflineto{\pgfpoint{378.692413pt}{328.065918pt}}
\pgfusepath{stroke}
\pgfpathmoveto{\pgfpoint{378.795197pt}{328.008942pt}}
\pgflineto{\pgfpoint{378.658386pt}{327.883270pt}}
\pgfusepath{stroke}
\pgfpathmoveto{\pgfpoint{378.695007pt}{333.899567pt}}
\pgflineto{\pgfpoint{378.720032pt}{334.068604pt}}
\pgfusepath{stroke}
\pgfpathmoveto{\pgfpoint{378.816437pt}{334.019775pt}}
\pgflineto{\pgfpoint{378.695007pt}{333.899567pt}}
\pgfusepath{stroke}
\pgfpathmoveto{\pgfpoint{378.723816pt}{339.913300pt}}
\pgflineto{\pgfpoint{378.742126pt}{340.070496pt}}
\pgfusepath{stroke}
\pgfpathmoveto{\pgfpoint{378.832764pt}{340.028076pt}}
\pgflineto{\pgfpoint{378.723816pt}{339.913300pt}}
\pgfusepath{stroke}
\pgfpathmoveto{\pgfpoint{378.746918pt}{345.924500pt}}
\pgflineto{\pgfpoint{378.760132pt}{346.071289pt}}
\pgfusepath{stroke}
\pgfpathmoveto{\pgfpoint{378.845551pt}{346.034027pt}}
\pgflineto{\pgfpoint{378.746918pt}{345.924500pt}}
\pgfusepath{stroke}
\pgfpathmoveto{\pgfpoint{378.765747pt}{351.933350pt}}
\pgflineto{\pgfpoint{378.775024pt}{352.071045pt}}
\pgfusepath{stroke}
\pgfpathmoveto{\pgfpoint{378.855774pt}{352.037903pt}}
\pgflineto{\pgfpoint{378.765747pt}{351.933350pt}}
\pgfusepath{stroke}
\pgfpathmoveto{\pgfpoint{378.781342pt}{357.940063pt}}
\pgflineto{\pgfpoint{378.787537pt}{358.069702pt}}
\pgfusepath{stroke}
\pgfpathmoveto{\pgfpoint{378.864075pt}{358.040070pt}}
\pgflineto{\pgfpoint{378.781342pt}{357.940063pt}}
\pgfusepath{stroke}
\pgfpathmoveto{\pgfpoint{378.794434pt}{363.944946pt}}
\pgflineto{\pgfpoint{378.798157pt}{364.067413pt}}
\pgfusepath{stroke}
\pgfpathmoveto{\pgfpoint{378.870880pt}{364.040680pt}}
\pgflineto{\pgfpoint{378.794434pt}{363.944946pt}}
\pgfusepath{stroke}
\pgfpathmoveto{\pgfpoint{378.805542pt}{369.948242pt}}
\pgflineto{\pgfpoint{378.807281pt}{370.064270pt}}
\pgfusepath{stroke}
\pgfpathmoveto{\pgfpoint{378.876556pt}{370.040039pt}}
\pgflineto{\pgfpoint{378.805542pt}{369.948242pt}}
\pgfusepath{stroke}
\pgfpathmoveto{\pgfpoint{384.831055pt}{77.052551pt}}
\pgflineto{\pgfpoint{384.876251pt}{76.996277pt}}
\pgfusepath{stroke}
\pgfpathmoveto{\pgfpoint{384.833466pt}{76.980423pt}}
\pgflineto{\pgfpoint{384.831055pt}{77.052551pt}}
\pgfusepath{stroke}
\pgfpathmoveto{\pgfpoint{384.826630pt}{83.044922pt}}
\pgflineto{\pgfpoint{384.873810pt}{82.987640pt}}
\pgfusepath{stroke}
\pgfpathmoveto{\pgfpoint{384.829987pt}{82.970764pt}}
\pgflineto{\pgfpoint{384.826630pt}{83.044922pt}}
\pgfusepath{stroke}
\pgfpathmoveto{\pgfpoint{384.821777pt}{89.037445pt}}
\pgflineto{\pgfpoint{384.871094pt}{88.979088pt}}
\pgfusepath{stroke}
\pgfpathmoveto{\pgfpoint{384.826233pt}{88.961166pt}}
\pgflineto{\pgfpoint{384.821777pt}{89.037445pt}}
\pgfusepath{stroke}
\pgfpathmoveto{\pgfpoint{384.816528pt}{95.030098pt}}
\pgflineto{\pgfpoint{384.868164pt}{94.970695pt}}
\pgfusepath{stroke}
\pgfpathmoveto{\pgfpoint{384.822205pt}{94.951599pt}}
\pgflineto{\pgfpoint{384.816528pt}{95.030098pt}}
\pgfusepath{stroke}
\pgfpathmoveto{\pgfpoint{384.810822pt}{101.022842pt}}
\pgflineto{\pgfpoint{384.864899pt}{100.962418pt}}
\pgfusepath{stroke}
\pgfpathmoveto{\pgfpoint{384.817841pt}{100.942062pt}}
\pgflineto{\pgfpoint{384.810822pt}{101.022842pt}}
\pgfusepath{stroke}
\pgfpathmoveto{\pgfpoint{384.804626pt}{107.015701pt}}
\pgflineto{\pgfpoint{384.861328pt}{106.954247pt}}
\pgfusepath{stroke}
\pgfpathmoveto{\pgfpoint{384.813110pt}{106.932518pt}}
\pgflineto{\pgfpoint{384.804626pt}{107.015701pt}}
\pgfusepath{stroke}
\pgfpathmoveto{\pgfpoint{384.797882pt}{113.008591pt}}
\pgflineto{\pgfpoint{384.857422pt}{112.946190pt}}
\pgfusepath{stroke}
\pgfpathmoveto{\pgfpoint{384.808075pt}{112.922951pt}}
\pgflineto{\pgfpoint{384.797882pt}{113.008591pt}}
\pgfusepath{stroke}
\pgfpathmoveto{\pgfpoint{384.790527pt}{119.001541pt}}
\pgflineto{\pgfpoint{384.853088pt}{118.938225pt}}
\pgfusepath{stroke}
\pgfpathmoveto{\pgfpoint{384.802582pt}{118.913361pt}}
\pgflineto{\pgfpoint{384.790527pt}{119.001541pt}}
\pgfusepath{stroke}
\pgfpathmoveto{\pgfpoint{384.782562pt}{124.994484pt}}
\pgflineto{\pgfpoint{384.848358pt}{124.930321pt}}
\pgfusepath{stroke}
\pgfpathmoveto{\pgfpoint{384.796692pt}{124.903679pt}}
\pgflineto{\pgfpoint{384.782562pt}{124.994484pt}}
\pgfusepath{stroke}
\pgfpathmoveto{\pgfpoint{384.773895pt}{130.987381pt}}
\pgflineto{\pgfpoint{384.843140pt}{130.922455pt}}
\pgfusepath{stroke}
\pgfpathmoveto{\pgfpoint{384.790344pt}{130.893906pt}}
\pgflineto{\pgfpoint{384.773895pt}{130.987381pt}}
\pgfusepath{stroke}
\pgfpathmoveto{\pgfpoint{384.764496pt}{136.980164pt}}
\pgflineto{\pgfpoint{384.837463pt}{136.914612pt}}
\pgfusepath{stroke}
\pgfpathmoveto{\pgfpoint{384.783508pt}{136.883972pt}}
\pgflineto{\pgfpoint{384.764496pt}{136.980164pt}}
\pgfusepath{stroke}
\pgfpathmoveto{\pgfpoint{384.754333pt}{142.972778pt}}
\pgflineto{\pgfpoint{384.831207pt}{142.906738pt}}
\pgfusepath{stroke}
\pgfpathmoveto{\pgfpoint{384.776215pt}{142.873840pt}}
\pgflineto{\pgfpoint{384.754333pt}{142.972778pt}}
\pgfusepath{stroke}
\pgfpathmoveto{\pgfpoint{384.743378pt}{148.965149pt}}
\pgflineto{\pgfpoint{384.824402pt}{148.898773pt}}
\pgfusepath{stroke}
\pgfpathmoveto{\pgfpoint{384.768402pt}{148.863434pt}}
\pgflineto{\pgfpoint{384.743378pt}{148.965149pt}}
\pgfusepath{stroke}
\pgfpathmoveto{\pgfpoint{384.731598pt}{154.957153pt}}
\pgflineto{\pgfpoint{384.817017pt}{154.890686pt}}
\pgfusepath{stroke}
\pgfpathmoveto{\pgfpoint{384.760071pt}{154.852722pt}}
\pgflineto{\pgfpoint{384.731598pt}{154.957153pt}}
\pgfusepath{stroke}
\pgfpathmoveto{\pgfpoint{384.718994pt}{160.948746pt}}
\pgflineto{\pgfpoint{384.809052pt}{160.882401pt}}
\pgfusepath{stroke}
\pgfpathmoveto{\pgfpoint{384.751221pt}{160.841629pt}}
\pgflineto{\pgfpoint{384.718994pt}{160.948746pt}}
\pgfusepath{stroke}
\pgfpathmoveto{\pgfpoint{384.705566pt}{166.939804pt}}
\pgflineto{\pgfpoint{384.800476pt}{166.873840pt}}
\pgfusepath{stroke}
\pgfpathmoveto{\pgfpoint{384.741913pt}{166.830109pt}}
\pgflineto{\pgfpoint{384.705566pt}{166.939804pt}}
\pgfusepath{stroke}
\pgfpathmoveto{\pgfpoint{384.691406pt}{172.930267pt}}
\pgflineto{\pgfpoint{384.791321pt}{172.864975pt}}
\pgfusepath{stroke}
\pgfpathmoveto{\pgfpoint{384.732178pt}{172.818085pt}}
\pgflineto{\pgfpoint{384.691406pt}{172.930267pt}}
\pgfusepath{stroke}
\pgfpathmoveto{\pgfpoint{384.676544pt}{178.920044pt}}
\pgflineto{\pgfpoint{384.781647pt}{178.855743pt}}
\pgfusepath{stroke}
\pgfpathmoveto{\pgfpoint{384.722046pt}{178.805542pt}}
\pgflineto{\pgfpoint{384.676544pt}{178.920044pt}}
\pgfusepath{stroke}
\pgfpathmoveto{\pgfpoint{384.661102pt}{184.909149pt}}
\pgflineto{\pgfpoint{384.771484pt}{184.846100pt}}
\pgfusepath{stroke}
\pgfpathmoveto{\pgfpoint{384.711609pt}{184.792480pt}}
\pgflineto{\pgfpoint{384.661102pt}{184.909149pt}}
\pgfusepath{stroke}
\pgfpathmoveto{\pgfpoint{384.645142pt}{190.897583pt}}
\pgflineto{\pgfpoint{384.760925pt}{190.836090pt}}
\pgfusepath{stroke}
\pgfpathmoveto{\pgfpoint{384.700867pt}{190.778915pt}}
\pgflineto{\pgfpoint{384.645142pt}{190.897583pt}}
\pgfusepath{stroke}
\pgfpathmoveto{\pgfpoint{384.628723pt}{196.885483pt}}
\pgflineto{\pgfpoint{384.750000pt}{196.825760pt}}
\pgfusepath{stroke}
\pgfpathmoveto{\pgfpoint{384.689941pt}{196.764938pt}}
\pgflineto{\pgfpoint{384.628723pt}{196.885483pt}}
\pgfusepath{stroke}
\pgfpathmoveto{\pgfpoint{384.611908pt}{202.873032pt}}
\pgflineto{\pgfpoint{384.738770pt}{202.815247pt}}
\pgfusepath{stroke}
\pgfpathmoveto{\pgfpoint{384.678741pt}{202.750687pt}}
\pgflineto{\pgfpoint{384.611908pt}{202.873032pt}}
\pgfusepath{stroke}
\pgfpathmoveto{\pgfpoint{384.594604pt}{208.860519pt}}
\pgflineto{\pgfpoint{384.727203pt}{208.804749pt}}
\pgfusepath{stroke}
\pgfpathmoveto{\pgfpoint{384.667206pt}{208.736343pt}}
\pgflineto{\pgfpoint{384.594604pt}{208.860519pt}}
\pgfusepath{stroke}
\pgfpathmoveto{\pgfpoint{384.576508pt}{214.848312pt}}
\pgflineto{\pgfpoint{384.715149pt}{214.794540pt}}
\pgfusepath{stroke}
\pgfpathmoveto{\pgfpoint{384.655151pt}{214.722107pt}}
\pgflineto{\pgfpoint{384.576508pt}{214.848312pt}}
\pgfusepath{stroke}
\pgfpathmoveto{\pgfpoint{384.557068pt}{220.836823pt}}
\pgflineto{\pgfpoint{384.702271pt}{220.784958pt}}
\pgfusepath{stroke}
\pgfpathmoveto{\pgfpoint{384.642120pt}{220.708206pt}}
\pgflineto{\pgfpoint{384.557068pt}{220.836823pt}}
\pgfusepath{stroke}
\pgfpathmoveto{\pgfpoint{384.535400pt}{226.826355pt}}
\pgflineto{\pgfpoint{384.688049pt}{226.776306pt}}
\pgfusepath{stroke}
\pgfpathmoveto{\pgfpoint{384.627472pt}{226.694748pt}}
\pgflineto{\pgfpoint{384.535400pt}{226.826355pt}}
\pgfusepath{stroke}
\pgfpathmoveto{\pgfpoint{384.510132pt}{232.817032pt}}
\pgflineto{\pgfpoint{384.671509pt}{232.768814pt}}
\pgfusepath{stroke}
\pgfpathmoveto{\pgfpoint{384.610291pt}{232.681641pt}}
\pgflineto{\pgfpoint{384.510132pt}{232.817032pt}}
\pgfusepath{stroke}
\pgfpathmoveto{\pgfpoint{384.479279pt}{238.808472pt}}
\pgflineto{\pgfpoint{384.651337pt}{238.762451pt}}
\pgfusepath{stroke}
\pgfpathmoveto{\pgfpoint{384.589325pt}{238.668427pt}}
\pgflineto{\pgfpoint{384.479279pt}{238.808472pt}}
\pgfusepath{stroke}
\pgfpathmoveto{\pgfpoint{384.440277pt}{244.799332pt}}
\pgflineto{\pgfpoint{384.625702pt}{244.756622pt}}
\pgfusepath{stroke}
\pgfpathmoveto{\pgfpoint{384.562988pt}{244.653915pt}}
\pgflineto{\pgfpoint{384.440277pt}{244.799332pt}}
\pgfusepath{stroke}
\pgfpathmoveto{\pgfpoint{384.390137pt}{250.786606pt}}
\pgflineto{\pgfpoint{384.592194pt}{250.749619pt}}
\pgfusepath{stroke}
\pgfpathmoveto{\pgfpoint{384.529602pt}{250.635773pt}}
\pgflineto{\pgfpoint{384.390137pt}{250.786606pt}}
\pgfusepath{stroke}
\pgfpathmoveto{\pgfpoint{384.326202pt}{256.764526pt}}
\pgflineto{\pgfpoint{384.548492pt}{256.737946pt}}
\pgfusepath{stroke}
\pgfpathmoveto{\pgfpoint{384.488068pt}{256.609863pt}}
\pgflineto{\pgfpoint{384.326202pt}{256.764526pt}}
\pgfusepath{stroke}
\pgfpathmoveto{\pgfpoint{384.248230pt}{262.723572pt}}
\pgflineto{\pgfpoint{384.493317pt}{262.715210pt}}
\pgfusepath{stroke}
\pgfpathmoveto{\pgfpoint{384.439270pt}{262.569824pt}}
\pgflineto{\pgfpoint{384.248230pt}{262.723572pt}}
\pgfusepath{stroke}
\pgfpathmoveto{\pgfpoint{384.162720pt}{268.650940pt}}
\pgflineto{\pgfpoint{384.429626pt}{268.672211pt}}
\pgfusepath{stroke}
\pgfpathmoveto{\pgfpoint{384.388977pt}{268.507812pt}}
\pgflineto{\pgfpoint{384.162720pt}{268.650940pt}}
\pgfusepath{stroke}
\pgfpathmoveto{\pgfpoint{384.087799pt}{274.536346pt}}
\pgflineto{\pgfpoint{384.368286pt}{274.600037pt}}
\pgfusepath{stroke}
\pgfpathmoveto{\pgfpoint{384.350403pt}{274.419006pt}}
\pgflineto{\pgfpoint{384.087799pt}{274.536346pt}}
\pgfusepath{stroke}
\pgfpathmoveto{\pgfpoint{384.051208pt}{280.384491pt}}
\pgflineto{\pgfpoint{384.328369pt}{280.498779pt}}
\pgfusepath{stroke}
\pgfpathmoveto{\pgfpoint{384.341522pt}{280.309631pt}}
\pgflineto{\pgfpoint{384.051208pt}{280.384491pt}}
\pgfusepath{stroke}
\pgfpathmoveto{\pgfpoint{384.073334pt}{286.222778pt}}
\pgflineto{\pgfpoint{384.326538pt}{286.384460pt}}
\pgfusepath{stroke}
\pgfpathmoveto{\pgfpoint{384.372894pt}{286.200195pt}}
\pgflineto{\pgfpoint{384.073334pt}{286.222778pt}}
\pgfusepath{stroke}
\pgfpathmoveto{\pgfpoint{384.148285pt}{292.086639pt}}
\pgflineto{\pgfpoint{384.362732pt}{292.281281pt}}
\pgfusepath{stroke}
\pgfpathmoveto{\pgfpoint{384.436615pt}{292.113678pt}}
\pgflineto{\pgfpoint{384.148285pt}{292.086639pt}}
\pgfusepath{stroke}
\pgfpathmoveto{\pgfpoint{384.249542pt}{297.994629pt}}
\pgflineto{\pgfpoint{384.421356pt}{298.204620pt}}
\pgfusepath{stroke}
\pgfpathmoveto{\pgfpoint{384.513000pt}{298.059509pt}}
\pgflineto{\pgfpoint{384.249542pt}{297.994629pt}}
\pgfusepath{stroke}
\pgfpathmoveto{\pgfpoint{384.351746pt}{303.943787pt}}
\pgflineto{\pgfpoint{384.485229pt}{304.155273pt}}
\pgfusepath{stroke}
\pgfpathmoveto{\pgfpoint{384.585419pt}{304.032898pt}}
\pgflineto{\pgfpoint{384.351746pt}{303.943787pt}}
\pgfusepath{stroke}
\pgfpathmoveto{\pgfpoint{384.441528pt}{309.921509pt}}
\pgflineto{\pgfpoint{384.543945pt}{310.126373pt}}
\pgfusepath{stroke}
\pgfpathmoveto{\pgfpoint{384.646362pt}{310.023956pt}}
\pgflineto{\pgfpoint{384.441528pt}{309.921509pt}}
\pgfusepath{stroke}
\pgfpathmoveto{\pgfpoint{384.515350pt}{315.915894pt}}
\pgflineto{\pgfpoint{384.593903pt}{316.110352pt}}
\pgfusepath{stroke}
\pgfpathmoveto{\pgfpoint{384.694855pt}{316.024323pt}}
\pgflineto{\pgfpoint{384.515350pt}{315.915894pt}}
\pgfusepath{stroke}
\pgfpathmoveto{\pgfpoint{384.574493pt}{321.918762pt}}
\pgflineto{\pgfpoint{384.635010pt}{322.101593pt}}
\pgfusepath{stroke}
\pgfpathmoveto{\pgfpoint{384.732605pt}{322.028717pt}}
\pgflineto{\pgfpoint{384.574493pt}{321.918762pt}}
\pgfusepath{stroke}
\pgfpathmoveto{\pgfpoint{384.621613pt}{327.925262pt}}
\pgflineto{\pgfpoint{384.668488pt}{328.096619pt}}
\pgfusepath{stroke}
\pgfpathmoveto{\pgfpoint{384.761932pt}{328.034210pt}}
\pgflineto{\pgfpoint{384.621613pt}{327.925262pt}}
\pgfusepath{stroke}
\pgfpathmoveto{\pgfpoint{384.659302pt}{333.932770pt}}
\pgflineto{\pgfpoint{384.695801pt}{334.093353pt}}
\pgfusepath{stroke}
\pgfpathmoveto{\pgfpoint{384.784851pt}{334.039307pt}}
\pgflineto{\pgfpoint{384.659302pt}{333.932770pt}}
\pgfusepath{stroke}
\pgfpathmoveto{\pgfpoint{384.689728pt}{339.939972pt}}
\pgflineto{\pgfpoint{384.718262pt}{340.090698pt}}
\pgfusepath{stroke}
\pgfpathmoveto{\pgfpoint{384.802979pt}{340.043427pt}}
\pgflineto{\pgfpoint{384.689728pt}{339.939972pt}}
\pgfusepath{stroke}
\pgfpathmoveto{\pgfpoint{384.714600pt}{345.946228pt}}
\pgflineto{\pgfpoint{384.736908pt}{346.088013pt}}
\pgfusepath{stroke}
\pgfpathmoveto{\pgfpoint{384.817505pt}{346.046295pt}}
\pgflineto{\pgfpoint{384.714600pt}{345.946228pt}}
\pgfusepath{stroke}
\pgfpathmoveto{\pgfpoint{384.735168pt}{351.951324pt}}
\pgflineto{\pgfpoint{384.752533pt}{352.085022pt}}
\pgfusepath{stroke}
\pgfpathmoveto{\pgfpoint{384.829285pt}{352.047852pt}}
\pgflineto{\pgfpoint{384.735168pt}{351.951324pt}}
\pgfusepath{stroke}
\pgfpathmoveto{\pgfpoint{384.752350pt}{357.955139pt}}
\pgflineto{\pgfpoint{384.765778pt}{358.081604pt}}
\pgfusepath{stroke}
\pgfpathmoveto{\pgfpoint{384.838989pt}{358.048218pt}}
\pgflineto{\pgfpoint{384.752350pt}{357.955139pt}}
\pgfusepath{stroke}
\pgfpathmoveto{\pgfpoint{384.766907pt}{363.957703pt}}
\pgflineto{\pgfpoint{384.777130pt}{364.077606pt}}
\pgfusepath{stroke}
\pgfpathmoveto{\pgfpoint{384.847046pt}{364.047516pt}}
\pgflineto{\pgfpoint{384.766907pt}{363.957703pt}}
\pgfusepath{stroke}
\pgfpathmoveto{\pgfpoint{384.779327pt}{369.959167pt}}
\pgflineto{\pgfpoint{384.786926pt}{370.073120pt}}
\pgfusepath{stroke}
\pgfpathmoveto{\pgfpoint{384.853790pt}{370.045746pt}}
\pgflineto{\pgfpoint{384.779327pt}{369.959167pt}}
\pgfusepath{stroke}
\pgfpathmoveto{\pgfpoint{390.814423pt}{77.048203pt}}
\pgflineto{\pgfpoint{390.860596pt}{76.993881pt}}
\pgfusepath{stroke}
\pgfpathmoveto{\pgfpoint{390.818787pt}{76.977051pt}}
\pgflineto{\pgfpoint{390.814423pt}{77.048203pt}}
\pgfusepath{stroke}
\pgfpathmoveto{\pgfpoint{390.809875pt}{83.040237pt}}
\pgflineto{\pgfpoint{390.858032pt}{82.985001pt}}
\pgfusepath{stroke}
\pgfpathmoveto{\pgfpoint{390.815247pt}{82.967148pt}}
\pgflineto{\pgfpoint{390.809875pt}{83.040237pt}}
\pgfusepath{stroke}
\pgfpathmoveto{\pgfpoint{390.804932pt}{89.032364pt}}
\pgflineto{\pgfpoint{390.855225pt}{88.976219pt}}
\pgfusepath{stroke}
\pgfpathmoveto{\pgfpoint{390.811462pt}{88.957268pt}}
\pgflineto{\pgfpoint{390.804932pt}{89.032364pt}}
\pgfusepath{stroke}
\pgfpathmoveto{\pgfpoint{390.799561pt}{95.024574pt}}
\pgflineto{\pgfpoint{390.852142pt}{94.967545pt}}
\pgfusepath{stroke}
\pgfpathmoveto{\pgfpoint{390.807404pt}{94.947395pt}}
\pgflineto{\pgfpoint{390.799561pt}{95.024574pt}}
\pgfusepath{stroke}
\pgfpathmoveto{\pgfpoint{390.793793pt}{101.016869pt}}
\pgflineto{\pgfpoint{390.848785pt}{100.958954pt}}
\pgfusepath{stroke}
\pgfpathmoveto{\pgfpoint{390.803040pt}{100.937523pt}}
\pgflineto{\pgfpoint{390.793793pt}{101.016869pt}}
\pgfusepath{stroke}
\pgfpathmoveto{\pgfpoint{390.787537pt}{107.009193pt}}
\pgflineto{\pgfpoint{390.845123pt}{106.950462pt}}
\pgfusepath{stroke}
\pgfpathmoveto{\pgfpoint{390.798370pt}{106.927650pt}}
\pgflineto{\pgfpoint{390.787537pt}{107.009193pt}}
\pgfusepath{stroke}
\pgfpathmoveto{\pgfpoint{390.780762pt}{113.001549pt}}
\pgflineto{\pgfpoint{390.841125pt}{112.942032pt}}
\pgfusepath{stroke}
\pgfpathmoveto{\pgfpoint{390.793335pt}{112.917709pt}}
\pgflineto{\pgfpoint{390.780762pt}{113.001549pt}}
\pgfusepath{stroke}
\pgfpathmoveto{\pgfpoint{390.773407pt}{118.993904pt}}
\pgflineto{\pgfpoint{390.836731pt}{118.933670pt}}
\pgfusepath{stroke}
\pgfpathmoveto{\pgfpoint{390.787933pt}{118.907715pt}}
\pgflineto{\pgfpoint{390.773407pt}{118.993904pt}}
\pgfusepath{stroke}
\pgfpathmoveto{\pgfpoint{390.765472pt}{124.986206pt}}
\pgflineto{\pgfpoint{390.831970pt}{124.925339pt}}
\pgfusepath{stroke}
\pgfpathmoveto{\pgfpoint{390.782166pt}{124.897621pt}}
\pgflineto{\pgfpoint{390.765472pt}{124.986206pt}}
\pgfusepath{stroke}
\pgfpathmoveto{\pgfpoint{390.756897pt}{130.978424pt}}
\pgflineto{\pgfpoint{390.826752pt}{130.917023pt}}
\pgfusepath{stroke}
\pgfpathmoveto{\pgfpoint{390.775940pt}{130.887390pt}}
\pgflineto{\pgfpoint{390.756897pt}{130.978424pt}}
\pgfusepath{stroke}
\pgfpathmoveto{\pgfpoint{390.747681pt}{136.970490pt}}
\pgflineto{\pgfpoint{390.821045pt}{136.908676pt}}
\pgfusepath{stroke}
\pgfpathmoveto{\pgfpoint{390.769287pt}{136.876999pt}}
\pgflineto{\pgfpoint{390.747681pt}{136.970490pt}}
\pgfusepath{stroke}
\pgfpathmoveto{\pgfpoint{390.737732pt}{142.962341pt}}
\pgflineto{\pgfpoint{390.814880pt}{142.900269pt}}
\pgfusepath{stroke}
\pgfpathmoveto{\pgfpoint{390.762207pt}{142.866394pt}}
\pgflineto{\pgfpoint{390.737732pt}{142.962341pt}}
\pgfusepath{stroke}
\pgfpathmoveto{\pgfpoint{390.727081pt}{148.953949pt}}
\pgflineto{\pgfpoint{390.808228pt}{148.891769pt}}
\pgfusepath{stroke}
\pgfpathmoveto{\pgfpoint{390.754700pt}{148.855530pt}}
\pgflineto{\pgfpoint{390.727081pt}{148.953949pt}}
\pgfusepath{stroke}
\pgfpathmoveto{\pgfpoint{390.715698pt}{154.945175pt}}
\pgflineto{\pgfpoint{390.800995pt}{154.883102pt}}
\pgfusepath{stroke}
\pgfpathmoveto{\pgfpoint{390.746704pt}{154.844330pt}}
\pgflineto{\pgfpoint{390.715698pt}{154.945175pt}}
\pgfusepath{stroke}
\pgfpathmoveto{\pgfpoint{390.703552pt}{160.935989pt}}
\pgflineto{\pgfpoint{390.793243pt}{160.874252pt}}
\pgfusepath{stroke}
\pgfpathmoveto{\pgfpoint{390.738281pt}{160.832779pt}}
\pgflineto{\pgfpoint{390.703552pt}{160.935989pt}}
\pgfusepath{stroke}
\pgfpathmoveto{\pgfpoint{390.690674pt}{166.926300pt}}
\pgflineto{\pgfpoint{390.784973pt}{166.865143pt}}
\pgfusepath{stroke}
\pgfpathmoveto{\pgfpoint{390.729431pt}{166.820816pt}}
\pgflineto{\pgfpoint{390.690674pt}{166.926300pt}}
\pgfusepath{stroke}
\pgfpathmoveto{\pgfpoint{390.677124pt}{172.916061pt}}
\pgflineto{\pgfpoint{390.776154pt}{172.855743pt}}
\pgfusepath{stroke}
\pgfpathmoveto{\pgfpoint{390.720154pt}{172.808395pt}}
\pgflineto{\pgfpoint{390.677124pt}{172.916061pt}}
\pgfusepath{stroke}
\pgfpathmoveto{\pgfpoint{390.662933pt}{178.905212pt}}
\pgflineto{\pgfpoint{390.766846pt}{178.846008pt}}
\pgfusepath{stroke}
\pgfpathmoveto{\pgfpoint{390.710510pt}{178.795502pt}}
\pgflineto{\pgfpoint{390.662933pt}{178.905212pt}}
\pgfusepath{stroke}
\pgfpathmoveto{\pgfpoint{390.648132pt}{184.893738pt}}
\pgflineto{\pgfpoint{390.757050pt}{184.835907pt}}
\pgfusepath{stroke}
\pgfpathmoveto{\pgfpoint{390.700562pt}{184.782120pt}}
\pgflineto{\pgfpoint{390.648132pt}{184.893738pt}}
\pgfusepath{stroke}
\pgfpathmoveto{\pgfpoint{390.632751pt}{190.881668pt}}
\pgflineto{\pgfpoint{390.746826pt}{190.825500pt}}
\pgfusepath{stroke}
\pgfpathmoveto{\pgfpoint{390.690308pt}{190.768280pt}}
\pgflineto{\pgfpoint{390.632751pt}{190.881668pt}}
\pgfusepath{stroke}
\pgfpathmoveto{\pgfpoint{390.616821pt}{196.869064pt}}
\pgflineto{\pgfpoint{390.736145pt}{196.814789pt}}
\pgfusepath{stroke}
\pgfpathmoveto{\pgfpoint{390.679718pt}{196.754028pt}}
\pgflineto{\pgfpoint{390.616821pt}{196.869064pt}}
\pgfusepath{stroke}
\pgfpathmoveto{\pgfpoint{390.600281pt}{202.856049pt}}
\pgflineto{\pgfpoint{390.725037pt}{202.803864pt}}
\pgfusepath{stroke}
\pgfpathmoveto{\pgfpoint{390.668762pt}{202.739441pt}}
\pgflineto{\pgfpoint{390.600281pt}{202.856049pt}}
\pgfusepath{stroke}
\pgfpathmoveto{\pgfpoint{390.582947pt}{208.842773pt}}
\pgflineto{\pgfpoint{390.713379pt}{208.792862pt}}
\pgfusepath{stroke}
\pgfpathmoveto{\pgfpoint{390.657349pt}{208.724579pt}}
\pgflineto{\pgfpoint{390.582947pt}{208.842773pt}}
\pgfusepath{stroke}
\pgfpathmoveto{\pgfpoint{390.564636pt}{214.829422pt}}
\pgflineto{\pgfpoint{390.701050pt}{214.781906pt}}
\pgfusepath{stroke}
\pgfpathmoveto{\pgfpoint{390.645264pt}{214.709564pt}}
\pgflineto{\pgfpoint{390.564636pt}{214.829422pt}}
\pgfusepath{stroke}
\pgfpathmoveto{\pgfpoint{390.544739pt}{220.816116pt}}
\pgflineto{\pgfpoint{390.687683pt}{220.771164pt}}
\pgfusepath{stroke}
\pgfpathmoveto{\pgfpoint{390.632111pt}{220.694397pt}}
\pgflineto{\pgfpoint{390.544739pt}{220.816116pt}}
\pgfusepath{stroke}
\pgfpathmoveto{\pgfpoint{390.522552pt}{226.802826pt}}
\pgflineto{\pgfpoint{390.672760pt}{226.760681pt}}
\pgfusepath{stroke}
\pgfpathmoveto{\pgfpoint{390.617432pt}{226.678986pt}}
\pgflineto{\pgfpoint{390.522552pt}{226.802826pt}}
\pgfusepath{stroke}
\pgfpathmoveto{\pgfpoint{390.497070pt}{232.789276pt}}
\pgflineto{\pgfpoint{390.655640pt}{232.750397pt}}
\pgfusepath{stroke}
\pgfpathmoveto{\pgfpoint{390.600616pt}{232.663040pt}}
\pgflineto{\pgfpoint{390.497070pt}{232.789276pt}}
\pgfusepath{stroke}
\pgfpathmoveto{\pgfpoint{390.467041pt}{238.774658pt}}
\pgflineto{\pgfpoint{390.635315pt}{238.739914pt}}
\pgfusepath{stroke}
\pgfpathmoveto{\pgfpoint{390.580841pt}{238.645874pt}}
\pgflineto{\pgfpoint{390.467041pt}{238.774658pt}}
\pgfusepath{stroke}
\pgfpathmoveto{\pgfpoint{390.430939pt}{244.757370pt}}
\pgflineto{\pgfpoint{390.610718pt}{244.728317pt}}
\pgfusepath{stroke}
\pgfpathmoveto{\pgfpoint{390.557312pt}{244.626266pt}}
\pgflineto{\pgfpoint{390.430939pt}{244.757370pt}}
\pgfusepath{stroke}
\pgfpathmoveto{\pgfpoint{390.387634pt}{250.734634pt}}
\pgflineto{\pgfpoint{390.580688pt}{250.713898pt}}
\pgfusepath{stroke}
\pgfpathmoveto{\pgfpoint{390.529633pt}{250.602203pt}}
\pgflineto{\pgfpoint{390.387634pt}{250.734634pt}}
\pgfusepath{stroke}
\pgfpathmoveto{\pgfpoint{390.336853pt}{256.702087pt}}
\pgflineto{\pgfpoint{390.544586pt}{256.693756pt}}
\pgfusepath{stroke}
\pgfpathmoveto{\pgfpoint{390.498047pt}{256.570801pt}}
\pgflineto{\pgfpoint{390.336853pt}{256.702087pt}}
\pgfusepath{stroke}
\pgfpathmoveto{\pgfpoint{390.280731pt}{262.654022pt}}
\pgflineto{\pgfpoint{390.503174pt}{262.663879pt}}
\pgfusepath{stroke}
\pgfpathmoveto{\pgfpoint{390.464600pt}{262.528442pt}}
\pgflineto{\pgfpoint{390.280731pt}{262.654022pt}}
\pgfusepath{stroke}
\pgfpathmoveto{\pgfpoint{390.225464pt}{268.584595pt}}
\pgflineto{\pgfpoint{390.459991pt}{268.619629pt}}
\pgfusepath{stroke}
\pgfpathmoveto{\pgfpoint{390.434113pt}{268.471924pt}}
\pgflineto{\pgfpoint{390.225464pt}{268.584595pt}}
\pgfusepath{stroke}
\pgfpathmoveto{\pgfpoint{390.182251pt}{274.491364pt}}
\pgflineto{\pgfpoint{390.422180pt}{274.558197pt}}
\pgfusepath{stroke}
\pgfpathmoveto{\pgfpoint{390.414276pt}{274.400879pt}}
\pgflineto{\pgfpoint{390.182251pt}{274.491364pt}}
\pgfusepath{stroke}
\pgfpathmoveto{\pgfpoint{390.164276pt}{280.379761pt}}
\pgflineto{\pgfpoint{390.399170pt}{280.481720pt}}
\pgfusepath{stroke}
\pgfpathmoveto{\pgfpoint{390.413361pt}{280.320404pt}}
\pgflineto{\pgfpoint{390.164276pt}{280.379761pt}}
\pgfusepath{stroke}
\pgfpathmoveto{\pgfpoint{390.179993pt}{286.263977pt}}
\pgflineto{\pgfpoint{390.398132pt}{286.398682pt}}
\pgfusepath{stroke}
\pgfpathmoveto{\pgfpoint{390.435333pt}{286.240875pt}}
\pgflineto{\pgfpoint{390.179993pt}{286.263977pt}}
\pgfusepath{stroke}
\pgfpathmoveto{\pgfpoint{390.226868pt}{292.160950pt}}
\pgflineto{\pgfpoint{390.419312pt}{292.320740pt}}
\pgfusepath{stroke}
\pgfpathmoveto{\pgfpoint{390.476685pt}{292.173309pt}}
\pgflineto{\pgfpoint{390.226868pt}{292.160950pt}}
\pgfusepath{stroke}
\pgfpathmoveto{\pgfpoint{390.293427pt}{298.081696pt}}
\pgflineto{\pgfpoint{390.456207pt}{298.256409pt}}
\pgfusepath{stroke}
\pgfpathmoveto{\pgfpoint{390.528503pt}{298.123810pt}}
\pgflineto{\pgfpoint{390.293427pt}{298.081696pt}}
\pgfusepath{stroke}
\pgfpathmoveto{\pgfpoint{390.366272pt}{304.027985pt}}
\pgflineto{\pgfpoint{390.500183pt}{304.208435pt}}
\pgfusepath{stroke}
\pgfpathmoveto{\pgfpoint{390.581665pt}{304.092010pt}}
\pgflineto{\pgfpoint{390.366272pt}{304.027985pt}}
\pgfusepath{stroke}
\pgfpathmoveto{\pgfpoint{390.435913pt}{309.995361pt}}
\pgflineto{\pgfpoint{390.544312pt}{310.174866pt}}
\pgfusepath{stroke}
\pgfpathmoveto{\pgfpoint{390.630341pt}{310.073914pt}}
\pgflineto{\pgfpoint{390.435913pt}{309.995361pt}}
\pgfusepath{stroke}
\pgfpathmoveto{\pgfpoint{390.497620pt}{315.977631pt}}
\pgflineto{\pgfpoint{390.584839pt}{316.152100pt}}
\pgfusepath{stroke}
\pgfpathmoveto{\pgfpoint{390.672089pt}{316.064880pt}}
\pgflineto{\pgfpoint{390.497620pt}{315.977631pt}}
\pgfusepath{stroke}
\pgfpathmoveto{\pgfpoint{390.550232pt}{321.969299pt}}
\pgflineto{\pgfpoint{390.620361pt}{322.136627pt}}
\pgfusepath{stroke}
\pgfpathmoveto{\pgfpoint{390.706726pt}{322.061066pt}}
\pgflineto{\pgfpoint{390.550232pt}{321.969299pt}}
\pgfusepath{stroke}
\pgfpathmoveto{\pgfpoint{390.594177pt}{327.966431pt}}
\pgflineto{\pgfpoint{390.650696pt}{328.125732pt}}
\pgfusepath{stroke}
\pgfpathmoveto{\pgfpoint{390.734985pt}{328.059937pt}}
\pgflineto{\pgfpoint{390.594177pt}{327.966431pt}}
\pgfusepath{stroke}
\pgfpathmoveto{\pgfpoint{390.630737pt}{333.966400pt}}
\pgflineto{\pgfpoint{390.676422pt}{334.117584pt}}
\pgfusepath{stroke}
\pgfpathmoveto{\pgfpoint{390.757996pt}{334.059937pt}}
\pgflineto{\pgfpoint{390.630737pt}{333.966400pt}}
\pgfusepath{stroke}
\pgfpathmoveto{\pgfpoint{390.661133pt}{339.967651pt}}
\pgflineto{\pgfpoint{390.698181pt}{340.110962pt}}
\pgfusepath{stroke}
\pgfpathmoveto{\pgfpoint{390.776764pt}{340.060059pt}}
\pgflineto{\pgfpoint{390.661133pt}{339.967651pt}}
\pgfusepath{stroke}
\pgfpathmoveto{\pgfpoint{390.686523pt}{345.969238pt}}
\pgflineto{\pgfpoint{390.716675pt}{346.105103pt}}
\pgfusepath{stroke}
\pgfpathmoveto{\pgfpoint{390.792175pt}{346.059845pt}}
\pgflineto{\pgfpoint{390.686523pt}{345.969238pt}}
\pgfusepath{stroke}
\pgfpathmoveto{\pgfpoint{390.707886pt}{351.970581pt}}
\pgflineto{\pgfpoint{390.732422pt}{352.099548pt}}
\pgfusepath{stroke}
\pgfpathmoveto{\pgfpoint{390.804901pt}{352.059021pt}}
\pgflineto{\pgfpoint{390.707886pt}{351.970581pt}}
\pgfusepath{stroke}
\pgfpathmoveto{\pgfpoint{390.726013pt}{357.971436pt}}
\pgflineto{\pgfpoint{390.745941pt}{358.093994pt}}
\pgfusepath{stroke}
\pgfpathmoveto{\pgfpoint{390.815521pt}{358.057556pt}}
\pgflineto{\pgfpoint{390.726013pt}{357.971436pt}}
\pgfusepath{stroke}
\pgfpathmoveto{\pgfpoint{390.741516pt}{363.971619pt}}
\pgflineto{\pgfpoint{390.757660pt}{364.088379pt}}
\pgfusepath{stroke}
\pgfpathmoveto{\pgfpoint{390.824463pt}{364.055298pt}}
\pgflineto{\pgfpoint{390.741516pt}{363.971619pt}}
\pgfusepath{stroke}
\pgfpathmoveto{\pgfpoint{390.754822pt}{369.971130pt}}
\pgflineto{\pgfpoint{390.767853pt}{370.082458pt}}
\pgfusepath{stroke}
\pgfpathmoveto{\pgfpoint{390.832031pt}{370.052368pt}}
\pgflineto{\pgfpoint{390.754822pt}{369.971130pt}}
\pgfusepath{stroke}
\pgfpathmoveto{\pgfpoint{396.798126pt}{77.043747pt}}
\pgflineto{\pgfpoint{396.845154pt}{76.991379pt}}
\pgfusepath{stroke}
\pgfpathmoveto{\pgfpoint{396.804321pt}{76.973633pt}}
\pgflineto{\pgfpoint{396.798126pt}{77.043747pt}}
\pgfusepath{stroke}
\pgfpathmoveto{\pgfpoint{396.793457pt}{83.035431pt}}
\pgflineto{\pgfpoint{396.842499pt}{82.982269pt}}
\pgfusepath{stroke}
\pgfpathmoveto{\pgfpoint{396.800781pt}{82.963486pt}}
\pgflineto{\pgfpoint{396.793457pt}{83.035431pt}}
\pgfusepath{stroke}
\pgfpathmoveto{\pgfpoint{396.788452pt}{89.027184pt}}
\pgflineto{\pgfpoint{396.839569pt}{88.973244pt}}
\pgfusepath{stroke}
\pgfpathmoveto{\pgfpoint{396.796997pt}{88.953354pt}}
\pgflineto{\pgfpoint{396.788452pt}{89.027184pt}}
\pgfusepath{stroke}
\pgfpathmoveto{\pgfpoint{396.783051pt}{95.018990pt}}
\pgflineto{\pgfpoint{396.836426pt}{94.964287pt}}
\pgfusepath{stroke}
\pgfpathmoveto{\pgfpoint{396.792938pt}{94.943192pt}}
\pgflineto{\pgfpoint{396.783051pt}{95.018990pt}}
\pgfusepath{stroke}
\pgfpathmoveto{\pgfpoint{396.777222pt}{101.010826pt}}
\pgflineto{\pgfpoint{396.833008pt}{100.955406pt}}
\pgfusepath{stroke}
\pgfpathmoveto{\pgfpoint{396.788605pt}{100.933029pt}}
\pgflineto{\pgfpoint{396.777222pt}{101.010826pt}}
\pgfusepath{stroke}
\pgfpathmoveto{\pgfpoint{396.770935pt}{107.002670pt}}
\pgflineto{\pgfpoint{396.829254pt}{106.946602pt}}
\pgfusepath{stroke}
\pgfpathmoveto{\pgfpoint{396.783966pt}{106.922821pt}}
\pgflineto{\pgfpoint{396.770935pt}{107.002670pt}}
\pgfusepath{stroke}
\pgfpathmoveto{\pgfpoint{396.764160pt}{112.994507pt}}
\pgflineto{\pgfpoint{396.825195pt}{112.937820pt}}
\pgfusepath{stroke}
\pgfpathmoveto{\pgfpoint{396.778992pt}{112.912537pt}}
\pgflineto{\pgfpoint{396.764160pt}{112.994507pt}}
\pgfusepath{stroke}
\pgfpathmoveto{\pgfpoint{396.756897pt}{118.986298pt}}
\pgflineto{\pgfpoint{396.820801pt}{118.929077pt}}
\pgfusepath{stroke}
\pgfpathmoveto{\pgfpoint{396.773682pt}{118.902176pt}}
\pgflineto{\pgfpoint{396.756897pt}{118.986298pt}}
\pgfusepath{stroke}
\pgfpathmoveto{\pgfpoint{396.749023pt}{124.978012pt}}
\pgflineto{\pgfpoint{396.816010pt}{124.920341pt}}
\pgfusepath{stroke}
\pgfpathmoveto{\pgfpoint{396.768005pt}{124.891701pt}}
\pgflineto{\pgfpoint{396.749023pt}{124.978012pt}}
\pgfusepath{stroke}
\pgfpathmoveto{\pgfpoint{396.740601pt}{130.969604pt}}
\pgflineto{\pgfpoint{396.810791pt}{130.911591pt}}
\pgfusepath{stroke}
\pgfpathmoveto{\pgfpoint{396.761963pt}{130.881073pt}}
\pgflineto{\pgfpoint{396.740601pt}{130.969604pt}}
\pgfusepath{stroke}
\pgfpathmoveto{\pgfpoint{396.731567pt}{136.960999pt}}
\pgflineto{\pgfpoint{396.805176pt}{136.902786pt}}
\pgfusepath{stroke}
\pgfpathmoveto{\pgfpoint{396.755524pt}{136.870255pt}}
\pgflineto{\pgfpoint{396.731567pt}{136.960999pt}}
\pgfusepath{stroke}
\pgfpathmoveto{\pgfpoint{396.721863pt}{142.952194pt}}
\pgflineto{\pgfpoint{396.799103pt}{142.893906pt}}
\pgfusepath{stroke}
\pgfpathmoveto{\pgfpoint{396.748688pt}{142.859222pt}}
\pgflineto{\pgfpoint{396.721863pt}{142.952194pt}}
\pgfusepath{stroke}
\pgfpathmoveto{\pgfpoint{396.711517pt}{148.943085pt}}
\pgflineto{\pgfpoint{396.792542pt}{148.884903pt}}
\pgfusepath{stroke}
\pgfpathmoveto{\pgfpoint{396.741455pt}{148.847900pt}}
\pgflineto{\pgfpoint{396.711517pt}{148.943085pt}}
\pgfusepath{stroke}
\pgfpathmoveto{\pgfpoint{396.700500pt}{154.933624pt}}
\pgflineto{\pgfpoint{396.785522pt}{154.875732pt}}
\pgfusepath{stroke}
\pgfpathmoveto{\pgfpoint{396.733765pt}{154.836304pt}}
\pgflineto{\pgfpoint{396.700500pt}{154.933624pt}}
\pgfusepath{stroke}
\pgfpathmoveto{\pgfpoint{396.688843pt}{160.923737pt}}
\pgflineto{\pgfpoint{396.777985pt}{160.866348pt}}
\pgfusepath{stroke}
\pgfpathmoveto{\pgfpoint{396.725739pt}{160.824326pt}}
\pgflineto{\pgfpoint{396.688843pt}{160.923737pt}}
\pgfusepath{stroke}
\pgfpathmoveto{\pgfpoint{396.676483pt}{166.913361pt}}
\pgflineto{\pgfpoint{396.769989pt}{166.856735pt}}
\pgfusepath{stroke}
\pgfpathmoveto{\pgfpoint{396.717285pt}{166.811966pt}}
\pgflineto{\pgfpoint{396.676483pt}{166.913361pt}}
\pgfusepath{stroke}
\pgfpathmoveto{\pgfpoint{396.663513pt}{172.902466pt}}
\pgflineto{\pgfpoint{396.761475pt}{172.846832pt}}
\pgfusepath{stroke}
\pgfpathmoveto{\pgfpoint{396.708496pt}{172.799179pt}}
\pgflineto{\pgfpoint{396.663513pt}{172.902466pt}}
\pgfusepath{stroke}
\pgfpathmoveto{\pgfpoint{396.649902pt}{178.890991pt}}
\pgflineto{\pgfpoint{396.752502pt}{178.836609pt}}
\pgfusepath{stroke}
\pgfpathmoveto{\pgfpoint{396.699341pt}{178.785934pt}}
\pgflineto{\pgfpoint{396.649902pt}{178.890991pt}}
\pgfusepath{stroke}
\pgfpathmoveto{\pgfpoint{396.635681pt}{184.878937pt}}
\pgflineto{\pgfpoint{396.743042pt}{184.826065pt}}
\pgfusepath{stroke}
\pgfpathmoveto{\pgfpoint{396.689850pt}{184.772232pt}}
\pgflineto{\pgfpoint{396.635681pt}{184.878937pt}}
\pgfusepath{stroke}
\pgfpathmoveto{\pgfpoint{396.620880pt}{190.866272pt}}
\pgflineto{\pgfpoint{396.733124pt}{190.815186pt}}
\pgfusepath{stroke}
\pgfpathmoveto{\pgfpoint{396.680023pt}{190.758072pt}}
\pgflineto{\pgfpoint{396.620880pt}{190.866272pt}}
\pgfusepath{stroke}
\pgfpathmoveto{\pgfpoint{396.605469pt}{196.853058pt}}
\pgflineto{\pgfpoint{396.722748pt}{196.804016pt}}
\pgfusepath{stroke}
\pgfpathmoveto{\pgfpoint{396.669861pt}{196.743469pt}}
\pgflineto{\pgfpoint{396.605469pt}{196.853058pt}}
\pgfusepath{stroke}
\pgfpathmoveto{\pgfpoint{396.589355pt}{202.839325pt}}
\pgflineto{\pgfpoint{396.711792pt}{202.792572pt}}
\pgfusepath{stroke}
\pgfpathmoveto{\pgfpoint{396.659241pt}{202.728439pt}}
\pgflineto{\pgfpoint{396.589355pt}{202.839325pt}}
\pgfusepath{stroke}
\pgfpathmoveto{\pgfpoint{396.572327pt}{208.825119pt}}
\pgflineto{\pgfpoint{396.700256pt}{208.780914pt}}
\pgfusepath{stroke}
\pgfpathmoveto{\pgfpoint{396.648163pt}{208.712997pt}}
\pgflineto{\pgfpoint{396.572327pt}{208.825119pt}}
\pgfusepath{stroke}
\pgfpathmoveto{\pgfpoint{396.554260pt}{214.810486pt}}
\pgflineto{\pgfpoint{396.687927pt}{214.769073pt}}
\pgfusepath{stroke}
\pgfpathmoveto{\pgfpoint{396.636353pt}{214.697144pt}}
\pgflineto{\pgfpoint{396.554260pt}{214.810486pt}}
\pgfusepath{stroke}
\pgfpathmoveto{\pgfpoint{396.534607pt}{220.795349pt}}
\pgflineto{\pgfpoint{396.674530pt}{220.757080pt}}
\pgfusepath{stroke}
\pgfpathmoveto{\pgfpoint{396.623596pt}{220.680786pt}}
\pgflineto{\pgfpoint{396.534607pt}{220.795349pt}}
\pgfusepath{stroke}
\pgfpathmoveto{\pgfpoint{396.512939pt}{226.779480pt}}
\pgflineto{\pgfpoint{396.659698pt}{226.744843pt}}
\pgfusepath{stroke}
\pgfpathmoveto{\pgfpoint{396.609558pt}{226.663727pt}}
\pgflineto{\pgfpoint{396.512939pt}{226.779480pt}}
\pgfusepath{stroke}
\pgfpathmoveto{\pgfpoint{396.488586pt}{232.762436pt}}
\pgflineto{\pgfpoint{396.642914pt}{232.732086pt}}
\pgfusepath{stroke}
\pgfpathmoveto{\pgfpoint{396.593842pt}{232.645569pt}}
\pgflineto{\pgfpoint{396.488586pt}{232.762436pt}}
\pgfusepath{stroke}
\pgfpathmoveto{\pgfpoint{396.460815pt}{238.743301pt}}
\pgflineto{\pgfpoint{396.623627pt}{238.718338pt}}
\pgfusepath{stroke}
\pgfpathmoveto{\pgfpoint{396.576080pt}{238.625641pt}}
\pgflineto{\pgfpoint{396.460815pt}{238.743301pt}}
\pgfusepath{stroke}
\pgfpathmoveto{\pgfpoint{396.428986pt}{244.720596pt}}
\pgflineto{\pgfpoint{396.601257pt}{244.702667pt}}
\pgfusepath{stroke}
\pgfpathmoveto{\pgfpoint{396.556030pt}{244.602890pt}}
\pgflineto{\pgfpoint{396.428986pt}{244.720596pt}}
\pgfusepath{stroke}
\pgfpathmoveto{\pgfpoint{396.392975pt}{250.692169pt}}
\pgflineto{\pgfpoint{396.575439pt}{250.683640pt}}
\pgfusepath{stroke}
\pgfpathmoveto{\pgfpoint{396.533813pt}{250.575867pt}}
\pgflineto{\pgfpoint{396.392975pt}{250.692169pt}}
\pgfusepath{stroke}
\pgfpathmoveto{\pgfpoint{396.353577pt}{256.655151pt}}
\pgflineto{\pgfpoint{396.546417pt}{256.659241pt}}
\pgfusepath{stroke}
\pgfpathmoveto{\pgfpoint{396.510315pt}{256.542725pt}}
\pgflineto{\pgfpoint{396.353577pt}{256.655151pt}}
\pgfusepath{stroke}
\pgfpathmoveto{\pgfpoint{396.313263pt}{262.606354pt}}
\pgflineto{\pgfpoint{396.515503pt}{262.627106pt}}
\pgfusepath{stroke}
\pgfpathmoveto{\pgfpoint{396.487488pt}{262.501617pt}}
\pgflineto{\pgfpoint{396.313263pt}{262.606354pt}}
\pgfusepath{stroke}
\pgfpathmoveto{\pgfpoint{396.276733pt}{268.543427pt}}
\pgflineto{\pgfpoint{396.485474pt}{268.585144pt}}
\pgfusepath{stroke}
\pgfpathmoveto{\pgfpoint{396.468750pt}{268.451538pt}}
\pgflineto{\pgfpoint{396.276733pt}{268.543427pt}}
\pgfusepath{stroke}
\pgfpathmoveto{\pgfpoint{396.250580pt}{274.466492pt}}
\pgflineto{\pgfpoint{396.460876pt}{274.532684pt}}
\pgfusepath{stroke}
\pgfpathmoveto{\pgfpoint{396.458527pt}{274.393250pt}}
\pgflineto{\pgfpoint{396.250580pt}{274.466492pt}}
\pgfusepath{stroke}
\pgfpathmoveto{\pgfpoint{396.241638pt}{280.379730pt}}
\pgflineto{\pgfpoint{396.446655pt}{280.471741pt}}
\pgfusepath{stroke}
\pgfpathmoveto{\pgfpoint{396.460815pt}{280.330353pt}}
\pgflineto{\pgfpoint{396.241638pt}{280.379730pt}}
\pgfusepath{stroke}
\pgfpathmoveto{\pgfpoint{396.253723pt}{286.291138pt}}
\pgflineto{\pgfpoint{396.446198pt}{286.407257pt}}
\pgfusepath{stroke}
\pgfpathmoveto{\pgfpoint{396.477386pt}{286.268524pt}}
\pgflineto{\pgfpoint{396.253723pt}{286.291138pt}}
\pgfusepath{stroke}
\pgfpathmoveto{\pgfpoint{396.285645pt}{292.209747pt}}
\pgflineto{\pgfpoint{396.459808pt}{292.345367pt}}
\pgfusepath{stroke}
\pgfpathmoveto{\pgfpoint{396.506348pt}{292.213745pt}}
\pgflineto{\pgfpoint{396.285645pt}{292.209747pt}}
\pgfusepath{stroke}
\pgfpathmoveto{\pgfpoint{396.331818pt}{298.142181pt}}
\pgflineto{\pgfpoint{396.484436pt}{298.291107pt}}
\pgfusepath{stroke}
\pgfpathmoveto{\pgfpoint{396.543274pt}{298.169739pt}}
\pgflineto{\pgfpoint{396.331818pt}{298.142181pt}}
\pgfusepath{stroke}
\pgfpathmoveto{\pgfpoint{396.384949pt}{304.090820pt}}
\pgflineto{\pgfpoint{396.515503pt}{304.246887pt}}
\pgfusepath{stroke}
\pgfpathmoveto{\pgfpoint{396.583038pt}{304.137329pt}}
\pgflineto{\pgfpoint{396.384949pt}{304.090820pt}}
\pgfusepath{stroke}
\pgfpathmoveto{\pgfpoint{396.438751pt}{310.054504pt}}
\pgflineto{\pgfpoint{396.548706pt}{310.212616pt}}
\pgfusepath{stroke}
\pgfpathmoveto{\pgfpoint{396.621582pt}{310.115021pt}}
\pgflineto{\pgfpoint{396.438751pt}{310.054504pt}}
\pgfusepath{stroke}
\pgfpathmoveto{\pgfpoint{396.489288pt}{316.030212pt}}
\pgflineto{\pgfpoint{396.581055pt}{316.186707pt}}
\pgfusepath{stroke}
\pgfpathmoveto{\pgfpoint{396.656616pt}{316.100342pt}}
\pgflineto{\pgfpoint{396.489288pt}{316.030212pt}}
\pgfusepath{stroke}
\pgfpathmoveto{\pgfpoint{396.534607pt}{322.014618pt}}
\pgflineto{\pgfpoint{396.610901pt}{322.167206pt}}
\pgfusepath{stroke}
\pgfpathmoveto{\pgfpoint{396.687195pt}{322.090912pt}}
\pgflineto{\pgfpoint{396.534607pt}{322.014618pt}}
\pgfusepath{stroke}
\pgfpathmoveto{\pgfpoint{396.574188pt}{328.004883pt}}
\pgflineto{\pgfpoint{396.637573pt}{328.152252pt}}
\pgfusepath{stroke}
\pgfpathmoveto{\pgfpoint{396.713318pt}{328.084747pt}}
\pgflineto{\pgfpoint{396.574188pt}{328.004883pt}}
\pgfusepath{stroke}
\pgfpathmoveto{\pgfpoint{396.608276pt}{333.998871pt}}
\pgflineto{\pgfpoint{396.660980pt}{334.140381pt}}
\pgfusepath{stroke}
\pgfpathmoveto{\pgfpoint{396.735352pt}{334.080444pt}}
\pgflineto{\pgfpoint{396.608276pt}{333.998871pt}}
\pgfusepath{stroke}
\pgfpathmoveto{\pgfpoint{396.637482pt}{339.995056pt}}
\pgflineto{\pgfpoint{396.681396pt}{340.130493pt}}
\pgfusepath{stroke}
\pgfpathmoveto{\pgfpoint{396.753876pt}{340.077057pt}}
\pgflineto{\pgfpoint{396.637482pt}{339.995056pt}}
\pgfusepath{stroke}
\pgfpathmoveto{\pgfpoint{396.662415pt}{345.992432pt}}
\pgflineto{\pgfpoint{396.699097pt}{346.121887pt}}
\pgfusepath{stroke}
\pgfpathmoveto{\pgfpoint{396.769440pt}{346.074005pt}}
\pgflineto{\pgfpoint{396.662415pt}{345.992432pt}}
\pgfusepath{stroke}
\pgfpathmoveto{\pgfpoint{396.683838pt}{351.990356pt}}
\pgflineto{\pgfpoint{396.714508pt}{352.114044pt}}
\pgfusepath{stroke}
\pgfpathmoveto{\pgfpoint{396.782593pt}{352.070923pt}}
\pgflineto{\pgfpoint{396.683838pt}{351.990356pt}}
\pgfusepath{stroke}
\pgfpathmoveto{\pgfpoint{396.702271pt}{357.988373pt}}
\pgflineto{\pgfpoint{396.727936pt}{358.106567pt}}
\pgfusepath{stroke}
\pgfpathmoveto{\pgfpoint{396.793732pt}{358.067566pt}}
\pgflineto{\pgfpoint{396.702271pt}{357.988373pt}}
\pgfusepath{stroke}
\pgfpathmoveto{\pgfpoint{396.718201pt}{363.986237pt}}
\pgflineto{\pgfpoint{396.739685pt}{364.099335pt}}
\pgfusepath{stroke}
\pgfpathmoveto{\pgfpoint{396.803223pt}{364.063843pt}}
\pgflineto{\pgfpoint{396.718201pt}{363.986237pt}}
\pgfusepath{stroke}
\pgfpathmoveto{\pgfpoint{396.732056pt}{369.983826pt}}
\pgflineto{\pgfpoint{396.750000pt}{370.092102pt}}
\pgfusepath{stroke}
\pgfpathmoveto{\pgfpoint{396.811401pt}{370.059662pt}}
\pgflineto{\pgfpoint{396.732056pt}{369.983826pt}}
\pgfusepath{stroke}
\pgfpathmoveto{\pgfpoint{402.782166pt}{77.039230pt}}
\pgflineto{\pgfpoint{402.829926pt}{76.988770pt}}
\pgfusepath{stroke}
\pgfpathmoveto{\pgfpoint{402.790100pt}{76.970215pt}}
\pgflineto{\pgfpoint{402.782166pt}{77.039230pt}}
\pgfusepath{stroke}
\pgfpathmoveto{\pgfpoint{402.777435pt}{83.030579pt}}
\pgflineto{\pgfpoint{402.827179pt}{82.979446pt}}
\pgfusepath{stroke}
\pgfpathmoveto{\pgfpoint{402.786560pt}{82.959839pt}}
\pgflineto{\pgfpoint{402.777435pt}{83.030579pt}}
\pgfusepath{stroke}
\pgfpathmoveto{\pgfpoint{402.772400pt}{89.021957pt}}
\pgflineto{\pgfpoint{402.824219pt}{88.970177pt}}
\pgfusepath{stroke}
\pgfpathmoveto{\pgfpoint{402.782776pt}{88.949448pt}}
\pgflineto{\pgfpoint{402.772400pt}{89.021957pt}}
\pgfusepath{stroke}
\pgfpathmoveto{\pgfpoint{402.766937pt}{95.013359pt}}
\pgflineto{\pgfpoint{402.820984pt}{94.960960pt}}
\pgfusepath{stroke}
\pgfpathmoveto{\pgfpoint{402.778748pt}{94.939026pt}}
\pgflineto{\pgfpoint{402.766937pt}{95.013359pt}}
\pgfusepath{stroke}
\pgfpathmoveto{\pgfpoint{402.761108pt}{101.004776pt}}
\pgflineto{\pgfpoint{402.817505pt}{100.951813pt}}
\pgfusepath{stroke}
\pgfpathmoveto{\pgfpoint{402.774445pt}{100.928558pt}}
\pgflineto{\pgfpoint{402.761108pt}{101.004776pt}}
\pgfusepath{stroke}
\pgfpathmoveto{\pgfpoint{402.754822pt}{106.996170pt}}
\pgflineto{\pgfpoint{402.813721pt}{106.942688pt}}
\pgfusepath{stroke}
\pgfpathmoveto{\pgfpoint{402.769867pt}{106.918045pt}}
\pgflineto{\pgfpoint{402.754822pt}{106.996170pt}}
\pgfusepath{stroke}
\pgfpathmoveto{\pgfpoint{402.748108pt}{112.987518pt}}
\pgflineto{\pgfpoint{402.809631pt}{112.933586pt}}
\pgfusepath{stroke}
\pgfpathmoveto{\pgfpoint{402.764984pt}{112.907448pt}}
\pgflineto{\pgfpoint{402.748108pt}{112.987518pt}}
\pgfusepath{stroke}
\pgfpathmoveto{\pgfpoint{402.740906pt}{118.978783pt}}
\pgflineto{\pgfpoint{402.805237pt}{118.924484pt}}
\pgfusepath{stroke}
\pgfpathmoveto{\pgfpoint{402.759766pt}{118.896751pt}}
\pgflineto{\pgfpoint{402.740906pt}{118.978783pt}}
\pgfusepath{stroke}
\pgfpathmoveto{\pgfpoint{402.733154pt}{124.969940pt}}
\pgflineto{\pgfpoint{402.800446pt}{124.915359pt}}
\pgfusepath{stroke}
\pgfpathmoveto{\pgfpoint{402.754242pt}{124.885910pt}}
\pgflineto{\pgfpoint{402.733154pt}{124.969940pt}}
\pgfusepath{stroke}
\pgfpathmoveto{\pgfpoint{402.724915pt}{130.960953pt}}
\pgflineto{\pgfpoint{402.795288pt}{130.906204pt}}
\pgfusepath{stroke}
\pgfpathmoveto{\pgfpoint{402.748352pt}{130.874908pt}}
\pgflineto{\pgfpoint{402.724915pt}{130.960953pt}}
\pgfusepath{stroke}
\pgfpathmoveto{\pgfpoint{402.716064pt}{136.951767pt}}
\pgflineto{\pgfpoint{402.789734pt}{136.896973pt}}
\pgfusepath{stroke}
\pgfpathmoveto{\pgfpoint{402.742126pt}{136.863739pt}}
\pgflineto{\pgfpoint{402.716064pt}{136.951767pt}}
\pgfusepath{stroke}
\pgfpathmoveto{\pgfpoint{402.706665pt}{142.942322pt}}
\pgflineto{\pgfpoint{402.783783pt}{142.887665pt}}
\pgfusepath{stroke}
\pgfpathmoveto{\pgfpoint{402.735565pt}{142.852310pt}}
\pgflineto{\pgfpoint{402.706665pt}{142.942322pt}}
\pgfusepath{stroke}
\pgfpathmoveto{\pgfpoint{402.696625pt}{148.932587pt}}
\pgflineto{\pgfpoint{402.777374pt}{148.878204pt}}
\pgfusepath{stroke}
\pgfpathmoveto{\pgfpoint{402.728577pt}{148.840607pt}}
\pgflineto{\pgfpoint{402.696625pt}{148.932587pt}}
\pgfusepath{stroke}
\pgfpathmoveto{\pgfpoint{402.686005pt}{154.922485pt}}
\pgflineto{\pgfpoint{402.770538pt}{154.868561pt}}
\pgfusepath{stroke}
\pgfpathmoveto{\pgfpoint{402.721283pt}{154.828613pt}}
\pgflineto{\pgfpoint{402.686005pt}{154.922485pt}}
\pgfusepath{stroke}
\pgfpathmoveto{\pgfpoint{402.674774pt}{160.911957pt}}
\pgflineto{\pgfpoint{402.763245pt}{160.858704pt}}
\pgfusepath{stroke}
\pgfpathmoveto{\pgfpoint{402.713593pt}{160.816284pt}}
\pgflineto{\pgfpoint{402.674774pt}{160.911957pt}}
\pgfusepath{stroke}
\pgfpathmoveto{\pgfpoint{402.662933pt}{166.900970pt}}
\pgflineto{\pgfpoint{402.755493pt}{166.848618pt}}
\pgfusepath{stroke}
\pgfpathmoveto{\pgfpoint{402.705566pt}{166.803558pt}}
\pgflineto{\pgfpoint{402.662933pt}{166.900970pt}}
\pgfusepath{stroke}
\pgfpathmoveto{\pgfpoint{402.650513pt}{172.889465pt}}
\pgflineto{\pgfpoint{402.747284pt}{172.838242pt}}
\pgfusepath{stroke}
\pgfpathmoveto{\pgfpoint{402.697205pt}{172.790421pt}}
\pgflineto{\pgfpoint{402.650513pt}{172.889465pt}}
\pgfusepath{stroke}
\pgfpathmoveto{\pgfpoint{402.637482pt}{178.877380pt}}
\pgflineto{\pgfpoint{402.738617pt}{178.827560pt}}
\pgfusepath{stroke}
\pgfpathmoveto{\pgfpoint{402.688477pt}{178.776840pt}}
\pgflineto{\pgfpoint{402.637482pt}{178.877380pt}}
\pgfusepath{stroke}
\pgfpathmoveto{\pgfpoint{402.623871pt}{184.864731pt}}
\pgflineto{\pgfpoint{402.729492pt}{184.816544pt}}
\pgfusepath{stroke}
\pgfpathmoveto{\pgfpoint{402.679474pt}{184.762802pt}}
\pgflineto{\pgfpoint{402.623871pt}{184.864731pt}}
\pgfusepath{stroke}
\pgfpathmoveto{\pgfpoint{402.609650pt}{190.851456pt}}
\pgflineto{\pgfpoint{402.719910pt}{190.805206pt}}
\pgfusepath{stroke}
\pgfpathmoveto{\pgfpoint{402.670105pt}{190.748306pt}}
\pgflineto{\pgfpoint{402.609650pt}{190.851456pt}}
\pgfusepath{stroke}
\pgfpathmoveto{\pgfpoint{402.594818pt}{196.837570pt}}
\pgflineto{\pgfpoint{402.709839pt}{196.793518pt}}
\pgfusepath{stroke}
\pgfpathmoveto{\pgfpoint{402.660400pt}{196.733322pt}}
\pgflineto{\pgfpoint{402.594818pt}{196.837570pt}}
\pgfusepath{stroke}
\pgfpathmoveto{\pgfpoint{402.579224pt}{202.823059pt}}
\pgflineto{\pgfpoint{402.699188pt}{202.781494pt}}
\pgfusepath{stroke}
\pgfpathmoveto{\pgfpoint{402.650269pt}{202.717834pt}}
\pgflineto{\pgfpoint{402.579224pt}{202.823059pt}}
\pgfusepath{stroke}
\pgfpathmoveto{\pgfpoint{402.562836pt}{208.807907pt}}
\pgflineto{\pgfpoint{402.687927pt}{208.769119pt}}
\pgfusepath{stroke}
\pgfpathmoveto{\pgfpoint{402.639648pt}{208.701813pt}}
\pgflineto{\pgfpoint{402.562836pt}{208.807907pt}}
\pgfusepath{stroke}
\pgfpathmoveto{\pgfpoint{402.545349pt}{214.792007pt}}
\pgflineto{\pgfpoint{402.675873pt}{214.756393pt}}
\pgfusepath{stroke}
\pgfpathmoveto{\pgfpoint{402.628387pt}{214.685196pt}}
\pgflineto{\pgfpoint{402.545349pt}{214.792007pt}}
\pgfusepath{stroke}
\pgfpathmoveto{\pgfpoint{402.526550pt}{220.775238pt}}
\pgflineto{\pgfpoint{402.662842pt}{220.743210pt}}
\pgfusepath{stroke}
\pgfpathmoveto{\pgfpoint{402.616364pt}{220.667847pt}}
\pgflineto{\pgfpoint{402.526550pt}{220.775238pt}}
\pgfusepath{stroke}
\pgfpathmoveto{\pgfpoint{402.506042pt}{226.757278pt}}
\pgflineto{\pgfpoint{402.648529pt}{226.729431pt}}
\pgfusepath{stroke}
\pgfpathmoveto{\pgfpoint{402.603333pt}{226.649506pt}}
\pgflineto{\pgfpoint{402.506042pt}{226.757278pt}}
\pgfusepath{stroke}
\pgfpathmoveto{\pgfpoint{402.483521pt}{232.737595pt}}
\pgflineto{\pgfpoint{402.632690pt}{232.714737pt}}
\pgfusepath{stroke}
\pgfpathmoveto{\pgfpoint{402.589172pt}{232.629807pt}}
\pgflineto{\pgfpoint{402.483521pt}{232.737595pt}}
\pgfusepath{stroke}
\pgfpathmoveto{\pgfpoint{402.458679pt}{238.715378pt}}
\pgflineto{\pgfpoint{402.615021pt}{238.698608pt}}
\pgfusepath{stroke}
\pgfpathmoveto{\pgfpoint{402.573669pt}{238.608154pt}}
\pgflineto{\pgfpoint{402.458679pt}{238.715378pt}}
\pgfusepath{stroke}
\pgfpathmoveto{\pgfpoint{402.431396pt}{244.689438pt}}
\pgflineto{\pgfpoint{402.595306pt}{244.680267pt}}
\pgfusepath{stroke}
\pgfpathmoveto{\pgfpoint{402.557007pt}{244.583755pt}}
\pgflineto{\pgfpoint{402.431396pt}{244.689438pt}}
\pgfusepath{stroke}
\pgfpathmoveto{\pgfpoint{402.401978pt}{250.658264pt}}
\pgflineto{\pgfpoint{402.573608pt}{250.658661pt}}
\pgfusepath{stroke}
\pgfpathmoveto{\pgfpoint{402.539520pt}{250.555618pt}}
\pgflineto{\pgfpoint{402.401978pt}{250.658264pt}}
\pgfusepath{stroke}
\pgfpathmoveto{\pgfpoint{402.371613pt}{256.620056pt}}
\pgflineto{\pgfpoint{402.550476pt}{256.632477pt}}
\pgfusepath{stroke}
\pgfpathmoveto{\pgfpoint{402.522156pt}{256.522705pt}}
\pgflineto{\pgfpoint{402.371613pt}{256.620056pt}}
\pgfusepath{stroke}
\pgfpathmoveto{\pgfpoint{402.342407pt}{262.573242pt}}
\pgflineto{\pgfpoint{402.527161pt}{262.600433pt}}
\pgfusepath{stroke}
\pgfpathmoveto{\pgfpoint{402.506531pt}{262.484131pt}}
\pgflineto{\pgfpoint{402.342407pt}{262.573242pt}}
\pgfusepath{stroke}
\pgfpathmoveto{\pgfpoint{402.317627pt}{268.516998pt}}
\pgflineto{\pgfpoint{402.505737pt}{268.561615pt}}
\pgfusepath{stroke}
\pgfpathmoveto{\pgfpoint{402.494873pt}{268.439819pt}}
\pgflineto{\pgfpoint{402.317627pt}{268.516998pt}}
\pgfusepath{stroke}
\pgfpathmoveto{\pgfpoint{402.301270pt}{274.452209pt}}
\pgflineto{\pgfpoint{402.488983pt}{274.516144pt}}
\pgfusepath{stroke}
\pgfpathmoveto{\pgfpoint{402.489807pt}{274.390717pt}}
\pgflineto{\pgfpoint{402.301270pt}{274.452209pt}}
\pgfusepath{stroke}
\pgfpathmoveto{\pgfpoint{402.296997pt}{280.381805pt}}
\pgflineto{\pgfpoint{402.479675pt}{280.465607pt}}
\pgfusepath{stroke}
\pgfpathmoveto{\pgfpoint{402.493439pt}{280.339233pt}}
\pgflineto{\pgfpoint{402.296997pt}{280.381805pt}}
\pgfusepath{stroke}
\pgfpathmoveto{\pgfpoint{402.306763pt}{286.310608pt}}
\pgflineto{\pgfpoint{402.479675pt}{286.412994pt}}
\pgfusepath{stroke}
\pgfpathmoveto{\pgfpoint{402.506500pt}{286.288788pt}}
\pgflineto{\pgfpoint{402.306763pt}{286.310608pt}}
\pgfusepath{stroke}
\pgfpathmoveto{\pgfpoint{402.329926pt}{292.243896pt}}
\pgflineto{\pgfpoint{402.489044pt}{292.361877pt}}
\pgfusepath{stroke}
\pgfpathmoveto{\pgfpoint{402.528015pt}{292.242798pt}}
\pgflineto{\pgfpoint{402.329926pt}{292.243896pt}}
\pgfusepath{stroke}
\pgfpathmoveto{\pgfpoint{402.363464pt}{298.185791pt}}
\pgflineto{\pgfpoint{402.506317pt}{298.315308pt}}
\pgfusepath{stroke}
\pgfpathmoveto{\pgfpoint{402.555450pt}{298.203735pt}}
\pgflineto{\pgfpoint{402.363464pt}{298.185791pt}}
\pgfusepath{stroke}
\pgfpathmoveto{\pgfpoint{402.403259pt}{304.138367pt}}
\pgflineto{\pgfpoint{402.528931pt}{304.275208pt}}
\pgfusepath{stroke}
\pgfpathmoveto{\pgfpoint{402.585876pt}{304.172424pt}}
\pgflineto{\pgfpoint{402.403259pt}{304.138367pt}}
\pgfusepath{stroke}
\pgfpathmoveto{\pgfpoint{402.445251pt}{310.101624pt}}
\pgflineto{\pgfpoint{402.554199pt}{310.241943pt}}
\pgfusepath{stroke}
\pgfpathmoveto{\pgfpoint{402.616608pt}{310.148499pt}}
\pgflineto{\pgfpoint{402.445251pt}{310.101624pt}}
\pgfusepath{stroke}
\pgfpathmoveto{\pgfpoint{402.486420pt}{316.074188pt}}
\pgflineto{\pgfpoint{402.579956pt}{316.214996pt}}
\pgfusepath{stroke}
\pgfpathmoveto{\pgfpoint{402.645752pt}{316.130707pt}}
\pgflineto{\pgfpoint{402.486420pt}{316.074188pt}}
\pgfusepath{stroke}
\pgfpathmoveto{\pgfpoint{402.524902pt}{322.054199pt}}
\pgflineto{\pgfpoint{402.604736pt}{322.193329pt}}
\pgfusepath{stroke}
\pgfpathmoveto{\pgfpoint{402.672241pt}{322.117584pt}}
\pgflineto{\pgfpoint{402.524902pt}{322.054199pt}}
\pgfusepath{stroke}
\pgfpathmoveto{\pgfpoint{402.559753pt}{328.039764pt}}
\pgflineto{\pgfpoint{402.627747pt}{328.175720pt}}
\pgfusepath{stroke}
\pgfpathmoveto{\pgfpoint{402.695740pt}{328.107727pt}}
\pgflineto{\pgfpoint{402.559753pt}{328.039764pt}}
\pgfusepath{stroke}
\pgfpathmoveto{\pgfpoint{402.590759pt}{334.029236pt}}
\pgflineto{\pgfpoint{402.648590pt}{334.161194pt}}
\pgfusepath{stroke}
\pgfpathmoveto{\pgfpoint{402.716217pt}{334.100098pt}}
\pgflineto{\pgfpoint{402.590759pt}{334.029236pt}}
\pgfusepath{stroke}
\pgfpathmoveto{\pgfpoint{402.618042pt}{340.021362pt}}
\pgflineto{\pgfpoint{402.667267pt}{340.148804pt}}
\pgfusepath{stroke}
\pgfpathmoveto{\pgfpoint{402.733917pt}{340.093781pt}}
\pgflineto{\pgfpoint{402.618042pt}{340.021362pt}}
\pgfusepath{stroke}
\pgfpathmoveto{\pgfpoint{402.641907pt}{346.015167pt}}
\pgflineto{\pgfpoint{402.683868pt}{346.138000pt}}
\pgfusepath{stroke}
\pgfpathmoveto{\pgfpoint{402.749146pt}{346.088257pt}}
\pgflineto{\pgfpoint{402.641907pt}{346.015167pt}}
\pgfusepath{stroke}
\pgfpathmoveto{\pgfpoint{402.662781pt}{352.010071pt}}
\pgflineto{\pgfpoint{402.698547pt}{352.128143pt}}
\pgfusepath{stroke}
\pgfpathmoveto{\pgfpoint{402.762268pt}{352.083069pt}}
\pgflineto{\pgfpoint{402.662781pt}{352.010071pt}}
\pgfusepath{stroke}
\pgfpathmoveto{\pgfpoint{402.681000pt}{358.005493pt}}
\pgflineto{\pgfpoint{402.711578pt}{358.119019pt}}
\pgfusepath{stroke}
\pgfpathmoveto{\pgfpoint{402.773560pt}{358.077972pt}}
\pgflineto{\pgfpoint{402.681000pt}{358.005493pt}}
\pgfusepath{stroke}
\pgfpathmoveto{\pgfpoint{402.696991pt}{364.001190pt}}
\pgflineto{\pgfpoint{402.723114pt}{364.110291pt}}
\pgfusepath{stroke}
\pgfpathmoveto{\pgfpoint{402.783356pt}{364.072784pt}}
\pgflineto{\pgfpoint{402.696991pt}{364.001190pt}}
\pgfusepath{stroke}
\pgfpathmoveto{\pgfpoint{402.711060pt}{369.996887pt}}
\pgflineto{\pgfpoint{402.733368pt}{370.101776pt}}
\pgfusepath{stroke}
\pgfpathmoveto{\pgfpoint{402.791840pt}{370.067413pt}}
\pgflineto{\pgfpoint{402.711060pt}{369.996887pt}}
\pgfusepath{stroke}
\pgfpathmoveto{\pgfpoint{408.766510pt}{77.034668pt}}
\pgflineto{\pgfpoint{408.814880pt}{76.986115pt}}
\pgfusepath{stroke}
\pgfpathmoveto{\pgfpoint{408.776062pt}{76.966766pt}}
\pgflineto{\pgfpoint{408.766510pt}{77.034668pt}}
\pgfusepath{stroke}
\pgfpathmoveto{\pgfpoint{408.761749pt}{83.025696pt}}
\pgflineto{\pgfpoint{408.812103pt}{82.976562pt}}
\pgfusepath{stroke}
\pgfpathmoveto{\pgfpoint{408.772552pt}{82.956177pt}}
\pgflineto{\pgfpoint{408.761749pt}{83.025696pt}}
\pgfusepath{stroke}
\pgfpathmoveto{\pgfpoint{408.756683pt}{89.016708pt}}
\pgflineto{\pgfpoint{408.809082pt}{88.967064pt}}
\pgfusepath{stroke}
\pgfpathmoveto{\pgfpoint{408.768799pt}{88.945564pt}}
\pgflineto{\pgfpoint{408.756683pt}{89.016708pt}}
\pgfusepath{stroke}
\pgfpathmoveto{\pgfpoint{408.751221pt}{95.007736pt}}
\pgflineto{\pgfpoint{408.805817pt}{94.957588pt}}
\pgfusepath{stroke}
\pgfpathmoveto{\pgfpoint{408.764832pt}{94.934891pt}}
\pgflineto{\pgfpoint{408.751221pt}{95.007736pt}}
\pgfusepath{stroke}
\pgfpathmoveto{\pgfpoint{408.745422pt}{100.998749pt}}
\pgflineto{\pgfpoint{408.802307pt}{100.948166pt}}
\pgfusepath{stroke}
\pgfpathmoveto{\pgfpoint{408.760559pt}{100.924171pt}}
\pgflineto{\pgfpoint{408.745422pt}{100.998749pt}}
\pgfusepath{stroke}
\pgfpathmoveto{\pgfpoint{408.739197pt}{106.989700pt}}
\pgflineto{\pgfpoint{408.798492pt}{106.938744pt}}
\pgfusepath{stroke}
\pgfpathmoveto{\pgfpoint{408.756073pt}{106.913361pt}}
\pgflineto{\pgfpoint{408.739197pt}{106.989700pt}}
\pgfusepath{stroke}
\pgfpathmoveto{\pgfpoint{408.732544pt}{112.980591pt}}
\pgflineto{\pgfpoint{408.794434pt}{112.929337pt}}
\pgfusepath{stroke}
\pgfpathmoveto{\pgfpoint{408.751282pt}{112.902466pt}}
\pgflineto{\pgfpoint{408.732544pt}{112.980591pt}}
\pgfusepath{stroke}
\pgfpathmoveto{\pgfpoint{408.725464pt}{118.971382pt}}
\pgflineto{\pgfpoint{408.790009pt}{118.919907pt}}
\pgfusepath{stroke}
\pgfpathmoveto{\pgfpoint{408.746216pt}{118.891449pt}}
\pgflineto{\pgfpoint{408.725464pt}{118.971382pt}}
\pgfusepath{stroke}
\pgfpathmoveto{\pgfpoint{408.717834pt}{124.962029pt}}
\pgflineto{\pgfpoint{408.785278pt}{124.910431pt}}
\pgfusepath{stroke}
\pgfpathmoveto{\pgfpoint{408.740845pt}{124.880295pt}}
\pgflineto{\pgfpoint{408.717834pt}{124.962029pt}}
\pgfusepath{stroke}
\pgfpathmoveto{\pgfpoint{408.709778pt}{130.952515pt}}
\pgflineto{\pgfpoint{408.780182pt}{130.900909pt}}
\pgfusepath{stroke}
\pgfpathmoveto{\pgfpoint{408.735138pt}{130.868958pt}}
\pgflineto{\pgfpoint{408.709778pt}{130.952515pt}}
\pgfusepath{stroke}
\pgfpathmoveto{\pgfpoint{408.701172pt}{136.942764pt}}
\pgflineto{\pgfpoint{408.774719pt}{136.891281pt}}
\pgfusepath{stroke}
\pgfpathmoveto{\pgfpoint{408.729126pt}{136.857437pt}}
\pgflineto{\pgfpoint{408.701172pt}{136.942764pt}}
\pgfusepath{stroke}
\pgfpathmoveto{\pgfpoint{408.692017pt}{142.932770pt}}
\pgflineto{\pgfpoint{408.768890pt}{142.881546pt}}
\pgfusepath{stroke}
\pgfpathmoveto{\pgfpoint{408.722778pt}{142.845673pt}}
\pgflineto{\pgfpoint{408.692017pt}{142.932770pt}}
\pgfusepath{stroke}
\pgfpathmoveto{\pgfpoint{408.682343pt}{148.922455pt}}
\pgflineto{\pgfpoint{408.762665pt}{148.871674pt}}
\pgfusepath{stroke}
\pgfpathmoveto{\pgfpoint{408.716125pt}{148.833633pt}}
\pgflineto{\pgfpoint{408.682343pt}{148.922455pt}}
\pgfusepath{stroke}
\pgfpathmoveto{\pgfpoint{408.672119pt}{154.911758pt}}
\pgflineto{\pgfpoint{408.756012pt}{154.861588pt}}
\pgfusepath{stroke}
\pgfpathmoveto{\pgfpoint{408.709106pt}{154.821289pt}}
\pgflineto{\pgfpoint{408.672119pt}{154.911758pt}}
\pgfusepath{stroke}
\pgfpathmoveto{\pgfpoint{408.661316pt}{160.900665pt}}
\pgflineto{\pgfpoint{408.748932pt}{160.851303pt}}
\pgfusepath{stroke}
\pgfpathmoveto{\pgfpoint{408.701813pt}{160.808624pt}}
\pgflineto{\pgfpoint{408.661316pt}{160.900665pt}}
\pgfusepath{stroke}
\pgfpathmoveto{\pgfpoint{408.649963pt}{166.889099pt}}
\pgflineto{\pgfpoint{408.741455pt}{166.840775pt}}
\pgfusepath{stroke}
\pgfpathmoveto{\pgfpoint{408.694153pt}{166.795563pt}}
\pgflineto{\pgfpoint{408.649963pt}{166.889099pt}}
\pgfusepath{stroke}
\pgfpathmoveto{\pgfpoint{408.638092pt}{172.877014pt}}
\pgflineto{\pgfpoint{408.733551pt}{172.829956pt}}
\pgfusepath{stroke}
\pgfpathmoveto{\pgfpoint{408.686218pt}{172.782104pt}}
\pgflineto{\pgfpoint{408.638092pt}{172.877014pt}}
\pgfusepath{stroke}
\pgfpathmoveto{\pgfpoint{408.625641pt}{178.864365pt}}
\pgflineto{\pgfpoint{408.725189pt}{178.818832pt}}
\pgfusepath{stroke}
\pgfpathmoveto{\pgfpoint{408.677979pt}{178.768204pt}}
\pgflineto{\pgfpoint{408.625641pt}{178.864365pt}}
\pgfusepath{stroke}
\pgfpathmoveto{\pgfpoint{408.612640pt}{184.851135pt}}
\pgflineto{\pgfpoint{408.716431pt}{184.807373pt}}
\pgfusepath{stroke}
\pgfpathmoveto{\pgfpoint{408.669403pt}{184.753860pt}}
\pgflineto{\pgfpoint{408.612640pt}{184.851135pt}}
\pgfusepath{stroke}
\pgfpathmoveto{\pgfpoint{408.599060pt}{190.837250pt}}
\pgflineto{\pgfpoint{408.707184pt}{190.795547pt}}
\pgfusepath{stroke}
\pgfpathmoveto{\pgfpoint{408.660522pt}{190.739014pt}}
\pgflineto{\pgfpoint{408.599060pt}{190.837250pt}}
\pgfusepath{stroke}
\pgfpathmoveto{\pgfpoint{408.584869pt}{196.822708pt}}
\pgflineto{\pgfpoint{408.697449pt}{196.783356pt}}
\pgfusepath{stroke}
\pgfpathmoveto{\pgfpoint{408.651337pt}{196.723663pt}}
\pgflineto{\pgfpoint{408.584869pt}{196.822708pt}}
\pgfusepath{stroke}
\pgfpathmoveto{\pgfpoint{408.570007pt}{202.807449pt}}
\pgflineto{\pgfpoint{408.687225pt}{202.770737pt}}
\pgfusepath{stroke}
\pgfpathmoveto{\pgfpoint{408.641754pt}{202.707748pt}}
\pgflineto{\pgfpoint{408.570007pt}{202.807449pt}}
\pgfusepath{stroke}
\pgfpathmoveto{\pgfpoint{408.554352pt}{208.791382pt}}
\pgflineto{\pgfpoint{408.676361pt}{208.757690pt}}
\pgfusepath{stroke}
\pgfpathmoveto{\pgfpoint{408.631744pt}{208.691223pt}}
\pgflineto{\pgfpoint{408.554352pt}{208.791382pt}}
\pgfusepath{stroke}
\pgfpathmoveto{\pgfpoint{408.537781pt}{214.774399pt}}
\pgflineto{\pgfpoint{408.664795pt}{214.744125pt}}
\pgfusepath{stroke}
\pgfpathmoveto{\pgfpoint{408.621216pt}{214.673981pt}}
\pgflineto{\pgfpoint{408.537781pt}{214.774399pt}}
\pgfusepath{stroke}
\pgfpathmoveto{\pgfpoint{408.520142pt}{220.756287pt}}
\pgflineto{\pgfpoint{408.652344pt}{220.729935pt}}
\pgfusepath{stroke}
\pgfpathmoveto{\pgfpoint{408.610107pt}{220.655869pt}}
\pgflineto{\pgfpoint{408.520142pt}{220.756287pt}}
\pgfusepath{stroke}
\pgfpathmoveto{\pgfpoint{408.501190pt}{226.736725pt}}
\pgflineto{\pgfpoint{408.638947pt}{226.714920pt}}
\pgfusepath{stroke}
\pgfpathmoveto{\pgfpoint{408.598297pt}{226.636627pt}}
\pgflineto{\pgfpoint{408.501190pt}{226.736725pt}}
\pgfusepath{stroke}
\pgfpathmoveto{\pgfpoint{408.480865pt}{232.715225pt}}
\pgflineto{\pgfpoint{408.624329pt}{232.698761pt}}
\pgfusepath{stroke}
\pgfpathmoveto{\pgfpoint{408.585785pt}{232.615952pt}}
\pgflineto{\pgfpoint{408.480865pt}{232.715225pt}}
\pgfusepath{stroke}
\pgfpathmoveto{\pgfpoint{408.459045pt}{238.691101pt}}
\pgflineto{\pgfpoint{408.608490pt}{238.681015pt}}
\pgfusepath{stroke}
\pgfpathmoveto{\pgfpoint{408.572571pt}{238.593369pt}}
\pgflineto{\pgfpoint{408.459045pt}{238.691101pt}}
\pgfusepath{stroke}
\pgfpathmoveto{\pgfpoint{408.435974pt}{244.663498pt}}
\pgflineto{\pgfpoint{408.591400pt}{244.661102pt}}
\pgfusepath{stroke}
\pgfpathmoveto{\pgfpoint{408.558899pt}{244.568314pt}}
\pgflineto{\pgfpoint{408.435974pt}{244.663498pt}}
\pgfusepath{stroke}
\pgfpathmoveto{\pgfpoint{408.412170pt}{250.631378pt}}
\pgflineto{\pgfpoint{408.573303pt}{250.638245pt}}
\pgfusepath{stroke}
\pgfpathmoveto{\pgfpoint{408.545227pt}{250.540176pt}}
\pgflineto{\pgfpoint{408.412170pt}{250.631378pt}}
\pgfusepath{stroke}
\pgfpathmoveto{\pgfpoint{408.388702pt}{256.593719pt}}
\pgflineto{\pgfpoint{408.554840pt}{256.611664pt}}
\pgfusepath{stroke}
\pgfpathmoveto{\pgfpoint{408.532379pt}{256.508392pt}}
\pgflineto{\pgfpoint{408.388702pt}{256.593719pt}}
\pgfusepath{stroke}
\pgfpathmoveto{\pgfpoint{408.367249pt}{262.549805pt}}
\pgflineto{\pgfpoint{408.537018pt}{262.580688pt}}
\pgfusepath{stroke}
\pgfpathmoveto{\pgfpoint{408.521606pt}{262.472656pt}}
\pgflineto{\pgfpoint{408.367249pt}{262.549805pt}}
\pgfusepath{stroke}
\pgfpathmoveto{\pgfpoint{408.350037pt}{268.499512pt}}
\pgflineto{\pgfpoint{408.521332pt}{268.544983pt}}
\pgfusepath{stroke}
\pgfpathmoveto{\pgfpoint{408.514343pt}{268.433105pt}}
\pgflineto{\pgfpoint{408.350037pt}{268.499512pt}}
\pgfusepath{stroke}
\pgfpathmoveto{\pgfpoint{408.339539pt}{274.443787pt}}
\pgflineto{\pgfpoint{408.509491pt}{274.504944pt}}
\pgfusepath{stroke}
\pgfpathmoveto{\pgfpoint{408.512207pt}{274.390717pt}}
\pgflineto{\pgfpoint{408.339539pt}{274.443787pt}}
\pgfusepath{stroke}
\pgfpathmoveto{\pgfpoint{408.337830pt}{280.384766pt}}
\pgflineto{\pgfpoint{408.503174pt}{280.461670pt}}
\pgfusepath{stroke}
\pgfpathmoveto{\pgfpoint{408.516266pt}{280.347107pt}}
\pgflineto{\pgfpoint{408.337830pt}{280.384766pt}}
\pgfusepath{stroke}
\pgfpathmoveto{\pgfpoint{408.346008pt}{286.325439pt}}
\pgflineto{\pgfpoint{408.503387pt}{286.417145pt}}
\pgfusepath{stroke}
\pgfpathmoveto{\pgfpoint{408.526947pt}{286.304382pt}}
\pgflineto{\pgfpoint{408.346008pt}{286.325439pt}}
\pgfusepath{stroke}
\pgfpathmoveto{\pgfpoint{408.363617pt}{292.269104pt}}
\pgflineto{\pgfpoint{408.510223pt}{292.373535pt}}
\pgfusepath{stroke}
\pgfpathmoveto{\pgfpoint{408.543579pt}{292.264679pt}}
\pgflineto{\pgfpoint{408.363617pt}{292.269104pt}}
\pgfusepath{stroke}
\pgfpathmoveto{\pgfpoint{408.388977pt}{298.218414pt}}
\pgflineto{\pgfpoint{408.522858pt}{298.332886pt}}
\pgfusepath{stroke}
\pgfpathmoveto{\pgfpoint{408.564758pt}{298.229645pt}}
\pgflineto{\pgfpoint{408.388977pt}{298.218414pt}}
\pgfusepath{stroke}
\pgfpathmoveto{\pgfpoint{408.419556pt}{304.175018pt}}
\pgflineto{\pgfpoint{408.539795pt}{304.296448pt}}
\pgfusepath{stroke}
\pgfpathmoveto{\pgfpoint{408.588593pt}{304.200043pt}}
\pgflineto{\pgfpoint{408.419556pt}{304.175018pt}}
\pgfusepath{stroke}
\pgfpathmoveto{\pgfpoint{408.452759pt}{310.139282pt}}
\pgflineto{\pgfpoint{408.559326pt}{310.264862pt}}
\pgfusepath{stroke}
\pgfpathmoveto{\pgfpoint{408.613342pt}{310.175842pt}}
\pgflineto{\pgfpoint{408.452759pt}{310.139282pt}}
\pgfusepath{stroke}
\pgfpathmoveto{\pgfpoint{408.486389pt}{316.110718pt}}
\pgflineto{\pgfpoint{408.579926pt}{316.238007pt}}
\pgfusepath{stroke}
\pgfpathmoveto{\pgfpoint{408.637573pt}{316.156433pt}}
\pgflineto{\pgfpoint{408.486389pt}{316.110718pt}}
\pgfusepath{stroke}
\pgfpathmoveto{\pgfpoint{408.518860pt}{322.088287pt}}
\pgflineto{\pgfpoint{408.600433pt}{322.215363pt}}
\pgfusepath{stroke}
\pgfpathmoveto{\pgfpoint{408.660370pt}{322.140991pt}}
\pgflineto{\pgfpoint{408.518860pt}{322.088287pt}}
\pgfusepath{stroke}
\pgfpathmoveto{\pgfpoint{408.549225pt}{328.070740pt}}
\pgflineto{\pgfpoint{408.620087pt}{328.196228pt}}
\pgfusepath{stroke}
\pgfpathmoveto{\pgfpoint{408.681183pt}{328.128601pt}}
\pgflineto{\pgfpoint{408.549225pt}{328.070740pt}}
\pgfusepath{stroke}
\pgfpathmoveto{\pgfpoint{408.576965pt}{334.056976pt}}
\pgflineto{\pgfpoint{408.638428pt}{334.179871pt}}
\pgfusepath{stroke}
\pgfpathmoveto{\pgfpoint{408.699860pt}{334.118408pt}}
\pgflineto{\pgfpoint{408.576965pt}{334.056976pt}}
\pgfusepath{stroke}
\pgfpathmoveto{\pgfpoint{408.601990pt}{340.045959pt}}
\pgflineto{\pgfpoint{408.655243pt}{340.165649pt}}
\pgfusepath{stroke}
\pgfpathmoveto{\pgfpoint{408.716400pt}{340.109772pt}}
\pgflineto{\pgfpoint{408.601990pt}{340.045959pt}}
\pgfusepath{stroke}
\pgfpathmoveto{\pgfpoint{408.624390pt}{346.036926pt}}
\pgflineto{\pgfpoint{408.670532pt}{346.153046pt}}
\pgfusepath{stroke}
\pgfpathmoveto{\pgfpoint{408.730957pt}{346.102173pt}}
\pgflineto{\pgfpoint{408.624390pt}{346.036926pt}}
\pgfusepath{stroke}
\pgfpathmoveto{\pgfpoint{408.644287pt}{352.029175pt}}
\pgflineto{\pgfpoint{408.684296pt}{352.141602pt}}
\pgfusepath{stroke}
\pgfpathmoveto{\pgfpoint{408.743744pt}{352.095154pt}}
\pgflineto{\pgfpoint{408.644287pt}{352.029175pt}}
\pgfusepath{stroke}
\pgfpathmoveto{\pgfpoint{408.661987pt}{358.022369pt}}
\pgflineto{\pgfpoint{408.696686pt}{358.131042pt}}
\pgfusepath{stroke}
\pgfpathmoveto{\pgfpoint{408.754944pt}{358.088470pt}}
\pgflineto{\pgfpoint{408.661987pt}{358.022369pt}}
\pgfusepath{stroke}
\pgfpathmoveto{\pgfpoint{408.677704pt}{364.016083pt}}
\pgflineto{\pgfpoint{408.707825pt}{364.121002pt}}
\pgfusepath{stroke}
\pgfpathmoveto{\pgfpoint{408.764740pt}{364.081940pt}}
\pgflineto{\pgfpoint{408.677704pt}{364.016083pt}}
\pgfusepath{stroke}
\pgfpathmoveto{\pgfpoint{408.691681pt}{370.010101pt}}
\pgflineto{\pgfpoint{408.717834pt}{370.111389pt}}
\pgfusepath{stroke}
\pgfpathmoveto{\pgfpoint{408.773376pt}{370.075439pt}}
\pgflineto{\pgfpoint{408.691681pt}{370.010101pt}}
\pgfusepath{stroke}
\pgfpathmoveto{\pgfpoint{414.751160pt}{77.030075pt}}
\pgflineto{\pgfpoint{414.800079pt}{76.983383pt}}
\pgfusepath{stroke}
\pgfpathmoveto{\pgfpoint{414.762268pt}{76.963379pt}}
\pgflineto{\pgfpoint{414.751160pt}{77.030075pt}}
\pgfusepath{stroke}
\pgfpathmoveto{\pgfpoint{414.746399pt}{83.020767pt}}
\pgflineto{\pgfpoint{414.797241pt}{82.973633pt}}
\pgfusepath{stroke}
\pgfpathmoveto{\pgfpoint{414.758789pt}{82.952576pt}}
\pgflineto{\pgfpoint{414.746399pt}{83.020767pt}}
\pgfusepath{stroke}
\pgfpathmoveto{\pgfpoint{414.741333pt}{89.011467pt}}
\pgflineto{\pgfpoint{414.794189pt}{88.963905pt}}
\pgfusepath{stroke}
\pgfpathmoveto{\pgfpoint{414.755066pt}{88.941711pt}}
\pgflineto{\pgfpoint{414.741333pt}{89.011467pt}}
\pgfusepath{stroke}
\pgfpathmoveto{\pgfpoint{414.735931pt}{95.002121pt}}
\pgflineto{\pgfpoint{414.790894pt}{94.954201pt}}
\pgfusepath{stroke}
\pgfpathmoveto{\pgfpoint{414.751160pt}{94.930794pt}}
\pgflineto{\pgfpoint{414.735931pt}{95.002121pt}}
\pgfusepath{stroke}
\pgfpathmoveto{\pgfpoint{414.730164pt}{100.992752pt}}
\pgflineto{\pgfpoint{414.787354pt}{100.944511pt}}
\pgfusepath{stroke}
\pgfpathmoveto{\pgfpoint{414.746979pt}{100.919830pt}}
\pgflineto{\pgfpoint{414.730164pt}{100.992752pt}}
\pgfusepath{stroke}
\pgfpathmoveto{\pgfpoint{414.723999pt}{106.983307pt}}
\pgflineto{\pgfpoint{414.783569pt}{106.934822pt}}
\pgfusepath{stroke}
\pgfpathmoveto{\pgfpoint{414.742554pt}{106.908775pt}}
\pgflineto{\pgfpoint{414.723999pt}{106.983307pt}}
\pgfusepath{stroke}
\pgfpathmoveto{\pgfpoint{414.717438pt}{112.973770pt}}
\pgflineto{\pgfpoint{414.779510pt}{112.925102pt}}
\pgfusepath{stroke}
\pgfpathmoveto{\pgfpoint{414.737885pt}{112.897598pt}}
\pgflineto{\pgfpoint{414.717438pt}{112.973770pt}}
\pgfusepath{stroke}
\pgfpathmoveto{\pgfpoint{414.710449pt}{118.964104pt}}
\pgflineto{\pgfpoint{414.775146pt}{118.915352pt}}
\pgfusepath{stroke}
\pgfpathmoveto{\pgfpoint{414.732941pt}{118.886292pt}}
\pgflineto{\pgfpoint{414.710449pt}{118.964104pt}}
\pgfusepath{stroke}
\pgfpathmoveto{\pgfpoint{414.703033pt}{124.954285pt}}
\pgflineto{\pgfpoint{414.770447pt}{124.905556pt}}
\pgfusepath{stroke}
\pgfpathmoveto{\pgfpoint{414.727722pt}{124.874847pt}}
\pgflineto{\pgfpoint{414.703033pt}{124.954285pt}}
\pgfusepath{stroke}
\pgfpathmoveto{\pgfpoint{414.695160pt}{130.944275pt}}
\pgflineto{\pgfpoint{414.765442pt}{130.895676pt}}
\pgfusepath{stroke}
\pgfpathmoveto{\pgfpoint{414.722229pt}{130.863220pt}}
\pgflineto{\pgfpoint{414.695160pt}{130.944275pt}}
\pgfusepath{stroke}
\pgfpathmoveto{\pgfpoint{414.686798pt}{136.934036pt}}
\pgflineto{\pgfpoint{414.760101pt}{136.885681pt}}
\pgfusepath{stroke}
\pgfpathmoveto{\pgfpoint{414.716431pt}{136.851379pt}}
\pgflineto{\pgfpoint{414.686798pt}{136.934036pt}}
\pgfusepath{stroke}
\pgfpathmoveto{\pgfpoint{414.677948pt}{142.923523pt}}
\pgflineto{\pgfpoint{414.754395pt}{142.875580pt}}
\pgfusepath{stroke}
\pgfpathmoveto{\pgfpoint{414.710358pt}{142.839294pt}}
\pgflineto{\pgfpoint{414.677948pt}{142.923523pt}}
\pgfusepath{stroke}
\pgfpathmoveto{\pgfpoint{414.668610pt}{148.912674pt}}
\pgflineto{\pgfpoint{414.748322pt}{148.865311pt}}
\pgfusepath{stroke}
\pgfpathmoveto{\pgfpoint{414.703979pt}{148.826965pt}}
\pgflineto{\pgfpoint{414.668610pt}{148.912674pt}}
\pgfusepath{stroke}
\pgfpathmoveto{\pgfpoint{414.658783pt}{154.901459pt}}
\pgflineto{\pgfpoint{414.741882pt}{154.854843pt}}
\pgfusepath{stroke}
\pgfpathmoveto{\pgfpoint{414.697296pt}{154.814301pt}}
\pgflineto{\pgfpoint{414.658783pt}{154.901459pt}}
\pgfusepath{stroke}
\pgfpathmoveto{\pgfpoint{414.648438pt}{160.889832pt}}
\pgflineto{\pgfpoint{414.735046pt}{160.844162pt}}
\pgfusepath{stroke}
\pgfpathmoveto{\pgfpoint{414.690338pt}{160.801331pt}}
\pgflineto{\pgfpoint{414.648438pt}{160.889832pt}}
\pgfusepath{stroke}
\pgfpathmoveto{\pgfpoint{414.637573pt}{166.877731pt}}
\pgflineto{\pgfpoint{414.727844pt}{166.833221pt}}
\pgfusepath{stroke}
\pgfpathmoveto{\pgfpoint{414.683075pt}{166.787964pt}}
\pgflineto{\pgfpoint{414.637573pt}{166.877731pt}}
\pgfusepath{stroke}
\pgfpathmoveto{\pgfpoint{414.626221pt}{172.865128pt}}
\pgflineto{\pgfpoint{414.720215pt}{172.821991pt}}
\pgfusepath{stroke}
\pgfpathmoveto{\pgfpoint{414.675537pt}{172.774216pt}}
\pgflineto{\pgfpoint{414.626221pt}{172.865128pt}}
\pgfusepath{stroke}
\pgfpathmoveto{\pgfpoint{414.614349pt}{178.851944pt}}
\pgflineto{\pgfpoint{414.712219pt}{178.810455pt}}
\pgfusepath{stroke}
\pgfpathmoveto{\pgfpoint{414.667725pt}{178.760040pt}}
\pgflineto{\pgfpoint{414.614349pt}{178.851944pt}}
\pgfusepath{stroke}
\pgfpathmoveto{\pgfpoint{414.601990pt}{184.838165pt}}
\pgflineto{\pgfpoint{414.703796pt}{184.798553pt}}
\pgfusepath{stroke}
\pgfpathmoveto{\pgfpoint{414.659668pt}{184.745392pt}}
\pgflineto{\pgfpoint{414.601990pt}{184.838165pt}}
\pgfusepath{stroke}
\pgfpathmoveto{\pgfpoint{414.589081pt}{190.823730pt}}
\pgflineto{\pgfpoint{414.694946pt}{190.786285pt}}
\pgfusepath{stroke}
\pgfpathmoveto{\pgfpoint{414.651306pt}{190.730255pt}}
\pgflineto{\pgfpoint{414.589081pt}{190.823730pt}}
\pgfusepath{stroke}
\pgfpathmoveto{\pgfpoint{414.575592pt}{196.808563pt}}
\pgflineto{\pgfpoint{414.685638pt}{196.773590pt}}
\pgfusepath{stroke}
\pgfpathmoveto{\pgfpoint{414.642639pt}{196.714569pt}}
\pgflineto{\pgfpoint{414.575592pt}{196.808563pt}}
\pgfusepath{stroke}
\pgfpathmoveto{\pgfpoint{414.561523pt}{202.792618pt}}
\pgflineto{\pgfpoint{414.675842pt}{202.760437pt}}
\pgfusepath{stroke}
\pgfpathmoveto{\pgfpoint{414.633667pt}{202.698288pt}}
\pgflineto{\pgfpoint{414.561523pt}{202.792618pt}}
\pgfusepath{stroke}
\pgfpathmoveto{\pgfpoint{414.546783pt}{208.775787pt}}
\pgflineto{\pgfpoint{414.665497pt}{208.746765pt}}
\pgfusepath{stroke}
\pgfpathmoveto{\pgfpoint{414.624359pt}{208.681335pt}}
\pgflineto{\pgfpoint{414.546783pt}{208.775787pt}}
\pgfusepath{stroke}
\pgfpathmoveto{\pgfpoint{414.531311pt}{214.757904pt}}
\pgflineto{\pgfpoint{414.654541pt}{214.732498pt}}
\pgfusepath{stroke}
\pgfpathmoveto{\pgfpoint{414.614655pt}{214.663620pt}}
\pgflineto{\pgfpoint{414.531311pt}{214.757904pt}}
\pgfusepath{stroke}
\pgfpathmoveto{\pgfpoint{414.514984pt}{220.738770pt}}
\pgflineto{\pgfpoint{414.642914pt}{220.717468pt}}
\pgfusepath{stroke}
\pgfpathmoveto{\pgfpoint{414.604553pt}{220.644974pt}}
\pgflineto{\pgfpoint{414.514984pt}{220.738770pt}}
\pgfusepath{stroke}
\pgfpathmoveto{\pgfpoint{414.497772pt}{226.718079pt}}
\pgflineto{\pgfpoint{414.630493pt}{226.701508pt}}
\pgfusepath{stroke}
\pgfpathmoveto{\pgfpoint{414.593994pt}{226.625183pt}}
\pgflineto{\pgfpoint{414.497772pt}{226.718079pt}}
\pgfusepath{stroke}
\pgfpathmoveto{\pgfpoint{414.479675pt}{232.695435pt}}
\pgflineto{\pgfpoint{414.617279pt}{232.684326pt}}
\pgfusepath{stroke}
\pgfpathmoveto{\pgfpoint{414.583099pt}{232.603989pt}}
\pgflineto{\pgfpoint{414.479675pt}{232.695435pt}}
\pgfusepath{stroke}
\pgfpathmoveto{\pgfpoint{414.460754pt}{238.670288pt}}
\pgflineto{\pgfpoint{414.603241pt}{238.665588pt}}
\pgfusepath{stroke}
\pgfpathmoveto{\pgfpoint{414.571930pt}{238.581055pt}}
\pgflineto{\pgfpoint{414.460754pt}{238.670288pt}}
\pgfusepath{stroke}
\pgfpathmoveto{\pgfpoint{414.441406pt}{244.642044pt}}
\pgflineto{\pgfpoint{414.588562pt}{244.644836pt}}
\pgfusepath{stroke}
\pgfpathmoveto{\pgfpoint{414.560791pt}{244.555984pt}}
\pgflineto{\pgfpoint{414.441406pt}{244.642044pt}}
\pgfusepath{stroke}
\pgfpathmoveto{\pgfpoint{414.422119pt}{250.610062pt}}
\pgflineto{\pgfpoint{414.573517pt}{250.621567pt}}
\pgfusepath{stroke}
\pgfpathmoveto{\pgfpoint{414.550140pt}{250.528442pt}}
\pgflineto{\pgfpoint{414.422119pt}{250.610062pt}}
\pgfusepath{stroke}
\pgfpathmoveto{\pgfpoint{414.403870pt}{256.573792pt}}
\pgflineto{\pgfpoint{414.558655pt}{256.595337pt}}
\pgfusepath{stroke}
\pgfpathmoveto{\pgfpoint{414.540649pt}{256.498138pt}}
\pgflineto{\pgfpoint{414.403870pt}{256.573792pt}}
\pgfusepath{stroke}
\pgfpathmoveto{\pgfpoint{414.387878pt}{262.532928pt}}
\pgflineto{\pgfpoint{414.544800pt}{262.565796pt}}
\pgfusepath{stroke}
\pgfpathmoveto{\pgfpoint{414.533142pt}{262.465027pt}}
\pgflineto{\pgfpoint{414.387878pt}{262.532928pt}}
\pgfusepath{stroke}
\pgfpathmoveto{\pgfpoint{414.375671pt}{268.487671pt}}
\pgflineto{\pgfpoint{414.533020pt}{268.532898pt}}
\pgfusepath{stroke}
\pgfpathmoveto{\pgfpoint{414.528687pt}{268.429443pt}}
\pgflineto{\pgfpoint{414.375671pt}{268.487671pt}}
\pgfusepath{stroke}
\pgfpathmoveto{\pgfpoint{414.368835pt}{274.438843pt}}
\pgflineto{\pgfpoint{414.524414pt}{274.497040pt}}
\pgfusepath{stroke}
\pgfpathmoveto{\pgfpoint{414.528198pt}{274.392059pt}}
\pgflineto{\pgfpoint{414.368835pt}{274.438843pt}}
\pgfusepath{stroke}
\pgfpathmoveto{\pgfpoint{414.368591pt}{280.388031pt}}
\pgflineto{\pgfpoint{414.519958pt}{280.459076pt}}
\pgfusepath{stroke}
\pgfpathmoveto{\pgfpoint{414.532318pt}{280.354065pt}}
\pgflineto{\pgfpoint{414.368591pt}{280.388031pt}}
\pgfusepath{stroke}
\pgfpathmoveto{\pgfpoint{414.375549pt}{286.337158pt}}
\pgflineto{\pgfpoint{414.520325pt}{286.420319pt}}
\pgfusepath{stroke}
\pgfpathmoveto{\pgfpoint{414.541260pt}{286.316833pt}}
\pgflineto{\pgfpoint{414.375549pt}{286.337158pt}}
\pgfusepath{stroke}
\pgfpathmoveto{\pgfpoint{414.389435pt}{292.288391pt}}
\pgflineto{\pgfpoint{414.525513pt}{292.382202pt}}
\pgfusepath{stroke}
\pgfpathmoveto{\pgfpoint{414.554565pt}{292.281799pt}}
\pgflineto{\pgfpoint{414.389435pt}{292.288391pt}}
\pgfusepath{stroke}
\pgfpathmoveto{\pgfpoint{414.409210pt}{298.243561pt}}
\pgflineto{\pgfpoint{414.535065pt}{298.346008pt}}
\pgfusepath{stroke}
\pgfpathmoveto{\pgfpoint{414.571350pt}{298.250000pt}}
\pgflineto{\pgfpoint{414.409210pt}{298.243561pt}}
\pgfusepath{stroke}
\pgfpathmoveto{\pgfpoint{414.433289pt}{304.203827pt}}
\pgflineto{\pgfpoint{414.548065pt}{304.312775pt}}
\pgfusepath{stroke}
\pgfpathmoveto{\pgfpoint{414.590485pt}{304.222137pt}}
\pgflineto{\pgfpoint{414.433289pt}{304.203827pt}}
\pgfusepath{stroke}
\pgfpathmoveto{\pgfpoint{414.459961pt}{310.169739pt}}
\pgflineto{\pgfpoint{414.563416pt}{310.282990pt}}
\pgfusepath{stroke}
\pgfpathmoveto{\pgfpoint{414.610687pt}{310.198273pt}}
\pgflineto{\pgfpoint{414.459961pt}{310.169739pt}}
\pgfusepath{stroke}
\pgfpathmoveto{\pgfpoint{414.487640pt}{316.141113pt}}
\pgflineto{\pgfpoint{414.580048pt}{316.256775pt}}
\pgfusepath{stroke}
\pgfpathmoveto{\pgfpoint{414.630951pt}{316.178192pt}}
\pgflineto{\pgfpoint{414.487640pt}{316.141113pt}}
\pgfusepath{stroke}
\pgfpathmoveto{\pgfpoint{414.515045pt}{322.117493pt}}
\pgflineto{\pgfpoint{414.597046pt}{322.233887pt}}
\pgfusepath{stroke}
\pgfpathmoveto{\pgfpoint{414.650513pt}{322.161377pt}}
\pgflineto{\pgfpoint{414.515045pt}{322.117493pt}}
\pgfusepath{stroke}
\pgfpathmoveto{\pgfpoint{414.541351pt}{328.098022pt}}
\pgflineto{\pgfpoint{414.613770pt}{328.213928pt}}
\pgfusepath{stroke}
\pgfpathmoveto{\pgfpoint{414.668823pt}{328.147278pt}}
\pgflineto{\pgfpoint{414.541351pt}{328.098022pt}}
\pgfusepath{stroke}
\pgfpathmoveto{\pgfpoint{414.565948pt}{334.082001pt}}
\pgflineto{\pgfpoint{414.629761pt}{334.196411pt}}
\pgfusepath{stroke}
\pgfpathmoveto{\pgfpoint{414.685669pt}{334.135254pt}}
\pgflineto{\pgfpoint{414.565948pt}{334.082001pt}}
\pgfusepath{stroke}
\pgfpathmoveto{\pgfpoint{414.588654pt}{340.068665pt}}
\pgflineto{\pgfpoint{414.644775pt}{340.180908pt}}
\pgfusepath{stroke}
\pgfpathmoveto{\pgfpoint{414.700897pt}{340.124786pt}}
\pgflineto{\pgfpoint{414.588654pt}{340.068665pt}}
\pgfusepath{stroke}
\pgfpathmoveto{\pgfpoint{414.609344pt}{346.057312pt}}
\pgflineto{\pgfpoint{414.658691pt}{346.166992pt}}
\pgfusepath{stroke}
\pgfpathmoveto{\pgfpoint{414.714600pt}{346.115417pt}}
\pgflineto{\pgfpoint{414.609344pt}{346.057312pt}}
\pgfusepath{stroke}
\pgfpathmoveto{\pgfpoint{414.628082pt}{352.047485pt}}
\pgflineto{\pgfpoint{414.671417pt}{352.154236pt}}
\pgfusepath{stroke}
\pgfpathmoveto{\pgfpoint{414.726807pt}{352.106873pt}}
\pgflineto{\pgfpoint{414.628082pt}{352.047485pt}}
\pgfusepath{stroke}
\pgfpathmoveto{\pgfpoint{414.644958pt}{358.038727pt}}
\pgflineto{\pgfpoint{414.683075pt}{358.142456pt}}
\pgfusepath{stroke}
\pgfpathmoveto{\pgfpoint{414.737671pt}{358.098816pt}}
\pgflineto{\pgfpoint{414.644958pt}{358.038727pt}}
\pgfusepath{stroke}
\pgfpathmoveto{\pgfpoint{414.660156pt}{364.030701pt}}
\pgflineto{\pgfpoint{414.693665pt}{364.131348pt}}
\pgfusepath{stroke}
\pgfpathmoveto{\pgfpoint{414.747345pt}{364.091095pt}}
\pgflineto{\pgfpoint{414.660156pt}{364.030701pt}}
\pgfusepath{stroke}
\pgfpathmoveto{\pgfpoint{414.673828pt}{370.023163pt}}
\pgflineto{\pgfpoint{414.703308pt}{370.120697pt}}
\pgfusepath{stroke}
\pgfpathmoveto{\pgfpoint{414.755920pt}{370.083496pt}}
\pgflineto{\pgfpoint{414.673828pt}{370.023163pt}}
\pgfusepath{stroke}
\pgfpathmoveto{\pgfpoint{420.736145pt}{77.025482pt}}
\pgflineto{\pgfpoint{420.785461pt}{76.980621pt}}
\pgfusepath{stroke}
\pgfpathmoveto{\pgfpoint{420.748657pt}{76.959991pt}}
\pgflineto{\pgfpoint{420.736145pt}{77.025482pt}}
\pgfusepath{stroke}
\pgfpathmoveto{\pgfpoint{420.731384pt}{83.015884pt}}
\pgflineto{\pgfpoint{420.782593pt}{82.970673pt}}
\pgfusepath{stroke}
\pgfpathmoveto{\pgfpoint{420.745239pt}{82.948990pt}}
\pgflineto{\pgfpoint{420.731384pt}{83.015884pt}}
\pgfusepath{stroke}
\pgfpathmoveto{\pgfpoint{420.726318pt}{89.006241pt}}
\pgflineto{\pgfpoint{420.779510pt}{88.960739pt}}
\pgfusepath{stroke}
\pgfpathmoveto{\pgfpoint{420.741577pt}{88.937927pt}}
\pgflineto{\pgfpoint{420.726318pt}{89.006241pt}}
\pgfusepath{stroke}
\pgfpathmoveto{\pgfpoint{420.720947pt}{94.996574pt}}
\pgflineto{\pgfpoint{420.776215pt}{94.950790pt}}
\pgfusepath{stroke}
\pgfpathmoveto{\pgfpoint{420.737701pt}{94.926804pt}}
\pgflineto{\pgfpoint{420.720947pt}{94.996574pt}}
\pgfusepath{stroke}
\pgfpathmoveto{\pgfpoint{420.715271pt}{100.986832pt}}
\pgflineto{\pgfpoint{420.772705pt}{100.940857pt}}
\pgfusepath{stroke}
\pgfpathmoveto{\pgfpoint{420.733643pt}{100.915596pt}}
\pgflineto{\pgfpoint{420.715271pt}{100.986832pt}}
\pgfusepath{stroke}
\pgfpathmoveto{\pgfpoint{420.709198pt}{106.976997pt}}
\pgflineto{\pgfpoint{420.768921pt}{106.930893pt}}
\pgfusepath{stroke}
\pgfpathmoveto{\pgfpoint{420.729309pt}{106.904282pt}}
\pgflineto{\pgfpoint{420.709198pt}{106.976997pt}}
\pgfusepath{stroke}
\pgfpathmoveto{\pgfpoint{420.702759pt}{112.967064pt}}
\pgflineto{\pgfpoint{420.764893pt}{112.920906pt}}
\pgfusepath{stroke}
\pgfpathmoveto{\pgfpoint{420.724762pt}{112.892853pt}}
\pgflineto{\pgfpoint{420.702759pt}{112.967064pt}}
\pgfusepath{stroke}
\pgfpathmoveto{\pgfpoint{420.695923pt}{118.956985pt}}
\pgflineto{\pgfpoint{420.760559pt}{118.910866pt}}
\pgfusepath{stroke}
\pgfpathmoveto{\pgfpoint{420.719971pt}{118.881302pt}}
\pgflineto{\pgfpoint{420.695923pt}{118.956985pt}}
\pgfusepath{stroke}
\pgfpathmoveto{\pgfpoint{420.688690pt}{124.946739pt}}
\pgflineto{\pgfpoint{420.755981pt}{124.900757pt}}
\pgfusepath{stroke}
\pgfpathmoveto{\pgfpoint{420.714905pt}{124.869583pt}}
\pgflineto{\pgfpoint{420.688690pt}{124.946739pt}}
\pgfusepath{stroke}
\pgfpathmoveto{\pgfpoint{420.681030pt}{130.936279pt}}
\pgflineto{\pgfpoint{420.751038pt}{130.890564pt}}
\pgfusepath{stroke}
\pgfpathmoveto{\pgfpoint{420.709595pt}{130.857666pt}}
\pgflineto{\pgfpoint{420.681030pt}{130.936279pt}}
\pgfusepath{stroke}
\pgfpathmoveto{\pgfpoint{420.672913pt}{136.925568pt}}
\pgflineto{\pgfpoint{420.745850pt}{136.880234pt}}
\pgfusepath{stroke}
\pgfpathmoveto{\pgfpoint{420.704041pt}{136.845551pt}}
\pgflineto{\pgfpoint{420.672913pt}{136.925568pt}}
\pgfusepath{stroke}
\pgfpathmoveto{\pgfpoint{420.664368pt}{142.914581pt}}
\pgflineto{\pgfpoint{420.740295pt}{142.869781pt}}
\pgfusepath{stroke}
\pgfpathmoveto{\pgfpoint{420.698212pt}{142.833191pt}}
\pgflineto{\pgfpoint{420.664368pt}{142.914581pt}}
\pgfusepath{stroke}
\pgfpathmoveto{\pgfpoint{420.655396pt}{148.903259pt}}
\pgflineto{\pgfpoint{420.734375pt}{148.859146pt}}
\pgfusepath{stroke}
\pgfpathmoveto{\pgfpoint{420.692108pt}{148.820587pt}}
\pgflineto{\pgfpoint{420.655396pt}{148.903259pt}}
\pgfusepath{stroke}
\pgfpathmoveto{\pgfpoint{420.645935pt}{154.891571pt}}
\pgflineto{\pgfpoint{420.728149pt}{154.848328pt}}
\pgfusepath{stroke}
\pgfpathmoveto{\pgfpoint{420.685760pt}{154.807648pt}}
\pgflineto{\pgfpoint{420.645935pt}{154.891571pt}}
\pgfusepath{stroke}
\pgfpathmoveto{\pgfpoint{420.636047pt}{160.879471pt}}
\pgflineto{\pgfpoint{420.721558pt}{160.837265pt}}
\pgfusepath{stroke}
\pgfpathmoveto{\pgfpoint{420.679138pt}{160.794403pt}}
\pgflineto{\pgfpoint{420.636047pt}{160.879471pt}}
\pgfusepath{stroke}
\pgfpathmoveto{\pgfpoint{420.625671pt}{166.866882pt}}
\pgflineto{\pgfpoint{420.714600pt}{166.825943pt}}
\pgfusepath{stroke}
\pgfpathmoveto{\pgfpoint{420.672272pt}{166.780777pt}}
\pgflineto{\pgfpoint{420.625671pt}{166.866882pt}}
\pgfusepath{stroke}
\pgfpathmoveto{\pgfpoint{420.614868pt}{172.853790pt}}
\pgflineto{\pgfpoint{420.707306pt}{172.814331pt}}
\pgfusepath{stroke}
\pgfpathmoveto{\pgfpoint{420.665161pt}{172.766754pt}}
\pgflineto{\pgfpoint{420.614868pt}{172.853790pt}}
\pgfusepath{stroke}
\pgfpathmoveto{\pgfpoint{420.603577pt}{178.840134pt}}
\pgflineto{\pgfpoint{420.699646pt}{178.802399pt}}
\pgfusepath{stroke}
\pgfpathmoveto{\pgfpoint{420.657776pt}{178.752319pt}}
\pgflineto{\pgfpoint{420.603577pt}{178.840134pt}}
\pgfusepath{stroke}
\pgfpathmoveto{\pgfpoint{420.591858pt}{184.825851pt}}
\pgflineto{\pgfpoint{420.691589pt}{184.790115pt}}
\pgfusepath{stroke}
\pgfpathmoveto{\pgfpoint{420.650208pt}{184.737427pt}}
\pgflineto{\pgfpoint{420.591858pt}{184.825851pt}}
\pgfusepath{stroke}
\pgfpathmoveto{\pgfpoint{420.579620pt}{190.810913pt}}
\pgflineto{\pgfpoint{420.683167pt}{190.777420pt}}
\pgfusepath{stroke}
\pgfpathmoveto{\pgfpoint{420.642334pt}{190.722015pt}}
\pgflineto{\pgfpoint{420.579620pt}{190.810913pt}}
\pgfusepath{stroke}
\pgfpathmoveto{\pgfpoint{420.566956pt}{196.795212pt}}
\pgflineto{\pgfpoint{420.674286pt}{196.764282pt}}
\pgfusepath{stroke}
\pgfpathmoveto{\pgfpoint{420.634277pt}{196.706055pt}}
\pgflineto{\pgfpoint{420.566956pt}{196.795212pt}}
\pgfusepath{stroke}
\pgfpathmoveto{\pgfpoint{420.553711pt}{202.778687pt}}
\pgflineto{\pgfpoint{420.665009pt}{202.750656pt}}
\pgfusepath{stroke}
\pgfpathmoveto{\pgfpoint{420.625946pt}{202.689484pt}}
\pgflineto{\pgfpoint{420.553711pt}{202.778687pt}}
\pgfusepath{stroke}
\pgfpathmoveto{\pgfpoint{420.539978pt}{208.761215pt}}
\pgflineto{\pgfpoint{420.655273pt}{208.736450pt}}
\pgfusepath{stroke}
\pgfpathmoveto{\pgfpoint{420.617371pt}{208.672226pt}}
\pgflineto{\pgfpoint{420.539978pt}{208.761215pt}}
\pgfusepath{stroke}
\pgfpathmoveto{\pgfpoint{420.525635pt}{214.742645pt}}
\pgflineto{\pgfpoint{420.645020pt}{214.721588pt}}
\pgfusepath{stroke}
\pgfpathmoveto{\pgfpoint{420.608490pt}{214.654175pt}}
\pgflineto{\pgfpoint{420.525635pt}{214.742645pt}}
\pgfusepath{stroke}
\pgfpathmoveto{\pgfpoint{420.510742pt}{220.722794pt}}
\pgflineto{\pgfpoint{420.634216pt}{220.705933pt}}
\pgfusepath{stroke}
\pgfpathmoveto{\pgfpoint{420.599396pt}{220.635208pt}}
\pgflineto{\pgfpoint{420.510742pt}{220.722794pt}}
\pgfusepath{stroke}
\pgfpathmoveto{\pgfpoint{420.495239pt}{226.701385pt}}
\pgflineto{\pgfpoint{420.622864pt}{226.689301pt}}
\pgfusepath{stroke}
\pgfpathmoveto{\pgfpoint{420.590088pt}{226.615143pt}}
\pgflineto{\pgfpoint{420.495239pt}{226.701385pt}}
\pgfusepath{stroke}
\pgfpathmoveto{\pgfpoint{420.479309pt}{232.678101pt}}
\pgflineto{\pgfpoint{420.611023pt}{232.671463pt}}
\pgfusepath{stroke}
\pgfpathmoveto{\pgfpoint{420.580688pt}{232.593765pt}}
\pgflineto{\pgfpoint{420.479309pt}{232.678101pt}}
\pgfusepath{stroke}
\pgfpathmoveto{\pgfpoint{420.463013pt}{238.652573pt}}
\pgflineto{\pgfpoint{420.598694pt}{238.652161pt}}
\pgfusepath{stroke}
\pgfpathmoveto{\pgfpoint{420.571320pt}{238.570847pt}}
\pgflineto{\pgfpoint{420.463013pt}{238.652573pt}}
\pgfusepath{stroke}
\pgfpathmoveto{\pgfpoint{420.446808pt}{244.624359pt}}
\pgflineto{\pgfpoint{420.586121pt}{244.631073pt}}
\pgfusepath{stroke}
\pgfpathmoveto{\pgfpoint{420.562286pt}{244.546158pt}}
\pgflineto{\pgfpoint{420.446808pt}{244.624359pt}}
\pgfusepath{stroke}
\pgfpathmoveto{\pgfpoint{420.431152pt}{250.593094pt}}
\pgflineto{\pgfpoint{420.573547pt}{250.607895pt}}
\pgfusepath{stroke}
\pgfpathmoveto{\pgfpoint{420.553955pt}{250.519485pt}}
\pgflineto{\pgfpoint{420.431152pt}{250.593094pt}}
\pgfusepath{stroke}
\pgfpathmoveto{\pgfpoint{420.416809pt}{256.558502pt}}
\pgflineto{\pgfpoint{420.561493pt}{256.582336pt}}
\pgfusepath{stroke}
\pgfpathmoveto{\pgfpoint{420.546875pt}{256.490753pt}}
\pgflineto{\pgfpoint{420.416809pt}{256.558502pt}}
\pgfusepath{stroke}
\pgfpathmoveto{\pgfpoint{420.404694pt}{262.520538pt}}
\pgflineto{\pgfpoint{420.550598pt}{262.554291pt}}
\pgfusepath{stroke}
\pgfpathmoveto{\pgfpoint{420.541656pt}{262.460022pt}}
\pgflineto{\pgfpoint{420.404694pt}{262.520538pt}}
\pgfusepath{stroke}
\pgfpathmoveto{\pgfpoint{420.395905pt}{268.479492pt}}
\pgflineto{\pgfpoint{420.541504pt}{268.523834pt}}
\pgfusepath{stroke}
\pgfpathmoveto{\pgfpoint{420.539001pt}{268.427582pt}}
\pgflineto{\pgfpoint{420.395905pt}{268.479492pt}}
\pgfusepath{stroke}
\pgfpathmoveto{\pgfpoint{420.391418pt}{274.436066pt}}
\pgflineto{\pgfpoint{420.535065pt}{274.491302pt}}
\pgfusepath{stroke}
\pgfpathmoveto{\pgfpoint{420.539490pt}{274.394043pt}}
\pgflineto{\pgfpoint{420.391418pt}{274.436066pt}}
\pgfusepath{stroke}
\pgfpathmoveto{\pgfpoint{420.392029pt}{280.391357pt}}
\pgflineto{\pgfpoint{420.531860pt}{280.457336pt}}
\pgfusepath{stroke}
\pgfpathmoveto{\pgfpoint{420.543488pt}{280.360229pt}}
\pgflineto{\pgfpoint{420.392029pt}{280.391357pt}}
\pgfusepath{stroke}
\pgfpathmoveto{\pgfpoint{420.398041pt}{286.346741pt}}
\pgflineto{\pgfpoint{420.532288pt}{286.422852pt}}
\pgfusepath{stroke}
\pgfpathmoveto{\pgfpoint{420.551086pt}{286.327087pt}}
\pgflineto{\pgfpoint{420.398041pt}{286.346741pt}}
\pgfusepath{stroke}
\pgfpathmoveto{\pgfpoint{420.409302pt}{292.303680pt}}
\pgflineto{\pgfpoint{420.536377pt}{292.388794pt}}
\pgfusepath{stroke}
\pgfpathmoveto{\pgfpoint{420.562012pt}{292.295563pt}}
\pgflineto{\pgfpoint{420.409302pt}{292.303680pt}}
\pgfusepath{stroke}
\pgfpathmoveto{\pgfpoint{420.425110pt}{298.263428pt}}
\pgflineto{\pgfpoint{420.543793pt}{298.356140pt}}
\pgfusepath{stroke}
\pgfpathmoveto{\pgfpoint{420.575684pt}{298.266388pt}}
\pgflineto{\pgfpoint{420.425110pt}{298.263428pt}}
\pgfusepath{stroke}
\pgfpathmoveto{\pgfpoint{420.444489pt}{304.226929pt}}
\pgflineto{\pgfpoint{420.554016pt}{304.325562pt}}
\pgfusepath{stroke}
\pgfpathmoveto{\pgfpoint{420.591309pt}{304.240112pt}}
\pgflineto{\pgfpoint{420.444489pt}{304.226929pt}}
\pgfusepath{stroke}
\pgfpathmoveto{\pgfpoint{420.466248pt}{310.194580pt}}
\pgflineto{\pgfpoint{420.566284pt}{310.297516pt}}
\pgfusepath{stroke}
\pgfpathmoveto{\pgfpoint{420.608032pt}{310.216919pt}}
\pgflineto{\pgfpoint{420.466248pt}{310.194580pt}}
\pgfusepath{stroke}
\pgfpathmoveto{\pgfpoint{420.489197pt}{316.166534pt}}
\pgflineto{\pgfpoint{420.579834pt}{316.272156pt}}
\pgfusepath{stroke}
\pgfpathmoveto{\pgfpoint{420.625092pt}{316.196655pt}}
\pgflineto{\pgfpoint{420.489197pt}{316.166534pt}}
\pgfusepath{stroke}
\pgfpathmoveto{\pgfpoint{420.512451pt}{322.142456pt}}
\pgflineto{\pgfpoint{420.593994pt}{322.249451pt}}
\pgfusepath{stroke}
\pgfpathmoveto{\pgfpoint{420.641907pt}{322.179108pt}}
\pgflineto{\pgfpoint{420.512451pt}{322.142456pt}}
\pgfusepath{stroke}
\pgfpathmoveto{\pgfpoint{420.535156pt}{328.121918pt}}
\pgflineto{\pgfpoint{420.608246pt}{328.229156pt}}
\pgfusepath{stroke}
\pgfpathmoveto{\pgfpoint{420.657959pt}{328.163879pt}}
\pgflineto{\pgfpoint{420.535156pt}{328.121918pt}}
\pgfusepath{stroke}
\pgfpathmoveto{\pgfpoint{420.556885pt}{334.104401pt}}
\pgflineto{\pgfpoint{420.622131pt}{334.210999pt}}
\pgfusepath{stroke}
\pgfpathmoveto{\pgfpoint{420.673035pt}{334.150513pt}}
\pgflineto{\pgfpoint{420.556885pt}{334.104401pt}}
\pgfusepath{stroke}
\pgfpathmoveto{\pgfpoint{420.577332pt}{340.089355pt}}
\pgflineto{\pgfpoint{420.635437pt}{340.194611pt}}
\pgfusepath{stroke}
\pgfpathmoveto{\pgfpoint{420.686951pt}{340.138672pt}}
\pgflineto{\pgfpoint{420.577332pt}{340.089355pt}}
\pgfusepath{stroke}
\pgfpathmoveto{\pgfpoint{420.596252pt}{346.076294pt}}
\pgflineto{\pgfpoint{420.647980pt}{346.179688pt}}
\pgfusepath{stroke}
\pgfpathmoveto{\pgfpoint{420.699707pt}{346.127991pt}}
\pgflineto{\pgfpoint{420.596252pt}{346.076294pt}}
\pgfusepath{stroke}
\pgfpathmoveto{\pgfpoint{420.613678pt}{352.064728pt}}
\pgflineto{\pgfpoint{420.659668pt}{352.165985pt}}
\pgfusepath{stroke}
\pgfpathmoveto{\pgfpoint{420.711243pt}{352.118134pt}}
\pgflineto{\pgfpoint{420.613678pt}{352.064728pt}}
\pgfusepath{stroke}
\pgfpathmoveto{\pgfpoint{420.629608pt}{358.054352pt}}
\pgflineto{\pgfpoint{420.670502pt}{358.153198pt}}
\pgfusepath{stroke}
\pgfpathmoveto{\pgfpoint{420.721649pt}{358.108917pt}}
\pgflineto{\pgfpoint{420.629608pt}{358.054352pt}}
\pgfusepath{stroke}
\pgfpathmoveto{\pgfpoint{420.644135pt}{364.044830pt}}
\pgflineto{\pgfpoint{420.680481pt}{364.141174pt}}
\pgfusepath{stroke}
\pgfpathmoveto{\pgfpoint{420.731018pt}{364.100098pt}}
\pgflineto{\pgfpoint{420.644135pt}{364.044830pt}}
\pgfusepath{stroke}
\pgfpathmoveto{\pgfpoint{420.657349pt}{370.035950pt}}
\pgflineto{\pgfpoint{420.689636pt}{370.129700pt}}
\pgfusepath{stroke}
\pgfpathmoveto{\pgfpoint{420.739441pt}{370.091553pt}}
\pgflineto{\pgfpoint{420.657349pt}{370.035950pt}}
\pgfusepath{stroke}
\pgfpathmoveto{\pgfpoint{426.721375pt}{77.020905pt}}
\pgflineto{\pgfpoint{426.771027pt}{76.977859pt}}
\pgfusepath{stroke}
\pgfpathmoveto{\pgfpoint{426.735260pt}{76.956665pt}}
\pgflineto{\pgfpoint{426.721375pt}{77.020905pt}}
\pgfusepath{stroke}
\pgfpathmoveto{\pgfpoint{426.716644pt}{83.011002pt}}
\pgflineto{\pgfpoint{426.768127pt}{82.967697pt}}
\pgfusepath{stroke}
\pgfpathmoveto{\pgfpoint{426.731873pt}{82.945465pt}}
\pgflineto{\pgfpoint{426.716644pt}{83.011002pt}}
\pgfusepath{stroke}
\pgfpathmoveto{\pgfpoint{426.711670pt}{89.001060pt}}
\pgflineto{\pgfpoint{426.765076pt}{88.957550pt}}
\pgfusepath{stroke}
\pgfpathmoveto{\pgfpoint{426.728271pt}{88.934196pt}}
\pgflineto{\pgfpoint{426.711670pt}{89.001060pt}}
\pgfusepath{stroke}
\pgfpathmoveto{\pgfpoint{426.706360pt}{94.991074pt}}
\pgflineto{\pgfpoint{426.761780pt}{94.947395pt}}
\pgfusepath{stroke}
\pgfpathmoveto{\pgfpoint{426.724487pt}{94.922867pt}}
\pgflineto{\pgfpoint{426.706360pt}{94.991074pt}}
\pgfusepath{stroke}
\pgfpathmoveto{\pgfpoint{426.700745pt}{100.980995pt}}
\pgflineto{\pgfpoint{426.758301pt}{100.937225pt}}
\pgfusepath{stroke}
\pgfpathmoveto{\pgfpoint{426.720520pt}{100.911453pt}}
\pgflineto{\pgfpoint{426.700745pt}{100.980995pt}}
\pgfusepath{stroke}
\pgfpathmoveto{\pgfpoint{426.694763pt}{106.970818pt}}
\pgflineto{\pgfpoint{426.754517pt}{106.927017pt}}
\pgfusepath{stroke}
\pgfpathmoveto{\pgfpoint{426.716309pt}{106.899918pt}}
\pgflineto{\pgfpoint{426.694763pt}{106.970818pt}}
\pgfusepath{stroke}
\pgfpathmoveto{\pgfpoint{426.688446pt}{112.960510pt}}
\pgflineto{\pgfpoint{426.750549pt}{112.916763pt}}
\pgfusepath{stroke}
\pgfpathmoveto{\pgfpoint{426.711884pt}{112.888252pt}}
\pgflineto{\pgfpoint{426.688446pt}{112.960510pt}}
\pgfusepath{stroke}
\pgfpathmoveto{\pgfpoint{426.681763pt}{118.950043pt}}
\pgflineto{\pgfpoint{426.746277pt}{118.906441pt}}
\pgfusepath{stroke}
\pgfpathmoveto{\pgfpoint{426.707214pt}{118.876465pt}}
\pgflineto{\pgfpoint{426.681763pt}{118.950043pt}}
\pgfusepath{stroke}
\pgfpathmoveto{\pgfpoint{426.674744pt}{124.939392pt}}
\pgflineto{\pgfpoint{426.741760pt}{124.896049pt}}
\pgfusepath{stroke}
\pgfpathmoveto{\pgfpoint{426.702362pt}{124.864494pt}}
\pgflineto{\pgfpoint{426.674744pt}{124.939392pt}}
\pgfusepath{stroke}
\pgfpathmoveto{\pgfpoint{426.667297pt}{130.928528pt}}
\pgflineto{\pgfpoint{426.736969pt}{130.885544pt}}
\pgfusepath{stroke}
\pgfpathmoveto{\pgfpoint{426.697266pt}{130.852341pt}}
\pgflineto{\pgfpoint{426.667297pt}{130.928528pt}}
\pgfusepath{stroke}
\pgfpathmoveto{\pgfpoint{426.659485pt}{136.917404pt}}
\pgflineto{\pgfpoint{426.731873pt}{136.874924pt}}
\pgfusepath{stroke}
\pgfpathmoveto{\pgfpoint{426.691895pt}{136.839981pt}}
\pgflineto{\pgfpoint{426.659485pt}{136.917404pt}}
\pgfusepath{stroke}
\pgfpathmoveto{\pgfpoint{426.651245pt}{142.905975pt}}
\pgflineto{\pgfpoint{426.726501pt}{142.864136pt}}
\pgfusepath{stroke}
\pgfpathmoveto{\pgfpoint{426.686340pt}{142.827362pt}}
\pgflineto{\pgfpoint{426.651245pt}{142.905975pt}}
\pgfusepath{stroke}
\pgfpathmoveto{\pgfpoint{426.642639pt}{148.894226pt}}
\pgflineto{\pgfpoint{426.720764pt}{148.853180pt}}
\pgfusepath{stroke}
\pgfpathmoveto{\pgfpoint{426.680542pt}{148.814484pt}}
\pgflineto{\pgfpoint{426.642639pt}{148.894226pt}}
\pgfusepath{stroke}
\pgfpathmoveto{\pgfpoint{426.633575pt}{154.882111pt}}
\pgflineto{\pgfpoint{426.714783pt}{154.842026pt}}
\pgfusepath{stroke}
\pgfpathmoveto{\pgfpoint{426.674500pt}{154.801331pt}}
\pgflineto{\pgfpoint{426.633575pt}{154.882111pt}}
\pgfusepath{stroke}
\pgfpathmoveto{\pgfpoint{426.624115pt}{160.869568pt}}
\pgflineto{\pgfpoint{426.708435pt}{160.830643pt}}
\pgfusepath{stroke}
\pgfpathmoveto{\pgfpoint{426.668213pt}{160.787827pt}}
\pgflineto{\pgfpoint{426.624115pt}{160.869568pt}}
\pgfusepath{stroke}
\pgfpathmoveto{\pgfpoint{426.614258pt}{166.856567pt}}
\pgflineto{\pgfpoint{426.701752pt}{166.818970pt}}
\pgfusepath{stroke}
\pgfpathmoveto{\pgfpoint{426.661713pt}{166.773972pt}}
\pgflineto{\pgfpoint{426.614258pt}{166.856567pt}}
\pgfusepath{stroke}
\pgfpathmoveto{\pgfpoint{426.603973pt}{172.843018pt}}
\pgflineto{\pgfpoint{426.694763pt}{172.807007pt}}
\pgfusepath{stroke}
\pgfpathmoveto{\pgfpoint{426.654999pt}{172.759735pt}}
\pgflineto{\pgfpoint{426.603973pt}{172.843018pt}}
\pgfusepath{stroke}
\pgfpathmoveto{\pgfpoint{426.593262pt}{178.828934pt}}
\pgflineto{\pgfpoint{426.687439pt}{178.794724pt}}
\pgfusepath{stroke}
\pgfpathmoveto{\pgfpoint{426.648071pt}{178.745071pt}}
\pgflineto{\pgfpoint{426.593262pt}{178.828934pt}}
\pgfusepath{stroke}
\pgfpathmoveto{\pgfpoint{426.582184pt}{184.814224pt}}
\pgflineto{\pgfpoint{426.679749pt}{184.782074pt}}
\pgfusepath{stroke}
\pgfpathmoveto{\pgfpoint{426.640961pt}{184.729950pt}}
\pgflineto{\pgfpoint{426.582184pt}{184.814224pt}}
\pgfusepath{stroke}
\pgfpathmoveto{\pgfpoint{426.570679pt}{190.798828pt}}
\pgflineto{\pgfpoint{426.671753pt}{190.768997pt}}
\pgfusepath{stroke}
\pgfpathmoveto{\pgfpoint{426.633636pt}{190.714325pt}}
\pgflineto{\pgfpoint{426.570679pt}{190.798828pt}}
\pgfusepath{stroke}
\pgfpathmoveto{\pgfpoint{426.558777pt}{196.782669pt}}
\pgflineto{\pgfpoint{426.663391pt}{196.755478pt}}
\pgfusepath{stroke}
\pgfpathmoveto{\pgfpoint{426.626160pt}{196.698151pt}}
\pgflineto{\pgfpoint{426.558777pt}{196.782669pt}}
\pgfusepath{stroke}
\pgfpathmoveto{\pgfpoint{426.546448pt}{202.765671pt}}
\pgflineto{\pgfpoint{426.654633pt}{202.741425pt}}
\pgfusepath{stroke}
\pgfpathmoveto{\pgfpoint{426.618469pt}{202.681366pt}}
\pgflineto{\pgfpoint{426.546448pt}{202.765671pt}}
\pgfusepath{stroke}
\pgfpathmoveto{\pgfpoint{426.533722pt}{208.747726pt}}
\pgflineto{\pgfpoint{426.645508pt}{208.726791pt}}
\pgfusepath{stroke}
\pgfpathmoveto{\pgfpoint{426.610596pt}{208.663910pt}}
\pgflineto{\pgfpoint{426.533722pt}{208.747726pt}}
\pgfusepath{stroke}
\pgfpathmoveto{\pgfpoint{426.520569pt}{214.728668pt}}
\pgflineto{\pgfpoint{426.635986pt}{214.711472pt}}
\pgfusepath{stroke}
\pgfpathmoveto{\pgfpoint{426.602600pt}{214.645660pt}}
\pgflineto{\pgfpoint{426.520569pt}{214.728668pt}}
\pgfusepath{stroke}
\pgfpathmoveto{\pgfpoint{426.507080pt}{220.708359pt}}
\pgflineto{\pgfpoint{426.626099pt}{220.695343pt}}
\pgfusepath{stroke}
\pgfpathmoveto{\pgfpoint{426.594482pt}{220.626541pt}}
\pgflineto{\pgfpoint{426.507080pt}{220.708359pt}}
\pgfusepath{stroke}
\pgfpathmoveto{\pgfpoint{426.493225pt}{226.686554pt}}
\pgflineto{\pgfpoint{426.615814pt}{226.678268pt}}
\pgfusepath{stroke}
\pgfpathmoveto{\pgfpoint{426.586304pt}{226.606384pt}}
\pgflineto{\pgfpoint{426.493225pt}{226.686554pt}}
\pgfusepath{stroke}
\pgfpathmoveto{\pgfpoint{426.479248pt}{232.663040pt}}
\pgflineto{\pgfpoint{426.605225pt}{232.660065pt}}
\pgfusepath{stroke}
\pgfpathmoveto{\pgfpoint{426.578247pt}{232.585068pt}}
\pgflineto{\pgfpoint{426.479248pt}{232.663040pt}}
\pgfusepath{stroke}
\pgfpathmoveto{\pgfpoint{426.465271pt}{238.637512pt}}
\pgflineto{\pgfpoint{426.594421pt}{238.640518pt}}
\pgfusepath{stroke}
\pgfpathmoveto{\pgfpoint{426.570404pt}{238.562424pt}}
\pgflineto{\pgfpoint{426.465271pt}{238.637512pt}}
\pgfusepath{stroke}
\pgfpathmoveto{\pgfpoint{426.451660pt}{244.609741pt}}
\pgflineto{\pgfpoint{426.583618pt}{244.619431pt}}
\pgfusepath{stroke}
\pgfpathmoveto{\pgfpoint{426.563049pt}{244.538315pt}}
\pgflineto{\pgfpoint{426.451660pt}{244.609741pt}}
\pgfusepath{stroke}
\pgfpathmoveto{\pgfpoint{426.438904pt}{250.579483pt}}
\pgflineto{\pgfpoint{426.573120pt}{250.596603pt}}
\pgfusepath{stroke}
\pgfpathmoveto{\pgfpoint{426.556519pt}{250.512634pt}}
\pgflineto{\pgfpoint{426.438904pt}{250.579483pt}}
\pgfusepath{stroke}
\pgfpathmoveto{\pgfpoint{426.427490pt}{256.546661pt}}
\pgflineto{\pgfpoint{426.563232pt}{256.571899pt}}
\pgfusepath{stroke}
\pgfpathmoveto{\pgfpoint{426.551239pt}{256.485413pt}}
\pgflineto{\pgfpoint{426.427490pt}{256.546661pt}}
\pgfusepath{stroke}
\pgfpathmoveto{\pgfpoint{426.418213pt}{262.511322pt}}
\pgflineto{\pgfpoint{426.554504pt}{262.545288pt}}
\pgfusepath{stroke}
\pgfpathmoveto{\pgfpoint{426.547638pt}{262.456726pt}}
\pgflineto{\pgfpoint{426.418213pt}{262.511322pt}}
\pgfusepath{stroke}
\pgfpathmoveto{\pgfpoint{426.411804pt}{268.473755pt}}
\pgflineto{\pgfpoint{426.547394pt}{268.516876pt}}
\pgfusepath{stroke}
\pgfpathmoveto{\pgfpoint{426.546143pt}{268.426910pt}}
\pgflineto{\pgfpoint{426.411804pt}{268.473755pt}}
\pgfusepath{stroke}
\pgfpathmoveto{\pgfpoint{426.408875pt}{274.434570pt}}
\pgflineto{\pgfpoint{426.542419pt}{274.487000pt}}
\pgfusepath{stroke}
\pgfpathmoveto{\pgfpoint{426.547180pt}{274.396362pt}}
\pgflineto{\pgfpoint{426.408875pt}{274.434570pt}}
\pgfusepath{stroke}
\pgfpathmoveto{\pgfpoint{426.409912pt}{280.394592pt}}
\pgflineto{\pgfpoint{426.540039pt}{280.456116pt}}
\pgfusepath{stroke}
\pgfpathmoveto{\pgfpoint{426.550964pt}{280.365723pt}}
\pgflineto{\pgfpoint{426.409912pt}{280.394592pt}}
\pgfusepath{stroke}
\pgfpathmoveto{\pgfpoint{426.415161pt}{286.354736pt}}
\pgflineto{\pgfpoint{426.540527pt}{286.424927pt}}
\pgfusepath{stroke}
\pgfpathmoveto{\pgfpoint{426.557556pt}{286.335693pt}}
\pgflineto{\pgfpoint{426.415161pt}{286.354736pt}}
\pgfusepath{stroke}
\pgfpathmoveto{\pgfpoint{426.424500pt}{292.316101pt}}
\pgflineto{\pgfpoint{426.543793pt}{292.394043pt}}
\pgfusepath{stroke}
\pgfpathmoveto{\pgfpoint{426.566711pt}{292.306885pt}}
\pgflineto{\pgfpoint{426.424500pt}{292.316101pt}}
\pgfusepath{stroke}
\pgfpathmoveto{\pgfpoint{426.437469pt}{298.279541pt}}
\pgflineto{\pgfpoint{426.549744pt}{298.364136pt}}
\pgfusepath{stroke}
\pgfpathmoveto{\pgfpoint{426.578033pt}{298.279846pt}}
\pgflineto{\pgfpoint{426.437469pt}{298.279541pt}}
\pgfusepath{stroke}
\pgfpathmoveto{\pgfpoint{426.453308pt}{304.245758pt}}
\pgflineto{\pgfpoint{426.557922pt}{304.335785pt}}
\pgfusepath{stroke}
\pgfpathmoveto{\pgfpoint{426.591003pt}{304.255005pt}}
\pgflineto{\pgfpoint{426.453308pt}{304.245758pt}}
\pgfusepath{stroke}
\pgfpathmoveto{\pgfpoint{426.471313pt}{310.215149pt}}
\pgflineto{\pgfpoint{426.567871pt}{310.309296pt}}
\pgfusepath{stroke}
\pgfpathmoveto{\pgfpoint{426.605042pt}{310.232544pt}}
\pgflineto{\pgfpoint{426.471313pt}{310.215149pt}}
\pgfusepath{stroke}
\pgfpathmoveto{\pgfpoint{426.490570pt}{316.187897pt}}
\pgflineto{\pgfpoint{426.579041pt}{316.284912pt}}
\pgfusepath{stroke}
\pgfpathmoveto{\pgfpoint{426.619537pt}{316.212433pt}}
\pgflineto{\pgfpoint{426.490570pt}{316.187897pt}}
\pgfusepath{stroke}
\pgfpathmoveto{\pgfpoint{426.510345pt}{322.163849pt}}
\pgflineto{\pgfpoint{426.590881pt}{322.262573pt}}
\pgfusepath{stroke}
\pgfpathmoveto{\pgfpoint{426.634033pt}{322.194519pt}}
\pgflineto{\pgfpoint{426.510345pt}{322.163849pt}}
\pgfusepath{stroke}
\pgfpathmoveto{\pgfpoint{426.530029pt}{328.142761pt}}
\pgflineto{\pgfpoint{426.603058pt}{328.242279pt}}
\pgfusepath{stroke}
\pgfpathmoveto{\pgfpoint{426.648132pt}{328.178558pt}}
\pgflineto{\pgfpoint{426.530029pt}{328.142761pt}}
\pgfusepath{stroke}
\pgfpathmoveto{\pgfpoint{426.549194pt}{334.124329pt}}
\pgflineto{\pgfpoint{426.615112pt}{334.223755pt}}
\pgfusepath{stroke}
\pgfpathmoveto{\pgfpoint{426.661621pt}{334.164307pt}}
\pgflineto{\pgfpoint{426.549194pt}{334.124329pt}}
\pgfusepath{stroke}
\pgfpathmoveto{\pgfpoint{426.567505pt}{340.108063pt}}
\pgflineto{\pgfpoint{426.626892pt}{340.206818pt}}
\pgfusepath{stroke}
\pgfpathmoveto{\pgfpoint{426.674255pt}{340.151428pt}}
\pgflineto{\pgfpoint{426.567505pt}{340.108063pt}}
\pgfusepath{stroke}
\pgfpathmoveto{\pgfpoint{426.584717pt}{346.093689pt}}
\pgflineto{\pgfpoint{426.638123pt}{346.191223pt}}
\pgfusepath{stroke}
\pgfpathmoveto{\pgfpoint{426.685974pt}{346.139648pt}}
\pgflineto{\pgfpoint{426.584717pt}{346.093689pt}}
\pgfusepath{stroke}
\pgfpathmoveto{\pgfpoint{426.600800pt}{352.080811pt}}
\pgflineto{\pgfpoint{426.648804pt}{352.176758pt}}
\pgfusepath{stroke}
\pgfpathmoveto{\pgfpoint{426.696777pt}{352.128784pt}}
\pgflineto{\pgfpoint{426.600800pt}{352.080811pt}}
\pgfusepath{stroke}
\pgfpathmoveto{\pgfpoint{426.615723pt}{358.069092pt}}
\pgflineto{\pgfpoint{426.658783pt}{358.163239pt}}
\pgfusepath{stroke}
\pgfpathmoveto{\pgfpoint{426.706635pt}{358.118561pt}}
\pgflineto{\pgfpoint{426.615723pt}{358.069092pt}}
\pgfusepath{stroke}
\pgfpathmoveto{\pgfpoint{426.629425pt}{364.058350pt}}
\pgflineto{\pgfpoint{426.668091pt}{364.150452pt}}
\pgfusepath{stroke}
\pgfpathmoveto{\pgfpoint{426.715637pt}{364.108826pt}}
\pgflineto{\pgfpoint{426.629425pt}{364.058350pt}}
\pgfusepath{stroke}
\pgfpathmoveto{\pgfpoint{426.642059pt}{370.048279pt}}
\pgflineto{\pgfpoint{426.676758pt}{370.138245pt}}
\pgfusepath{stroke}
\pgfpathmoveto{\pgfpoint{426.723816pt}{370.099426pt}}
\pgflineto{\pgfpoint{426.642059pt}{370.048279pt}}
\pgfusepath{stroke}
\pgfpathmoveto{\pgfpoint{432.706879pt}{77.016327pt}}
\pgflineto{\pgfpoint{432.756775pt}{76.975067pt}}
\pgfusepath{stroke}
\pgfpathmoveto{\pgfpoint{432.722046pt}{76.953369pt}}
\pgflineto{\pgfpoint{432.706879pt}{77.016327pt}}
\pgfusepath{stroke}
\pgfpathmoveto{\pgfpoint{432.702209pt}{83.006165pt}}
\pgflineto{\pgfpoint{432.753906pt}{82.964722pt}}
\pgfusepath{stroke}
\pgfpathmoveto{\pgfpoint{432.718689pt}{82.941986pt}}
\pgflineto{\pgfpoint{432.702209pt}{83.006165pt}}
\pgfusepath{stroke}
\pgfpathmoveto{\pgfpoint{432.697296pt}{88.995956pt}}
\pgflineto{\pgfpoint{432.750824pt}{88.954384pt}}
\pgfusepath{stroke}
\pgfpathmoveto{\pgfpoint{432.715179pt}{88.930565pt}}
\pgflineto{\pgfpoint{432.697296pt}{88.995956pt}}
\pgfusepath{stroke}
\pgfpathmoveto{\pgfpoint{432.692017pt}{94.985641pt}}
\pgflineto{\pgfpoint{432.747589pt}{94.944008pt}}
\pgfusepath{stroke}
\pgfpathmoveto{\pgfpoint{432.711487pt}{94.919029pt}}
\pgflineto{\pgfpoint{432.692017pt}{94.985641pt}}
\pgfusepath{stroke}
\pgfpathmoveto{\pgfpoint{432.686523pt}{100.975258pt}}
\pgflineto{\pgfpoint{432.744080pt}{100.933609pt}}
\pgfusepath{stroke}
\pgfpathmoveto{\pgfpoint{432.707581pt}{100.907410pt}}
\pgflineto{\pgfpoint{432.686523pt}{100.975258pt}}
\pgfusepath{stroke}
\pgfpathmoveto{\pgfpoint{432.680664pt}{106.964752pt}}
\pgflineto{\pgfpoint{432.740387pt}{106.923172pt}}
\pgfusepath{stroke}
\pgfpathmoveto{\pgfpoint{432.703491pt}{106.895668pt}}
\pgflineto{\pgfpoint{432.680664pt}{106.964752pt}}
\pgfusepath{stroke}
\pgfpathmoveto{\pgfpoint{432.674530pt}{112.954094pt}}
\pgflineto{\pgfpoint{432.736450pt}{112.912682pt}}
\pgfusepath{stroke}
\pgfpathmoveto{\pgfpoint{432.699219pt}{112.883804pt}}
\pgflineto{\pgfpoint{432.674530pt}{112.954094pt}}
\pgfusepath{stroke}
\pgfpathmoveto{\pgfpoint{432.668030pt}{118.943275pt}}
\pgflineto{\pgfpoint{432.732300pt}{118.902107pt}}
\pgfusepath{stroke}
\pgfpathmoveto{\pgfpoint{432.694733pt}{118.871788pt}}
\pgflineto{\pgfpoint{432.668030pt}{118.943275pt}}
\pgfusepath{stroke}
\pgfpathmoveto{\pgfpoint{432.661194pt}{124.932274pt}}
\pgflineto{\pgfpoint{432.727875pt}{124.891441pt}}
\pgfusepath{stroke}
\pgfpathmoveto{\pgfpoint{432.690063pt}{124.859604pt}}
\pgflineto{\pgfpoint{432.661194pt}{124.932274pt}}
\pgfusepath{stroke}
\pgfpathmoveto{\pgfpoint{432.653992pt}{130.921021pt}}
\pgflineto{\pgfpoint{432.723175pt}{130.880676pt}}
\pgfusepath{stroke}
\pgfpathmoveto{\pgfpoint{432.685120pt}{130.847229pt}}
\pgflineto{\pgfpoint{432.653992pt}{130.921021pt}}
\pgfusepath{stroke}
\pgfpathmoveto{\pgfpoint{432.646423pt}{136.909515pt}}
\pgflineto{\pgfpoint{432.718231pt}{136.869751pt}}
\pgfusepath{stroke}
\pgfpathmoveto{\pgfpoint{432.680023pt}{136.834625pt}}
\pgflineto{\pgfpoint{432.646423pt}{136.909515pt}}
\pgfusepath{stroke}
\pgfpathmoveto{\pgfpoint{432.638550pt}{142.897720pt}}
\pgflineto{\pgfpoint{432.713013pt}{142.858673pt}}
\pgfusepath{stroke}
\pgfpathmoveto{\pgfpoint{432.674683pt}{142.821808pt}}
\pgflineto{\pgfpoint{432.638550pt}{142.897720pt}}
\pgfusepath{stroke}
\pgfpathmoveto{\pgfpoint{432.630249pt}{148.885590pt}}
\pgflineto{\pgfpoint{432.707520pt}{148.847427pt}}
\pgfusepath{stroke}
\pgfpathmoveto{\pgfpoint{432.669159pt}{148.808716pt}}
\pgflineto{\pgfpoint{432.630249pt}{148.885590pt}}
\pgfusepath{stroke}
\pgfpathmoveto{\pgfpoint{432.621582pt}{154.873077pt}}
\pgflineto{\pgfpoint{432.701721pt}{154.835968pt}}
\pgfusepath{stroke}
\pgfpathmoveto{\pgfpoint{432.663422pt}{154.795334pt}}
\pgflineto{\pgfpoint{432.621582pt}{154.873077pt}}
\pgfusepath{stroke}
\pgfpathmoveto{\pgfpoint{432.612610pt}{160.860138pt}}
\pgflineto{\pgfpoint{432.695618pt}{160.824265pt}}
\pgfusepath{stroke}
\pgfpathmoveto{\pgfpoint{432.657501pt}{160.781616pt}}
\pgflineto{\pgfpoint{432.612610pt}{160.860138pt}}
\pgfusepath{stroke}
\pgfpathmoveto{\pgfpoint{432.603210pt}{166.846741pt}}
\pgflineto{\pgfpoint{432.689209pt}{166.812286pt}}
\pgfusepath{stroke}
\pgfpathmoveto{\pgfpoint{432.651367pt}{166.767578pt}}
\pgflineto{\pgfpoint{432.603210pt}{166.846741pt}}
\pgfusepath{stroke}
\pgfpathmoveto{\pgfpoint{432.593475pt}{172.832825pt}}
\pgflineto{\pgfpoint{432.682556pt}{172.800018pt}}
\pgfusepath{stroke}
\pgfpathmoveto{\pgfpoint{432.645050pt}{172.753128pt}}
\pgflineto{\pgfpoint{432.593475pt}{172.832825pt}}
\pgfusepath{stroke}
\pgfpathmoveto{\pgfpoint{432.583374pt}{178.818359pt}}
\pgflineto{\pgfpoint{432.675598pt}{178.787415pt}}
\pgfusepath{stroke}
\pgfpathmoveto{\pgfpoint{432.638580pt}{178.738281pt}}
\pgflineto{\pgfpoint{432.583374pt}{178.818359pt}}
\pgfusepath{stroke}
\pgfpathmoveto{\pgfpoint{432.572937pt}{184.803268pt}}
\pgflineto{\pgfpoint{432.668304pt}{184.774429pt}}
\pgfusepath{stroke}
\pgfpathmoveto{\pgfpoint{432.631927pt}{184.722977pt}}
\pgflineto{\pgfpoint{432.572937pt}{184.803268pt}}
\pgfusepath{stroke}
\pgfpathmoveto{\pgfpoint{432.562164pt}{190.787506pt}}
\pgflineto{\pgfpoint{432.660706pt}{190.761047pt}}
\pgfusepath{stroke}
\pgfpathmoveto{\pgfpoint{432.625122pt}{190.707184pt}}
\pgflineto{\pgfpoint{432.562164pt}{190.787506pt}}
\pgfusepath{stroke}
\pgfpathmoveto{\pgfpoint{432.551025pt}{196.770981pt}}
\pgflineto{\pgfpoint{432.652832pt}{196.747177pt}}
\pgfusepath{stroke}
\pgfpathmoveto{\pgfpoint{432.618225pt}{196.690857pt}}
\pgflineto{\pgfpoint{432.551025pt}{196.770981pt}}
\pgfusepath{stroke}
\pgfpathmoveto{\pgfpoint{432.539612pt}{202.753616pt}}
\pgflineto{\pgfpoint{432.644653pt}{202.732788pt}}
\pgfusepath{stroke}
\pgfpathmoveto{\pgfpoint{432.611145pt}{202.673935pt}}
\pgflineto{\pgfpoint{432.539612pt}{202.753616pt}}
\pgfusepath{stroke}
\pgfpathmoveto{\pgfpoint{432.527893pt}{208.735321pt}}
\pgflineto{\pgfpoint{432.636169pt}{208.717819pt}}
\pgfusepath{stroke}
\pgfpathmoveto{\pgfpoint{432.604004pt}{208.656342pt}}
\pgflineto{\pgfpoint{432.527893pt}{208.735321pt}}
\pgfusepath{stroke}
\pgfpathmoveto{\pgfpoint{432.515930pt}{214.715958pt}}
\pgflineto{\pgfpoint{432.627380pt}{214.702164pt}}
\pgfusepath{stroke}
\pgfpathmoveto{\pgfpoint{432.596802pt}{214.638031pt}}
\pgflineto{\pgfpoint{432.515930pt}{214.715958pt}}
\pgfusepath{stroke}
\pgfpathmoveto{\pgfpoint{432.503723pt}{220.695389pt}}
\pgflineto{\pgfpoint{432.618317pt}{220.685715pt}}
\pgfusepath{stroke}
\pgfpathmoveto{\pgfpoint{432.589600pt}{220.618881pt}}
\pgflineto{\pgfpoint{432.503723pt}{220.695389pt}}
\pgfusepath{stroke}
\pgfpathmoveto{\pgfpoint{432.491425pt}{226.673447pt}}
\pgflineto{\pgfpoint{432.609070pt}{226.668381pt}}
\pgfusepath{stroke}
\pgfpathmoveto{\pgfpoint{432.582489pt}{226.598801pt}}
\pgflineto{\pgfpoint{432.491425pt}{226.673447pt}}
\pgfusepath{stroke}
\pgfpathmoveto{\pgfpoint{432.479187pt}{232.649963pt}}
\pgflineto{\pgfpoint{432.599640pt}{232.649994pt}}
\pgfusepath{stroke}
\pgfpathmoveto{\pgfpoint{432.575562pt}{232.577698pt}}
\pgflineto{\pgfpoint{432.479187pt}{232.649963pt}}
\pgfusepath{stroke}
\pgfpathmoveto{\pgfpoint{432.467163pt}{238.624725pt}}
\pgflineto{\pgfpoint{432.590210pt}{238.630432pt}}
\pgfusepath{stroke}
\pgfpathmoveto{\pgfpoint{432.569031pt}{238.555466pt}}
\pgflineto{\pgfpoint{432.467163pt}{238.624725pt}}
\pgfusepath{stroke}
\pgfpathmoveto{\pgfpoint{432.455719pt}{244.597610pt}}
\pgflineto{\pgfpoint{432.580933pt}{244.609543pt}}
\pgfusepath{stroke}
\pgfpathmoveto{\pgfpoint{432.563049pt}{244.532043pt}}
\pgflineto{\pgfpoint{432.455719pt}{244.597610pt}}
\pgfusepath{stroke}
\pgfpathmoveto{\pgfpoint{432.445221pt}{250.568497pt}}
\pgflineto{\pgfpoint{432.572052pt}{250.587219pt}}
\pgfusepath{stroke}
\pgfpathmoveto{\pgfpoint{432.557892pt}{250.507385pt}}
\pgflineto{\pgfpoint{432.445221pt}{250.568497pt}}
\pgfusepath{stroke}
\pgfpathmoveto{\pgfpoint{432.436127pt}{256.537354pt}}
\pgflineto{\pgfpoint{432.563873pt}{256.563385pt}}
\pgfusepath{stroke}
\pgfpathmoveto{\pgfpoint{432.553955pt}{256.481537pt}}
\pgflineto{\pgfpoint{432.436127pt}{256.537354pt}}
\pgfusepath{stroke}
\pgfpathmoveto{\pgfpoint{432.428894pt}{262.504333pt}}
\pgflineto{\pgfpoint{432.556763pt}{262.538086pt}}
\pgfusepath{stroke}
\pgfpathmoveto{\pgfpoint{432.551453pt}{262.454620pt}}
\pgflineto{\pgfpoint{432.428894pt}{262.504333pt}}
\pgfusepath{stroke}
\pgfpathmoveto{\pgfpoint{432.424133pt}{268.469696pt}}
\pgflineto{\pgfpoint{432.551086pt}{268.511444pt}}
\pgfusepath{stroke}
\pgfpathmoveto{\pgfpoint{432.550720pt}{268.426941pt}}
\pgflineto{\pgfpoint{432.424133pt}{268.469696pt}}
\pgfusepath{stroke}
\pgfpathmoveto{\pgfpoint{432.422241pt}{274.433929pt}}
\pgflineto{\pgfpoint{432.547241pt}{274.483734pt}}
\pgfusepath{stroke}
\pgfpathmoveto{\pgfpoint{432.552094pt}{274.398804pt}}
\pgflineto{\pgfpoint{432.422241pt}{274.433929pt}}
\pgfusepath{stroke}
\pgfpathmoveto{\pgfpoint{432.423553pt}{280.397644pt}}
\pgflineto{\pgfpoint{432.545441pt}{280.455322pt}}
\pgfusepath{stroke}
\pgfpathmoveto{\pgfpoint{432.555664pt}{280.370667pt}}
\pgflineto{\pgfpoint{432.423553pt}{280.397644pt}}
\pgfusepath{stroke}
\pgfpathmoveto{\pgfpoint{432.428192pt}{286.361542pt}}
\pgflineto{\pgfpoint{432.545837pt}{286.426636pt}}
\pgfusepath{stroke}
\pgfpathmoveto{\pgfpoint{432.561401pt}{286.343018pt}}
\pgflineto{\pgfpoint{432.428192pt}{286.361542pt}}
\pgfusepath{stroke}
\pgfpathmoveto{\pgfpoint{432.436096pt}{292.326355pt}}
\pgflineto{\pgfpoint{432.548553pt}{292.398254pt}}
\pgfusepath{stroke}
\pgfpathmoveto{\pgfpoint{432.569183pt}{292.316376pt}}
\pgflineto{\pgfpoint{432.436096pt}{292.326355pt}}
\pgfusepath{stroke}
\pgfpathmoveto{\pgfpoint{432.446838pt}{298.292786pt}}
\pgflineto{\pgfpoint{432.553406pt}{298.370575pt}}
\pgfusepath{stroke}
\pgfpathmoveto{\pgfpoint{432.578735pt}{298.291107pt}}
\pgflineto{\pgfpoint{432.446838pt}{298.292786pt}}
\pgfusepath{stroke}
\pgfpathmoveto{\pgfpoint{432.460083pt}{304.261353pt}}
\pgflineto{\pgfpoint{432.560059pt}{304.344055pt}}
\pgfusepath{stroke}
\pgfpathmoveto{\pgfpoint{432.589691pt}{304.267517pt}}
\pgflineto{\pgfpoint{432.460083pt}{304.261353pt}}
\pgfusepath{stroke}
\pgfpathmoveto{\pgfpoint{432.475128pt}{310.232361pt}}
\pgflineto{\pgfpoint{432.568237pt}{310.318970pt}}
\pgfusepath{stroke}
\pgfpathmoveto{\pgfpoint{432.601562pt}{310.245789pt}}
\pgflineto{\pgfpoint{432.475128pt}{310.232361pt}}
\pgfusepath{stroke}
\pgfpathmoveto{\pgfpoint{432.491394pt}{316.205994pt}}
\pgflineto{\pgfpoint{432.577545pt}{316.295532pt}}
\pgfusepath{stroke}
\pgfpathmoveto{\pgfpoint{432.614014pt}{316.225952pt}}
\pgflineto{\pgfpoint{432.491394pt}{316.205994pt}}
\pgfusepath{stroke}
\pgfpathmoveto{\pgfpoint{432.508362pt}{322.182281pt}}
\pgflineto{\pgfpoint{432.587555pt}{322.273743pt}}
\pgfusepath{stroke}
\pgfpathmoveto{\pgfpoint{432.626587pt}{322.207947pt}}
\pgflineto{\pgfpoint{432.508362pt}{322.182281pt}}
\pgfusepath{stroke}
\pgfpathmoveto{\pgfpoint{432.525452pt}{328.161011pt}}
\pgflineto{\pgfpoint{432.597961pt}{328.253601pt}}
\pgfusepath{stroke}
\pgfpathmoveto{\pgfpoint{432.639008pt}{328.191589pt}}
\pgflineto{\pgfpoint{432.525452pt}{328.161011pt}}
\pgfusepath{stroke}
\pgfpathmoveto{\pgfpoint{432.542358pt}{334.141998pt}}
\pgflineto{\pgfpoint{432.608459pt}{334.234955pt}}
\pgfusepath{stroke}
\pgfpathmoveto{\pgfpoint{432.651001pt}{334.176697pt}}
\pgflineto{\pgfpoint{432.542358pt}{334.141998pt}}
\pgfusepath{stroke}
\pgfpathmoveto{\pgfpoint{432.558716pt}{340.124969pt}}
\pgflineto{\pgfpoint{432.618835pt}{340.217682pt}}
\pgfusepath{stroke}
\pgfpathmoveto{\pgfpoint{432.662445pt}{340.163086pt}}
\pgflineto{\pgfpoint{432.558716pt}{340.124969pt}}
\pgfusepath{stroke}
\pgfpathmoveto{\pgfpoint{432.574341pt}{346.109619pt}}
\pgflineto{\pgfpoint{432.628906pt}{346.201660pt}}
\pgfusepath{stroke}
\pgfpathmoveto{\pgfpoint{432.673187pt}{346.150513pt}}
\pgflineto{\pgfpoint{432.574341pt}{346.109619pt}}
\pgfusepath{stroke}
\pgfpathmoveto{\pgfpoint{432.589111pt}{352.095703pt}}
\pgflineto{\pgfpoint{432.638550pt}{352.186646pt}}
\pgfusepath{stroke}
\pgfpathmoveto{\pgfpoint{432.683228pt}{352.138794pt}}
\pgflineto{\pgfpoint{432.589111pt}{352.095703pt}}
\pgfusepath{stroke}
\pgfpathmoveto{\pgfpoint{432.602936pt}{358.082947pt}}
\pgflineto{\pgfpoint{432.647766pt}{358.172546pt}}
\pgfusepath{stroke}
\pgfpathmoveto{\pgfpoint{432.692505pt}{358.127747pt}}
\pgflineto{\pgfpoint{432.602936pt}{358.082947pt}}
\pgfusepath{stroke}
\pgfpathmoveto{\pgfpoint{432.615845pt}{364.071167pt}}
\pgflineto{\pgfpoint{432.656372pt}{364.159119pt}}
\pgfusepath{stroke}
\pgfpathmoveto{\pgfpoint{432.701050pt}{364.117218pt}}
\pgflineto{\pgfpoint{432.615845pt}{364.071167pt}}
\pgfusepath{stroke}
\pgfpathmoveto{\pgfpoint{432.627838pt}{370.060089pt}}
\pgflineto{\pgfpoint{432.664490pt}{370.146362pt}}
\pgfusepath{stroke}
\pgfpathmoveto{\pgfpoint{432.708923pt}{370.107117pt}}
\pgflineto{\pgfpoint{432.627838pt}{370.060089pt}}
\pgfusepath{stroke}
\pgfpathmoveto{\pgfpoint{438.692657pt}{77.011810pt}}
\pgflineto{\pgfpoint{438.742737pt}{76.972275pt}}
\pgfusepath{stroke}
\pgfpathmoveto{\pgfpoint{438.708984pt}{76.950134pt}}
\pgflineto{\pgfpoint{438.692657pt}{77.011810pt}}
\pgfusepath{stroke}
\pgfpathmoveto{\pgfpoint{438.688049pt}{83.001389pt}}
\pgflineto{\pgfpoint{438.739838pt}{82.961761pt}}
\pgfusepath{stroke}
\pgfpathmoveto{\pgfpoint{438.705719pt}{82.938599pt}}
\pgflineto{\pgfpoint{438.688049pt}{83.001389pt}}
\pgfusepath{stroke}
\pgfpathmoveto{\pgfpoint{438.683197pt}{88.990906pt}}
\pgflineto{\pgfpoint{438.736786pt}{88.951225pt}}
\pgfusepath{stroke}
\pgfpathmoveto{\pgfpoint{438.702271pt}{88.926994pt}}
\pgflineto{\pgfpoint{438.683197pt}{88.990906pt}}
\pgfusepath{stroke}
\pgfpathmoveto{\pgfpoint{438.678040pt}{94.980324pt}}
\pgflineto{\pgfpoint{438.733551pt}{94.940659pt}}
\pgfusepath{stroke}
\pgfpathmoveto{\pgfpoint{438.698669pt}{94.915283pt}}
\pgflineto{\pgfpoint{438.678040pt}{94.980324pt}}
\pgfusepath{stroke}
\pgfpathmoveto{\pgfpoint{438.672638pt}{100.969627pt}}
\pgflineto{\pgfpoint{438.730103pt}{100.930054pt}}
\pgfusepath{stroke}
\pgfpathmoveto{\pgfpoint{438.694855pt}{100.903481pt}}
\pgflineto{\pgfpoint{438.672638pt}{100.969627pt}}
\pgfusepath{stroke}
\pgfpathmoveto{\pgfpoint{438.666931pt}{106.958832pt}}
\pgflineto{\pgfpoint{438.726501pt}{106.919395pt}}
\pgfusepath{stroke}
\pgfpathmoveto{\pgfpoint{438.690918pt}{106.891548pt}}
\pgflineto{\pgfpoint{438.666931pt}{106.958832pt}}
\pgfusepath{stroke}
\pgfpathmoveto{\pgfpoint{438.660919pt}{112.947861pt}}
\pgflineto{\pgfpoint{438.722626pt}{112.908661pt}}
\pgfusepath{stroke}
\pgfpathmoveto{\pgfpoint{438.686768pt}{112.879486pt}}
\pgflineto{\pgfpoint{438.660919pt}{112.947861pt}}
\pgfusepath{stroke}
\pgfpathmoveto{\pgfpoint{438.654633pt}{118.936722pt}}
\pgflineto{\pgfpoint{438.718536pt}{118.897865pt}}
\pgfusepath{stroke}
\pgfpathmoveto{\pgfpoint{438.682434pt}{118.867279pt}}
\pgflineto{\pgfpoint{438.654633pt}{118.936722pt}}
\pgfusepath{stroke}
\pgfpathmoveto{\pgfpoint{438.648010pt}{124.925369pt}}
\pgflineto{\pgfpoint{438.714233pt}{124.886948pt}}
\pgfusepath{stroke}
\pgfpathmoveto{\pgfpoint{438.677917pt}{124.854897pt}}
\pgflineto{\pgfpoint{438.648010pt}{124.925369pt}}
\pgfusepath{stroke}
\pgfpathmoveto{\pgfpoint{438.641052pt}{130.913788pt}}
\pgflineto{\pgfpoint{438.709656pt}{130.875916pt}}
\pgfusepath{stroke}
\pgfpathmoveto{\pgfpoint{438.673218pt}{130.842316pt}}
\pgflineto{\pgfpoint{438.641052pt}{130.913788pt}}
\pgfusepath{stroke}
\pgfpathmoveto{\pgfpoint{438.633789pt}{136.901947pt}}
\pgflineto{\pgfpoint{438.704865pt}{136.864746pt}}
\pgfusepath{stroke}
\pgfpathmoveto{\pgfpoint{438.668335pt}{136.829529pt}}
\pgflineto{\pgfpoint{438.633789pt}{136.901947pt}}
\pgfusepath{stroke}
\pgfpathmoveto{\pgfpoint{438.626190pt}{142.889801pt}}
\pgflineto{\pgfpoint{438.699799pt}{142.853409pt}}
\pgfusepath{stroke}
\pgfpathmoveto{\pgfpoint{438.663269pt}{142.816513pt}}
\pgflineto{\pgfpoint{438.626190pt}{142.889801pt}}
\pgfusepath{stroke}
\pgfpathmoveto{\pgfpoint{438.618286pt}{148.877319pt}}
\pgflineto{\pgfpoint{438.694519pt}{148.841888pt}}
\pgfusepath{stroke}
\pgfpathmoveto{\pgfpoint{438.657990pt}{148.803223pt}}
\pgflineto{\pgfpoint{438.618286pt}{148.877319pt}}
\pgfusepath{stroke}
\pgfpathmoveto{\pgfpoint{438.610016pt}{154.864441pt}}
\pgflineto{\pgfpoint{438.688904pt}{154.830139pt}}
\pgfusepath{stroke}
\pgfpathmoveto{\pgfpoint{438.652557pt}{154.789642pt}}
\pgflineto{\pgfpoint{438.610016pt}{154.864441pt}}
\pgfusepath{stroke}
\pgfpathmoveto{\pgfpoint{438.601440pt}{160.851166pt}}
\pgflineto{\pgfpoint{438.683105pt}{160.818161pt}}
\pgfusepath{stroke}
\pgfpathmoveto{\pgfpoint{438.646973pt}{160.775772pt}}
\pgflineto{\pgfpoint{438.601440pt}{160.851166pt}}
\pgfusepath{stroke}
\pgfpathmoveto{\pgfpoint{438.592529pt}{166.837433pt}}
\pgflineto{\pgfpoint{438.677002pt}{166.805908pt}}
\pgfusepath{stroke}
\pgfpathmoveto{\pgfpoint{438.641205pt}{166.761536pt}}
\pgflineto{\pgfpoint{438.592529pt}{166.837433pt}}
\pgfusepath{stroke}
\pgfpathmoveto{\pgfpoint{438.583313pt}{172.823196pt}}
\pgflineto{\pgfpoint{438.670654pt}{172.793365pt}}
\pgfusepath{stroke}
\pgfpathmoveto{\pgfpoint{438.635284pt}{172.746933pt}}
\pgflineto{\pgfpoint{438.583313pt}{172.823196pt}}
\pgfusepath{stroke}
\pgfpathmoveto{\pgfpoint{438.573792pt}{178.808395pt}}
\pgflineto{\pgfpoint{438.664001pt}{178.780487pt}}
\pgfusepath{stroke}
\pgfpathmoveto{\pgfpoint{438.629211pt}{178.731934pt}}
\pgflineto{\pgfpoint{438.573792pt}{178.808395pt}}
\pgfusepath{stroke}
\pgfpathmoveto{\pgfpoint{438.564026pt}{184.792999pt}}
\pgflineto{\pgfpoint{438.657135pt}{184.767212pt}}
\pgfusepath{stroke}
\pgfpathmoveto{\pgfpoint{438.623047pt}{184.716507pt}}
\pgflineto{\pgfpoint{438.564026pt}{184.792999pt}}
\pgfusepath{stroke}
\pgfpathmoveto{\pgfpoint{438.553955pt}{190.776932pt}}
\pgflineto{\pgfpoint{438.649994pt}{190.753540pt}}
\pgfusepath{stroke}
\pgfpathmoveto{\pgfpoint{438.616760pt}{190.700592pt}}
\pgflineto{\pgfpoint{438.553955pt}{190.776932pt}}
\pgfusepath{stroke}
\pgfpathmoveto{\pgfpoint{438.543640pt}{196.760117pt}}
\pgflineto{\pgfpoint{438.642609pt}{196.739410pt}}
\pgfusepath{stroke}
\pgfpathmoveto{\pgfpoint{438.610352pt}{196.684158pt}}
\pgflineto{\pgfpoint{438.543640pt}{196.760117pt}}
\pgfusepath{stroke}
\pgfpathmoveto{\pgfpoint{438.533051pt}{202.742508pt}}
\pgflineto{\pgfpoint{438.634949pt}{202.724747pt}}
\pgfusepath{stroke}
\pgfpathmoveto{\pgfpoint{438.603943pt}{202.667160pt}}
\pgflineto{\pgfpoint{438.533051pt}{202.742508pt}}
\pgfusepath{stroke}
\pgfpathmoveto{\pgfpoint{438.522308pt}{208.723984pt}}
\pgflineto{\pgfpoint{438.627106pt}{208.709518pt}}
\pgfusepath{stroke}
\pgfpathmoveto{\pgfpoint{438.597473pt}{208.649536pt}}
\pgflineto{\pgfpoint{438.522308pt}{208.723984pt}}
\pgfusepath{stroke}
\pgfpathmoveto{\pgfpoint{438.511414pt}{214.704453pt}}
\pgflineto{\pgfpoint{438.619019pt}{214.693619pt}}
\pgfusepath{stroke}
\pgfpathmoveto{\pgfpoint{438.591003pt}{214.631226pt}}
\pgflineto{\pgfpoint{438.511414pt}{214.704453pt}}
\pgfusepath{stroke}
\pgfpathmoveto{\pgfpoint{438.500488pt}{220.683792pt}}
\pgflineto{\pgfpoint{438.610809pt}{220.676987pt}}
\pgfusepath{stroke}
\pgfpathmoveto{\pgfpoint{438.584656pt}{220.612152pt}}
\pgflineto{\pgfpoint{438.500488pt}{220.683792pt}}
\pgfusepath{stroke}
\pgfpathmoveto{\pgfpoint{438.489563pt}{226.661896pt}}
\pgflineto{\pgfpoint{438.602478pt}{226.659515pt}}
\pgfusepath{stroke}
\pgfpathmoveto{\pgfpoint{438.578430pt}{226.592255pt}}
\pgflineto{\pgfpoint{438.489563pt}{226.661896pt}}
\pgfusepath{stroke}
\pgfpathmoveto{\pgfpoint{438.478851pt}{232.638626pt}}
\pgflineto{\pgfpoint{438.594116pt}{232.641113pt}}
\pgfusepath{stroke}
\pgfpathmoveto{\pgfpoint{438.572540pt}{232.571472pt}}
\pgflineto{\pgfpoint{438.478851pt}{232.638626pt}}
\pgfusepath{stroke}
\pgfpathmoveto{\pgfpoint{438.468536pt}{238.613861pt}}
\pgflineto{\pgfpoint{438.585815pt}{238.621658pt}}
\pgfusepath{stroke}
\pgfpathmoveto{\pgfpoint{438.567047pt}{238.549728pt}}
\pgflineto{\pgfpoint{438.468536pt}{238.613861pt}}
\pgfusepath{stroke}
\pgfpathmoveto{\pgfpoint{438.458862pt}{244.587509pt}}
\pgflineto{\pgfpoint{438.577820pt}{244.601089pt}}
\pgfusepath{stroke}
\pgfpathmoveto{\pgfpoint{438.562164pt}{244.527008pt}}
\pgflineto{\pgfpoint{438.458862pt}{244.587509pt}}
\pgfusepath{stroke}
\pgfpathmoveto{\pgfpoint{438.450195pt}{250.559540pt}}
\pgflineto{\pgfpoint{438.570251pt}{250.579330pt}}
\pgfusepath{stroke}
\pgfpathmoveto{\pgfpoint{438.558105pt}{250.503326pt}}
\pgflineto{\pgfpoint{438.450195pt}{250.559540pt}}
\pgfusepath{stroke}
\pgfpathmoveto{\pgfpoint{438.442810pt}{256.529999pt}}
\pgflineto{\pgfpoint{438.563446pt}{256.556366pt}}
\pgfusepath{stroke}
\pgfpathmoveto{\pgfpoint{438.555115pt}{256.478729pt}}
\pgflineto{\pgfpoint{438.442810pt}{256.529999pt}}
\pgfusepath{stroke}
\pgfpathmoveto{\pgfpoint{438.437164pt}{262.498993pt}}
\pgflineto{\pgfpoint{438.557556pt}{262.532257pt}}
\pgfusepath{stroke}
\pgfpathmoveto{\pgfpoint{438.553467pt}{262.453339pt}}
\pgflineto{\pgfpoint{438.437164pt}{262.498993pt}}
\pgfusepath{stroke}
\pgfpathmoveto{\pgfpoint{438.433594pt}{268.466827pt}}
\pgflineto{\pgfpoint{438.552979pt}{268.507111pt}}
\pgfusepath{stroke}
\pgfpathmoveto{\pgfpoint{438.553284pt}{268.427429pt}}
\pgflineto{\pgfpoint{438.433594pt}{268.466827pt}}
\pgfusepath{stroke}
\pgfpathmoveto{\pgfpoint{438.432373pt}{274.433838pt}}
\pgflineto{\pgfpoint{438.549896pt}{274.481171pt}}
\pgfusepath{stroke}
\pgfpathmoveto{\pgfpoint{438.554810pt}{274.401215pt}}
\pgflineto{\pgfpoint{438.432373pt}{274.433838pt}}
\pgfusepath{stroke}
\pgfpathmoveto{\pgfpoint{438.433838pt}{280.400513pt}}
\pgflineto{\pgfpoint{438.548523pt}{280.454712pt}}
\pgfusepath{stroke}
\pgfpathmoveto{\pgfpoint{438.558075pt}{280.375061pt}}
\pgflineto{\pgfpoint{438.433838pt}{280.400513pt}}
\pgfusepath{stroke}
\pgfpathmoveto{\pgfpoint{438.437958pt}{286.367401pt}}
\pgflineto{\pgfpoint{438.548950pt}{286.428101pt}}
\pgfusepath{stroke}
\pgfpathmoveto{\pgfpoint{438.563171pt}{286.349365pt}}
\pgflineto{\pgfpoint{438.437958pt}{286.367401pt}}
\pgfusepath{stroke}
\pgfpathmoveto{\pgfpoint{438.444702pt}{292.335022pt}}
\pgflineto{\pgfpoint{438.551178pt}{292.401672pt}}
\pgfusepath{stroke}
\pgfpathmoveto{\pgfpoint{438.569885pt}{292.324463pt}}
\pgflineto{\pgfpoint{438.444702pt}{292.335022pt}}
\pgfusepath{stroke}
\pgfpathmoveto{\pgfpoint{438.453796pt}{298.303894pt}}
\pgflineto{\pgfpoint{438.555176pt}{298.375854pt}}
\pgfusepath{stroke}
\pgfpathmoveto{\pgfpoint{438.578064pt}{298.300659pt}}
\pgflineto{\pgfpoint{438.453796pt}{298.303894pt}}
\pgfusepath{stroke}
\pgfpathmoveto{\pgfpoint{438.464935pt}{304.274445pt}}
\pgflineto{\pgfpoint{438.560699pt}{304.350891pt}}
\pgfusepath{stroke}
\pgfpathmoveto{\pgfpoint{438.587402pt}{304.278137pt}}
\pgflineto{\pgfpoint{438.464935pt}{304.274445pt}}
\pgfusepath{stroke}
\pgfpathmoveto{\pgfpoint{438.477722pt}{310.246918pt}}
\pgflineto{\pgfpoint{438.567505pt}{310.327026pt}}
\pgfusepath{stroke}
\pgfpathmoveto{\pgfpoint{438.597595pt}{310.257141pt}}
\pgflineto{\pgfpoint{438.477722pt}{310.246918pt}}
\pgfusepath{stroke}
\pgfpathmoveto{\pgfpoint{438.491608pt}{316.221497pt}}
\pgflineto{\pgfpoint{438.575317pt}{316.304474pt}}
\pgfusepath{stroke}
\pgfpathmoveto{\pgfpoint{438.608337pt}{316.237671pt}}
\pgflineto{\pgfpoint{438.491608pt}{316.221497pt}}
\pgfusepath{stroke}
\pgfpathmoveto{\pgfpoint{438.506226pt}{322.198212pt}}
\pgflineto{\pgfpoint{438.583832pt}{322.283234pt}}
\pgfusepath{stroke}
\pgfpathmoveto{\pgfpoint{438.619324pt}{322.219666pt}}
\pgflineto{\pgfpoint{438.506226pt}{322.198212pt}}
\pgfusepath{stroke}
\pgfpathmoveto{\pgfpoint{438.521179pt}{328.177002pt}}
\pgflineto{\pgfpoint{438.592773pt}{328.263367pt}}
\pgfusepath{stroke}
\pgfpathmoveto{\pgfpoint{438.630280pt}{328.203125pt}}
\pgflineto{\pgfpoint{438.521179pt}{328.177002pt}}
\pgfusepath{stroke}
\pgfpathmoveto{\pgfpoint{438.536072pt}{334.157715pt}}
\pgflineto{\pgfpoint{438.601929pt}{334.244751pt}}
\pgfusepath{stroke}
\pgfpathmoveto{\pgfpoint{438.640991pt}{334.187836pt}}
\pgflineto{\pgfpoint{438.536072pt}{334.157715pt}}
\pgfusepath{stroke}
\pgfpathmoveto{\pgfpoint{438.550720pt}{340.140167pt}}
\pgflineto{\pgfpoint{438.611084pt}{340.227356pt}}
\pgfusepath{stroke}
\pgfpathmoveto{\pgfpoint{438.651337pt}{340.173676pt}}
\pgflineto{\pgfpoint{438.550720pt}{340.140167pt}}
\pgfusepath{stroke}
\pgfpathmoveto{\pgfpoint{438.564819pt}{346.124146pt}}
\pgflineto{\pgfpoint{438.620087pt}{346.210999pt}}
\pgfusepath{stroke}
\pgfpathmoveto{\pgfpoint{438.661163pt}{346.160461pt}}
\pgflineto{\pgfpoint{438.564819pt}{346.124146pt}}
\pgfusepath{stroke}
\pgfpathmoveto{\pgfpoint{438.578339pt}{352.109436pt}}
\pgflineto{\pgfpoint{438.628845pt}{352.195648pt}}
\pgfusepath{stroke}
\pgfpathmoveto{\pgfpoint{438.670441pt}{352.148102pt}}
\pgflineto{\pgfpoint{438.578339pt}{352.109436pt}}
\pgfusepath{stroke}
\pgfpathmoveto{\pgfpoint{438.591125pt}{358.095856pt}}
\pgflineto{\pgfpoint{438.637207pt}{358.181091pt}}
\pgfusepath{stroke}
\pgfpathmoveto{\pgfpoint{438.679138pt}{358.136414pt}}
\pgflineto{\pgfpoint{438.591125pt}{358.095856pt}}
\pgfusepath{stroke}
\pgfpathmoveto{\pgfpoint{438.603210pt}{364.083191pt}}
\pgflineto{\pgfpoint{438.645203pt}{364.167236pt}}
\pgfusepath{stroke}
\pgfpathmoveto{\pgfpoint{438.687195pt}{364.125214pt}}
\pgflineto{\pgfpoint{438.603210pt}{364.083191pt}}
\pgfusepath{stroke}
\pgfpathmoveto{\pgfpoint{438.614502pt}{370.071320pt}}
\pgflineto{\pgfpoint{438.652771pt}{370.153961pt}}
\pgfusepath{stroke}
\pgfpathmoveto{\pgfpoint{438.694702pt}{370.114471pt}}
\pgflineto{\pgfpoint{438.614502pt}{370.071320pt}}
\pgfusepath{stroke}
\pgfpathmoveto{\pgfpoint{444.678650pt}{77.007370pt}}
\pgflineto{\pgfpoint{444.728821pt}{76.969482pt}}
\pgfusepath{stroke}
\pgfpathmoveto{\pgfpoint{444.696075pt}{76.946991pt}}
\pgflineto{\pgfpoint{444.678650pt}{77.007370pt}}
\pgfusepath{stroke}
\pgfpathmoveto{\pgfpoint{444.674133pt}{82.996704pt}}
\pgflineto{\pgfpoint{444.725983pt}{82.958801pt}}
\pgfusepath{stroke}
\pgfpathmoveto{\pgfpoint{444.692871pt}{82.935287pt}}
\pgflineto{\pgfpoint{444.674133pt}{82.996704pt}}
\pgfusepath{stroke}
\pgfpathmoveto{\pgfpoint{444.669342pt}{88.985962pt}}
\pgflineto{\pgfpoint{444.722961pt}{88.948097pt}}
\pgfusepath{stroke}
\pgfpathmoveto{\pgfpoint{444.689514pt}{88.923508pt}}
\pgflineto{\pgfpoint{444.669342pt}{88.985962pt}}
\pgfusepath{stroke}
\pgfpathmoveto{\pgfpoint{444.664307pt}{94.975113pt}}
\pgflineto{\pgfpoint{444.719727pt}{94.937347pt}}
\pgfusepath{stroke}
\pgfpathmoveto{\pgfpoint{444.686005pt}{94.911636pt}}
\pgflineto{\pgfpoint{444.664307pt}{94.975113pt}}
\pgfusepath{stroke}
\pgfpathmoveto{\pgfpoint{444.659027pt}{100.964142pt}}
\pgflineto{\pgfpoint{444.716370pt}{100.926537pt}}
\pgfusepath{stroke}
\pgfpathmoveto{\pgfpoint{444.682343pt}{100.899666pt}}
\pgflineto{\pgfpoint{444.659027pt}{100.964142pt}}
\pgfusepath{stroke}
\pgfpathmoveto{\pgfpoint{444.653473pt}{106.953064pt}}
\pgflineto{\pgfpoint{444.712769pt}{106.915688pt}}
\pgfusepath{stroke}
\pgfpathmoveto{\pgfpoint{444.678497pt}{106.887573pt}}
\pgflineto{\pgfpoint{444.653473pt}{106.953064pt}}
\pgfusepath{stroke}
\pgfpathmoveto{\pgfpoint{444.647644pt}{112.941803pt}}
\pgflineto{\pgfpoint{444.708984pt}{112.904747pt}}
\pgfusepath{stroke}
\pgfpathmoveto{\pgfpoint{444.674500pt}{112.875328pt}}
\pgflineto{\pgfpoint{444.647644pt}{112.941803pt}}
\pgfusepath{stroke}
\pgfpathmoveto{\pgfpoint{444.641541pt}{118.930359pt}}
\pgflineto{\pgfpoint{444.705017pt}{118.893707pt}}
\pgfusepath{stroke}
\pgfpathmoveto{\pgfpoint{444.670319pt}{118.862946pt}}
\pgflineto{\pgfpoint{444.641541pt}{118.930359pt}}
\pgfusepath{stroke}
\pgfpathmoveto{\pgfpoint{444.635132pt}{124.918709pt}}
\pgflineto{\pgfpoint{444.700806pt}{124.882576pt}}
\pgfusepath{stroke}
\pgfpathmoveto{\pgfpoint{444.666016pt}{124.850388pt}}
\pgflineto{\pgfpoint{444.635132pt}{124.918709pt}}
\pgfusepath{stroke}
\pgfpathmoveto{\pgfpoint{444.628418pt}{130.906815pt}}
\pgflineto{\pgfpoint{444.696411pt}{130.871307pt}}
\pgfusepath{stroke}
\pgfpathmoveto{\pgfpoint{444.661499pt}{130.837631pt}}
\pgflineto{\pgfpoint{444.628418pt}{130.906815pt}}
\pgfusepath{stroke}
\pgfpathmoveto{\pgfpoint{444.621460pt}{136.894653pt}}
\pgflineto{\pgfpoint{444.691742pt}{136.859894pt}}
\pgfusepath{stroke}
\pgfpathmoveto{\pgfpoint{444.656830pt}{136.824677pt}}
\pgflineto{\pgfpoint{444.621460pt}{136.894653pt}}
\pgfusepath{stroke}
\pgfpathmoveto{\pgfpoint{444.614197pt}{142.882202pt}}
\pgflineto{\pgfpoint{444.686890pt}{142.848328pt}}
\pgfusepath{stroke}
\pgfpathmoveto{\pgfpoint{444.652008pt}{142.811462pt}}
\pgflineto{\pgfpoint{444.614197pt}{142.882202pt}}
\pgfusepath{stroke}
\pgfpathmoveto{\pgfpoint{444.606567pt}{148.869415pt}}
\pgflineto{\pgfpoint{444.681763pt}{148.836548pt}}
\pgfusepath{stroke}
\pgfpathmoveto{\pgfpoint{444.647034pt}{148.798004pt}}
\pgflineto{\pgfpoint{444.606567pt}{148.869415pt}}
\pgfusepath{stroke}
\pgfpathmoveto{\pgfpoint{444.598724pt}{154.856232pt}}
\pgflineto{\pgfpoint{444.676422pt}{154.824570pt}}
\pgfusepath{stroke}
\pgfpathmoveto{\pgfpoint{444.641907pt}{154.784286pt}}
\pgflineto{\pgfpoint{444.598724pt}{154.856232pt}}
\pgfusepath{stroke}
\pgfpathmoveto{\pgfpoint{444.590576pt}{160.842651pt}}
\pgflineto{\pgfpoint{444.670837pt}{160.812332pt}}
\pgfusepath{stroke}
\pgfpathmoveto{\pgfpoint{444.636597pt}{160.770248pt}}
\pgflineto{\pgfpoint{444.590576pt}{160.842651pt}}
\pgfusepath{stroke}
\pgfpathmoveto{\pgfpoint{444.582153pt}{166.828629pt}}
\pgflineto{\pgfpoint{444.665039pt}{166.799835pt}}
\pgfusepath{stroke}
\pgfpathmoveto{\pgfpoint{444.631195pt}{166.755875pt}}
\pgflineto{\pgfpoint{444.582153pt}{166.828629pt}}
\pgfusepath{stroke}
\pgfpathmoveto{\pgfpoint{444.573486pt}{172.814102pt}}
\pgflineto{\pgfpoint{444.658997pt}{172.787048pt}}
\pgfusepath{stroke}
\pgfpathmoveto{\pgfpoint{444.625641pt}{172.741150pt}}
\pgflineto{\pgfpoint{444.573486pt}{172.814102pt}}
\pgfusepath{stroke}
\pgfpathmoveto{\pgfpoint{444.564545pt}{178.799042pt}}
\pgflineto{\pgfpoint{444.652710pt}{178.773911pt}}
\pgfusepath{stroke}
\pgfpathmoveto{\pgfpoint{444.619995pt}{178.726044pt}}
\pgflineto{\pgfpoint{444.564545pt}{178.799042pt}}
\pgfusepath{stroke}
\pgfpathmoveto{\pgfpoint{444.555389pt}{184.783386pt}}
\pgflineto{\pgfpoint{444.646240pt}{184.760406pt}}
\pgfusepath{stroke}
\pgfpathmoveto{\pgfpoint{444.614258pt}{184.710495pt}}
\pgflineto{\pgfpoint{444.555389pt}{184.783386pt}}
\pgfusepath{stroke}
\pgfpathmoveto{\pgfpoint{444.546021pt}{190.767090pt}}
\pgflineto{\pgfpoint{444.639526pt}{190.746506pt}}
\pgfusepath{stroke}
\pgfpathmoveto{\pgfpoint{444.608459pt}{190.694504pt}}
\pgflineto{\pgfpoint{444.546021pt}{190.767090pt}}
\pgfusepath{stroke}
\pgfpathmoveto{\pgfpoint{444.536438pt}{196.750076pt}}
\pgflineto{\pgfpoint{444.632629pt}{196.732147pt}}
\pgfusepath{stroke}
\pgfpathmoveto{\pgfpoint{444.602631pt}{196.678024pt}}
\pgflineto{\pgfpoint{444.536438pt}{196.750076pt}}
\pgfusepath{stroke}
\pgfpathmoveto{\pgfpoint{444.526733pt}{202.732285pt}}
\pgflineto{\pgfpoint{444.625519pt}{202.717285pt}}
\pgfusepath{stroke}
\pgfpathmoveto{\pgfpoint{444.596741pt}{202.660995pt}}
\pgflineto{\pgfpoint{444.526733pt}{202.732285pt}}
\pgfusepath{stroke}
\pgfpathmoveto{\pgfpoint{444.516876pt}{208.713638pt}}
\pgflineto{\pgfpoint{444.618256pt}{208.701859pt}}
\pgfusepath{stroke}
\pgfpathmoveto{\pgfpoint{444.590912pt}{208.643402pt}}
\pgflineto{\pgfpoint{444.516876pt}{208.713638pt}}
\pgfusepath{stroke}
\pgfpathmoveto{\pgfpoint{444.507019pt}{214.694061pt}}
\pgflineto{\pgfpoint{444.610901pt}{214.685822pt}}
\pgfusepath{stroke}
\pgfpathmoveto{\pgfpoint{444.585144pt}{214.625168pt}}
\pgflineto{\pgfpoint{444.507019pt}{214.694061pt}}
\pgfusepath{stroke}
\pgfpathmoveto{\pgfpoint{444.497192pt}{220.673447pt}}
\pgflineto{\pgfpoint{444.603394pt}{220.669083pt}}
\pgfusepath{stroke}
\pgfpathmoveto{\pgfpoint{444.579529pt}{220.606247pt}}
\pgflineto{\pgfpoint{444.497192pt}{220.673447pt}}
\pgfusepath{stroke}
\pgfpathmoveto{\pgfpoint{444.487549pt}{226.651718pt}}
\pgflineto{\pgfpoint{444.595886pt}{226.651596pt}}
\pgfusepath{stroke}
\pgfpathmoveto{\pgfpoint{444.574158pt}{226.586609pt}}
\pgflineto{\pgfpoint{444.487549pt}{226.651718pt}}
\pgfusepath{stroke}
\pgfpathmoveto{\pgfpoint{444.478149pt}{232.628784pt}}
\pgflineto{\pgfpoint{444.588440pt}{232.633270pt}}
\pgfusepath{stroke}
\pgfpathmoveto{\pgfpoint{444.569092pt}{232.566193pt}}
\pgflineto{\pgfpoint{444.478149pt}{232.628784pt}}
\pgfusepath{stroke}
\pgfpathmoveto{\pgfpoint{444.469238pt}{238.604568pt}}
\pgflineto{\pgfpoint{444.581177pt}{238.614029pt}}
\pgfusepath{stroke}
\pgfpathmoveto{\pgfpoint{444.564484pt}{238.544983pt}}
\pgflineto{\pgfpoint{444.469238pt}{238.604568pt}}
\pgfusepath{stroke}
\pgfpathmoveto{\pgfpoint{444.461060pt}{244.579025pt}}
\pgflineto{\pgfpoint{444.574249pt}{244.593842pt}}
\pgfusepath{stroke}
\pgfpathmoveto{\pgfpoint{444.560486pt}{244.522964pt}}
\pgflineto{\pgfpoint{444.461060pt}{244.579025pt}}
\pgfusepath{stroke}
\pgfpathmoveto{\pgfpoint{444.453796pt}{250.552185pt}}
\pgflineto{\pgfpoint{444.567810pt}{250.572662pt}}
\pgfusepath{stroke}
\pgfpathmoveto{\pgfpoint{444.557281pt}{250.500183pt}}
\pgflineto{\pgfpoint{444.453796pt}{250.552185pt}}
\pgfusepath{stroke}
\pgfpathmoveto{\pgfpoint{444.447815pt}{256.524078pt}}
\pgflineto{\pgfpoint{444.562012pt}{256.550507pt}}
\pgfusepath{stroke}
\pgfpathmoveto{\pgfpoint{444.555054pt}{256.476685pt}}
\pgflineto{\pgfpoint{444.447815pt}{256.524078pt}}
\pgfusepath{stroke}
\pgfpathmoveto{\pgfpoint{444.443298pt}{262.494873pt}}
\pgflineto{\pgfpoint{444.557129pt}{262.527466pt}}
\pgfusepath{stroke}
\pgfpathmoveto{\pgfpoint{444.553955pt}{262.452637pt}}
\pgflineto{\pgfpoint{444.443298pt}{262.494873pt}}
\pgfusepath{stroke}
\pgfpathmoveto{\pgfpoint{444.440613pt}{268.464783pt}}
\pgflineto{\pgfpoint{444.553345pt}{268.503601pt}}
\pgfusepath{stroke}
\pgfpathmoveto{\pgfpoint{444.554108pt}{268.428192pt}}
\pgflineto{\pgfpoint{444.440613pt}{268.464783pt}}
\pgfusepath{stroke}
\pgfpathmoveto{\pgfpoint{444.439941pt}{274.434082pt}}
\pgflineto{\pgfpoint{444.550842pt}{274.479156pt}}
\pgfusepath{stroke}
\pgfpathmoveto{\pgfpoint{444.555725pt}{274.403564pt}}
\pgflineto{\pgfpoint{444.439941pt}{274.434082pt}}
\pgfusepath{stroke}
\pgfpathmoveto{\pgfpoint{444.441406pt}{280.403198pt}}
\pgflineto{\pgfpoint{444.549774pt}{280.454315pt}}
\pgfusepath{stroke}
\pgfpathmoveto{\pgfpoint{444.558777pt}{280.379059pt}}
\pgflineto{\pgfpoint{444.441406pt}{280.403198pt}}
\pgfusepath{stroke}
\pgfpathmoveto{\pgfpoint{444.445068pt}{286.372498pt}}
\pgflineto{\pgfpoint{444.550171pt}{286.429352pt}}
\pgfusepath{stroke}
\pgfpathmoveto{\pgfpoint{444.563263pt}{286.354919pt}}
\pgflineto{\pgfpoint{444.445068pt}{286.372498pt}}
\pgfusepath{stroke}
\pgfpathmoveto{\pgfpoint{444.450897pt}{292.342407pt}}
\pgflineto{\pgfpoint{444.552063pt}{292.404541pt}}
\pgfusepath{stroke}
\pgfpathmoveto{\pgfpoint{444.569153pt}{292.331421pt}}
\pgflineto{\pgfpoint{444.450897pt}{292.342407pt}}
\pgfusepath{stroke}
\pgfpathmoveto{\pgfpoint{444.458710pt}{298.313324pt}}
\pgflineto{\pgfpoint{444.555389pt}{298.380219pt}}
\pgfusepath{stroke}
\pgfpathmoveto{\pgfpoint{444.576172pt}{298.308838pt}}
\pgflineto{\pgfpoint{444.458710pt}{298.313324pt}}
\pgfusepath{stroke}
\pgfpathmoveto{\pgfpoint{444.468201pt}{304.285553pt}}
\pgflineto{\pgfpoint{444.559998pt}{304.356567pt}}
\pgfusepath{stroke}
\pgfpathmoveto{\pgfpoint{444.584290pt}{304.287292pt}}
\pgflineto{\pgfpoint{444.468201pt}{304.285553pt}}
\pgfusepath{stroke}
\pgfpathmoveto{\pgfpoint{444.479126pt}{310.259338pt}}
\pgflineto{\pgfpoint{444.565735pt}{310.333801pt}}
\pgfusepath{stroke}
\pgfpathmoveto{\pgfpoint{444.593109pt}{310.266968pt}}
\pgflineto{\pgfpoint{444.479126pt}{310.259338pt}}
\pgfusepath{stroke}
\pgfpathmoveto{\pgfpoint{444.491119pt}{316.234833pt}}
\pgflineto{\pgfpoint{444.572388pt}{316.312042pt}}
\pgfusepath{stroke}
\pgfpathmoveto{\pgfpoint{444.602448pt}{316.247864pt}}
\pgflineto{\pgfpoint{444.491119pt}{316.234833pt}}
\pgfusepath{stroke}
\pgfpathmoveto{\pgfpoint{444.503784pt}{322.212067pt}}
\pgflineto{\pgfpoint{444.579651pt}{322.291382pt}}
\pgfusepath{stroke}
\pgfpathmoveto{\pgfpoint{444.612061pt}{322.230011pt}}
\pgflineto{\pgfpoint{444.503784pt}{322.212067pt}}
\pgfusepath{stroke}
\pgfpathmoveto{\pgfpoint{444.516876pt}{328.191040pt}}
\pgflineto{\pgfpoint{444.587402pt}{328.271851pt}}
\pgfusepath{stroke}
\pgfpathmoveto{\pgfpoint{444.621765pt}{328.213379pt}}
\pgflineto{\pgfpoint{444.516876pt}{328.191040pt}}
\pgfusepath{stroke}
\pgfpathmoveto{\pgfpoint{444.530090pt}{334.171692pt}}
\pgflineto{\pgfpoint{444.595398pt}{334.253357pt}}
\pgfusepath{stroke}
\pgfpathmoveto{\pgfpoint{444.631378pt}{334.197845pt}}
\pgflineto{\pgfpoint{444.530090pt}{334.171692pt}}
\pgfusepath{stroke}
\pgfpathmoveto{\pgfpoint{444.543152pt}{340.153839pt}}
\pgflineto{\pgfpoint{444.603516pt}{340.235931pt}}
\pgfusepath{stroke}
\pgfpathmoveto{\pgfpoint{444.640686pt}{340.183319pt}}
\pgflineto{\pgfpoint{444.543152pt}{340.153839pt}}
\pgfusepath{stroke}
\pgfpathmoveto{\pgfpoint{444.555939pt}{346.137329pt}}
\pgflineto{\pgfpoint{444.611542pt}{346.219452pt}}
\pgfusepath{stroke}
\pgfpathmoveto{\pgfpoint{444.649689pt}{346.169647pt}}
\pgflineto{\pgfpoint{444.555939pt}{346.137329pt}}
\pgfusepath{stroke}
\pgfpathmoveto{\pgfpoint{444.568268pt}{352.122070pt}}
\pgflineto{\pgfpoint{444.619446pt}{352.203796pt}}
\pgfusepath{stroke}
\pgfpathmoveto{\pgfpoint{444.658264pt}{352.156769pt}}
\pgflineto{\pgfpoint{444.568268pt}{352.122070pt}}
\pgfusepath{stroke}
\pgfpathmoveto{\pgfpoint{444.580078pt}{358.107849pt}}
\pgflineto{\pgfpoint{444.627075pt}{358.188934pt}}
\pgfusepath{stroke}
\pgfpathmoveto{\pgfpoint{444.666351pt}{358.144501pt}}
\pgflineto{\pgfpoint{444.580078pt}{358.107849pt}}
\pgfusepath{stroke}
\pgfpathmoveto{\pgfpoint{444.591309pt}{364.094513pt}}
\pgflineto{\pgfpoint{444.634460pt}{364.174744pt}}
\pgfusepath{stroke}
\pgfpathmoveto{\pgfpoint{444.673950pt}{364.132812pt}}
\pgflineto{\pgfpoint{444.591309pt}{364.094513pt}}
\pgfusepath{stroke}
\pgfpathmoveto{\pgfpoint{444.601929pt}{370.081909pt}}
\pgflineto{\pgfpoint{444.641479pt}{370.161072pt}}
\pgfusepath{stroke}
\pgfpathmoveto{\pgfpoint{444.681091pt}{370.121490pt}}
\pgflineto{\pgfpoint{444.601929pt}{370.081909pt}}
\pgfusepath{stroke}
\end{pgfscope}
\end{pgfpicture}

  \caption{Campo vectorial}
  \label{fig:vecfield}
\end{figure}

\end{document}